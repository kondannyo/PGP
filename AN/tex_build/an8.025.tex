
\begin{absolutelynopagebreak}
\setstretch{.7}
{\PaliGlossA{aṅguttara nikāya 8}}\\
\begin{addmargin}[1em]{2em}
\setstretch{.5}
{\PaliGlossB{Numbered Discourses 8}}\\
\end{addmargin}
\end{absolutelynopagebreak}

\begin{absolutelynopagebreak}
\setstretch{.7}
{\PaliGlossA{3. gahapativagga}}\\
\begin{addmargin}[1em]{2em}
\setstretch{.5}
{\PaliGlossB{3. Householders}}\\
\end{addmargin}
\end{absolutelynopagebreak}

\begin{absolutelynopagebreak}
\setstretch{.7}
{\PaliGlossA{25. mahānāmasutta}}\\
\begin{addmargin}[1em]{2em}
\setstretch{.5}
{\PaliGlossB{25. With Mahānāma}}\\
\end{addmargin}
\end{absolutelynopagebreak}

\begin{absolutelynopagebreak}
\setstretch{.7}
{\PaliGlossA{ekaṃ samayaṃ bhagavā sakkesu viharati kapilavatthusmiṃ nigrodhārāme.}}\\
\begin{addmargin}[1em]{2em}
\setstretch{.5}
{\PaliGlossB{At one time the Buddha was staying in the land of the Sakyans, near Kapilavatthu in the Banyan Tree Monastery.}}\\
\end{addmargin}
\end{absolutelynopagebreak}

\begin{absolutelynopagebreak}
\setstretch{.7}
{\PaliGlossA{atha kho mahānāmo sakko yena bhagavā tenupasaṅkami; upasaṅkamitvā bhagavantaṃ abhivādetvā ekamantaṃ nisīdi. ekamantaṃ nisinno kho mahānāmo sakko bhagavantaṃ etadavoca:}}\\
\begin{addmargin}[1em]{2em}
\setstretch{.5}
{\PaliGlossB{Then Mahānāma the Sakyan went up to the Buddha, bowed, sat down to one side, and said to him:}}\\
\end{addmargin}
\end{absolutelynopagebreak}

\begin{absolutelynopagebreak}
\setstretch{.7}
{\PaliGlossA{“kittāvatā nu kho, bhante, upāsako hotī”ti?}}\\
\begin{addmargin}[1em]{2em}
\setstretch{.5}
{\PaliGlossB{“Sir, how is a lay follower defined?”}}\\
\end{addmargin}
\end{absolutelynopagebreak}

\begin{absolutelynopagebreak}
\setstretch{.7}
{\PaliGlossA{“yato kho, mahānāma, buddhaṃ saraṇaṃ gato hoti, dhammaṃ saraṇaṃ gato hoti, saṅghaṃ saraṇaṃ gato hoti;}}\\
\begin{addmargin}[1em]{2em}
\setstretch{.5}
{\PaliGlossB{“Mahānāma, when you’ve gone for refuge to the Buddha, the teaching, and the Saṅgha,}}\\
\end{addmargin}
\end{absolutelynopagebreak}

\begin{absolutelynopagebreak}
\setstretch{.7}
{\PaliGlossA{ettāvatā kho, mahānāma, upāsako hotī”ti.}}\\
\begin{addmargin}[1em]{2em}
\setstretch{.5}
{\PaliGlossB{you’re considered to be a lay follower.”}}\\
\end{addmargin}
\end{absolutelynopagebreak}

\begin{absolutelynopagebreak}
\setstretch{.7}
{\PaliGlossA{“kittāvatā pana, bhante, upāsako sīlavā hotī”ti?}}\\
\begin{addmargin}[1em]{2em}
\setstretch{.5}
{\PaliGlossB{“But how is an ethical lay follower defined?”}}\\
\end{addmargin}
\end{absolutelynopagebreak}

\begin{absolutelynopagebreak}
\setstretch{.7}
{\PaliGlossA{“yato kho, mahānāma, upāsako pāṇātipātā paṭivirato hoti, adinnādānā paṭivirato hoti, kāmesumicchācārā paṭivirato hoti, musāvādā paṭivirato hoti, surāmerayamajjapamādaṭṭhānā paṭivirato hoti;}}\\
\begin{addmargin}[1em]{2em}
\setstretch{.5}
{\PaliGlossB{“When a lay follower doesn’t kill living creatures, steal, commit sexual misconduct, lie, or use alcoholic drinks that cause negligence,}}\\
\end{addmargin}
\end{absolutelynopagebreak}

\begin{absolutelynopagebreak}
\setstretch{.7}
{\PaliGlossA{ettāvatā kho, mahānāma, upāsako sīlavā hotī”ti.}}\\
\begin{addmargin}[1em]{2em}
\setstretch{.5}
{\PaliGlossB{they’re considered to be an ethical lay follower.”}}\\
\end{addmargin}
\end{absolutelynopagebreak}

\begin{absolutelynopagebreak}
\setstretch{.7}
{\PaliGlossA{“kittāvatā pana, bhante, upāsako attahitāya paṭipanno hoti, no parahitāyā”ti?}}\\
\begin{addmargin}[1em]{2em}
\setstretch{.5}
{\PaliGlossB{“But how do we define a lay follower who is practicing to benefit themselves, not others?”}}\\
\end{addmargin}
\end{absolutelynopagebreak}

\begin{absolutelynopagebreak}
\setstretch{.7}
{\PaliGlossA{“yato kho, mahānāma, upāsako attanāva saddhāsampanno hoti, no paraṃ saddhāsampadāya samādapeti;}}\\
\begin{addmargin}[1em]{2em}
\setstretch{.5}
{\PaliGlossB{“A lay follower is accomplished in faith, but doesn’t encourage others to do the same.}}\\
\end{addmargin}
\end{absolutelynopagebreak}

\begin{absolutelynopagebreak}
\setstretch{.7}
{\PaliGlossA{attanāva sīlasampanno hoti, no paraṃ sīlasampadāya samādapeti;}}\\
\begin{addmargin}[1em]{2em}
\setstretch{.5}
{\PaliGlossB{They’re accomplished in ethical conduct, but don’t encourage others to do the same.}}\\
\end{addmargin}
\end{absolutelynopagebreak}

\begin{absolutelynopagebreak}
\setstretch{.7}
{\PaliGlossA{attanāva cāgasampanno hoti, no paraṃ cāgasampadāya samādapeti;}}\\
\begin{addmargin}[1em]{2em}
\setstretch{.5}
{\PaliGlossB{They’re accomplished in generosity, but don’t encourage others to do the same.}}\\
\end{addmargin}
\end{absolutelynopagebreak}

\begin{absolutelynopagebreak}
\setstretch{.7}
{\PaliGlossA{attanāva bhikkhūnaṃ dassanakāmo hoti, no paraṃ bhikkhūnaṃ dassane samādapeti;}}\\
\begin{addmargin}[1em]{2em}
\setstretch{.5}
{\PaliGlossB{They like to see the mendicants, but don’t encourage others to do the same.}}\\
\end{addmargin}
\end{absolutelynopagebreak}

\begin{absolutelynopagebreak}
\setstretch{.7}
{\PaliGlossA{attanāva saddhammaṃ sotukāmo hoti, no paraṃ saddhammassavane samādapeti;}}\\
\begin{addmargin}[1em]{2em}
\setstretch{.5}
{\PaliGlossB{They like to hear the true teaching, but don’t encourage others to do the same.}}\\
\end{addmargin}
\end{absolutelynopagebreak}

\begin{absolutelynopagebreak}
\setstretch{.7}
{\PaliGlossA{attanāva sutānaṃ dhammānaṃ dhāraṇajātiko hoti, no paraṃ dhammadhāraṇāya samādapeti;}}\\
\begin{addmargin}[1em]{2em}
\setstretch{.5}
{\PaliGlossB{They readily memorize the teachings they’ve heard, but don’t encourage others to do the same.}}\\
\end{addmargin}
\end{absolutelynopagebreak}

\begin{absolutelynopagebreak}
\setstretch{.7}
{\PaliGlossA{attanāva sutānaṃ dhammānaṃ atthūpaparikkhitā hoti, no paraṃ atthūpaparikkhāya samādapeti;}}\\
\begin{addmargin}[1em]{2em}
\setstretch{.5}
{\PaliGlossB{They examine the meaning of the teachings they’ve memorized, but don’t encourage others to do the same.}}\\
\end{addmargin}
\end{absolutelynopagebreak}

\begin{absolutelynopagebreak}
\setstretch{.7}
{\PaliGlossA{attanāva atthamaññāya dhammamaññāya dhammānudhammappaṭipanno hoti, no paraṃ dhammānudhammappaṭipattiyā samādapeti.}}\\
\begin{addmargin}[1em]{2em}
\setstretch{.5}
{\PaliGlossB{Understanding the meaning and the teaching, they practice accordingly, but they don’t encourage others to do the same.}}\\
\end{addmargin}
\end{absolutelynopagebreak}

\begin{absolutelynopagebreak}
\setstretch{.7}
{\PaliGlossA{ettāvatā kho, mahānāma, upāsako attahitāya paṭipanno hoti, no parahitāyā”ti.}}\\
\begin{addmargin}[1em]{2em}
\setstretch{.5}
{\PaliGlossB{That’s how we define a lay follower who is practicing to benefit themselves, not others.”}}\\
\end{addmargin}
\end{absolutelynopagebreak}

\begin{absolutelynopagebreak}
\setstretch{.7}
{\PaliGlossA{“kittāvatā pana, bhante, upāsako attahitāya ca paṭipanno hoti parahitāya cā”ti?}}\\
\begin{addmargin}[1em]{2em}
\setstretch{.5}
{\PaliGlossB{“But how do we define a lay follower who is practicing to benefit both themselves and others?”}}\\
\end{addmargin}
\end{absolutelynopagebreak}

\begin{absolutelynopagebreak}
\setstretch{.7}
{\PaliGlossA{“yato kho, mahānāma, upāsako attanā ca saddhāsampanno hoti, parañca saddhāsampadāya samādapeti;}}\\
\begin{addmargin}[1em]{2em}
\setstretch{.5}
{\PaliGlossB{“A lay follower is accomplished in faith and encourages others to do the same.}}\\
\end{addmargin}
\end{absolutelynopagebreak}

\begin{absolutelynopagebreak}
\setstretch{.7}
{\PaliGlossA{attanā ca sīlasampanno hoti, parañca sīlasampadāya samādapeti;}}\\
\begin{addmargin}[1em]{2em}
\setstretch{.5}
{\PaliGlossB{They’re accomplished in ethical conduct and encourage others to do the same.}}\\
\end{addmargin}
\end{absolutelynopagebreak}

\begin{absolutelynopagebreak}
\setstretch{.7}
{\PaliGlossA{attanā ca cāgasampanno hoti, parañca cāgasampadāya samādapeti;}}\\
\begin{addmargin}[1em]{2em}
\setstretch{.5}
{\PaliGlossB{They’re accomplished in generosity and encourage others to do the same.}}\\
\end{addmargin}
\end{absolutelynopagebreak}

\begin{absolutelynopagebreak}
\setstretch{.7}
{\PaliGlossA{attanā ca bhikkhūnaṃ dassanakāmo hoti, parañca bhikkhūnaṃ dassane samādapeti;}}\\
\begin{addmargin}[1em]{2em}
\setstretch{.5}
{\PaliGlossB{They like to see the mendicants and encourage others to do the same.}}\\
\end{addmargin}
\end{absolutelynopagebreak}

\begin{absolutelynopagebreak}
\setstretch{.7}
{\PaliGlossA{attanā ca saddhammaṃ sotukāmo hoti, parañca saddhammassavane samādapeti;}}\\
\begin{addmargin}[1em]{2em}
\setstretch{.5}
{\PaliGlossB{They like to hear the true teaching and encourage others to do the same.}}\\
\end{addmargin}
\end{absolutelynopagebreak}

\begin{absolutelynopagebreak}
\setstretch{.7}
{\PaliGlossA{attanā ca sutānaṃ dhammānaṃ dhāraṇajātiko hoti, parañca dhammadhāraṇāya samādapeti;}}\\
\begin{addmargin}[1em]{2em}
\setstretch{.5}
{\PaliGlossB{They readily memorize the teachings they’ve heard and encourage others to do the same.}}\\
\end{addmargin}
\end{absolutelynopagebreak}

\begin{absolutelynopagebreak}
\setstretch{.7}
{\PaliGlossA{attanā ca sutānaṃ dhammānaṃ atthūpaparikkhitā hoti, parañca atthūpaparikkhāya samādapeti,}}\\
\begin{addmargin}[1em]{2em}
\setstretch{.5}
{\PaliGlossB{They examine the meaning of the teachings they’ve memorized and encourage others to do the same.}}\\
\end{addmargin}
\end{absolutelynopagebreak}

\begin{absolutelynopagebreak}
\setstretch{.7}
{\PaliGlossA{attanā ca atthamaññāya dhammamaññāya dhammānudhammappaṭipanno hoti, parañca dhammānudhammappaṭipattiyā samādapeti.}}\\
\begin{addmargin}[1em]{2em}
\setstretch{.5}
{\PaliGlossB{Understanding the meaning and the teaching, they practice accordingly and they encourage others to do the same.}}\\
\end{addmargin}
\end{absolutelynopagebreak}

\begin{absolutelynopagebreak}
\setstretch{.7}
{\PaliGlossA{ettāvatā kho, mahānāma, upāsako attahitāya ca paṭipanno hoti parahitāya cā”ti.}}\\
\begin{addmargin}[1em]{2em}
\setstretch{.5}
{\PaliGlossB{That’s how we define a lay follower who is practicing to benefit both themselves and others.”}}\\
\end{addmargin}
\end{absolutelynopagebreak}

\begin{absolutelynopagebreak}
\setstretch{.7}
{\PaliGlossA{pañcamaṃ.}}\\
\begin{addmargin}[1em]{2em}
\setstretch{.5}
{\PaliGlossB{    -}}\\
\end{addmargin}
\end{absolutelynopagebreak}
