
\begin{absolutelynopagebreak}
\setstretch{.7}
{\PaliGlossA{aṅguttara nikāya 6}}\\
\begin{addmargin}[1em]{2em}
\setstretch{.5}
{\PaliGlossB{Numbered Discourses 6}}\\
\end{addmargin}
\end{absolutelynopagebreak}

\begin{absolutelynopagebreak}
\setstretch{.7}
{\PaliGlossA{7. devatāvagga}}\\
\begin{addmargin}[1em]{2em}
\setstretch{.5}
{\PaliGlossB{7. A God}}\\
\end{addmargin}
\end{absolutelynopagebreak}

\begin{absolutelynopagebreak}
\setstretch{.7}
{\PaliGlossA{71. sakkhibhabbasutta}}\\
\begin{addmargin}[1em]{2em}
\setstretch{.5}
{\PaliGlossB{71. Capable of Realizing}}\\
\end{addmargin}
\end{absolutelynopagebreak}

\begin{absolutelynopagebreak}
\setstretch{.7}
{\PaliGlossA{“chahi, bhikkhave, dhammehi samannāgato bhikkhu abhabbo tatra tatreva sakkhibhabbataṃ pāpuṇituṃ sati sati āyatane.}}\\
\begin{addmargin}[1em]{2em}
\setstretch{.5}
{\PaliGlossB{“Mendicants, a mendicant with six qualities is incapable of realizing anything that can be realized, in each and every case.}}\\
\end{addmargin}
\end{absolutelynopagebreak}

\begin{absolutelynopagebreak}
\setstretch{.7}
{\PaliGlossA{katamehi chahi?}}\\
\begin{addmargin}[1em]{2em}
\setstretch{.5}
{\PaliGlossB{What six?}}\\
\end{addmargin}
\end{absolutelynopagebreak}

\begin{absolutelynopagebreak}
\setstretch{.7}
{\PaliGlossA{idha, bhikkhave, bhikkhu ‘ime hānabhāgiyā dhammā’ti yathābhūtaṃ nappajānāti, ‘ime ṭhitibhāgiyā dhammā’ti yathābhūtaṃ nappajānāti, ‘ime visesabhāgiyā dhammā’ti yathābhūtaṃ nappajānāti, ‘ime nibbedhabhāgiyā dhammā’ti yathābhūtaṃ nappajānāti, asakkaccakārī ca hoti, asappāyakārī ca.}}\\
\begin{addmargin}[1em]{2em}
\setstretch{.5}
{\PaliGlossB{It’s when a mendicant doesn’t truly understand which qualities make things worse, which keep things steady, which lead to distinction, and which lead to penetration. And they don’t practice carefully or do what’s suitable.}}\\
\end{addmargin}
\end{absolutelynopagebreak}

\begin{absolutelynopagebreak}
\setstretch{.7}
{\PaliGlossA{imehi kho, bhikkhave, chahi dhammehi samannāgato bhikkhu abhabbo tatra tatreva sakkhibhabbataṃ pāpuṇituṃ sati sati āyatane.}}\\
\begin{addmargin}[1em]{2em}
\setstretch{.5}
{\PaliGlossB{A mendicant with these six qualities is incapable of realizing anything that can be realized, in each and every case.}}\\
\end{addmargin}
\end{absolutelynopagebreak}

\begin{absolutelynopagebreak}
\setstretch{.7}
{\PaliGlossA{chahi, bhikkhave, dhammehi samannāgato bhikkhu bhabbo tatra tatreva sakkhibhabbataṃ pāpuṇituṃ sati sati āyatane.}}\\
\begin{addmargin}[1em]{2em}
\setstretch{.5}
{\PaliGlossB{A mendicant with six qualities is capable of realizing anything that can be realized, in each and every case.}}\\
\end{addmargin}
\end{absolutelynopagebreak}

\begin{absolutelynopagebreak}
\setstretch{.7}
{\PaliGlossA{katamehi chahi?}}\\
\begin{addmargin}[1em]{2em}
\setstretch{.5}
{\PaliGlossB{What six?}}\\
\end{addmargin}
\end{absolutelynopagebreak}

\begin{absolutelynopagebreak}
\setstretch{.7}
{\PaliGlossA{idha, bhikkhave, bhikkhu ‘ime hānabhāgiyā dhammā’ti yathābhūtaṃ pajānāti, ‘ime ṭhitibhāgiyā dhammā’ti yathābhūtaṃ pajānāti, ‘ime visesabhāgiyā dhammā’ti yathābhūtaṃ pajānāti, ‘ime nibbedhabhāgiyā dhammā’ti yathābhūtaṃ pajānāti, sakkaccakārī ca hoti, sappāyakārī ca.}}\\
\begin{addmargin}[1em]{2em}
\setstretch{.5}
{\PaliGlossB{It’s when a mendicant truly understands which qualities make things worse, which keep things steady, which lead to distinction, and which lead to penetration. And they practice carefully and do what’s suitable.}}\\
\end{addmargin}
\end{absolutelynopagebreak}

\begin{absolutelynopagebreak}
\setstretch{.7}
{\PaliGlossA{imehi kho, bhikkhave, chahi dhammehi samannāgato bhikkhu bhabbo tatra tatreva sakkhibhabbataṃ pāpuṇituṃ sati sati āyatane”ti.}}\\
\begin{addmargin}[1em]{2em}
\setstretch{.5}
{\PaliGlossB{A mendicant with these six qualities is capable of realizing anything that can be realized, in each and every case.”}}\\
\end{addmargin}
\end{absolutelynopagebreak}

\begin{absolutelynopagebreak}
\setstretch{.7}
{\PaliGlossA{sattamaṃ.}}\\
\begin{addmargin}[1em]{2em}
\setstretch{.5}
{\PaliGlossB{    -}}\\
\end{addmargin}
\end{absolutelynopagebreak}
