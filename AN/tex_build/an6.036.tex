
\begin{absolutelynopagebreak}
\setstretch{.7}
{\PaliGlossA{aṅguttara nikāya 6}}\\
\begin{addmargin}[1em]{2em}
\setstretch{.5}
{\PaliGlossB{Numbered Discourses 6}}\\
\end{addmargin}
\end{absolutelynopagebreak}

\begin{absolutelynopagebreak}
\setstretch{.7}
{\PaliGlossA{4. devatāvagga}}\\
\begin{addmargin}[1em]{2em}
\setstretch{.5}
{\PaliGlossB{4. Deities}}\\
\end{addmargin}
\end{absolutelynopagebreak}

\begin{absolutelynopagebreak}
\setstretch{.7}
{\PaliGlossA{36. vivādamūlasutta}}\\
\begin{addmargin}[1em]{2em}
\setstretch{.5}
{\PaliGlossB{36. Roots of Quarrels}}\\
\end{addmargin}
\end{absolutelynopagebreak}

\begin{absolutelynopagebreak}
\setstretch{.7}
{\PaliGlossA{“chayimāni, bhikkhave, vivādamūlāni.}}\\
\begin{addmargin}[1em]{2em}
\setstretch{.5}
{\PaliGlossB{“Mendicants, there are these six roots of quarrels.}}\\
\end{addmargin}
\end{absolutelynopagebreak}

\begin{absolutelynopagebreak}
\setstretch{.7}
{\PaliGlossA{katamāni cha?}}\\
\begin{addmargin}[1em]{2em}
\setstretch{.5}
{\PaliGlossB{What six?}}\\
\end{addmargin}
\end{absolutelynopagebreak}

\begin{absolutelynopagebreak}
\setstretch{.7}
{\PaliGlossA{idha, bhikkhave, bhikkhu kodhano hoti upanāhī.}}\\
\begin{addmargin}[1em]{2em}
\setstretch{.5}
{\PaliGlossB{Firstly, a mendicant is irritable and hostile.}}\\
\end{addmargin}
\end{absolutelynopagebreak}

\begin{absolutelynopagebreak}
\setstretch{.7}
{\PaliGlossA{yo so, bhikkhave, bhikkhu kodhano hoti upanāhī so sattharipi agāravo viharati appatisso, dhammepi agāravo viharati appatisso, saṃghepi agāravo viharati appatisso, sikkhāyapi na paripūrakārī hoti.}}\\
\begin{addmargin}[1em]{2em}
\setstretch{.5}
{\PaliGlossB{Such a mendicant lacks respect and reverence for the Teacher, the teaching, and the Saṅgha, and they don’t fulfill the training.}}\\
\end{addmargin}
\end{absolutelynopagebreak}

\begin{absolutelynopagebreak}
\setstretch{.7}
{\PaliGlossA{yo so, bhikkhave, bhikkhu satthari agāravo viharati appatisso, dhamme agāravo viharati appatisso, saṃghe agāravo viharati appatisso, sikkhāya na paripūrakārī so saṃghe vivādaṃ janeti, yo hoti vivādo bahujanāhitāya bahujanāsukhāya bahuno janassa anatthāya ahitāya dukkhāya devamanussānaṃ.}}\\
\begin{addmargin}[1em]{2em}
\setstretch{.5}
{\PaliGlossB{They create a dispute in the Saṅgha, which is for the hurt and unhappiness of the people, for the harm, hurt, and suffering of gods and humans.}}\\
\end{addmargin}
\end{absolutelynopagebreak}

\begin{absolutelynopagebreak}
\setstretch{.7}
{\PaliGlossA{evarūpañce tumhe, bhikkhave, vivādamūlaṃ ajjhattaṃ vā bahiddhā vā samanupasseyyātha. tatra tumhe, bhikkhave, tasseva pāpakassa vivādamūlassa pahānāya vāyameyyātha.}}\\
\begin{addmargin}[1em]{2em}
\setstretch{.5}
{\PaliGlossB{If you see such a root of quarrels in yourselves or others, you should try to give up this bad thing.}}\\
\end{addmargin}
\end{absolutelynopagebreak}

\begin{absolutelynopagebreak}
\setstretch{.7}
{\PaliGlossA{evarūpañce tumhe, bhikkhave, vivādamūlaṃ ajjhattaṃ vā bahiddhā vā na samanupasseyyātha, tatra tumhe, bhikkhave, tasseva pāpakassa vivādamūlassa āyatiṃ anavassavāya paṭipajjeyyātha.}}\\
\begin{addmargin}[1em]{2em}
\setstretch{.5}
{\PaliGlossB{If you don’t see it, you should practice so that it doesn’t come up in the future.}}\\
\end{addmargin}
\end{absolutelynopagebreak}

\begin{absolutelynopagebreak}
\setstretch{.7}
{\PaliGlossA{evametassa pāpakassa vivādamūlassa pahānaṃ hoti. evametassa pāpakassa vivādamūlassa āyatiṃ anavassavo hoti.}}\\
\begin{addmargin}[1em]{2em}
\setstretch{.5}
{\PaliGlossB{That’s how to give up this bad root of quarrels, so it doesn’t come up in the future.}}\\
\end{addmargin}
\end{absolutelynopagebreak}

\begin{absolutelynopagebreak}
\setstretch{.7}
{\PaliGlossA{puna caparaṃ, bhikkhave, bhikkhu makkhī hoti paḷāsī … pe …}}\\
\begin{addmargin}[1em]{2em}
\setstretch{.5}
{\PaliGlossB{Furthermore, a mendicant is offensive and contemptuous …}}\\
\end{addmargin}
\end{absolutelynopagebreak}

\begin{absolutelynopagebreak}
\setstretch{.7}
{\PaliGlossA{issukī hoti maccharī …}}\\
\begin{addmargin}[1em]{2em}
\setstretch{.5}
{\PaliGlossB{They’re jealous and stingy …}}\\
\end{addmargin}
\end{absolutelynopagebreak}

\begin{absolutelynopagebreak}
\setstretch{.7}
{\PaliGlossA{saṭho hoti māyāvī …}}\\
\begin{addmargin}[1em]{2em}
\setstretch{.5}
{\PaliGlossB{devious and deceitful …}}\\
\end{addmargin}
\end{absolutelynopagebreak}

\begin{absolutelynopagebreak}
\setstretch{.7}
{\PaliGlossA{pāpiccho hoti micchādiṭṭhi …}}\\
\begin{addmargin}[1em]{2em}
\setstretch{.5}
{\PaliGlossB{with wicked desires and wrong view …}}\\
\end{addmargin}
\end{absolutelynopagebreak}

\begin{absolutelynopagebreak}
\setstretch{.7}
{\PaliGlossA{sandiṭṭhiparāmāsī hoti ādhānaggāhī duppaṭinissaggī.}}\\
\begin{addmargin}[1em]{2em}
\setstretch{.5}
{\PaliGlossB{They’re attached to their own views, holding them tight, and refusing to let go.}}\\
\end{addmargin}
\end{absolutelynopagebreak}

\begin{absolutelynopagebreak}
\setstretch{.7}
{\PaliGlossA{yo so, bhikkhave, bhikkhu sandiṭṭhiparāmāsī hoti ādhānaggāhī duppaṭinissaggī, so sattharipi agāravo viharati appatisso, dhammepi agāravo viharati appatisso, saṅghepi agāravo viharati appatisso, sikkhāyapi na paripūrakārī hoti.}}\\
\begin{addmargin}[1em]{2em}
\setstretch{.5}
{\PaliGlossB{Such a mendicant lacks respect and reverence for the Teacher, the teaching, and the Saṅgha, and they don’t fulfill the training.}}\\
\end{addmargin}
\end{absolutelynopagebreak}

\begin{absolutelynopagebreak}
\setstretch{.7}
{\PaliGlossA{yo so, bhikkhave, bhikkhu satthari agāravo viharati appatisso, dhamme … saṅghe agāravo viharati appatisso, sikkhāya na paripūrakārī, so saṅghe vivādaṃ janeti, yo hoti vivādo bahujanāhitāya bahujanāsukhāya bahuno janassa anatthāya ahitāya dukkhāya devamanussānaṃ.}}\\
\begin{addmargin}[1em]{2em}
\setstretch{.5}
{\PaliGlossB{They create a dispute in the Saṅgha, which is for the hurt and unhappiness of the people, for the harm, hurt, and suffering of gods and humans.}}\\
\end{addmargin}
\end{absolutelynopagebreak}

\begin{absolutelynopagebreak}
\setstretch{.7}
{\PaliGlossA{evarūpañce tumhe, bhikkhave, vivādamūlaṃ ajjhattaṃ vā bahiddhā vā samanupasseyyātha. tatra tumhe, bhikkhave, tasseva pāpakassa vivādamūlassa pahānāya vāyameyyātha.}}\\
\begin{addmargin}[1em]{2em}
\setstretch{.5}
{\PaliGlossB{If you see such a root of quarrels in yourselves or others, you should try to give up this bad thing.}}\\
\end{addmargin}
\end{absolutelynopagebreak}

\begin{absolutelynopagebreak}
\setstretch{.7}
{\PaliGlossA{evarūpañce tumhe, bhikkhave, vivādamūlaṃ ajjhattaṃ vā bahiddhā vā na samanupasseyyātha. tatra tumhe, bhikkhave, tasseva pāpakassa vivādamūlassa āyatiṃ anavassavāya paṭipajjeyyātha.}}\\
\begin{addmargin}[1em]{2em}
\setstretch{.5}
{\PaliGlossB{If you don’t see it, you should practice so that it doesn’t come up in the future.}}\\
\end{addmargin}
\end{absolutelynopagebreak}

\begin{absolutelynopagebreak}
\setstretch{.7}
{\PaliGlossA{evametassa pāpakassa vivādamūlassa pahānaṃ hoti. evametassa pāpakassa vivādamūlassa āyatiṃ anavassavo hoti.}}\\
\begin{addmargin}[1em]{2em}
\setstretch{.5}
{\PaliGlossB{That’s how to give up this bad root of quarrels, so it doesn’t come up in the future.}}\\
\end{addmargin}
\end{absolutelynopagebreak}

\begin{absolutelynopagebreak}
\setstretch{.7}
{\PaliGlossA{imāni kho, bhikkhave, cha vivādamūlānī”ti.}}\\
\begin{addmargin}[1em]{2em}
\setstretch{.5}
{\PaliGlossB{These are the six roots of quarrels.”}}\\
\end{addmargin}
\end{absolutelynopagebreak}

\begin{absolutelynopagebreak}
\setstretch{.7}
{\PaliGlossA{chaṭṭhaṃ.}}\\
\begin{addmargin}[1em]{2em}
\setstretch{.5}
{\PaliGlossB{    -}}\\
\end{addmargin}
\end{absolutelynopagebreak}
