
\begin{absolutelynopagebreak}
\setstretch{.7}
{\PaliGlossA{aṅguttara nikāya 4}}\\
\begin{addmargin}[1em]{2em}
\setstretch{.5}
{\PaliGlossB{Numbered Discourses 4}}\\
\end{addmargin}
\end{absolutelynopagebreak}

\begin{absolutelynopagebreak}
\setstretch{.7}
{\PaliGlossA{13. bhayavagga}}\\
\begin{addmargin}[1em]{2em}
\setstretch{.5}
{\PaliGlossB{13. Fears}}\\
\end{addmargin}
\end{absolutelynopagebreak}

\begin{absolutelynopagebreak}
\setstretch{.7}
{\PaliGlossA{123. paṭhamanānākaraṇasutta}}\\
\begin{addmargin}[1em]{2em}
\setstretch{.5}
{\PaliGlossB{123. Difference (1st)}}\\
\end{addmargin}
\end{absolutelynopagebreak}

\begin{absolutelynopagebreak}
\setstretch{.7}
{\PaliGlossA{“cattārome, bhikkhave, puggalā santo saṃvijjamānā lokasmiṃ.}}\\
\begin{addmargin}[1em]{2em}
\setstretch{.5}
{\PaliGlossB{“Mendicants, these four people are found in the world.}}\\
\end{addmargin}
\end{absolutelynopagebreak}

\begin{absolutelynopagebreak}
\setstretch{.7}
{\PaliGlossA{katame cattāro?}}\\
\begin{addmargin}[1em]{2em}
\setstretch{.5}
{\PaliGlossB{What four?}}\\
\end{addmargin}
\end{absolutelynopagebreak}

\begin{absolutelynopagebreak}
\setstretch{.7}
{\PaliGlossA{idha, bhikkhave, ekacco puggalo vivicceva kāmehi vivicca akusalehi dhammehi savitakkaṃ savicāraṃ vivekajaṃ pītisukhaṃ paṭhamaṃ jhānaṃ upasampajja viharati.}}\\
\begin{addmargin}[1em]{2em}
\setstretch{.5}
{\PaliGlossB{Firstly, a mendicant, quite secluded from sensual pleasures, secluded from unskillful qualities, enters and remains in the first absorption, which has the rapture and bliss born of seclusion, while placing the mind and keeping it connected.}}\\
\end{addmargin}
\end{absolutelynopagebreak}

\begin{absolutelynopagebreak}
\setstretch{.7}
{\PaliGlossA{so tadassādeti, taṃ nikāmeti, tena ca vittiṃ āpajjati.}}\\
\begin{addmargin}[1em]{2em}
\setstretch{.5}
{\PaliGlossB{They enjoy it and like it and find it satisfying.}}\\
\end{addmargin}
\end{absolutelynopagebreak}

\begin{absolutelynopagebreak}
\setstretch{.7}
{\PaliGlossA{tattha ṭhito tadadhimutto tabbahulavihārī aparihīno kālaṃ kurumāno brahmakāyikānaṃ devānaṃ sahabyataṃ upapajjati.}}\\
\begin{addmargin}[1em]{2em}
\setstretch{.5}
{\PaliGlossB{If they abide in that, are committed to it, and meditate on it often without losing it, when they die they’re reborn in the company of the gods of Brahmā’s Host.}}\\
\end{addmargin}
\end{absolutelynopagebreak}

\begin{absolutelynopagebreak}
\setstretch{.7}
{\PaliGlossA{brahmakāyikānaṃ, bhikkhave, devānaṃ kappo āyuppamāṇaṃ.}}\\
\begin{addmargin}[1em]{2em}
\setstretch{.5}
{\PaliGlossB{The lifespan of the gods of Brahma’s Host is one eon.}}\\
\end{addmargin}
\end{absolutelynopagebreak}

\begin{absolutelynopagebreak}
\setstretch{.7}
{\PaliGlossA{tattha puthujjano yāvatāyukaṃ ṭhatvā yāvatakaṃ tesaṃ devānaṃ āyuppamāṇaṃ taṃ sabbaṃ khepetvā nirayampi gacchati tiracchānayonimpi gacchati pettivisayampi gacchati.}}\\
\begin{addmargin}[1em]{2em}
\setstretch{.5}
{\PaliGlossB{An ordinary person stays there until the lifespan of those gods is spent, then they go to hell or the animal realm or the ghost realm.}}\\
\end{addmargin}
\end{absolutelynopagebreak}

\begin{absolutelynopagebreak}
\setstretch{.7}
{\PaliGlossA{bhagavato pana sāvako tattha yāvatāyukaṃ ṭhatvā yāvatakaṃ tesaṃ devānaṃ āyuppamāṇaṃ taṃ sabbaṃ khepetvā tasmiṃyeva bhave parinibbāyati.}}\\
\begin{addmargin}[1em]{2em}
\setstretch{.5}
{\PaliGlossB{But a disciple of the Buddha stays there until the lifespan of those gods is spent, then they’re extinguished in that very life.}}\\
\end{addmargin}
\end{absolutelynopagebreak}

\begin{absolutelynopagebreak}
\setstretch{.7}
{\PaliGlossA{ayaṃ kho, bhikkhave, viseso ayaṃ adhippayāso idaṃ nānākaraṇaṃ sutavato ariyasāvakassa assutavatā puthujjanena, yadidaṃ gatiyā upapattiyā sati.}}\\
\begin{addmargin}[1em]{2em}
\setstretch{.5}
{\PaliGlossB{This is the difference between an educated noble disciple and an uneducated ordinary person, that is, when there is a place of rebirth.}}\\
\end{addmargin}
\end{absolutelynopagebreak}

\begin{absolutelynopagebreak}
\setstretch{.7}
{\PaliGlossA{puna caparaṃ, bhikkhave, idhekacco puggalo vitakkavicārānaṃ vūpasamā ajjhattaṃ sampasādanaṃ cetaso ekodibhāvaṃ avitakkaṃ avicāraṃ samādhijaṃ pītisukhaṃ dutiyaṃ jhānaṃ upasampajja viharati.}}\\
\begin{addmargin}[1em]{2em}
\setstretch{.5}
{\PaliGlossB{As the placing of the mind and keeping it connected are stilled, they enter and remain in the second absorption, which has the rapture and bliss born of immersion, with internal clarity and confidence, and unified mind, without placing the mind and keeping it connected.}}\\
\end{addmargin}
\end{absolutelynopagebreak}

\begin{absolutelynopagebreak}
\setstretch{.7}
{\PaliGlossA{so tadassādeti, taṃ nikāmeti, tena ca vittiṃ āpajjati.}}\\
\begin{addmargin}[1em]{2em}
\setstretch{.5}
{\PaliGlossB{They enjoy it and like it and find it satisfying.}}\\
\end{addmargin}
\end{absolutelynopagebreak}

\begin{absolutelynopagebreak}
\setstretch{.7}
{\PaliGlossA{tattha ṭhito tadadhimutto tabbahulavihārī aparihīno kālaṃ kurumāno ābhassarānaṃ devānaṃ sahabyataṃ upapajjati.}}\\
\begin{addmargin}[1em]{2em}
\setstretch{.5}
{\PaliGlossB{If they abide in that, are committed to it, and meditate on it often without losing it, when they die they’re reborn in the company of the gods of streaming radiance.}}\\
\end{addmargin}
\end{absolutelynopagebreak}

\begin{absolutelynopagebreak}
\setstretch{.7}
{\PaliGlossA{ābhassarānaṃ, bhikkhave, devānaṃ dve kappā āyuppamāṇaṃ.}}\\
\begin{addmargin}[1em]{2em}
\setstretch{.5}
{\PaliGlossB{The lifespan of the gods of streaming radiance is two eons.}}\\
\end{addmargin}
\end{absolutelynopagebreak}

\begin{absolutelynopagebreak}
\setstretch{.7}
{\PaliGlossA{tattha puthujjano yāvatāyukaṃ ṭhatvā yāvatakaṃ tesaṃ devānaṃ āyuppamāṇaṃ taṃ sabbaṃ khepetvā nirayampi gacchati tiracchānayonimpi gacchati pettivisayampi gacchati.}}\\
\begin{addmargin}[1em]{2em}
\setstretch{.5}
{\PaliGlossB{An ordinary person stays there until the lifespan of those gods is spent, then they go to hell or the animal realm or the ghost realm.}}\\
\end{addmargin}
\end{absolutelynopagebreak}

\begin{absolutelynopagebreak}
\setstretch{.7}
{\PaliGlossA{bhagavato pana sāvako tattha yāvatāyukaṃ ṭhatvā yāvatakaṃ tesaṃ devānaṃ āyuppamāṇaṃ taṃ sabbaṃ khepetvā tasmiṃyeva bhave parinibbāyati.}}\\
\begin{addmargin}[1em]{2em}
\setstretch{.5}
{\PaliGlossB{But a disciple of the Buddha stays there until the lifespan of those gods is spent, then they’re extinguished in that very life.}}\\
\end{addmargin}
\end{absolutelynopagebreak}

\begin{absolutelynopagebreak}
\setstretch{.7}
{\PaliGlossA{ayaṃ kho, bhikkhave, viseso ayaṃ adhippayāso idaṃ nānākaraṇaṃ sutavato ariyasāvakassa assutavatā puthujjanena, yadidaṃ gatiyā upapattiyā sati.}}\\
\begin{addmargin}[1em]{2em}
\setstretch{.5}
{\PaliGlossB{This is the difference between an educated noble disciple and an uneducated ordinary person, that is, when there is a place of rebirth.}}\\
\end{addmargin}
\end{absolutelynopagebreak}

\begin{absolutelynopagebreak}
\setstretch{.7}
{\PaliGlossA{puna caparaṃ, bhikkhave, idhekacco puggalo pītiyā ca virāgā upekkhako ca viharati sato ca sampajāno sukhañca kāyena paṭisaṃvedeti yaṃ taṃ ariyā ācikkhanti: ‘upekkhako satimā sukhavihārī’ti tatiyaṃ jhānaṃ upasampajja viharati.}}\\
\begin{addmargin}[1em]{2em}
\setstretch{.5}
{\PaliGlossB{Furthermore, with the fading away of rapture, they enter and remain in the third absorption, where they meditate with equanimity, mindful and aware, personally experiencing the bliss of which the noble ones declare, ‘Equanimous and mindful, one meditates in bliss.’}}\\
\end{addmargin}
\end{absolutelynopagebreak}

\begin{absolutelynopagebreak}
\setstretch{.7}
{\PaliGlossA{so tadassādeti, taṃ nikāmeti, tena ca vittiṃ āpajjati.}}\\
\begin{addmargin}[1em]{2em}
\setstretch{.5}
{\PaliGlossB{They enjoy it and like it and find it satisfying.}}\\
\end{addmargin}
\end{absolutelynopagebreak}

\begin{absolutelynopagebreak}
\setstretch{.7}
{\PaliGlossA{tattha ṭhito tadadhimutto tabbahulavihārī aparihīno, kālaṃ kurumāno subhakiṇhānaṃ devānaṃ sahabyataṃ upapajjati.}}\\
\begin{addmargin}[1em]{2em}
\setstretch{.5}
{\PaliGlossB{If they abide in that, are committed to it, and meditate on it often without losing it, when they die they’re reborn in the company of the gods replete with glory.}}\\
\end{addmargin}
\end{absolutelynopagebreak}

\begin{absolutelynopagebreak}
\setstretch{.7}
{\PaliGlossA{subhakiṇhānaṃ, bhikkhave, devānaṃ cattāro kappā āyuppamāṇaṃ.}}\\
\begin{addmargin}[1em]{2em}
\setstretch{.5}
{\PaliGlossB{The lifespan of the gods replete with glory is four eons.}}\\
\end{addmargin}
\end{absolutelynopagebreak}

\begin{absolutelynopagebreak}
\setstretch{.7}
{\PaliGlossA{tattha puthujjano yāvatāyukaṃ ṭhatvā yāvatakaṃ tesaṃ devānaṃ āyuppamāṇaṃ taṃ sabbaṃ khepetvā nirayampi gacchati tiracchānayonimpi gacchati pettivisayampi gacchati.}}\\
\begin{addmargin}[1em]{2em}
\setstretch{.5}
{\PaliGlossB{An ordinary person stays there until the lifespan of those gods is spent, then they go to hell or the animal realm or the ghost realm.}}\\
\end{addmargin}
\end{absolutelynopagebreak}

\begin{absolutelynopagebreak}
\setstretch{.7}
{\PaliGlossA{bhagavato pana sāvako tattha yāvatāyukaṃ ṭhatvā yāvatakaṃ tesaṃ devānaṃ āyuppamāṇaṃ taṃ sabbaṃ khepetvā tasmiṃyeva bhave parinibbāyati.}}\\
\begin{addmargin}[1em]{2em}
\setstretch{.5}
{\PaliGlossB{But a disciple of the Buddha stays there until the lifespan of those gods is spent, then they’re extinguished in that very life.}}\\
\end{addmargin}
\end{absolutelynopagebreak}

\begin{absolutelynopagebreak}
\setstretch{.7}
{\PaliGlossA{ayaṃ kho, bhikkhave, viseso ayaṃ adhippayāso idaṃ nānākaraṇaṃ sutavato ariyasāvakassa assutavatā puthujjanena, yadidaṃ gatiyā upapattiyā sati.}}\\
\begin{addmargin}[1em]{2em}
\setstretch{.5}
{\PaliGlossB{This is the difference between an educated noble disciple and an uneducated ordinary person, that is, when there is a place of rebirth.}}\\
\end{addmargin}
\end{absolutelynopagebreak}

\begin{absolutelynopagebreak}
\setstretch{.7}
{\PaliGlossA{puna caparaṃ, bhikkhave, idhekacco puggalo sukhassa ca pahānā dukkhassa ca pahānā pubbeva somanassadomanassānaṃ atthaṅgamā adukkhamasukhaṃ upekkhāsatipārisuddhiṃ catutthaṃ jhānaṃ upasampajja viharati.}}\\
\begin{addmargin}[1em]{2em}
\setstretch{.5}
{\PaliGlossB{Furthermore, giving up pleasure and pain, and ending former happiness and sadness, they enter and remain in the fourth absorption, without pleasure or pain, with pure equanimity and mindfulness.}}\\
\end{addmargin}
\end{absolutelynopagebreak}

\begin{absolutelynopagebreak}
\setstretch{.7}
{\PaliGlossA{so tadassādeti, taṃ nikāmeti, tena ca vittiṃ āpajjati.}}\\
\begin{addmargin}[1em]{2em}
\setstretch{.5}
{\PaliGlossB{They enjoy it and like it and find it satisfying.}}\\
\end{addmargin}
\end{absolutelynopagebreak}

\begin{absolutelynopagebreak}
\setstretch{.7}
{\PaliGlossA{tattha ṭhito tadadhimutto tabbahulavihārī aparihīno kālaṃ kurumāno vehapphalānaṃ devānaṃ sahabyataṃ upapajjati.}}\\
\begin{addmargin}[1em]{2em}
\setstretch{.5}
{\PaliGlossB{If they abide in that, are committed to it, and meditate on it often without losing it, when they die they’re reborn in the company of the gods of abundant fruit.}}\\
\end{addmargin}
\end{absolutelynopagebreak}

\begin{absolutelynopagebreak}
\setstretch{.7}
{\PaliGlossA{vehapphalānaṃ, bhikkhave, devānaṃ pañca kappasatāni āyuppamāṇaṃ.}}\\
\begin{addmargin}[1em]{2em}
\setstretch{.5}
{\PaliGlossB{The lifespan of the gods of abundant fruit is five hundred eons.}}\\
\end{addmargin}
\end{absolutelynopagebreak}

\begin{absolutelynopagebreak}
\setstretch{.7}
{\PaliGlossA{tattha puthujjano yāvatāyukaṃ ṭhatvā yāvatakaṃ tesaṃ devānaṃ āyuppamāṇaṃ taṃ sabbaṃ khepetvā nirayampi gacchati tiracchānayonimpi gacchati pettivisayampi gacchati.}}\\
\begin{addmargin}[1em]{2em}
\setstretch{.5}
{\PaliGlossB{An ordinary person stays there until the lifespan of those gods is spent, then they go to hell or the animal realm or the ghost realm.}}\\
\end{addmargin}
\end{absolutelynopagebreak}

\begin{absolutelynopagebreak}
\setstretch{.7}
{\PaliGlossA{bhagavato pana sāvako tattha yāvatāyukaṃ ṭhatvā yāvatakaṃ tesaṃ devānaṃ āyuppamāṇaṃ taṃ sabbaṃ khepetvā tasmiṃyeva bhave parinibbāyati.}}\\
\begin{addmargin}[1em]{2em}
\setstretch{.5}
{\PaliGlossB{But a disciple of the Buddha stays there until the lifespan of those gods is spent, then they’re extinguished in that very life.}}\\
\end{addmargin}
\end{absolutelynopagebreak}

\begin{absolutelynopagebreak}
\setstretch{.7}
{\PaliGlossA{ayaṃ kho, bhikkhave, viseso ayaṃ adhippayāso idaṃ nānākaraṇaṃ sutavato ariyasāvakassa assutavatā puthujjanena, yadidaṃ gatiyā upapattiyā sati.}}\\
\begin{addmargin}[1em]{2em}
\setstretch{.5}
{\PaliGlossB{This is the difference between an educated noble disciple and an uneducated ordinary person, that is, when there is a place of rebirth.}}\\
\end{addmargin}
\end{absolutelynopagebreak}

\begin{absolutelynopagebreak}
\setstretch{.7}
{\PaliGlossA{ime kho, bhikkhave, cattāro puggalā santo saṃvijjamānā lokasmin”ti.}}\\
\begin{addmargin}[1em]{2em}
\setstretch{.5}
{\PaliGlossB{These are the four people found in the world.”}}\\
\end{addmargin}
\end{absolutelynopagebreak}

\begin{absolutelynopagebreak}
\setstretch{.7}
{\PaliGlossA{tatiyaṃ.}}\\
\begin{addmargin}[1em]{2em}
\setstretch{.5}
{\PaliGlossB{    -}}\\
\end{addmargin}
\end{absolutelynopagebreak}
