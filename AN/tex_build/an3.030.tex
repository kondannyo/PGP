
\begin{absolutelynopagebreak}
\setstretch{.7}
{\PaliGlossA{aṅguttara nikāya 3}}\\
\begin{addmargin}[1em]{2em}
\setstretch{.5}
{\PaliGlossB{Numbered Discourses 3}}\\
\end{addmargin}
\end{absolutelynopagebreak}

\begin{absolutelynopagebreak}
\setstretch{.7}
{\PaliGlossA{3. puggalavagga}}\\
\begin{addmargin}[1em]{2em}
\setstretch{.5}
{\PaliGlossB{3. Persons}}\\
\end{addmargin}
\end{absolutelynopagebreak}

\begin{absolutelynopagebreak}
\setstretch{.7}
{\PaliGlossA{30. avakujjasutta}}\\
\begin{addmargin}[1em]{2em}
\setstretch{.5}
{\PaliGlossB{30. Upside-down}}\\
\end{addmargin}
\end{absolutelynopagebreak}

\begin{absolutelynopagebreak}
\setstretch{.7}
{\PaliGlossA{“tayome, bhikkhave, puggalā santo saṃvijjamānā lokasmiṃ.}}\\
\begin{addmargin}[1em]{2em}
\setstretch{.5}
{\PaliGlossB{“These three kinds of people are found in the world.}}\\
\end{addmargin}
\end{absolutelynopagebreak}

\begin{absolutelynopagebreak}
\setstretch{.7}
{\PaliGlossA{katame tayo?}}\\
\begin{addmargin}[1em]{2em}
\setstretch{.5}
{\PaliGlossB{What three?}}\\
\end{addmargin}
\end{absolutelynopagebreak}

\begin{absolutelynopagebreak}
\setstretch{.7}
{\PaliGlossA{avakujjapañño puggalo, ucchaṅgapañño puggalo, puthupañño puggalo.}}\\
\begin{addmargin}[1em]{2em}
\setstretch{.5}
{\PaliGlossB{One with upside-down wisdom, one with wisdom on their lap, and one with wide wisdom.}}\\
\end{addmargin}
\end{absolutelynopagebreak}

\begin{absolutelynopagebreak}
\setstretch{.7}
{\PaliGlossA{katamo ca, bhikkhave, avakujjapañño puggalo?}}\\
\begin{addmargin}[1em]{2em}
\setstretch{.5}
{\PaliGlossB{And who is the person with upside-down wisdom?}}\\
\end{addmargin}
\end{absolutelynopagebreak}

\begin{absolutelynopagebreak}
\setstretch{.7}
{\PaliGlossA{idha, bhikkhave, ekacco puggalo ārāmaṃ gantā hoti abhikkhaṇaṃ bhikkhūnaṃ santike dhammassavanāya.}}\\
\begin{addmargin}[1em]{2em}
\setstretch{.5}
{\PaliGlossB{It’s someone who often goes to the monastery to hear the teaching in the presence of the mendicants.}}\\
\end{addmargin}
\end{absolutelynopagebreak}

\begin{absolutelynopagebreak}
\setstretch{.7}
{\PaliGlossA{tassa bhikkhū dhammaṃ desenti ādikalyāṇaṃ majjhekalyāṇaṃ pariyosānakalyāṇaṃ sātthaṃ sabyañjanaṃ, kevalaparipuṇṇaṃ parisuddhaṃ brahmacariyaṃ pakāsenti.}}\\
\begin{addmargin}[1em]{2em}
\setstretch{.5}
{\PaliGlossB{The mendicants teach them Dhamma that’s good in the beginning, good in the middle, and good in the end, meaningful and well-phrased. And they reveal a spiritual practice that’s entirely full and pure.}}\\
\end{addmargin}
\end{absolutelynopagebreak}

\begin{absolutelynopagebreak}
\setstretch{.7}
{\PaliGlossA{so tasmiṃ āsane nisinno tassā kathāya neva ādiṃ manasi karoti, na majjhaṃ manasi karoti, na pariyosānaṃ manasi karoti;}}\\
\begin{addmargin}[1em]{2em}
\setstretch{.5}
{\PaliGlossB{But even while sitting there, that person doesn’t pay attention to the beginning, middle, or end of the discussion.}}\\
\end{addmargin}
\end{absolutelynopagebreak}

\begin{absolutelynopagebreak}
\setstretch{.7}
{\PaliGlossA{vuṭṭhitopi tamhā āsanā tassā kathāya neva ādiṃ manasi karoti, na majjhaṃ manasi karoti, na pariyosānaṃ manasi karoti.}}\\
\begin{addmargin}[1em]{2em}
\setstretch{.5}
{\PaliGlossB{And when they get up from their seat, they don’t pay attention to the beginning, middle, or end of the discussion.}}\\
\end{addmargin}
\end{absolutelynopagebreak}

\begin{absolutelynopagebreak}
\setstretch{.7}
{\PaliGlossA{seyyathāpi, bhikkhave, kumbho nikujjo tatra udakaṃ āsittaṃ vivaṭṭati, no saṇṭhāti;}}\\
\begin{addmargin}[1em]{2em}
\setstretch{.5}
{\PaliGlossB{It’s like when a pot full of water is tipped over, so the water drains out and doesn’t stay.}}\\
\end{addmargin}
\end{absolutelynopagebreak}

\begin{absolutelynopagebreak}
\setstretch{.7}
{\PaliGlossA{evamevaṃ kho, bhikkhave, idhekacco puggalo ārāmaṃ gantā hoti abhikkhaṇaṃ bhikkhūnaṃ santike dhammassavanāya.}}\\
\begin{addmargin}[1em]{2em}
\setstretch{.5}
{\PaliGlossB{In the same way, someone often goes to the monastery to hear the teaching in the presence of the mendicants.}}\\
\end{addmargin}
\end{absolutelynopagebreak}

\begin{absolutelynopagebreak}
\setstretch{.7}
{\PaliGlossA{tassa bhikkhū dhammaṃ desenti ādikalyāṇaṃ majjhekalyāṇaṃ pariyosānakalyāṇaṃ sātthaṃ sabyañjanaṃ, kevalaparipuṇṇaṃ parisuddhaṃ brahmacariyaṃ pakāsenti.}}\\
\begin{addmargin}[1em]{2em}
\setstretch{.5}
{\PaliGlossB{The mendicants teach them Dhamma that’s good in the beginning, good in the middle, and good in the end, meaningful and well-phrased. And they reveal a spiritual practice that’s entirely full and pure.}}\\
\end{addmargin}
\end{absolutelynopagebreak}

\begin{absolutelynopagebreak}
\setstretch{.7}
{\PaliGlossA{so tasmiṃ āsane nisinno tassā kathāya neva ādiṃ manasi karoti, na majjhaṃ manasi karoti, na pariyosānaṃ manasi karoti;}}\\
\begin{addmargin}[1em]{2em}
\setstretch{.5}
{\PaliGlossB{But even while sitting there, that person doesn’t pay attention to the discussion in the beginning, middle, or end.}}\\
\end{addmargin}
\end{absolutelynopagebreak}

\begin{absolutelynopagebreak}
\setstretch{.7}
{\PaliGlossA{vuṭṭhitopi tamhā āsanā tassā kathāya neva ādiṃ manasi karoti, na majjhaṃ manasi karoti, na pariyosānaṃ manasi karoti.}}\\
\begin{addmargin}[1em]{2em}
\setstretch{.5}
{\PaliGlossB{And when they get up from their seat, they don’t pay attention to the beginning, middle, or end of the discussion.}}\\
\end{addmargin}
\end{absolutelynopagebreak}

\begin{absolutelynopagebreak}
\setstretch{.7}
{\PaliGlossA{ayaṃ vuccati, bhikkhave, avakujjapañño puggalo.}}\\
\begin{addmargin}[1em]{2em}
\setstretch{.5}
{\PaliGlossB{This is called a person with upside-down wisdom.}}\\
\end{addmargin}
\end{absolutelynopagebreak}

\begin{absolutelynopagebreak}
\setstretch{.7}
{\PaliGlossA{katamo ca, bhikkhave, ucchaṅgapañño puggalo?}}\\
\begin{addmargin}[1em]{2em}
\setstretch{.5}
{\PaliGlossB{And who is the person with wisdom on their lap?}}\\
\end{addmargin}
\end{absolutelynopagebreak}

\begin{absolutelynopagebreak}
\setstretch{.7}
{\PaliGlossA{idha, bhikkhave, ekacco puggalo ārāmaṃ gantā hoti abhikkhaṇaṃ bhikkhūnaṃ santike dhammassavanāya.}}\\
\begin{addmargin}[1em]{2em}
\setstretch{.5}
{\PaliGlossB{It’s someone who often goes to the monastery to hear the teaching in the presence of the mendicants.}}\\
\end{addmargin}
\end{absolutelynopagebreak}

\begin{absolutelynopagebreak}
\setstretch{.7}
{\PaliGlossA{tassa bhikkhū dhammaṃ desenti ādikalyāṇaṃ majjhekalyāṇaṃ pariyosānakalyāṇaṃ sātthaṃ sabyañjanaṃ, kevalaparipuṇṇaṃ parisuddhaṃ brahmacariyaṃ pakāsenti.}}\\
\begin{addmargin}[1em]{2em}
\setstretch{.5}
{\PaliGlossB{The mendicants teach them Dhamma that’s good in the beginning, good in the middle, and good in the end, meaningful and well-phrased. And they reveal a spiritual practice that’s entirely full and pure.}}\\
\end{addmargin}
\end{absolutelynopagebreak}

\begin{absolutelynopagebreak}
\setstretch{.7}
{\PaliGlossA{so tasmiṃ āsane nisinno tassā kathāya ādimpi manasi karoti, majjhampi manasi karoti, pariyosānampi manasi karoti;}}\\
\begin{addmargin}[1em]{2em}
\setstretch{.5}
{\PaliGlossB{While sitting there, that person pays attention to the discussion in the beginning, middle, and end.}}\\
\end{addmargin}
\end{absolutelynopagebreak}

\begin{absolutelynopagebreak}
\setstretch{.7}
{\PaliGlossA{vuṭṭhito ca kho tamhā āsanā tassā kathāya neva ādiṃ manasi karoti, na majjhaṃ manasi karoti, na pariyosānaṃ manasi karoti.}}\\
\begin{addmargin}[1em]{2em}
\setstretch{.5}
{\PaliGlossB{But when they get up from their seat, they don’t pay attention to the beginning, middle, or end of the discussion.}}\\
\end{addmargin}
\end{absolutelynopagebreak}

\begin{absolutelynopagebreak}
\setstretch{.7}
{\PaliGlossA{seyyathāpi, bhikkhave, purisassa ucchaṅge nānākhajjakāni ākiṇṇāni—}}\\
\begin{addmargin}[1em]{2em}
\setstretch{.5}
{\PaliGlossB{It’s like a person who has different kinds of food crammed on their lap—}}\\
\end{addmargin}
\end{absolutelynopagebreak}

\begin{absolutelynopagebreak}
\setstretch{.7}
{\PaliGlossA{tilā taṇḍulā modakā badarā.}}\\
\begin{addmargin}[1em]{2em}
\setstretch{.5}
{\PaliGlossB{such as sesame, rice, sweets, or jujube—}}\\
\end{addmargin}
\end{absolutelynopagebreak}

\begin{absolutelynopagebreak}
\setstretch{.7}
{\PaliGlossA{so tamhā āsanā vuṭṭhahanto satisammosā pakireyya.}}\\
\begin{addmargin}[1em]{2em}
\setstretch{.5}
{\PaliGlossB{so that if they get up from the seat without mindfulness, everything gets scattered.}}\\
\end{addmargin}
\end{absolutelynopagebreak}

\begin{absolutelynopagebreak}
\setstretch{.7}
{\PaliGlossA{evamevaṃ kho, bhikkhave, idhekacco puggalo ārāmaṃ gantā hoti abhikkhaṇaṃ bhikkhūnaṃ santike dhammassavanāya.}}\\
\begin{addmargin}[1em]{2em}
\setstretch{.5}
{\PaliGlossB{In the same way, someone often goes to the monastery to hear the teaching in the presence of the mendicants.}}\\
\end{addmargin}
\end{absolutelynopagebreak}

\begin{absolutelynopagebreak}
\setstretch{.7}
{\PaliGlossA{tassa bhikkhū dhammaṃ desenti ādikalyāṇaṃ majjhekalyāṇaṃ pariyosānakalyāṇaṃ sātthaṃ sabyañjanaṃ, kevalaparipuṇṇaṃ parisuddhaṃ brahmacariyaṃ pakāsenti.}}\\
\begin{addmargin}[1em]{2em}
\setstretch{.5}
{\PaliGlossB{The mendicants teach them Dhamma that’s good in the beginning, good in the middle, and good in the end, meaningful and well-phrased. And they reveal a spiritual practice that’s entirely full and pure.}}\\
\end{addmargin}
\end{absolutelynopagebreak}

\begin{absolutelynopagebreak}
\setstretch{.7}
{\PaliGlossA{so tasmiṃ āsane nisinno tassā kathāya ādimpi manasi karoti, majjhampi manasi karoti, pariyosānampi manasi karoti;}}\\
\begin{addmargin}[1em]{2em}
\setstretch{.5}
{\PaliGlossB{While sitting there, that person pays attention to the discussion in the beginning, middle, and end.}}\\
\end{addmargin}
\end{absolutelynopagebreak}

\begin{absolutelynopagebreak}
\setstretch{.7}
{\PaliGlossA{vuṭṭhito ca kho tamhā āsanā tassā kathāya neva ādiṃ manasi karoti, na majjhaṃ manasi karoti, na pariyosānaṃ manasi karoti.}}\\
\begin{addmargin}[1em]{2em}
\setstretch{.5}
{\PaliGlossB{But when they get up from their seat, they don’t pay attention to the beginning, middle, or end of the discussion.}}\\
\end{addmargin}
\end{absolutelynopagebreak}

\begin{absolutelynopagebreak}
\setstretch{.7}
{\PaliGlossA{ayaṃ vuccati, bhikkhave, ucchaṅgapañño puggalo.}}\\
\begin{addmargin}[1em]{2em}
\setstretch{.5}
{\PaliGlossB{This is called a person with wisdom on their lap.}}\\
\end{addmargin}
\end{absolutelynopagebreak}

\begin{absolutelynopagebreak}
\setstretch{.7}
{\PaliGlossA{katamo ca, bhikkhave, puthupañño puggalo?}}\\
\begin{addmargin}[1em]{2em}
\setstretch{.5}
{\PaliGlossB{And who is the person with wide wisdom?}}\\
\end{addmargin}
\end{absolutelynopagebreak}

\begin{absolutelynopagebreak}
\setstretch{.7}
{\PaliGlossA{idha, bhikkhave, ekacco puggalo ārāmaṃ gantā hoti abhikkhaṇaṃ bhikkhūnaṃ santike dhammassavanāya.}}\\
\begin{addmargin}[1em]{2em}
\setstretch{.5}
{\PaliGlossB{It’s someone who often goes to the monastery to hear the teaching in the presence of the mendicants.}}\\
\end{addmargin}
\end{absolutelynopagebreak}

\begin{absolutelynopagebreak}
\setstretch{.7}
{\PaliGlossA{tassa bhikkhū dhammaṃ desenti ādikalyāṇaṃ majjhekalyāṇaṃ pariyosānakalyāṇaṃ sātthaṃ sabyañjanaṃ, kevalaparipuṇṇaṃ parisuddhaṃ brahmacariyaṃ pakāsenti.}}\\
\begin{addmargin}[1em]{2em}
\setstretch{.5}
{\PaliGlossB{The mendicants teach them Dhamma that’s good in the beginning, good in the middle, and good in the end, meaningful and well-phrased. And they reveal a spiritual practice that’s entirely full and pure.}}\\
\end{addmargin}
\end{absolutelynopagebreak}

\begin{absolutelynopagebreak}
\setstretch{.7}
{\PaliGlossA{so tasmiṃ āsane nisinno tassā kathāya ādimpi manasi karoti, majjhampi manasi karoti, pariyosānampi manasi karoti;}}\\
\begin{addmargin}[1em]{2em}
\setstretch{.5}
{\PaliGlossB{While sitting there, that person pays attention to the discussion in the beginning, middle, and end.}}\\
\end{addmargin}
\end{absolutelynopagebreak}

\begin{absolutelynopagebreak}
\setstretch{.7}
{\PaliGlossA{vuṭṭhitopi tamhā āsanā tassā kathāya ādimpi manasi karoti, majjhampi manasi karoti, pariyosānampi manasi karoti.}}\\
\begin{addmargin}[1em]{2em}
\setstretch{.5}
{\PaliGlossB{And when they get up from their seat, they continue to pay attention to the beginning, middle, or end of the discussion.}}\\
\end{addmargin}
\end{absolutelynopagebreak}

\begin{absolutelynopagebreak}
\setstretch{.7}
{\PaliGlossA{seyyathāpi, bhikkhave, kumbho ukkujjo tatra udakaṃ āsittaṃ saṇṭhāti no vivaṭṭati;}}\\
\begin{addmargin}[1em]{2em}
\setstretch{.5}
{\PaliGlossB{It’s like when a pot full of water is set straight, so the water stays and doesn’t drain out.}}\\
\end{addmargin}
\end{absolutelynopagebreak}

\begin{absolutelynopagebreak}
\setstretch{.7}
{\PaliGlossA{evamevaṃ kho, bhikkhave, idhekacco puggalo ārāmaṃ gantā hoti abhikkhaṇaṃ bhikkhūnaṃ santike dhammassavanāya.}}\\
\begin{addmargin}[1em]{2em}
\setstretch{.5}
{\PaliGlossB{In the same way, someone often goes to the monastery to hear the teaching in the presence of the mendicants.}}\\
\end{addmargin}
\end{absolutelynopagebreak}

\begin{absolutelynopagebreak}
\setstretch{.7}
{\PaliGlossA{tassa bhikkhū dhammaṃ desenti ādikalyāṇaṃ majjhekalyāṇaṃ pariyosānakalyāṇaṃ sātthaṃ sabyañjanaṃ, kevalaparipuṇṇaṃ parisuddhaṃ brahmacariyaṃ pakāsenti.}}\\
\begin{addmargin}[1em]{2em}
\setstretch{.5}
{\PaliGlossB{The mendicants teach them Dhamma that’s good in the beginning, good in the middle, and good in the end, meaningful and well-phrased. And they reveal a spiritual practice that’s entirely full and pure.}}\\
\end{addmargin}
\end{absolutelynopagebreak}

\begin{absolutelynopagebreak}
\setstretch{.7}
{\PaliGlossA{so tasmiṃ āsane nisinno tassā kathāya ādimpi manasi karoti, majjhampi manasi karoti, pariyosānampi manasi karoti;}}\\
\begin{addmargin}[1em]{2em}
\setstretch{.5}
{\PaliGlossB{While sitting there, that person pays attention to the discussion in the beginning, middle, and end.}}\\
\end{addmargin}
\end{absolutelynopagebreak}

\begin{absolutelynopagebreak}
\setstretch{.7}
{\PaliGlossA{vuṭṭhitopi tamhā āsanā tassā kathāya ādimpi manasi karoti, majjhampi manasi karoti, pariyosānampi manasi karoti.}}\\
\begin{addmargin}[1em]{2em}
\setstretch{.5}
{\PaliGlossB{And when they get up from their seat, they continue to pay attention to the beginning, middle, or end of the discussion.}}\\
\end{addmargin}
\end{absolutelynopagebreak}

\begin{absolutelynopagebreak}
\setstretch{.7}
{\PaliGlossA{ayaṃ vuccati, bhikkhave, puthupañño puggalo.}}\\
\begin{addmargin}[1em]{2em}
\setstretch{.5}
{\PaliGlossB{This is called a person with wide wisdom.}}\\
\end{addmargin}
\end{absolutelynopagebreak}

\begin{absolutelynopagebreak}
\setstretch{.7}
{\PaliGlossA{ime kho, bhikkhave, tayo puggalā santo saṃvijjamānā lokasminti.}}\\
\begin{addmargin}[1em]{2em}
\setstretch{.5}
{\PaliGlossB{These are the three kinds of people found in the world.}}\\
\end{addmargin}
\end{absolutelynopagebreak}

\begin{absolutelynopagebreak}
\setstretch{.7}
{\PaliGlossA{avakujjapañño puriso,}}\\
\begin{addmargin}[1em]{2em}
\setstretch{.5}
{\PaliGlossB{A person with upside-down wisdom,}}\\
\end{addmargin}
\end{absolutelynopagebreak}

\begin{absolutelynopagebreak}
\setstretch{.7}
{\PaliGlossA{dummedho avicakkhaṇo;}}\\
\begin{addmargin}[1em]{2em}
\setstretch{.5}
{\PaliGlossB{is stupid and cannot see,}}\\
\end{addmargin}
\end{absolutelynopagebreak}

\begin{absolutelynopagebreak}
\setstretch{.7}
{\PaliGlossA{abhikkhaṇampi ce hoti,}}\\
\begin{addmargin}[1em]{2em}
\setstretch{.5}
{\PaliGlossB{and even if they frequently}}\\
\end{addmargin}
\end{absolutelynopagebreak}

\begin{absolutelynopagebreak}
\setstretch{.7}
{\PaliGlossA{gantā bhikkhūna santike.}}\\
\begin{addmargin}[1em]{2em}
\setstretch{.5}
{\PaliGlossB{go into the mendicants’ presence,}}\\
\end{addmargin}
\end{absolutelynopagebreak}

\begin{absolutelynopagebreak}
\setstretch{.7}
{\PaliGlossA{ādiṃ kathāya majjhañca,}}\\
\begin{addmargin}[1em]{2em}
\setstretch{.5}
{\PaliGlossB{such a person can’t learn}}\\
\end{addmargin}
\end{absolutelynopagebreak}

\begin{absolutelynopagebreak}
\setstretch{.7}
{\PaliGlossA{pariyosānañca tādiso;}}\\
\begin{addmargin}[1em]{2em}
\setstretch{.5}
{\PaliGlossB{the beginning, middle, or end}}\\
\end{addmargin}
\end{absolutelynopagebreak}

\begin{absolutelynopagebreak}
\setstretch{.7}
{\PaliGlossA{uggahetuṃ na sakkoti,}}\\
\begin{addmargin}[1em]{2em}
\setstretch{.5}
{\PaliGlossB{of the discussion,}}\\
\end{addmargin}
\end{absolutelynopagebreak}

\begin{absolutelynopagebreak}
\setstretch{.7}
{\PaliGlossA{paññā hissa na vijjati.}}\\
\begin{addmargin}[1em]{2em}
\setstretch{.5}
{\PaliGlossB{for their wisdom is lacking.}}\\
\end{addmargin}
\end{absolutelynopagebreak}

\begin{absolutelynopagebreak}
\setstretch{.7}
{\PaliGlossA{ucchaṅgapañño puriso,}}\\
\begin{addmargin}[1em]{2em}
\setstretch{.5}
{\PaliGlossB{The person with wisdom on their lap}}\\
\end{addmargin}
\end{absolutelynopagebreak}

\begin{absolutelynopagebreak}
\setstretch{.7}
{\PaliGlossA{seyyo etena vuccati;}}\\
\begin{addmargin}[1em]{2em}
\setstretch{.5}
{\PaliGlossB{is better than that, it’s said;}}\\
\end{addmargin}
\end{absolutelynopagebreak}

\begin{absolutelynopagebreak}
\setstretch{.7}
{\PaliGlossA{abhikkhaṇampi ce hoti,}}\\
\begin{addmargin}[1em]{2em}
\setstretch{.5}
{\PaliGlossB{but even if they frequently}}\\
\end{addmargin}
\end{absolutelynopagebreak}

\begin{absolutelynopagebreak}
\setstretch{.7}
{\PaliGlossA{gantā bhikkhūna santike.}}\\
\begin{addmargin}[1em]{2em}
\setstretch{.5}
{\PaliGlossB{go into the mendicants’ presence,}}\\
\end{addmargin}
\end{absolutelynopagebreak}

\begin{absolutelynopagebreak}
\setstretch{.7}
{\PaliGlossA{ādiṃ kathāya majjhañca,}}\\
\begin{addmargin}[1em]{2em}
\setstretch{.5}
{\PaliGlossB{such a person can only learn}}\\
\end{addmargin}
\end{absolutelynopagebreak}

\begin{absolutelynopagebreak}
\setstretch{.7}
{\PaliGlossA{pariyosānañca tādiso;}}\\
\begin{addmargin}[1em]{2em}
\setstretch{.5}
{\PaliGlossB{the beginning, middle, and end}}\\
\end{addmargin}
\end{absolutelynopagebreak}

\begin{absolutelynopagebreak}
\setstretch{.7}
{\PaliGlossA{nisinno āsane tasmiṃ,}}\\
\begin{addmargin}[1em]{2em}
\setstretch{.5}
{\PaliGlossB{while sitting in that seat;}}\\
\end{addmargin}
\end{absolutelynopagebreak}

\begin{absolutelynopagebreak}
\setstretch{.7}
{\PaliGlossA{uggahetvāna byañjanaṃ;}}\\
\begin{addmargin}[1em]{2em}
\setstretch{.5}
{\PaliGlossB{but they’ve only grasped the phrasing,}}\\
\end{addmargin}
\end{absolutelynopagebreak}

\begin{absolutelynopagebreak}
\setstretch{.7}
{\PaliGlossA{vuṭṭhito nappajānāti,}}\\
\begin{addmargin}[1em]{2em}
\setstretch{.5}
{\PaliGlossB{for when they get up their understanding fails,}}\\
\end{addmargin}
\end{absolutelynopagebreak}

\begin{absolutelynopagebreak}
\setstretch{.7}
{\PaliGlossA{gahitaṃ hissa mussati.}}\\
\begin{addmargin}[1em]{2em}
\setstretch{.5}
{\PaliGlossB{and what they’ve learned is lost.}}\\
\end{addmargin}
\end{absolutelynopagebreak}

\begin{absolutelynopagebreak}
\setstretch{.7}
{\PaliGlossA{puthupañño ca puriso,}}\\
\begin{addmargin}[1em]{2em}
\setstretch{.5}
{\PaliGlossB{The person with wide wisdom}}\\
\end{addmargin}
\end{absolutelynopagebreak}

\begin{absolutelynopagebreak}
\setstretch{.7}
{\PaliGlossA{seyyo etehi vuccati;}}\\
\begin{addmargin}[1em]{2em}
\setstretch{.5}
{\PaliGlossB{is better than that, it’s said;}}\\
\end{addmargin}
\end{absolutelynopagebreak}

\begin{absolutelynopagebreak}
\setstretch{.7}
{\PaliGlossA{abhikkhaṇampi ce hoti,}}\\
\begin{addmargin}[1em]{2em}
\setstretch{.5}
{\PaliGlossB{and if they, too, frequently}}\\
\end{addmargin}
\end{absolutelynopagebreak}

\begin{absolutelynopagebreak}
\setstretch{.7}
{\PaliGlossA{gantā bhikkhūna santike.}}\\
\begin{addmargin}[1em]{2em}
\setstretch{.5}
{\PaliGlossB{go into the mendicants’ presence,}}\\
\end{addmargin}
\end{absolutelynopagebreak}

\begin{absolutelynopagebreak}
\setstretch{.7}
{\PaliGlossA{ādiṃ kathāya majjhañca,}}\\
\begin{addmargin}[1em]{2em}
\setstretch{.5}
{\PaliGlossB{such a person can learn}}\\
\end{addmargin}
\end{absolutelynopagebreak}

\begin{absolutelynopagebreak}
\setstretch{.7}
{\PaliGlossA{pariyosānañca tādiso;}}\\
\begin{addmargin}[1em]{2em}
\setstretch{.5}
{\PaliGlossB{the beginning, middle, and end}}\\
\end{addmargin}
\end{absolutelynopagebreak}

\begin{absolutelynopagebreak}
\setstretch{.7}
{\PaliGlossA{nisinno āsane tasmiṃ,}}\\
\begin{addmargin}[1em]{2em}
\setstretch{.5}
{\PaliGlossB{while sitting in that seat;}}\\
\end{addmargin}
\end{absolutelynopagebreak}

\begin{absolutelynopagebreak}
\setstretch{.7}
{\PaliGlossA{uggahetvāna byañjanaṃ.}}\\
\begin{addmargin}[1em]{2em}
\setstretch{.5}
{\PaliGlossB{and when they’ve grasped the phrasing,}}\\
\end{addmargin}
\end{absolutelynopagebreak}

\begin{absolutelynopagebreak}
\setstretch{.7}
{\PaliGlossA{dhāreti seṭṭhasaṅkappo,}}\\
\begin{addmargin}[1em]{2em}
\setstretch{.5}
{\PaliGlossB{they remember it with the best of intentions.}}\\
\end{addmargin}
\end{absolutelynopagebreak}

\begin{absolutelynopagebreak}
\setstretch{.7}
{\PaliGlossA{abyaggamānaso naro;}}\\
\begin{addmargin}[1em]{2em}
\setstretch{.5}
{\PaliGlossB{That peaceful-hearted person,}}\\
\end{addmargin}
\end{absolutelynopagebreak}

\begin{absolutelynopagebreak}
\setstretch{.7}
{\PaliGlossA{dhammānudhammappaṭipanno,}}\\
\begin{addmargin}[1em]{2em}
\setstretch{.5}
{\PaliGlossB{practicing in line with the teaching,}}\\
\end{addmargin}
\end{absolutelynopagebreak}

\begin{absolutelynopagebreak}
\setstretch{.7}
{\PaliGlossA{dukkhassantakaro siyā”ti.}}\\
\begin{addmargin}[1em]{2em}
\setstretch{.5}
{\PaliGlossB{would make an end of suffering.”}}\\
\end{addmargin}
\end{absolutelynopagebreak}

\begin{absolutelynopagebreak}
\setstretch{.7}
{\PaliGlossA{dasamaṃ.}}\\
\begin{addmargin}[1em]{2em}
\setstretch{.5}
{\PaliGlossB{    -}}\\
\end{addmargin}
\end{absolutelynopagebreak}

\begin{absolutelynopagebreak}
\setstretch{.7}
{\PaliGlossA{puggalavaggo tatiyo.}}\\
\begin{addmargin}[1em]{2em}
\setstretch{.5}
{\PaliGlossB{    -}}\\
\end{addmargin}
\end{absolutelynopagebreak}

\begin{absolutelynopagebreak}
\setstretch{.7}
{\PaliGlossA{samiddha gilāna saṅkhārā,}}\\
\begin{addmargin}[1em]{2em}
\setstretch{.5}
{\PaliGlossB{    -}}\\
\end{addmargin}
\end{absolutelynopagebreak}

\begin{absolutelynopagebreak}
\setstretch{.7}
{\PaliGlossA{bahukārā vajirena ca;}}\\
\begin{addmargin}[1em]{2em}
\setstretch{.5}
{\PaliGlossB{    -}}\\
\end{addmargin}
\end{absolutelynopagebreak}

\begin{absolutelynopagebreak}
\setstretch{.7}
{\PaliGlossA{sevi jiguccha gūthabhāṇī,}}\\
\begin{addmargin}[1em]{2em}
\setstretch{.5}
{\PaliGlossB{    -}}\\
\end{addmargin}
\end{absolutelynopagebreak}

\begin{absolutelynopagebreak}
\setstretch{.7}
{\PaliGlossA{andho ca avakujjatāti.}}\\
\begin{addmargin}[1em]{2em}
\setstretch{.5}
{\PaliGlossB{    -}}\\
\end{addmargin}
\end{absolutelynopagebreak}
