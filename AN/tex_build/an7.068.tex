
\begin{absolutelynopagebreak}
\setstretch{.7}
{\PaliGlossA{aṅguttara nikāya 7}}\\
\begin{addmargin}[1em]{2em}
\setstretch{.5}
{\PaliGlossB{Numbered Discourses 7}}\\
\end{addmargin}
\end{absolutelynopagebreak}

\begin{absolutelynopagebreak}
\setstretch{.7}
{\PaliGlossA{7. mahāvagga}}\\
\begin{addmargin}[1em]{2em}
\setstretch{.5}
{\PaliGlossB{7. The Great Chapter}}\\
\end{addmargin}
\end{absolutelynopagebreak}

\begin{absolutelynopagebreak}
\setstretch{.7}
{\PaliGlossA{68. dhammaññūsutta}}\\
\begin{addmargin}[1em]{2em}
\setstretch{.5}
{\PaliGlossB{68. One Who Knows the Teachings}}\\
\end{addmargin}
\end{absolutelynopagebreak}

\begin{absolutelynopagebreak}
\setstretch{.7}
{\PaliGlossA{“sattahi, bhikkhave, dhammehi samannāgato bhikkhu āhuneyyo hoti … pe … anuttaraṃ puññakkhettaṃ lokassa.}}\\
\begin{addmargin}[1em]{2em}
\setstretch{.5}
{\PaliGlossB{“A mendicant with seven qualities is worthy of offerings dedicated to the gods, worthy of hospitality, worthy of a religious donation, worthy of veneration with joined palms, and is the supreme field of merit for the world.}}\\
\end{addmargin}
\end{absolutelynopagebreak}

\begin{absolutelynopagebreak}
\setstretch{.7}
{\PaliGlossA{katamehi sattahi?}}\\
\begin{addmargin}[1em]{2em}
\setstretch{.5}
{\PaliGlossB{What seven?}}\\
\end{addmargin}
\end{absolutelynopagebreak}

\begin{absolutelynopagebreak}
\setstretch{.7}
{\PaliGlossA{idha, bhikkhave, bhikkhu dhammaññū ca hoti atthaññū ca attaññū ca mattaññū ca kālaññū ca parisaññū ca puggalaparoparaññū ca.}}\\
\begin{addmargin}[1em]{2em}
\setstretch{.5}
{\PaliGlossB{It’s when a mendicant knows the teachings, knows the meaning, has self-knowledge, knows moderation, knows the right time, knows assemblies, and knows people high and low.}}\\
\end{addmargin}
\end{absolutelynopagebreak}

\begin{absolutelynopagebreak}
\setstretch{.7}
{\PaliGlossA{kathañca, bhikkhave, bhikkhu dhammaññū hoti?}}\\
\begin{addmargin}[1em]{2em}
\setstretch{.5}
{\PaliGlossB{And how is a mendicant one who knows the teachings?}}\\
\end{addmargin}
\end{absolutelynopagebreak}

\begin{absolutelynopagebreak}
\setstretch{.7}
{\PaliGlossA{idha, bhikkhave, bhikkhu dhammaṃ jānāti—}}\\
\begin{addmargin}[1em]{2em}
\setstretch{.5}
{\PaliGlossB{It’s when a mendicant knows the teachings:}}\\
\end{addmargin}
\end{absolutelynopagebreak}

\begin{absolutelynopagebreak}
\setstretch{.7}
{\PaliGlossA{suttaṃ geyyaṃ veyyākaraṇaṃ gāthaṃ udānaṃ itivuttakaṃ jātakaṃ abbhutadhammaṃ vedallaṃ.}}\\
\begin{addmargin}[1em]{2em}
\setstretch{.5}
{\PaliGlossB{statements, songs, discussions, verses, inspired exclamations, legends, stories of past lives, amazing stories, and classifications.}}\\
\end{addmargin}
\end{absolutelynopagebreak}

\begin{absolutelynopagebreak}
\setstretch{.7}
{\PaliGlossA{no ce, bhikkhave, bhikkhu dhammaṃ jāneyya—}}\\
\begin{addmargin}[1em]{2em}
\setstretch{.5}
{\PaliGlossB{If a mendicant did not know these teachings,}}\\
\end{addmargin}
\end{absolutelynopagebreak}

\begin{absolutelynopagebreak}
\setstretch{.7}
{\PaliGlossA{suttaṃ geyyaṃ … pe … abbhutadhammaṃ vedallaṃ, nayidha ‘dhammaññū’ti vucceyya.}}\\
\begin{addmargin}[1em]{2em}
\setstretch{.5}
{\PaliGlossB{they would not be called ‘one who knows the teachings’.}}\\
\end{addmargin}
\end{absolutelynopagebreak}

\begin{absolutelynopagebreak}
\setstretch{.7}
{\PaliGlossA{yasmā ca kho, bhikkhave, bhikkhu dhammaṃ jānāti—}}\\
\begin{addmargin}[1em]{2em}
\setstretch{.5}
{\PaliGlossB{But because they do know these teachings,}}\\
\end{addmargin}
\end{absolutelynopagebreak}

\begin{absolutelynopagebreak}
\setstretch{.7}
{\PaliGlossA{suttaṃ geyyaṃ … pe … abbhutadhammaṃ vedallaṃ, tasmā ‘dhammaññū’ti vuccati.}}\\
\begin{addmargin}[1em]{2em}
\setstretch{.5}
{\PaliGlossB{they are called ‘one who knows the teachings’.}}\\
\end{addmargin}
\end{absolutelynopagebreak}

\begin{absolutelynopagebreak}
\setstretch{.7}
{\PaliGlossA{iti dhammaññū. (1)}}\\
\begin{addmargin}[1em]{2em}
\setstretch{.5}
{\PaliGlossB{Such is the one who knows the teachings.}}\\
\end{addmargin}
\end{absolutelynopagebreak}

\begin{absolutelynopagebreak}
\setstretch{.7}
{\PaliGlossA{atthaññū ca kathaṃ hoti?}}\\
\begin{addmargin}[1em]{2em}
\setstretch{.5}
{\PaliGlossB{And how are they one who knows the meaning?}}\\
\end{addmargin}
\end{absolutelynopagebreak}

\begin{absolutelynopagebreak}
\setstretch{.7}
{\PaliGlossA{idha, bhikkhave, bhikkhu tassa tasseva bhāsitassa atthaṃ jānāti:}}\\
\begin{addmargin}[1em]{2em}
\setstretch{.5}
{\PaliGlossB{It’s when a mendicant knows the meaning of this or that statement:}}\\
\end{addmargin}
\end{absolutelynopagebreak}

\begin{absolutelynopagebreak}
\setstretch{.7}
{\PaliGlossA{‘ayaṃ imassa bhāsitassa attho, ayaṃ imassa bhāsitassa attho’ti.}}\\
\begin{addmargin}[1em]{2em}
\setstretch{.5}
{\PaliGlossB{‘This is what that statement means; that is what this statement means.’}}\\
\end{addmargin}
\end{absolutelynopagebreak}

\begin{absolutelynopagebreak}
\setstretch{.7}
{\PaliGlossA{no ce, bhikkhave, bhikkhu tassa tasseva bhāsitassa atthaṃ jāneyya:}}\\
\begin{addmargin}[1em]{2em}
\setstretch{.5}
{\PaliGlossB{If a mendicant did not know the meaning of this or that statement,}}\\
\end{addmargin}
\end{absolutelynopagebreak}

\begin{absolutelynopagebreak}
\setstretch{.7}
{\PaliGlossA{‘ayaṃ imassa bhāsitassa attho, ayaṃ imassa bhāsitassa attho’ti, nayidha ‘atthaññū’ti vucceyya.}}\\
\begin{addmargin}[1em]{2em}
\setstretch{.5}
{\PaliGlossB{they would not be called ‘one who knows the meaning’.}}\\
\end{addmargin}
\end{absolutelynopagebreak}

\begin{absolutelynopagebreak}
\setstretch{.7}
{\PaliGlossA{yasmā ca kho, bhikkhave, bhikkhu tassa tasseva bhāsitassa atthaṃ jānāti:}}\\
\begin{addmargin}[1em]{2em}
\setstretch{.5}
{\PaliGlossB{But because they do know the meaning of this or that statement,}}\\
\end{addmargin}
\end{absolutelynopagebreak}

\begin{absolutelynopagebreak}
\setstretch{.7}
{\PaliGlossA{‘ayaṃ imassa bhāsitassa attho, ayaṃ imassa bhāsitassa attho’ti, tasmā ‘atthaññū’ti vuccati.}}\\
\begin{addmargin}[1em]{2em}
\setstretch{.5}
{\PaliGlossB{they are called ‘one who knows the meaning’.}}\\
\end{addmargin}
\end{absolutelynopagebreak}

\begin{absolutelynopagebreak}
\setstretch{.7}
{\PaliGlossA{iti dhammaññū, atthaññū. (2)}}\\
\begin{addmargin}[1em]{2em}
\setstretch{.5}
{\PaliGlossB{Such is the one who knows the teachings and the one who knows the meaning.}}\\
\end{addmargin}
\end{absolutelynopagebreak}

\begin{absolutelynopagebreak}
\setstretch{.7}
{\PaliGlossA{attaññū ca kathaṃ hoti?}}\\
\begin{addmargin}[1em]{2em}
\setstretch{.5}
{\PaliGlossB{And how are they one who has self-knowledge?}}\\
\end{addmargin}
\end{absolutelynopagebreak}

\begin{absolutelynopagebreak}
\setstretch{.7}
{\PaliGlossA{idha, bhikkhave, bhikkhu attānaṃ jānāti:}}\\
\begin{addmargin}[1em]{2em}
\setstretch{.5}
{\PaliGlossB{It’s when a mendicant has self-knowledge:}}\\
\end{addmargin}
\end{absolutelynopagebreak}

\begin{absolutelynopagebreak}
\setstretch{.7}
{\PaliGlossA{‘ettakomhi saddhāya sīlena sutena cāgena paññāya paṭibhānenā’ti.}}\\
\begin{addmargin}[1em]{2em}
\setstretch{.5}
{\PaliGlossB{‘This is the extent of my faith, ethics, learning, generosity, wisdom, and eloquence.’}}\\
\end{addmargin}
\end{absolutelynopagebreak}

\begin{absolutelynopagebreak}
\setstretch{.7}
{\PaliGlossA{no ce, bhikkhave, bhikkhu attānaṃ jāneyya:}}\\
\begin{addmargin}[1em]{2em}
\setstretch{.5}
{\PaliGlossB{If a mendicant did not have self-knowledge,}}\\
\end{addmargin}
\end{absolutelynopagebreak}

\begin{absolutelynopagebreak}
\setstretch{.7}
{\PaliGlossA{‘ettakomhi saddhāya sīlena sutena cāgena paññāya paṭibhānenā’ti, nayidha ‘attaññū’ti vucceyya.}}\\
\begin{addmargin}[1em]{2em}
\setstretch{.5}
{\PaliGlossB{they would not be called ‘one who has self-knowledge’.}}\\
\end{addmargin}
\end{absolutelynopagebreak}

\begin{absolutelynopagebreak}
\setstretch{.7}
{\PaliGlossA{yasmā ca, bhikkhave, bhikkhu attānaṃ jānāti:}}\\
\begin{addmargin}[1em]{2em}
\setstretch{.5}
{\PaliGlossB{But because they do have self-knowledge,}}\\
\end{addmargin}
\end{absolutelynopagebreak}

\begin{absolutelynopagebreak}
\setstretch{.7}
{\PaliGlossA{‘ettakomhi saddhāya sīlena sutena cāgena paññāya paṭibhānenā’ti, tasmā ‘attaññū’ti vuccati.}}\\
\begin{addmargin}[1em]{2em}
\setstretch{.5}
{\PaliGlossB{they are called ‘one who has self-knowledge’.}}\\
\end{addmargin}
\end{absolutelynopagebreak}

\begin{absolutelynopagebreak}
\setstretch{.7}
{\PaliGlossA{iti dhammaññū, atthaññū, attaññū. (3)}}\\
\begin{addmargin}[1em]{2em}
\setstretch{.5}
{\PaliGlossB{Such is the one who knows the teachings, the one who knows the meaning, and the one who has self-knowledge.}}\\
\end{addmargin}
\end{absolutelynopagebreak}

\begin{absolutelynopagebreak}
\setstretch{.7}
{\PaliGlossA{mattaññū ca kathaṃ hoti?}}\\
\begin{addmargin}[1em]{2em}
\setstretch{.5}
{\PaliGlossB{And how are they one who knows moderation?}}\\
\end{addmargin}
\end{absolutelynopagebreak}

\begin{absolutelynopagebreak}
\setstretch{.7}
{\PaliGlossA{idha, bhikkhave, bhikkhu mattaṃ jānāti cīvarapiṇḍapātasenāsanagilānappaccayabhesajjaparikkhārānaṃ paṭiggahaṇāya.}}\\
\begin{addmargin}[1em]{2em}
\setstretch{.5}
{\PaliGlossB{It’s when a mendicant knows moderation when receiving robes, alms-food, lodgings, and medicines and supplies for the sick.}}\\
\end{addmargin}
\end{absolutelynopagebreak}

\begin{absolutelynopagebreak}
\setstretch{.7}
{\PaliGlossA{no ce, bhikkhave, bhikkhu mattaṃ jāneyya cīvarapiṇḍapātasenāsanagilānappaccayabhesajjaparikkhārānaṃ paṭiggahaṇāya, nayidha ‘mattaññū’ti vucceyya.}}\\
\begin{addmargin}[1em]{2em}
\setstretch{.5}
{\PaliGlossB{If a mendicant did not know moderation, they would not be called ‘one who knows moderation’.}}\\
\end{addmargin}
\end{absolutelynopagebreak}

\begin{absolutelynopagebreak}
\setstretch{.7}
{\PaliGlossA{yasmā ca kho, bhikkhave, bhikkhu mattaṃ jānāti cīvarapiṇḍapātasenāsanagilānappaccayabhesajjaparikkhārānaṃ paṭiggahaṇāya, tasmā ‘mattaññū’ti vuccati.}}\\
\begin{addmargin}[1em]{2em}
\setstretch{.5}
{\PaliGlossB{But because they do know moderation, they are called ‘one who knows moderation’.}}\\
\end{addmargin}
\end{absolutelynopagebreak}

\begin{absolutelynopagebreak}
\setstretch{.7}
{\PaliGlossA{iti dhammaññū, atthaññū, attaññū, mattaññū. (4)}}\\
\begin{addmargin}[1em]{2em}
\setstretch{.5}
{\PaliGlossB{Such is the one who knows the teachings, the one who knows the meaning, the one who has self-knowledge, and the one who knows moderation.}}\\
\end{addmargin}
\end{absolutelynopagebreak}

\begin{absolutelynopagebreak}
\setstretch{.7}
{\PaliGlossA{kālaññū ca kathaṃ hoti?}}\\
\begin{addmargin}[1em]{2em}
\setstretch{.5}
{\PaliGlossB{And how are they one who knows the right time?}}\\
\end{addmargin}
\end{absolutelynopagebreak}

\begin{absolutelynopagebreak}
\setstretch{.7}
{\PaliGlossA{idha, bhikkhave, bhikkhu kālaṃ jānāti:}}\\
\begin{addmargin}[1em]{2em}
\setstretch{.5}
{\PaliGlossB{It’s when a mendicant knows the right time:}}\\
\end{addmargin}
\end{absolutelynopagebreak}

\begin{absolutelynopagebreak}
\setstretch{.7}
{\PaliGlossA{‘ayaṃ kālo uddesassa, ayaṃ kālo paripucchāya, ayaṃ kālo yogassa, ayaṃ kālo paṭisallānassā’ti.}}\\
\begin{addmargin}[1em]{2em}
\setstretch{.5}
{\PaliGlossB{‘This is the time for recitation; this is the time for questioning; this is the time for meditation; this is the time for retreat.’}}\\
\end{addmargin}
\end{absolutelynopagebreak}

\begin{absolutelynopagebreak}
\setstretch{.7}
{\PaliGlossA{no ce, bhikkhave, bhikkhu kālaṃ jāneyya:}}\\
\begin{addmargin}[1em]{2em}
\setstretch{.5}
{\PaliGlossB{If a mendicant did not know the right time,}}\\
\end{addmargin}
\end{absolutelynopagebreak}

\begin{absolutelynopagebreak}
\setstretch{.7}
{\PaliGlossA{‘ayaṃ kālo uddesassa, ayaṃ kālo paripucchāya, ayaṃ kālo yogassa, ayaṃ kālo paṭisallānassā’ti, nayidha ‘kālaññū’ti vucceyya.}}\\
\begin{addmargin}[1em]{2em}
\setstretch{.5}
{\PaliGlossB{they would not be called ‘one who knows the right time’.}}\\
\end{addmargin}
\end{absolutelynopagebreak}

\begin{absolutelynopagebreak}
\setstretch{.7}
{\PaliGlossA{yasmā ca kho, bhikkhave, bhikkhu kālaṃ jānāti:}}\\
\begin{addmargin}[1em]{2em}
\setstretch{.5}
{\PaliGlossB{But because they do know the right time,}}\\
\end{addmargin}
\end{absolutelynopagebreak}

\begin{absolutelynopagebreak}
\setstretch{.7}
{\PaliGlossA{‘ayaṃ kālo uddesassa, ayaṃ kālo paripucchāya, ayaṃ kālo yogassa, ayaṃ kālo paṭisallānassā’ti, tasmā ‘kālaññū’ti vuccati.}}\\
\begin{addmargin}[1em]{2em}
\setstretch{.5}
{\PaliGlossB{they are called ‘one who knows the right time’.}}\\
\end{addmargin}
\end{absolutelynopagebreak}

\begin{absolutelynopagebreak}
\setstretch{.7}
{\PaliGlossA{iti dhammaññū, atthaññū, attaññū, mattaññū, kālaññū. (5)}}\\
\begin{addmargin}[1em]{2em}
\setstretch{.5}
{\PaliGlossB{Such is the one who knows the teachings, the one who knows the meaning, the one who has self-knowledge, the one who knows moderation, and the one who knows the right time.}}\\
\end{addmargin}
\end{absolutelynopagebreak}

\begin{absolutelynopagebreak}
\setstretch{.7}
{\PaliGlossA{parisaññū ca kathaṃ hoti?}}\\
\begin{addmargin}[1em]{2em}
\setstretch{.5}
{\PaliGlossB{And how are they one who knows assemblies?}}\\
\end{addmargin}
\end{absolutelynopagebreak}

\begin{absolutelynopagebreak}
\setstretch{.7}
{\PaliGlossA{idha, bhikkhave, bhikkhu parisaṃ jānāti:}}\\
\begin{addmargin}[1em]{2em}
\setstretch{.5}
{\PaliGlossB{It’s when a mendicant knows assemblies:}}\\
\end{addmargin}
\end{absolutelynopagebreak}

\begin{absolutelynopagebreak}
\setstretch{.7}
{\PaliGlossA{‘ayaṃ khattiyaparisā, ayaṃ brāhmaṇaparisā, ayaṃ gahapatiparisā, ayaṃ samaṇaparisā.}}\\
\begin{addmargin}[1em]{2em}
\setstretch{.5}
{\PaliGlossB{‘This is an assembly of aristocrats, of brahmins, of householders, or of ascetics.}}\\
\end{addmargin}
\end{absolutelynopagebreak}

\begin{absolutelynopagebreak}
\setstretch{.7}
{\PaliGlossA{tattha evaṃ upasaṅkamitabbaṃ, evaṃ ṭhātabbaṃ, evaṃ kattabbaṃ, evaṃ nisīditabbaṃ, evaṃ bhāsitabbaṃ, evaṃ tuṇhī bhavitabban’ti.}}\\
\begin{addmargin}[1em]{2em}
\setstretch{.5}
{\PaliGlossB{This one should be approached in this way. This is how to stand, to act, to sit, to speak, or to stay silent when there.’}}\\
\end{addmargin}
\end{absolutelynopagebreak}

\begin{absolutelynopagebreak}
\setstretch{.7}
{\PaliGlossA{no ce, bhikkhave, bhikkhu parisaṃ jāneyya:}}\\
\begin{addmargin}[1em]{2em}
\setstretch{.5}
{\PaliGlossB{If a mendicant did not know assemblies,}}\\
\end{addmargin}
\end{absolutelynopagebreak}

\begin{absolutelynopagebreak}
\setstretch{.7}
{\PaliGlossA{‘ayaṃ khattiyaparisā … pe …}}\\
\begin{addmargin}[1em]{2em}
\setstretch{.5}
{\PaliGlossB{    -}}\\
\end{addmargin}
\end{absolutelynopagebreak}

\begin{absolutelynopagebreak}
\setstretch{.7}
{\PaliGlossA{evaṃ tuṇhī bhavitabban’ti, nayidha ‘parisaññū’ti vucceyya.}}\\
\begin{addmargin}[1em]{2em}
\setstretch{.5}
{\PaliGlossB{they would not be called ‘one who knows assemblies’.}}\\
\end{addmargin}
\end{absolutelynopagebreak}

\begin{absolutelynopagebreak}
\setstretch{.7}
{\PaliGlossA{yasmā ca kho, bhikkhave, bhikkhu parisaṃ jānāti:}}\\
\begin{addmargin}[1em]{2em}
\setstretch{.5}
{\PaliGlossB{But because they do know assemblies,}}\\
\end{addmargin}
\end{absolutelynopagebreak}

\begin{absolutelynopagebreak}
\setstretch{.7}
{\PaliGlossA{‘ayaṃ khattiyaparisā, ayaṃ brāhmaṇaparisā, ayaṃ gahapatiparisā, ayaṃ samaṇaparisā.}}\\
\begin{addmargin}[1em]{2em}
\setstretch{.5}
{\PaliGlossB{    -}}\\
\end{addmargin}
\end{absolutelynopagebreak}

\begin{absolutelynopagebreak}
\setstretch{.7}
{\PaliGlossA{tattha evaṃ upasaṅkamitabbaṃ, evaṃ ṭhātabbaṃ, evaṃ kattabbaṃ, evaṃ nisīditabbaṃ, evaṃ bhāsitabbaṃ, evaṃ tuṇhī bhavitabban’ti, tasmā ‘parisaññū’ti vuccati.}}\\
\begin{addmargin}[1em]{2em}
\setstretch{.5}
{\PaliGlossB{they are called ‘one who knows assemblies’.}}\\
\end{addmargin}
\end{absolutelynopagebreak}

\begin{absolutelynopagebreak}
\setstretch{.7}
{\PaliGlossA{iti dhammaññū, atthaññū, attaññū, mattaññū, kālaññū, parisaññū. (6)}}\\
\begin{addmargin}[1em]{2em}
\setstretch{.5}
{\PaliGlossB{Such is the one who knows the teachings, the one who knows the meaning, the one who has self-knowledge, the one who knows moderation, the one who knows the right time, and the one who knows assemblies.}}\\
\end{addmargin}
\end{absolutelynopagebreak}

\begin{absolutelynopagebreak}
\setstretch{.7}
{\PaliGlossA{puggalaparoparaññū ca kathaṃ hoti?}}\\
\begin{addmargin}[1em]{2em}
\setstretch{.5}
{\PaliGlossB{And how are they one who knows people high and low?}}\\
\end{addmargin}
\end{absolutelynopagebreak}

\begin{absolutelynopagebreak}
\setstretch{.7}
{\PaliGlossA{idha, bhikkhave, bhikkhuno dvayena puggalā viditā honti.}}\\
\begin{addmargin}[1em]{2em}
\setstretch{.5}
{\PaliGlossB{It’s when a mendicant understands people in terms of pairs.}}\\
\end{addmargin}
\end{absolutelynopagebreak}

\begin{absolutelynopagebreak}
\setstretch{.7}
{\PaliGlossA{dve puggalā—}}\\
\begin{addmargin}[1em]{2em}
\setstretch{.5}
{\PaliGlossB{Two people:}}\\
\end{addmargin}
\end{absolutelynopagebreak}

\begin{absolutelynopagebreak}
\setstretch{.7}
{\PaliGlossA{eko ariyānaṃ dassanakāmo, eko ariyānaṃ na dassanakāmo.}}\\
\begin{addmargin}[1em]{2em}
\setstretch{.5}
{\PaliGlossB{one likes to see the noble ones, one does not.}}\\
\end{addmargin}
\end{absolutelynopagebreak}

\begin{absolutelynopagebreak}
\setstretch{.7}
{\PaliGlossA{yvāyaṃ puggalo ariyānaṃ na dassanakāmo,}}\\
\begin{addmargin}[1em]{2em}
\setstretch{.5}
{\PaliGlossB{The person who doesn’t like to see the noble ones}}\\
\end{addmargin}
\end{absolutelynopagebreak}

\begin{absolutelynopagebreak}
\setstretch{.7}
{\PaliGlossA{evaṃ so tenaṅgena gārayho.}}\\
\begin{addmargin}[1em]{2em}
\setstretch{.5}
{\PaliGlossB{is reprehensible in that respect.}}\\
\end{addmargin}
\end{absolutelynopagebreak}

\begin{absolutelynopagebreak}
\setstretch{.7}
{\PaliGlossA{yvāyaṃ puggalo ariyānaṃ dassanakāmo,}}\\
\begin{addmargin}[1em]{2em}
\setstretch{.5}
{\PaliGlossB{The person who does like to see the noble ones}}\\
\end{addmargin}
\end{absolutelynopagebreak}

\begin{absolutelynopagebreak}
\setstretch{.7}
{\PaliGlossA{evaṃ so tenaṅgena pāsaṃso. (7.1)}}\\
\begin{addmargin}[1em]{2em}
\setstretch{.5}
{\PaliGlossB{is praiseworthy in that respect.}}\\
\end{addmargin}
\end{absolutelynopagebreak}

\begin{absolutelynopagebreak}
\setstretch{.7}
{\PaliGlossA{dve puggalā ariyānaṃ dassanakāmā—}}\\
\begin{addmargin}[1em]{2em}
\setstretch{.5}
{\PaliGlossB{Two people like to see the noble ones:}}\\
\end{addmargin}
\end{absolutelynopagebreak}

\begin{absolutelynopagebreak}
\setstretch{.7}
{\PaliGlossA{eko saddhammaṃ sotukāmo, eko saddhammaṃ na sotukāmo.}}\\
\begin{addmargin}[1em]{2em}
\setstretch{.5}
{\PaliGlossB{one likes to hear the true teaching, one does not.}}\\
\end{addmargin}
\end{absolutelynopagebreak}

\begin{absolutelynopagebreak}
\setstretch{.7}
{\PaliGlossA{yvāyaṃ puggalo saddhammaṃ na sotukāmo,}}\\
\begin{addmargin}[1em]{2em}
\setstretch{.5}
{\PaliGlossB{The person who doesn’t like to hear the true teaching}}\\
\end{addmargin}
\end{absolutelynopagebreak}

\begin{absolutelynopagebreak}
\setstretch{.7}
{\PaliGlossA{evaṃ so tenaṅgena gārayho.}}\\
\begin{addmargin}[1em]{2em}
\setstretch{.5}
{\PaliGlossB{is reprehensible in that respect.}}\\
\end{addmargin}
\end{absolutelynopagebreak}

\begin{absolutelynopagebreak}
\setstretch{.7}
{\PaliGlossA{yvāyaṃ puggalo saddhammaṃ sotukāmo,}}\\
\begin{addmargin}[1em]{2em}
\setstretch{.5}
{\PaliGlossB{The person who does like to hear the true teaching}}\\
\end{addmargin}
\end{absolutelynopagebreak}

\begin{absolutelynopagebreak}
\setstretch{.7}
{\PaliGlossA{evaṃ so tenaṅgena pāsaṃso. (7.2)}}\\
\begin{addmargin}[1em]{2em}
\setstretch{.5}
{\PaliGlossB{is praiseworthy in that respect.}}\\
\end{addmargin}
\end{absolutelynopagebreak}

\begin{absolutelynopagebreak}
\setstretch{.7}
{\PaliGlossA{dve puggalā saddhammaṃ sotukāmā—}}\\
\begin{addmargin}[1em]{2em}
\setstretch{.5}
{\PaliGlossB{Two people like to hear the true teaching:}}\\
\end{addmargin}
\end{absolutelynopagebreak}

\begin{absolutelynopagebreak}
\setstretch{.7}
{\PaliGlossA{eko ohitasoto dhammaṃ suṇāti, eko anohitasoto dhammaṃ suṇāti.}}\\
\begin{addmargin}[1em]{2em}
\setstretch{.5}
{\PaliGlossB{one lends an ear to the teaching, one does not.}}\\
\end{addmargin}
\end{absolutelynopagebreak}

\begin{absolutelynopagebreak}
\setstretch{.7}
{\PaliGlossA{yvāyaṃ puggalo anohitasoto dhammaṃ suṇāti,}}\\
\begin{addmargin}[1em]{2em}
\setstretch{.5}
{\PaliGlossB{The person who doesn’t lend an ear to the teaching}}\\
\end{addmargin}
\end{absolutelynopagebreak}

\begin{absolutelynopagebreak}
\setstretch{.7}
{\PaliGlossA{evaṃ so tenaṅgena gārayho.}}\\
\begin{addmargin}[1em]{2em}
\setstretch{.5}
{\PaliGlossB{is reprehensible in that respect.}}\\
\end{addmargin}
\end{absolutelynopagebreak}

\begin{absolutelynopagebreak}
\setstretch{.7}
{\PaliGlossA{yvāyaṃ puggalo ohitasoto dhammaṃ suṇāti,}}\\
\begin{addmargin}[1em]{2em}
\setstretch{.5}
{\PaliGlossB{The person who does lend an ear to the teaching}}\\
\end{addmargin}
\end{absolutelynopagebreak}

\begin{absolutelynopagebreak}
\setstretch{.7}
{\PaliGlossA{evaṃ so tenaṅgena pāsaṃso. (7.3)}}\\
\begin{addmargin}[1em]{2em}
\setstretch{.5}
{\PaliGlossB{is praiseworthy in that respect.}}\\
\end{addmargin}
\end{absolutelynopagebreak}

\begin{absolutelynopagebreak}
\setstretch{.7}
{\PaliGlossA{dve puggalā ohitasotā dhammaṃ suṇanti—}}\\
\begin{addmargin}[1em]{2em}
\setstretch{.5}
{\PaliGlossB{Two people lend an ear to the teaching:}}\\
\end{addmargin}
\end{absolutelynopagebreak}

\begin{absolutelynopagebreak}
\setstretch{.7}
{\PaliGlossA{eko sutvā dhammaṃ dhāreti, eko sutvā dhammaṃ na dhāreti.}}\\
\begin{addmargin}[1em]{2em}
\setstretch{.5}
{\PaliGlossB{one remembers the teaching they’ve heard, one does not.}}\\
\end{addmargin}
\end{absolutelynopagebreak}

\begin{absolutelynopagebreak}
\setstretch{.7}
{\PaliGlossA{yvāyaṃ puggalo sutvā na dhammaṃ dhāreti,}}\\
\begin{addmargin}[1em]{2em}
\setstretch{.5}
{\PaliGlossB{The person who doesn’t remember the teaching they’ve heard}}\\
\end{addmargin}
\end{absolutelynopagebreak}

\begin{absolutelynopagebreak}
\setstretch{.7}
{\PaliGlossA{evaṃ so tenaṅgena gārayho.}}\\
\begin{addmargin}[1em]{2em}
\setstretch{.5}
{\PaliGlossB{is reprehensible in that respect.}}\\
\end{addmargin}
\end{absolutelynopagebreak}

\begin{absolutelynopagebreak}
\setstretch{.7}
{\PaliGlossA{yvāyaṃ puggalo sutvā dhammaṃ dhāreti,}}\\
\begin{addmargin}[1em]{2em}
\setstretch{.5}
{\PaliGlossB{The person who does remember the teaching they’ve heard}}\\
\end{addmargin}
\end{absolutelynopagebreak}

\begin{absolutelynopagebreak}
\setstretch{.7}
{\PaliGlossA{evaṃ so tenaṅgena pāsaṃso. (7.4)}}\\
\begin{addmargin}[1em]{2em}
\setstretch{.5}
{\PaliGlossB{is praiseworthy in that respect.}}\\
\end{addmargin}
\end{absolutelynopagebreak}

\begin{absolutelynopagebreak}
\setstretch{.7}
{\PaliGlossA{dve puggalā sutvā dhammaṃ dhārenti—}}\\
\begin{addmargin}[1em]{2em}
\setstretch{.5}
{\PaliGlossB{Two people remember the teaching they’ve heard:}}\\
\end{addmargin}
\end{absolutelynopagebreak}

\begin{absolutelynopagebreak}
\setstretch{.7}
{\PaliGlossA{eko dhātānaṃ dhammānaṃ atthaṃ upaparikkhati, eko dhātānaṃ dhammānaṃ atthaṃ na upaparikkhati.}}\\
\begin{addmargin}[1em]{2em}
\setstretch{.5}
{\PaliGlossB{one reflects on the meaning of the teachings they have remembered, one does not.}}\\
\end{addmargin}
\end{absolutelynopagebreak}

\begin{absolutelynopagebreak}
\setstretch{.7}
{\PaliGlossA{yvāyaṃ puggalo dhātānaṃ dhammānaṃ atthaṃ na upaparikkhati,}}\\
\begin{addmargin}[1em]{2em}
\setstretch{.5}
{\PaliGlossB{The person who does not reflect on the meaning of the teachings they have remembered}}\\
\end{addmargin}
\end{absolutelynopagebreak}

\begin{absolutelynopagebreak}
\setstretch{.7}
{\PaliGlossA{evaṃ so tenaṅgena gārayho.}}\\
\begin{addmargin}[1em]{2em}
\setstretch{.5}
{\PaliGlossB{is reprehensible in that respect.}}\\
\end{addmargin}
\end{absolutelynopagebreak}

\begin{absolutelynopagebreak}
\setstretch{.7}
{\PaliGlossA{yvāyaṃ puggalo dhātānaṃ dhammānaṃ atthaṃ upaparikkhati,}}\\
\begin{addmargin}[1em]{2em}
\setstretch{.5}
{\PaliGlossB{The person who does reflect on the meaning of the teachings they have remembered}}\\
\end{addmargin}
\end{absolutelynopagebreak}

\begin{absolutelynopagebreak}
\setstretch{.7}
{\PaliGlossA{evaṃ so tenaṅgena pāsaṃso. (7.5)}}\\
\begin{addmargin}[1em]{2em}
\setstretch{.5}
{\PaliGlossB{is praiseworthy in that respect.}}\\
\end{addmargin}
\end{absolutelynopagebreak}

\begin{absolutelynopagebreak}
\setstretch{.7}
{\PaliGlossA{dve puggalā dhātānaṃ dhammānaṃ atthaṃ upaparikkhanti—}}\\
\begin{addmargin}[1em]{2em}
\setstretch{.5}
{\PaliGlossB{Two people reflect on the meaning of the teachings they have remembered:}}\\
\end{addmargin}
\end{absolutelynopagebreak}

\begin{absolutelynopagebreak}
\setstretch{.7}
{\PaliGlossA{eko atthamaññāya dhammamaññāya dhammānudhammappaṭipanno, eko atthamaññāya dhammamaññāya na dhammānudhammappaṭipanno.}}\\
\begin{addmargin}[1em]{2em}
\setstretch{.5}
{\PaliGlossB{one understands the meaning and the teaching and practices accordingly, one understands the meaning and the teaching but does not practice accordingly.}}\\
\end{addmargin}
\end{absolutelynopagebreak}

\begin{absolutelynopagebreak}
\setstretch{.7}
{\PaliGlossA{yvāyaṃ puggalo atthamaññāya dhammamaññāya na dhammānudhammappaṭipanno,}}\\
\begin{addmargin}[1em]{2em}
\setstretch{.5}
{\PaliGlossB{The person who understands the meaning and the teaching but does not practice accordingly}}\\
\end{addmargin}
\end{absolutelynopagebreak}

\begin{absolutelynopagebreak}
\setstretch{.7}
{\PaliGlossA{evaṃ so tenaṅgena gārayho.}}\\
\begin{addmargin}[1em]{2em}
\setstretch{.5}
{\PaliGlossB{is reprehensible in that respect.}}\\
\end{addmargin}
\end{absolutelynopagebreak}

\begin{absolutelynopagebreak}
\setstretch{.7}
{\PaliGlossA{yvāyaṃ puggalo atthamaññāya dhammamaññāya dhammānudhammappaṭipanno,}}\\
\begin{addmargin}[1em]{2em}
\setstretch{.5}
{\PaliGlossB{The person who understands the meaning and the teaching and practices accordingly}}\\
\end{addmargin}
\end{absolutelynopagebreak}

\begin{absolutelynopagebreak}
\setstretch{.7}
{\PaliGlossA{evaṃ so tenaṅgena pāsaṃso. (7.6)}}\\
\begin{addmargin}[1em]{2em}
\setstretch{.5}
{\PaliGlossB{is praiseworthy in that respect.}}\\
\end{addmargin}
\end{absolutelynopagebreak}

\begin{absolutelynopagebreak}
\setstretch{.7}
{\PaliGlossA{dve puggalā atthamaññāya dhammamaññāya dhammānudhammappaṭipannā—}}\\
\begin{addmargin}[1em]{2em}
\setstretch{.5}
{\PaliGlossB{Two people understand the meaning and the teaching and practice accordingly:}}\\
\end{addmargin}
\end{absolutelynopagebreak}

\begin{absolutelynopagebreak}
\setstretch{.7}
{\PaliGlossA{eko attahitāya paṭipanno no parahitāya, eko attahitāya ca paṭipanno parahitāya ca.}}\\
\begin{addmargin}[1em]{2em}
\setstretch{.5}
{\PaliGlossB{one practices to benefit themselves but not others, and one practices to benefit both themselves and others.}}\\
\end{addmargin}
\end{absolutelynopagebreak}

\begin{absolutelynopagebreak}
\setstretch{.7}
{\PaliGlossA{yvāyaṃ puggalo attahitāya paṭipanno no parahitāya,}}\\
\begin{addmargin}[1em]{2em}
\setstretch{.5}
{\PaliGlossB{The person who practices to benefit themselves but not others}}\\
\end{addmargin}
\end{absolutelynopagebreak}

\begin{absolutelynopagebreak}
\setstretch{.7}
{\PaliGlossA{evaṃ so tenaṅgena gārayho.}}\\
\begin{addmargin}[1em]{2em}
\setstretch{.5}
{\PaliGlossB{is reprehensible in that respect.}}\\
\end{addmargin}
\end{absolutelynopagebreak}

\begin{absolutelynopagebreak}
\setstretch{.7}
{\PaliGlossA{yvāyaṃ puggalo attahitāya ca paṭipanno parahitāya ca,}}\\
\begin{addmargin}[1em]{2em}
\setstretch{.5}
{\PaliGlossB{The person who practices to benefit both themselves and others}}\\
\end{addmargin}
\end{absolutelynopagebreak}

\begin{absolutelynopagebreak}
\setstretch{.7}
{\PaliGlossA{evaṃ so tenaṅgena pāsaṃso. (7.7)}}\\
\begin{addmargin}[1em]{2em}
\setstretch{.5}
{\PaliGlossB{is praiseworthy in that respect.}}\\
\end{addmargin}
\end{absolutelynopagebreak}

\begin{absolutelynopagebreak}
\setstretch{.7}
{\PaliGlossA{evaṃ kho, bhikkhave, bhikkhuno dvayena puggalā viditā honti.}}\\
\begin{addmargin}[1em]{2em}
\setstretch{.5}
{\PaliGlossB{That’s how a mendicant understands people in terms of pairs.}}\\
\end{addmargin}
\end{absolutelynopagebreak}

\begin{absolutelynopagebreak}
\setstretch{.7}
{\PaliGlossA{evaṃ, bhikkhave, bhikkhu puggalaparoparaññū hoti.}}\\
\begin{addmargin}[1em]{2em}
\setstretch{.5}
{\PaliGlossB{That’s how a mendicant is one who knows people high and low.}}\\
\end{addmargin}
\end{absolutelynopagebreak}

\begin{absolutelynopagebreak}
\setstretch{.7}
{\PaliGlossA{imehi kho, bhikkhave, sattahi dhammehi samannāgato bhikkhu āhuneyyo hoti pāhuneyyo … pe … anuttaraṃ puññakkhettaṃ lokassā”ti.}}\\
\begin{addmargin}[1em]{2em}
\setstretch{.5}
{\PaliGlossB{A mendicant with these seven factors is worthy of offerings dedicated to the gods, worthy of hospitality, worthy of a religious donation, worthy of veneration with joined palms, and is the supreme field of merit for the world.”}}\\
\end{addmargin}
\end{absolutelynopagebreak}

\begin{absolutelynopagebreak}
\setstretch{.7}
{\PaliGlossA{catutthaṃ.}}\\
\begin{addmargin}[1em]{2em}
\setstretch{.5}
{\PaliGlossB{    -}}\\
\end{addmargin}
\end{absolutelynopagebreak}
