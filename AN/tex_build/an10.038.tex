
\begin{absolutelynopagebreak}
\setstretch{.7}
{\PaliGlossA{aṅguttara nikāya 10}}\\
\begin{addmargin}[1em]{2em}
\setstretch{.5}
{\PaliGlossB{Numbered Discourses 10}}\\
\end{addmargin}
\end{absolutelynopagebreak}

\begin{absolutelynopagebreak}
\setstretch{.7}
{\PaliGlossA{4. upālivagga}}\\
\begin{addmargin}[1em]{2em}
\setstretch{.5}
{\PaliGlossB{4. With Upāli}}\\
\end{addmargin}
\end{absolutelynopagebreak}

\begin{absolutelynopagebreak}
\setstretch{.7}
{\PaliGlossA{38. saṅghasāmaggīsutta}}\\
\begin{addmargin}[1em]{2em}
\setstretch{.5}
{\PaliGlossB{38. Harmony in the Saṅgha}}\\
\end{addmargin}
\end{absolutelynopagebreak}

\begin{absolutelynopagebreak}
\setstretch{.7}
{\PaliGlossA{“‘saṃghasāmaggī, saṃghasāmaggī’ti, bhante, vuccati.}}\\
\begin{addmargin}[1em]{2em}
\setstretch{.5}
{\PaliGlossB{“Sir, they speak of ‘harmony in the Saṅgha’.}}\\
\end{addmargin}
\end{absolutelynopagebreak}

\begin{absolutelynopagebreak}
\setstretch{.7}
{\PaliGlossA{kittāvatā nu kho, bhante, saṃgho samaggo hotī”ti?}}\\
\begin{addmargin}[1em]{2em}
\setstretch{.5}
{\PaliGlossB{How is harmony in the Saṅgha defined?”}}\\
\end{addmargin}
\end{absolutelynopagebreak}

\begin{absolutelynopagebreak}
\setstretch{.7}
{\PaliGlossA{“idhupāli, bhikkhū adhammaṃ adhammoti dīpenti, dhammaṃ dhammoti dīpenti, avinayaṃ avinayoti dīpenti, vinayaṃ vinayoti dīpenti, abhāsitaṃ alapitaṃ tathāgatena abhāsitaṃ alapitaṃ tathāgatenāti dīpenti, bhāsitaṃ lapitaṃ tathāgatena bhāsitaṃ lapitaṃ tathāgatenāti dīpenti, anāciṇṇaṃ tathāgatena anāciṇṇaṃ tathāgatenāti dīpenti, āciṇṇaṃ tathāgatena āciṇṇaṃ tathāgatenāti dīpenti, apaññattaṃ tathāgatena apaññattaṃ tathāgatenāti dīpenti, paññattaṃ tathāgatena paññattaṃ tathāgatenāti dīpenti.}}\\
\begin{addmargin}[1em]{2em}
\setstretch{.5}
{\PaliGlossB{“Upāli, it’s when a mendicant explains what is not the teaching as not the teaching, and what is the teaching as the teaching. They explain what is not the training as not the training, and what is the training as the training. They explain what was not spoken and stated by the Realized One as not spoken and stated by the Realized One, and what was spoken and stated by the Realized One as spoken and stated by the Realized One. They explain what was not practiced by the Realized One as not practiced by the Realized One, and what was practiced by the Realized One as practiced by the Realized One. They explain what was not prescribed by the Realized One as not prescribed by the Realized One, and what was prescribed by the Realized One as prescribed by the Realized One.}}\\
\end{addmargin}
\end{absolutelynopagebreak}

\begin{absolutelynopagebreak}
\setstretch{.7}
{\PaliGlossA{te imehi dasahi vatthūhi na avakassanti na apakassanti na āveni kammāni karonti na āveni pātimokkhaṃ uddisanti.}}\\
\begin{addmargin}[1em]{2em}
\setstretch{.5}
{\PaliGlossB{On these ten grounds they don’t split off and go their own way. They don’t perform legal acts autonomously or recite the monastic code autonomously.}}\\
\end{addmargin}
\end{absolutelynopagebreak}

\begin{absolutelynopagebreak}
\setstretch{.7}
{\PaliGlossA{ettāvatā kho, upāli, saṃgho samaggo hotī”ti.}}\\
\begin{addmargin}[1em]{2em}
\setstretch{.5}
{\PaliGlossB{That is how harmony in the Saṅgha is defined.”}}\\
\end{addmargin}
\end{absolutelynopagebreak}

\begin{absolutelynopagebreak}
\setstretch{.7}
{\PaliGlossA{aṭṭhamaṃ.}}\\
\begin{addmargin}[1em]{2em}
\setstretch{.5}
{\PaliGlossB{    -}}\\
\end{addmargin}
\end{absolutelynopagebreak}
