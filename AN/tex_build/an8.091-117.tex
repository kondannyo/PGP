
\begin{absolutelynopagebreak}
\setstretch{.7}
{\PaliGlossA{aṅguttara nikāya 8}}\\
\begin{addmargin}[1em]{2em}
\setstretch{.5}
{\PaliGlossB{Numbered Discourses 8}}\\
\end{addmargin}
\end{absolutelynopagebreak}

\begin{absolutelynopagebreak}
\setstretch{.7}
{\PaliGlossA{10. sāmaññavagga}}\\
\begin{addmargin}[1em]{2em}
\setstretch{.5}
{\PaliGlossB{10. Similarity}}\\
\end{addmargin}
\end{absolutelynopagebreak}

\begin{absolutelynopagebreak}
\setstretch{.7}
{\PaliGlossA{91–117}}\\
\begin{addmargin}[1em]{2em}
\setstretch{.5}
{\PaliGlossB{91–117}}\\
\end{addmargin}
\end{absolutelynopagebreak}

\begin{absolutelynopagebreak}
\setstretch{.7}
{\PaliGlossA{atha kho bojjhā upāsikā, sirīmā, padumā, sutanā, manujā, uttarā, muttā, khemā, rucī, cundī, bimbī, sumanā, mallikā, tissā, tissamātā, soṇā, soṇāya mātā, kāṇā, kāṇamātā, uttarā nandamātā, visākhā migāramātā, khujjuttarā upāsikā, sāmāvatī upāsikā, suppavāsā koliyadhītā, suppiyā upāsikā, nakulamātā gahapatānī. (1–26.)}}\\
\begin{addmargin}[1em]{2em}
\setstretch{.5}
{\PaliGlossB{And then the lay woman Bojjhā … Sirīmā … Padumā … Sutanā … Manujā … Uttarā … Muttā … Khemā … Somā … Rucī … Cundī … Bimbī … Sumanā … Mallikā … Tissā … Tissamātā … Soṇā … Soṇā’s mother … Kāṇā … Kāṇamātā … Uttarā Nanda’s mother … Visākhā Migāra’s mother … the lay woman Khujjuttarā … the lay woman Sāmāvatī … Suppavāsā the Koliyan … the lay woman Suppiyā … the housewife Nakula’s mother …}}\\
\end{addmargin}
\end{absolutelynopagebreak}

\begin{absolutelynopagebreak}
\setstretch{.7}
{\PaliGlossA{sāmaññavaggo pañcamo.}}\\
\begin{addmargin}[1em]{2em}
\setstretch{.5}
{\PaliGlossB{    -}}\\
\end{addmargin}
\end{absolutelynopagebreak}

\begin{absolutelynopagebreak}
\setstretch{.7}
{\PaliGlossA{dutiyo paṇṇāsako samatto.}}\\
\begin{addmargin}[1em]{2em}
\setstretch{.5}
{\PaliGlossB{    -}}\\
\end{addmargin}
\end{absolutelynopagebreak}
