
\begin{absolutelynopagebreak}
\setstretch{.7}
{\PaliGlossA{aṅguttara nikāya 6}}\\
\begin{addmargin}[1em]{2em}
\setstretch{.5}
{\PaliGlossB{Numbered Discourses 6}}\\
\end{addmargin}
\end{absolutelynopagebreak}

\begin{absolutelynopagebreak}
\setstretch{.7}
{\PaliGlossA{6. mahāvagga}}\\
\begin{addmargin}[1em]{2em}
\setstretch{.5}
{\PaliGlossB{6. The Great Chapter}}\\
\end{addmargin}
\end{absolutelynopagebreak}

\begin{absolutelynopagebreak}
\setstretch{.7}
{\PaliGlossA{61. majjhesutta}}\\
\begin{addmargin}[1em]{2em}
\setstretch{.5}
{\PaliGlossB{61. In the Middle}}\\
\end{addmargin}
\end{absolutelynopagebreak}

\begin{absolutelynopagebreak}
\setstretch{.7}
{\PaliGlossA{evaṃ me sutaṃ—}}\\
\begin{addmargin}[1em]{2em}
\setstretch{.5}
{\PaliGlossB{So I have heard.}}\\
\end{addmargin}
\end{absolutelynopagebreak}

\begin{absolutelynopagebreak}
\setstretch{.7}
{\PaliGlossA{ekaṃ samayaṃ bhagavā bārāṇasiyaṃ viharati isipatane migadāye.}}\\
\begin{addmargin}[1em]{2em}
\setstretch{.5}
{\PaliGlossB{At one time the Buddha was staying near Benares, in the deer park at Isipatana.}}\\
\end{addmargin}
\end{absolutelynopagebreak}

\begin{absolutelynopagebreak}
\setstretch{.7}
{\PaliGlossA{tena kho pana samayena sambahulānaṃ therānaṃ bhikkhūnaṃ pacchābhattaṃ piṇḍapātapaṭikkantānaṃ maṇḍalamāḷe sannisinnānaṃ sannipatitānaṃ ayamantarākathā udapādi:}}\\
\begin{addmargin}[1em]{2em}
\setstretch{.5}
{\PaliGlossB{Now at that time, after the meal, on return from alms-round, several senior mendicants sat together in the pavilion and this discussion came up among them,}}\\
\end{addmargin}
\end{absolutelynopagebreak}

\begin{absolutelynopagebreak}
\setstretch{.7}
{\PaliGlossA{“vuttamidaṃ, āvuso, bhagavatā pārāyane metteyyapañhe:}}\\
\begin{addmargin}[1em]{2em}
\setstretch{.5}
{\PaliGlossB{“Reverends, this was said by the Buddha in ‘The Way to the Beyond’, in ‘The Questions of Metteyya’:}}\\
\end{addmargin}
\end{absolutelynopagebreak}

\begin{absolutelynopagebreak}
\setstretch{.7}
{\PaliGlossA{‘yo ubhonte viditvāna,}}\\
\begin{addmargin}[1em]{2em}
\setstretch{.5}
{\PaliGlossB{‘The sage has known both ends,}}\\
\end{addmargin}
\end{absolutelynopagebreak}

\begin{absolutelynopagebreak}
\setstretch{.7}
{\PaliGlossA{majjhe mantā na lippati;}}\\
\begin{addmargin}[1em]{2em}
\setstretch{.5}
{\PaliGlossB{and is not stuck in the middle.}}\\
\end{addmargin}
\end{absolutelynopagebreak}

\begin{absolutelynopagebreak}
\setstretch{.7}
{\PaliGlossA{taṃ brūmi mahāpurisoti,}}\\
\begin{addmargin}[1em]{2em}
\setstretch{.5}
{\PaliGlossB{He is a great man, I declare,}}\\
\end{addmargin}
\end{absolutelynopagebreak}

\begin{absolutelynopagebreak}
\setstretch{.7}
{\PaliGlossA{sodha sibbini maccagā’”ti.}}\\
\begin{addmargin}[1em]{2em}
\setstretch{.5}
{\PaliGlossB{he has escaped the seamstress here.’}}\\
\end{addmargin}
\end{absolutelynopagebreak}

\begin{absolutelynopagebreak}
\setstretch{.7}
{\PaliGlossA{“katamo nu kho, āvuso, eko anto, katamo dutiyo anto, kiṃ majjhe, kā sibbinī”ti?}}\\
\begin{addmargin}[1em]{2em}
\setstretch{.5}
{\PaliGlossB{But what is one end? What’s the second end? What’s the middle? And who is the seamstress?”}}\\
\end{addmargin}
\end{absolutelynopagebreak}

\begin{absolutelynopagebreak}
\setstretch{.7}
{\PaliGlossA{evaṃ vutte, aññataro bhikkhu there bhikkhū etadavoca:}}\\
\begin{addmargin}[1em]{2em}
\setstretch{.5}
{\PaliGlossB{When this was said, one of the mendicants said to the senior mendicants:}}\\
\end{addmargin}
\end{absolutelynopagebreak}

\begin{absolutelynopagebreak}
\setstretch{.7}
{\PaliGlossA{“phasso kho, āvuso, eko anto, phassasamudayo dutiyo anto, phassanirodho majjhe, taṇhā sibbinī;}}\\
\begin{addmargin}[1em]{2em}
\setstretch{.5}
{\PaliGlossB{“Contact, reverends, is one end. The origin of contact is the second end. The cessation of contact is the middle. And craving is the seamstress,}}\\
\end{addmargin}
\end{absolutelynopagebreak}

\begin{absolutelynopagebreak}
\setstretch{.7}
{\PaliGlossA{taṇhā hi naṃ sibbati tassa tasseva bhavassa abhinibbattiyā.}}\\
\begin{addmargin}[1em]{2em}
\setstretch{.5}
{\PaliGlossB{for craving weaves one to rebirth in this or that state of existence.}}\\
\end{addmargin}
\end{absolutelynopagebreak}

\begin{absolutelynopagebreak}
\setstretch{.7}
{\PaliGlossA{ettāvatā kho, āvuso, bhikkhu abhiññeyyaṃ abhijānāti, pariññeyyaṃ parijānāti, abhiññeyyaṃ abhijānanto pariññeyyaṃ parijānanto diṭṭheva dhamme dukkhassantakaro hotī”ti. (1)}}\\
\begin{addmargin}[1em]{2em}
\setstretch{.5}
{\PaliGlossB{That’s how a mendicant directly knows what should be directly known and completely understands what should be completely understood. Knowing and understanding thus they make an end of suffering in this very life.”}}\\
\end{addmargin}
\end{absolutelynopagebreak}

\begin{absolutelynopagebreak}
\setstretch{.7}
{\PaliGlossA{evaṃ vutte, aññataro bhikkhu there bhikkhū etadavoca:}}\\
\begin{addmargin}[1em]{2em}
\setstretch{.5}
{\PaliGlossB{When this was said, one of the mendicants said to the senior mendicants:}}\\
\end{addmargin}
\end{absolutelynopagebreak}

\begin{absolutelynopagebreak}
\setstretch{.7}
{\PaliGlossA{“atītaṃ kho, āvuso, eko anto, anāgataṃ dutiyo anto, paccuppannaṃ majjhe, taṇhā sibbinī;}}\\
\begin{addmargin}[1em]{2em}
\setstretch{.5}
{\PaliGlossB{“The past, reverends, is one end. The future is the second end. The present is the middle. And craving is the seamstress …}}\\
\end{addmargin}
\end{absolutelynopagebreak}

\begin{absolutelynopagebreak}
\setstretch{.7}
{\PaliGlossA{taṇhā hi naṃ sibbati tassa tasseva bhavassa abhinibbattiyā.}}\\
\begin{addmargin}[1em]{2em}
\setstretch{.5}
{\PaliGlossB{    -}}\\
\end{addmargin}
\end{absolutelynopagebreak}

\begin{absolutelynopagebreak}
\setstretch{.7}
{\PaliGlossA{ettāvatā kho, āvuso, bhikkhu abhiññeyyaṃ abhijānāti, pariññeyyaṃ parijānāti, abhiññeyyaṃ abhijānanto, pariññeyyaṃ parijānanto diṭṭheva dhamme dukkhassantakaro hotī”ti. (2)}}\\
\begin{addmargin}[1em]{2em}
\setstretch{.5}
{\PaliGlossB{That’s how a mendicant directly knows … an end of suffering in this very life.”}}\\
\end{addmargin}
\end{absolutelynopagebreak}

\begin{absolutelynopagebreak}
\setstretch{.7}
{\PaliGlossA{evaṃ vutte, aññataro bhikkhu there bhikkhū etadavoca:}}\\
\begin{addmargin}[1em]{2em}
\setstretch{.5}
{\PaliGlossB{When this was said, one of the mendicants said to the senior mendicants:}}\\
\end{addmargin}
\end{absolutelynopagebreak}

\begin{absolutelynopagebreak}
\setstretch{.7}
{\PaliGlossA{“sukhā, āvuso, vedanā eko anto, dukkhā vedanā dutiyo anto, adukkhamasukhā vedanā majjhe, taṇhā sibbinī;}}\\
\begin{addmargin}[1em]{2em}
\setstretch{.5}
{\PaliGlossB{“Pleasant feeling, reverends, is one end. Painful feeling is the second end. Neutral feeling is the middle. And craving is the seamstress …}}\\
\end{addmargin}
\end{absolutelynopagebreak}

\begin{absolutelynopagebreak}
\setstretch{.7}
{\PaliGlossA{taṇhā hi naṃ sibbati tassa tasseva bhavassa abhinibbattiyā.}}\\
\begin{addmargin}[1em]{2em}
\setstretch{.5}
{\PaliGlossB{    -}}\\
\end{addmargin}
\end{absolutelynopagebreak}

\begin{absolutelynopagebreak}
\setstretch{.7}
{\PaliGlossA{ettāvatā kho, āvuso, bhikkhu abhiññeyyaṃ abhijānāti, pariññeyyaṃ parijānāti, abhiññeyyaṃ abhijānanto, pariññeyyaṃ parijānanto diṭṭheva dhamme dukkhassantakaro hotī”ti. (3)}}\\
\begin{addmargin}[1em]{2em}
\setstretch{.5}
{\PaliGlossB{That’s how a mendicant directly knows … an end of suffering in this very life.”}}\\
\end{addmargin}
\end{absolutelynopagebreak}

\begin{absolutelynopagebreak}
\setstretch{.7}
{\PaliGlossA{evaṃ vutte, aññataro bhikkhu there bhikkhū etadavoca:}}\\
\begin{addmargin}[1em]{2em}
\setstretch{.5}
{\PaliGlossB{When this was said, one of the mendicants said to the senior mendicants:}}\\
\end{addmargin}
\end{absolutelynopagebreak}

\begin{absolutelynopagebreak}
\setstretch{.7}
{\PaliGlossA{“nāmaṃ kho, āvuso, eko anto, rūpaṃ dutiyo anto, viññāṇaṃ majjhe, taṇhā sibbinī;}}\\
\begin{addmargin}[1em]{2em}
\setstretch{.5}
{\PaliGlossB{“Name, reverends, is one end. Form is the second end. Consciousness is the middle. And craving is the seamstress …}}\\
\end{addmargin}
\end{absolutelynopagebreak}

\begin{absolutelynopagebreak}
\setstretch{.7}
{\PaliGlossA{taṇhā hi naṃ sibbati tassa tasseva bhavassa abhinibbattiyā.}}\\
\begin{addmargin}[1em]{2em}
\setstretch{.5}
{\PaliGlossB{    -}}\\
\end{addmargin}
\end{absolutelynopagebreak}

\begin{absolutelynopagebreak}
\setstretch{.7}
{\PaliGlossA{ettāvatā kho, āvuso, bhikkhu abhiññeyyaṃ abhijānāti, pariññeyyaṃ parijānāti, abhiññeyyaṃ abhijānanto pariññeyyaṃ parijānanto diṭṭheva dhamme dukkhassantakaro hotī”ti. (4)}}\\
\begin{addmargin}[1em]{2em}
\setstretch{.5}
{\PaliGlossB{That’s how a mendicant directly knows … an end of suffering in this very life.”}}\\
\end{addmargin}
\end{absolutelynopagebreak}

\begin{absolutelynopagebreak}
\setstretch{.7}
{\PaliGlossA{evaṃ vutte aññataro bhikkhu there bhikkhū etadavoca:}}\\
\begin{addmargin}[1em]{2em}
\setstretch{.5}
{\PaliGlossB{When this was said, one of the mendicants said to the senior mendicants:}}\\
\end{addmargin}
\end{absolutelynopagebreak}

\begin{absolutelynopagebreak}
\setstretch{.7}
{\PaliGlossA{“cha kho, āvuso, ajjhattikāni āyatanāni eko anto, cha bāhirāni āyatanāni dutiyo anto, viññāṇaṃ majjhe, taṇhā sibbinī;}}\\
\begin{addmargin}[1em]{2em}
\setstretch{.5}
{\PaliGlossB{“The six interior sense fields, reverends, are one end. The six exterior sense fields are the second end. Consciousness is the middle. And craving is the seamstress …}}\\
\end{addmargin}
\end{absolutelynopagebreak}

\begin{absolutelynopagebreak}
\setstretch{.7}
{\PaliGlossA{taṇhā hi naṃ sibbati tassa tasseva bhavassa abhinibbattiyā.}}\\
\begin{addmargin}[1em]{2em}
\setstretch{.5}
{\PaliGlossB{    -}}\\
\end{addmargin}
\end{absolutelynopagebreak}

\begin{absolutelynopagebreak}
\setstretch{.7}
{\PaliGlossA{ettāvatā kho, āvuso, bhikkhu abhiññeyyaṃ abhijānāti, pariññeyyaṃ parijānāti, abhiññeyyaṃ abhijānanto pariññeyyaṃ parijānanto diṭṭheva dhamme dukkhassantakaro hotī”ti. (5)}}\\
\begin{addmargin}[1em]{2em}
\setstretch{.5}
{\PaliGlossB{That’s how a mendicant directly knows … an end of suffering in this very life.”}}\\
\end{addmargin}
\end{absolutelynopagebreak}

\begin{absolutelynopagebreak}
\setstretch{.7}
{\PaliGlossA{evaṃ vutte aññataro bhikkhu there bhikkhū etadavoca:}}\\
\begin{addmargin}[1em]{2em}
\setstretch{.5}
{\PaliGlossB{When this was said, one of the mendicants said to the senior mendicants:}}\\
\end{addmargin}
\end{absolutelynopagebreak}

\begin{absolutelynopagebreak}
\setstretch{.7}
{\PaliGlossA{“sakkāyo kho, āvuso, eko anto, sakkāyasamudayo dutiyo anto, sakkāyanirodho majjhe, taṇhā sibbinī;}}\\
\begin{addmargin}[1em]{2em}
\setstretch{.5}
{\PaliGlossB{“Identity, reverends, is one end. The origin of identity is the second end. The cessation of identity is the middle. And craving is the seamstress,}}\\
\end{addmargin}
\end{absolutelynopagebreak}

\begin{absolutelynopagebreak}
\setstretch{.7}
{\PaliGlossA{taṇhā hi naṃ sibbati tassa tasseva bhavassa abhinibbattiyā.}}\\
\begin{addmargin}[1em]{2em}
\setstretch{.5}
{\PaliGlossB{for craving weaves one to rebirth in this or that state of existence.}}\\
\end{addmargin}
\end{absolutelynopagebreak}

\begin{absolutelynopagebreak}
\setstretch{.7}
{\PaliGlossA{ettāvatā kho, āvuso, bhikkhu abhiññeyyaṃ abhijānāti, pariññeyyaṃ parijānāti, abhiññeyyaṃ abhijānanto pariññeyyaṃ parijānanto diṭṭheva dhamme dukkhassantakaro hotī”ti. (6)}}\\
\begin{addmargin}[1em]{2em}
\setstretch{.5}
{\PaliGlossB{That’s how a mendicant directly knows what should be directly known and completely understands what should be completely understood. Knowing and understanding thus they make an end of suffering in this very life.”}}\\
\end{addmargin}
\end{absolutelynopagebreak}

\begin{absolutelynopagebreak}
\setstretch{.7}
{\PaliGlossA{evaṃ vutte, aññataro bhikkhu there bhikkhū etadavoca:}}\\
\begin{addmargin}[1em]{2em}
\setstretch{.5}
{\PaliGlossB{When this was said, one of the mendicants said to the senior mendicants:}}\\
\end{addmargin}
\end{absolutelynopagebreak}

\begin{absolutelynopagebreak}
\setstretch{.7}
{\PaliGlossA{“byākataṃ kho, āvuso, amhehi sabbeheva yathāsakaṃ paṭibhānaṃ.}}\\
\begin{addmargin}[1em]{2em}
\setstretch{.5}
{\PaliGlossB{“Each of us has spoken from the heart.}}\\
\end{addmargin}
\end{absolutelynopagebreak}

\begin{absolutelynopagebreak}
\setstretch{.7}
{\PaliGlossA{āyāmāvuso, yena bhagavā tenupasaṅkamissāma; upasaṅkamitvā bhagavato etamatthaṃ ārocessāma.}}\\
\begin{addmargin}[1em]{2em}
\setstretch{.5}
{\PaliGlossB{Come, reverends, let’s go to the Buddha, and inform him about this.}}\\
\end{addmargin}
\end{absolutelynopagebreak}

\begin{absolutelynopagebreak}
\setstretch{.7}
{\PaliGlossA{yathā no bhagavā byākarissati tathā naṃ dhāressāmā”ti. (7)}}\\
\begin{addmargin}[1em]{2em}
\setstretch{.5}
{\PaliGlossB{As he answers, so we’ll remember it.”}}\\
\end{addmargin}
\end{absolutelynopagebreak}

\begin{absolutelynopagebreak}
\setstretch{.7}
{\PaliGlossA{“evamāvuso”ti kho therā bhikkhū tassa bhikkhuno paccassosuṃ.}}\\
\begin{addmargin}[1em]{2em}
\setstretch{.5}
{\PaliGlossB{“Yes, reverend,” those senior mendicants replied.}}\\
\end{addmargin}
\end{absolutelynopagebreak}

\begin{absolutelynopagebreak}
\setstretch{.7}
{\PaliGlossA{atha kho therā bhikkhū yena bhagavā tenupasaṅkamiṃsu; upasaṅkamitvā bhagavantaṃ abhivādetvā ekamantaṃ nisīdiṃsu.}}\\
\begin{addmargin}[1em]{2em}
\setstretch{.5}
{\PaliGlossB{Then those senior mendicants went up to the Buddha, bowed, sat down to one side,}}\\
\end{addmargin}
\end{absolutelynopagebreak}

\begin{absolutelynopagebreak}
\setstretch{.7}
{\PaliGlossA{ekamantaṃ nisinnā kho therā bhikkhū yāvatako ahosi sabbeheva saddhiṃ kathāsallāpo, taṃ sabbaṃ bhagavato ārocesuṃ.}}\\
\begin{addmargin}[1em]{2em}
\setstretch{.5}
{\PaliGlossB{and informed the Buddha of all they had discussed. They asked,}}\\
\end{addmargin}
\end{absolutelynopagebreak}

\begin{absolutelynopagebreak}
\setstretch{.7}
{\PaliGlossA{“kassa nu kho, bhante, subhāsitan”ti?}}\\
\begin{addmargin}[1em]{2em}
\setstretch{.5}
{\PaliGlossB{“Sir, who has spoken well?”}}\\
\end{addmargin}
\end{absolutelynopagebreak}

\begin{absolutelynopagebreak}
\setstretch{.7}
{\PaliGlossA{“sabbesaṃ vo, bhikkhave, subhāsitaṃ pariyāyena, api ca yaṃ mayā sandhāya bhāsitaṃ pārāyane metteyyapañhe:}}\\
\begin{addmargin}[1em]{2em}
\setstretch{.5}
{\PaliGlossB{“Mendicants, you’ve all spoken well in a way. However, this is what I was referring to in ‘The Way to the Beyond’, in ‘The Questions of Metteyya’ when I said:}}\\
\end{addmargin}
\end{absolutelynopagebreak}

\begin{absolutelynopagebreak}
\setstretch{.7}
{\PaliGlossA{‘yo ubhonte viditvāna,}}\\
\begin{addmargin}[1em]{2em}
\setstretch{.5}
{\PaliGlossB{‘The sage has known both ends,}}\\
\end{addmargin}
\end{absolutelynopagebreak}

\begin{absolutelynopagebreak}
\setstretch{.7}
{\PaliGlossA{majjhe mantā na lippati;}}\\
\begin{addmargin}[1em]{2em}
\setstretch{.5}
{\PaliGlossB{and is not stuck in the middle.}}\\
\end{addmargin}
\end{absolutelynopagebreak}

\begin{absolutelynopagebreak}
\setstretch{.7}
{\PaliGlossA{taṃ brūmi mahāpurisoti,}}\\
\begin{addmargin}[1em]{2em}
\setstretch{.5}
{\PaliGlossB{He is a great man, I declare,}}\\
\end{addmargin}
\end{absolutelynopagebreak}

\begin{absolutelynopagebreak}
\setstretch{.7}
{\PaliGlossA{sodha sibbinimaccagā’ti.}}\\
\begin{addmargin}[1em]{2em}
\setstretch{.5}
{\PaliGlossB{he has escaped the seamstress here.’}}\\
\end{addmargin}
\end{absolutelynopagebreak}

\begin{absolutelynopagebreak}
\setstretch{.7}
{\PaliGlossA{taṃ suṇātha, sādhukaṃ manasi karotha, bhāsissāmī”ti.}}\\
\begin{addmargin}[1em]{2em}
\setstretch{.5}
{\PaliGlossB{Listen and pay close attention, I will speak.”}}\\
\end{addmargin}
\end{absolutelynopagebreak}

\begin{absolutelynopagebreak}
\setstretch{.7}
{\PaliGlossA{“evaṃ, bhante”ti kho therā bhikkhū bhagavato paccassosuṃ.}}\\
\begin{addmargin}[1em]{2em}
\setstretch{.5}
{\PaliGlossB{“Yes, sir,” they replied.}}\\
\end{addmargin}
\end{absolutelynopagebreak}

\begin{absolutelynopagebreak}
\setstretch{.7}
{\PaliGlossA{bhagavā etadavoca:}}\\
\begin{addmargin}[1em]{2em}
\setstretch{.5}
{\PaliGlossB{The Buddha said this:}}\\
\end{addmargin}
\end{absolutelynopagebreak}

\begin{absolutelynopagebreak}
\setstretch{.7}
{\PaliGlossA{“phasso kho, bhikkhave, eko anto, phassasamudayo dutiyo anto, phassanirodho majjhe, taṇhā sibbinī;}}\\
\begin{addmargin}[1em]{2em}
\setstretch{.5}
{\PaliGlossB{“Contact, mendicants, is one end. The origin of contact is the second end. The cessation of contact is the middle. And craving is the seamstress,}}\\
\end{addmargin}
\end{absolutelynopagebreak}

\begin{absolutelynopagebreak}
\setstretch{.7}
{\PaliGlossA{taṇhā hi naṃ sibbati tassa tasseva bhavassa abhinibbattiyā.}}\\
\begin{addmargin}[1em]{2em}
\setstretch{.5}
{\PaliGlossB{for craving weaves one to rebirth in this or that state of existence.}}\\
\end{addmargin}
\end{absolutelynopagebreak}

\begin{absolutelynopagebreak}
\setstretch{.7}
{\PaliGlossA{ettāvatā kho, bhikkhave, bhikkhu abhiññeyyaṃ abhijānāti, pariññeyyaṃ parijānāti, abhiññeyyaṃ abhijānanto pariññeyyaṃ parijānanto diṭṭheva dhamme dukkhassantakaro hotī”ti.}}\\
\begin{addmargin}[1em]{2em}
\setstretch{.5}
{\PaliGlossB{That’s how a mendicant directly knows what should be directly known and completely understands what should be completely understood. Knowing and understanding thus they make an end of suffering in this very life.”}}\\
\end{addmargin}
\end{absolutelynopagebreak}

\begin{absolutelynopagebreak}
\setstretch{.7}
{\PaliGlossA{sattamaṃ.}}\\
\begin{addmargin}[1em]{2em}
\setstretch{.5}
{\PaliGlossB{    -}}\\
\end{addmargin}
\end{absolutelynopagebreak}
