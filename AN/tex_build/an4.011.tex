
\begin{absolutelynopagebreak}
\setstretch{.7}
{\PaliGlossA{aṅguttara nikāya 4}}\\
\begin{addmargin}[1em]{2em}
\setstretch{.5}
{\PaliGlossB{Numbered Discourses 4}}\\
\end{addmargin}
\end{absolutelynopagebreak}

\begin{absolutelynopagebreak}
\setstretch{.7}
{\PaliGlossA{2. caravagga}}\\
\begin{addmargin}[1em]{2em}
\setstretch{.5}
{\PaliGlossB{2. Walking}}\\
\end{addmargin}
\end{absolutelynopagebreak}

\begin{absolutelynopagebreak}
\setstretch{.7}
{\PaliGlossA{11. carasutta}}\\
\begin{addmargin}[1em]{2em}
\setstretch{.5}
{\PaliGlossB{11. Walking}}\\
\end{addmargin}
\end{absolutelynopagebreak}

\begin{absolutelynopagebreak}
\setstretch{.7}
{\PaliGlossA{“carato cepi, bhikkhave, bhikkhuno uppajjati kāmavitakko vā byāpādavitakko vā vihiṃsāvitakko vā.}}\\
\begin{addmargin}[1em]{2em}
\setstretch{.5}
{\PaliGlossB{“Mendicants, suppose a mendicant has a sensual, malicious, or cruel thought while walking.}}\\
\end{addmargin}
\end{absolutelynopagebreak}

\begin{absolutelynopagebreak}
\setstretch{.7}
{\PaliGlossA{tañce bhikkhu adhivāseti, nappajahati na vinodeti na byantīkaroti na anabhāvaṃ gameti, carampi, bhikkhave, bhikkhu evaṃbhūto ‘anātāpī anottāpī satataṃ samitaṃ kusīto hīnavīriyo’ti vuccati.}}\\
\begin{addmargin}[1em]{2em}
\setstretch{.5}
{\PaliGlossB{They tolerate it and don’t give it up, get rid of it, eliminate it, and obliterate it. Such a mendicant is said to be ‘not keen or prudent, always lazy, and lacking energy’ when walking.}}\\
\end{addmargin}
\end{absolutelynopagebreak}

\begin{absolutelynopagebreak}
\setstretch{.7}
{\PaliGlossA{ṭhitassa cepi, bhikkhave, bhikkhuno uppajjati kāmavitakko vā byāpādavitakko vā vihiṃsāvitakko vā.}}\\
\begin{addmargin}[1em]{2em}
\setstretch{.5}
{\PaliGlossB{Suppose a mendicant has a sensual, malicious, or cruel thought while standing …}}\\
\end{addmargin}
\end{absolutelynopagebreak}

\begin{absolutelynopagebreak}
\setstretch{.7}
{\PaliGlossA{tañce bhikkhu adhivāseti, nappajahati na vinodeti na byantīkaroti na anabhāvaṃ gameti, ṭhitopi, bhikkhave, bhikkhu evaṃbhūto ‘anātāpī anottāpī satataṃ samitaṃ kusīto hīnavīriyo’ti vuccati.}}\\
\begin{addmargin}[1em]{2em}
\setstretch{.5}
{\PaliGlossB{    -}}\\
\end{addmargin}
\end{absolutelynopagebreak}

\begin{absolutelynopagebreak}
\setstretch{.7}
{\PaliGlossA{nisinnassa cepi, bhikkhave, bhikkhuno uppajjati kāmavitakko vā byāpādavitakko vā vihiṃsāvitakko vā.}}\\
\begin{addmargin}[1em]{2em}
\setstretch{.5}
{\PaliGlossB{sitting …}}\\
\end{addmargin}
\end{absolutelynopagebreak}

\begin{absolutelynopagebreak}
\setstretch{.7}
{\PaliGlossA{tañce bhikkhu adhivāseti, nappajahati na vinodeti na byantīkaroti na anabhāvaṃ gameti, nisinnopi, bhikkhave, bhikkhu evaṃbhūto ‘anātāpī anottāpī satataṃ samitaṃ kusīto hīnavīriyo’ti vuccati.}}\\
\begin{addmargin}[1em]{2em}
\setstretch{.5}
{\PaliGlossB{    -}}\\
\end{addmargin}
\end{absolutelynopagebreak}

\begin{absolutelynopagebreak}
\setstretch{.7}
{\PaliGlossA{sayānassa cepi, bhikkhave, bhikkhuno jāgarassa uppajjati kāmavitakko vā byāpādavitakko vā vihiṃsāvitakko vā.}}\\
\begin{addmargin}[1em]{2em}
\setstretch{.5}
{\PaliGlossB{or when lying down while awake.}}\\
\end{addmargin}
\end{absolutelynopagebreak}

\begin{absolutelynopagebreak}
\setstretch{.7}
{\PaliGlossA{tañce bhikkhu adhivāseti, nappajahati na vinodeti na byantīkaroti na anabhāvaṃ gameti, sayānopi, bhikkhave, bhikkhu jāgaro evaṃbhūto ‘anātāpī anottāpī satataṃ samitaṃ kusīto hīnavīriyo’ti vuccati.}}\\
\begin{addmargin}[1em]{2em}
\setstretch{.5}
{\PaliGlossB{They tolerate it and don’t give it up, get rid of it, eliminate it, and obliterate it. Such a mendicant is said to be ‘not keen or prudent, always lazy, and lacking energy’ when lying down while awake.}}\\
\end{addmargin}
\end{absolutelynopagebreak}

\begin{absolutelynopagebreak}
\setstretch{.7}
{\PaliGlossA{carato cepi, bhikkhave, bhikkhuno uppajjati kāmavitakko vā byāpādavitakko vā vihiṃsāvitakko vā.}}\\
\begin{addmargin}[1em]{2em}
\setstretch{.5}
{\PaliGlossB{Suppose a mendicant has a sensual, malicious, or cruel thought while walking.}}\\
\end{addmargin}
\end{absolutelynopagebreak}

\begin{absolutelynopagebreak}
\setstretch{.7}
{\PaliGlossA{tañce bhikkhu nādhivāseti, pajahati vinodeti byantīkaroti anabhāvaṃ gameti;}}\\
\begin{addmargin}[1em]{2em}
\setstretch{.5}
{\PaliGlossB{They don’t tolerate it, but give it up, get rid of it, eliminate it, and obliterate it.}}\\
\end{addmargin}
\end{absolutelynopagebreak}

\begin{absolutelynopagebreak}
\setstretch{.7}
{\PaliGlossA{carampi, bhikkhave, bhikkhu evaṃbhūto ‘ātāpī ottāpī satataṃ samitaṃ āraddhavīriyo pahitatto’ti vuccati.}}\\
\begin{addmargin}[1em]{2em}
\setstretch{.5}
{\PaliGlossB{Such a mendicant is said to be ‘keen and prudent, always energetic and determined’ when walking.}}\\
\end{addmargin}
\end{absolutelynopagebreak}

\begin{absolutelynopagebreak}
\setstretch{.7}
{\PaliGlossA{ṭhitassa cepi, bhikkhave, bhikkhuno uppajjati kāmavitakko vā byāpādavitakko vā vihiṃsāvitakko vā.}}\\
\begin{addmargin}[1em]{2em}
\setstretch{.5}
{\PaliGlossB{Suppose a mendicant has a sensual, malicious, or cruel thought while standing …}}\\
\end{addmargin}
\end{absolutelynopagebreak}

\begin{absolutelynopagebreak}
\setstretch{.7}
{\PaliGlossA{tañce bhikkhu nādhivāseti, pajahati vinodeti byantīkaroti anabhāvaṃ gameti;}}\\
\begin{addmargin}[1em]{2em}
\setstretch{.5}
{\PaliGlossB{    -}}\\
\end{addmargin}
\end{absolutelynopagebreak}

\begin{absolutelynopagebreak}
\setstretch{.7}
{\PaliGlossA{ṭhitopi, bhikkhave, bhikkhu evaṃbhūto ‘ātāpī ottāpī satataṃ samitaṃ āraddhavīriyo pahitatto’ti vuccati.}}\\
\begin{addmargin}[1em]{2em}
\setstretch{.5}
{\PaliGlossB{    -}}\\
\end{addmargin}
\end{absolutelynopagebreak}

\begin{absolutelynopagebreak}
\setstretch{.7}
{\PaliGlossA{nisinnassa cepi, bhikkhave, bhikkhuno uppajjati kāmavitakko vā byāpādavitakko vā vihiṃsāvitakko vā.}}\\
\begin{addmargin}[1em]{2em}
\setstretch{.5}
{\PaliGlossB{sitting …}}\\
\end{addmargin}
\end{absolutelynopagebreak}

\begin{absolutelynopagebreak}
\setstretch{.7}
{\PaliGlossA{tañce bhikkhu nādhivāseti, pajahati vinodeti byantīkaroti anabhāvaṃ gameti;}}\\
\begin{addmargin}[1em]{2em}
\setstretch{.5}
{\PaliGlossB{    -}}\\
\end{addmargin}
\end{absolutelynopagebreak}

\begin{absolutelynopagebreak}
\setstretch{.7}
{\PaliGlossA{nisinnopi, bhikkhave, bhikkhu evaṃbhūto ‘ātāpī ottāpī satataṃ samitaṃ āraddhavīriyo pahitatto’ti vuccati.}}\\
\begin{addmargin}[1em]{2em}
\setstretch{.5}
{\PaliGlossB{    -}}\\
\end{addmargin}
\end{absolutelynopagebreak}

\begin{absolutelynopagebreak}
\setstretch{.7}
{\PaliGlossA{sayānassa cepi, bhikkhave, bhikkhuno jāgarassa uppajjati kāmavitakko vā byāpādavitakko vā vihiṃsāvitakko vā.}}\\
\begin{addmargin}[1em]{2em}
\setstretch{.5}
{\PaliGlossB{or when lying down while awake.}}\\
\end{addmargin}
\end{absolutelynopagebreak}

\begin{absolutelynopagebreak}
\setstretch{.7}
{\PaliGlossA{tañce bhikkhu nādhivāseti, pajahati vinodeti byantīkaroti anabhāvaṃ gameti;}}\\
\begin{addmargin}[1em]{2em}
\setstretch{.5}
{\PaliGlossB{They don’t tolerate it, but give it up, get rid of it, eliminate it, and obliterate it.}}\\
\end{addmargin}
\end{absolutelynopagebreak}

\begin{absolutelynopagebreak}
\setstretch{.7}
{\PaliGlossA{sayānopi, bhikkhave, bhikkhu jāgaro evaṃbhūto ‘ātāpī ottāpī satataṃ samitaṃ āraddhavīriyo pahitatto’ti vuccatīti.}}\\
\begin{addmargin}[1em]{2em}
\setstretch{.5}
{\PaliGlossB{Such a mendicant is said to be ‘keen and prudent, always energetic and determined’ when lying down while awake.”}}\\
\end{addmargin}
\end{absolutelynopagebreak}

\begin{absolutelynopagebreak}
\setstretch{.7}
{\PaliGlossA{caraṃ vā yadi vā tiṭṭhaṃ,}}\\
\begin{addmargin}[1em]{2em}
\setstretch{.5}
{\PaliGlossB{Whether walking or standing,}}\\
\end{addmargin}
\end{absolutelynopagebreak}

\begin{absolutelynopagebreak}
\setstretch{.7}
{\PaliGlossA{nisinno uda vā sayaṃ;}}\\
\begin{addmargin}[1em]{2em}
\setstretch{.5}
{\PaliGlossB{sitting or lying down,}}\\
\end{addmargin}
\end{absolutelynopagebreak}

\begin{absolutelynopagebreak}
\setstretch{.7}
{\PaliGlossA{yo vitakkaṃ vitakketi,}}\\
\begin{addmargin}[1em]{2em}
\setstretch{.5}
{\PaliGlossB{if you think a bad thought}}\\
\end{addmargin}
\end{absolutelynopagebreak}

\begin{absolutelynopagebreak}
\setstretch{.7}
{\PaliGlossA{pāpakaṃ gehanissitaṃ.}}\\
\begin{addmargin}[1em]{2em}
\setstretch{.5}
{\PaliGlossB{to do with the lay life,}}\\
\end{addmargin}
\end{absolutelynopagebreak}

\begin{absolutelynopagebreak}
\setstretch{.7}
{\PaliGlossA{kummaggappaṭipanno so,}}\\
\begin{addmargin}[1em]{2em}
\setstretch{.5}
{\PaliGlossB{you’re on the wrong path,}}\\
\end{addmargin}
\end{absolutelynopagebreak}

\begin{absolutelynopagebreak}
\setstretch{.7}
{\PaliGlossA{mohaneyyesu mucchito;}}\\
\begin{addmargin}[1em]{2em}
\setstretch{.5}
{\PaliGlossB{lost among things that delude.}}\\
\end{addmargin}
\end{absolutelynopagebreak}

\begin{absolutelynopagebreak}
\setstretch{.7}
{\PaliGlossA{abhabbo tādiso bhikkhu,}}\\
\begin{addmargin}[1em]{2em}
\setstretch{.5}
{\PaliGlossB{Such a mendicant is incapable}}\\
\end{addmargin}
\end{absolutelynopagebreak}

\begin{absolutelynopagebreak}
\setstretch{.7}
{\PaliGlossA{phuṭṭhuṃ sambodhimuttamaṃ.}}\\
\begin{addmargin}[1em]{2em}
\setstretch{.5}
{\PaliGlossB{of touching the highest awakening.}}\\
\end{addmargin}
\end{absolutelynopagebreak}

\begin{absolutelynopagebreak}
\setstretch{.7}
{\PaliGlossA{yo ca caraṃ vā tiṭṭhaṃ vā,}}\\
\begin{addmargin}[1em]{2em}
\setstretch{.5}
{\PaliGlossB{But one who, whether standing or walking,}}\\
\end{addmargin}
\end{absolutelynopagebreak}

\begin{absolutelynopagebreak}
\setstretch{.7}
{\PaliGlossA{nisinno uda vā sayaṃ;}}\\
\begin{addmargin}[1em]{2em}
\setstretch{.5}
{\PaliGlossB{sitting or lying down,}}\\
\end{addmargin}
\end{absolutelynopagebreak}

\begin{absolutelynopagebreak}
\setstretch{.7}
{\PaliGlossA{vitakkaṃ samayitvāna,}}\\
\begin{addmargin}[1em]{2em}
\setstretch{.5}
{\PaliGlossB{has calmed their thoughts,}}\\
\end{addmargin}
\end{absolutelynopagebreak}

\begin{absolutelynopagebreak}
\setstretch{.7}
{\PaliGlossA{vitakkūpasame rato;}}\\
\begin{addmargin}[1em]{2em}
\setstretch{.5}
{\PaliGlossB{loving peace of mind;}}\\
\end{addmargin}
\end{absolutelynopagebreak}

\begin{absolutelynopagebreak}
\setstretch{.7}
{\PaliGlossA{bhabbo so tādiso bhikkhu,}}\\
\begin{addmargin}[1em]{2em}
\setstretch{.5}
{\PaliGlossB{such a mendicant is capable}}\\
\end{addmargin}
\end{absolutelynopagebreak}

\begin{absolutelynopagebreak}
\setstretch{.7}
{\PaliGlossA{phuṭṭhuṃ sambodhimuttaman”ti.}}\\
\begin{addmargin}[1em]{2em}
\setstretch{.5}
{\PaliGlossB{of touching the highest awakening.”}}\\
\end{addmargin}
\end{absolutelynopagebreak}

\begin{absolutelynopagebreak}
\setstretch{.7}
{\PaliGlossA{paṭhamaṃ.}}\\
\begin{addmargin}[1em]{2em}
\setstretch{.5}
{\PaliGlossB{    -}}\\
\end{addmargin}
\end{absolutelynopagebreak}
