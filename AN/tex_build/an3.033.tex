
\begin{absolutelynopagebreak}
\setstretch{.7}
{\PaliGlossA{aṅguttara nikāya 3}}\\
\begin{addmargin}[1em]{2em}
\setstretch{.5}
{\PaliGlossB{Numbered Discourses 3}}\\
\end{addmargin}
\end{absolutelynopagebreak}

\begin{absolutelynopagebreak}
\setstretch{.7}
{\PaliGlossA{4. devadūtavagga}}\\
\begin{addmargin}[1em]{2em}
\setstretch{.5}
{\PaliGlossB{4. Messengers of the Gods}}\\
\end{addmargin}
\end{absolutelynopagebreak}

\begin{absolutelynopagebreak}
\setstretch{.7}
{\PaliGlossA{33. sāriputtasutta}}\\
\begin{addmargin}[1em]{2em}
\setstretch{.5}
{\PaliGlossB{33. With Sāriputta}}\\
\end{addmargin}
\end{absolutelynopagebreak}

\begin{absolutelynopagebreak}
\setstretch{.7}
{\PaliGlossA{atha kho āyasmā sāriputto yena bhagavā tenupasaṅkami; upasaṅkamitvā bhagavantaṃ abhivādetvā ekamantaṃ nisīdi. ekamantaṃ nisinnaṃ kho āyasmantaṃ sāriputtaṃ bhagavā etadavoca:}}\\
\begin{addmargin}[1em]{2em}
\setstretch{.5}
{\PaliGlossB{Then Venerable Sāriputta went up to the Buddha, bowed, and sat down to one side. The Buddha said to him,}}\\
\end{addmargin}
\end{absolutelynopagebreak}

\begin{absolutelynopagebreak}
\setstretch{.7}
{\PaliGlossA{“saṅkhittenapi kho ahaṃ, sāriputta, dhammaṃ deseyyaṃ;}}\\
\begin{addmargin}[1em]{2em}
\setstretch{.5}
{\PaliGlossB{“Maybe I’ll teach Dhamma in brief,}}\\
\end{addmargin}
\end{absolutelynopagebreak}

\begin{absolutelynopagebreak}
\setstretch{.7}
{\PaliGlossA{vitthārenapi kho ahaṃ, sāriputta, dhammaṃ deseyyaṃ;}}\\
\begin{addmargin}[1em]{2em}
\setstretch{.5}
{\PaliGlossB{maybe in detail,}}\\
\end{addmargin}
\end{absolutelynopagebreak}

\begin{absolutelynopagebreak}
\setstretch{.7}
{\PaliGlossA{saṅkhittavitthārenapi kho ahaṃ, sāriputta, dhammaṃ deseyyaṃ;}}\\
\begin{addmargin}[1em]{2em}
\setstretch{.5}
{\PaliGlossB{maybe both in brief and in detail.}}\\
\end{addmargin}
\end{absolutelynopagebreak}

\begin{absolutelynopagebreak}
\setstretch{.7}
{\PaliGlossA{aññātāro ca dullabhā”ti.}}\\
\begin{addmargin}[1em]{2em}
\setstretch{.5}
{\PaliGlossB{But it’s hard to find anyone who understands.”}}\\
\end{addmargin}
\end{absolutelynopagebreak}

\begin{absolutelynopagebreak}
\setstretch{.7}
{\PaliGlossA{“etassa, bhagavā, kālo, etassa, sugata, kālo}}\\
\begin{addmargin}[1em]{2em}
\setstretch{.5}
{\PaliGlossB{“Now is the time, Blessed One! Now is the time, Holy One!}}\\
\end{addmargin}
\end{absolutelynopagebreak}

\begin{absolutelynopagebreak}
\setstretch{.7}
{\PaliGlossA{yaṃ bhagavā saṅkhittenapi dhammaṃ deseyya, vitthārenapi dhammaṃ deseyya, saṅkhittavitthārenapi dhammaṃ deseyya.}}\\
\begin{addmargin}[1em]{2em}
\setstretch{.5}
{\PaliGlossB{Let the Buddha teach Dhamma in brief, in detail, and both in brief and in detail.}}\\
\end{addmargin}
\end{absolutelynopagebreak}

\begin{absolutelynopagebreak}
\setstretch{.7}
{\PaliGlossA{bhavissanti dhammassa aññātāro”ti.}}\\
\begin{addmargin}[1em]{2em}
\setstretch{.5}
{\PaliGlossB{There will be those who understand the teaching!”}}\\
\end{addmargin}
\end{absolutelynopagebreak}

\begin{absolutelynopagebreak}
\setstretch{.7}
{\PaliGlossA{“tasmātiha, sāriputta, evaṃ sikkhitabbaṃ:}}\\
\begin{addmargin}[1em]{2em}
\setstretch{.5}
{\PaliGlossB{“So, Sāriputta, you should train like this:}}\\
\end{addmargin}
\end{absolutelynopagebreak}

\begin{absolutelynopagebreak}
\setstretch{.7}
{\PaliGlossA{‘imasmiñca saviññāṇake kāye ahaṅkāramamaṅkāramānānusayā na bhavissanti, bahiddhā ca sabbanimittesu ahaṅkāramamaṅkāramānānusayā na bhavissanti, yañca cetovimuttiṃ paññāvimuttiṃ upasampajja viharato ahaṅkāramamaṅkāramānānusayā na honti tañca cetovimuttiṃ paññāvimuttiṃ upasampajja viharissāmā’ti.}}\\
\begin{addmargin}[1em]{2em}
\setstretch{.5}
{\PaliGlossB{‘There’ll be no ego, possessiveness, or underlying tendency to conceit for this conscious body; and no ego, possessiveness, or underlying tendency to conceit for all external stimuli; and we’ll live having achieved the freedom of heart and freedom by wisdom where ego, possessiveness, and underlying tendency to conceit are no more.’}}\\
\end{addmargin}
\end{absolutelynopagebreak}

\begin{absolutelynopagebreak}
\setstretch{.7}
{\PaliGlossA{evañhi kho, sāriputta, sikkhitabbaṃ.}}\\
\begin{addmargin}[1em]{2em}
\setstretch{.5}
{\PaliGlossB{That’s how you should train.}}\\
\end{addmargin}
\end{absolutelynopagebreak}

\begin{absolutelynopagebreak}
\setstretch{.7}
{\PaliGlossA{yato ca kho, sāriputta, bhikkhuno imasmiñca saviññāṇake kāye ahaṅkāramamaṅkāramānānusayā na honti, bahiddhā ca sabbanimittesu ahaṅkāramamaṅkāramānānusayā na honti, yañca cetovimuttiṃ paññāvimuttiṃ upasampajja viharato ahaṅkāramamaṅkāramānānusayā na honti tañca cetovimuttiṃ paññāvimuttiṃ upasampajja viharati;}}\\
\begin{addmargin}[1em]{2em}
\setstretch{.5}
{\PaliGlossB{When a mendicant has no ego, possessiveness, or underlying tendency to conceit for this conscious body; and no ego, possessiveness, or underlying tendency to conceit for all external stimuli; and they live having attained the freedom of heart and freedom by wisdom where ego, possessiveness, and underlying tendency to conceit are no more—}}\\
\end{addmargin}
\end{absolutelynopagebreak}

\begin{absolutelynopagebreak}
\setstretch{.7}
{\PaliGlossA{ayaṃ vuccati, sāriputta:}}\\
\begin{addmargin}[1em]{2em}
\setstretch{.5}
{\PaliGlossB{they’re called}}\\
\end{addmargin}
\end{absolutelynopagebreak}

\begin{absolutelynopagebreak}
\setstretch{.7}
{\PaliGlossA{‘bhikkhu acchecchi taṇhaṃ, vivattayi saṃyojanaṃ, sammā mānābhisamayā antamakāsi dukkhassa’.}}\\
\begin{addmargin}[1em]{2em}
\setstretch{.5}
{\PaliGlossB{a mendicant who has cut off craving, untied the fetters, and by rightly comprehending conceit has made an end of suffering.}}\\
\end{addmargin}
\end{absolutelynopagebreak}

\begin{absolutelynopagebreak}
\setstretch{.7}
{\PaliGlossA{idañca pana metaṃ, sāriputta, sandhāya bhāsitaṃ pārāyane udayapañhe:}}\\
\begin{addmargin}[1em]{2em}
\setstretch{.5}
{\PaliGlossB{And Sāriputta, this is what I was referring to in ‘The Way to the Beyond’, in ‘The Questions of Udaya’ when I said:}}\\
\end{addmargin}
\end{absolutelynopagebreak}

\begin{absolutelynopagebreak}
\setstretch{.7}
{\PaliGlossA{‘pahānaṃ kāmasaññānaṃ,}}\\
\begin{addmargin}[1em]{2em}
\setstretch{.5}
{\PaliGlossB{‘The abandoning of both}}\\
\end{addmargin}
\end{absolutelynopagebreak}

\begin{absolutelynopagebreak}
\setstretch{.7}
{\PaliGlossA{domanassāna cūbhayaṃ;}}\\
\begin{addmargin}[1em]{2em}
\setstretch{.5}
{\PaliGlossB{sensual perceptions and aversion;}}\\
\end{addmargin}
\end{absolutelynopagebreak}

\begin{absolutelynopagebreak}
\setstretch{.7}
{\PaliGlossA{thinassa ca panūdanaṃ,}}\\
\begin{addmargin}[1em]{2em}
\setstretch{.5}
{\PaliGlossB{the dispelling of dullness,}}\\
\end{addmargin}
\end{absolutelynopagebreak}

\begin{absolutelynopagebreak}
\setstretch{.7}
{\PaliGlossA{kukkuccānaṃ nivāraṇaṃ.}}\\
\begin{addmargin}[1em]{2em}
\setstretch{.5}
{\PaliGlossB{and the cessation of remorse.}}\\
\end{addmargin}
\end{absolutelynopagebreak}

\begin{absolutelynopagebreak}
\setstretch{.7}
{\PaliGlossA{upekkhāsatisaṃsuddhaṃ,}}\\
\begin{addmargin}[1em]{2em}
\setstretch{.5}
{\PaliGlossB{Pure equanimity and mindfulness,}}\\
\end{addmargin}
\end{absolutelynopagebreak}

\begin{absolutelynopagebreak}
\setstretch{.7}
{\PaliGlossA{dhammatakkapurejavaṃ;}}\\
\begin{addmargin}[1em]{2em}
\setstretch{.5}
{\PaliGlossB{preceded by investigation of principles—}}\\
\end{addmargin}
\end{absolutelynopagebreak}

\begin{absolutelynopagebreak}
\setstretch{.7}
{\PaliGlossA{aññāvimokkhaṃ pabrūmi,}}\\
\begin{addmargin}[1em]{2em}
\setstretch{.5}
{\PaliGlossB{this, I declare, is liberation by enlightenment,}}\\
\end{addmargin}
\end{absolutelynopagebreak}

\begin{absolutelynopagebreak}
\setstretch{.7}
{\PaliGlossA{avijjāya pabhedanan’”ti.}}\\
\begin{addmargin}[1em]{2em}
\setstretch{.5}
{\PaliGlossB{with the smashing of ignorance.’”}}\\
\end{addmargin}
\end{absolutelynopagebreak}

\begin{absolutelynopagebreak}
\setstretch{.7}
{\PaliGlossA{tatiyaṃ.}}\\
\begin{addmargin}[1em]{2em}
\setstretch{.5}
{\PaliGlossB{    -}}\\
\end{addmargin}
\end{absolutelynopagebreak}
