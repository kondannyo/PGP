
\begin{absolutelynopagebreak}
\setstretch{.7}
{\PaliGlossA{aṅguttara nikāya 3}}\\
\begin{addmargin}[1em]{2em}
\setstretch{.5}
{\PaliGlossB{Numbered Discourses 3}}\\
\end{addmargin}
\end{absolutelynopagebreak}

\begin{absolutelynopagebreak}
\setstretch{.7}
{\PaliGlossA{7. mahāvagga}}\\
\begin{addmargin}[1em]{2em}
\setstretch{.5}
{\PaliGlossB{7. The Great Chapter}}\\
\end{addmargin}
\end{absolutelynopagebreak}

\begin{absolutelynopagebreak}
\setstretch{.7}
{\PaliGlossA{67. kathāvatthusutta}}\\
\begin{addmargin}[1em]{2em}
\setstretch{.5}
{\PaliGlossB{67. Topics of Discussion}}\\
\end{addmargin}
\end{absolutelynopagebreak}

\begin{absolutelynopagebreak}
\setstretch{.7}
{\PaliGlossA{“tīṇimāni, bhikkhave, kathāvatthūni.}}\\
\begin{addmargin}[1em]{2em}
\setstretch{.5}
{\PaliGlossB{“There are, mendicants, these three topics of discussion.}}\\
\end{addmargin}
\end{absolutelynopagebreak}

\begin{absolutelynopagebreak}
\setstretch{.7}
{\PaliGlossA{katamāni tīṇi?}}\\
\begin{addmargin}[1em]{2em}
\setstretch{.5}
{\PaliGlossB{What three?}}\\
\end{addmargin}
\end{absolutelynopagebreak}

\begin{absolutelynopagebreak}
\setstretch{.7}
{\PaliGlossA{atītaṃ vā, bhikkhave, addhānaṃ ārabbha kathaṃ katheyya:}}\\
\begin{addmargin}[1em]{2em}
\setstretch{.5}
{\PaliGlossB{You might discuss the past:}}\\
\end{addmargin}
\end{absolutelynopagebreak}

\begin{absolutelynopagebreak}
\setstretch{.7}
{\PaliGlossA{‘evaṃ ahosi atītamaddhānan’ti.}}\\
\begin{addmargin}[1em]{2em}
\setstretch{.5}
{\PaliGlossB{‘That is how it was in the past.’}}\\
\end{addmargin}
\end{absolutelynopagebreak}

\begin{absolutelynopagebreak}
\setstretch{.7}
{\PaliGlossA{anāgataṃ vā, bhikkhave, addhānaṃ ārabbha kathaṃ katheyya:}}\\
\begin{addmargin}[1em]{2em}
\setstretch{.5}
{\PaliGlossB{You might discuss the future:}}\\
\end{addmargin}
\end{absolutelynopagebreak}

\begin{absolutelynopagebreak}
\setstretch{.7}
{\PaliGlossA{‘evaṃ bhavissati anāgatamaddhānan’ti.}}\\
\begin{addmargin}[1em]{2em}
\setstretch{.5}
{\PaliGlossB{‘That is how it will be in the future.’}}\\
\end{addmargin}
\end{absolutelynopagebreak}

\begin{absolutelynopagebreak}
\setstretch{.7}
{\PaliGlossA{etarahi vā, bhikkhave, paccuppannaṃ addhānaṃ ārabbha kathaṃ katheyya:}}\\
\begin{addmargin}[1em]{2em}
\setstretch{.5}
{\PaliGlossB{Or you might discuss the present:}}\\
\end{addmargin}
\end{absolutelynopagebreak}

\begin{absolutelynopagebreak}
\setstretch{.7}
{\PaliGlossA{‘evaṃ hoti etarahi paccuppannamaddhānan’ti.}}\\
\begin{addmargin}[1em]{2em}
\setstretch{.5}
{\PaliGlossB{‘This is how it is at present.’}}\\
\end{addmargin}
\end{absolutelynopagebreak}

\begin{absolutelynopagebreak}
\setstretch{.7}
{\PaliGlossA{kathāsampayogena, bhikkhave, puggalo veditabbo yadi vā kaccho yadi vā akacchoti.}}\\
\begin{addmargin}[1em]{2em}
\setstretch{.5}
{\PaliGlossB{You can know whether or not a person is competent to hold a discussion by seeing how they take part in a discussion.}}\\
\end{addmargin}
\end{absolutelynopagebreak}

\begin{absolutelynopagebreak}
\setstretch{.7}
{\PaliGlossA{sacāyaṃ, bhikkhave, puggalo pañhaṃ puṭṭho samāno ekaṃsabyākaraṇīyaṃ pañhaṃ na ekaṃsena byākaroti, vibhajjabyākaraṇīyaṃ pañhaṃ na vibhajja byākaroti, paṭipucchābyākaraṇīyaṃ pañhaṃ na paṭipucchā byākaroti, ṭhapanīyaṃ pañhaṃ na ṭhapeti, evaṃ santāyaṃ, bhikkhave, puggalo akaccho hoti.}}\\
\begin{addmargin}[1em]{2em}
\setstretch{.5}
{\PaliGlossB{When a person is asked a question, if it needs to be answered with a generalization and they don’t answer it generally; or if it needs analysis and they answer without analyzing it; or if it needs a counter-question and they answer without a counter-question; or if it should be set aside and they don’t set it aside, then that person is not competent to hold a discussion.}}\\
\end{addmargin}
\end{absolutelynopagebreak}

\begin{absolutelynopagebreak}
\setstretch{.7}
{\PaliGlossA{sace panāyaṃ, bhikkhave, puggalo pañhaṃ puṭṭho samāno ekaṃsabyākaraṇīyaṃ pañhaṃ ekaṃsena byākaroti, vibhajjabyākaraṇīyaṃ pañhaṃ vibhajja byākaroti, paṭipucchābyākaraṇīyaṃ pañhaṃ paṭipucchā byākaroti, ṭhapanīyaṃ pañhaṃ ṭhapeti, evaṃ santāyaṃ, bhikkhave, puggalo kaccho hoti.}}\\
\begin{addmargin}[1em]{2em}
\setstretch{.5}
{\PaliGlossB{When a person is asked a question, if it needs to be answered with a generalization and they answer it generally; or if it needs analysis and they answer after analyzing it; or if it needs a counter-question and they answer with a counter-question; or if it should be set aside and they set it aside, then that person is competent to hold a discussion.}}\\
\end{addmargin}
\end{absolutelynopagebreak}

\begin{absolutelynopagebreak}
\setstretch{.7}
{\PaliGlossA{kathāsampayogena, bhikkhave, puggalo veditabbo yadi vā kaccho yadi vā akacchoti.}}\\
\begin{addmargin}[1em]{2em}
\setstretch{.5}
{\PaliGlossB{You can know whether or not a person is competent to hold a discussion by seeing how they take part in a discussion.}}\\
\end{addmargin}
\end{absolutelynopagebreak}

\begin{absolutelynopagebreak}
\setstretch{.7}
{\PaliGlossA{sacāyaṃ, bhikkhave, puggalo pañhaṃ puṭṭho samāno ṭhānāṭhāne na saṇṭhāti parikappe na saṇṭhāti aññātavāde na saṇṭhāti paṭipadāya na saṇṭhāti, evaṃ santāyaṃ, bhikkhave, puggalo akaccho hoti.}}\\
\begin{addmargin}[1em]{2em}
\setstretch{.5}
{\PaliGlossB{When a person is asked a question, if they’re not consistent about what their position is and what it isn’t; about what they propose; about speaking from what they know; and about the appropriate procedure, then that person is not competent to hold a discussion.}}\\
\end{addmargin}
\end{absolutelynopagebreak}

\begin{absolutelynopagebreak}
\setstretch{.7}
{\PaliGlossA{sace panāyaṃ, bhikkhave, puggalo pañhaṃ puṭṭho samāno ṭhānāṭhāne saṇṭhāti parikappe saṇṭhāti aññātavāde saṇṭhāti paṭipadāya saṇṭhāti, evaṃ santāyaṃ, bhikkhave, puggalo kaccho hoti.}}\\
\begin{addmargin}[1em]{2em}
\setstretch{.5}
{\PaliGlossB{When a person is asked a question, if they are consistent about what their position is and what it isn’t; about what they propose; about speaking from what they know; and about the appropriate procedure, then that person is competent to hold a discussion.}}\\
\end{addmargin}
\end{absolutelynopagebreak}

\begin{absolutelynopagebreak}
\setstretch{.7}
{\PaliGlossA{kathāsampayogena, bhikkhave, puggalo veditabbo yadi vā kaccho yadi vā akacchoti.}}\\
\begin{addmargin}[1em]{2em}
\setstretch{.5}
{\PaliGlossB{You can know whether or not a person is competent to hold a discussion by seeing how they take part in a discussion.}}\\
\end{addmargin}
\end{absolutelynopagebreak}

\begin{absolutelynopagebreak}
\setstretch{.7}
{\PaliGlossA{sacāyaṃ, bhikkhave, puggalo pañhaṃ puṭṭho samāno aññenaññaṃ paṭicarati, bahiddhā kathaṃ apanāmeti, kopañca dosañca appaccayañca pātukaroti, evaṃ santāyaṃ, bhikkhave, puggalo akaccho hoti.}}\\
\begin{addmargin}[1em]{2em}
\setstretch{.5}
{\PaliGlossB{When a person is asked a question, if they dodge the issue; distract the discussion with irrelevant points; or display annoyance, hate, and bitterness, then that person is not competent to hold a discussion.}}\\
\end{addmargin}
\end{absolutelynopagebreak}

\begin{absolutelynopagebreak}
\setstretch{.7}
{\PaliGlossA{sace panāyaṃ, bhikkhave, puggalo pañhaṃ puṭṭho samāno na aññenaññaṃ paṭicarati na bahiddhā kathaṃ apanāmeti, na kopañca dosañca appaccayañca pātukaroti, evaṃ santāyaṃ, bhikkhave, puggalo kaccho hoti.}}\\
\begin{addmargin}[1em]{2em}
\setstretch{.5}
{\PaliGlossB{When a person is asked a question, if they don’t dodge the issue; distract the discussion with irrelevant points; or display annoyance, hate, and bitterness, then that person is competent to hold a discussion.}}\\
\end{addmargin}
\end{absolutelynopagebreak}

\begin{absolutelynopagebreak}
\setstretch{.7}
{\PaliGlossA{kathāsampayogena, bhikkhave, puggalo veditabbo yadi vā kaccho yadi vā akacchoti.}}\\
\begin{addmargin}[1em]{2em}
\setstretch{.5}
{\PaliGlossB{You can know whether or not a person is competent to hold a discussion by seeing how they take part in a discussion.}}\\
\end{addmargin}
\end{absolutelynopagebreak}

\begin{absolutelynopagebreak}
\setstretch{.7}
{\PaliGlossA{sacāyaṃ, bhikkhave, puggalo pañhaṃ puṭṭho samāno abhiharati abhimaddati anupajagghati khalitaṃ gaṇhāti, evaṃ santāyaṃ, bhikkhave, puggalo akaccho hoti.}}\\
\begin{addmargin}[1em]{2em}
\setstretch{.5}
{\PaliGlossB{When a person is asked a question, if they intimidate, crush, mock, or seize on trivial mistakes, then that person is not competent to hold a discussion.}}\\
\end{addmargin}
\end{absolutelynopagebreak}

\begin{absolutelynopagebreak}
\setstretch{.7}
{\PaliGlossA{sace panāyaṃ, bhikkhave, puggalo pañhaṃ puṭṭho samāno nābhiharati nābhimaddati na anupajagghati na khalitaṃ gaṇhāti, evaṃ santāyaṃ, bhikkhave, puggalo kaccho hoti.}}\\
\begin{addmargin}[1em]{2em}
\setstretch{.5}
{\PaliGlossB{When a person is asked a question, if they don’t intimidate, crush, mock, or seize on trivial mistakes, then that person is competent to hold a discussion.}}\\
\end{addmargin}
\end{absolutelynopagebreak}

\begin{absolutelynopagebreak}
\setstretch{.7}
{\PaliGlossA{kathāsampayogena, bhikkhave, puggalo veditabbo yadi vā saupaniso yadi vā anupanisoti.}}\\
\begin{addmargin}[1em]{2em}
\setstretch{.5}
{\PaliGlossB{You can know whether or not a person has what’s required by seeing how they take part in a discussion.}}\\
\end{addmargin}
\end{absolutelynopagebreak}

\begin{absolutelynopagebreak}
\setstretch{.7}
{\PaliGlossA{anohitasoto, bhikkhave, anupaniso hoti, ohitasoto saupaniso hoti.}}\\
\begin{addmargin}[1em]{2em}
\setstretch{.5}
{\PaliGlossB{If they lend an ear they have what’s required; if they don’t lend an ear they don’t have what’s required.}}\\
\end{addmargin}
\end{absolutelynopagebreak}

\begin{absolutelynopagebreak}
\setstretch{.7}
{\PaliGlossA{so saupaniso samāno abhijānāti ekaṃ dhammaṃ, parijānāti ekaṃ dhammaṃ, pajahati ekaṃ dhammaṃ, sacchikaroti ekaṃ dhammaṃ.}}\\
\begin{addmargin}[1em]{2em}
\setstretch{.5}
{\PaliGlossB{Someone who has what’s required directly knows one thing, completely understands one thing, gives up one thing, and realizes one thing—}}\\
\end{addmargin}
\end{absolutelynopagebreak}

\begin{absolutelynopagebreak}
\setstretch{.7}
{\PaliGlossA{so abhijānanto ekaṃ dhammaṃ, parijānanto ekaṃ dhammaṃ, pajahanto ekaṃ dhammaṃ, sacchikaronto ekaṃ dhammaṃ sammāvimuttiṃ phusati.}}\\
\begin{addmargin}[1em]{2em}
\setstretch{.5}
{\PaliGlossB{and then they experience complete freedom.}}\\
\end{addmargin}
\end{absolutelynopagebreak}

\begin{absolutelynopagebreak}
\setstretch{.7}
{\PaliGlossA{etadatthā, bhikkhave, kathā;}}\\
\begin{addmargin}[1em]{2em}
\setstretch{.5}
{\PaliGlossB{This is the purpose of discussion,}}\\
\end{addmargin}
\end{absolutelynopagebreak}

\begin{absolutelynopagebreak}
\setstretch{.7}
{\PaliGlossA{etadatthā mantanā;}}\\
\begin{addmargin}[1em]{2em}
\setstretch{.5}
{\PaliGlossB{consultation,}}\\
\end{addmargin}
\end{absolutelynopagebreak}

\begin{absolutelynopagebreak}
\setstretch{.7}
{\PaliGlossA{etadatthā upanisā;}}\\
\begin{addmargin}[1em]{2em}
\setstretch{.5}
{\PaliGlossB{the requirements,}}\\
\end{addmargin}
\end{absolutelynopagebreak}

\begin{absolutelynopagebreak}
\setstretch{.7}
{\PaliGlossA{etadatthaṃ sotāvadhānaṃ, yadidaṃ anupādā cittassa vimokkhoti.}}\\
\begin{addmargin}[1em]{2em}
\setstretch{.5}
{\PaliGlossB{and listening well, that is, the liberation of the mind by not grasping.}}\\
\end{addmargin}
\end{absolutelynopagebreak}

\begin{absolutelynopagebreak}
\setstretch{.7}
{\PaliGlossA{ye viruddhā sallapanti,}}\\
\begin{addmargin}[1em]{2em}
\setstretch{.5}
{\PaliGlossB{Those who converse with hostility,}}\\
\end{addmargin}
\end{absolutelynopagebreak}

\begin{absolutelynopagebreak}
\setstretch{.7}
{\PaliGlossA{viniviṭṭhā samussitā;}}\\
\begin{addmargin}[1em]{2em}
\setstretch{.5}
{\PaliGlossB{too sure of themselves, arrogant,}}\\
\end{addmargin}
\end{absolutelynopagebreak}

\begin{absolutelynopagebreak}
\setstretch{.7}
{\PaliGlossA{anariyaguṇamāsajja,}}\\
\begin{addmargin}[1em]{2em}
\setstretch{.5}
{\PaliGlossB{ignoble, attacking virtues,}}\\
\end{addmargin}
\end{absolutelynopagebreak}

\begin{absolutelynopagebreak}
\setstretch{.7}
{\PaliGlossA{aññoññavivaresino.}}\\
\begin{addmargin}[1em]{2em}
\setstretch{.5}
{\PaliGlossB{they look for flaws in each other.}}\\
\end{addmargin}
\end{absolutelynopagebreak}

\begin{absolutelynopagebreak}
\setstretch{.7}
{\PaliGlossA{dubbhāsitaṃ vikkhalitaṃ,}}\\
\begin{addmargin}[1em]{2em}
\setstretch{.5}
{\PaliGlossB{They rejoice together when their opponent}}\\
\end{addmargin}
\end{absolutelynopagebreak}

\begin{absolutelynopagebreak}
\setstretch{.7}
{\PaliGlossA{sampamohaṃ parājayaṃ;}}\\
\begin{addmargin}[1em]{2em}
\setstretch{.5}
{\PaliGlossB{speaks poorly and makes a mistake,}}\\
\end{addmargin}
\end{absolutelynopagebreak}

\begin{absolutelynopagebreak}
\setstretch{.7}
{\PaliGlossA{aññoññassābhinandanti,}}\\
\begin{addmargin}[1em]{2em}
\setstretch{.5}
{\PaliGlossB{becoming confused and defeated—}}\\
\end{addmargin}
\end{absolutelynopagebreak}

\begin{absolutelynopagebreak}
\setstretch{.7}
{\PaliGlossA{tadariyo kathanācare.}}\\
\begin{addmargin}[1em]{2em}
\setstretch{.5}
{\PaliGlossB{but the noble ones don’t discuss like this.}}\\
\end{addmargin}
\end{absolutelynopagebreak}

\begin{absolutelynopagebreak}
\setstretch{.7}
{\PaliGlossA{sace cassa kathākāmo,}}\\
\begin{addmargin}[1em]{2em}
\setstretch{.5}
{\PaliGlossB{If an astute person wants to hold a discussion}}\\
\end{addmargin}
\end{absolutelynopagebreak}

\begin{absolutelynopagebreak}
\setstretch{.7}
{\PaliGlossA{kālamaññāya paṇḍito;}}\\
\begin{addmargin}[1em]{2em}
\setstretch{.5}
{\PaliGlossB{connected with the teaching and its meaning—}}\\
\end{addmargin}
\end{absolutelynopagebreak}

\begin{absolutelynopagebreak}
\setstretch{.7}
{\PaliGlossA{dhammaṭṭhapaṭisaṃyuttā,}}\\
\begin{addmargin}[1em]{2em}
\setstretch{.5}
{\PaliGlossB{the kind of discussion that noble ones hold—}}\\
\end{addmargin}
\end{absolutelynopagebreak}

\begin{absolutelynopagebreak}
\setstretch{.7}
{\PaliGlossA{yā ariyacaritā kathā.}}\\
\begin{addmargin}[1em]{2em}
\setstretch{.5}
{\PaliGlossB{then that wise one should start the discussion,}}\\
\end{addmargin}
\end{absolutelynopagebreak}

\begin{absolutelynopagebreak}
\setstretch{.7}
{\PaliGlossA{taṃ kathaṃ kathaye dhīro,}}\\
\begin{addmargin}[1em]{2em}
\setstretch{.5}
{\PaliGlossB{knowing when the time is right,}}\\
\end{addmargin}
\end{absolutelynopagebreak}

\begin{absolutelynopagebreak}
\setstretch{.7}
{\PaliGlossA{aviruddho anussito;}}\\
\begin{addmargin}[1em]{2em}
\setstretch{.5}
{\PaliGlossB{neither hostile nor arrogant.}}\\
\end{addmargin}
\end{absolutelynopagebreak}

\begin{absolutelynopagebreak}
\setstretch{.7}
{\PaliGlossA{anunnatena manasā,}}\\
\begin{addmargin}[1em]{2em}
\setstretch{.5}
{\PaliGlossB{Not over-excited,}}\\
\end{addmargin}
\end{absolutelynopagebreak}

\begin{absolutelynopagebreak}
\setstretch{.7}
{\PaliGlossA{apaḷāso asāhaso.}}\\
\begin{addmargin}[1em]{2em}
\setstretch{.5}
{\PaliGlossB{contemptuous, or aggressive,}}\\
\end{addmargin}
\end{absolutelynopagebreak}

\begin{absolutelynopagebreak}
\setstretch{.7}
{\PaliGlossA{anusūyāyamāno so,}}\\
\begin{addmargin}[1em]{2em}
\setstretch{.5}
{\PaliGlossB{or with a mind full of jealousy,}}\\
\end{addmargin}
\end{absolutelynopagebreak}

\begin{absolutelynopagebreak}
\setstretch{.7}
{\PaliGlossA{sammadaññāya bhāsati;}}\\
\begin{addmargin}[1em]{2em}
\setstretch{.5}
{\PaliGlossB{they’d speak from what they rightly know.}}\\
\end{addmargin}
\end{absolutelynopagebreak}

\begin{absolutelynopagebreak}
\setstretch{.7}
{\PaliGlossA{subhāsitaṃ anumodeyya,}}\\
\begin{addmargin}[1em]{2em}
\setstretch{.5}
{\PaliGlossB{They agree with what was well spoken,}}\\
\end{addmargin}
\end{absolutelynopagebreak}

\begin{absolutelynopagebreak}
\setstretch{.7}
{\PaliGlossA{dubbhaṭṭhe nāpasādaye.}}\\
\begin{addmargin}[1em]{2em}
\setstretch{.5}
{\PaliGlossB{without criticizing what was poorly said.}}\\
\end{addmargin}
\end{absolutelynopagebreak}

\begin{absolutelynopagebreak}
\setstretch{.7}
{\PaliGlossA{upārambhaṃ na sikkheyya,}}\\
\begin{addmargin}[1em]{2em}
\setstretch{.5}
{\PaliGlossB{They’d not persist in finding faults,}}\\
\end{addmargin}
\end{absolutelynopagebreak}

\begin{absolutelynopagebreak}
\setstretch{.7}
{\PaliGlossA{khalitañca na gāhaye;}}\\
\begin{addmargin}[1em]{2em}
\setstretch{.5}
{\PaliGlossB{nor seize on trivial mistakes,}}\\
\end{addmargin}
\end{absolutelynopagebreak}

\begin{absolutelynopagebreak}
\setstretch{.7}
{\PaliGlossA{nābhihare nābhimadde,}}\\
\begin{addmargin}[1em]{2em}
\setstretch{.5}
{\PaliGlossB{neither intimidating nor crushing the other,}}\\
\end{addmargin}
\end{absolutelynopagebreak}

\begin{absolutelynopagebreak}
\setstretch{.7}
{\PaliGlossA{na vācaṃ payutaṃ bhaṇe.}}\\
\begin{addmargin}[1em]{2em}
\setstretch{.5}
{\PaliGlossB{nor would they speak with sly implications.}}\\
\end{addmargin}
\end{absolutelynopagebreak}

\begin{absolutelynopagebreak}
\setstretch{.7}
{\PaliGlossA{aññātatthaṃ pasādatthaṃ,}}\\
\begin{addmargin}[1em]{2em}
\setstretch{.5}
{\PaliGlossB{Good people consult}}\\
\end{addmargin}
\end{absolutelynopagebreak}

\begin{absolutelynopagebreak}
\setstretch{.7}
{\PaliGlossA{sataṃ ve hoti mantanā;}}\\
\begin{addmargin}[1em]{2em}
\setstretch{.5}
{\PaliGlossB{for the sake of knowledge and clarity.}}\\
\end{addmargin}
\end{absolutelynopagebreak}

\begin{absolutelynopagebreak}
\setstretch{.7}
{\PaliGlossA{evaṃ kho ariyā mantenti,}}\\
\begin{addmargin}[1em]{2em}
\setstretch{.5}
{\PaliGlossB{That’s how the noble ones consult,}}\\
\end{addmargin}
\end{absolutelynopagebreak}

\begin{absolutelynopagebreak}
\setstretch{.7}
{\PaliGlossA{esā ariyāna mantanā;}}\\
\begin{addmargin}[1em]{2em}
\setstretch{.5}
{\PaliGlossB{this is a noble consultation.}}\\
\end{addmargin}
\end{absolutelynopagebreak}

\begin{absolutelynopagebreak}
\setstretch{.7}
{\PaliGlossA{etadaññāya medhāvī,}}\\
\begin{addmargin}[1em]{2em}
\setstretch{.5}
{\PaliGlossB{Knowing this, an intelligent person}}\\
\end{addmargin}
\end{absolutelynopagebreak}

\begin{absolutelynopagebreak}
\setstretch{.7}
{\PaliGlossA{na samusseyya mantaye”ti.}}\\
\begin{addmargin}[1em]{2em}
\setstretch{.5}
{\PaliGlossB{would consult without arrogance.”}}\\
\end{addmargin}
\end{absolutelynopagebreak}

\begin{absolutelynopagebreak}
\setstretch{.7}
{\PaliGlossA{sattamaṃ.}}\\
\begin{addmargin}[1em]{2em}
\setstretch{.5}
{\PaliGlossB{    -}}\\
\end{addmargin}
\end{absolutelynopagebreak}
