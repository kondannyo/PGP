
\begin{absolutelynopagebreak}
\setstretch{.7}
{\PaliGlossA{aṅguttara nikāya 7}}\\
\begin{addmargin}[1em]{2em}
\setstretch{.5}
{\PaliGlossB{Numbered Discourses 7}}\\
\end{addmargin}
\end{absolutelynopagebreak}

\begin{absolutelynopagebreak}
\setstretch{.7}
{\PaliGlossA{4. devatāvagga}}\\
\begin{addmargin}[1em]{2em}
\setstretch{.5}
{\PaliGlossB{4. Deities}}\\
\end{addmargin}
\end{absolutelynopagebreak}

\begin{absolutelynopagebreak}
\setstretch{.7}
{\PaliGlossA{43. dutiyaniddasasutta}}\\
\begin{addmargin}[1em]{2em}
\setstretch{.5}
{\PaliGlossB{43. Graduation (2nd)}}\\
\end{addmargin}
\end{absolutelynopagebreak}

\begin{absolutelynopagebreak}
\setstretch{.7}
{\PaliGlossA{evaṃ me sutaṃ—}}\\
\begin{addmargin}[1em]{2em}
\setstretch{.5}
{\PaliGlossB{So I have heard.}}\\
\end{addmargin}
\end{absolutelynopagebreak}

\begin{absolutelynopagebreak}
\setstretch{.7}
{\PaliGlossA{ekaṃ samayaṃ bhagavā kosambiyaṃ viharati ghositārāme.}}\\
\begin{addmargin}[1em]{2em}
\setstretch{.5}
{\PaliGlossB{At one time the Buddha was staying near Kosambi, in Ghosita’s Monastery.}}\\
\end{addmargin}
\end{absolutelynopagebreak}

\begin{absolutelynopagebreak}
\setstretch{.7}
{\PaliGlossA{atha kho āyasmā ānando pubbaṇhasamayaṃ nivāsetvā pattacīvaramādāya kosambiṃ piṇḍāya pāvisi.}}\\
\begin{addmargin}[1em]{2em}
\setstretch{.5}
{\PaliGlossB{Then Venerable Ānanda robed up in the morning and, taking his bowl and robe, entered Kosambi for alms.}}\\
\end{addmargin}
\end{absolutelynopagebreak}

\begin{absolutelynopagebreak}
\setstretch{.7}
{\PaliGlossA{atha kho āyasmato ānandassa etadahosi:}}\\
\begin{addmargin}[1em]{2em}
\setstretch{.5}
{\PaliGlossB{Then it occurred to him,}}\\
\end{addmargin}
\end{absolutelynopagebreak}

\begin{absolutelynopagebreak}
\setstretch{.7}
{\PaliGlossA{“atippago kho tāva kosambiyaṃ piṇḍāya carituṃ.}}\\
\begin{addmargin}[1em]{2em}
\setstretch{.5}
{\PaliGlossB{“It’s too early to wander for alms in Kosambi.}}\\
\end{addmargin}
\end{absolutelynopagebreak}

\begin{absolutelynopagebreak}
\setstretch{.7}
{\PaliGlossA{yannūnāhaṃ yena aññatitthiyānaṃ paribbājakānaṃ ārāmo tenupasaṅkameyyan”ti.}}\\
\begin{addmargin}[1em]{2em}
\setstretch{.5}
{\PaliGlossB{Why don’t I go to the monastery of the wanderers who follow other paths?”}}\\
\end{addmargin}
\end{absolutelynopagebreak}

\begin{absolutelynopagebreak}
\setstretch{.7}
{\PaliGlossA{atha kho āyasmā ānando yena aññatitthiyānaṃ paribbājakānaṃ ārāmo tenupasaṅkami; upasaṅkamitvā tehi aññatitthiyehi paribbājakehi saddhiṃ sammodi.}}\\
\begin{addmargin}[1em]{2em}
\setstretch{.5}
{\PaliGlossB{Then he went to the monastery of the wanderers who follow other paths, and exchanged greetings with the wanderers there.}}\\
\end{addmargin}
\end{absolutelynopagebreak}

\begin{absolutelynopagebreak}
\setstretch{.7}
{\PaliGlossA{sammodanīyaṃ kathaṃ sāraṇīyaṃ vītisāretvā ekamantaṃ nisīdi.}}\\
\begin{addmargin}[1em]{2em}
\setstretch{.5}
{\PaliGlossB{When the greetings and polite conversation were over, he sat down to one side.}}\\
\end{addmargin}
\end{absolutelynopagebreak}

\begin{absolutelynopagebreak}
\setstretch{.7}
{\PaliGlossA{tena kho pana samayena tesaṃ aññatitthiyānaṃ paribbājakānaṃ sannisinnānaṃ sannipatitānaṃ ayamantarākathā udapādi:}}\\
\begin{addmargin}[1em]{2em}
\setstretch{.5}
{\PaliGlossB{Now at that time while those wanderers who follow other paths were sitting together this discussion came up among them,}}\\
\end{addmargin}
\end{absolutelynopagebreak}

\begin{absolutelynopagebreak}
\setstretch{.7}
{\PaliGlossA{“yo hi koci, āvuso, dvādasavassāni paripuṇṇaṃ parisuddhaṃ brahmacariyaṃ carati, ‘niddaso bhikkhū’ti alaṃvacanāyā”ti.}}\\
\begin{addmargin}[1em]{2em}
\setstretch{.5}
{\PaliGlossB{“Reverends, anyone who lives the full and pure spiritual life for twelve years is qualified to be called a ‘graduate mendicant’.”}}\\
\end{addmargin}
\end{absolutelynopagebreak}

\begin{absolutelynopagebreak}
\setstretch{.7}
{\PaliGlossA{atha kho āyasmā ānando tesaṃ aññatitthiyānaṃ paribbājakānaṃ bhāsitaṃ neva abhinandi nappaṭikkosi.}}\\
\begin{addmargin}[1em]{2em}
\setstretch{.5}
{\PaliGlossB{Ānanda neither approved nor dismissed that statement of the wanderers who follow other paths.}}\\
\end{addmargin}
\end{absolutelynopagebreak}

\begin{absolutelynopagebreak}
\setstretch{.7}
{\PaliGlossA{anabhinanditvā appaṭikkositvā uṭṭhāyāsanā pakkāmi:}}\\
\begin{addmargin}[1em]{2em}
\setstretch{.5}
{\PaliGlossB{He got up from his seat, thinking,}}\\
\end{addmargin}
\end{absolutelynopagebreak}

\begin{absolutelynopagebreak}
\setstretch{.7}
{\PaliGlossA{“bhagavato santike etassa bhāsitassa atthaṃ ājānissāmī”ti.}}\\
\begin{addmargin}[1em]{2em}
\setstretch{.5}
{\PaliGlossB{“I will learn the meaning of this statement from the Buddha himself.”}}\\
\end{addmargin}
\end{absolutelynopagebreak}

\begin{absolutelynopagebreak}
\setstretch{.7}
{\PaliGlossA{atha kho āyasmā ānando kosambiyaṃ piṇḍāya caritvā pacchābhattaṃ piṇḍapātapaṭikkanto yena bhagavā tenupasaṅkami; upasaṅkamitvā bhagavantaṃ abhivādetvā ekamantaṃ nisīdi. ekamantaṃ nisinno kho āyasmā ānando bhagavantaṃ etadavoca:}}\\
\begin{addmargin}[1em]{2em}
\setstretch{.5}
{\PaliGlossB{Then Ānanda wandered for alms in Kosambi. After the meal, on his return from alms-round, he went to the Buddha, bowed, sat down to one side, and told him what had happened, adding:}}\\
\end{addmargin}
\end{absolutelynopagebreak}

\begin{absolutelynopagebreak}
\setstretch{.7}
{\PaliGlossA{“idhāhaṃ, bhante, pubbaṇhasamayaṃ nivāsetvā pattacīvaramādāya kosambiṃ piṇḍāya pāvisiṃ.}}\\
\begin{addmargin}[1em]{2em}
\setstretch{.5}
{\PaliGlossB{    -}}\\
\end{addmargin}
\end{absolutelynopagebreak}

\begin{absolutelynopagebreak}
\setstretch{.7}
{\PaliGlossA{tassa mayhaṃ, bhante, etadahosi:}}\\
\begin{addmargin}[1em]{2em}
\setstretch{.5}
{\PaliGlossB{    -}}\\
\end{addmargin}
\end{absolutelynopagebreak}

\begin{absolutelynopagebreak}
\setstretch{.7}
{\PaliGlossA{‘atippago kho tāva kosambiyaṃ piṇḍāya carituṃ.}}\\
\begin{addmargin}[1em]{2em}
\setstretch{.5}
{\PaliGlossB{    -}}\\
\end{addmargin}
\end{absolutelynopagebreak}

\begin{absolutelynopagebreak}
\setstretch{.7}
{\PaliGlossA{yannūnāhaṃ yena aññatitthiyānaṃ paribbājakānaṃ ārāmo tenupasaṅkameyyan’ti … pe …}}\\
\begin{addmargin}[1em]{2em}
\setstretch{.5}
{\PaliGlossB{    -}}\\
\end{addmargin}
\end{absolutelynopagebreak}

\begin{absolutelynopagebreak}
\setstretch{.7}
{\PaliGlossA{tehi saddhiṃ sammodiṃ.}}\\
\begin{addmargin}[1em]{2em}
\setstretch{.5}
{\PaliGlossB{    -}}\\
\end{addmargin}
\end{absolutelynopagebreak}

\begin{absolutelynopagebreak}
\setstretch{.7}
{\PaliGlossA{sammodanīyaṃ kathaṃ sāraṇīyaṃ vītisāretvā ekamantaṃ nisīdiṃ.}}\\
\begin{addmargin}[1em]{2em}
\setstretch{.5}
{\PaliGlossB{    -}}\\
\end{addmargin}
\end{absolutelynopagebreak}

\begin{absolutelynopagebreak}
\setstretch{.7}
{\PaliGlossA{tena kho pana, bhante, samayena tesaṃ aññatitthiyānaṃ paribbājakānaṃ sannisinnānaṃ sannipatitānaṃ ayamantarākathā udapādi:}}\\
\begin{addmargin}[1em]{2em}
\setstretch{.5}
{\PaliGlossB{    -}}\\
\end{addmargin}
\end{absolutelynopagebreak}

\begin{absolutelynopagebreak}
\setstretch{.7}
{\PaliGlossA{‘yo hi koci, āvuso, dvādasavassāni paripuṇṇaṃ parisuddhaṃ brahmacariyaṃ carati, niddaso bhikkhūti alaṃvacanāyā’ti.}}\\
\begin{addmargin}[1em]{2em}
\setstretch{.5}
{\PaliGlossB{    -}}\\
\end{addmargin}
\end{absolutelynopagebreak}

\begin{absolutelynopagebreak}
\setstretch{.7}
{\PaliGlossA{atha khvāhaṃ, bhante, tesaṃ aññatitthiyānaṃ paribbājakānaṃ bhāsitaṃ neva abhinandiṃ nappaṭikkosiṃ.}}\\
\begin{addmargin}[1em]{2em}
\setstretch{.5}
{\PaliGlossB{    -}}\\
\end{addmargin}
\end{absolutelynopagebreak}

\begin{absolutelynopagebreak}
\setstretch{.7}
{\PaliGlossA{anabhinanditvā, appaṭikkositvā uṭṭhāyāsanā pakkamiṃ:}}\\
\begin{addmargin}[1em]{2em}
\setstretch{.5}
{\PaliGlossB{    -}}\\
\end{addmargin}
\end{absolutelynopagebreak}

\begin{absolutelynopagebreak}
\setstretch{.7}
{\PaliGlossA{‘bhagavato santike etassa bhāsitassa atthaṃ ājānissāmī’ti.}}\\
\begin{addmargin}[1em]{2em}
\setstretch{.5}
{\PaliGlossB{    -}}\\
\end{addmargin}
\end{absolutelynopagebreak}

\begin{absolutelynopagebreak}
\setstretch{.7}
{\PaliGlossA{sakkā nu kho, bhante, imasmiṃ dhammavinaye kevalaṃ vassagaṇanamattena niddaso bhikkhu paññāpetun”ti?}}\\
\begin{addmargin}[1em]{2em}
\setstretch{.5}
{\PaliGlossB{“Sir, in this teaching and training can we describe a mendicant as a ‘graduate’ solely because they have completed a certain number of years?”}}\\
\end{addmargin}
\end{absolutelynopagebreak}

\begin{absolutelynopagebreak}
\setstretch{.7}
{\PaliGlossA{“na kho, ānanda, sakkā imasmiṃ dhammavinaye kevalaṃ vassagaṇanamattena niddaso bhikkhu paññāpetuṃ.}}\\
\begin{addmargin}[1em]{2em}
\setstretch{.5}
{\PaliGlossB{“No, Ānanda, we cannot.}}\\
\end{addmargin}
\end{absolutelynopagebreak}

\begin{absolutelynopagebreak}
\setstretch{.7}
{\PaliGlossA{satta kho imāni, ānanda, niddasavatthūni mayā sayaṃ abhiññā sacchikatvā paveditāni.}}\\
\begin{addmargin}[1em]{2em}
\setstretch{.5}
{\PaliGlossB{These are the seven qualifications for graduation that I make known after realizing them with my own insight.}}\\
\end{addmargin}
\end{absolutelynopagebreak}

\begin{absolutelynopagebreak}
\setstretch{.7}
{\PaliGlossA{katamāni satta?}}\\
\begin{addmargin}[1em]{2em}
\setstretch{.5}
{\PaliGlossB{What seven?}}\\
\end{addmargin}
\end{absolutelynopagebreak}

\begin{absolutelynopagebreak}
\setstretch{.7}
{\PaliGlossA{idhānanda, bhikkhu, saddho hoti, hirīmā hoti, ottappī hoti, bahussuto hoti, āraddhavīriyo hoti, satimā hoti, paññavā hoti.}}\\
\begin{addmargin}[1em]{2em}
\setstretch{.5}
{\PaliGlossB{It’s when someone is faithful, conscientious, prudent, learned, energetic, mindful, and wise.}}\\
\end{addmargin}
\end{absolutelynopagebreak}

\begin{absolutelynopagebreak}
\setstretch{.7}
{\PaliGlossA{imāni kho, ānanda, satta niddasavatthūni mayā sayaṃ abhiññā sacchikatvā paveditāni.}}\\
\begin{addmargin}[1em]{2em}
\setstretch{.5}
{\PaliGlossB{These are the seven qualifications for graduation that I make known after realizing them with my own insight.}}\\
\end{addmargin}
\end{absolutelynopagebreak}

\begin{absolutelynopagebreak}
\setstretch{.7}
{\PaliGlossA{imehi kho, ānanda, sattahi niddasavatthūhi samannāgato bhikkhu dvādasa cepi vassāni paripuṇṇaṃ parisuddhaṃ brahmacariyaṃ carati, ‘niddaso bhikkhū’ti alaṃvacanāya;}}\\
\begin{addmargin}[1em]{2em}
\setstretch{.5}
{\PaliGlossB{A mendicant who has these seven qualifications for graduation is qualified to be called a ‘graduate mendicant’. This is so whether they have lived the full and pure spiritual life for twelve years,}}\\
\end{addmargin}
\end{absolutelynopagebreak}

\begin{absolutelynopagebreak}
\setstretch{.7}
{\PaliGlossA{catubbīsati cepi vassāni paripuṇṇaṃ parisuddhaṃ brahmacariyaṃ carati, ‘niddaso bhikkhū’ti alaṃvacanāya;}}\\
\begin{addmargin}[1em]{2em}
\setstretch{.5}
{\PaliGlossB{twenty-four years,}}\\
\end{addmargin}
\end{absolutelynopagebreak}

\begin{absolutelynopagebreak}
\setstretch{.7}
{\PaliGlossA{chattiṃsati cepi vassāni paripuṇṇaṃ parisuddhaṃ brahmacariyaṃ carati, ‘niddaso bhikkhū’ti alaṃvacanāya, aṭṭhacattārīsañcepi vassāni paripuṇṇaṃ parisuddhaṃ brahmacariyaṃ carati, ‘niddaso bhikkhū’ti alaṃvacanāyā”ti.}}\\
\begin{addmargin}[1em]{2em}
\setstretch{.5}
{\PaliGlossB{thirty-six years, or forty-eight years.”}}\\
\end{addmargin}
\end{absolutelynopagebreak}

\begin{absolutelynopagebreak}
\setstretch{.7}
{\PaliGlossA{dvādasamaṃ.}}\\
\begin{addmargin}[1em]{2em}
\setstretch{.5}
{\PaliGlossB{    -}}\\
\end{addmargin}
\end{absolutelynopagebreak}

\begin{absolutelynopagebreak}
\setstretch{.7}
{\PaliGlossA{devatāvaggo catuttho.}}\\
\begin{addmargin}[1em]{2em}
\setstretch{.5}
{\PaliGlossB{    -}}\\
\end{addmargin}
\end{absolutelynopagebreak}

\begin{absolutelynopagebreak}
\setstretch{.7}
{\PaliGlossA{appamādo hirī ceva,}}\\
\begin{addmargin}[1em]{2em}
\setstretch{.5}
{\PaliGlossB{    -}}\\
\end{addmargin}
\end{absolutelynopagebreak}

\begin{absolutelynopagebreak}
\setstretch{.7}
{\PaliGlossA{dve suvacā duve mittā;}}\\
\begin{addmargin}[1em]{2em}
\setstretch{.5}
{\PaliGlossB{    -}}\\
\end{addmargin}
\end{absolutelynopagebreak}

\begin{absolutelynopagebreak}
\setstretch{.7}
{\PaliGlossA{dve paṭisambhidā dve vasā,}}\\
\begin{addmargin}[1em]{2em}
\setstretch{.5}
{\PaliGlossB{    -}}\\
\end{addmargin}
\end{absolutelynopagebreak}

\begin{absolutelynopagebreak}
\setstretch{.7}
{\PaliGlossA{duve niddasavatthunāti.}}\\
\begin{addmargin}[1em]{2em}
\setstretch{.5}
{\PaliGlossB{    -}}\\
\end{addmargin}
\end{absolutelynopagebreak}
