
\begin{absolutelynopagebreak}
\setstretch{.7}
{\PaliGlossA{aṅguttara nikāya 3}}\\
\begin{addmargin}[1em]{2em}
\setstretch{.5}
{\PaliGlossB{Numbered Discourses 3}}\\
\end{addmargin}
\end{absolutelynopagebreak}

\begin{absolutelynopagebreak}
\setstretch{.7}
{\PaliGlossA{4. devadūtavagga}}\\
\begin{addmargin}[1em]{2em}
\setstretch{.5}
{\PaliGlossB{4. Messengers of the Gods}}\\
\end{addmargin}
\end{absolutelynopagebreak}

\begin{absolutelynopagebreak}
\setstretch{.7}
{\PaliGlossA{32. ānandasutta}}\\
\begin{addmargin}[1em]{2em}
\setstretch{.5}
{\PaliGlossB{32. With Ānanda}}\\
\end{addmargin}
\end{absolutelynopagebreak}

\begin{absolutelynopagebreak}
\setstretch{.7}
{\PaliGlossA{atha kho āyasmā ānando yena bhagavā tenupasaṅkami; upasaṅkamitvā bhagavantaṃ abhivādetvā ekamantaṃ nisīdi. ekamantaṃ nisinno kho āyasmā ānando bhagavantaṃ etadavoca:}}\\
\begin{addmargin}[1em]{2em}
\setstretch{.5}
{\PaliGlossB{Then Venerable Ānanda went up to the Buddha, bowed, sat down to one side, and said to the Buddha:}}\\
\end{addmargin}
\end{absolutelynopagebreak}

\begin{absolutelynopagebreak}
\setstretch{.7}
{\PaliGlossA{“siyā nu kho, bhante, bhikkhuno tathārūpo samādhipaṭilābho yathā imasmiñca saviññāṇake kāye ahaṅkāramamaṅkāramānānusayā nāssu, bahiddhā ca sabbanimittesu ahaṅkāramamaṅkāramānānusayā nāssu;}}\\
\begin{addmargin}[1em]{2em}
\setstretch{.5}
{\PaliGlossB{“Could it be, sir, that a mendicant might gain a state of immersion such that there’s no ego, possessiveness, or underlying tendency to conceit for this conscious body; and no ego, possessiveness, or underlying tendency to conceit for all external stimuli;}}\\
\end{addmargin}
\end{absolutelynopagebreak}

\begin{absolutelynopagebreak}
\setstretch{.7}
{\PaliGlossA{yañca cetovimuttiṃ paññāvimuttiṃ upasampajja viharato ahaṅkāramamaṅkāramānānusayā na honti tañca cetovimuttiṃ paññāvimuttiṃ upasampajja vihareyyā”ti?}}\\
\begin{addmargin}[1em]{2em}
\setstretch{.5}
{\PaliGlossB{and that they’d live having attained the freedom of heart and freedom by wisdom where ego, possessiveness, and underlying tendency to conceit are no more?”}}\\
\end{addmargin}
\end{absolutelynopagebreak}

\begin{absolutelynopagebreak}
\setstretch{.7}
{\PaliGlossA{“siyā, ānanda, bhikkhuno tathārūpo samādhipaṭilābho yathā imasmiñca saviññāṇake kāye ahaṅkāramamaṅkāramānānusayā nāssu, bahiddhā ca sabbanimittesu ahaṅkāramamaṅkāramānānusayā nāssu;}}\\
\begin{addmargin}[1em]{2em}
\setstretch{.5}
{\PaliGlossB{“It could be, Ānanda, that a mendicant gains a state of immersion such that they have no ego, possessiveness, or underlying tendency to conceit for this conscious body; and no ego, possessiveness, or underlying tendency to conceit for all external stimuli;}}\\
\end{addmargin}
\end{absolutelynopagebreak}

\begin{absolutelynopagebreak}
\setstretch{.7}
{\PaliGlossA{yañca cetovimuttiṃ paññāvimuttiṃ upasampajja viharato ahaṅkāramamaṅkāramānānusayā na honti tañca cetovimuttiṃ paññāvimuttiṃ upasampajja vihareyyā”ti.}}\\
\begin{addmargin}[1em]{2em}
\setstretch{.5}
{\PaliGlossB{and that they’d live having attained the freedom of heart and freedom by wisdom where ego, possessiveness, and underlying tendency to conceit are no more.”}}\\
\end{addmargin}
\end{absolutelynopagebreak}

\begin{absolutelynopagebreak}
\setstretch{.7}
{\PaliGlossA{“yathā kathaṃ pana, bhante, siyā bhikkhuno tathārūpo samādhipaṭilābho yathā imasmiñca saviññāṇake kāye ahaṅkāramamaṅkāramānānusayā nāssu, bahiddhā ca sabbanimittesu ahaṅkāramamaṅkāramānānusayā nāssu;}}\\
\begin{addmargin}[1em]{2em}
\setstretch{.5}
{\PaliGlossB{“But how could this be, sir?”}}\\
\end{addmargin}
\end{absolutelynopagebreak}

\begin{absolutelynopagebreak}
\setstretch{.7}
{\PaliGlossA{yañca cetovimuttiṃ paññāvimuttiṃ upasampajja viharato ahaṅkāramamaṅkāramānānusayā na honti tañca cetovimuttiṃ paññāvimuttiṃ upasampajja vihareyyā”ti?}}\\
\begin{addmargin}[1em]{2em}
\setstretch{.5}
{\PaliGlossB{    -}}\\
\end{addmargin}
\end{absolutelynopagebreak}

\begin{absolutelynopagebreak}
\setstretch{.7}
{\PaliGlossA{“idhānanda, bhikkhuno evaṃ hoti:}}\\
\begin{addmargin}[1em]{2em}
\setstretch{.5}
{\PaliGlossB{“Ānanda, it’s when a mendicant thinks:}}\\
\end{addmargin}
\end{absolutelynopagebreak}

\begin{absolutelynopagebreak}
\setstretch{.7}
{\PaliGlossA{‘etaṃ santaṃ etaṃ paṇītaṃ yadidaṃ sabbasaṅkhārasamatho sabbūpadhipaṭinissaggo taṇhākkhayo virāgo nirodho nibbānan’ti.}}\\
\begin{addmargin}[1em]{2em}
\setstretch{.5}
{\PaliGlossB{‘This is peaceful; this is sublime—that is, the stilling of all activities, the letting go of all attachments, the ending of craving, fading away, cessation, extinguishment.’}}\\
\end{addmargin}
\end{absolutelynopagebreak}

\begin{absolutelynopagebreak}
\setstretch{.7}
{\PaliGlossA{evaṃ kho, ānanda, siyā bhikkhuno tathārūpo samādhipaṭilābho yathā imasmiñca saviññāṇake kāye ahaṅkāramamaṅkāramānānusayā nāssu, bahiddhā ca sabbanimittesu ahaṅkāramamaṅkāramānānusayā nāssu;}}\\
\begin{addmargin}[1em]{2em}
\setstretch{.5}
{\PaliGlossB{That’s how, Ānanda, a mendicant might gain a state of immersion such that there’s no ego, possessiveness, or underlying tendency to conceit for this conscious body; and no ego, possessiveness, or underlying tendency to conceit for all external stimuli;}}\\
\end{addmargin}
\end{absolutelynopagebreak}

\begin{absolutelynopagebreak}
\setstretch{.7}
{\PaliGlossA{yañca cetovimuttiṃ paññāvimuttiṃ upasampajja viharato ahaṅkāramamaṅkāramānānusayā na honti tañca cetovimuttiṃ paññāvimuttiṃ upasampajja vihareyyāti.}}\\
\begin{addmargin}[1em]{2em}
\setstretch{.5}
{\PaliGlossB{and that they’d live having achieved the freedom of heart and freedom by wisdom where ego, possessiveness, and underlying tendency to conceit are no more.}}\\
\end{addmargin}
\end{absolutelynopagebreak}

\begin{absolutelynopagebreak}
\setstretch{.7}
{\PaliGlossA{idañca pana metaṃ, ānanda, sandhāya bhāsitaṃ pārāyane puṇṇakapañhe:}}\\
\begin{addmargin}[1em]{2em}
\setstretch{.5}
{\PaliGlossB{And Ānanda, this is what I was referring to in ‘The Way to the Beyond’, in ‘The Questions of Puṇṇaka’ when I said:}}\\
\end{addmargin}
\end{absolutelynopagebreak}

\begin{absolutelynopagebreak}
\setstretch{.7}
{\PaliGlossA{‘saṅkhāya lokasmiṃ paroparāni,}}\\
\begin{addmargin}[1em]{2em}
\setstretch{.5}
{\PaliGlossB{‘Having surveyed the world high and low,}}\\
\end{addmargin}
\end{absolutelynopagebreak}

\begin{absolutelynopagebreak}
\setstretch{.7}
{\PaliGlossA{yassiñjitaṃ natthi kuhiñci loke;}}\\
\begin{addmargin}[1em]{2em}
\setstretch{.5}
{\PaliGlossB{they’re not shaken by anything in the world.}}\\
\end{addmargin}
\end{absolutelynopagebreak}

\begin{absolutelynopagebreak}
\setstretch{.7}
{\PaliGlossA{santo vidhūmo anīgho nirāso,}}\\
\begin{addmargin}[1em]{2em}
\setstretch{.5}
{\PaliGlossB{Peaceful, unclouded, untroubled, with no need for hope—}}\\
\end{addmargin}
\end{absolutelynopagebreak}

\begin{absolutelynopagebreak}
\setstretch{.7}
{\PaliGlossA{atāri so jātijaranti brūmī’”ti.}}\\
\begin{addmargin}[1em]{2em}
\setstretch{.5}
{\PaliGlossB{they’ve crossed over birth and old age, I declare.’”}}\\
\end{addmargin}
\end{absolutelynopagebreak}

\begin{absolutelynopagebreak}
\setstretch{.7}
{\PaliGlossA{dutiyaṃ.}}\\
\begin{addmargin}[1em]{2em}
\setstretch{.5}
{\PaliGlossB{    -}}\\
\end{addmargin}
\end{absolutelynopagebreak}
