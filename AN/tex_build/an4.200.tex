
\begin{absolutelynopagebreak}
\setstretch{.7}
{\PaliGlossA{aṅguttara nikāya 4}}\\
\begin{addmargin}[1em]{2em}
\setstretch{.5}
{\PaliGlossB{Numbered Discourses 4}}\\
\end{addmargin}
\end{absolutelynopagebreak}

\begin{absolutelynopagebreak}
\setstretch{.7}
{\PaliGlossA{20. mahāvagga}}\\
\begin{addmargin}[1em]{2em}
\setstretch{.5}
{\PaliGlossB{20. The Great Chapter}}\\
\end{addmargin}
\end{absolutelynopagebreak}

\begin{absolutelynopagebreak}
\setstretch{.7}
{\PaliGlossA{200. pemasutta}}\\
\begin{addmargin}[1em]{2em}
\setstretch{.5}
{\PaliGlossB{200. Love and Hate}}\\
\end{addmargin}
\end{absolutelynopagebreak}

\begin{absolutelynopagebreak}
\setstretch{.7}
{\PaliGlossA{“cattārimāni, bhikkhave, pemāni jāyanti.}}\\
\begin{addmargin}[1em]{2em}
\setstretch{.5}
{\PaliGlossB{“Mendicants, these four things are born of love and hate.}}\\
\end{addmargin}
\end{absolutelynopagebreak}

\begin{absolutelynopagebreak}
\setstretch{.7}
{\PaliGlossA{katamāni cattāri?}}\\
\begin{addmargin}[1em]{2em}
\setstretch{.5}
{\PaliGlossB{What four?}}\\
\end{addmargin}
\end{absolutelynopagebreak}

\begin{absolutelynopagebreak}
\setstretch{.7}
{\PaliGlossA{pemā pemaṃ jāyati, pemā doso jāyati, dosā pemaṃ jāyati, dosā doso jāyati.}}\\
\begin{addmargin}[1em]{2em}
\setstretch{.5}
{\PaliGlossB{~Love is born of love, ~hate is born of love, ~love is born of hate, and ~hate is born of hate.}}\\
\end{addmargin}
\end{absolutelynopagebreak}

\begin{absolutelynopagebreak}
\setstretch{.7}
{\PaliGlossA{kathañca, bhikkhave, pemā pemaṃ jāyati?}}\\
\begin{addmargin}[1em]{2em}
\setstretch{.5}
{\PaliGlossB{And how is love born of love?}}\\
\end{addmargin}
\end{absolutelynopagebreak}

\begin{absolutelynopagebreak}
\setstretch{.7}
{\PaliGlossA{idha, bhikkhave, puggalo puggalassa iṭṭho hoti kanto manāpo.}}\\
\begin{addmargin}[1em]{2em}
\setstretch{.5}
{\PaliGlossB{It’s when someone likes, loves, and cares for a person.}}\\
\end{addmargin}
\end{absolutelynopagebreak}

\begin{absolutelynopagebreak}
\setstretch{.7}
{\PaliGlossA{taṃ pare iṭṭhena kantena manāpena samudācaranti.}}\\
\begin{addmargin}[1em]{2em}
\setstretch{.5}
{\PaliGlossB{Others treat that person with liking, love, and care.}}\\
\end{addmargin}
\end{absolutelynopagebreak}

\begin{absolutelynopagebreak}
\setstretch{.7}
{\PaliGlossA{tassa evaṃ hoti:}}\\
\begin{addmargin}[1em]{2em}
\setstretch{.5}
{\PaliGlossB{They think:}}\\
\end{addmargin}
\end{absolutelynopagebreak}

\begin{absolutelynopagebreak}
\setstretch{.7}
{\PaliGlossA{‘yo kho myāyaṃ puggalo iṭṭho kanto manāpo, taṃ pare iṭṭhena kantena manāpena samudācarantī’ti.}}\\
\begin{addmargin}[1em]{2em}
\setstretch{.5}
{\PaliGlossB{‘These others like the person I like.’}}\\
\end{addmargin}
\end{absolutelynopagebreak}

\begin{absolutelynopagebreak}
\setstretch{.7}
{\PaliGlossA{so tesu pemaṃ janeti.}}\\
\begin{addmargin}[1em]{2em}
\setstretch{.5}
{\PaliGlossB{And so love for them springs up.}}\\
\end{addmargin}
\end{absolutelynopagebreak}

\begin{absolutelynopagebreak}
\setstretch{.7}
{\PaliGlossA{evaṃ kho, bhikkhave, pemā pemaṃ jāyati.}}\\
\begin{addmargin}[1em]{2em}
\setstretch{.5}
{\PaliGlossB{That’s how love is born of love.}}\\
\end{addmargin}
\end{absolutelynopagebreak}

\begin{absolutelynopagebreak}
\setstretch{.7}
{\PaliGlossA{kathañca, bhikkhave, pemā doso jāyati?}}\\
\begin{addmargin}[1em]{2em}
\setstretch{.5}
{\PaliGlossB{And how is hate born of love?}}\\
\end{addmargin}
\end{absolutelynopagebreak}

\begin{absolutelynopagebreak}
\setstretch{.7}
{\PaliGlossA{idha, bhikkhave, puggalo puggalassa iṭṭho hoti kanto manāpo.}}\\
\begin{addmargin}[1em]{2em}
\setstretch{.5}
{\PaliGlossB{It’s when someone likes, loves, and cares for a person.}}\\
\end{addmargin}
\end{absolutelynopagebreak}

\begin{absolutelynopagebreak}
\setstretch{.7}
{\PaliGlossA{taṃ pare aniṭṭhena akantena amanāpena samudācaranti.}}\\
\begin{addmargin}[1em]{2em}
\setstretch{.5}
{\PaliGlossB{Others treat that person with disliking, loathing, and detestation.}}\\
\end{addmargin}
\end{absolutelynopagebreak}

\begin{absolutelynopagebreak}
\setstretch{.7}
{\PaliGlossA{tassa evaṃ hoti:}}\\
\begin{addmargin}[1em]{2em}
\setstretch{.5}
{\PaliGlossB{They think:}}\\
\end{addmargin}
\end{absolutelynopagebreak}

\begin{absolutelynopagebreak}
\setstretch{.7}
{\PaliGlossA{‘yo kho myāyaṃ puggalo iṭṭho kanto manāpo, taṃ pare aniṭṭhena akantena amanāpena samudācarantī’ti.}}\\
\begin{addmargin}[1em]{2em}
\setstretch{.5}
{\PaliGlossB{‘These others dislike the person I like.’}}\\
\end{addmargin}
\end{absolutelynopagebreak}

\begin{absolutelynopagebreak}
\setstretch{.7}
{\PaliGlossA{so tesu dosaṃ janeti.}}\\
\begin{addmargin}[1em]{2em}
\setstretch{.5}
{\PaliGlossB{And so hate for them springs up.}}\\
\end{addmargin}
\end{absolutelynopagebreak}

\begin{absolutelynopagebreak}
\setstretch{.7}
{\PaliGlossA{evaṃ kho, bhikkhave, pemā doso jāyati.}}\\
\begin{addmargin}[1em]{2em}
\setstretch{.5}
{\PaliGlossB{That’s how hate is born of love.}}\\
\end{addmargin}
\end{absolutelynopagebreak}

\begin{absolutelynopagebreak}
\setstretch{.7}
{\PaliGlossA{kathañca, bhikkhave, dosā pemaṃ jāyati?}}\\
\begin{addmargin}[1em]{2em}
\setstretch{.5}
{\PaliGlossB{And how is love born of hate?}}\\
\end{addmargin}
\end{absolutelynopagebreak}

\begin{absolutelynopagebreak}
\setstretch{.7}
{\PaliGlossA{idha, bhikkhave, puggalo puggalassa aniṭṭho hoti akanto amanāpo.}}\\
\begin{addmargin}[1em]{2em}
\setstretch{.5}
{\PaliGlossB{It’s when someone dislikes, loathes, and detests a person.}}\\
\end{addmargin}
\end{absolutelynopagebreak}

\begin{absolutelynopagebreak}
\setstretch{.7}
{\PaliGlossA{taṃ pare aniṭṭhena akantena amanāpena samudācaranti.}}\\
\begin{addmargin}[1em]{2em}
\setstretch{.5}
{\PaliGlossB{Others treat that person with disliking, loathing, and detestation.}}\\
\end{addmargin}
\end{absolutelynopagebreak}

\begin{absolutelynopagebreak}
\setstretch{.7}
{\PaliGlossA{tassa evaṃ hoti:}}\\
\begin{addmargin}[1em]{2em}
\setstretch{.5}
{\PaliGlossB{They think:}}\\
\end{addmargin}
\end{absolutelynopagebreak}

\begin{absolutelynopagebreak}
\setstretch{.7}
{\PaliGlossA{‘yo kho myāyaṃ puggalo aniṭṭho akanto amanāpo, taṃ pare aniṭṭhena akantena amanāpena samudācarantī’ti.}}\\
\begin{addmargin}[1em]{2em}
\setstretch{.5}
{\PaliGlossB{‘These others dislike the person I dislike.’}}\\
\end{addmargin}
\end{absolutelynopagebreak}

\begin{absolutelynopagebreak}
\setstretch{.7}
{\PaliGlossA{so tesu pemaṃ janeti.}}\\
\begin{addmargin}[1em]{2em}
\setstretch{.5}
{\PaliGlossB{And so love for them springs up.}}\\
\end{addmargin}
\end{absolutelynopagebreak}

\begin{absolutelynopagebreak}
\setstretch{.7}
{\PaliGlossA{evaṃ kho, bhikkhave, dosā pemaṃ jāyati.}}\\
\begin{addmargin}[1em]{2em}
\setstretch{.5}
{\PaliGlossB{That’s how love is born of hate.}}\\
\end{addmargin}
\end{absolutelynopagebreak}

\begin{absolutelynopagebreak}
\setstretch{.7}
{\PaliGlossA{kathañca, bhikkhave, dosā doso jāyati?}}\\
\begin{addmargin}[1em]{2em}
\setstretch{.5}
{\PaliGlossB{And how is hate born of hate?}}\\
\end{addmargin}
\end{absolutelynopagebreak}

\begin{absolutelynopagebreak}
\setstretch{.7}
{\PaliGlossA{idha, bhikkhave, puggalo puggalassa aniṭṭho hoti akanto amanāpo.}}\\
\begin{addmargin}[1em]{2em}
\setstretch{.5}
{\PaliGlossB{It’s when someone dislikes, loathes, and detests a person.}}\\
\end{addmargin}
\end{absolutelynopagebreak}

\begin{absolutelynopagebreak}
\setstretch{.7}
{\PaliGlossA{taṃ pare iṭṭhena kantena manāpena samudācaranti.}}\\
\begin{addmargin}[1em]{2em}
\setstretch{.5}
{\PaliGlossB{Others treat that person with liking, love, and care.}}\\
\end{addmargin}
\end{absolutelynopagebreak}

\begin{absolutelynopagebreak}
\setstretch{.7}
{\PaliGlossA{tassa evaṃ hoti:}}\\
\begin{addmargin}[1em]{2em}
\setstretch{.5}
{\PaliGlossB{They think:}}\\
\end{addmargin}
\end{absolutelynopagebreak}

\begin{absolutelynopagebreak}
\setstretch{.7}
{\PaliGlossA{‘yo kho myāyaṃ puggalo aniṭṭho akanto amanāpo, taṃ pare iṭṭhena kantena manāpena samudācarantī’ti.}}\\
\begin{addmargin}[1em]{2em}
\setstretch{.5}
{\PaliGlossB{‘These others like the person I dislike.’}}\\
\end{addmargin}
\end{absolutelynopagebreak}

\begin{absolutelynopagebreak}
\setstretch{.7}
{\PaliGlossA{so tesu dosaṃ janeti.}}\\
\begin{addmargin}[1em]{2em}
\setstretch{.5}
{\PaliGlossB{And so hate for them springs up.}}\\
\end{addmargin}
\end{absolutelynopagebreak}

\begin{absolutelynopagebreak}
\setstretch{.7}
{\PaliGlossA{evaṃ kho, bhikkhave, dosā doso jāyati.}}\\
\begin{addmargin}[1em]{2em}
\setstretch{.5}
{\PaliGlossB{That’s how hate is born of hate.}}\\
\end{addmargin}
\end{absolutelynopagebreak}

\begin{absolutelynopagebreak}
\setstretch{.7}
{\PaliGlossA{imāni kho, bhikkhave, cattāri pemāni jāyanti.}}\\
\begin{addmargin}[1em]{2em}
\setstretch{.5}
{\PaliGlossB{These are the four things that are born of love and hate.}}\\
\end{addmargin}
\end{absolutelynopagebreak}

\begin{absolutelynopagebreak}
\setstretch{.7}
{\PaliGlossA{yasmiṃ, bhikkhave, samaye bhikkhu vivicceva kāmehi … pe … paṭhamaṃ jhānaṃ upasampajja viharati,}}\\
\begin{addmargin}[1em]{2em}
\setstretch{.5}
{\PaliGlossB{A time comes when a mendicant … enters and remains in the first absorption.}}\\
\end{addmargin}
\end{absolutelynopagebreak}

\begin{absolutelynopagebreak}
\setstretch{.7}
{\PaliGlossA{yampissa pemā pemaṃ jāyati tampissa tasmiṃ samaye na hoti, yopissa pemā doso jāyati sopissa tasmiṃ samaye na hoti, yampissa dosā pemaṃ jāyati tampissa tasmiṃ samaye na hoti, yopissa dosā doso jāyati sopissa tasmiṃ samaye na hoti.}}\\
\begin{addmargin}[1em]{2em}
\setstretch{.5}
{\PaliGlossB{At that time they have no love born of love, hate born of love, love born of hate, or hate born of hate.}}\\
\end{addmargin}
\end{absolutelynopagebreak}

\begin{absolutelynopagebreak}
\setstretch{.7}
{\PaliGlossA{yasmiṃ, bhikkhave, samaye bhikkhu vitakkavicārānaṃ vūpasamā … pe … dutiyaṃ jhānaṃ … pe … tatiyaṃ jhānaṃ … pe … catutthaṃ jhānaṃ upasampajja viharati,}}\\
\begin{addmargin}[1em]{2em}
\setstretch{.5}
{\PaliGlossB{A time comes when a mendicant … enters and remains in the second absorption … third absorption … fourth absorption.}}\\
\end{addmargin}
\end{absolutelynopagebreak}

\begin{absolutelynopagebreak}
\setstretch{.7}
{\PaliGlossA{yampissa pemā pemaṃ jāyati tampissa tasmiṃ samaye na hoti, yopissa pemā doso jāyati sopissa tasmiṃ samaye na hoti, yampissa dosā pemaṃ jāyati tampissa tasmiṃ samaye na hoti, yopissa dosā doso jāyati sopissa tasmiṃ samaye na hoti.}}\\
\begin{addmargin}[1em]{2em}
\setstretch{.5}
{\PaliGlossB{At that time they have no love born of love, hate born of love, love born of hate, or hate born of hate.}}\\
\end{addmargin}
\end{absolutelynopagebreak}

\begin{absolutelynopagebreak}
\setstretch{.7}
{\PaliGlossA{yasmiṃ, bhikkhave, samaye bhikkhu āsavānaṃ khayā anāsavaṃ cetovimuttiṃ paññāvimuttiṃ diṭṭheva dhamme sayaṃ abhiññā sacchikatvā upasampajja viharati, yampissa pemā pemaṃ jāyati tampissa pahīnaṃ hoti ucchinnamūlaṃ tālāvatthukataṃ anabhāvaṃkataṃ āyatiṃ anuppādadhammaṃ, yopissa pemā doso jāyati sopissa pahīno hoti ucchinnamūlo tālāvatthukato anabhāvaṃkato āyatiṃ anuppādadhammo,}}\\
\begin{addmargin}[1em]{2em}
\setstretch{.5}
{\PaliGlossB{A time comes when a mendicant realizes the undefiled freedom of heart and freedom by wisdom in this very life. And they live having realized it with their own insight due to the ending of defilements.}}\\
\end{addmargin}
\end{absolutelynopagebreak}

\begin{absolutelynopagebreak}
\setstretch{.7}
{\PaliGlossA{yampissa dosā pemaṃ jāyati tampissa pahīnaṃ hoti ucchinnamūlaṃ tālāvatthukataṃ anabhāvaṃkataṃ āyatiṃ anuppādadhammaṃ, yopissa dosā doso jāyati sopissa pahīno hoti ucchinnamūlo tālāvatthukato anabhāvaṃkato āyatiṃ anuppādadhammo.}}\\
\begin{addmargin}[1em]{2em}
\setstretch{.5}
{\PaliGlossB{At that time any love born of love, hate born of love, love born of hate, or hate born of hate is given up, cut off at the root, made like a palm stump, obliterated, and unable to arise in the future.}}\\
\end{addmargin}
\end{absolutelynopagebreak}

\begin{absolutelynopagebreak}
\setstretch{.7}
{\PaliGlossA{ayaṃ vuccati, bhikkhave, bhikkhu neva usseneti na paṭiseneti na dhūpāyati na pajjalati na sampajjhāyati.}}\\
\begin{addmargin}[1em]{2em}
\setstretch{.5}
{\PaliGlossB{This is called a mendicant who doesn’t draw close or push back or fume or ignite or burn up.}}\\
\end{addmargin}
\end{absolutelynopagebreak}

\begin{absolutelynopagebreak}
\setstretch{.7}
{\PaliGlossA{kathañca, bhikkhave, bhikkhu usseneti?}}\\
\begin{addmargin}[1em]{2em}
\setstretch{.5}
{\PaliGlossB{And how does a mendicant draw close?}}\\
\end{addmargin}
\end{absolutelynopagebreak}

\begin{absolutelynopagebreak}
\setstretch{.7}
{\PaliGlossA{idha, bhikkhave, bhikkhu rūpaṃ attato samanupassati, rūpavantaṃ vā attānaṃ, attani vā rūpaṃ, rūpasmiṃ vā attānaṃ;}}\\
\begin{addmargin}[1em]{2em}
\setstretch{.5}
{\PaliGlossB{It’s when a mendicant regards form as self, self as having form, form in self, or self in form.}}\\
\end{addmargin}
\end{absolutelynopagebreak}

\begin{absolutelynopagebreak}
\setstretch{.7}
{\PaliGlossA{vedanaṃ attato samanupassati, vedanāvantaṃ vā attānaṃ, attani vā vedanaṃ, vedanāya vā attānaṃ;}}\\
\begin{addmargin}[1em]{2em}
\setstretch{.5}
{\PaliGlossB{They regard feeling as self, self as having feeling, feeling in self, or self in feeling.}}\\
\end{addmargin}
\end{absolutelynopagebreak}

\begin{absolutelynopagebreak}
\setstretch{.7}
{\PaliGlossA{saññaṃ attato samanupassati, saññāvantaṃ vā attānaṃ, attani vā saññaṃ, saññāya vā attānaṃ;}}\\
\begin{addmargin}[1em]{2em}
\setstretch{.5}
{\PaliGlossB{They regard perception as self, self as having perception, perception in self, or self in perception.}}\\
\end{addmargin}
\end{absolutelynopagebreak}

\begin{absolutelynopagebreak}
\setstretch{.7}
{\PaliGlossA{saṅkhāre attato samanupassati, saṅkhāravantaṃ vā attānaṃ, attani vā saṅkhāre, saṅkhāresu vā attānaṃ;}}\\
\begin{addmargin}[1em]{2em}
\setstretch{.5}
{\PaliGlossB{They regard choices as self, self as having choices, choices in self, or self in choices.}}\\
\end{addmargin}
\end{absolutelynopagebreak}

\begin{absolutelynopagebreak}
\setstretch{.7}
{\PaliGlossA{viññāṇaṃ attato samanupassati, viññāṇavantaṃ vā attānaṃ, attani vā viññāṇaṃ, viññāṇasmiṃ vā attānaṃ.}}\\
\begin{addmargin}[1em]{2em}
\setstretch{.5}
{\PaliGlossB{They regard consciousness as self, self as having consciousness, consciousness in self, or self in consciousness.}}\\
\end{addmargin}
\end{absolutelynopagebreak}

\begin{absolutelynopagebreak}
\setstretch{.7}
{\PaliGlossA{evaṃ kho, bhikkhave, bhikkhu usseneti.}}\\
\begin{addmargin}[1em]{2em}
\setstretch{.5}
{\PaliGlossB{That’s how a mendicant draws close.}}\\
\end{addmargin}
\end{absolutelynopagebreak}

\begin{absolutelynopagebreak}
\setstretch{.7}
{\PaliGlossA{kathañca, bhikkhave, bhikkhu na usseneti?}}\\
\begin{addmargin}[1em]{2em}
\setstretch{.5}
{\PaliGlossB{And how does a mendicant not draw close?}}\\
\end{addmargin}
\end{absolutelynopagebreak}

\begin{absolutelynopagebreak}
\setstretch{.7}
{\PaliGlossA{idha, bhikkhave, bhikkhu na rūpaṃ attato samanupassati, na rūpavantaṃ vā attānaṃ, na attani vā rūpaṃ, na rūpasmiṃ vā attānaṃ;}}\\
\begin{addmargin}[1em]{2em}
\setstretch{.5}
{\PaliGlossB{It’s when a mendicant doesn’t regard form as self, self as having form, form in self, or self in form.}}\\
\end{addmargin}
\end{absolutelynopagebreak}

\begin{absolutelynopagebreak}
\setstretch{.7}
{\PaliGlossA{na vedanaṃ attato samanupassati, na vedanāvantaṃ vā attānaṃ, na attani vā vedanaṃ, na vedanāya vā attānaṃ;}}\\
\begin{addmargin}[1em]{2em}
\setstretch{.5}
{\PaliGlossB{They don’t regard feeling as self, self as having feeling, feeling in self, or self in feeling.}}\\
\end{addmargin}
\end{absolutelynopagebreak}

\begin{absolutelynopagebreak}
\setstretch{.7}
{\PaliGlossA{na saññaṃ attato samanupassati, na saññāvantaṃ vā attānaṃ, na attani vā saññaṃ, na saññāya vā attānaṃ;}}\\
\begin{addmargin}[1em]{2em}
\setstretch{.5}
{\PaliGlossB{They don’t regard perception as self, self as having perception, perception in self, or self in perception.}}\\
\end{addmargin}
\end{absolutelynopagebreak}

\begin{absolutelynopagebreak}
\setstretch{.7}
{\PaliGlossA{na saṅkhāre attato samanupassati, na saṅkhāravantaṃ vā attānaṃ, na attani vā saṅkhāre, na saṅkhāresu vā attānaṃ;}}\\
\begin{addmargin}[1em]{2em}
\setstretch{.5}
{\PaliGlossB{They don’t regard choices as self, self as having choices, choices in self, or self in choices.}}\\
\end{addmargin}
\end{absolutelynopagebreak}

\begin{absolutelynopagebreak}
\setstretch{.7}
{\PaliGlossA{na viññāṇaṃ attato samanupassati, na viññāṇavantaṃ vā attānaṃ, na attani vā viññāṇaṃ, na viññāṇasmiṃ vā attānaṃ.}}\\
\begin{addmargin}[1em]{2em}
\setstretch{.5}
{\PaliGlossB{They don’t regard consciousness as self, self as having consciousness, consciousness in self, or self in consciousness.}}\\
\end{addmargin}
\end{absolutelynopagebreak}

\begin{absolutelynopagebreak}
\setstretch{.7}
{\PaliGlossA{evaṃ kho, bhikkhave, bhikkhu na usseneti.}}\\
\begin{addmargin}[1em]{2em}
\setstretch{.5}
{\PaliGlossB{That’s how a mendicant doesn’t draw close.}}\\
\end{addmargin}
\end{absolutelynopagebreak}

\begin{absolutelynopagebreak}
\setstretch{.7}
{\PaliGlossA{kathañca, bhikkhave, bhikkhu paṭiseneti?}}\\
\begin{addmargin}[1em]{2em}
\setstretch{.5}
{\PaliGlossB{And how does a mendicant push back?}}\\
\end{addmargin}
\end{absolutelynopagebreak}

\begin{absolutelynopagebreak}
\setstretch{.7}
{\PaliGlossA{idha, bhikkhave, bhikkhu akkosantaṃ paccakkosati, rosantaṃ paṭirosati, bhaṇḍantaṃ paṭibhaṇḍati.}}\\
\begin{addmargin}[1em]{2em}
\setstretch{.5}
{\PaliGlossB{It’s when someone abuses, annoys, or argues with a mendicant, and the mendicant abuses, annoys, or argues back at them.}}\\
\end{addmargin}
\end{absolutelynopagebreak}

\begin{absolutelynopagebreak}
\setstretch{.7}
{\PaliGlossA{evaṃ kho, bhikkhave, bhikkhu paṭiseneti.}}\\
\begin{addmargin}[1em]{2em}
\setstretch{.5}
{\PaliGlossB{That’s how a mendicant pushes back.}}\\
\end{addmargin}
\end{absolutelynopagebreak}

\begin{absolutelynopagebreak}
\setstretch{.7}
{\PaliGlossA{kathañca, bhikkhave, bhikkhu na paṭiseneti?}}\\
\begin{addmargin}[1em]{2em}
\setstretch{.5}
{\PaliGlossB{And how does a mendicant not push back?}}\\
\end{addmargin}
\end{absolutelynopagebreak}

\begin{absolutelynopagebreak}
\setstretch{.7}
{\PaliGlossA{idha, bhikkhave, bhikkhu akkosantaṃ na paccakkosati, rosantaṃ na paṭirosati, bhaṇḍantaṃ na paṭibhaṇḍati.}}\\
\begin{addmargin}[1em]{2em}
\setstretch{.5}
{\PaliGlossB{It’s when someone abuses, annoys, or argues with a mendicant, and the mendicant doesn’t abuse, annoy, or argue back at them.}}\\
\end{addmargin}
\end{absolutelynopagebreak}

\begin{absolutelynopagebreak}
\setstretch{.7}
{\PaliGlossA{evaṃ kho, bhikkhave, bhikkhu na paṭiseneti.}}\\
\begin{addmargin}[1em]{2em}
\setstretch{.5}
{\PaliGlossB{That’s how a mendicant doesn’t push back.}}\\
\end{addmargin}
\end{absolutelynopagebreak}

\begin{absolutelynopagebreak}
\setstretch{.7}
{\PaliGlossA{kathañca, bhikkhave, bhikkhu dhūpāyati?}}\\
\begin{addmargin}[1em]{2em}
\setstretch{.5}
{\PaliGlossB{And how does a mendicant fume?}}\\
\end{addmargin}
\end{absolutelynopagebreak}

\begin{absolutelynopagebreak}
\setstretch{.7}
{\PaliGlossA{asmīti, bhikkhave, sati itthasmīti hoti, evaṃsmīti hoti, aññathāsmīti hoti, asasmīti hoti, satasmīti hoti, santi hoti, itthaṃ santi hoti, evaṃ santi hoti, aññathā santi hoti, apihaṃ santi hoti, apihaṃ itthaṃ santi hoti, apihaṃ evaṃ santi hoti, apihaṃ aññathā santi hoti, bhavissanti hoti, itthaṃ bhavissanti hoti, evaṃ bhavissanti hoti, aññathā bhavissanti hoti.}}\\
\begin{addmargin}[1em]{2em}
\setstretch{.5}
{\PaliGlossB{When there is the concept ‘I am’, there are the concepts ‘I am such’, ‘I am thus’, ‘I am otherwise’; ‘I am fleeting’, ‘I am lasting’; ‘mine’, ‘such is mine’, ‘thus is mine’, ‘otherwise is mine’; ‘also mine’, ‘such is also mine’, ‘thus is also mine’, ‘otherwise is also mine’; ‘I will be’, ‘I will be such’, ‘I will be thus’, ‘I will be otherwise’.}}\\
\end{addmargin}
\end{absolutelynopagebreak}

\begin{absolutelynopagebreak}
\setstretch{.7}
{\PaliGlossA{evaṃ kho, bhikkhave, bhikkhu dhūpāyati.}}\\
\begin{addmargin}[1em]{2em}
\setstretch{.5}
{\PaliGlossB{That’s how a mendicant fumes.}}\\
\end{addmargin}
\end{absolutelynopagebreak}

\begin{absolutelynopagebreak}
\setstretch{.7}
{\PaliGlossA{kathañca, bhikkhave, bhikkhu na dhūpāyati?}}\\
\begin{addmargin}[1em]{2em}
\setstretch{.5}
{\PaliGlossB{And how does a mendicant not fume?}}\\
\end{addmargin}
\end{absolutelynopagebreak}

\begin{absolutelynopagebreak}
\setstretch{.7}
{\PaliGlossA{asmīti, bhikkhave, asati itthasmīti na hoti, evaṃsmīti na hoti, aññathāsmīti na hoti, asasmīti na hoti, satasmīti na hoti, santi na hoti, itthaṃ santi na hoti, evaṃ santi na hoti, aññathā santi na hoti, apihaṃ santi na hoti, apihaṃ itthaṃ santi na hoti, apihaṃ evaṃ santi na hoti, apihaṃ aññathā santi na hoti, bhavissanti na hoti, itthaṃ bhavissanti na hoti, evaṃ bhavissanti na hoti, aññathā bhavissanti na hoti.}}\\
\begin{addmargin}[1em]{2em}
\setstretch{.5}
{\PaliGlossB{When there is no concept ‘I am’, there are no concepts ‘I am such’, ‘I am thus’, ‘I am otherwise’; ‘I am fleeting’, ‘I am lasting’; ‘mine’, ‘such is mine’, ‘thus is mine’, ‘otherwise is mine’; ‘also mine’, ‘such is also mine’, ‘thus is also mine’, ‘otherwise is also mine’; ‘I will be’, ‘I will be such’, ‘I will be thus’, ‘I will be otherwise’.}}\\
\end{addmargin}
\end{absolutelynopagebreak}

\begin{absolutelynopagebreak}
\setstretch{.7}
{\PaliGlossA{evaṃ kho, bhikkhave, bhikkhu na dhūpāyati.}}\\
\begin{addmargin}[1em]{2em}
\setstretch{.5}
{\PaliGlossB{That’s how a mendicant doesn’t fume.}}\\
\end{addmargin}
\end{absolutelynopagebreak}

\begin{absolutelynopagebreak}
\setstretch{.7}
{\PaliGlossA{kathañca, bhikkhave, bhikkhu pajjalati?}}\\
\begin{addmargin}[1em]{2em}
\setstretch{.5}
{\PaliGlossB{And how is a mendicant ignited?}}\\
\end{addmargin}
\end{absolutelynopagebreak}

\begin{absolutelynopagebreak}
\setstretch{.7}
{\PaliGlossA{iminā asmīti, bhikkhave, sati iminā itthasmīti hoti, iminā evaṃsmīti hoti, iminā aññathāsmīti hoti, iminā asasmīti hoti, iminā satasmīti hoti, iminā santi hoti, iminā itthaṃ santi hoti, iminā evaṃ santi hoti, iminā aññathā santi hoti, iminā apihaṃ santi hoti, iminā apihaṃ itthaṃ santi hoti, iminā apihaṃ evaṃ santi hoti, iminā apihaṃ aññathā santi hoti, iminā bhavissanti hoti, iminā itthaṃ bhavissanti hoti, iminā evaṃ bhavissanti hoti, iminā aññathā bhavissanti hoti.}}\\
\begin{addmargin}[1em]{2em}
\setstretch{.5}
{\PaliGlossB{When there is the concept ‘I am because of this’, there are the concepts ‘I am such because of this’, ‘I am thus because of this’, ‘I am otherwise because of this’; ‘I am fleeting because of this’, ‘I am lasting because of this’; ‘mine because of this’, ‘such is mine because of this’, ‘thus is mine because of this’, ‘otherwise is mine because of this’; ‘also mine because of this’, ‘such is also mine because of this’, ‘thus is also mine because of this’, ‘otherwise is also mine because of this’; ‘I will be because of this’, ‘I will be such because of this’, ‘I will be thus because of this’, ‘I will be otherwise because of this’.}}\\
\end{addmargin}
\end{absolutelynopagebreak}

\begin{absolutelynopagebreak}
\setstretch{.7}
{\PaliGlossA{evaṃ kho, bhikkhave, bhikkhu pajjalati.}}\\
\begin{addmargin}[1em]{2em}
\setstretch{.5}
{\PaliGlossB{That’s how a mendicant is ignited.}}\\
\end{addmargin}
\end{absolutelynopagebreak}

\begin{absolutelynopagebreak}
\setstretch{.7}
{\PaliGlossA{kathañca, bhikkhave, bhikkhu na pajjalati?}}\\
\begin{addmargin}[1em]{2em}
\setstretch{.5}
{\PaliGlossB{And how is a mendicant not ignited?}}\\
\end{addmargin}
\end{absolutelynopagebreak}

\begin{absolutelynopagebreak}
\setstretch{.7}
{\PaliGlossA{iminā asmīti, bhikkhave, asati iminā itthasmīti na hoti, iminā evaṃsmīti na hoti, iminā aññathāsmīti na hoti, iminā asasmīti na hoti, iminā satasmīti na hoti, iminā santi na hoti, iminā itthaṃ santi na hoti, iminā evaṃ santi na hoti, iminā aññathā santi na hoti, iminā apihaṃ santi na hoti, iminā apihaṃ itthaṃ santi na hoti, iminā apihaṃ evaṃ santi na hoti, iminā apihaṃ aññathā santi na hoti, iminā bhavissanti na hoti, iminā itthaṃ bhavissanti na hoti, iminā evaṃ bhavissanti na hoti, iminā aññathā bhavissanti na hoti.}}\\
\begin{addmargin}[1em]{2em}
\setstretch{.5}
{\PaliGlossB{When there is no concept ‘I am because of this’, there are no concepts ‘I am such because of this’, ‘I am thus because of this’, ‘I am otherwise because of this’; ‘I am fleeting because of this’, ‘I am lasting because of this’; ‘mine because of this’, ‘such is mine because of this’, ‘thus is mine because of this’, ‘otherwise is mine because of this’; ‘also mine because of this’, ‘such is also mine because of this’, ‘thus is also mine because of this’, ‘otherwise is also mine because of this’; ‘I will be because of this’, ‘I will be such because of this’, ‘I will be thus because of this’, ‘I will be otherwise because of this’.}}\\
\end{addmargin}
\end{absolutelynopagebreak}

\begin{absolutelynopagebreak}
\setstretch{.7}
{\PaliGlossA{evaṃ kho, bhikkhave, bhikkhu na pajjalati.}}\\
\begin{addmargin}[1em]{2em}
\setstretch{.5}
{\PaliGlossB{That’s how a mendicant is not ignited.}}\\
\end{addmargin}
\end{absolutelynopagebreak}

\begin{absolutelynopagebreak}
\setstretch{.7}
{\PaliGlossA{kathañca, bhikkhave, bhikkhu sampajjhāyati?}}\\
\begin{addmargin}[1em]{2em}
\setstretch{.5}
{\PaliGlossB{And how does a mendicant burn up?}}\\
\end{addmargin}
\end{absolutelynopagebreak}

\begin{absolutelynopagebreak}
\setstretch{.7}
{\PaliGlossA{idha, bhikkhave, bhikkhuno asmimāno pahīno na hoti ucchinnamūlo tālāvatthukato anabhāvaṅkato āyatiṃ anuppādadhammo.}}\\
\begin{addmargin}[1em]{2em}
\setstretch{.5}
{\PaliGlossB{It’s when a mendicant hasn’t given up the conceit ‘I am’, cut it off at the root, made it like a palm stump, obliterated it, so it’s unable to arise in the future.}}\\
\end{addmargin}
\end{absolutelynopagebreak}

\begin{absolutelynopagebreak}
\setstretch{.7}
{\PaliGlossA{evaṃ kho, bhikkhave, bhikkhu sampajjhāyati.}}\\
\begin{addmargin}[1em]{2em}
\setstretch{.5}
{\PaliGlossB{That’s how a mendicant is burned up.}}\\
\end{addmargin}
\end{absolutelynopagebreak}

\begin{absolutelynopagebreak}
\setstretch{.7}
{\PaliGlossA{kathañca, bhikkhave, bhikkhu na sampajjhāyati?}}\\
\begin{addmargin}[1em]{2em}
\setstretch{.5}
{\PaliGlossB{And how does a mendicant not burn up?}}\\
\end{addmargin}
\end{absolutelynopagebreak}

\begin{absolutelynopagebreak}
\setstretch{.7}
{\PaliGlossA{idha, bhikkhave, bhikkhuno asmimāno pahīno hoti ucchinnamūlo tālāvatthukato anabhāvaṅkato āyatiṃ anuppādadhammo.}}\\
\begin{addmargin}[1em]{2em}
\setstretch{.5}
{\PaliGlossB{It’s when a mendicant has given up the conceit ‘I am’, cut it off at the root, made it like a palm stump, obliterated it, so it’s unable to arise in the future.}}\\
\end{addmargin}
\end{absolutelynopagebreak}

\begin{absolutelynopagebreak}
\setstretch{.7}
{\PaliGlossA{evaṃ kho, bhikkhave, bhikkhu na sampajjhāyatī”ti.}}\\
\begin{addmargin}[1em]{2em}
\setstretch{.5}
{\PaliGlossB{That’s how a mendicant is not burned up.”}}\\
\end{addmargin}
\end{absolutelynopagebreak}

\begin{absolutelynopagebreak}
\setstretch{.7}
{\PaliGlossA{dasamaṃ.}}\\
\begin{addmargin}[1em]{2em}
\setstretch{.5}
{\PaliGlossB{    -}}\\
\end{addmargin}
\end{absolutelynopagebreak}

\begin{absolutelynopagebreak}
\setstretch{.7}
{\PaliGlossA{mahāvaggo pañcamo.}}\\
\begin{addmargin}[1em]{2em}
\setstretch{.5}
{\PaliGlossB{    -}}\\
\end{addmargin}
\end{absolutelynopagebreak}

\begin{absolutelynopagebreak}
\setstretch{.7}
{\PaliGlossA{sotānugataṃ ṭhānaṃ,}}\\
\begin{addmargin}[1em]{2em}
\setstretch{.5}
{\PaliGlossB{    -}}\\
\end{addmargin}
\end{absolutelynopagebreak}

\begin{absolutelynopagebreak}
\setstretch{.7}
{\PaliGlossA{bhaddiya sāmugiya vappa sāḷhā ca;}}\\
\begin{addmargin}[1em]{2em}
\setstretch{.5}
{\PaliGlossB{    -}}\\
\end{addmargin}
\end{absolutelynopagebreak}

\begin{absolutelynopagebreak}
\setstretch{.7}
{\PaliGlossA{mallika attantāpo,}}\\
\begin{addmargin}[1em]{2em}
\setstretch{.5}
{\PaliGlossB{    -}}\\
\end{addmargin}
\end{absolutelynopagebreak}

\begin{absolutelynopagebreak}
\setstretch{.7}
{\PaliGlossA{taṇhā pemena ca dasā teti.}}\\
\begin{addmargin}[1em]{2em}
\setstretch{.5}
{\PaliGlossB{    -}}\\
\end{addmargin}
\end{absolutelynopagebreak}

\begin{absolutelynopagebreak}
\setstretch{.7}
{\PaliGlossA{catuttho mahāpaṇṇāsako samatto.}}\\
\begin{addmargin}[1em]{2em}
\setstretch{.5}
{\PaliGlossB{    -}}\\
\end{addmargin}
\end{absolutelynopagebreak}
