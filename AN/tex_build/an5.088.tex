
\begin{absolutelynopagebreak}
\setstretch{.7}
{\PaliGlossA{aṅguttara nikāya 5}}\\
\begin{addmargin}[1em]{2em}
\setstretch{.5}
{\PaliGlossB{Numbered Discourses 5}}\\
\end{addmargin}
\end{absolutelynopagebreak}

\begin{absolutelynopagebreak}
\setstretch{.7}
{\PaliGlossA{9. theravagga}}\\
\begin{addmargin}[1em]{2em}
\setstretch{.5}
{\PaliGlossB{9. Senior Mendicants}}\\
\end{addmargin}
\end{absolutelynopagebreak}

\begin{absolutelynopagebreak}
\setstretch{.7}
{\PaliGlossA{88. therasutta}}\\
\begin{addmargin}[1em]{2em}
\setstretch{.5}
{\PaliGlossB{88. Senior Mendicants}}\\
\end{addmargin}
\end{absolutelynopagebreak}

\begin{absolutelynopagebreak}
\setstretch{.7}
{\PaliGlossA{“pañcahi, bhikkhave, dhammehi samannāgato thero bhikkhu bahujanaahitāya paṭipanno hoti bahujanaasukhāya bahuno janassa anatthāya ahitāya dukkhāya devamanussānaṃ.}}\\
\begin{addmargin}[1em]{2em}
\setstretch{.5}
{\PaliGlossB{“Mendicants, a senior mendicant who has five qualities is acting for the hurt and unhappiness of the people, for the harm, hurt, and suffering of gods and humans.}}\\
\end{addmargin}
\end{absolutelynopagebreak}

\begin{absolutelynopagebreak}
\setstretch{.7}
{\PaliGlossA{katamehi pañcahi?}}\\
\begin{addmargin}[1em]{2em}
\setstretch{.5}
{\PaliGlossB{What five?}}\\
\end{addmargin}
\end{absolutelynopagebreak}

\begin{absolutelynopagebreak}
\setstretch{.7}
{\PaliGlossA{thero hoti rattaññū cirapabbajito;}}\\
\begin{addmargin}[1em]{2em}
\setstretch{.5}
{\PaliGlossB{They are senior and have long gone forth.}}\\
\end{addmargin}
\end{absolutelynopagebreak}

\begin{absolutelynopagebreak}
\setstretch{.7}
{\PaliGlossA{ñāto hoti yasassī sagahaṭṭhapabbajitānaṃ bahujanaparivāro;}}\\
\begin{addmargin}[1em]{2em}
\setstretch{.5}
{\PaliGlossB{They’re well-known, famous, with a large following that includes both laypeople and renunciates.}}\\
\end{addmargin}
\end{absolutelynopagebreak}

\begin{absolutelynopagebreak}
\setstretch{.7}
{\PaliGlossA{lābhī hoti cīvarapiṇḍapātasenāsanagilānappaccayabhesajjaparikkhārānaṃ;}}\\
\begin{addmargin}[1em]{2em}
\setstretch{.5}
{\PaliGlossB{They receive robes, alms-food, lodgings, and medicines and supplies for the sick.}}\\
\end{addmargin}
\end{absolutelynopagebreak}

\begin{absolutelynopagebreak}
\setstretch{.7}
{\PaliGlossA{bahussuto hoti sutadharo sutasannicayo, ye te dhammā ādikalyāṇā majjhekalyāṇā pariyosānakalyāṇā sātthaṃ sabyañjanaṃ kevalaparipuṇṇaṃ parisuddhaṃ brahmacariyaṃ abhivadanti, tathārūpāssa dhammā bahussutā honti dhātā vacasā paricitā manasānupekkhitā diṭṭhiyā appaṭividdhā;}}\\
\begin{addmargin}[1em]{2em}
\setstretch{.5}
{\PaliGlossB{They’re very learned, remembering and keeping what they’ve learned. These teachings are good in the beginning, good in the middle, and good in the end, meaningful and well-phrased, describing a spiritual practice that’s entirely full and pure. They are very learned in such teachings, remembering them, reinforcing them by recitation, mentally scrutinizing them, and understanding them with view.}}\\
\end{addmargin}
\end{absolutelynopagebreak}

\begin{absolutelynopagebreak}
\setstretch{.7}
{\PaliGlossA{micchādiṭṭhiko hoti viparītadassano, so bahujanaṃ saddhammā vuṭṭhāpetvā asaddhamme patiṭṭhāpeti.}}\\
\begin{addmargin}[1em]{2em}
\setstretch{.5}
{\PaliGlossB{But they have wrong view and distorted perspective. They draw many people away from the true teaching and establish them in false teachings.}}\\
\end{addmargin}
\end{absolutelynopagebreak}

\begin{absolutelynopagebreak}
\setstretch{.7}
{\PaliGlossA{thero bhikkhu rattaññū cirapabbajito itipissa diṭṭhānugatiṃ āpajjanti, ñāto thero bhikkhu yasassī sagahaṭṭhapabbajitānaṃ bahujanaparivāro itipissa diṭṭhānugatiṃ āpajjanti, lābhī thero bhikkhu cīvarapiṇḍapātasenāsanagilānappaccayabhesajjaparikkhārānaṃ itipissa diṭṭhānugatiṃ āpajjanti, bahussuto thero bhikkhu sutadharo sutasannicayo itipissa diṭṭhānugatiṃ āpajjanti.}}\\
\begin{addmargin}[1em]{2em}
\setstretch{.5}
{\PaliGlossB{People follow their example, thinking that the senior mendicant is senior and has long gone forth. Or that they’re well-known, famous, with a large following that includes both laypeople and renunciates. Or that they receive robes, alms-food, lodgings, and medicines and supplies for the sick. Or that they’re very learned, remembering and keeping what they’ve learned.}}\\
\end{addmargin}
\end{absolutelynopagebreak}

\begin{absolutelynopagebreak}
\setstretch{.7}
{\PaliGlossA{imehi kho, bhikkhave, pañcahi dhammehi samannāgato thero bhikkhu bahujanaahitāya paṭipanno hoti bahujanaasukhāya bahuno janassa anatthāya ahitāya dukkhāya devamanussānaṃ.}}\\
\begin{addmargin}[1em]{2em}
\setstretch{.5}
{\PaliGlossB{A senior mendicant who has these five qualities is acting for the hurt and unhappiness of the people, for the harm, hurt, and suffering of gods and humans.}}\\
\end{addmargin}
\end{absolutelynopagebreak}

\begin{absolutelynopagebreak}
\setstretch{.7}
{\PaliGlossA{pañcahi, bhikkhave, dhammehi samannāgato thero bhikkhu bahujanahitāya paṭipanno hoti bahujanasukhāya bahuno janassa atthāya hitāya sukhāya devamanussānaṃ.}}\\
\begin{addmargin}[1em]{2em}
\setstretch{.5}
{\PaliGlossB{A senior mendicant who has five qualities is acting for the welfare and happiness of the people, for the benefit, welfare, and happiness of gods and humans.}}\\
\end{addmargin}
\end{absolutelynopagebreak}

\begin{absolutelynopagebreak}
\setstretch{.7}
{\PaliGlossA{katamehi pañcahi?}}\\
\begin{addmargin}[1em]{2em}
\setstretch{.5}
{\PaliGlossB{What five?}}\\
\end{addmargin}
\end{absolutelynopagebreak}

\begin{absolutelynopagebreak}
\setstretch{.7}
{\PaliGlossA{thero hoti rattaññū cirapabbajito;}}\\
\begin{addmargin}[1em]{2em}
\setstretch{.5}
{\PaliGlossB{They are senior and have long gone forth.}}\\
\end{addmargin}
\end{absolutelynopagebreak}

\begin{absolutelynopagebreak}
\setstretch{.7}
{\PaliGlossA{ñāto hoti yasassī sagahaṭṭhapabbajitānaṃ bahujanaparivāro;}}\\
\begin{addmargin}[1em]{2em}
\setstretch{.5}
{\PaliGlossB{They’re well-known, famous, with a large following, including both laypeople and renunciates.}}\\
\end{addmargin}
\end{absolutelynopagebreak}

\begin{absolutelynopagebreak}
\setstretch{.7}
{\PaliGlossA{lābhī hoti cīvarapiṇḍapātasenāsanagilānappaccayabhesajjaparikkhārānaṃ;}}\\
\begin{addmargin}[1em]{2em}
\setstretch{.5}
{\PaliGlossB{They receive robes, alms-food, lodgings, and medicines and supplies for the sick.}}\\
\end{addmargin}
\end{absolutelynopagebreak}

\begin{absolutelynopagebreak}
\setstretch{.7}
{\PaliGlossA{bahussuto hoti sutadharo sutasannicayo, ye te dhammā ādikalyāṇā majjhekalyāṇā pariyosānakalyāṇā sātthaṃ sabyañjanaṃ kevalaparipuṇṇaṃ parisuddhaṃ brahmacariyaṃ abhivadanti, tathārūpāssa dhammā bahussutā honti dhātā vacasā paricitā manasānupekkhitā diṭṭhiyā suppaṭividdhā;}}\\
\begin{addmargin}[1em]{2em}
\setstretch{.5}
{\PaliGlossB{They’re very learned, remembering and keeping what they’ve learned. These teachings are good in the beginning, good in the middle, and good in the end, meaningful and well-phrased, describing a spiritual practice that’s entirely full and pure. They are very learned in such teachings, remembering them, reinforcing them by recitation, mentally scrutinizing them, and comprehending them theoretically.}}\\
\end{addmargin}
\end{absolutelynopagebreak}

\begin{absolutelynopagebreak}
\setstretch{.7}
{\PaliGlossA{sammādiṭṭhiko hoti aviparītadassano, so bahujanaṃ asaddhammā vuṭṭhāpetvā saddhamme patiṭṭhāpeti.}}\\
\begin{addmargin}[1em]{2em}
\setstretch{.5}
{\PaliGlossB{And they have right view and an undistorted perspective. They draw many people away from false teachings and establish them in the true teaching.}}\\
\end{addmargin}
\end{absolutelynopagebreak}

\begin{absolutelynopagebreak}
\setstretch{.7}
{\PaliGlossA{thero bhikkhu rattaññū cirapabbajito itipissa diṭṭhānugatiṃ āpajjanti, ñāto thero bhikkhu yasassī sagahaṭṭhapabbajitānaṃ bahujanaparivāro itipissa diṭṭhānugatiṃ āpajjanti, lābhī thero bhikkhu cīvarapiṇḍapātasenāsanagilānappaccayabhesajjaparikkhārānaṃ itipissa diṭṭhānugatiṃ āpajjanti, bahussuto thero bhikkhu sutadharo sutasannicayo itipissa diṭṭhānugatiṃ āpajjanti.}}\\
\begin{addmargin}[1em]{2em}
\setstretch{.5}
{\PaliGlossB{People follow their example, thinking that the senior mendicant is senior and has long gone forth. Or that they’re well-known, famous, with a large following that includes both laypeople and renunciates. Or that they receive robes, alms-food, lodgings, and medicines and supplies for the sick. Or that they’re very learned, remembering and keeping what they’ve learned.}}\\
\end{addmargin}
\end{absolutelynopagebreak}

\begin{absolutelynopagebreak}
\setstretch{.7}
{\PaliGlossA{imehi kho, bhikkhave, pañcahi dhammehi samannāgato thero bhikkhu bahujanahitāya paṭipanno hoti bahujanasukhāya bahuno janassa atthāya hitāya sukhāya devamanussānan”ti.}}\\
\begin{addmargin}[1em]{2em}
\setstretch{.5}
{\PaliGlossB{A senior mendicant who has these five qualities is acting for the welfare and happiness of the people, for the benefit, welfare, and happiness of gods and humans.”}}\\
\end{addmargin}
\end{absolutelynopagebreak}

\begin{absolutelynopagebreak}
\setstretch{.7}
{\PaliGlossA{aṭṭhamaṃ.}}\\
\begin{addmargin}[1em]{2em}
\setstretch{.5}
{\PaliGlossB{    -}}\\
\end{addmargin}
\end{absolutelynopagebreak}
