
\begin{absolutelynopagebreak}
\setstretch{.7}
{\PaliGlossA{aṅguttara nikāya 7}}\\
\begin{addmargin}[1em]{2em}
\setstretch{.5}
{\PaliGlossB{Numbered Discourses 7}}\\
\end{addmargin}
\end{absolutelynopagebreak}

\begin{absolutelynopagebreak}
\setstretch{.7}
{\PaliGlossA{6. abyākatavagga}}\\
\begin{addmargin}[1em]{2em}
\setstretch{.5}
{\PaliGlossB{6. The Undeclared Points}}\\
\end{addmargin}
\end{absolutelynopagebreak}

\begin{absolutelynopagebreak}
\setstretch{.7}
{\PaliGlossA{61. pacalāyamānasutta}}\\
\begin{addmargin}[1em]{2em}
\setstretch{.5}
{\PaliGlossB{61. Nodding Off}}\\
\end{addmargin}
\end{absolutelynopagebreak}

\begin{absolutelynopagebreak}
\setstretch{.7}
{\PaliGlossA{evaṃ me sutaṃ—}}\\
\begin{addmargin}[1em]{2em}
\setstretch{.5}
{\PaliGlossB{So I have heard.}}\\
\end{addmargin}
\end{absolutelynopagebreak}

\begin{absolutelynopagebreak}
\setstretch{.7}
{\PaliGlossA{ekaṃ samayaṃ bhagavā bhaggesu viharati susumāragire bhesakaḷāvane migadāye.}}\\
\begin{addmargin}[1em]{2em}
\setstretch{.5}
{\PaliGlossB{At one time the Buddha was staying in the land of the Bhaggas on Crocodile Hill, in the deer park at Bhesakaḷā’s Wood.}}\\
\end{addmargin}
\end{absolutelynopagebreak}

\begin{absolutelynopagebreak}
\setstretch{.7}
{\PaliGlossA{tena kho pana samayena āyasmā mahāmoggallāno magadhesu kallavāḷaputtagāme pacalāyamāno nisinno hoti.}}\\
\begin{addmargin}[1em]{2em}
\setstretch{.5}
{\PaliGlossB{Now at that time, in the land of the Magadhans near Kallavāḷamutta Village, Venerable Mahāmoggallāna was nodding off while meditating.}}\\
\end{addmargin}
\end{absolutelynopagebreak}

\begin{absolutelynopagebreak}
\setstretch{.7}
{\PaliGlossA{addasā kho bhagavā dibbena cakkhunā visuddhena atikkantamānusakena āyasmantaṃ mahāmoggallānaṃ magadhesu kallavāḷaputtagāme pacalāyamānaṃ nisinnaṃ.}}\\
\begin{addmargin}[1em]{2em}
\setstretch{.5}
{\PaliGlossB{The Buddha saw him with his clairvoyance that is purified and superhuman.}}\\
\end{addmargin}
\end{absolutelynopagebreak}

\begin{absolutelynopagebreak}
\setstretch{.7}
{\PaliGlossA{disvā—}}\\
\begin{addmargin}[1em]{2em}
\setstretch{.5}
{\PaliGlossB{    -}}\\
\end{addmargin}
\end{absolutelynopagebreak}

\begin{absolutelynopagebreak}
\setstretch{.7}
{\PaliGlossA{seyyathāpi nāma balavā puriso samiñjitaṃ vā bāhaṃ pasāreyya, pasāritaṃ vā bāhaṃ samiñjeyya; evamevaṃ—bhaggesu susumāragire bhesakaḷāvane migadāye antarahito magadhesu kallavāḷaputtagāme āyasmato mahāmoggallānassa sammukhe pāturahosi.}}\\
\begin{addmargin}[1em]{2em}
\setstretch{.5}
{\PaliGlossB{Then, as easily as a strong person would extend or contract their arm, he vanished from the deer park at Bhesakaḷā’s Wood in the land of the Bhaggas and reappeared in front of Mahāmoggallāna near Kallavāḷamutta Village in the land of the Magadhans.}}\\
\end{addmargin}
\end{absolutelynopagebreak}

\begin{absolutelynopagebreak}
\setstretch{.7}
{\PaliGlossA{nisīdi bhagavā paññatte āsane.}}\\
\begin{addmargin}[1em]{2em}
\setstretch{.5}
{\PaliGlossB{He sat on the seat spread out}}\\
\end{addmargin}
\end{absolutelynopagebreak}

\begin{absolutelynopagebreak}
\setstretch{.7}
{\PaliGlossA{nisajja kho bhagavā āyasmantaṃ mahāmoggallānaṃ etadavoca:}}\\
\begin{addmargin}[1em]{2em}
\setstretch{.5}
{\PaliGlossB{and said to Mahāmoggallāna,}}\\
\end{addmargin}
\end{absolutelynopagebreak}

\begin{absolutelynopagebreak}
\setstretch{.7}
{\PaliGlossA{“pacalāyasi no tvaṃ, moggallāna, pacalāyasi no tvaṃ, moggallānā”ti?}}\\
\begin{addmargin}[1em]{2em}
\setstretch{.5}
{\PaliGlossB{“Are you nodding off, Moggallāna? Are you nodding off?”}}\\
\end{addmargin}
\end{absolutelynopagebreak}

\begin{absolutelynopagebreak}
\setstretch{.7}
{\PaliGlossA{“evaṃ, bhante”.}}\\
\begin{addmargin}[1em]{2em}
\setstretch{.5}
{\PaliGlossB{“Yes, sir.”}}\\
\end{addmargin}
\end{absolutelynopagebreak}

\begin{absolutelynopagebreak}
\setstretch{.7}
{\PaliGlossA{“tasmātiha, moggallāna, yathāsaññissa te viharato taṃ middhaṃ okkamati, taṃ saññaṃ mā manasākāsi, taṃ saññaṃ mā bahulamakāsi.}}\\
\begin{addmargin}[1em]{2em}
\setstretch{.5}
{\PaliGlossB{“So, Moggallāna, don’t focus on or cultivate the perception that you were meditating on when you fell drowsy.}}\\
\end{addmargin}
\end{absolutelynopagebreak}

\begin{absolutelynopagebreak}
\setstretch{.7}
{\PaliGlossA{ṭhānaṃ kho panetaṃ, moggallāna, vijjati yaṃ te evaṃ viharato taṃ middhaṃ pahīyetha. (1)}}\\
\begin{addmargin}[1em]{2em}
\setstretch{.5}
{\PaliGlossB{It’s possible that you’ll give up drowsiness in this way.}}\\
\end{addmargin}
\end{absolutelynopagebreak}

\begin{absolutelynopagebreak}
\setstretch{.7}
{\PaliGlossA{no ce te evaṃ viharato taṃ middhaṃ pahīyetha, tato tvaṃ, moggallāna, yathāsutaṃ yathāpariyattaṃ dhammaṃ cetasā anuvitakkeyyāsi anuvicāreyyāsi, manasā anupekkheyyāsi.}}\\
\begin{addmargin}[1em]{2em}
\setstretch{.5}
{\PaliGlossB{But what if that doesn’t work? Then think about and consider the teaching as you’ve learned and memorized it, examining it with your mind.}}\\
\end{addmargin}
\end{absolutelynopagebreak}

\begin{absolutelynopagebreak}
\setstretch{.7}
{\PaliGlossA{ṭhānaṃ kho panetaṃ vijjati yaṃ te evaṃ viharato taṃ middhaṃ pahīyetha. (2)}}\\
\begin{addmargin}[1em]{2em}
\setstretch{.5}
{\PaliGlossB{It’s possible that you’ll give up drowsiness in this way.}}\\
\end{addmargin}
\end{absolutelynopagebreak}

\begin{absolutelynopagebreak}
\setstretch{.7}
{\PaliGlossA{no ce te evaṃ viharato taṃ middhaṃ pahīyetha, tato tvaṃ, moggallāna, yathāsutaṃ yathāpariyattaṃ dhammaṃ vitthārena sajjhāyaṃ kareyyāsi.}}\\
\begin{addmargin}[1em]{2em}
\setstretch{.5}
{\PaliGlossB{But what if that doesn’t work? Then recite in detail the teaching as you’ve learned and memorized it.}}\\
\end{addmargin}
\end{absolutelynopagebreak}

\begin{absolutelynopagebreak}
\setstretch{.7}
{\PaliGlossA{ṭhānaṃ kho panetaṃ vijjati yaṃ te evaṃ viharato taṃ middhaṃ pahīyetha. (3)}}\\
\begin{addmargin}[1em]{2em}
\setstretch{.5}
{\PaliGlossB{It’s possible that you’ll give up drowsiness in this way.}}\\
\end{addmargin}
\end{absolutelynopagebreak}

\begin{absolutelynopagebreak}
\setstretch{.7}
{\PaliGlossA{no ce te evaṃ viharato taṃ middhaṃ pahīyetha, tato tvaṃ, moggallāna, ubho kaṇṇasotāni āviñcheyyāsi, pāṇinā gattāni anumajjeyyāsi.}}\\
\begin{addmargin}[1em]{2em}
\setstretch{.5}
{\PaliGlossB{But what if that doesn’t work? Then pinch your ears and rub your limbs.}}\\
\end{addmargin}
\end{absolutelynopagebreak}

\begin{absolutelynopagebreak}
\setstretch{.7}
{\PaliGlossA{ṭhānaṃ kho panetaṃ vijjati yaṃ te evaṃ viharato taṃ middhaṃ pahīyetha. (4)}}\\
\begin{addmargin}[1em]{2em}
\setstretch{.5}
{\PaliGlossB{It’s possible that you’ll give up drowsiness in this way.}}\\
\end{addmargin}
\end{absolutelynopagebreak}

\begin{absolutelynopagebreak}
\setstretch{.7}
{\PaliGlossA{no ce te evaṃ viharato taṃ middhaṃ pahīyetha, tato tvaṃ, moggallāna, uṭṭhāyāsanā udakena akkhīni anumajjitvā disā anuvilokeyyāsi, nakkhattāni tārakarūpāni ullokeyyāsi.}}\\
\begin{addmargin}[1em]{2em}
\setstretch{.5}
{\PaliGlossB{But what if that doesn’t work? Then get up from your seat, flush your eyes with water, look around in every direction, and look up at the stars and constellations.}}\\
\end{addmargin}
\end{absolutelynopagebreak}

\begin{absolutelynopagebreak}
\setstretch{.7}
{\PaliGlossA{ṭhānaṃ kho panetaṃ vijjati yaṃ te evaṃ viharato taṃ middhaṃ pahīyetha. (5)}}\\
\begin{addmargin}[1em]{2em}
\setstretch{.5}
{\PaliGlossB{It’s possible that you’ll give up drowsiness in this way.}}\\
\end{addmargin}
\end{absolutelynopagebreak}

\begin{absolutelynopagebreak}
\setstretch{.7}
{\PaliGlossA{no ce te evaṃ viharato taṃ middhaṃ pahīyetha, tato tvaṃ, moggallāna, ālokasaññaṃ manasi kareyyāsi, divāsaññaṃ adhiṭṭhaheyyāsi—}}\\
\begin{addmargin}[1em]{2em}
\setstretch{.5}
{\PaliGlossB{But what if that doesn’t work? Then focus on the perception of light, concentrating on the perception of day,}}\\
\end{addmargin}
\end{absolutelynopagebreak}

\begin{absolutelynopagebreak}
\setstretch{.7}
{\PaliGlossA{yathā divā tathā rattiṃ yathā rattiṃ tathā divā.}}\\
\begin{addmargin}[1em]{2em}
\setstretch{.5}
{\PaliGlossB{regardless of whether it’s night or day.}}\\
\end{addmargin}
\end{absolutelynopagebreak}

\begin{absolutelynopagebreak}
\setstretch{.7}
{\PaliGlossA{iti vivaṭena cetasā apariyonaddhena sappabhāsaṃ cittaṃ bhāveyyāsi.}}\\
\begin{addmargin}[1em]{2em}
\setstretch{.5}
{\PaliGlossB{And so, with an open and unenveloped heart, develop a mind that’s full of radiance.}}\\
\end{addmargin}
\end{absolutelynopagebreak}

\begin{absolutelynopagebreak}
\setstretch{.7}
{\PaliGlossA{ṭhānaṃ kho panetaṃ vijjati yaṃ te evaṃ viharato taṃ middhaṃ pahīyetha. (6)}}\\
\begin{addmargin}[1em]{2em}
\setstretch{.5}
{\PaliGlossB{It’s possible that you’ll give up drowsiness in this way.}}\\
\end{addmargin}
\end{absolutelynopagebreak}

\begin{absolutelynopagebreak}
\setstretch{.7}
{\PaliGlossA{no ce te evaṃ viharato taṃ middhaṃ pahīyetha, tato tvaṃ, moggallāna, pacchāpuresaññī caṅkamaṃ adhiṭṭhaheyyāsi antogatehi indriyehi abahigatena mānasena.}}\\
\begin{addmargin}[1em]{2em}
\setstretch{.5}
{\PaliGlossB{But what if that doesn’t work? Then walk meditation concentrating on the perception of continuity, your faculties directed inwards and your mind not scattered outside.}}\\
\end{addmargin}
\end{absolutelynopagebreak}

\begin{absolutelynopagebreak}
\setstretch{.7}
{\PaliGlossA{ṭhānaṃ kho panetaṃ vijjati yaṃ te evaṃ viharato taṃ middhaṃ pahīyetha. (7)}}\\
\begin{addmargin}[1em]{2em}
\setstretch{.5}
{\PaliGlossB{It’s possible that you’ll give up drowsiness in this way.}}\\
\end{addmargin}
\end{absolutelynopagebreak}

\begin{absolutelynopagebreak}
\setstretch{.7}
{\PaliGlossA{no ce te evaṃ viharato taṃ middhaṃ pahīyetha, tato tvaṃ, moggallāna, dakkhiṇena passena sīhaseyyaṃ kappeyyāsi pāde pādaṃ accādhāya sato sampajāno uṭṭhānasaññaṃ manasi karitvā.}}\\
\begin{addmargin}[1em]{2em}
\setstretch{.5}
{\PaliGlossB{But what if that doesn’t work? Then lie down in the lion’s posture—on the right side, placing one foot on top of the other—mindful and aware, and focused on the time of getting up.}}\\
\end{addmargin}
\end{absolutelynopagebreak}

\begin{absolutelynopagebreak}
\setstretch{.7}
{\PaliGlossA{paṭibuddhena ca te, moggallāna, khippaññeva paccuṭṭhātabbaṃ:}}\\
\begin{addmargin}[1em]{2em}
\setstretch{.5}
{\PaliGlossB{When you wake, you should get up quickly, thinking:}}\\
\end{addmargin}
\end{absolutelynopagebreak}

\begin{absolutelynopagebreak}
\setstretch{.7}
{\PaliGlossA{‘na seyyasukhaṃ na passasukhaṃ na middhasukhaṃ anuyutto viharissāmī’ti.}}\\
\begin{addmargin}[1em]{2em}
\setstretch{.5}
{\PaliGlossB{‘I will not live attached to the pleasures of sleeping, lying down, and drowsing.’}}\\
\end{addmargin}
\end{absolutelynopagebreak}

\begin{absolutelynopagebreak}
\setstretch{.7}
{\PaliGlossA{evañhi te, moggallāna, sikkhitabbaṃ.}}\\
\begin{addmargin}[1em]{2em}
\setstretch{.5}
{\PaliGlossB{That’s how you should train.}}\\
\end{addmargin}
\end{absolutelynopagebreak}

\begin{absolutelynopagebreak}
\setstretch{.7}
{\PaliGlossA{tasmātiha, moggallāna, evaṃ sikkhitabbaṃ:}}\\
\begin{addmargin}[1em]{2em}
\setstretch{.5}
{\PaliGlossB{So you should train like this:}}\\
\end{addmargin}
\end{absolutelynopagebreak}

\begin{absolutelynopagebreak}
\setstretch{.7}
{\PaliGlossA{‘na uccāsoṇḍaṃ paggahetvā kulāni upasaṅkamissāmī’ti.}}\\
\begin{addmargin}[1em]{2em}
\setstretch{.5}
{\PaliGlossB{‘I will not approach families with my head swollen with vanity.’}}\\
\end{addmargin}
\end{absolutelynopagebreak}

\begin{absolutelynopagebreak}
\setstretch{.7}
{\PaliGlossA{evañhi te, moggallāna, sikkhitabbaṃ.}}\\
\begin{addmargin}[1em]{2em}
\setstretch{.5}
{\PaliGlossB{That’s how you should train.}}\\
\end{addmargin}
\end{absolutelynopagebreak}

\begin{absolutelynopagebreak}
\setstretch{.7}
{\PaliGlossA{sace, moggallāna, bhikkhu uccāsoṇḍaṃ paggahetvā kulāni upasaṅkamati, santi hi, moggallāna, kulesu kiccakaraṇīyāni.}}\\
\begin{addmargin}[1em]{2em}
\setstretch{.5}
{\PaliGlossB{What happens if a mendicant approaches families with a head swollen with vanity? Well, families have business to attend to,}}\\
\end{addmargin}
\end{absolutelynopagebreak}

\begin{absolutelynopagebreak}
\setstretch{.7}
{\PaliGlossA{yehi manussā āgataṃ bhikkhuṃ na manasi karonti, tatra bhikkhussa evaṃ hoti:}}\\
\begin{addmargin}[1em]{2em}
\setstretch{.5}
{\PaliGlossB{so people might not notice when a mendicant arrives. In that case the mendicant thinks:}}\\
\end{addmargin}
\end{absolutelynopagebreak}

\begin{absolutelynopagebreak}
\setstretch{.7}
{\PaliGlossA{‘kosu nāma idāni maṃ imasmiṃ kule paribhindi, virattarūpā dānime mayi manussā’ti.}}\\
\begin{addmargin}[1em]{2em}
\setstretch{.5}
{\PaliGlossB{‘Who on earth has turned this family against me? It seems they don’t like me any more.’}}\\
\end{addmargin}
\end{absolutelynopagebreak}

\begin{absolutelynopagebreak}
\setstretch{.7}
{\PaliGlossA{itissa alābhena maṅkubhāvo, maṅkubhūtassa uddhaccaṃ, uddhatassa asaṃvaro, asaṃvutassa ārā cittaṃ samādhimhā.}}\\
\begin{addmargin}[1em]{2em}
\setstretch{.5}
{\PaliGlossB{And so, because they don’t get anything they feel dismayed. Being dismayed, they become restless. Being restless, they lose restraint. And without restraint the mind is far from immersion.}}\\
\end{addmargin}
\end{absolutelynopagebreak}

\begin{absolutelynopagebreak}
\setstretch{.7}
{\PaliGlossA{tasmātiha, moggallāna, evaṃ sikkhitabbaṃ:}}\\
\begin{addmargin}[1em]{2em}
\setstretch{.5}
{\PaliGlossB{So you should train like this:}}\\
\end{addmargin}
\end{absolutelynopagebreak}

\begin{absolutelynopagebreak}
\setstretch{.7}
{\PaliGlossA{‘na viggāhikakathaṃ kathessāmī’ti.}}\\
\begin{addmargin}[1em]{2em}
\setstretch{.5}
{\PaliGlossB{‘I won’t get into arguments.’}}\\
\end{addmargin}
\end{absolutelynopagebreak}

\begin{absolutelynopagebreak}
\setstretch{.7}
{\PaliGlossA{evañhi te, moggallāna, sikkhitabbaṃ.}}\\
\begin{addmargin}[1em]{2em}
\setstretch{.5}
{\PaliGlossB{That’s how you should train.}}\\
\end{addmargin}
\end{absolutelynopagebreak}

\begin{absolutelynopagebreak}
\setstretch{.7}
{\PaliGlossA{viggāhikāya, moggallāna, kathāya sati kathābāhullaṃ pāṭikaṅkhaṃ, kathābāhulle sati uddhaccaṃ, uddhatassa asaṃvaro, asaṃvutassa ārā cittaṃ samādhimhā.}}\\
\begin{addmargin}[1em]{2em}
\setstretch{.5}
{\PaliGlossB{When there’s an argument, you can expect there’ll be lots of talking. When there’s lots of talking, people become restless. Being restless, they lose restraint. And without restraint the mind is far from immersion.}}\\
\end{addmargin}
\end{absolutelynopagebreak}

\begin{absolutelynopagebreak}
\setstretch{.7}
{\PaliGlossA{nāhaṃ, moggallāna, sabbeheva saṃsaggaṃ vaṇṇayāmi.}}\\
\begin{addmargin}[1em]{2em}
\setstretch{.5}
{\PaliGlossB{Moggallāna, I don’t praise all kinds of closeness.}}\\
\end{addmargin}
\end{absolutelynopagebreak}

\begin{absolutelynopagebreak}
\setstretch{.7}
{\PaliGlossA{na panāhaṃ, moggallāna, sabbeheva saṃsaggaṃ na vaṇṇayāmi.}}\\
\begin{addmargin}[1em]{2em}
\setstretch{.5}
{\PaliGlossB{Nor do I criticize all kinds of closeness.}}\\
\end{addmargin}
\end{absolutelynopagebreak}

\begin{absolutelynopagebreak}
\setstretch{.7}
{\PaliGlossA{sagahaṭṭhapabbajitehi kho ahaṃ, moggallāna, saṃsaggaṃ na vaṇṇayāmi.}}\\
\begin{addmargin}[1em]{2em}
\setstretch{.5}
{\PaliGlossB{I don’t praise closeness with laypeople and renunciates.}}\\
\end{addmargin}
\end{absolutelynopagebreak}

\begin{absolutelynopagebreak}
\setstretch{.7}
{\PaliGlossA{yāni ca kho tāni senāsanāni appasaddāni appanigghosāni vijanavātāni manussarāhasseyyakāni paṭisallānasāruppāni tathārūpehi senāsanehi saṃsaggaṃ vaṇṇayāmī”ti.}}\\
\begin{addmargin}[1em]{2em}
\setstretch{.5}
{\PaliGlossB{I do praise closeness with those lodgings that are quiet and still, far from the madding crowd, remote from human settlements, and fit for retreat.”}}\\
\end{addmargin}
\end{absolutelynopagebreak}

\begin{absolutelynopagebreak}
\setstretch{.7}
{\PaliGlossA{evaṃ vutte, āyasmā mahāmoggallāno bhagavantaṃ etadavoca:}}\\
\begin{addmargin}[1em]{2em}
\setstretch{.5}
{\PaliGlossB{When he said this, Venerable Moggallāna asked the Buddha,}}\\
\end{addmargin}
\end{absolutelynopagebreak}

\begin{absolutelynopagebreak}
\setstretch{.7}
{\PaliGlossA{“kittāvatā nu kho, bhante, bhikkhu saṅkhittena taṇhāsaṅkhayavimutto hoti accantaniṭṭho accantayogakkhemī accantabrahmacārī accantapariyosāno seṭṭho devamanussānan”ti?}}\\
\begin{addmargin}[1em]{2em}
\setstretch{.5}
{\PaliGlossB{“Sir, how do you briefly define a mendicant who is freed through the ending of craving, who has reached the ultimate end, the ultimate sanctuary, the ultimate spiritual life, the ultimate goal, and is best among gods and humans?”}}\\
\end{addmargin}
\end{absolutelynopagebreak}

\begin{absolutelynopagebreak}
\setstretch{.7}
{\PaliGlossA{“idha, moggallāna, bhikkhuno sutaṃ hoti:}}\\
\begin{addmargin}[1em]{2em}
\setstretch{.5}
{\PaliGlossB{“It’s when a mendicant has heard:}}\\
\end{addmargin}
\end{absolutelynopagebreak}

\begin{absolutelynopagebreak}
\setstretch{.7}
{\PaliGlossA{‘sabbe dhammā nālaṃ abhinivesāyā’ti;}}\\
\begin{addmargin}[1em]{2em}
\setstretch{.5}
{\PaliGlossB{‘Nothing is worth clinging on to.’}}\\
\end{addmargin}
\end{absolutelynopagebreak}

\begin{absolutelynopagebreak}
\setstretch{.7}
{\PaliGlossA{evañcetaṃ, moggallāna, bhikkhuno sutaṃ hoti:}}\\
\begin{addmargin}[1em]{2em}
\setstretch{.5}
{\PaliGlossB{When a mendicant has heard that}}\\
\end{addmargin}
\end{absolutelynopagebreak}

\begin{absolutelynopagebreak}
\setstretch{.7}
{\PaliGlossA{‘sabbe dhammā nālaṃ abhinivesāyā’ti.}}\\
\begin{addmargin}[1em]{2em}
\setstretch{.5}
{\PaliGlossB{nothing is worth clinging on to,}}\\
\end{addmargin}
\end{absolutelynopagebreak}

\begin{absolutelynopagebreak}
\setstretch{.7}
{\PaliGlossA{so sabbaṃ dhammaṃ abhijānāti, sabbaṃ dhammaṃ abhiññāya sabbaṃ dhammaṃ parijānāti. sabbaṃ dhammaṃ pariññāya yaṃ kiñci vedanaṃ vediyati sukhaṃ vā dukkhaṃ vā adukkhamasukhaṃ vā.}}\\
\begin{addmargin}[1em]{2em}
\setstretch{.5}
{\PaliGlossB{they directly know all things. Directly knowing all things, they completely understand all things. Having completely understood all things, when they experience any kind of feeling—pleasant, unpleasant, or neutral—}}\\
\end{addmargin}
\end{absolutelynopagebreak}

\begin{absolutelynopagebreak}
\setstretch{.7}
{\PaliGlossA{so tāsu vedanāsu aniccānupassī viharati, virāgānupassī viharati, nirodhānupassī viharati, paṭinissaggānupassī viharati.}}\\
\begin{addmargin}[1em]{2em}
\setstretch{.5}
{\PaliGlossB{they meditate observing impermanence, dispassion, cessation, and letting go in those feelings.}}\\
\end{addmargin}
\end{absolutelynopagebreak}

\begin{absolutelynopagebreak}
\setstretch{.7}
{\PaliGlossA{so tāsu vedanāsu aniccānupassī viharanto virāgānupassī viharanto nirodhānupassī viharanto paṭinissaggānupassī viharanto na kiñci loke upādiyati,}}\\
\begin{addmargin}[1em]{2em}
\setstretch{.5}
{\PaliGlossB{Meditating in this way, they don’t grasp at anything in the world.}}\\
\end{addmargin}
\end{absolutelynopagebreak}

\begin{absolutelynopagebreak}
\setstretch{.7}
{\PaliGlossA{anupādiyaṃ na paritassati, aparitassaṃ paccattaṃyeva parinibbāyati.}}\\
\begin{addmargin}[1em]{2em}
\setstretch{.5}
{\PaliGlossB{Not grasping, they’re not anxious. Not being anxious, they personally become extinguished.}}\\
\end{addmargin}
\end{absolutelynopagebreak}

\begin{absolutelynopagebreak}
\setstretch{.7}
{\PaliGlossA{‘khīṇā jāti, vusitaṃ brahmacariyaṃ, kataṃ karaṇīyaṃ, nāparaṃ itthattāyā’ti pajānāti.}}\\
\begin{addmargin}[1em]{2em}
\setstretch{.5}
{\PaliGlossB{They understand: ‘Rebirth is ended, the spiritual journey has been completed, what had to be done has been done, there is no return to any state of existence.’}}\\
\end{addmargin}
\end{absolutelynopagebreak}

\begin{absolutelynopagebreak}
\setstretch{.7}
{\PaliGlossA{ettāvatā kho, moggallāna, bhikkhu saṃkhittena taṇhāsaṅkhayavimutto hoti accantaniṭṭho accantayogakkhemī accantabrahmacārī accantapariyosāno seṭṭho devamanussānan”ti.}}\\
\begin{addmargin}[1em]{2em}
\setstretch{.5}
{\PaliGlossB{That’s how I briefly define a mendicant who is freed through the ending of craving, who has reached the ultimate end, the ultimate sanctuary, the ultimate spiritual life, the ultimate goal, and is best among gods and humans.”}}\\
\end{addmargin}
\end{absolutelynopagebreak}

\begin{absolutelynopagebreak}
\setstretch{.7}
{\PaliGlossA{aṭṭhamaṃ.}}\\
\begin{addmargin}[1em]{2em}
\setstretch{.5}
{\PaliGlossB{    -}}\\
\end{addmargin}
\end{absolutelynopagebreak}
