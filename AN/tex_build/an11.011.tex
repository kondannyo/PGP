
\begin{absolutelynopagebreak}
\setstretch{.7}
{\PaliGlossA{aṅguttara nikāya 11}}\\
\begin{addmargin}[1em]{2em}
\setstretch{.5}
{\PaliGlossB{Numbered Discourses 11}}\\
\end{addmargin}
\end{absolutelynopagebreak}

\begin{absolutelynopagebreak}
\setstretch{.7}
{\PaliGlossA{2. anussativagga}}\\
\begin{addmargin}[1em]{2em}
\setstretch{.5}
{\PaliGlossB{2. Recollection}}\\
\end{addmargin}
\end{absolutelynopagebreak}

\begin{absolutelynopagebreak}
\setstretch{.7}
{\PaliGlossA{11. paṭhamamahānāmasutta}}\\
\begin{addmargin}[1em]{2em}
\setstretch{.5}
{\PaliGlossB{11. With Mahānāma (1st)}}\\
\end{addmargin}
\end{absolutelynopagebreak}

\begin{absolutelynopagebreak}
\setstretch{.7}
{\PaliGlossA{ekaṃ samayaṃ bhagavā sakkesu viharati kapilavatthusmiṃ nigrodhārāme.}}\\
\begin{addmargin}[1em]{2em}
\setstretch{.5}
{\PaliGlossB{At one time the Buddha was staying in the land of the Sakyans, near Kapilavatthu in the Banyan Tree Monastery.}}\\
\end{addmargin}
\end{absolutelynopagebreak}

\begin{absolutelynopagebreak}
\setstretch{.7}
{\PaliGlossA{tena kho pana samayena sambahulā bhikkhū bhagavato cīvarakammaṃ karonti:}}\\
\begin{addmargin}[1em]{2em}
\setstretch{.5}
{\PaliGlossB{At that time several mendicants were making a robe for the Buddha, thinking that}}\\
\end{addmargin}
\end{absolutelynopagebreak}

\begin{absolutelynopagebreak}
\setstretch{.7}
{\PaliGlossA{“niṭṭhitacīvaro bhagavā temāsaccayena cārikaṃ pakkamissatī”ti.}}\\
\begin{addmargin}[1em]{2em}
\setstretch{.5}
{\PaliGlossB{when his robe was finished and the three months of the rains residence had passed the Buddha would set out wandering.}}\\
\end{addmargin}
\end{absolutelynopagebreak}

\begin{absolutelynopagebreak}
\setstretch{.7}
{\PaliGlossA{assosi kho mahānāmo sakko:}}\\
\begin{addmargin}[1em]{2em}
\setstretch{.5}
{\PaliGlossB{Mahānāma the Sakyan heard about this.}}\\
\end{addmargin}
\end{absolutelynopagebreak}

\begin{absolutelynopagebreak}
\setstretch{.7}
{\PaliGlossA{“sambahulā kira bhikkhū bhagavato cīvarakammaṃ karonti:}}\\
\begin{addmargin}[1em]{2em}
\setstretch{.5}
{\PaliGlossB{    -}}\\
\end{addmargin}
\end{absolutelynopagebreak}

\begin{absolutelynopagebreak}
\setstretch{.7}
{\PaliGlossA{‘niṭṭhitacīvaro bhagavā temāsaccayena cārikaṃ pakkamissatī’”ti.}}\\
\begin{addmargin}[1em]{2em}
\setstretch{.5}
{\PaliGlossB{    -}}\\
\end{addmargin}
\end{absolutelynopagebreak}

\begin{absolutelynopagebreak}
\setstretch{.7}
{\PaliGlossA{atha kho mahānāmo sakko yena bhagavā tenupasaṅkami; upasaṅkamitvā bhagavantaṃ abhivādetvā ekamantaṃ nisīdi. ekamantaṃ nisinno kho mahānāmo sakko bhagavantaṃ etadavoca:}}\\
\begin{addmargin}[1em]{2em}
\setstretch{.5}
{\PaliGlossB{He went up to the Buddha, bowed, sat down to one side, and said to him:}}\\
\end{addmargin}
\end{absolutelynopagebreak}

\begin{absolutelynopagebreak}
\setstretch{.7}
{\PaliGlossA{“sutaṃ metaṃ, bhante:}}\\
\begin{addmargin}[1em]{2em}
\setstretch{.5}
{\PaliGlossB{“Sir, I have heard that}}\\
\end{addmargin}
\end{absolutelynopagebreak}

\begin{absolutelynopagebreak}
\setstretch{.7}
{\PaliGlossA{‘sambahulā kira bhikkhū bhagavato cīvarakammaṃ karonti—}}\\
\begin{addmargin}[1em]{2em}
\setstretch{.5}
{\PaliGlossB{several mendicants are making a robe for the Buddha, thinking that}}\\
\end{addmargin}
\end{absolutelynopagebreak}

\begin{absolutelynopagebreak}
\setstretch{.7}
{\PaliGlossA{niṭṭhitacīvaro bhagavā temāsaccayena cārikaṃ pakkamissatī’ti.}}\\
\begin{addmargin}[1em]{2em}
\setstretch{.5}
{\PaliGlossB{when his robe was finished and the three months of the rains residence had passed the Buddha would set out wandering.}}\\
\end{addmargin}
\end{absolutelynopagebreak}

\begin{absolutelynopagebreak}
\setstretch{.7}
{\PaliGlossA{tesaṃ no, bhante, nānāvihārehi viharataṃ kenassa vihārena vihātabban”ti?}}\\
\begin{addmargin}[1em]{2em}
\setstretch{.5}
{\PaliGlossB{Now, we spend our life in various ways. Which of these should we practice?”}}\\
\end{addmargin}
\end{absolutelynopagebreak}

\begin{absolutelynopagebreak}
\setstretch{.7}
{\PaliGlossA{“sādhu sādhu, mahānāma.}}\\
\begin{addmargin}[1em]{2em}
\setstretch{.5}
{\PaliGlossB{“Good, good, Mahānāma!}}\\
\end{addmargin}
\end{absolutelynopagebreak}

\begin{absolutelynopagebreak}
\setstretch{.7}
{\PaliGlossA{etaṃ kho, mahānāma, tumhākaṃ patirūpaṃ kulaputtānaṃ, yaṃ tumhe tathāgataṃ upasaṅkamitvā puccheyyātha:}}\\
\begin{addmargin}[1em]{2em}
\setstretch{.5}
{\PaliGlossB{It’s appropriate that gentlemen such as you come to me and ask:}}\\
\end{addmargin}
\end{absolutelynopagebreak}

\begin{absolutelynopagebreak}
\setstretch{.7}
{\PaliGlossA{‘tesaṃ no, bhante, nānāvihārehi viharataṃ kenassa vihārena vihātabban’ti?}}\\
\begin{addmargin}[1em]{2em}
\setstretch{.5}
{\PaliGlossB{‘We spend our life in various ways. Which of these should we practice?’}}\\
\end{addmargin}
\end{absolutelynopagebreak}

\begin{absolutelynopagebreak}
\setstretch{.7}
{\PaliGlossA{saddho kho, mahānāma, ārādhako hoti, no assaddho;}}\\
\begin{addmargin}[1em]{2em}
\setstretch{.5}
{\PaliGlossB{The faithful succeed, not the faithless.}}\\
\end{addmargin}
\end{absolutelynopagebreak}

\begin{absolutelynopagebreak}
\setstretch{.7}
{\PaliGlossA{āraddhavīriyo ārādhako hoti, no kusīto;}}\\
\begin{addmargin}[1em]{2em}
\setstretch{.5}
{\PaliGlossB{The energetic succeed, not the lazy.}}\\
\end{addmargin}
\end{absolutelynopagebreak}

\begin{absolutelynopagebreak}
\setstretch{.7}
{\PaliGlossA{upaṭṭhitassati ārādhako hoti, no muṭṭhassati;}}\\
\begin{addmargin}[1em]{2em}
\setstretch{.5}
{\PaliGlossB{The mindful succeed, not the unmindful.}}\\
\end{addmargin}
\end{absolutelynopagebreak}

\begin{absolutelynopagebreak}
\setstretch{.7}
{\PaliGlossA{samāhito ārādhako hoti, no asamāhito;}}\\
\begin{addmargin}[1em]{2em}
\setstretch{.5}
{\PaliGlossB{Those with immersion succeed, not those without immersion.}}\\
\end{addmargin}
\end{absolutelynopagebreak}

\begin{absolutelynopagebreak}
\setstretch{.7}
{\PaliGlossA{paññavā ārādhako hoti, no duppañño.}}\\
\begin{addmargin}[1em]{2em}
\setstretch{.5}
{\PaliGlossB{The wise succeed, not the witless.}}\\
\end{addmargin}
\end{absolutelynopagebreak}

\begin{absolutelynopagebreak}
\setstretch{.7}
{\PaliGlossA{imesu kho tvaṃ, mahānāma, pañcasu dhammesu patiṭṭhāya cha dhamme uttari bhāveyyāsi.}}\\
\begin{addmargin}[1em]{2em}
\setstretch{.5}
{\PaliGlossB{When you’re grounded on these five things, go on to develop six further things.}}\\
\end{addmargin}
\end{absolutelynopagebreak}

\begin{absolutelynopagebreak}
\setstretch{.7}
{\PaliGlossA{idha tvaṃ, mahānāma, tathāgataṃ anussareyyāsi:}}\\
\begin{addmargin}[1em]{2em}
\setstretch{.5}
{\PaliGlossB{Firstly, you should recollect the Realized One:}}\\
\end{addmargin}
\end{absolutelynopagebreak}

\begin{absolutelynopagebreak}
\setstretch{.7}
{\PaliGlossA{‘itipi so bhagavā arahaṃ sammāsambuddho vijjācaraṇasampanno sugato lokavidū anuttaro purisadammasārathi satthā devamanussānaṃ buddho bhagavā’ti.}}\\
\begin{addmargin}[1em]{2em}
\setstretch{.5}
{\PaliGlossB{‘That Blessed One is perfected, a fully awakened Buddha, accomplished in knowledge and conduct, holy, knower of the world, supreme guide for those who wish to train, teacher of gods and humans, awakened, blessed.’}}\\
\end{addmargin}
\end{absolutelynopagebreak}

\begin{absolutelynopagebreak}
\setstretch{.7}
{\PaliGlossA{yasmiṃ, mahānāma, samaye ariyasāvako tathāgataṃ anussarati, nevassa tasmiṃ samaye rāgapariyuṭṭhitaṃ cittaṃ hoti, na dosapariyuṭṭhitaṃ cittaṃ hoti, na mohapariyuṭṭhitaṃ cittaṃ hoti;}}\\
\begin{addmargin}[1em]{2em}
\setstretch{.5}
{\PaliGlossB{When a noble disciple recollects the Realized One their mind is not full of greed, hate, and delusion.}}\\
\end{addmargin}
\end{absolutelynopagebreak}

\begin{absolutelynopagebreak}
\setstretch{.7}
{\PaliGlossA{ujugatamevassa tasmiṃ samaye cittaṃ hoti tathāgataṃ ārabbha.}}\\
\begin{addmargin}[1em]{2em}
\setstretch{.5}
{\PaliGlossB{At that time their mind is unswerving, based on the Realized One.}}\\
\end{addmargin}
\end{absolutelynopagebreak}

\begin{absolutelynopagebreak}
\setstretch{.7}
{\PaliGlossA{ujugatacitto kho pana, mahānāma, ariyasāvako labhati atthavedaṃ, labhati dhammavedaṃ, labhati dhammūpasaṃhitaṃ pāmojjaṃ.}}\\
\begin{addmargin}[1em]{2em}
\setstretch{.5}
{\PaliGlossB{A noble disciple whose mind is unswerving finds joy in the meaning and the teaching, and finds joy connected with the teaching.}}\\
\end{addmargin}
\end{absolutelynopagebreak}

\begin{absolutelynopagebreak}
\setstretch{.7}
{\PaliGlossA{pamuditassa pīti jāyati, pītimanassa kāyo passambhati, passaddhakāyo sukhaṃ vediyati, sukhino cittaṃ samādhiyati.}}\\
\begin{addmargin}[1em]{2em}
\setstretch{.5}
{\PaliGlossB{When they’re joyful, rapture springs up. When the mind is full of rapture, the body becomes tranquil. When the body is tranquil, they feel bliss. And when they’re blissful, the mind becomes immersed in samādhi.}}\\
\end{addmargin}
\end{absolutelynopagebreak}

\begin{absolutelynopagebreak}
\setstretch{.7}
{\PaliGlossA{ayaṃ vuccati, mahānāma, ariyasāvako visamagatāya pajāya samappatto viharati, sabyāpajjāya pajāya abyāpajjo viharati, dhammasotasamāpanno buddhānussatiṃ bhāveti.}}\\
\begin{addmargin}[1em]{2em}
\setstretch{.5}
{\PaliGlossB{This is called a noble disciple who lives in balance among people who are unbalanced, and lives untroubled among people who are troubled. They’ve entered the stream of the teaching and developed the recollection of the Buddha.}}\\
\end{addmargin}
\end{absolutelynopagebreak}

\begin{absolutelynopagebreak}
\setstretch{.7}
{\PaliGlossA{puna caparaṃ tvaṃ, mahānāma, dhammaṃ anussareyyāsi:}}\\
\begin{addmargin}[1em]{2em}
\setstretch{.5}
{\PaliGlossB{Furthermore, you should recollect the teaching:}}\\
\end{addmargin}
\end{absolutelynopagebreak}

\begin{absolutelynopagebreak}
\setstretch{.7}
{\PaliGlossA{‘svākkhāto bhagavatā dhammo sandiṭṭhiko akāliko ehipassiko opaneyyiko paccattaṃ veditabbo viññūhī’ti.}}\\
\begin{addmargin}[1em]{2em}
\setstretch{.5}
{\PaliGlossB{‘The teaching is well explained by the Buddha—visible in this very life, immediately effective, inviting inspection, relevant, so that sensible people can know it for themselves.’}}\\
\end{addmargin}
\end{absolutelynopagebreak}

\begin{absolutelynopagebreak}
\setstretch{.7}
{\PaliGlossA{yasmiṃ, mahānāma, samaye ariyasāvako dhammaṃ anussarati, nevassa tasmiṃ samaye rāgapariyuṭṭhitaṃ cittaṃ hoti, na dosapariyuṭṭhitaṃ cittaṃ hoti, na mohapariyuṭṭhitaṃ cittaṃ hoti;}}\\
\begin{addmargin}[1em]{2em}
\setstretch{.5}
{\PaliGlossB{When a noble disciple recollects the teaching their mind is not full of greed, hate, and delusion. …}}\\
\end{addmargin}
\end{absolutelynopagebreak}

\begin{absolutelynopagebreak}
\setstretch{.7}
{\PaliGlossA{ujugatamevassa tasmiṃ samaye cittaṃ hoti dhammaṃ ārabbha.}}\\
\begin{addmargin}[1em]{2em}
\setstretch{.5}
{\PaliGlossB{    -}}\\
\end{addmargin}
\end{absolutelynopagebreak}

\begin{absolutelynopagebreak}
\setstretch{.7}
{\PaliGlossA{ujugatacitto kho pana, mahānāma, ariyasāvako labhati atthavedaṃ, labhati dhammavedaṃ, labhati dhammūpasaṃhitaṃ pāmojjaṃ.}}\\
\begin{addmargin}[1em]{2em}
\setstretch{.5}
{\PaliGlossB{    -}}\\
\end{addmargin}
\end{absolutelynopagebreak}

\begin{absolutelynopagebreak}
\setstretch{.7}
{\PaliGlossA{pamuditassa pīti jāyati, pītimanassa kāyo passambhati, passaddhakāyo sukhaṃ vediyati, sukhino cittaṃ samādhiyati.}}\\
\begin{addmargin}[1em]{2em}
\setstretch{.5}
{\PaliGlossB{    -}}\\
\end{addmargin}
\end{absolutelynopagebreak}

\begin{absolutelynopagebreak}
\setstretch{.7}
{\PaliGlossA{ayaṃ vuccati, mahānāma, ariyasāvako visamagatāya pajāya samappatto viharati, sabyāpajjāya pajāya abyāpajjo viharati, dhammasotasamāpanno dhammānussatiṃ bhāveti.}}\\
\begin{addmargin}[1em]{2em}
\setstretch{.5}
{\PaliGlossB{This is called a noble disciple who lives in balance among people who are unbalanced, and lives untroubled among people who are troubled. They’ve entered the stream of the teaching and developed the recollection of the teaching.}}\\
\end{addmargin}
\end{absolutelynopagebreak}

\begin{absolutelynopagebreak}
\setstretch{.7}
{\PaliGlossA{puna caparaṃ tvaṃ, mahānāma, saṅghaṃ anussareyyāsi:}}\\
\begin{addmargin}[1em]{2em}
\setstretch{.5}
{\PaliGlossB{Furthermore, you should recollect the Saṅgha:}}\\
\end{addmargin}
\end{absolutelynopagebreak}

\begin{absolutelynopagebreak}
\setstretch{.7}
{\PaliGlossA{‘suppaṭipanno bhagavato sāvakasaṅgho, ujuppaṭipanno bhagavato sāvakasaṅgho, ñāyappaṭipanno bhagavato sāvakasaṅgho, sāmīcippaṭipanno bhagavato sāvakasaṅgho, yadidaṃ cattāri purisayugāni aṭṭha purisapuggalā, esa bhagavato sāvakasaṅgho āhuneyyo pāhuneyyo dakkhiṇeyyo añjalikaraṇīyo anuttaraṃ puññakkhettaṃ lokassā’ti.}}\\
\begin{addmargin}[1em]{2em}
\setstretch{.5}
{\PaliGlossB{‘The Saṅgha of the Buddha’s disciples is practicing the way that’s good, straightforward, methodical, and proper. It consists of the four pairs, the eight individuals. This is the Saṅgha of the Buddha’s disciples that is worthy of offerings dedicated to the gods, worthy of hospitality, worthy of a religious donation, worthy of greeting with joined palms, and is the supreme field of merit for the world.’}}\\
\end{addmargin}
\end{absolutelynopagebreak}

\begin{absolutelynopagebreak}
\setstretch{.7}
{\PaliGlossA{yasmiṃ, mahānāma, samaye ariyasāvako saṅghaṃ anussarati, nevassa tasmiṃ samaye rāgapariyuṭṭhitaṃ cittaṃ hoti, na dosapariyuṭṭhitaṃ cittaṃ hoti, na mohapariyuṭṭhitaṃ cittaṃ hoti;}}\\
\begin{addmargin}[1em]{2em}
\setstretch{.5}
{\PaliGlossB{When a noble disciple recollects the Saṅgha their mind is not full of greed, hate, and delusion. …}}\\
\end{addmargin}
\end{absolutelynopagebreak}

\begin{absolutelynopagebreak}
\setstretch{.7}
{\PaliGlossA{ujugatamevassa tasmiṃ samaye cittaṃ hoti saṅghaṃ ārabbha.}}\\
\begin{addmargin}[1em]{2em}
\setstretch{.5}
{\PaliGlossB{    -}}\\
\end{addmargin}
\end{absolutelynopagebreak}

\begin{absolutelynopagebreak}
\setstretch{.7}
{\PaliGlossA{ujugatacitto kho pana, mahānāma, ariyasāvako labhati atthavedaṃ, labhati dhammavedaṃ, labhati dhammūpasaṃhitaṃ pāmojjaṃ.}}\\
\begin{addmargin}[1em]{2em}
\setstretch{.5}
{\PaliGlossB{    -}}\\
\end{addmargin}
\end{absolutelynopagebreak}

\begin{absolutelynopagebreak}
\setstretch{.7}
{\PaliGlossA{pamuditassa pīti jāyati, pītimanassa kāyo passambhati, passaddhakāyo sukhaṃ vediyati, sukhino cittaṃ samādhiyati.}}\\
\begin{addmargin}[1em]{2em}
\setstretch{.5}
{\PaliGlossB{    -}}\\
\end{addmargin}
\end{absolutelynopagebreak}

\begin{absolutelynopagebreak}
\setstretch{.7}
{\PaliGlossA{ayaṃ vuccati, mahānāma, ariyasāvako visamagatāya pajāya samappatto viharati, sabyāpajjāya pajāya abyāpajjo viharati, dhammasotasamāpanno saṅghānussatiṃ bhāveti.}}\\
\begin{addmargin}[1em]{2em}
\setstretch{.5}
{\PaliGlossB{This is called a noble disciple who lives in balance among people who are unbalanced, and lives untroubled among people who are troubled. They’ve entered the stream of the teaching and developed the recollection of the Saṅgha.}}\\
\end{addmargin}
\end{absolutelynopagebreak}

\begin{absolutelynopagebreak}
\setstretch{.7}
{\PaliGlossA{puna caparaṃ tvaṃ, mahānāma, attano sīlāni anussareyyāsi akhaṇḍāni acchiddāni asabalāni akammāsāni bhujissāni viññuppasatthāni aparāmaṭṭhāni samādhisaṃvattanikāni.}}\\
\begin{addmargin}[1em]{2em}
\setstretch{.5}
{\PaliGlossB{Furthermore, a noble disciple recollects their own ethical conduct, which is unbroken, impeccable, spotless, and unmarred, liberating, praised by sensible people, not mistaken, and leading to immersion.}}\\
\end{addmargin}
\end{absolutelynopagebreak}

\begin{absolutelynopagebreak}
\setstretch{.7}
{\PaliGlossA{yasmiṃ, mahānāma, samaye ariyasāvako sīlaṃ anussarati, nevassa tasmiṃ samaye rāgapariyuṭṭhitaṃ cittaṃ hoti, na dosapariyuṭṭhitaṃ cittaṃ hoti, na mohapariyuṭṭhitaṃ cittaṃ hoti;}}\\
\begin{addmargin}[1em]{2em}
\setstretch{.5}
{\PaliGlossB{When a noble disciple recollects their ethical conduct their mind is not full of greed, hate, and delusion. …}}\\
\end{addmargin}
\end{absolutelynopagebreak}

\begin{absolutelynopagebreak}
\setstretch{.7}
{\PaliGlossA{ujugatamevassa tasmiṃ samaye cittaṃ hoti sīlaṃ ārabbha.}}\\
\begin{addmargin}[1em]{2em}
\setstretch{.5}
{\PaliGlossB{    -}}\\
\end{addmargin}
\end{absolutelynopagebreak}

\begin{absolutelynopagebreak}
\setstretch{.7}
{\PaliGlossA{ujugatacitto kho pana, mahānāma, ariyasāvako labhati atthavedaṃ, labhati dhammavedaṃ, labhati dhammūpasaṃhitaṃ pāmojjaṃ.}}\\
\begin{addmargin}[1em]{2em}
\setstretch{.5}
{\PaliGlossB{    -}}\\
\end{addmargin}
\end{absolutelynopagebreak}

\begin{absolutelynopagebreak}
\setstretch{.7}
{\PaliGlossA{pamuditassa pīti jāyati, pītimanassa kāyo passambhati, passaddhakāyo sukhaṃ vediyati, sukhino cittaṃ samādhiyati.}}\\
\begin{addmargin}[1em]{2em}
\setstretch{.5}
{\PaliGlossB{    -}}\\
\end{addmargin}
\end{absolutelynopagebreak}

\begin{absolutelynopagebreak}
\setstretch{.7}
{\PaliGlossA{ayaṃ vuccati, mahānāma, ariyasāvako visamagatāya pajāya samappatto viharati, sabyāpajjāya pajāya abyāpajjo viharati, dhammasotasamāpanno sīlānussatiṃ bhāveti.}}\\
\begin{addmargin}[1em]{2em}
\setstretch{.5}
{\PaliGlossB{This is called a noble disciple who lives in balance among people who are unbalanced, and lives untroubled among people who are troubled. They’ve entered the stream of the teaching and developed the recollection of their ethical conduct.}}\\
\end{addmargin}
\end{absolutelynopagebreak}

\begin{absolutelynopagebreak}
\setstretch{.7}
{\PaliGlossA{puna caparaṃ tvaṃ, mahānāma, attano cāgaṃ anussareyyāsi:}}\\
\begin{addmargin}[1em]{2em}
\setstretch{.5}
{\PaliGlossB{Furthermore, you should recollect your own generosity:}}\\
\end{addmargin}
\end{absolutelynopagebreak}

\begin{absolutelynopagebreak}
\setstretch{.7}
{\PaliGlossA{‘lābhā vata me, suladdhaṃ vata me,}}\\
\begin{addmargin}[1em]{2em}
\setstretch{.5}
{\PaliGlossB{‘I’m so fortunate, so very fortunate.}}\\
\end{addmargin}
\end{absolutelynopagebreak}

\begin{absolutelynopagebreak}
\setstretch{.7}
{\PaliGlossA{yohaṃ maccheramalapariyuṭṭhitāya pajāya vigatamalamaccherena cetasā agāraṃ ajjhāvasāmi muttacāgo payatapāṇi vossaggarato yācayogo dānasaṃvibhāgarato’ti.}}\\
\begin{addmargin}[1em]{2em}
\setstretch{.5}
{\PaliGlossB{Among people with hearts full of the stain of stinginess I live at home rid of stinginess, freely generous, open-handed, loving to let go, committed to charity, loving to give and to share.’}}\\
\end{addmargin}
\end{absolutelynopagebreak}

\begin{absolutelynopagebreak}
\setstretch{.7}
{\PaliGlossA{yasmiṃ, mahānāma, samaye ariyasāvako cāgaṃ anussarati, nevassa tasmiṃ samaye rāgapariyuṭṭhitaṃ cittaṃ hoti, na dosapariyuṭṭhitaṃ cittaṃ hoti, na mohapariyuṭṭhitaṃ cittaṃ hoti;}}\\
\begin{addmargin}[1em]{2em}
\setstretch{.5}
{\PaliGlossB{When a noble disciple recollects their own generosity their mind is not full of greed, hate, and delusion. …}}\\
\end{addmargin}
\end{absolutelynopagebreak}

\begin{absolutelynopagebreak}
\setstretch{.7}
{\PaliGlossA{ujugatamevassa tasmiṃ samaye cittaṃ hoti cāgaṃ ārabbha.}}\\
\begin{addmargin}[1em]{2em}
\setstretch{.5}
{\PaliGlossB{    -}}\\
\end{addmargin}
\end{absolutelynopagebreak}

\begin{absolutelynopagebreak}
\setstretch{.7}
{\PaliGlossA{ujugatacitto kho pana, mahānāma, ariyasāvako labhati atthavedaṃ, labhati dhammavedaṃ, labhati dhammūpasaṃhitaṃ pāmojjaṃ.}}\\
\begin{addmargin}[1em]{2em}
\setstretch{.5}
{\PaliGlossB{    -}}\\
\end{addmargin}
\end{absolutelynopagebreak}

\begin{absolutelynopagebreak}
\setstretch{.7}
{\PaliGlossA{pamuditassa pīti jāyati, pītimanassa kāyo passambhati, passaddhakāyo sukhaṃ vediyati, sukhino cittaṃ samādhiyati.}}\\
\begin{addmargin}[1em]{2em}
\setstretch{.5}
{\PaliGlossB{    -}}\\
\end{addmargin}
\end{absolutelynopagebreak}

\begin{absolutelynopagebreak}
\setstretch{.7}
{\PaliGlossA{ayaṃ vuccati, mahānāma, ariyasāvako visamagatāya pajāya samappatto viharati, sabyāpajjāya pajāya abyāpajjo viharati, dhammasotasamāpanno cāgānussatiṃ bhāveti.}}\\
\begin{addmargin}[1em]{2em}
\setstretch{.5}
{\PaliGlossB{This is called a noble disciple who lives in balance among people who are unbalanced, and lives untroubled among people who are troubled. They’ve entered the stream of the teaching and developed the recollection of generosity.}}\\
\end{addmargin}
\end{absolutelynopagebreak}

\begin{absolutelynopagebreak}
\setstretch{.7}
{\PaliGlossA{puna caparaṃ tvaṃ, mahānāma, devatā anussareyyāsi:}}\\
\begin{addmargin}[1em]{2em}
\setstretch{.5}
{\PaliGlossB{Furthermore, you should recollect the deities:}}\\
\end{addmargin}
\end{absolutelynopagebreak}

\begin{absolutelynopagebreak}
\setstretch{.7}
{\PaliGlossA{‘santi devā cātumahārājikā, santi devā tāvatiṃsā, santi devā yāmā, santi devā tusitā, santi devā nimmānaratino, santi devā paranimmitavasavattino, santi devā brahmakāyikā, santi devā tatuttari.}}\\
\begin{addmargin}[1em]{2em}
\setstretch{.5}
{\PaliGlossB{‘There are the Gods of the Four Great Kings, the Gods of the Thirty-Three, the Gods of Yama, the Joyful Gods, the Gods Who Love to Create, the Gods Who Control the Creations of Others, the Gods of Brahmā’s Host, and gods even higher than these.}}\\
\end{addmargin}
\end{absolutelynopagebreak}

\begin{absolutelynopagebreak}
\setstretch{.7}
{\PaliGlossA{yathārūpāya saddhāya samannāgatā tā devatā ito cutā tatthūpapannā, mayhampi tathārūpā saddhā saṃvijjati.}}\\
\begin{addmargin}[1em]{2em}
\setstretch{.5}
{\PaliGlossB{When those deities passed away from here, they were reborn there because of their faith, ethics, learning, generosity, and wisdom. I, too, have the same kind of faith, ethics, learning, generosity, and wisdom.’}}\\
\end{addmargin}
\end{absolutelynopagebreak}

\begin{absolutelynopagebreak}
\setstretch{.7}
{\PaliGlossA{yathārūpena sīlena samannāgatā tā devatā ito cutā tatthūpapannā, mayhampi tathārūpaṃ sīlaṃ saṃvijjati.}}\\
\begin{addmargin}[1em]{2em}
\setstretch{.5}
{\PaliGlossB{    -}}\\
\end{addmargin}
\end{absolutelynopagebreak}

\begin{absolutelynopagebreak}
\setstretch{.7}
{\PaliGlossA{yathārūpena sutena samannāgatā tā devatā ito cutā tatthūpapannā, mayhampi tathārūpaṃ sutaṃ saṃvijjati.}}\\
\begin{addmargin}[1em]{2em}
\setstretch{.5}
{\PaliGlossB{    -}}\\
\end{addmargin}
\end{absolutelynopagebreak}

\begin{absolutelynopagebreak}
\setstretch{.7}
{\PaliGlossA{yathārūpena cāgena samannāgatā tā devatā ito cutā tatthūpapannā, mayhampi tathārūpo cāgo saṃvijjati.}}\\
\begin{addmargin}[1em]{2em}
\setstretch{.5}
{\PaliGlossB{    -}}\\
\end{addmargin}
\end{absolutelynopagebreak}

\begin{absolutelynopagebreak}
\setstretch{.7}
{\PaliGlossA{yathārūpāya paññāya samannāgatā tā devatā ito cutā tatthūpapannā, mayhampi tathārūpā paññā saṃvijjatī’ti.}}\\
\begin{addmargin}[1em]{2em}
\setstretch{.5}
{\PaliGlossB{    -}}\\
\end{addmargin}
\end{absolutelynopagebreak}

\begin{absolutelynopagebreak}
\setstretch{.7}
{\PaliGlossA{yasmiṃ, mahānāma, samaye ariyasāvako attano ca tāsañca devatānaṃ saddhañca sīlañca sutañca cāgañca paññañca anussarati, nevassa tasmiṃ samaye rāgapariyuṭṭhitaṃ cittaṃ hoti, na dosapariyuṭṭhitaṃ cittaṃ hoti, na mohapariyuṭṭhitaṃ cittaṃ hoti;}}\\
\begin{addmargin}[1em]{2em}
\setstretch{.5}
{\PaliGlossB{When a noble disciple recollects the faith, ethics, learning, generosity, and wisdom of both themselves and the deities their mind is not full of greed, hate, and delusion.}}\\
\end{addmargin}
\end{absolutelynopagebreak}

\begin{absolutelynopagebreak}
\setstretch{.7}
{\PaliGlossA{ujugatamevassa tasmiṃ samaye cittaṃ hoti devatā ārabbha.}}\\
\begin{addmargin}[1em]{2em}
\setstretch{.5}
{\PaliGlossB{At that time their mind is unswerving, based on the deities.}}\\
\end{addmargin}
\end{absolutelynopagebreak}

\begin{absolutelynopagebreak}
\setstretch{.7}
{\PaliGlossA{ujugatacitto kho pana, mahānāma, ariyasāvako labhati atthavedaṃ, labhati dhammavedaṃ, labhati dhammūpasaṃhitaṃ pāmojjaṃ.}}\\
\begin{addmargin}[1em]{2em}
\setstretch{.5}
{\PaliGlossB{A noble disciple whose mind is unswerving finds joy in the meaning and the teaching, and finds joy connected with the teaching.}}\\
\end{addmargin}
\end{absolutelynopagebreak}

\begin{absolutelynopagebreak}
\setstretch{.7}
{\PaliGlossA{pamuditassa pīti jāyati, pītimanassa kāyo passambhati, passaddhakāyo sukhaṃ vediyati, sukhino cittaṃ samādhiyati.}}\\
\begin{addmargin}[1em]{2em}
\setstretch{.5}
{\PaliGlossB{When they’re joyful, rapture springs up. When the mind is full of rapture, the body becomes tranquil. When the body is tranquil, they feel bliss. And when they’re blissful, the mind becomes immersed in samādhi.}}\\
\end{addmargin}
\end{absolutelynopagebreak}

\begin{absolutelynopagebreak}
\setstretch{.7}
{\PaliGlossA{ayaṃ vuccati, mahānāma, ariyasāvako visamagatāya pajāya samappatto viharati, sabyāpajjāya pajāya abyāpajjo viharati, dhammasotasamāpanno devatānussatiṃ bhāvetī”ti.}}\\
\begin{addmargin}[1em]{2em}
\setstretch{.5}
{\PaliGlossB{This is called a noble disciple who lives in balance among people who are unbalanced, and lives untroubled among people who are troubled. They’ve entered the stream of the teaching and developed the recollection of the deities.”}}\\
\end{addmargin}
\end{absolutelynopagebreak}

\begin{absolutelynopagebreak}
\setstretch{.7}
{\PaliGlossA{paṭhamaṃ.}}\\
\begin{addmargin}[1em]{2em}
\setstretch{.5}
{\PaliGlossB{    -}}\\
\end{addmargin}
\end{absolutelynopagebreak}
