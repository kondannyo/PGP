
\begin{absolutelynopagebreak}
\setstretch{.7}
{\PaliGlossA{aṅguttara nikāya 8}}\\
\begin{addmargin}[1em]{2em}
\setstretch{.5}
{\PaliGlossB{Numbered Discourses 8}}\\
\end{addmargin}
\end{absolutelynopagebreak}

\begin{absolutelynopagebreak}
\setstretch{.7}
{\PaliGlossA{1. mettāvagga}}\\
\begin{addmargin}[1em]{2em}
\setstretch{.5}
{\PaliGlossB{1. Love}}\\
\end{addmargin}
\end{absolutelynopagebreak}

\begin{absolutelynopagebreak}
\setstretch{.7}
{\PaliGlossA{3. paṭhamaappiyasutta}}\\
\begin{addmargin}[1em]{2em}
\setstretch{.5}
{\PaliGlossB{3. Disliked (1st)}}\\
\end{addmargin}
\end{absolutelynopagebreak}

\begin{absolutelynopagebreak}
\setstretch{.7}
{\PaliGlossA{“aṭṭhahi, bhikkhave, dhammehi samannāgato bhikkhu sabrahmacārīnaṃ appiyo ca hoti amanāpo ca agaru ca abhāvanīyo ca.}}\\
\begin{addmargin}[1em]{2em}
\setstretch{.5}
{\PaliGlossB{“Mendicants, a mendicant with eight qualities is disliked and disapproved by their spiritual companions, not respected or admired.}}\\
\end{addmargin}
\end{absolutelynopagebreak}

\begin{absolutelynopagebreak}
\setstretch{.7}
{\PaliGlossA{katamehi aṭṭhahi?}}\\
\begin{addmargin}[1em]{2em}
\setstretch{.5}
{\PaliGlossB{What eight?}}\\
\end{addmargin}
\end{absolutelynopagebreak}

\begin{absolutelynopagebreak}
\setstretch{.7}
{\PaliGlossA{idha, bhikkhave, bhikkhu appiyapasaṃsī ca hoti, piyagarahī ca, lābhakāmo ca, sakkārakāmo ca, ahiriko ca, anottappī ca, pāpiccho ca, micchādiṭṭhi ca.}}\\
\begin{addmargin}[1em]{2em}
\setstretch{.5}
{\PaliGlossB{It’s when a mendicant praises the disliked and criticizes the liked. They desire material possessions and honor. They lack conscience and prudence. They have wicked desires and wrong view.}}\\
\end{addmargin}
\end{absolutelynopagebreak}

\begin{absolutelynopagebreak}
\setstretch{.7}
{\PaliGlossA{imehi kho, bhikkhave, aṭṭhahi dhammehi samannāgato bhikkhu sabrahmacārīnaṃ appiyo ca hoti amanāpo ca agaru ca abhāvanīyo ca.}}\\
\begin{addmargin}[1em]{2em}
\setstretch{.5}
{\PaliGlossB{A mendicant with these eight qualities is disliked and disapproved by their spiritual companions, not respected or admired.}}\\
\end{addmargin}
\end{absolutelynopagebreak}

\begin{absolutelynopagebreak}
\setstretch{.7}
{\PaliGlossA{aṭṭhahi, bhikkhave, dhammehi samannāgato bhikkhu sabrahmacārīnaṃ piyo ca hoti manāpo ca garu ca bhāvanīyo ca.}}\\
\begin{addmargin}[1em]{2em}
\setstretch{.5}
{\PaliGlossB{A mendicant with eight qualities is liked and approved by their spiritual companions, and respected and admired.}}\\
\end{addmargin}
\end{absolutelynopagebreak}

\begin{absolutelynopagebreak}
\setstretch{.7}
{\PaliGlossA{katamehi aṭṭhahi?}}\\
\begin{addmargin}[1em]{2em}
\setstretch{.5}
{\PaliGlossB{What eight?}}\\
\end{addmargin}
\end{absolutelynopagebreak}

\begin{absolutelynopagebreak}
\setstretch{.7}
{\PaliGlossA{idha, bhikkhave, bhikkhu na appiyapasaṃsī ca hoti, na piyagarahī ca, na lābhakāmo ca, na sakkārakāmo ca, hirīmā ca hoti, ottappī ca, appiccho ca, sammādiṭṭhi ca.}}\\
\begin{addmargin}[1em]{2em}
\setstretch{.5}
{\PaliGlossB{It’s when a mendicant doesn’t praise the disliked and criticize the liked. They don’t desire material possessions and honor. They have conscience and prudence. They have few desires and right view.}}\\
\end{addmargin}
\end{absolutelynopagebreak}

\begin{absolutelynopagebreak}
\setstretch{.7}
{\PaliGlossA{imehi kho, bhikkhave, aṭṭhahi dhammehi samannāgato bhikkhu sabrahmacārīnaṃ piyo ca hoti manāpo ca garu ca bhāvanīyo cā”ti.}}\\
\begin{addmargin}[1em]{2em}
\setstretch{.5}
{\PaliGlossB{A mendicant with these eight qualities is liked and approved by their spiritual companions, and respected and admired.”}}\\
\end{addmargin}
\end{absolutelynopagebreak}

\begin{absolutelynopagebreak}
\setstretch{.7}
{\PaliGlossA{tatiyaṃ.}}\\
\begin{addmargin}[1em]{2em}
\setstretch{.5}
{\PaliGlossB{    -}}\\
\end{addmargin}
\end{absolutelynopagebreak}
