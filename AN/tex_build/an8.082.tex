
\begin{absolutelynopagebreak}
\setstretch{.7}
{\PaliGlossA{aṅguttara nikāya 8}}\\
\begin{addmargin}[1em]{2em}
\setstretch{.5}
{\PaliGlossB{Numbered Discourses 8}}\\
\end{addmargin}
\end{absolutelynopagebreak}

\begin{absolutelynopagebreak}
\setstretch{.7}
{\PaliGlossA{9. sativagga}}\\
\begin{addmargin}[1em]{2em}
\setstretch{.5}
{\PaliGlossB{9. Mindfulness}}\\
\end{addmargin}
\end{absolutelynopagebreak}

\begin{absolutelynopagebreak}
\setstretch{.7}
{\PaliGlossA{82. puṇṇiyasutta}}\\
\begin{addmargin}[1em]{2em}
\setstretch{.5}
{\PaliGlossB{82. With Puṇṇiya}}\\
\end{addmargin}
\end{absolutelynopagebreak}

\begin{absolutelynopagebreak}
\setstretch{.7}
{\PaliGlossA{atha kho āyasmā puṇṇiyo yena bhagavā tenupasaṅkami; upasaṅkamitvā bhagavantaṃ abhivādetvā ekamantaṃ nisīdi. ekamantaṃ nisinno kho āyasmā puṇṇiyo bhagavantaṃ etadavoca:}}\\
\begin{addmargin}[1em]{2em}
\setstretch{.5}
{\PaliGlossB{Then Venerable Puṇṇiya went up to the Buddha, bowed, sat down to one side, and said to him:}}\\
\end{addmargin}
\end{absolutelynopagebreak}

\begin{absolutelynopagebreak}
\setstretch{.7}
{\PaliGlossA{“ko nu kho, bhante, hetu ko paccayo yena appekadā tathāgataṃ dhammadesanā paṭibhāti, appekadā na paṭibhātī”ti?}}\\
\begin{addmargin}[1em]{2em}
\setstretch{.5}
{\PaliGlossB{“Sir, what is the cause, what is the reason why sometimes the Realized One feels inspired to teach, and other times not?”}}\\
\end{addmargin}
\end{absolutelynopagebreak}

\begin{absolutelynopagebreak}
\setstretch{.7}
{\PaliGlossA{“saddho ca, puṇṇiya, bhikkhu hoti, no cupasaṅkamitā;}}\\
\begin{addmargin}[1em]{2em}
\setstretch{.5}
{\PaliGlossB{“Puṇṇiya, when a mendicant has faith but doesn’t approach,}}\\
\end{addmargin}
\end{absolutelynopagebreak}

\begin{absolutelynopagebreak}
\setstretch{.7}
{\PaliGlossA{neva tathāgataṃ dhammadesanā paṭibhāti.}}\\
\begin{addmargin}[1em]{2em}
\setstretch{.5}
{\PaliGlossB{the Realized One doesn’t feel inspired to teach.}}\\
\end{addmargin}
\end{absolutelynopagebreak}

\begin{absolutelynopagebreak}
\setstretch{.7}
{\PaliGlossA{yato ca kho, puṇṇiya, bhikkhu saddho ca hoti, upasaṅkamitā ca;}}\\
\begin{addmargin}[1em]{2em}
\setstretch{.5}
{\PaliGlossB{But when a mendicant has faith and approaches,}}\\
\end{addmargin}
\end{absolutelynopagebreak}

\begin{absolutelynopagebreak}
\setstretch{.7}
{\PaliGlossA{evaṃ tathāgataṃ dhammadesanā paṭibhāti.}}\\
\begin{addmargin}[1em]{2em}
\setstretch{.5}
{\PaliGlossB{the Realized One feels inspired to teach.}}\\
\end{addmargin}
\end{absolutelynopagebreak}

\begin{absolutelynopagebreak}
\setstretch{.7}
{\PaliGlossA{saddho ca, puṇṇiya, bhikkhu hoti, upasaṅkamitā ca, no ca payirupāsitā … pe …}}\\
\begin{addmargin}[1em]{2em}
\setstretch{.5}
{\PaliGlossB{When a mendicant has faith and approaches, but doesn’t pay homage …}}\\
\end{addmargin}
\end{absolutelynopagebreak}

\begin{absolutelynopagebreak}
\setstretch{.7}
{\PaliGlossA{payirupāsitā ca, no ca paripucchitā …}}\\
\begin{addmargin}[1em]{2em}
\setstretch{.5}
{\PaliGlossB{they pay homage, but don’t ask questions …}}\\
\end{addmargin}
\end{absolutelynopagebreak}

\begin{absolutelynopagebreak}
\setstretch{.7}
{\PaliGlossA{paripucchitā ca, no ca ohitasoto dhammaṃ suṇāti …}}\\
\begin{addmargin}[1em]{2em}
\setstretch{.5}
{\PaliGlossB{they ask questions, but don’t lend an ear …}}\\
\end{addmargin}
\end{absolutelynopagebreak}

\begin{absolutelynopagebreak}
\setstretch{.7}
{\PaliGlossA{ohitasoto ca dhammaṃ suṇāti, no ca sutvā dhammaṃ dhāreti …}}\\
\begin{addmargin}[1em]{2em}
\setstretch{.5}
{\PaliGlossB{they lend an ear, but don’t remember the teaching they’ve heard …}}\\
\end{addmargin}
\end{absolutelynopagebreak}

\begin{absolutelynopagebreak}
\setstretch{.7}
{\PaliGlossA{sutvā ca dhammaṃ dhāreti, no ca dhātānaṃ dhammānaṃ atthaṃ upaparikkhati …}}\\
\begin{addmargin}[1em]{2em}
\setstretch{.5}
{\PaliGlossB{they remember the teaching they’ve heard, but don’t reflect on the meaning of the teachings they’ve remembered …}}\\
\end{addmargin}
\end{absolutelynopagebreak}

\begin{absolutelynopagebreak}
\setstretch{.7}
{\PaliGlossA{dhātānañca dhammānaṃ atthaṃ upaparikkhati, no ca atthamaññāya dhammamaññāya dhammānudhammappaṭipanno hoti.}}\\
\begin{addmargin}[1em]{2em}
\setstretch{.5}
{\PaliGlossB{they reflect on the meaning of the teachings they’ve remembered, but, having understood the meaning and the teaching, they don’t practice accordingly.}}\\
\end{addmargin}
\end{absolutelynopagebreak}

\begin{absolutelynopagebreak}
\setstretch{.7}
{\PaliGlossA{neva tāva tathāgataṃ dhammadesanā paṭibhāti.}}\\
\begin{addmargin}[1em]{2em}
\setstretch{.5}
{\PaliGlossB{The Realized One doesn’t feel inspired to teach.}}\\
\end{addmargin}
\end{absolutelynopagebreak}

\begin{absolutelynopagebreak}
\setstretch{.7}
{\PaliGlossA{yato ca kho, puṇṇiya, bhikkhu saddho ca hoti, upasaṅkamitā ca, payirupāsitā ca, paripucchitā ca, ohitasoto ca dhammaṃ suṇāti, sutvā ca dhammaṃ dhāreti, dhātānañca dhammānaṃ atthaṃ upaparikkhati, atthamaññāya dhammamaññāya dhammānudhammappaṭipanno ca hoti;}}\\
\begin{addmargin}[1em]{2em}
\setstretch{.5}
{\PaliGlossB{But when a mendicant has faith, approaches, pays homage, asks questions, lends an ear, remembers the teachings, reflects on the meaning, and practices accordingly,}}\\
\end{addmargin}
\end{absolutelynopagebreak}

\begin{absolutelynopagebreak}
\setstretch{.7}
{\PaliGlossA{evaṃ tathāgataṃ dhammadesanā paṭibhāti.}}\\
\begin{addmargin}[1em]{2em}
\setstretch{.5}
{\PaliGlossB{the Realized One feels inspired to teach.}}\\
\end{addmargin}
\end{absolutelynopagebreak}

\begin{absolutelynopagebreak}
\setstretch{.7}
{\PaliGlossA{imehi kho, puṇṇiya, aṭṭhahi dhammehi samannāgatā ekantapaṭibhānā tathāgataṃ dhammadesanā hotī”ti.}}\\
\begin{addmargin}[1em]{2em}
\setstretch{.5}
{\PaliGlossB{When someone has these eight qualities, the Realized One feels totally inspired to teach.”}}\\
\end{addmargin}
\end{absolutelynopagebreak}

\begin{absolutelynopagebreak}
\setstretch{.7}
{\PaliGlossA{dutiyaṃ.}}\\
\begin{addmargin}[1em]{2em}
\setstretch{.5}
{\PaliGlossB{    -}}\\
\end{addmargin}
\end{absolutelynopagebreak}
