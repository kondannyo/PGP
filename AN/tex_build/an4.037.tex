
\begin{absolutelynopagebreak}
\setstretch{.7}
{\PaliGlossA{aṅguttara nikāya 4}}\\
\begin{addmargin}[1em]{2em}
\setstretch{.5}
{\PaliGlossB{Numbered Discourses 4}}\\
\end{addmargin}
\end{absolutelynopagebreak}

\begin{absolutelynopagebreak}
\setstretch{.7}
{\PaliGlossA{4. cakkavagga}}\\
\begin{addmargin}[1em]{2em}
\setstretch{.5}
{\PaliGlossB{4. Situations}}\\
\end{addmargin}
\end{absolutelynopagebreak}

\begin{absolutelynopagebreak}
\setstretch{.7}
{\PaliGlossA{37. aparihāniyasutta}}\\
\begin{addmargin}[1em]{2em}
\setstretch{.5}
{\PaliGlossB{37. Non-decline}}\\
\end{addmargin}
\end{absolutelynopagebreak}

\begin{absolutelynopagebreak}
\setstretch{.7}
{\PaliGlossA{“catūhi, bhikkhave, dhammehi samannāgato bhikkhu abhabbo parihānāya nibbānasseva santike.}}\\
\begin{addmargin}[1em]{2em}
\setstretch{.5}
{\PaliGlossB{“Mendicants, a mendicant who has four qualities can’t decline, and is close to extinguishment.}}\\
\end{addmargin}
\end{absolutelynopagebreak}

\begin{absolutelynopagebreak}
\setstretch{.7}
{\PaliGlossA{katamehi catūhi?}}\\
\begin{addmargin}[1em]{2em}
\setstretch{.5}
{\PaliGlossB{What four?}}\\
\end{addmargin}
\end{absolutelynopagebreak}

\begin{absolutelynopagebreak}
\setstretch{.7}
{\PaliGlossA{idha, bhikkhave, bhikkhu sīlasampanno hoti, indriyesu guttadvāro hoti, bhojane mattaññū hoti, jāgariyaṃ anuyutto hoti.}}\\
\begin{addmargin}[1em]{2em}
\setstretch{.5}
{\PaliGlossB{A mendicant is accomplished in ethics, guards the sense doors, eats in moderation, and is dedicated to wakefulness.}}\\
\end{addmargin}
\end{absolutelynopagebreak}

\begin{absolutelynopagebreak}
\setstretch{.7}
{\PaliGlossA{kathañca, bhikkhave, bhikkhu sīlasampanno hoti?}}\\
\begin{addmargin}[1em]{2em}
\setstretch{.5}
{\PaliGlossB{And how is a mendicant accomplished in ethics?}}\\
\end{addmargin}
\end{absolutelynopagebreak}

\begin{absolutelynopagebreak}
\setstretch{.7}
{\PaliGlossA{idha, bhikkhave, bhikkhu sīlavā hoti pātimokkhasaṃvarasaṃvuto viharati ācāragocarasampanno aṇumattesu vajjesu bhayadassāvī, samādāya sikkhati sikkhāpadesu.}}\\
\begin{addmargin}[1em]{2em}
\setstretch{.5}
{\PaliGlossB{It’s when a mendicant is ethical, restrained in the monastic code, conducting themselves well and seeking alms in suitable places. Seeing danger in the slightest fault, they keep the rules they’ve undertaken.}}\\
\end{addmargin}
\end{absolutelynopagebreak}

\begin{absolutelynopagebreak}
\setstretch{.7}
{\PaliGlossA{evaṃ kho, bhikkhave, bhikkhu sīlasampanno hoti.}}\\
\begin{addmargin}[1em]{2em}
\setstretch{.5}
{\PaliGlossB{That’s how a mendicant is accomplished in ethics.}}\\
\end{addmargin}
\end{absolutelynopagebreak}

\begin{absolutelynopagebreak}
\setstretch{.7}
{\PaliGlossA{kathañca, bhikkhave, bhikkhu indriyesu guttadvāro hoti?}}\\
\begin{addmargin}[1em]{2em}
\setstretch{.5}
{\PaliGlossB{And how does a mendicant guard the sense doors?}}\\
\end{addmargin}
\end{absolutelynopagebreak}

\begin{absolutelynopagebreak}
\setstretch{.7}
{\PaliGlossA{idha, bhikkhave, bhikkhu cakkhunā rūpaṃ disvā na nimittaggāhī hoti nānubyañjanaggāhī.}}\\
\begin{addmargin}[1em]{2em}
\setstretch{.5}
{\PaliGlossB{When a mendicant sees a sight with their eyes, they don’t get caught up in the features and details.}}\\
\end{addmargin}
\end{absolutelynopagebreak}

\begin{absolutelynopagebreak}
\setstretch{.7}
{\PaliGlossA{yatvādhikaraṇamenaṃ cakkhundriyaṃ asaṃvutaṃ viharantaṃ abhijjhādomanassā pāpakā akusalā dhammā anvāssaveyyuṃ, tassa saṃvarāya paṭipajjati; rakkhati cakkhundriyaṃ; cakkhundriye saṃvaraṃ āpajjati.}}\\
\begin{addmargin}[1em]{2em}
\setstretch{.5}
{\PaliGlossB{If the faculty of sight were left unrestrained, bad unskillful qualities of desire and aversion would become overwhelming. For this reason, they practice restraint, protecting the faculty of sight, and achieving restraint over it.}}\\
\end{addmargin}
\end{absolutelynopagebreak}

\begin{absolutelynopagebreak}
\setstretch{.7}
{\PaliGlossA{sotena saddaṃ sutvā …}}\\
\begin{addmargin}[1em]{2em}
\setstretch{.5}
{\PaliGlossB{Hearing a sound with their ears …}}\\
\end{addmargin}
\end{absolutelynopagebreak}

\begin{absolutelynopagebreak}
\setstretch{.7}
{\PaliGlossA{ghānena gandhaṃ ghāyitvā …}}\\
\begin{addmargin}[1em]{2em}
\setstretch{.5}
{\PaliGlossB{Smelling an odor with their nose …}}\\
\end{addmargin}
\end{absolutelynopagebreak}

\begin{absolutelynopagebreak}
\setstretch{.7}
{\PaliGlossA{jivhāya rasaṃ sāyitvā …}}\\
\begin{addmargin}[1em]{2em}
\setstretch{.5}
{\PaliGlossB{Tasting a flavor with their tongue …}}\\
\end{addmargin}
\end{absolutelynopagebreak}

\begin{absolutelynopagebreak}
\setstretch{.7}
{\PaliGlossA{kāyena phoṭṭhabbaṃ phusitvā …}}\\
\begin{addmargin}[1em]{2em}
\setstretch{.5}
{\PaliGlossB{Feeling a touch with their body …}}\\
\end{addmargin}
\end{absolutelynopagebreak}

\begin{absolutelynopagebreak}
\setstretch{.7}
{\PaliGlossA{manasā dhammaṃ viññāya na nimittaggāhī hoti nānubyañjanaggāhī.}}\\
\begin{addmargin}[1em]{2em}
\setstretch{.5}
{\PaliGlossB{Knowing a thought with their mind, they don’t get caught up in the features and details.}}\\
\end{addmargin}
\end{absolutelynopagebreak}

\begin{absolutelynopagebreak}
\setstretch{.7}
{\PaliGlossA{yatvādhikaraṇamenaṃ manindriyaṃ asaṃvutaṃ viharantaṃ abhijjhādomanassā pāpakā akusalā dhammā anvāssaveyyuṃ, tassa saṃvarāya paṭipajjati; rakkhati manindriyaṃ; manindriye saṃvaraṃ āpajjati.}}\\
\begin{addmargin}[1em]{2em}
\setstretch{.5}
{\PaliGlossB{If the faculty of mind were left unrestrained, bad unskillful qualities of desire and aversion would become overwhelming. For this reason, they practice restraint, protecting the faculty of mind, and achieving restraint over it.}}\\
\end{addmargin}
\end{absolutelynopagebreak}

\begin{absolutelynopagebreak}
\setstretch{.7}
{\PaliGlossA{evaṃ kho, bhikkhave, bhikkhu indriyesu guttadvāro hoti.}}\\
\begin{addmargin}[1em]{2em}
\setstretch{.5}
{\PaliGlossB{That’s how a mendicant guards the sense doors.}}\\
\end{addmargin}
\end{absolutelynopagebreak}

\begin{absolutelynopagebreak}
\setstretch{.7}
{\PaliGlossA{kathañca, bhikkhave, bhikkhu bhojane mattaññū hoti?}}\\
\begin{addmargin}[1em]{2em}
\setstretch{.5}
{\PaliGlossB{And how does a mendicant eat in moderation?}}\\
\end{addmargin}
\end{absolutelynopagebreak}

\begin{absolutelynopagebreak}
\setstretch{.7}
{\PaliGlossA{idha, bhikkhave, bhikkhu paṭisaṅkhā yoniso āhāraṃ āhāreti:}}\\
\begin{addmargin}[1em]{2em}
\setstretch{.5}
{\PaliGlossB{It’s when a mendicant reflects properly on the food that they eat:}}\\
\end{addmargin}
\end{absolutelynopagebreak}

\begin{absolutelynopagebreak}
\setstretch{.7}
{\PaliGlossA{‘neva davāya na madāya na maṇḍanāya na vibhūsanāya; yāvadeva imassa kāyassa ṭhitiyā yāpanāya vihiṃsūparatiyā brahmacariyānuggahāya. iti purāṇañca vedanaṃ paṭihaṅkhāmi, navañca vedanaṃ na uppādessāmi, yātrā ca me bhavissati, anavajjatā ca phāsuvihāro cā’ti.}}\\
\begin{addmargin}[1em]{2em}
\setstretch{.5}
{\PaliGlossB{‘Not for fun, indulgence, adornment, or decoration, but only to sustain this body, to avoid harm, and to support spiritual practice. In this way, I shall put an end to old discomfort and not give rise to new discomfort, and I will live blamelessly and at ease.’}}\\
\end{addmargin}
\end{absolutelynopagebreak}

\begin{absolutelynopagebreak}
\setstretch{.7}
{\PaliGlossA{evaṃ kho, bhikkhave, bhikkhu bhojane mattaññū hoti.}}\\
\begin{addmargin}[1em]{2em}
\setstretch{.5}
{\PaliGlossB{That’s how a mendicant eats in moderation.}}\\
\end{addmargin}
\end{absolutelynopagebreak}

\begin{absolutelynopagebreak}
\setstretch{.7}
{\PaliGlossA{kathañca, bhikkhave, bhikkhu jāgariyaṃ anuyutto hoti?}}\\
\begin{addmargin}[1em]{2em}
\setstretch{.5}
{\PaliGlossB{And how is a mendicant dedicated to wakefulness?}}\\
\end{addmargin}
\end{absolutelynopagebreak}

\begin{absolutelynopagebreak}
\setstretch{.7}
{\PaliGlossA{idha, bhikkhave, bhikkhu divasaṃ caṅkamena nisajjāya āvaraṇīyehi dhammehi cittaṃ parisodheti;}}\\
\begin{addmargin}[1em]{2em}
\setstretch{.5}
{\PaliGlossB{It’s when a mendicant practices walking and sitting meditation by day, purifying their mind from obstacles.}}\\
\end{addmargin}
\end{absolutelynopagebreak}

\begin{absolutelynopagebreak}
\setstretch{.7}
{\PaliGlossA{rattiyā paṭhamaṃ yāmaṃ caṅkamena nisajjāya āvaraṇīyehi dhammehi cittaṃ parisodheti;}}\\
\begin{addmargin}[1em]{2em}
\setstretch{.5}
{\PaliGlossB{In the evening, they continue to practice walking and sitting meditation.}}\\
\end{addmargin}
\end{absolutelynopagebreak}

\begin{absolutelynopagebreak}
\setstretch{.7}
{\PaliGlossA{rattiyā majjhimaṃ yāmaṃ dakkhiṇena passena sīhaseyyaṃ kappeti, pāde pādaṃ accādhāya, sato sampajāno uṭṭhānasaññaṃ manasi karitvā;}}\\
\begin{addmargin}[1em]{2em}
\setstretch{.5}
{\PaliGlossB{In the middle of the night, they lie down in the lion’s posture—on the right side, placing one foot on top of the other—mindful and aware, and focused on the time of getting up.}}\\
\end{addmargin}
\end{absolutelynopagebreak}

\begin{absolutelynopagebreak}
\setstretch{.7}
{\PaliGlossA{rattiyā pacchimaṃ yāmaṃ paccuṭṭhāya caṅkamena nisajjāya āvaraṇīyehi dhammehi cittaṃ parisodheti.}}\\
\begin{addmargin}[1em]{2em}
\setstretch{.5}
{\PaliGlossB{In the last part of the night, they get up and continue to practice walking and sitting meditation, purifying their mind from obstacles.}}\\
\end{addmargin}
\end{absolutelynopagebreak}

\begin{absolutelynopagebreak}
\setstretch{.7}
{\PaliGlossA{evaṃ kho, bhikkhave, bhikkhu jāgariyaṃ anuyutto hoti.}}\\
\begin{addmargin}[1em]{2em}
\setstretch{.5}
{\PaliGlossB{This is how a mendicant is dedicated to wakefulness.}}\\
\end{addmargin}
\end{absolutelynopagebreak}

\begin{absolutelynopagebreak}
\setstretch{.7}
{\PaliGlossA{imehi kho, bhikkhave, catūhi dhammehi samannāgato bhikkhu abhabbo parihānāya, nibbānasseva santiketi.}}\\
\begin{addmargin}[1em]{2em}
\setstretch{.5}
{\PaliGlossB{A mendicant who has these four qualities can’t decline, and has drawn near to extinguishment.}}\\
\end{addmargin}
\end{absolutelynopagebreak}

\begin{absolutelynopagebreak}
\setstretch{.7}
{\PaliGlossA{sīle patiṭṭhito bhikkhu,}}\\
\begin{addmargin}[1em]{2em}
\setstretch{.5}
{\PaliGlossB{Established in ethics,}}\\
\end{addmargin}
\end{absolutelynopagebreak}

\begin{absolutelynopagebreak}
\setstretch{.7}
{\PaliGlossA{indriyesu ca saṃvuto;}}\\
\begin{addmargin}[1em]{2em}
\setstretch{.5}
{\PaliGlossB{restrained in the sense faculties,}}\\
\end{addmargin}
\end{absolutelynopagebreak}

\begin{absolutelynopagebreak}
\setstretch{.7}
{\PaliGlossA{bhojanamhi ca mattaññū,}}\\
\begin{addmargin}[1em]{2em}
\setstretch{.5}
{\PaliGlossB{eating in moderation,}}\\
\end{addmargin}
\end{absolutelynopagebreak}

\begin{absolutelynopagebreak}
\setstretch{.7}
{\PaliGlossA{jāgariyaṃ anuyuñjati.}}\\
\begin{addmargin}[1em]{2em}
\setstretch{.5}
{\PaliGlossB{and dedicated to wakefulness;}}\\
\end{addmargin}
\end{absolutelynopagebreak}

\begin{absolutelynopagebreak}
\setstretch{.7}
{\PaliGlossA{evaṃ vihārī ātāpī,}}\\
\begin{addmargin}[1em]{2em}
\setstretch{.5}
{\PaliGlossB{a mendicant lives like this, with keen energy,}}\\
\end{addmargin}
\end{absolutelynopagebreak}

\begin{absolutelynopagebreak}
\setstretch{.7}
{\PaliGlossA{ahorattamatandito;}}\\
\begin{addmargin}[1em]{2em}
\setstretch{.5}
{\PaliGlossB{tireless all night and day,}}\\
\end{addmargin}
\end{absolutelynopagebreak}

\begin{absolutelynopagebreak}
\setstretch{.7}
{\PaliGlossA{bhāvayaṃ kusalaṃ dhammaṃ,}}\\
\begin{addmargin}[1em]{2em}
\setstretch{.5}
{\PaliGlossB{developing skillful qualities,}}\\
\end{addmargin}
\end{absolutelynopagebreak}

\begin{absolutelynopagebreak}
\setstretch{.7}
{\PaliGlossA{yogakkhemassa pattiyā.}}\\
\begin{addmargin}[1em]{2em}
\setstretch{.5}
{\PaliGlossB{for the sake of finding sanctuary.}}\\
\end{addmargin}
\end{absolutelynopagebreak}

\begin{absolutelynopagebreak}
\setstretch{.7}
{\PaliGlossA{appamādarato bhikkhu,}}\\
\begin{addmargin}[1em]{2em}
\setstretch{.5}
{\PaliGlossB{A mendicant who loves to be diligent,}}\\
\end{addmargin}
\end{absolutelynopagebreak}

\begin{absolutelynopagebreak}
\setstretch{.7}
{\PaliGlossA{pamāde bhayadassi vā;}}\\
\begin{addmargin}[1em]{2em}
\setstretch{.5}
{\PaliGlossB{seeing fear in negligence,}}\\
\end{addmargin}
\end{absolutelynopagebreak}

\begin{absolutelynopagebreak}
\setstretch{.7}
{\PaliGlossA{abhabbo parihānāya,}}\\
\begin{addmargin}[1em]{2em}
\setstretch{.5}
{\PaliGlossB{can’t decline,}}\\
\end{addmargin}
\end{absolutelynopagebreak}

\begin{absolutelynopagebreak}
\setstretch{.7}
{\PaliGlossA{nibbānasseva santike”ti.}}\\
\begin{addmargin}[1em]{2em}
\setstretch{.5}
{\PaliGlossB{and has drawn near to extinguishment.”}}\\
\end{addmargin}
\end{absolutelynopagebreak}

\begin{absolutelynopagebreak}
\setstretch{.7}
{\PaliGlossA{sattamaṃ.}}\\
\begin{addmargin}[1em]{2em}
\setstretch{.5}
{\PaliGlossB{    -}}\\
\end{addmargin}
\end{absolutelynopagebreak}
