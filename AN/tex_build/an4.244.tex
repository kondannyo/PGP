
\begin{absolutelynopagebreak}
\setstretch{.7}
{\PaliGlossA{aṅguttara nikāya 4}}\\
\begin{addmargin}[1em]{2em}
\setstretch{.5}
{\PaliGlossB{Numbered Discourses 4}}\\
\end{addmargin}
\end{absolutelynopagebreak}

\begin{absolutelynopagebreak}
\setstretch{.7}
{\PaliGlossA{25. āpattibhayavagga}}\\
\begin{addmargin}[1em]{2em}
\setstretch{.5}
{\PaliGlossB{25. Perils of Offenses}}\\
\end{addmargin}
\end{absolutelynopagebreak}

\begin{absolutelynopagebreak}
\setstretch{.7}
{\PaliGlossA{244. āpattibhayasutta}}\\
\begin{addmargin}[1em]{2em}
\setstretch{.5}
{\PaliGlossB{244. Perils of Offenses}}\\
\end{addmargin}
\end{absolutelynopagebreak}

\begin{absolutelynopagebreak}
\setstretch{.7}
{\PaliGlossA{“cattārimāni, bhikkhave, āpattibhayāni.}}\\
\begin{addmargin}[1em]{2em}
\setstretch{.5}
{\PaliGlossB{“Mendicants, there are these four perils of offenses.}}\\
\end{addmargin}
\end{absolutelynopagebreak}

\begin{absolutelynopagebreak}
\setstretch{.7}
{\PaliGlossA{katamāni cattāri?}}\\
\begin{addmargin}[1em]{2em}
\setstretch{.5}
{\PaliGlossB{What four?}}\\
\end{addmargin}
\end{absolutelynopagebreak}

\begin{absolutelynopagebreak}
\setstretch{.7}
{\PaliGlossA{seyyathāpi, bhikkhave, coraṃ āgucāriṃ gahetvā rañño dasseyyuṃ:}}\\
\begin{addmargin}[1em]{2em}
\setstretch{.5}
{\PaliGlossB{Suppose they were to arrest a bandit, a criminal and present him to the king, saying:}}\\
\end{addmargin}
\end{absolutelynopagebreak}

\begin{absolutelynopagebreak}
\setstretch{.7}
{\PaliGlossA{‘ayaṃ te, deva, coro āgucārī.}}\\
\begin{addmargin}[1em]{2em}
\setstretch{.5}
{\PaliGlossB{‘Your Majesty, this is a bandit, a criminal.}}\\
\end{addmargin}
\end{absolutelynopagebreak}

\begin{absolutelynopagebreak}
\setstretch{.7}
{\PaliGlossA{imassa devo daṇḍaṃ paṇetū’ti.}}\\
\begin{addmargin}[1em]{2em}
\setstretch{.5}
{\PaliGlossB{May Your Majesty punish them!’}}\\
\end{addmargin}
\end{absolutelynopagebreak}

\begin{absolutelynopagebreak}
\setstretch{.7}
{\PaliGlossA{tamenaṃ rājā evaṃ vadeyya:}}\\
\begin{addmargin}[1em]{2em}
\setstretch{.5}
{\PaliGlossB{The king would say:}}\\
\end{addmargin}
\end{absolutelynopagebreak}

\begin{absolutelynopagebreak}
\setstretch{.7}
{\PaliGlossA{‘gacchatha, bho, imaṃ purisaṃ daḷhāya rajjuyā pacchābāhaṃ gāḷhabandhanaṃ bandhitvā khuramuṇḍaṃ karitvā kharassarena paṇavena rathikāya rathikaṃ siṅghāṭakena siṅghāṭakaṃ parinetvā dakkhiṇena dvārena nikkhāmetvā dakkhiṇato nagarassa sīsaṃ chindathā’ti.}}\\
\begin{addmargin}[1em]{2em}
\setstretch{.5}
{\PaliGlossB{‘Go, my men, and tie this man’s arms tightly behind his back with a strong rope. Shave his head and march him from street to street and square to square to the beating of a harsh drum. Then take him out the south gate and there, to the south of the city, chop off his head.’}}\\
\end{addmargin}
\end{absolutelynopagebreak}

\begin{absolutelynopagebreak}
\setstretch{.7}
{\PaliGlossA{tamenaṃ rañño purisā daḷhāya rajjuyā pacchābāhaṃ gāḷhabandhanaṃ bandhitvā khuramuṇḍaṃ karitvā kharassarena paṇavena rathikāya rathikaṃ siṅghāṭakena siṅghāṭakaṃ parinetvā dakkhiṇena dvārena nikkhāmetvā dakkhiṇato nagarassa sīsaṃ chindeyyuṃ.}}\\
\begin{addmargin}[1em]{2em}
\setstretch{.5}
{\PaliGlossB{The king’s men would do as they were told.}}\\
\end{addmargin}
\end{absolutelynopagebreak}

\begin{absolutelynopagebreak}
\setstretch{.7}
{\PaliGlossA{tatraññatarassa thalaṭṭhassa purisassa evamassa:}}\\
\begin{addmargin}[1em]{2em}
\setstretch{.5}
{\PaliGlossB{Then a bystander might think:}}\\
\end{addmargin}
\end{absolutelynopagebreak}

\begin{absolutelynopagebreak}
\setstretch{.7}
{\PaliGlossA{‘pāpakaṃ vata, bho, ayaṃ puriso kammaṃ akāsi gārayhaṃ sīsacchejjaṃ.}}\\
\begin{addmargin}[1em]{2em}
\setstretch{.5}
{\PaliGlossB{‘This man must have done a truly bad and reprehensible deed, a capital offense.}}\\
\end{addmargin}
\end{absolutelynopagebreak}

\begin{absolutelynopagebreak}
\setstretch{.7}
{\PaliGlossA{yatra hi nāma rañño purisā daḷhāya rajjuyā pacchābāhaṃ gāḷhabandhanaṃ bandhitvā khuramuṇḍaṃ karitvā kharassarena paṇavena rathikāya rathikaṃ siṅghāṭakena siṅghāṭakaṃ parinetvā dakkhiṇena dvārena nikkhāmetvā dakkhiṇato nagarassa sīsaṃ chindissanti.}}\\
\begin{addmargin}[1em]{2em}
\setstretch{.5}
{\PaliGlossB{    -}}\\
\end{addmargin}
\end{absolutelynopagebreak}

\begin{absolutelynopagebreak}
\setstretch{.7}
{\PaliGlossA{so vatassāhaṃ evarūpaṃ pāpakammaṃ na kareyyaṃ gārayhaṃ sīsacchejjan’ti.}}\\
\begin{addmargin}[1em]{2em}
\setstretch{.5}
{\PaliGlossB{There’s no way I’d ever do such a bad and reprehensible deed, a capital offense.’}}\\
\end{addmargin}
\end{absolutelynopagebreak}

\begin{absolutelynopagebreak}
\setstretch{.7}
{\PaliGlossA{evamevaṃ kho, bhikkhave, yassa kassaci bhikkhussa vā bhikkhuniyā vā evaṃ tibbā bhayasaññā paccupaṭṭhitā hoti pārājikesu dhammesu.}}\\
\begin{addmargin}[1em]{2em}
\setstretch{.5}
{\PaliGlossB{In the same way, take any monk or nun who has set up such an acute perception of peril regarding expulsion offenses.}}\\
\end{addmargin}
\end{absolutelynopagebreak}

\begin{absolutelynopagebreak}
\setstretch{.7}
{\PaliGlossA{tassetaṃ pāṭikaṅkhaṃ—anāpanno vā pārājikaṃ dhammaṃ na āpajjissati, āpanno vā pārājikaṃ dhammaṃ yathādhammaṃ paṭikarissati.}}\\
\begin{addmargin}[1em]{2em}
\setstretch{.5}
{\PaliGlossB{It can be expected that if they haven’t committed an expulsion offense they won’t, and if they committed one they will deal with it properly.}}\\
\end{addmargin}
\end{absolutelynopagebreak}

\begin{absolutelynopagebreak}
\setstretch{.7}
{\PaliGlossA{seyyathāpi, bhikkhave, puriso kāḷavatthaṃ paridhāya kese pakiritvā musalaṃ khandhe āropetvā mahājanakāyaṃ upasaṅkamitvā evaṃ vadeyya:}}\\
\begin{addmargin}[1em]{2em}
\setstretch{.5}
{\PaliGlossB{Suppose a man was to put on a black cloth, mess up his hair, and put a club on his shoulder. Then he approaches a large crowd and says:}}\\
\end{addmargin}
\end{absolutelynopagebreak}

\begin{absolutelynopagebreak}
\setstretch{.7}
{\PaliGlossA{‘ahaṃ, bhante, pāpakammaṃ akāsiṃ gārayhaṃ mosallaṃ, yena me āyasmanto attamanā honti taṃ karomī’ti.}}\\
\begin{addmargin}[1em]{2em}
\setstretch{.5}
{\PaliGlossB{‘Sirs, I’ve done a bad and reprehensible deed, deserving of clubbing. I submit to your pleasure.’}}\\
\end{addmargin}
\end{absolutelynopagebreak}

\begin{absolutelynopagebreak}
\setstretch{.7}
{\PaliGlossA{tatraññatarassa thalaṭṭhassa purisassa evamassa:}}\\
\begin{addmargin}[1em]{2em}
\setstretch{.5}
{\PaliGlossB{Then a bystander might think:}}\\
\end{addmargin}
\end{absolutelynopagebreak}

\begin{absolutelynopagebreak}
\setstretch{.7}
{\PaliGlossA{‘pāpakaṃ vata, bho, ayaṃ puriso kammaṃ akāsi gārayhaṃ mosallaṃ.}}\\
\begin{addmargin}[1em]{2em}
\setstretch{.5}
{\PaliGlossB{‘This man must have done a truly bad and reprehensible deed, deserving of clubbing. …}}\\
\end{addmargin}
\end{absolutelynopagebreak}

\begin{absolutelynopagebreak}
\setstretch{.7}
{\PaliGlossA{yatra hi nāma kāḷavatthaṃ paridhāya kese pakiritvā musalaṃ khandhe āropetvā mahājanakāyaṃ upasaṅkamitvā evaṃ vakkhati:}}\\
\begin{addmargin}[1em]{2em}
\setstretch{.5}
{\PaliGlossB{    -}}\\
\end{addmargin}
\end{absolutelynopagebreak}

\begin{absolutelynopagebreak}
\setstretch{.7}
{\PaliGlossA{“ahaṃ, bhante, pāpakammaṃ akāsiṃ gārayhaṃ mosallaṃ, yena me āyasmanto attamanā honti taṃ karomī”ti.}}\\
\begin{addmargin}[1em]{2em}
\setstretch{.5}
{\PaliGlossB{    -}}\\
\end{addmargin}
\end{absolutelynopagebreak}

\begin{absolutelynopagebreak}
\setstretch{.7}
{\PaliGlossA{so vatassāhaṃ evarūpaṃ pāpakammaṃ na kareyyaṃ gārayhaṃ mosallan’ti.}}\\
\begin{addmargin}[1em]{2em}
\setstretch{.5}
{\PaliGlossB{There’s no way I’d ever do such a bad and reprehensible deed, deserving of clubbing.’}}\\
\end{addmargin}
\end{absolutelynopagebreak}

\begin{absolutelynopagebreak}
\setstretch{.7}
{\PaliGlossA{evamevaṃ kho, bhikkhave, yassa kassaci bhikkhussa vā bhikkhuniyā vā evaṃ tibbā bhayasaññā paccupaṭṭhitā hoti saṃghādisesesu dhammesu, tassetaṃ pāṭikaṅkhaṃ—anāpanno vā saṃghādisesaṃ dhammaṃ na āpajjissati, āpanno vā saṃghādisesaṃ dhammaṃ yathādhammaṃ paṭikarissati.}}\\
\begin{addmargin}[1em]{2em}
\setstretch{.5}
{\PaliGlossB{In the same way, take any monk or nun who has set up such an acute perception of peril regarding suspension offenses. It can be expected that if they haven’t committed a suspension offense they won’t, and if they committed one they will deal with it properly.}}\\
\end{addmargin}
\end{absolutelynopagebreak}

\begin{absolutelynopagebreak}
\setstretch{.7}
{\PaliGlossA{seyyathāpi, bhikkhave, puriso kāḷavatthaṃ paridhāya kese pakiritvā bhasmapuṭaṃ khandhe āropetvā mahājanakāyaṃ upasaṅkamitvā evaṃ vadeyya:}}\\
\begin{addmargin}[1em]{2em}
\setstretch{.5}
{\PaliGlossB{Suppose a man was to put on a black cloth, mess up his hair, and put a sack of ashes on his shoulder. Then he approaches a large crowd and says:}}\\
\end{addmargin}
\end{absolutelynopagebreak}

\begin{absolutelynopagebreak}
\setstretch{.7}
{\PaliGlossA{‘ahaṃ, bhante, pāpakammaṃ akāsiṃ gārayhaṃ bhasmapuṭaṃ.}}\\
\begin{addmargin}[1em]{2em}
\setstretch{.5}
{\PaliGlossB{‘Sirs, I’ve done a bad and reprehensible deed, deserving of a sack of ashes.}}\\
\end{addmargin}
\end{absolutelynopagebreak}

\begin{absolutelynopagebreak}
\setstretch{.7}
{\PaliGlossA{yena me āyasmanto attamanā honti taṃ karomī’ti.}}\\
\begin{addmargin}[1em]{2em}
\setstretch{.5}
{\PaliGlossB{I submit to your pleasure.’}}\\
\end{addmargin}
\end{absolutelynopagebreak}

\begin{absolutelynopagebreak}
\setstretch{.7}
{\PaliGlossA{tatraññatarassa thalaṭṭhassa purisassa evamassa:}}\\
\begin{addmargin}[1em]{2em}
\setstretch{.5}
{\PaliGlossB{Then a bystander might think:}}\\
\end{addmargin}
\end{absolutelynopagebreak}

\begin{absolutelynopagebreak}
\setstretch{.7}
{\PaliGlossA{‘pāpakaṃ vata, bho, ayaṃ puriso kammaṃ akāsi gārayhaṃ bhasmapuṭaṃ.}}\\
\begin{addmargin}[1em]{2em}
\setstretch{.5}
{\PaliGlossB{‘This man must have done a truly bad and reprehensible deed, deserving of a sack of ashes. …}}\\
\end{addmargin}
\end{absolutelynopagebreak}

\begin{absolutelynopagebreak}
\setstretch{.7}
{\PaliGlossA{yatra hi nāma kāḷavatthaṃ paridhāya kese pakiritvā bhasmapuṭaṃ khandhe āropetvā mahājanakāyaṃ upasaṅkamitvā evaṃ vakkhati:}}\\
\begin{addmargin}[1em]{2em}
\setstretch{.5}
{\PaliGlossB{    -}}\\
\end{addmargin}
\end{absolutelynopagebreak}

\begin{absolutelynopagebreak}
\setstretch{.7}
{\PaliGlossA{“ahaṃ, bhante, pāpakammaṃ akāsiṃ gārayhaṃ bhasmapuṭaṃ;}}\\
\begin{addmargin}[1em]{2em}
\setstretch{.5}
{\PaliGlossB{    -}}\\
\end{addmargin}
\end{absolutelynopagebreak}

\begin{absolutelynopagebreak}
\setstretch{.7}
{\PaliGlossA{yena me āyasmanto attamanā honti taṃ karomī”ti.}}\\
\begin{addmargin}[1em]{2em}
\setstretch{.5}
{\PaliGlossB{    -}}\\
\end{addmargin}
\end{absolutelynopagebreak}

\begin{absolutelynopagebreak}
\setstretch{.7}
{\PaliGlossA{so vatassāhaṃ evarūpaṃ pāpakammaṃ na kareyyaṃ gārayhaṃ bhasmapuṭan’ti.}}\\
\begin{addmargin}[1em]{2em}
\setstretch{.5}
{\PaliGlossB{There’s no way I’d ever do such a bad and reprehensible deed, deserving of a sack of ashes.’}}\\
\end{addmargin}
\end{absolutelynopagebreak}

\begin{absolutelynopagebreak}
\setstretch{.7}
{\PaliGlossA{evamevaṃ kho, bhikkhave, yassa kassaci bhikkhussa vā bhikkhuniyā vā evaṃ tibbā bhayasaññā paccupaṭṭhitā hoti pācittiyesu dhammesu, tassetaṃ pāṭikaṅkhaṃ—anāpanno vā pācittiyaṃ dhammaṃ na āpajjissati, āpanno vā pācittiyaṃ dhammaṃ yathādhammaṃ paṭikarissati.}}\\
\begin{addmargin}[1em]{2em}
\setstretch{.5}
{\PaliGlossB{In the same way, take any monk or nun who has set up such an acute perception of peril regarding confessable offenses. It can be expected that if they haven’t committed a confessable offense they won’t, and if they committed one they will deal with it properly.}}\\
\end{addmargin}
\end{absolutelynopagebreak}

\begin{absolutelynopagebreak}
\setstretch{.7}
{\PaliGlossA{seyyathāpi, bhikkhave, puriso kāḷavatthaṃ paridhāya kese pakiritvā mahājanakāyaṃ upasaṅkamitvā evaṃ vadeyya:}}\\
\begin{addmargin}[1em]{2em}
\setstretch{.5}
{\PaliGlossB{Suppose a man was to put on a black cloth and mess up his hair. Then he approaches a large crowd and says:}}\\
\end{addmargin}
\end{absolutelynopagebreak}

\begin{absolutelynopagebreak}
\setstretch{.7}
{\PaliGlossA{‘ahaṃ, bhante, pāpakammaṃ akāsiṃ gārayhaṃ upavajjaṃ.}}\\
\begin{addmargin}[1em]{2em}
\setstretch{.5}
{\PaliGlossB{‘Sirs, I’ve done a bad and reprehensible deed, deserving of criticism.}}\\
\end{addmargin}
\end{absolutelynopagebreak}

\begin{absolutelynopagebreak}
\setstretch{.7}
{\PaliGlossA{yena me āyasmanto attamanā honti taṃ karomī’ti.}}\\
\begin{addmargin}[1em]{2em}
\setstretch{.5}
{\PaliGlossB{I submit to your pleasure.’}}\\
\end{addmargin}
\end{absolutelynopagebreak}

\begin{absolutelynopagebreak}
\setstretch{.7}
{\PaliGlossA{tatraññatarassa thalaṭṭhassa purisassa evamassa:}}\\
\begin{addmargin}[1em]{2em}
\setstretch{.5}
{\PaliGlossB{Then a bystander might think:}}\\
\end{addmargin}
\end{absolutelynopagebreak}

\begin{absolutelynopagebreak}
\setstretch{.7}
{\PaliGlossA{‘pāpakaṃ vata, bho, ayaṃ puriso kammaṃ akāsi gārayhaṃ upavajjaṃ.}}\\
\begin{addmargin}[1em]{2em}
\setstretch{.5}
{\PaliGlossB{‘This man must have done a truly bad and reprehensible deed, deserving of criticism. …}}\\
\end{addmargin}
\end{absolutelynopagebreak}

\begin{absolutelynopagebreak}
\setstretch{.7}
{\PaliGlossA{yatra hi nāma kāḷavatthaṃ paridhāya kese pakiritvā mahājanakāyaṃ upasaṅkamitvā evaṃ vakkhati:}}\\
\begin{addmargin}[1em]{2em}
\setstretch{.5}
{\PaliGlossB{    -}}\\
\end{addmargin}
\end{absolutelynopagebreak}

\begin{absolutelynopagebreak}
\setstretch{.7}
{\PaliGlossA{“ahaṃ, bhante, pāpakammaṃ akāsiṃ gārayhaṃ upavajjaṃ;}}\\
\begin{addmargin}[1em]{2em}
\setstretch{.5}
{\PaliGlossB{    -}}\\
\end{addmargin}
\end{absolutelynopagebreak}

\begin{absolutelynopagebreak}
\setstretch{.7}
{\PaliGlossA{yena me āyasmanto attamanā honti taṃ karomī”ti.}}\\
\begin{addmargin}[1em]{2em}
\setstretch{.5}
{\PaliGlossB{    -}}\\
\end{addmargin}
\end{absolutelynopagebreak}

\begin{absolutelynopagebreak}
\setstretch{.7}
{\PaliGlossA{so vatassāhaṃ evarūpaṃ pāpakammaṃ na kareyyaṃ gārayhaṃ upavajjan’ti.}}\\
\begin{addmargin}[1em]{2em}
\setstretch{.5}
{\PaliGlossB{There’s no way I’d ever do such a bad and reprehensible deed, deserving of criticism.’}}\\
\end{addmargin}
\end{absolutelynopagebreak}

\begin{absolutelynopagebreak}
\setstretch{.7}
{\PaliGlossA{evamevaṃ kho, bhikkhave, yassa kassaci bhikkhussa vā bhikkhuniyā vā evaṃ tibbā bhayasaññā paccupaṭṭhitā hoti pāṭidesanīyesu dhammesu, tassetaṃ pāṭikaṅkhaṃ—anāpanno vā pāṭidesanīyaṃ dhammaṃ na āpajjissati, āpanno vā pāṭidesanīyaṃ dhammaṃ yathādhammaṃ paṭikarissati.}}\\
\begin{addmargin}[1em]{2em}
\setstretch{.5}
{\PaliGlossB{In the same way, take any monk or nun who has set up such an acute perception of peril regarding acknowledgable offenses. It can be expected that if they haven’t committed an acknowledgeable offense they won’t, and if they committed one they will deal with it properly.}}\\
\end{addmargin}
\end{absolutelynopagebreak}

\begin{absolutelynopagebreak}
\setstretch{.7}
{\PaliGlossA{imāni kho, bhikkhave, cattāri āpattibhayānī”ti.}}\\
\begin{addmargin}[1em]{2em}
\setstretch{.5}
{\PaliGlossB{These are the four perils of offenses.”}}\\
\end{addmargin}
\end{absolutelynopagebreak}

\begin{absolutelynopagebreak}
\setstretch{.7}
{\PaliGlossA{dutiyaṃ.}}\\
\begin{addmargin}[1em]{2em}
\setstretch{.5}
{\PaliGlossB{    -}}\\
\end{addmargin}
\end{absolutelynopagebreak}
