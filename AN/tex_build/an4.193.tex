
\begin{absolutelynopagebreak}
\setstretch{.7}
{\PaliGlossA{aṅguttara nikāya 4}}\\
\begin{addmargin}[1em]{2em}
\setstretch{.5}
{\PaliGlossB{Numbered Discourses 4}}\\
\end{addmargin}
\end{absolutelynopagebreak}

\begin{absolutelynopagebreak}
\setstretch{.7}
{\PaliGlossA{20. mahāvagga}}\\
\begin{addmargin}[1em]{2em}
\setstretch{.5}
{\PaliGlossB{20. The Great Chapter}}\\
\end{addmargin}
\end{absolutelynopagebreak}

\begin{absolutelynopagebreak}
\setstretch{.7}
{\PaliGlossA{193. bhaddiyasutta}}\\
\begin{addmargin}[1em]{2em}
\setstretch{.5}
{\PaliGlossB{193. With Bhaddiya}}\\
\end{addmargin}
\end{absolutelynopagebreak}

\begin{absolutelynopagebreak}
\setstretch{.7}
{\PaliGlossA{ekaṃ samayaṃ bhagavā vesāliyaṃ viharati mahāvane kūṭāgārasālāyaṃ.}}\\
\begin{addmargin}[1em]{2em}
\setstretch{.5}
{\PaliGlossB{At one time the Buddha was staying near Vesālī, at the Great Wood, in the hall with the peaked roof.}}\\
\end{addmargin}
\end{absolutelynopagebreak}

\begin{absolutelynopagebreak}
\setstretch{.7}
{\PaliGlossA{atha kho bhaddiyo licchavi yena bhagavā tenupasaṅkami; upasaṅkamitvā bhagavantaṃ abhivādetvā ekamantaṃ nisīdi. ekamantaṃ nisinno kho bhaddiyo licchavi bhagavantaṃ etadavoca:}}\\
\begin{addmargin}[1em]{2em}
\setstretch{.5}
{\PaliGlossB{Then Bhaddiya the Licchavi went up to the Buddha, bowed, sat down to one side, and said to him:}}\\
\end{addmargin}
\end{absolutelynopagebreak}

\begin{absolutelynopagebreak}
\setstretch{.7}
{\PaliGlossA{“sutaṃ metaṃ, bhante:}}\\
\begin{addmargin}[1em]{2em}
\setstretch{.5}
{\PaliGlossB{“Sir, I have heard this:}}\\
\end{addmargin}
\end{absolutelynopagebreak}

\begin{absolutelynopagebreak}
\setstretch{.7}
{\PaliGlossA{‘māyāvī samaṇo gotamo āvaṭṭaniṃ māyaṃ jānāti yāya aññatitthiyānaṃ sāvake āvaṭṭetī’ti.}}\\
\begin{addmargin}[1em]{2em}
\setstretch{.5}
{\PaliGlossB{‘The ascetic Gotama is a magician. He knows a conversion magic, and uses it to convert the disciples of those who follow other paths.’}}\\
\end{addmargin}
\end{absolutelynopagebreak}

\begin{absolutelynopagebreak}
\setstretch{.7}
{\PaliGlossA{ye te, bhante, evamāhaṃsu:}}\\
\begin{addmargin}[1em]{2em}
\setstretch{.5}
{\PaliGlossB{    -}}\\
\end{addmargin}
\end{absolutelynopagebreak}

\begin{absolutelynopagebreak}
\setstretch{.7}
{\PaliGlossA{‘māyāvī samaṇo gotamo āvaṭṭaniṃ māyaṃ jānāti yāya aññatitthiyānaṃ sāvake āvaṭṭetī’ti, kacci te, bhante, bhagavato vuttavādino, na ca bhagavantaṃ abhūtena abbhācikkhanti, dhammassa ca anudhammaṃ byākaronti, na ca koci sahadhammiko vādānupāto gārayhaṃ ṭhānaṃ āgacchati, anabbhakkhātukāmā hi mayaṃ, bhante, bhagavantan”ti?}}\\
\begin{addmargin}[1em]{2em}
\setstretch{.5}
{\PaliGlossB{I trust that those who say this repeat what the Buddha has said, and do not misrepresent him with an untruth? Is their explanation in line with the teaching? Are there any legitimate grounds for rebuke and criticism?”}}\\
\end{addmargin}
\end{absolutelynopagebreak}

\begin{absolutelynopagebreak}
\setstretch{.7}
{\PaliGlossA{“etha tumhe, bhaddiya, mā anussavena, mā paramparāya, mā itikirāya, mā piṭakasampadānena, mā takkahetu, mā nayahetu, mā ākāraparivitakkena, mā diṭṭhinijjhānakkhantiyā, mā bhabbarūpatāya, mā ‘samaṇo no garū’ti.}}\\
\begin{addmargin}[1em]{2em}
\setstretch{.5}
{\PaliGlossB{“Please, Bhaddiya, don’t go by oral transmission, don’t go by lineage, don’t go by testament, don’t go by canonical authority, don’t rely on logic, don’t rely on inference, don’t go by reasoned contemplation, don’t go by the acceptance of a view after consideration, don’t go by the appearance of competence, and don’t think ‘The ascetic is our respected teacher.’}}\\
\end{addmargin}
\end{absolutelynopagebreak}

\begin{absolutelynopagebreak}
\setstretch{.7}
{\PaliGlossA{yadā tumhe, bhaddiya, attanāva jāneyyātha:}}\\
\begin{addmargin}[1em]{2em}
\setstretch{.5}
{\PaliGlossB{But when you know for yourselves:}}\\
\end{addmargin}
\end{absolutelynopagebreak}

\begin{absolutelynopagebreak}
\setstretch{.7}
{\PaliGlossA{‘ime dhammā akusalā, ime dhammā sāvajjā, ime dhammā viññugarahitā, ime dhammā samattā samādinnā ahitāya dukkhāya saṃvattantī’ti, atha tumhe, bhaddiya, pajaheyyātha.}}\\
\begin{addmargin}[1em]{2em}
\setstretch{.5}
{\PaliGlossB{‘These things are unskillful, blameworthy, criticized by sensible people, and when you undertake them, they lead to harm and suffering’, then you should give them up.}}\\
\end{addmargin}
\end{absolutelynopagebreak}

\begin{absolutelynopagebreak}
\setstretch{.7}
{\PaliGlossA{taṃ kiṃ maññatha, bhaddiya,}}\\
\begin{addmargin}[1em]{2em}
\setstretch{.5}
{\PaliGlossB{What do you think, Bhaddiya?}}\\
\end{addmargin}
\end{absolutelynopagebreak}

\begin{absolutelynopagebreak}
\setstretch{.7}
{\PaliGlossA{lobho purisassa ajjhattaṃ uppajjamāno uppajjati hitāya vā ahitāya vā”ti?}}\\
\begin{addmargin}[1em]{2em}
\setstretch{.5}
{\PaliGlossB{Does greed come up in a person for their welfare or harm?”}}\\
\end{addmargin}
\end{absolutelynopagebreak}

\begin{absolutelynopagebreak}
\setstretch{.7}
{\PaliGlossA{“ahitāya, bhante”.}}\\
\begin{addmargin}[1em]{2em}
\setstretch{.5}
{\PaliGlossB{“Harm, sir.”}}\\
\end{addmargin}
\end{absolutelynopagebreak}

\begin{absolutelynopagebreak}
\setstretch{.7}
{\PaliGlossA{“luddho panāyaṃ, bhaddiya, purisapuggalo lobhena abhibhūto pariyādinnacitto pāṇampi hanati, adinnampi ādiyati, paradārampi gacchati, musāpi bhaṇati, parampi tathattāya samādapeti yaṃsa hoti dīgharattaṃ ahitāya dukkhāyā”ti.}}\\
\begin{addmargin}[1em]{2em}
\setstretch{.5}
{\PaliGlossB{“A greedy individual—overcome by greed—kills living creatures, steals, commits adultery, lies, and encourages others to do the same. Is that for their lasting harm and suffering?”}}\\
\end{addmargin}
\end{absolutelynopagebreak}

\begin{absolutelynopagebreak}
\setstretch{.7}
{\PaliGlossA{“evaṃ, bhante”.}}\\
\begin{addmargin}[1em]{2em}
\setstretch{.5}
{\PaliGlossB{“Yes, sir.”}}\\
\end{addmargin}
\end{absolutelynopagebreak}

\begin{absolutelynopagebreak}
\setstretch{.7}
{\PaliGlossA{“taṃ kiṃ maññatha, bhaddiya,}}\\
\begin{addmargin}[1em]{2em}
\setstretch{.5}
{\PaliGlossB{“What do you think, Bhaddiya?}}\\
\end{addmargin}
\end{absolutelynopagebreak}

\begin{absolutelynopagebreak}
\setstretch{.7}
{\PaliGlossA{doso purisassa … pe … moho purisassa … pe … sārambho purisassa ajjhattaṃ uppajjamāno uppajjati hitāya vā ahitāya vā”ti?}}\\
\begin{addmargin}[1em]{2em}
\setstretch{.5}
{\PaliGlossB{Does hate … or delusion … or aggression come up in a person for their welfare or harm?”}}\\
\end{addmargin}
\end{absolutelynopagebreak}

\begin{absolutelynopagebreak}
\setstretch{.7}
{\PaliGlossA{“ahitāya, bhante”.}}\\
\begin{addmargin}[1em]{2em}
\setstretch{.5}
{\PaliGlossB{“Harm, sir.”}}\\
\end{addmargin}
\end{absolutelynopagebreak}

\begin{absolutelynopagebreak}
\setstretch{.7}
{\PaliGlossA{“sāraddho panāyaṃ, bhaddiya, purisapuggalo sārambhena abhibhūto pariyādinnacitto pāṇampi hanati, adinnampi ādiyati, paradārampi gacchati, musāpi bhaṇati, parampi tathattāya samādapeti yaṃsa hoti dīgharattaṃ ahitāya dukkhāyā”ti.}}\\
\begin{addmargin}[1em]{2em}
\setstretch{.5}
{\PaliGlossB{“An aggressive individual kills living creatures, steals, commits adultery, lies, and encourages others to do the same. Is that for their lasting harm and suffering?”}}\\
\end{addmargin}
\end{absolutelynopagebreak}

\begin{absolutelynopagebreak}
\setstretch{.7}
{\PaliGlossA{“evaṃ, bhante”.}}\\
\begin{addmargin}[1em]{2em}
\setstretch{.5}
{\PaliGlossB{“Yes, sir.”}}\\
\end{addmargin}
\end{absolutelynopagebreak}

\begin{absolutelynopagebreak}
\setstretch{.7}
{\PaliGlossA{“taṃ kiṃ maññatha, bhaddiya, ime dhammā kusalā vā akusalā vā”ti?}}\\
\begin{addmargin}[1em]{2em}
\setstretch{.5}
{\PaliGlossB{“What do you think, Bhaddiya, are these things skillful or unskillful?”}}\\
\end{addmargin}
\end{absolutelynopagebreak}

\begin{absolutelynopagebreak}
\setstretch{.7}
{\PaliGlossA{“akusalā, bhante”.}}\\
\begin{addmargin}[1em]{2em}
\setstretch{.5}
{\PaliGlossB{“Unskillful, sir.”}}\\
\end{addmargin}
\end{absolutelynopagebreak}

\begin{absolutelynopagebreak}
\setstretch{.7}
{\PaliGlossA{“sāvajjā vā anavajjā vā”ti?}}\\
\begin{addmargin}[1em]{2em}
\setstretch{.5}
{\PaliGlossB{“Blameworthy or blameless?”}}\\
\end{addmargin}
\end{absolutelynopagebreak}

\begin{absolutelynopagebreak}
\setstretch{.7}
{\PaliGlossA{“sāvajjā, bhante”.}}\\
\begin{addmargin}[1em]{2em}
\setstretch{.5}
{\PaliGlossB{“Blameworthy, sir.”}}\\
\end{addmargin}
\end{absolutelynopagebreak}

\begin{absolutelynopagebreak}
\setstretch{.7}
{\PaliGlossA{“viññugarahitā vā viññuppasatthā vā”ti?}}\\
\begin{addmargin}[1em]{2em}
\setstretch{.5}
{\PaliGlossB{“Criticized or praised by sensible people?”}}\\
\end{addmargin}
\end{absolutelynopagebreak}

\begin{absolutelynopagebreak}
\setstretch{.7}
{\PaliGlossA{“viññugarahitā, bhante”.}}\\
\begin{addmargin}[1em]{2em}
\setstretch{.5}
{\PaliGlossB{“Criticized by sensible people, sir.”}}\\
\end{addmargin}
\end{absolutelynopagebreak}

\begin{absolutelynopagebreak}
\setstretch{.7}
{\PaliGlossA{“samattā samādinnā ahitāya dukkhāya saṃvattanti, no vā?}}\\
\begin{addmargin}[1em]{2em}
\setstretch{.5}
{\PaliGlossB{“When you undertake them, do they lead to harm and suffering, or not?}}\\
\end{addmargin}
\end{absolutelynopagebreak}

\begin{absolutelynopagebreak}
\setstretch{.7}
{\PaliGlossA{kathaṃ vā ettha hotī”ti?}}\\
\begin{addmargin}[1em]{2em}
\setstretch{.5}
{\PaliGlossB{Or how do you see this?”}}\\
\end{addmargin}
\end{absolutelynopagebreak}

\begin{absolutelynopagebreak}
\setstretch{.7}
{\PaliGlossA{“samattā, bhante, samādinnā ahitāya dukkhāya saṃvattanti.}}\\
\begin{addmargin}[1em]{2em}
\setstretch{.5}
{\PaliGlossB{“When you undertake them, they lead to harm and suffering.}}\\
\end{addmargin}
\end{absolutelynopagebreak}

\begin{absolutelynopagebreak}
\setstretch{.7}
{\PaliGlossA{evaṃ no ettha hotī”ti.}}\\
\begin{addmargin}[1em]{2em}
\setstretch{.5}
{\PaliGlossB{That’s how we see it.”}}\\
\end{addmargin}
\end{absolutelynopagebreak}

\begin{absolutelynopagebreak}
\setstretch{.7}
{\PaliGlossA{“iti kho, bhaddiya, yaṃ taṃ te avocumhā—}}\\
\begin{addmargin}[1em]{2em}
\setstretch{.5}
{\PaliGlossB{“So, Bhaddiya, when we said:}}\\
\end{addmargin}
\end{absolutelynopagebreak}

\begin{absolutelynopagebreak}
\setstretch{.7}
{\PaliGlossA{etha tumhe, bhaddiya, mā anussavena, mā paramparāya, mā itikirāya, mā piṭakasampadānena, mā takkahetu, mā nayahetu, mā ākāraparivitakkena, mā diṭṭhinijjhānakkhantiyā, mā bhabbarūpatāya, mā ‘samaṇo no garū’ti.}}\\
\begin{addmargin}[1em]{2em}
\setstretch{.5}
{\PaliGlossB{‘Please, Bhaddiya, don’t go by oral transmission, don’t go by lineage, don’t go by testament, don’t go by canonical authority, don’t rely on logic, don’t rely on inference, don’t go by reasoned contemplation, don’t go by the acceptance of a view after consideration, don’t go by the appearance of competence, and don’t think “The ascetic is our respected teacher.”}}\\
\end{addmargin}
\end{absolutelynopagebreak}

\begin{absolutelynopagebreak}
\setstretch{.7}
{\PaliGlossA{yadā tumhe, bhaddiya, attanāva jāneyyātha:}}\\
\begin{addmargin}[1em]{2em}
\setstretch{.5}
{\PaliGlossB{But when you know for yourselves:}}\\
\end{addmargin}
\end{absolutelynopagebreak}

\begin{absolutelynopagebreak}
\setstretch{.7}
{\PaliGlossA{‘ime dhammā akusalā, ime dhammā sāvajjā, ime dhammā viññugarahitā, ime dhammā samattā samādinnā ahitāya dukkhāya saṃvattantīti, atha tumhe, bhaddiya, pajaheyyāthā’ti,}}\\
\begin{addmargin}[1em]{2em}
\setstretch{.5}
{\PaliGlossB{“These things are unskillful, blameworthy, criticized by sensible people, and when you undertake them, they lead to harm and suffering”, then you should give them up.’}}\\
\end{addmargin}
\end{absolutelynopagebreak}

\begin{absolutelynopagebreak}
\setstretch{.7}
{\PaliGlossA{iti yaṃ taṃ vuttaṃ idametaṃ paṭicca vuttaṃ.}}\\
\begin{addmargin}[1em]{2em}
\setstretch{.5}
{\PaliGlossB{That’s what I said, and this is why I said it.}}\\
\end{addmargin}
\end{absolutelynopagebreak}

\begin{absolutelynopagebreak}
\setstretch{.7}
{\PaliGlossA{etha tumhe, bhaddiya, mā anussavena, mā paramparāya, mā itikirāya, mā piṭakasampadānena, mā takkahetu, mā nayahetu, mā ākāraparivitakkena, mā diṭṭhinijjhānakkhantiyā, mā bhabbarūpatāya, mā ‘samaṇo no garū’ti.}}\\
\begin{addmargin}[1em]{2em}
\setstretch{.5}
{\PaliGlossB{Please, Bhaddiya, don’t rely on oral transmission …}}\\
\end{addmargin}
\end{absolutelynopagebreak}

\begin{absolutelynopagebreak}
\setstretch{.7}
{\PaliGlossA{yadā tumhe, bhaddiya, attanāva jāneyyātha:}}\\
\begin{addmargin}[1em]{2em}
\setstretch{.5}
{\PaliGlossB{But when you know for yourselves:}}\\
\end{addmargin}
\end{absolutelynopagebreak}

\begin{absolutelynopagebreak}
\setstretch{.7}
{\PaliGlossA{‘ime dhammā kusalā, ime dhammā anavajjā, ime dhammā viññuppasatthā, ime dhammā samattā samādinnā hitāya sukhāya saṃvattantī’ti, atha tumhe, bhaddiya, upasampajja vihareyyāthāti.}}\\
\begin{addmargin}[1em]{2em}
\setstretch{.5}
{\PaliGlossB{‘These things are skillful, blameless, praised by sensible people, and when you undertake them, they lead to welfare and happiness’, then you should acquire them and keep them.}}\\
\end{addmargin}
\end{absolutelynopagebreak}

\begin{absolutelynopagebreak}
\setstretch{.7}
{\PaliGlossA{taṃ kiṃ maññatha, bhaddiya,}}\\
\begin{addmargin}[1em]{2em}
\setstretch{.5}
{\PaliGlossB{What do you think, Bhaddiya?}}\\
\end{addmargin}
\end{absolutelynopagebreak}

\begin{absolutelynopagebreak}
\setstretch{.7}
{\PaliGlossA{alobho purisassa ajjhattaṃ uppajjamāno uppajjati hitāya vā ahitāya vā”ti?}}\\
\begin{addmargin}[1em]{2em}
\setstretch{.5}
{\PaliGlossB{Does contentment … love … understanding … benevolence come up in a person for their welfare or harm?”}}\\
\end{addmargin}
\end{absolutelynopagebreak}

\begin{absolutelynopagebreak}
\setstretch{.7}
{\PaliGlossA{“hitāya, bhante”.}}\\
\begin{addmargin}[1em]{2em}
\setstretch{.5}
{\PaliGlossB{    -}}\\
\end{addmargin}
\end{absolutelynopagebreak}

\begin{absolutelynopagebreak}
\setstretch{.7}
{\PaliGlossA{“aluddho panāyaṃ, bhaddiya, purisapuggalo lobhena anabhibhūto apariyādinnacitto neva pāṇaṃ hanati, na adinnaṃ ādiyati, na paradāraṃ gacchati, na musā bhaṇati, parampi tathattāya na samādapeti yaṃsa hoti dīgharattaṃ hitāya sukhāyā”ti.}}\\
\begin{addmargin}[1em]{2em}
\setstretch{.5}
{\PaliGlossB{    -}}\\
\end{addmargin}
\end{absolutelynopagebreak}

\begin{absolutelynopagebreak}
\setstretch{.7}
{\PaliGlossA{“evaṃ, bhante”.}}\\
\begin{addmargin}[1em]{2em}
\setstretch{.5}
{\PaliGlossB{    -}}\\
\end{addmargin}
\end{absolutelynopagebreak}

\begin{absolutelynopagebreak}
\setstretch{.7}
{\PaliGlossA{“taṃ kiṃ maññatha, bhaddiya, adoso purisassa … pe … amoho purisassa … pe … asārambho purisassa ajjhattaṃ uppajjamāno uppajjati hitāya vā ahitāya vā”ti?}}\\
\begin{addmargin}[1em]{2em}
\setstretch{.5}
{\PaliGlossB{    -}}\\
\end{addmargin}
\end{absolutelynopagebreak}

\begin{absolutelynopagebreak}
\setstretch{.7}
{\PaliGlossA{“hitāya, bhante”.}}\\
\begin{addmargin}[1em]{2em}
\setstretch{.5}
{\PaliGlossB{“Welfare, sir.”}}\\
\end{addmargin}
\end{absolutelynopagebreak}

\begin{absolutelynopagebreak}
\setstretch{.7}
{\PaliGlossA{“asāraddho panāyaṃ, bhaddiya, purisapuggalo sārambhena anabhibhūto apariyādinnacitto neva pāṇaṃ hanati, na adinnaṃ ādiyati, na paradāraṃ gacchati, na musā bhaṇati, parampi tathattāya na samādapeti yaṃsa hoti dīgharattaṃ hitāya sukhāyā”ti.}}\\
\begin{addmargin}[1em]{2em}
\setstretch{.5}
{\PaliGlossB{“An individual who is benevolent—not overcome by aggression—doesn’t kill living creatures, steal, commit adultery, lie, or encourage others to do the same. Is that for their lasting welfare and happiness?”}}\\
\end{addmargin}
\end{absolutelynopagebreak}

\begin{absolutelynopagebreak}
\setstretch{.7}
{\PaliGlossA{“evaṃ, bhante”.}}\\
\begin{addmargin}[1em]{2em}
\setstretch{.5}
{\PaliGlossB{“Yes, sir.”}}\\
\end{addmargin}
\end{absolutelynopagebreak}

\begin{absolutelynopagebreak}
\setstretch{.7}
{\PaliGlossA{“taṃ kiṃ maññatha, bhaddiya, ime dhammā kusalā vā akusalā vā”ti?}}\\
\begin{addmargin}[1em]{2em}
\setstretch{.5}
{\PaliGlossB{“What do you think, Bhaddiya, are these things skillful or unskillful?”}}\\
\end{addmargin}
\end{absolutelynopagebreak}

\begin{absolutelynopagebreak}
\setstretch{.7}
{\PaliGlossA{“kusalā, bhante”.}}\\
\begin{addmargin}[1em]{2em}
\setstretch{.5}
{\PaliGlossB{“Skillful, sir.”}}\\
\end{addmargin}
\end{absolutelynopagebreak}

\begin{absolutelynopagebreak}
\setstretch{.7}
{\PaliGlossA{“sāvajjā vā anavajjā vā”ti?}}\\
\begin{addmargin}[1em]{2em}
\setstretch{.5}
{\PaliGlossB{“Blameworthy or blameless?”}}\\
\end{addmargin}
\end{absolutelynopagebreak}

\begin{absolutelynopagebreak}
\setstretch{.7}
{\PaliGlossA{“anavajjā, bhante”.}}\\
\begin{addmargin}[1em]{2em}
\setstretch{.5}
{\PaliGlossB{“Blameless, sir.”}}\\
\end{addmargin}
\end{absolutelynopagebreak}

\begin{absolutelynopagebreak}
\setstretch{.7}
{\PaliGlossA{“viññugarahitā vā viññuppasatthā vā”ti?}}\\
\begin{addmargin}[1em]{2em}
\setstretch{.5}
{\PaliGlossB{“Criticized or praised by sensible people?”}}\\
\end{addmargin}
\end{absolutelynopagebreak}

\begin{absolutelynopagebreak}
\setstretch{.7}
{\PaliGlossA{“viññuppasatthā, bhante”.}}\\
\begin{addmargin}[1em]{2em}
\setstretch{.5}
{\PaliGlossB{“Praised by sensible people, sir.”}}\\
\end{addmargin}
\end{absolutelynopagebreak}

\begin{absolutelynopagebreak}
\setstretch{.7}
{\PaliGlossA{“samattā samādinnā hitāya sukhāya saṃvattanti no vā?}}\\
\begin{addmargin}[1em]{2em}
\setstretch{.5}
{\PaliGlossB{“When you undertake them, do they lead to welfare and happiness, or not?}}\\
\end{addmargin}
\end{absolutelynopagebreak}

\begin{absolutelynopagebreak}
\setstretch{.7}
{\PaliGlossA{kathaṃ vā ettha hotī”ti?}}\\
\begin{addmargin}[1em]{2em}
\setstretch{.5}
{\PaliGlossB{Or how do you see this?”}}\\
\end{addmargin}
\end{absolutelynopagebreak}

\begin{absolutelynopagebreak}
\setstretch{.7}
{\PaliGlossA{“samattā, bhante, samādinnā hitāya sukhāya saṃvattanti.}}\\
\begin{addmargin}[1em]{2em}
\setstretch{.5}
{\PaliGlossB{“When you undertake them, they lead to welfare and happiness.}}\\
\end{addmargin}
\end{absolutelynopagebreak}

\begin{absolutelynopagebreak}
\setstretch{.7}
{\PaliGlossA{evaṃ no ettha hotī”ti.}}\\
\begin{addmargin}[1em]{2em}
\setstretch{.5}
{\PaliGlossB{That’s how we see it.”}}\\
\end{addmargin}
\end{absolutelynopagebreak}

\begin{absolutelynopagebreak}
\setstretch{.7}
{\PaliGlossA{“iti kho, bhaddiya, yaṃ taṃ te avocumhā—}}\\
\begin{addmargin}[1em]{2em}
\setstretch{.5}
{\PaliGlossB{“So, Bhaddiya, when we said:}}\\
\end{addmargin}
\end{absolutelynopagebreak}

\begin{absolutelynopagebreak}
\setstretch{.7}
{\PaliGlossA{etha tumhe, bhaddiya, mā anussavena, mā paramparāya, mā itikirāya, mā piṭakasampadānena, mā takkahetu, mā nayahetu, mā ākāraparivitakkena, mā diṭṭhinijjhānakkhantiyā, mā bhabbarūpatāya, mā ‘samaṇo no garū’ti.}}\\
\begin{addmargin}[1em]{2em}
\setstretch{.5}
{\PaliGlossB{‘Please, Bhaddiya, don’t go by oral transmission, don’t go by lineage, don’t go by testament, don’t go by canonical authority, don’t rely on logic, don’t rely on inference, don’t go by reasoned contemplation, don’t go by the acceptance of a view after consideration, don’t go by the appearance of competence, and don’t think “The ascetic is our respected teacher.”}}\\
\end{addmargin}
\end{absolutelynopagebreak}

\begin{absolutelynopagebreak}
\setstretch{.7}
{\PaliGlossA{yadā tumhe, bhaddiya, attanāva jāneyyātha:}}\\
\begin{addmargin}[1em]{2em}
\setstretch{.5}
{\PaliGlossB{But when you know for yourselves:}}\\
\end{addmargin}
\end{absolutelynopagebreak}

\begin{absolutelynopagebreak}
\setstretch{.7}
{\PaliGlossA{‘ime dhammā kusalā, ime dhammā anavajjā, ime dhammā viññuppasatthā, ime dhammā samattā samādinnā hitāya sukhāya saṃvattantīti, atha tumhe, bhaddiya, upasampajja vihareyyāthā’ti,}}\\
\begin{addmargin}[1em]{2em}
\setstretch{.5}
{\PaliGlossB{“These things are skillful, blameless, praised by sensible people, and when you undertake them, they lead to welfare and happiness”, then you should acquire them and keep them.’}}\\
\end{addmargin}
\end{absolutelynopagebreak}

\begin{absolutelynopagebreak}
\setstretch{.7}
{\PaliGlossA{iti yaṃ taṃ vuttaṃ idametaṃ paṭicca vuttaṃ.}}\\
\begin{addmargin}[1em]{2em}
\setstretch{.5}
{\PaliGlossB{That’s what I said, and this is why I said it.}}\\
\end{addmargin}
\end{absolutelynopagebreak}

\begin{absolutelynopagebreak}
\setstretch{.7}
{\PaliGlossA{ye kho te, bhaddiya, loke santo sappurisā te sāvakaṃ evaṃ samādapenti:}}\\
\begin{addmargin}[1em]{2em}
\setstretch{.5}
{\PaliGlossB{The good people in the world encourage their disciples:}}\\
\end{addmargin}
\end{absolutelynopagebreak}

\begin{absolutelynopagebreak}
\setstretch{.7}
{\PaliGlossA{‘ehi tvaṃ, ambho purisa, lobhaṃ vineyya viharāhi.}}\\
\begin{addmargin}[1em]{2em}
\setstretch{.5}
{\PaliGlossB{‘Please, mister, live rid of greed.}}\\
\end{addmargin}
\end{absolutelynopagebreak}

\begin{absolutelynopagebreak}
\setstretch{.7}
{\PaliGlossA{lobhaṃ vineyya viharanto na lobhajaṃ kammaṃ karissasi kāyena vācāya manasā.}}\\
\begin{addmargin}[1em]{2em}
\setstretch{.5}
{\PaliGlossB{Then you won’t act out of greed by way of body, speech, or mind.}}\\
\end{addmargin}
\end{absolutelynopagebreak}

\begin{absolutelynopagebreak}
\setstretch{.7}
{\PaliGlossA{dosaṃ vineyya viharāhi.}}\\
\begin{addmargin}[1em]{2em}
\setstretch{.5}
{\PaliGlossB{Live rid of hate … delusion … aggression.}}\\
\end{addmargin}
\end{absolutelynopagebreak}

\begin{absolutelynopagebreak}
\setstretch{.7}
{\PaliGlossA{dosaṃ vineyya viharanto na dosajaṃ kammaṃ karissasi kāyena vācāya manasā.}}\\
\begin{addmargin}[1em]{2em}
\setstretch{.5}
{\PaliGlossB{Then you won’t act out of hate … delusion … aggression by way of body, speech, or mind.”}}\\
\end{addmargin}
\end{absolutelynopagebreak}

\begin{absolutelynopagebreak}
\setstretch{.7}
{\PaliGlossA{mohaṃ vineyya viharāhi.}}\\
\begin{addmargin}[1em]{2em}
\setstretch{.5}
{\PaliGlossB{    -}}\\
\end{addmargin}
\end{absolutelynopagebreak}

\begin{absolutelynopagebreak}
\setstretch{.7}
{\PaliGlossA{mohaṃ vineyya viharanto na mohajaṃ kammaṃ karissasi kāyena vācāya manasā.}}\\
\begin{addmargin}[1em]{2em}
\setstretch{.5}
{\PaliGlossB{    -}}\\
\end{addmargin}
\end{absolutelynopagebreak}

\begin{absolutelynopagebreak}
\setstretch{.7}
{\PaliGlossA{sārambhaṃ vineyya viharāhi.}}\\
\begin{addmargin}[1em]{2em}
\setstretch{.5}
{\PaliGlossB{    -}}\\
\end{addmargin}
\end{absolutelynopagebreak}

\begin{absolutelynopagebreak}
\setstretch{.7}
{\PaliGlossA{sārambhaṃ vineyya viharanto na sārambhajaṃ kammaṃ karissasi kāyena vācāya manasā’”ti.}}\\
\begin{addmargin}[1em]{2em}
\setstretch{.5}
{\PaliGlossB{    -}}\\
\end{addmargin}
\end{absolutelynopagebreak}

\begin{absolutelynopagebreak}
\setstretch{.7}
{\PaliGlossA{evaṃ vutte, bhaddiyo licchavi bhagavantaṃ etadavoca:}}\\
\begin{addmargin}[1em]{2em}
\setstretch{.5}
{\PaliGlossB{When he said this, Bhaddiya the Licchavi said to the Buddha,}}\\
\end{addmargin}
\end{absolutelynopagebreak}

\begin{absolutelynopagebreak}
\setstretch{.7}
{\PaliGlossA{“abhikkantaṃ, bhante … pe … upāsakaṃ maṃ, bhante, bhagavā dhāretu ajjatagge pāṇupetaṃ saraṇaṃ gatan”ti.}}\\
\begin{addmargin}[1em]{2em}
\setstretch{.5}
{\PaliGlossB{“Excellent, sir! … From this day forth, may the Buddha remember me as a lay follower who has gone for refuge for life.”}}\\
\end{addmargin}
\end{absolutelynopagebreak}

\begin{absolutelynopagebreak}
\setstretch{.7}
{\PaliGlossA{“api nu tāhaṃ, bhaddiya, evaṃ avacaṃ:}}\\
\begin{addmargin}[1em]{2em}
\setstretch{.5}
{\PaliGlossB{“Well, Bhaddiya, did I say to you:}}\\
\end{addmargin}
\end{absolutelynopagebreak}

\begin{absolutelynopagebreak}
\setstretch{.7}
{\PaliGlossA{‘ehi me tvaṃ, bhaddiya, sāvako hohi;}}\\
\begin{addmargin}[1em]{2em}
\setstretch{.5}
{\PaliGlossB{‘Please, Bhaddiya, be my disciple,}}\\
\end{addmargin}
\end{absolutelynopagebreak}

\begin{absolutelynopagebreak}
\setstretch{.7}
{\PaliGlossA{ahaṃ satthā bhavissāmī’”ti?}}\\
\begin{addmargin}[1em]{2em}
\setstretch{.5}
{\PaliGlossB{and I will be your teacher’?”}}\\
\end{addmargin}
\end{absolutelynopagebreak}

\begin{absolutelynopagebreak}
\setstretch{.7}
{\PaliGlossA{“no hetaṃ, bhante”.}}\\
\begin{addmargin}[1em]{2em}
\setstretch{.5}
{\PaliGlossB{“No, sir.”}}\\
\end{addmargin}
\end{absolutelynopagebreak}

\begin{absolutelynopagebreak}
\setstretch{.7}
{\PaliGlossA{“evaṃvādiṃ kho maṃ, bhaddiya, evamakkhāyiṃ eke samaṇabrāhmaṇā asatā tucchā musā abhūtena abbhācikkhanti:}}\\
\begin{addmargin}[1em]{2em}
\setstretch{.5}
{\PaliGlossB{“Though I speak and explain like this, certain ascetics and brahmins misrepresent me with the false, hollow, lying, untruthful claim:}}\\
\end{addmargin}
\end{absolutelynopagebreak}

\begin{absolutelynopagebreak}
\setstretch{.7}
{\PaliGlossA{‘māyāvī samaṇo gotamo āvaṭṭaniṃ māyaṃ jānāti yāya aññatitthiyānaṃ sāvake āvaṭṭetī’”ti.}}\\
\begin{addmargin}[1em]{2em}
\setstretch{.5}
{\PaliGlossB{‘The ascetic Gotama is a magician. He knows a conversion magic, and uses it to convert the disciples of those who follow other paths.’”}}\\
\end{addmargin}
\end{absolutelynopagebreak}

\begin{absolutelynopagebreak}
\setstretch{.7}
{\PaliGlossA{“bhaddikā, bhante, āvaṭṭanī māyā.}}\\
\begin{addmargin}[1em]{2em}
\setstretch{.5}
{\PaliGlossB{“Sir, this conversion magic is excellent.}}\\
\end{addmargin}
\end{absolutelynopagebreak}

\begin{absolutelynopagebreak}
\setstretch{.7}
{\PaliGlossA{kalyāṇī, bhante, āvaṭṭanī māyā.}}\\
\begin{addmargin}[1em]{2em}
\setstretch{.5}
{\PaliGlossB{This conversion magic is lovely!}}\\
\end{addmargin}
\end{absolutelynopagebreak}

\begin{absolutelynopagebreak}
\setstretch{.7}
{\PaliGlossA{piyā me, bhante, ñātisālohitā imāya āvaṭṭaniyā āvaṭṭeyyuṃ, piyānampi me assa ñātisālohitānaṃ dīgharattaṃ hitāya sukhāya.}}\\
\begin{addmargin}[1em]{2em}
\setstretch{.5}
{\PaliGlossB{If my loved ones—relatives and kin—were to be converted by this, it would be for their lasting welfare and happiness.}}\\
\end{addmargin}
\end{absolutelynopagebreak}

\begin{absolutelynopagebreak}
\setstretch{.7}
{\PaliGlossA{sabbe cepi, bhante, khattiyā imāya āvaṭṭaniyā āvaṭṭeyyuṃ, sabbesampissa khattiyānaṃ dīgharattaṃ hitāya sukhāya.}}\\
\begin{addmargin}[1em]{2em}
\setstretch{.5}
{\PaliGlossB{If all the aristocrats, brahmins, merchants, and workers were to be converted by this, it would be for their lasting welfare and happiness.”}}\\
\end{addmargin}
\end{absolutelynopagebreak}

\begin{absolutelynopagebreak}
\setstretch{.7}
{\PaliGlossA{sabbe cepi, bhante, brāhmaṇā … vessā … suddā imāya āvaṭṭaniyā āvaṭṭeyyuṃ, sabbesampissa suddānaṃ dīgharattaṃ hitāya sukhāyā”ti.}}\\
\begin{addmargin}[1em]{2em}
\setstretch{.5}
{\PaliGlossB{    -}}\\
\end{addmargin}
\end{absolutelynopagebreak}

\begin{absolutelynopagebreak}
\setstretch{.7}
{\PaliGlossA{“evametaṃ, bhaddiya, evametaṃ, bhaddiya.}}\\
\begin{addmargin}[1em]{2em}
\setstretch{.5}
{\PaliGlossB{“That’s so true, Bhaddiya! That’s so true, Bhaddiya!}}\\
\end{addmargin}
\end{absolutelynopagebreak}

\begin{absolutelynopagebreak}
\setstretch{.7}
{\PaliGlossA{sabbe cepi, bhaddiya, khattiyā imāya āvaṭṭaniyā āvaṭṭeyyuṃ akusaladhammappahānāya kusaladhammūpasampadāya, sabbesampissa khattiyānaṃ dīgharattaṃ hitāya sukhāya.}}\\
\begin{addmargin}[1em]{2em}
\setstretch{.5}
{\PaliGlossB{If all the aristocrats, brahmins, merchants, and workers were to be converted by this, it would be for their lasting welfare and happiness.}}\\
\end{addmargin}
\end{absolutelynopagebreak}

\begin{absolutelynopagebreak}
\setstretch{.7}
{\PaliGlossA{sabbe cepi, bhaddiya, brāhmaṇā … vessā … suddā āvaṭṭeyyuṃ akusaladhammappahānāya kusaladhammūpasampadāya, sabbesampissa suddānaṃ dīgharattaṃ hitāya sukhāya.}}\\
\begin{addmargin}[1em]{2em}
\setstretch{.5}
{\PaliGlossB{    -}}\\
\end{addmargin}
\end{absolutelynopagebreak}

\begin{absolutelynopagebreak}
\setstretch{.7}
{\PaliGlossA{sadevako cepi, bhaddiya, loko samārako sabrahmako sassamaṇabrāhmaṇī pajā sadevamanussā imāya āvaṭṭaniyā āvaṭṭeyyuṃ akusaladhammappahānāya kusaladhammūpasampadāya, sadevakassapissa lokassa samārakassa sabrahmakassa sassamaṇabrāhmaṇiyā pajāya sadevamanussāya dīgharattaṃ hitāya sukhāya.}}\\
\begin{addmargin}[1em]{2em}
\setstretch{.5}
{\PaliGlossB{If the whole world—with its gods, Māras and Brahmās, this population with its ascetics and brahmins, gods and humans—were to be converted by this, for giving up unskillful qualities and embracing skillful qualities, it would be for their lasting welfare and happiness.}}\\
\end{addmargin}
\end{absolutelynopagebreak}

\begin{absolutelynopagebreak}
\setstretch{.7}
{\PaliGlossA{ime cepi, bhaddiya, mahāsālā imāya āvaṭṭaniyā āvaṭṭeyyuṃ akusaladhammappahānāya kusaladhammūpasampadāya, imesampissa mahāsālānaṃ dīgharattaṃ hitāya sukhāya ().}}\\
\begin{addmargin}[1em]{2em}
\setstretch{.5}
{\PaliGlossB{If these great sal trees were to be converted by this, for giving up unskillful qualities and embracing skillful qualities, it would be for their lasting welfare and happiness—if they were sentient.}}\\
\end{addmargin}
\end{absolutelynopagebreak}

\begin{absolutelynopagebreak}
\setstretch{.7}
{\PaliGlossA{ko pana vādo manussabhūtassā”ti.}}\\
\begin{addmargin}[1em]{2em}
\setstretch{.5}
{\PaliGlossB{How much more then a human being!”}}\\
\end{addmargin}
\end{absolutelynopagebreak}

\begin{absolutelynopagebreak}
\setstretch{.7}
{\PaliGlossA{tatiyaṃ.}}\\
\begin{addmargin}[1em]{2em}
\setstretch{.5}
{\PaliGlossB{    -}}\\
\end{addmargin}
\end{absolutelynopagebreak}
