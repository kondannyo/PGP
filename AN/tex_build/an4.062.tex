
\begin{absolutelynopagebreak}
\setstretch{.7}
{\PaliGlossA{aṅguttara nikāya 4}}\\
\begin{addmargin}[1em]{2em}
\setstretch{.5}
{\PaliGlossB{Numbered Discourses 4}}\\
\end{addmargin}
\end{absolutelynopagebreak}

\begin{absolutelynopagebreak}
\setstretch{.7}
{\PaliGlossA{7. pattakammavagga}}\\
\begin{addmargin}[1em]{2em}
\setstretch{.5}
{\PaliGlossB{7. Deeds of Substance}}\\
\end{addmargin}
\end{absolutelynopagebreak}

\begin{absolutelynopagebreak}
\setstretch{.7}
{\PaliGlossA{62. ānaṇyasutta}}\\
\begin{addmargin}[1em]{2em}
\setstretch{.5}
{\PaliGlossB{62. Debtlessness}}\\
\end{addmargin}
\end{absolutelynopagebreak}

\begin{absolutelynopagebreak}
\setstretch{.7}
{\PaliGlossA{atha kho anāthapiṇḍiko gahapati yena bhagavā tenupasaṅkami; upasaṅkamitvā bhagavantaṃ abhivādetvā ekamantaṃ nisīdi. ekamantaṃ nisinnaṃ kho anāthapiṇḍikaṃ gahapatiṃ bhagavā etadavoca:}}\\
\begin{addmargin}[1em]{2em}
\setstretch{.5}
{\PaliGlossB{Then the householder Anāthapiṇḍika went up to the Buddha, bowed, and sat down to one side. The Buddha said to him:}}\\
\end{addmargin}
\end{absolutelynopagebreak}

\begin{absolutelynopagebreak}
\setstretch{.7}
{\PaliGlossA{“cattārimāni, gahapati, sukhāni adhigamanīyāni gihinā kāmabhoginā kālena kālaṃ samayena samayaṃ upādāya.}}\\
\begin{addmargin}[1em]{2em}
\setstretch{.5}
{\PaliGlossB{“Householder, these four kinds of happiness can be earned by a layperson who enjoys sensual pleasures, depending on time and occasion.}}\\
\end{addmargin}
\end{absolutelynopagebreak}

\begin{absolutelynopagebreak}
\setstretch{.7}
{\PaliGlossA{katamāni cattāri?}}\\
\begin{addmargin}[1em]{2em}
\setstretch{.5}
{\PaliGlossB{What four?}}\\
\end{addmargin}
\end{absolutelynopagebreak}

\begin{absolutelynopagebreak}
\setstretch{.7}
{\PaliGlossA{atthisukhaṃ, bhogasukhaṃ, ānaṇyasukhaṃ, anavajjasukhaṃ.}}\\
\begin{addmargin}[1em]{2em}
\setstretch{.5}
{\PaliGlossB{The happiness of ownership, using wealth, debtlessness, and blamelessness.}}\\
\end{addmargin}
\end{absolutelynopagebreak}

\begin{absolutelynopagebreak}
\setstretch{.7}
{\PaliGlossA{katamañca, gahapati, atthisukhaṃ?}}\\
\begin{addmargin}[1em]{2em}
\setstretch{.5}
{\PaliGlossB{And what is the happiness of ownership?}}\\
\end{addmargin}
\end{absolutelynopagebreak}

\begin{absolutelynopagebreak}
\setstretch{.7}
{\PaliGlossA{idha, gahapati, kulaputtassa bhogā honti uṭṭhānavīriyādhigatā bāhābalaparicitā sedāvakkhittā dhammikā dhammaladdhā.}}\\
\begin{addmargin}[1em]{2em}
\setstretch{.5}
{\PaliGlossB{It’s when a gentleman owns legitimate wealth that he has earned by his own efforts and initiative, built up with his own hands, gathered by the sweat of the brow.}}\\
\end{addmargin}
\end{absolutelynopagebreak}

\begin{absolutelynopagebreak}
\setstretch{.7}
{\PaliGlossA{so ‘bhogā me atthi uṭṭhānavīriyādhigatā bāhābalaparicitā sedāvakkhittā dhammikā dhammaladdhā’ti adhigacchati sukhaṃ, adhigacchati somanassaṃ.}}\\
\begin{addmargin}[1em]{2em}
\setstretch{.5}
{\PaliGlossB{When he reflects on this, he’s filled with pleasure and happiness.}}\\
\end{addmargin}
\end{absolutelynopagebreak}

\begin{absolutelynopagebreak}
\setstretch{.7}
{\PaliGlossA{idaṃ vuccati, gahapati, atthisukhaṃ.}}\\
\begin{addmargin}[1em]{2em}
\setstretch{.5}
{\PaliGlossB{This is called ‘the happiness of ownership’.}}\\
\end{addmargin}
\end{absolutelynopagebreak}

\begin{absolutelynopagebreak}
\setstretch{.7}
{\PaliGlossA{katamañca, gahapati, bhogasukhaṃ?}}\\
\begin{addmargin}[1em]{2em}
\setstretch{.5}
{\PaliGlossB{And what is the happiness of using wealth?}}\\
\end{addmargin}
\end{absolutelynopagebreak}

\begin{absolutelynopagebreak}
\setstretch{.7}
{\PaliGlossA{idha, gahapati, kulaputto uṭṭhānavīriyādhigatehi bhogehi bāhābalaparicitehi sedāvakkhittehi dhammikehi dhammaladdhehi paribhuñjati puññāni ca karoti.}}\\
\begin{addmargin}[1em]{2em}
\setstretch{.5}
{\PaliGlossB{It’s when a gentleman uses his legitimate wealth, and makes merit with it.}}\\
\end{addmargin}
\end{absolutelynopagebreak}

\begin{absolutelynopagebreak}
\setstretch{.7}
{\PaliGlossA{so ‘uṭṭhānavīriyādhigatehi bhogehi bāhābalaparicitehi sedāvakkhittehi dhammikehi dhammaladdhehi paribhuñjāmi puññāni ca karomī’ti adhigacchati sukhaṃ, adhigacchati somanassaṃ.}}\\
\begin{addmargin}[1em]{2em}
\setstretch{.5}
{\PaliGlossB{When he reflects on this, he’s filled with pleasure and happiness.}}\\
\end{addmargin}
\end{absolutelynopagebreak}

\begin{absolutelynopagebreak}
\setstretch{.7}
{\PaliGlossA{idaṃ vuccati, gahapati, bhogasukhaṃ.}}\\
\begin{addmargin}[1em]{2em}
\setstretch{.5}
{\PaliGlossB{This is called ‘the happiness of using wealth’.}}\\
\end{addmargin}
\end{absolutelynopagebreak}

\begin{absolutelynopagebreak}
\setstretch{.7}
{\PaliGlossA{katamañca, gahapati, ānaṇyasukhaṃ?}}\\
\begin{addmargin}[1em]{2em}
\setstretch{.5}
{\PaliGlossB{And what is the happiness of debtlessness?}}\\
\end{addmargin}
\end{absolutelynopagebreak}

\begin{absolutelynopagebreak}
\setstretch{.7}
{\PaliGlossA{idha, gahapati, kulaputto na kassaci kiñci dhāreti appaṃ vā bahuṃ vā.}}\\
\begin{addmargin}[1em]{2em}
\setstretch{.5}
{\PaliGlossB{It’s when a gentleman owes no debt, large or small, to anyone.}}\\
\end{addmargin}
\end{absolutelynopagebreak}

\begin{absolutelynopagebreak}
\setstretch{.7}
{\PaliGlossA{so ‘na kassaci kiñci dhāremi appaṃ vā bahuṃ vā’ti adhigacchati sukhaṃ, adhigacchati somanassaṃ.}}\\
\begin{addmargin}[1em]{2em}
\setstretch{.5}
{\PaliGlossB{When he reflects on this, he’s filled with pleasure and happiness.}}\\
\end{addmargin}
\end{absolutelynopagebreak}

\begin{absolutelynopagebreak}
\setstretch{.7}
{\PaliGlossA{idaṃ vuccati, gahapati, ānaṇyasukhaṃ.}}\\
\begin{addmargin}[1em]{2em}
\setstretch{.5}
{\PaliGlossB{This is called ‘the happiness of debtlessness’.}}\\
\end{addmargin}
\end{absolutelynopagebreak}

\begin{absolutelynopagebreak}
\setstretch{.7}
{\PaliGlossA{katamañca, gahapati, anavajjasukhaṃ?}}\\
\begin{addmargin}[1em]{2em}
\setstretch{.5}
{\PaliGlossB{And what is the happiness of blamelessness?}}\\
\end{addmargin}
\end{absolutelynopagebreak}

\begin{absolutelynopagebreak}
\setstretch{.7}
{\PaliGlossA{idha, gahapati, ariyasāvako anavajjena kāyakammena samannāgato hoti, anavajjena vacīkammena samannāgato hoti, anavajjena manokammena samannāgato hoti.}}\\
\begin{addmargin}[1em]{2em}
\setstretch{.5}
{\PaliGlossB{It’s when a noble disciple has blameless conduct by way of body, speech, and mind.}}\\
\end{addmargin}
\end{absolutelynopagebreak}

\begin{absolutelynopagebreak}
\setstretch{.7}
{\PaliGlossA{so ‘anavajjenamhi kāyakammena samannāgato, anavajjena vacīkammena samannāgato, anavajjena manokammena samannāgato’ti adhigacchati sukhaṃ, adhigacchati somanassaṃ.}}\\
\begin{addmargin}[1em]{2em}
\setstretch{.5}
{\PaliGlossB{When he reflects on this, he’s filled with pleasure and happiness.}}\\
\end{addmargin}
\end{absolutelynopagebreak}

\begin{absolutelynopagebreak}
\setstretch{.7}
{\PaliGlossA{idaṃ vuccati, gahapati, anavajjasukhaṃ.}}\\
\begin{addmargin}[1em]{2em}
\setstretch{.5}
{\PaliGlossB{This is called ‘the happiness of blamelessness’.}}\\
\end{addmargin}
\end{absolutelynopagebreak}

\begin{absolutelynopagebreak}
\setstretch{.7}
{\PaliGlossA{imāni kho, gahapati, cattāri sukhāni adhigamanīyāni gihinā kāmabhoginā kālena kālaṃ samayena samayaṃ upādāyāti.}}\\
\begin{addmargin}[1em]{2em}
\setstretch{.5}
{\PaliGlossB{These four kinds of happiness can be earned by a layperson who enjoys sensual pleasures, depending on time and occasion.}}\\
\end{addmargin}
\end{absolutelynopagebreak}

\begin{absolutelynopagebreak}
\setstretch{.7}
{\PaliGlossA{ānaṇyasukhaṃ ñatvāna,}}\\
\begin{addmargin}[1em]{2em}
\setstretch{.5}
{\PaliGlossB{Knowing the happiness of debtlessness,}}\\
\end{addmargin}
\end{absolutelynopagebreak}

\begin{absolutelynopagebreak}
\setstretch{.7}
{\PaliGlossA{atho atthisukhaṃ paraṃ;}}\\
\begin{addmargin}[1em]{2em}
\setstretch{.5}
{\PaliGlossB{and the extra happiness of possession,}}\\
\end{addmargin}
\end{absolutelynopagebreak}

\begin{absolutelynopagebreak}
\setstretch{.7}
{\PaliGlossA{bhuñjaṃ bhogasukhaṃ macco,}}\\
\begin{addmargin}[1em]{2em}
\setstretch{.5}
{\PaliGlossB{a mortal enjoying the happiness of using wealth,}}\\
\end{addmargin}
\end{absolutelynopagebreak}

\begin{absolutelynopagebreak}
\setstretch{.7}
{\PaliGlossA{tato paññā vipassati.}}\\
\begin{addmargin}[1em]{2em}
\setstretch{.5}
{\PaliGlossB{then sees clearly with wisdom.}}\\
\end{addmargin}
\end{absolutelynopagebreak}

\begin{absolutelynopagebreak}
\setstretch{.7}
{\PaliGlossA{vipassamāno jānāti,}}\\
\begin{addmargin}[1em]{2em}
\setstretch{.5}
{\PaliGlossB{Seeing clearly, a clever person knows}}\\
\end{addmargin}
\end{absolutelynopagebreak}

\begin{absolutelynopagebreak}
\setstretch{.7}
{\PaliGlossA{ubho bhoge sumedhaso;}}\\
\begin{addmargin}[1em]{2em}
\setstretch{.5}
{\PaliGlossB{both kinds of happiness:}}\\
\end{addmargin}
\end{absolutelynopagebreak}

\begin{absolutelynopagebreak}
\setstretch{.7}
{\PaliGlossA{anavajjasukhassetaṃ,}}\\
\begin{addmargin}[1em]{2em}
\setstretch{.5}
{\PaliGlossB{the other kind is not worth a sixteenth part}}\\
\end{addmargin}
\end{absolutelynopagebreak}

\begin{absolutelynopagebreak}
\setstretch{.7}
{\PaliGlossA{kalaṃ nāgghati soḷasin”ti.}}\\
\begin{addmargin}[1em]{2em}
\setstretch{.5}
{\PaliGlossB{of the happiness of blamelessness.”}}\\
\end{addmargin}
\end{absolutelynopagebreak}

\begin{absolutelynopagebreak}
\setstretch{.7}
{\PaliGlossA{dutiyaṃ.}}\\
\begin{addmargin}[1em]{2em}
\setstretch{.5}
{\PaliGlossB{    -}}\\
\end{addmargin}
\end{absolutelynopagebreak}
