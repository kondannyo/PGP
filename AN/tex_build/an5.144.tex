
\begin{absolutelynopagebreak}
\setstretch{.7}
{\PaliGlossA{aṅguttara nikāya 5}}\\
\begin{addmargin}[1em]{2em}
\setstretch{.5}
{\PaliGlossB{Numbered Discourses 5}}\\
\end{addmargin}
\end{absolutelynopagebreak}

\begin{absolutelynopagebreak}
\setstretch{.7}
{\PaliGlossA{15. tikaṇḍakīvagga}}\\
\begin{addmargin}[1em]{2em}
\setstretch{.5}
{\PaliGlossB{15. At Tikaṇḍakī}}\\
\end{addmargin}
\end{absolutelynopagebreak}

\begin{absolutelynopagebreak}
\setstretch{.7}
{\PaliGlossA{144. tikaṇḍakīsutta}}\\
\begin{addmargin}[1em]{2em}
\setstretch{.5}
{\PaliGlossB{144. At Tikaṇḍakī}}\\
\end{addmargin}
\end{absolutelynopagebreak}

\begin{absolutelynopagebreak}
\setstretch{.7}
{\PaliGlossA{ekaṃ samayaṃ bhagavā sākete viharati tikaṇḍakīvane.}}\\
\begin{addmargin}[1em]{2em}
\setstretch{.5}
{\PaliGlossB{At one time the Buddha was staying near Sāketa, in Tikaṇḍakī Wood.}}\\
\end{addmargin}
\end{absolutelynopagebreak}

\begin{absolutelynopagebreak}
\setstretch{.7}
{\PaliGlossA{tatra kho bhagavā bhikkhū āmantesi:}}\\
\begin{addmargin}[1em]{2em}
\setstretch{.5}
{\PaliGlossB{There the Buddha addressed the mendicants,}}\\
\end{addmargin}
\end{absolutelynopagebreak}

\begin{absolutelynopagebreak}
\setstretch{.7}
{\PaliGlossA{“bhikkhavo”ti.}}\\
\begin{addmargin}[1em]{2em}
\setstretch{.5}
{\PaliGlossB{“Mendicants!”}}\\
\end{addmargin}
\end{absolutelynopagebreak}

\begin{absolutelynopagebreak}
\setstretch{.7}
{\PaliGlossA{“bhadante”ti te bhikkhū bhagavato paccassosuṃ.}}\\
\begin{addmargin}[1em]{2em}
\setstretch{.5}
{\PaliGlossB{“Venerable sir,” they replied.}}\\
\end{addmargin}
\end{absolutelynopagebreak}

\begin{absolutelynopagebreak}
\setstretch{.7}
{\PaliGlossA{bhagavā etadavoca:}}\\
\begin{addmargin}[1em]{2em}
\setstretch{.5}
{\PaliGlossB{The Buddha said this:}}\\
\end{addmargin}
\end{absolutelynopagebreak}

\begin{absolutelynopagebreak}
\setstretch{.7}
{\PaliGlossA{“sādhu, bhikkhave, bhikkhu kālena kālaṃ appaṭikūle paṭikūlasaññī vihareyya.}}\\
\begin{addmargin}[1em]{2em}
\setstretch{.5}
{\PaliGlossB{“Mendicants, a mendicant would do well to meditate from time to time perceiving the following:}}\\
\end{addmargin}
\end{absolutelynopagebreak}

\begin{absolutelynopagebreak}
\setstretch{.7}
{\PaliGlossA{appaṭikūle paṭikūlasaññī vihareyya.}}\\
\begin{addmargin}[1em]{2em}
\setstretch{.5}
{\PaliGlossB{the repulsive in the unrepulsive,}}\\
\end{addmargin}
\end{absolutelynopagebreak}

\begin{absolutelynopagebreak}
\setstretch{.7}
{\PaliGlossA{sādhu, bhikkhave, bhikkhu kālena kālaṃ paṭikūle appaṭikūlasaññī vihareyya.}}\\
\begin{addmargin}[1em]{2em}
\setstretch{.5}
{\PaliGlossB{the unrepulsive in the repulsive,}}\\
\end{addmargin}
\end{absolutelynopagebreak}

\begin{absolutelynopagebreak}
\setstretch{.7}
{\PaliGlossA{sādhu, bhikkhave, bhikkhu kālena kālaṃ appaṭikūle ca paṭikūle ca paṭikūlasaññī vihareyya.}}\\
\begin{addmargin}[1em]{2em}
\setstretch{.5}
{\PaliGlossB{the repulsive in both the unrepulsive and the repulsive, and}}\\
\end{addmargin}
\end{absolutelynopagebreak}

\begin{absolutelynopagebreak}
\setstretch{.7}
{\PaliGlossA{sādhu, bhikkhave, bhikkhu kālena kālaṃ paṭikūle ca appaṭikūle ca appaṭikūlasaññī vihareyya.}}\\
\begin{addmargin}[1em]{2em}
\setstretch{.5}
{\PaliGlossB{the unrepulsive in both the repulsive and the unrepulsive.}}\\
\end{addmargin}
\end{absolutelynopagebreak}

\begin{absolutelynopagebreak}
\setstretch{.7}
{\PaliGlossA{sādhu, bhikkhave, bhikkhu kālena kālaṃ paṭikūlañca appaṭikūlañca tadubhayaṃ abhinivajjetvā upekkhako vihareyya sato sampajāno.}}\\
\begin{addmargin}[1em]{2em}
\setstretch{.5}
{\PaliGlossB{A mendicant would do well to meditate from time to time staying equanimous, mindful and aware, rejecting both the repulsive and the unrepulsive.}}\\
\end{addmargin}
\end{absolutelynopagebreak}

\begin{absolutelynopagebreak}
\setstretch{.7}
{\PaliGlossA{kiñca, bhikkhave, bhikkhu atthavasaṃ paṭicca appaṭikūle paṭikūlasaññī vihareyya?}}\\
\begin{addmargin}[1em]{2em}
\setstretch{.5}
{\PaliGlossB{For what reason should a mendicant meditate perceiving the repulsive in the unrepulsive?}}\\
\end{addmargin}
\end{absolutelynopagebreak}

\begin{absolutelynopagebreak}
\setstretch{.7}
{\PaliGlossA{‘mā me rajanīyesu dhammesu rāgo udapādī’ti—}}\\
\begin{addmargin}[1em]{2em}
\setstretch{.5}
{\PaliGlossB{‘May greed not arise in me for things that arouse greed.’}}\\
\end{addmargin}
\end{absolutelynopagebreak}

\begin{absolutelynopagebreak}
\setstretch{.7}
{\PaliGlossA{idaṃ kho, bhikkhave, bhikkhu atthavasaṃ paṭicca appaṭikūle paṭikūlasaññī vihareyya.}}\\
\begin{addmargin}[1em]{2em}
\setstretch{.5}
{\PaliGlossB{A mendicant should meditate perceiving the repulsive in the unrepulsive for this reason.}}\\
\end{addmargin}
\end{absolutelynopagebreak}

\begin{absolutelynopagebreak}
\setstretch{.7}
{\PaliGlossA{kiñca, bhikkhave, bhikkhu atthavasaṃ paṭicca paṭikūle appaṭikūlasaññī vihareyya?}}\\
\begin{addmargin}[1em]{2em}
\setstretch{.5}
{\PaliGlossB{For what reason should a mendicant meditate perceiving the unrepulsive in the repulsive?}}\\
\end{addmargin}
\end{absolutelynopagebreak}

\begin{absolutelynopagebreak}
\setstretch{.7}
{\PaliGlossA{‘mā me dosanīyesu dhammesu doso udapādī’ti—}}\\
\begin{addmargin}[1em]{2em}
\setstretch{.5}
{\PaliGlossB{‘May hate not arise in me for things that provoke hate.’ …}}\\
\end{addmargin}
\end{absolutelynopagebreak}

\begin{absolutelynopagebreak}
\setstretch{.7}
{\PaliGlossA{idaṃ kho, bhikkhave, bhikkhu atthavasaṃ paṭicca paṭikūle appaṭikūlasaññī vihareyya.}}\\
\begin{addmargin}[1em]{2em}
\setstretch{.5}
{\PaliGlossB{    -}}\\
\end{addmargin}
\end{absolutelynopagebreak}

\begin{absolutelynopagebreak}
\setstretch{.7}
{\PaliGlossA{kiñca, bhikkhave, bhikkhu atthavasaṃ paṭicca appaṭikūle ca paṭikūle ca paṭikūlasaññī vihareyya?}}\\
\begin{addmargin}[1em]{2em}
\setstretch{.5}
{\PaliGlossB{For what reason should a mendicant meditate perceiving the repulsive in both the unrepulsive and the repulsive?}}\\
\end{addmargin}
\end{absolutelynopagebreak}

\begin{absolutelynopagebreak}
\setstretch{.7}
{\PaliGlossA{‘mā me rajanīyesu dhammesu rāgo udapādi, mā me dosanīyesu dhammesu doso udapādī’ti—}}\\
\begin{addmargin}[1em]{2em}
\setstretch{.5}
{\PaliGlossB{‘May greed not arise in me for things that arouse greed. May hate not arise in me for things that provoke hate.’ …}}\\
\end{addmargin}
\end{absolutelynopagebreak}

\begin{absolutelynopagebreak}
\setstretch{.7}
{\PaliGlossA{idaṃ kho, bhikkhave, bhikkhu atthavasaṃ paṭicca appaṭikūle ca paṭikūle ca paṭikūlasaññī vihareyya.}}\\
\begin{addmargin}[1em]{2em}
\setstretch{.5}
{\PaliGlossB{    -}}\\
\end{addmargin}
\end{absolutelynopagebreak}

\begin{absolutelynopagebreak}
\setstretch{.7}
{\PaliGlossA{kiñca, bhikkhave, bhikkhu atthavasaṃ paṭicca paṭikūle ca appaṭikūle ca appaṭikūlasaññī vihareyya?}}\\
\begin{addmargin}[1em]{2em}
\setstretch{.5}
{\PaliGlossB{For what reason should a mendicant meditate perceiving the unrepulsive in both the repulsive and the unrepulsive?}}\\
\end{addmargin}
\end{absolutelynopagebreak}

\begin{absolutelynopagebreak}
\setstretch{.7}
{\PaliGlossA{‘mā me dosanīyesu dhammesu doso udapādi, mā me rajanīyesu dhammesu rāgo udapādī’ti—}}\\
\begin{addmargin}[1em]{2em}
\setstretch{.5}
{\PaliGlossB{‘May hate not arise in me for things that provoke hate. May greed not arise in me for things that arouse greed.’ …}}\\
\end{addmargin}
\end{absolutelynopagebreak}

\begin{absolutelynopagebreak}
\setstretch{.7}
{\PaliGlossA{idaṃ kho, bhikkhave, bhikkhu atthavasaṃ paṭicca paṭikūle ca appaṭikūle ca appaṭikūlasaññī vihareyya.}}\\
\begin{addmargin}[1em]{2em}
\setstretch{.5}
{\PaliGlossB{    -}}\\
\end{addmargin}
\end{absolutelynopagebreak}

\begin{absolutelynopagebreak}
\setstretch{.7}
{\PaliGlossA{kiñca, bhikkhave, bhikkhu atthavasaṃ paṭicca paṭikūlañca appaṭikūlañca tadubhayaṃ abhinivajjetvā upekkhako vihareyya?}}\\
\begin{addmargin}[1em]{2em}
\setstretch{.5}
{\PaliGlossB{For what reason should a mendicant meditate staying equanimous, mindful and aware, rejecting both the repulsive and the unrepulsive?}}\\
\end{addmargin}
\end{absolutelynopagebreak}

\begin{absolutelynopagebreak}
\setstretch{.7}
{\PaliGlossA{‘sato sampajāno mā me kvacani katthaci kiñcanaṃ rajanīyesu dhammesu rāgo udapādi, mā me kvacani katthaci kiñcanaṃ dosanīyesu dhammesu doso udapādi, mā me kvacani katthaci kiñcanaṃ mohanīyesu dhammesu moho udapādī’ti—}}\\
\begin{addmargin}[1em]{2em}
\setstretch{.5}
{\PaliGlossB{‘May no greed for things that arouse greed, hate for things that provoke hate, or delusion for things that promote delusion arise in me in any way at all.’}}\\
\end{addmargin}
\end{absolutelynopagebreak}

\begin{absolutelynopagebreak}
\setstretch{.7}
{\PaliGlossA{idaṃ kho, bhikkhave, bhikkhu atthavasaṃ paṭicca paṭikūlañca appaṭikūlañca tadubhayaṃ abhinivajjetvā upekkhako vihareyya sato sampajāno”ti.}}\\
\begin{addmargin}[1em]{2em}
\setstretch{.5}
{\PaliGlossB{For this reason a mendicant should meditate staying equanimous, mindful and aware, rejecting both the repulsive and the unrepulsive.”}}\\
\end{addmargin}
\end{absolutelynopagebreak}

\begin{absolutelynopagebreak}
\setstretch{.7}
{\PaliGlossA{catutthaṃ.}}\\
\begin{addmargin}[1em]{2em}
\setstretch{.5}
{\PaliGlossB{    -}}\\
\end{addmargin}
\end{absolutelynopagebreak}
