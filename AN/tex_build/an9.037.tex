
\begin{absolutelynopagebreak}
\setstretch{.7}
{\PaliGlossA{aṅguttara nikāya 9}}\\
\begin{addmargin}[1em]{2em}
\setstretch{.5}
{\PaliGlossB{Numbered Discourses 9}}\\
\end{addmargin}
\end{absolutelynopagebreak}

\begin{absolutelynopagebreak}
\setstretch{.7}
{\PaliGlossA{4. mahāvagga}}\\
\begin{addmargin}[1em]{2em}
\setstretch{.5}
{\PaliGlossB{4. The Great Chapter}}\\
\end{addmargin}
\end{absolutelynopagebreak}

\begin{absolutelynopagebreak}
\setstretch{.7}
{\PaliGlossA{37. ānandasutta}}\\
\begin{addmargin}[1em]{2em}
\setstretch{.5}
{\PaliGlossB{37. By Ānanda}}\\
\end{addmargin}
\end{absolutelynopagebreak}

\begin{absolutelynopagebreak}
\setstretch{.7}
{\PaliGlossA{ekaṃ samayaṃ āyasmā ānando kosambiyaṃ viharati ghositārāme.}}\\
\begin{addmargin}[1em]{2em}
\setstretch{.5}
{\PaliGlossB{At one time Venerable Ānanda was staying near Kosambi, in Ghosita’s Monastery.}}\\
\end{addmargin}
\end{absolutelynopagebreak}

\begin{absolutelynopagebreak}
\setstretch{.7}
{\PaliGlossA{tatra kho āyasmā ānando bhikkhū āmantesi:}}\\
\begin{addmargin}[1em]{2em}
\setstretch{.5}
{\PaliGlossB{There Ānanda addressed the mendicants:}}\\
\end{addmargin}
\end{absolutelynopagebreak}

\begin{absolutelynopagebreak}
\setstretch{.7}
{\PaliGlossA{“āvuso bhikkhave”ti.}}\\
\begin{addmargin}[1em]{2em}
\setstretch{.5}
{\PaliGlossB{“Reverends, mendicants!”}}\\
\end{addmargin}
\end{absolutelynopagebreak}

\begin{absolutelynopagebreak}
\setstretch{.7}
{\PaliGlossA{“āvuso”ti kho te bhikkhū āyasmato ānandassa paccassosuṃ.}}\\
\begin{addmargin}[1em]{2em}
\setstretch{.5}
{\PaliGlossB{“Reverend,” they replied.}}\\
\end{addmargin}
\end{absolutelynopagebreak}

\begin{absolutelynopagebreak}
\setstretch{.7}
{\PaliGlossA{āyasmā ānando etadavoca:}}\\
\begin{addmargin}[1em]{2em}
\setstretch{.5}
{\PaliGlossB{Ānanda said this:}}\\
\end{addmargin}
\end{absolutelynopagebreak}

\begin{absolutelynopagebreak}
\setstretch{.7}
{\PaliGlossA{“acchariyaṃ, āvuso, abbhutaṃ, āvuso.}}\\
\begin{addmargin}[1em]{2em}
\setstretch{.5}
{\PaliGlossB{“It’s incredible, reverends, it’s amazing!}}\\
\end{addmargin}
\end{absolutelynopagebreak}

\begin{absolutelynopagebreak}
\setstretch{.7}
{\PaliGlossA{yāvañcidaṃ tena bhagavatā jānatā passatā arahatā sammāsambuddhena sambādhe okāsādhigamo anubuddho sattānaṃ visuddhiyā sokaparidevānaṃ samatikkamāya dukkhadomanassānaṃ atthaṅgamāya ñāyassa adhigamāya nibbānassa sacchikiriyāya.}}\\
\begin{addmargin}[1em]{2em}
\setstretch{.5}
{\PaliGlossB{How this Blessed One who knows and sees, the perfected one, the fully awakened Buddha, has found an opening in a confined space. It’s in order to purify sentient beings, to get past sorrow and crying, to make an end of pain and sadness, to end the cycle of suffering, and to realize extinguishment.}}\\
\end{addmargin}
\end{absolutelynopagebreak}

\begin{absolutelynopagebreak}
\setstretch{.7}
{\PaliGlossA{tadeva nāma cakkhuṃ bhavissati te rūpā tañcāyatanaṃ no paṭisaṃvedissati.}}\\
\begin{addmargin}[1em]{2em}
\setstretch{.5}
{\PaliGlossB{The eye itself is actually present, and so are those sights. Yet one will not experience that sense-field.}}\\
\end{addmargin}
\end{absolutelynopagebreak}

\begin{absolutelynopagebreak}
\setstretch{.7}
{\PaliGlossA{tadeva nāma sotaṃ bhavissati te saddā tañcāyatanaṃ no paṭisaṃvedissati.}}\\
\begin{addmargin}[1em]{2em}
\setstretch{.5}
{\PaliGlossB{The ear itself is actually present, and so are those sounds. Yet one will not experience that sense-field.}}\\
\end{addmargin}
\end{absolutelynopagebreak}

\begin{absolutelynopagebreak}
\setstretch{.7}
{\PaliGlossA{tadeva nāma ghānaṃ bhavissati te gandhā tañcāyatanaṃ no paṭisaṃvedissati.}}\\
\begin{addmargin}[1em]{2em}
\setstretch{.5}
{\PaliGlossB{The nose itself is actually present, and so are those smells. Yet one will not experience that sense-field.}}\\
\end{addmargin}
\end{absolutelynopagebreak}

\begin{absolutelynopagebreak}
\setstretch{.7}
{\PaliGlossA{sāva nāma jivhā bhavissati te rasā tañcāyatanaṃ no paṭisaṃvedissati.}}\\
\begin{addmargin}[1em]{2em}
\setstretch{.5}
{\PaliGlossB{The tongue itself is actually present, and so are those tastes. Yet one will not experience that sense-field.}}\\
\end{addmargin}
\end{absolutelynopagebreak}

\begin{absolutelynopagebreak}
\setstretch{.7}
{\PaliGlossA{sova nāma kāyo bhavissati te phoṭṭhabbā tañcāyatanaṃ no paṭisaṃvedissatī”ti.}}\\
\begin{addmargin}[1em]{2em}
\setstretch{.5}
{\PaliGlossB{The body itself is actually present, and so are those touches. Yet one will not experience that sense-field.”}}\\
\end{addmargin}
\end{absolutelynopagebreak}

\begin{absolutelynopagebreak}
\setstretch{.7}
{\PaliGlossA{evaṃ vutte, āyasmā udāyī āyasmantaṃ ānandaṃ etadavoca:}}\\
\begin{addmargin}[1em]{2em}
\setstretch{.5}
{\PaliGlossB{When he said this, Venerable Udāyī said to Venerable Ānanda:}}\\
\end{addmargin}
\end{absolutelynopagebreak}

\begin{absolutelynopagebreak}
\setstretch{.7}
{\PaliGlossA{“saññīmeva nu kho, āvuso ānanda, tadāyatanaṃ no paṭisaṃvedeti udāhu asaññī”ti?}}\\
\begin{addmargin}[1em]{2em}
\setstretch{.5}
{\PaliGlossB{“Reverend Ānanda, is one who doesn’t experience that sense-field actually percipient or not?”}}\\
\end{addmargin}
\end{absolutelynopagebreak}

\begin{absolutelynopagebreak}
\setstretch{.7}
{\PaliGlossA{“saññīmeva kho, āvuso, tadāyatanaṃ no paṭisaṃvedeti, no asaññī”ti.}}\\
\begin{addmargin}[1em]{2em}
\setstretch{.5}
{\PaliGlossB{“Reverend, one who doesn’t experience that sense-field is actually percipient, not non-percipient.”}}\\
\end{addmargin}
\end{absolutelynopagebreak}

\begin{absolutelynopagebreak}
\setstretch{.7}
{\PaliGlossA{“kiṃsaññī panāvuso, tadāyatanaṃ no paṭisaṃvedetī”ti?}}\\
\begin{addmargin}[1em]{2em}
\setstretch{.5}
{\PaliGlossB{“But what does one who doesn’t experience that sense-field perceive?”}}\\
\end{addmargin}
\end{absolutelynopagebreak}

\begin{absolutelynopagebreak}
\setstretch{.7}
{\PaliGlossA{“idhāvuso, bhikkhu, sabbaso rūpasaññānaṃ samatikkamā paṭighasaññānaṃ atthaṅgamā nānattasaññānaṃ amanasikārā ‘ananto ākāso’ti ākāsānañcāyatanaṃ upasampajja viharati.}}\\
\begin{addmargin}[1em]{2em}
\setstretch{.5}
{\PaliGlossB{“It’s when a mendicant, going totally beyond perceptions of form, with the ending of perceptions of impingement, not focusing on perceptions of diversity, aware that ‘space is infinite’, enters and remains in the dimension of infinite space.}}\\
\end{addmargin}
\end{absolutelynopagebreak}

\begin{absolutelynopagebreak}
\setstretch{.7}
{\PaliGlossA{evaṃsaññīpi kho, āvuso, tadāyatanaṃ no paṭisaṃvedeti.}}\\
\begin{addmargin}[1em]{2em}
\setstretch{.5}
{\PaliGlossB{One who doesn’t experience that sense-field perceives in this way.}}\\
\end{addmargin}
\end{absolutelynopagebreak}

\begin{absolutelynopagebreak}
\setstretch{.7}
{\PaliGlossA{puna caparaṃ, āvuso, bhikkhu sabbaso ākāsānañcāyatanaṃ samatikkamma ‘anantaṃ viññāṇan’ti viññāṇañcāyatanaṃ upasampajja viharati.}}\\
\begin{addmargin}[1em]{2em}
\setstretch{.5}
{\PaliGlossB{Furthermore, a mendicant, going totally beyond the dimension of infinite space, aware that ‘consciousness is infinite’, enters and remains in the dimension of infinite consciousness.}}\\
\end{addmargin}
\end{absolutelynopagebreak}

\begin{absolutelynopagebreak}
\setstretch{.7}
{\PaliGlossA{evaṃsaññīpi kho, āvuso, tadāyatanaṃ no paṭisaṃvedeti.}}\\
\begin{addmargin}[1em]{2em}
\setstretch{.5}
{\PaliGlossB{One who doesn’t experience that sense-field perceives in this way.}}\\
\end{addmargin}
\end{absolutelynopagebreak}

\begin{absolutelynopagebreak}
\setstretch{.7}
{\PaliGlossA{puna caparaṃ, āvuso, bhikkhu sabbaso viññāṇañcāyatanaṃ samatikkamma ‘natthi kiñcī’ti ākiñcaññāyatanaṃ upasampajja viharati.}}\\
\begin{addmargin}[1em]{2em}
\setstretch{.5}
{\PaliGlossB{Furthermore, a mendicant, going totally beyond the dimension of infinite consciousness, aware that ‘there is nothing at all’, enters and remains in the dimension of nothingness.}}\\
\end{addmargin}
\end{absolutelynopagebreak}

\begin{absolutelynopagebreak}
\setstretch{.7}
{\PaliGlossA{evaṃsaññīpi kho, āvuso, tadāyatanaṃ no paṭisaṃvedetīti.}}\\
\begin{addmargin}[1em]{2em}
\setstretch{.5}
{\PaliGlossB{One who doesn’t experience that sense-field perceives in this way.}}\\
\end{addmargin}
\end{absolutelynopagebreak}

\begin{absolutelynopagebreak}
\setstretch{.7}
{\PaliGlossA{ekamidāhaṃ, āvuso, samayaṃ sākete viharāmi añjanavane migadāye.}}\\
\begin{addmargin}[1em]{2em}
\setstretch{.5}
{\PaliGlossB{Reverend, one time I was staying near Sāketa in the deer park in Añjana Wood.}}\\
\end{addmargin}
\end{absolutelynopagebreak}

\begin{absolutelynopagebreak}
\setstretch{.7}
{\PaliGlossA{atha kho, āvuso, jaṭilavāsikā bhikkhunī yenāhaṃ tenupasaṅkami; upasaṅkamitvā maṃ abhivādetvā ekamantaṃ aṭṭhāsi. ekamantaṃ ṭhitā kho, āvuso, jaṭilavāsikā bhikkhunī maṃ etadavoca:}}\\
\begin{addmargin}[1em]{2em}
\setstretch{.5}
{\PaliGlossB{Then the nun Jaṭilagāhikā came up to me, bowed, stood to one side, and said to me:}}\\
\end{addmargin}
\end{absolutelynopagebreak}

\begin{absolutelynopagebreak}
\setstretch{.7}
{\PaliGlossA{‘yāyaṃ, bhante ānanda, samādhi na cābhinato na cāpanato na ca sasaṅkhāraniggayhavāritagato,}}\\
\begin{addmargin}[1em]{2em}
\setstretch{.5}
{\PaliGlossB{‘Sir, Ānanda, regarding the immersion that does not lean forward or pull back, and is not held in place by forceful suppression.}}\\
\end{addmargin}
\end{absolutelynopagebreak}

\begin{absolutelynopagebreak}
\setstretch{.7}
{\PaliGlossA{vimuttattā ṭhito, ṭhitattā santusito, santusitattā no paritassati.}}\\
\begin{addmargin}[1em]{2em}
\setstretch{.5}
{\PaliGlossB{Being free, it’s stable. Being stable, it’s content. Being content, one is not anxious.}}\\
\end{addmargin}
\end{absolutelynopagebreak}

\begin{absolutelynopagebreak}
\setstretch{.7}
{\PaliGlossA{ayaṃ, bhante ānanda, samādhi kiṃphalo vutto bhagavatā’ti?}}\\
\begin{addmargin}[1em]{2em}
\setstretch{.5}
{\PaliGlossB{What did the Buddha say was the fruit of this immersion?’}}\\
\end{addmargin}
\end{absolutelynopagebreak}

\begin{absolutelynopagebreak}
\setstretch{.7}
{\PaliGlossA{evaṃ vutte, sohaṃ, āvuso, jaṭilavāsikaṃ bhikkhuniṃ etadavocaṃ:}}\\
\begin{addmargin}[1em]{2em}
\setstretch{.5}
{\PaliGlossB{When she said this, I said to her:}}\\
\end{addmargin}
\end{absolutelynopagebreak}

\begin{absolutelynopagebreak}
\setstretch{.7}
{\PaliGlossA{‘yāyaṃ, bhagini, samādhi na cābhinato na cāpanato na ca sasaṅkhāraniggayhavāritagato,}}\\
\begin{addmargin}[1em]{2em}
\setstretch{.5}
{\PaliGlossB{‘Sister, regarding the immersion that does not lean forward or pull back, and is not held in place by forceful suppression.}}\\
\end{addmargin}
\end{absolutelynopagebreak}

\begin{absolutelynopagebreak}
\setstretch{.7}
{\PaliGlossA{vimuttattā ṭhito, ṭhitattā santusito, santusitattā no paritassati.}}\\
\begin{addmargin}[1em]{2em}
\setstretch{.5}
{\PaliGlossB{Being free, it’s stable. Being stable, it’s content. Being content, one is not anxious.}}\\
\end{addmargin}
\end{absolutelynopagebreak}

\begin{absolutelynopagebreak}
\setstretch{.7}
{\PaliGlossA{ayaṃ, bhagini, samādhi aññāphalo vutto bhagavatā’ti.}}\\
\begin{addmargin}[1em]{2em}
\setstretch{.5}
{\PaliGlossB{The Buddha said that the fruit of this immersion is enlightenment.’}}\\
\end{addmargin}
\end{absolutelynopagebreak}

\begin{absolutelynopagebreak}
\setstretch{.7}
{\PaliGlossA{evaṃsaññīpi kho, āvuso, tadāyatanaṃ no paṭisaṃvedetī”ti.}}\\
\begin{addmargin}[1em]{2em}
\setstretch{.5}
{\PaliGlossB{One who doesn’t experience that sense-field perceives in this way, too.”}}\\
\end{addmargin}
\end{absolutelynopagebreak}

\begin{absolutelynopagebreak}
\setstretch{.7}
{\PaliGlossA{chaṭṭhaṃ.}}\\
\begin{addmargin}[1em]{2em}
\setstretch{.5}
{\PaliGlossB{    -}}\\
\end{addmargin}
\end{absolutelynopagebreak}
