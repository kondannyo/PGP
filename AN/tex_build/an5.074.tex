
\begin{absolutelynopagebreak}
\setstretch{.7}
{\PaliGlossA{aṅguttara nikāya 5}}\\
\begin{addmargin}[1em]{2em}
\setstretch{.5}
{\PaliGlossB{Numbered Discourses 5}}\\
\end{addmargin}
\end{absolutelynopagebreak}

\begin{absolutelynopagebreak}
\setstretch{.7}
{\PaliGlossA{8. yodhājīvavagga}}\\
\begin{addmargin}[1em]{2em}
\setstretch{.5}
{\PaliGlossB{8. Warriors}}\\
\end{addmargin}
\end{absolutelynopagebreak}

\begin{absolutelynopagebreak}
\setstretch{.7}
{\PaliGlossA{74. dutiyadhammavihārīsutta}}\\
\begin{addmargin}[1em]{2em}
\setstretch{.5}
{\PaliGlossB{74. One Who Lives by the Teaching (2nd)}}\\
\end{addmargin}
\end{absolutelynopagebreak}

\begin{absolutelynopagebreak}
\setstretch{.7}
{\PaliGlossA{atha kho aññataro bhikkhu yena bhagavā tenupasaṅkami; upasaṅkamitvā bhagavantaṃ abhivādetvā ekamantaṃ nisīdi. ekamantaṃ nisinno kho so bhikkhu bhagavantaṃ etadavoca:}}\\
\begin{addmargin}[1em]{2em}
\setstretch{.5}
{\PaliGlossB{Then a mendicant went up to the Buddha, bowed, sat down to one side, and said to him:}}\\
\end{addmargin}
\end{absolutelynopagebreak}

\begin{absolutelynopagebreak}
\setstretch{.7}
{\PaliGlossA{“‘dhammavihārī, dhammavihārī’ti, bhante, vuccati.}}\\
\begin{addmargin}[1em]{2em}
\setstretch{.5}
{\PaliGlossB{“Sir, they speak of ‘one who lives by the teaching’.}}\\
\end{addmargin}
\end{absolutelynopagebreak}

\begin{absolutelynopagebreak}
\setstretch{.7}
{\PaliGlossA{kittāvatā nu kho, bhante, bhikkhu dhammavihārī hotī”ti?}}\\
\begin{addmargin}[1em]{2em}
\setstretch{.5}
{\PaliGlossB{How is one who lives by the teaching defined?”}}\\
\end{addmargin}
\end{absolutelynopagebreak}

\begin{absolutelynopagebreak}
\setstretch{.7}
{\PaliGlossA{“idha, bhikkhu, bhikkhu dhammaṃ pariyāpuṇāti—}}\\
\begin{addmargin}[1em]{2em}
\setstretch{.5}
{\PaliGlossB{“Mendicant, take a mendicant who memorizes the teaching—}}\\
\end{addmargin}
\end{absolutelynopagebreak}

\begin{absolutelynopagebreak}
\setstretch{.7}
{\PaliGlossA{suttaṃ, geyyaṃ, veyyākaraṇaṃ, gāthaṃ, udānaṃ, itivuttakaṃ, jātakaṃ, abbhutadhammaṃ, vedallaṃ;}}\\
\begin{addmargin}[1em]{2em}
\setstretch{.5}
{\PaliGlossB{statements, songs, discussions, verses, inspired exclamations, legends, stories of past lives, amazing stories, and classifications.}}\\
\end{addmargin}
\end{absolutelynopagebreak}

\begin{absolutelynopagebreak}
\setstretch{.7}
{\PaliGlossA{uttari cassa paññāya atthaṃ nappajānāti.}}\\
\begin{addmargin}[1em]{2em}
\setstretch{.5}
{\PaliGlossB{But they don’t understand the higher meaning.}}\\
\end{addmargin}
\end{absolutelynopagebreak}

\begin{absolutelynopagebreak}
\setstretch{.7}
{\PaliGlossA{ayaṃ vuccati, bhikkhu: ‘bhikkhu pariyattibahulo, no dhammavihārī’.}}\\
\begin{addmargin}[1em]{2em}
\setstretch{.5}
{\PaliGlossB{That mendicant is called one who studies a lot, not one who lives by the teaching.}}\\
\end{addmargin}
\end{absolutelynopagebreak}

\begin{absolutelynopagebreak}
\setstretch{.7}
{\PaliGlossA{puna caparaṃ, bhikkhu, bhikkhu yathāsutaṃ yathāpariyattaṃ dhammaṃ vitthārena paresaṃ deseti, uttari cassa paññāya atthaṃ nappajānāti.}}\\
\begin{addmargin}[1em]{2em}
\setstretch{.5}
{\PaliGlossB{Furthermore, a mendicant teaches Dhamma in detail to others as they learned and memorized it. But they don’t understand the higher meaning.}}\\
\end{addmargin}
\end{absolutelynopagebreak}

\begin{absolutelynopagebreak}
\setstretch{.7}
{\PaliGlossA{ayaṃ vuccati, bhikkhu: ‘bhikkhu paññattibahulo, no dhammavihārī’.}}\\
\begin{addmargin}[1em]{2em}
\setstretch{.5}
{\PaliGlossB{That mendicant is called one who advocates a lot, not one who lives by the teaching.}}\\
\end{addmargin}
\end{absolutelynopagebreak}

\begin{absolutelynopagebreak}
\setstretch{.7}
{\PaliGlossA{puna caparaṃ, bhikkhu, bhikkhu yathāsutaṃ yathāpariyattaṃ dhammaṃ vitthārena sajjhāyaṃ karoti, uttari cassa paññāya atthaṃ nappajānāti.}}\\
\begin{addmargin}[1em]{2em}
\setstretch{.5}
{\PaliGlossB{Furthermore, a mendicant recites the teaching in detail as they learned and memorized it. But they don’t understand the higher meaning.}}\\
\end{addmargin}
\end{absolutelynopagebreak}

\begin{absolutelynopagebreak}
\setstretch{.7}
{\PaliGlossA{ayaṃ vuccati, bhikkhu: ‘bhikkhu sajjhāyabahulo, no dhammavihārī’.}}\\
\begin{addmargin}[1em]{2em}
\setstretch{.5}
{\PaliGlossB{That mendicant is called one who recites a lot, not one who lives by the teaching.}}\\
\end{addmargin}
\end{absolutelynopagebreak}

\begin{absolutelynopagebreak}
\setstretch{.7}
{\PaliGlossA{puna caparaṃ, bhikkhu, bhikkhu yathāsutaṃ yathāpariyattaṃ dhammaṃ cetasā anuvitakketi anuvicāreti manasānupekkhati, uttari cassa paññāya atthaṃ nappajānāti.}}\\
\begin{addmargin}[1em]{2em}
\setstretch{.5}
{\PaliGlossB{Furthermore, a mendicant thinks about and considers the teaching in their heart, examining it with the mind as they learned and memorized it. But they don’t understand the higher meaning.}}\\
\end{addmargin}
\end{absolutelynopagebreak}

\begin{absolutelynopagebreak}
\setstretch{.7}
{\PaliGlossA{ayaṃ vuccati, bhikkhu: ‘bhikkhu vitakkabahulo, no dhammavihārī’.}}\\
\begin{addmargin}[1em]{2em}
\setstretch{.5}
{\PaliGlossB{That mendicant is called one who thinks a lot, not one who lives by the teaching.}}\\
\end{addmargin}
\end{absolutelynopagebreak}

\begin{absolutelynopagebreak}
\setstretch{.7}
{\PaliGlossA{idha, bhikkhu, bhikkhu dhammaṃ pariyāpuṇāti—}}\\
\begin{addmargin}[1em]{2em}
\setstretch{.5}
{\PaliGlossB{Take a mendicant who memorizes the teaching—}}\\
\end{addmargin}
\end{absolutelynopagebreak}

\begin{absolutelynopagebreak}
\setstretch{.7}
{\PaliGlossA{suttaṃ, geyyaṃ, veyyākaraṇaṃ, gāthaṃ, udānaṃ, itivuttakaṃ, jātakaṃ, abbhutadhammaṃ, vedallaṃ;}}\\
\begin{addmargin}[1em]{2em}
\setstretch{.5}
{\PaliGlossB{statements, songs, discussions, verses, inspired exclamations, legends, stories of past lives, amazing stories, and classifications.}}\\
\end{addmargin}
\end{absolutelynopagebreak}

\begin{absolutelynopagebreak}
\setstretch{.7}
{\PaliGlossA{uttari cassa paññāya atthaṃ pajānāti.}}\\
\begin{addmargin}[1em]{2em}
\setstretch{.5}
{\PaliGlossB{And they do understand the higher meaning.}}\\
\end{addmargin}
\end{absolutelynopagebreak}

\begin{absolutelynopagebreak}
\setstretch{.7}
{\PaliGlossA{evaṃ kho, bhikkhu, bhikkhu dhammavihārī hoti.}}\\
\begin{addmargin}[1em]{2em}
\setstretch{.5}
{\PaliGlossB{That’s how a mendicant is one who lives by the teaching.}}\\
\end{addmargin}
\end{absolutelynopagebreak}

\begin{absolutelynopagebreak}
\setstretch{.7}
{\PaliGlossA{iti kho, bhikkhu, desito mayā pariyattibahulo, desito paññattibahulo, desito sajjhāyabahulo, desito vitakkabahulo, desito dhammavihārī.}}\\
\begin{addmargin}[1em]{2em}
\setstretch{.5}
{\PaliGlossB{So, mendicant, I’ve taught you the one who studies a lot, the one who advocates a lot, the one who recites a lot, the one who thinks a lot, and the one who lives by the teaching.}}\\
\end{addmargin}
\end{absolutelynopagebreak}

\begin{absolutelynopagebreak}
\setstretch{.7}
{\PaliGlossA{yaṃ kho, bhikkhu, satthārā karaṇīyaṃ sāvakānaṃ hitesinā anukampakena anukampaṃ upādāya, kataṃ vo taṃ mayā.}}\\
\begin{addmargin}[1em]{2em}
\setstretch{.5}
{\PaliGlossB{Out of compassion, I’ve done what a teacher should do who wants what’s best for their disciples.}}\\
\end{addmargin}
\end{absolutelynopagebreak}

\begin{absolutelynopagebreak}
\setstretch{.7}
{\PaliGlossA{etāni, bhikkhu, rukkhamūlāni, etāni suññāgārāni. jhāyatha bhikkhu, mā pamādattha, mā pacchā vippaṭisārino ahuvattha. ayaṃ vo amhākaṃ anusāsanī”ti.}}\\
\begin{addmargin}[1em]{2em}
\setstretch{.5}
{\PaliGlossB{Here are these roots of trees, and here are these empty huts. Practice absorption, mendicant! Don’t be negligent! Don’t regret it later! This is my instruction to you.”}}\\
\end{addmargin}
\end{absolutelynopagebreak}

\begin{absolutelynopagebreak}
\setstretch{.7}
{\PaliGlossA{catutthaṃ.}}\\
\begin{addmargin}[1em]{2em}
\setstretch{.5}
{\PaliGlossB{    -}}\\
\end{addmargin}
\end{absolutelynopagebreak}
