
\begin{absolutelynopagebreak}
\setstretch{.7}
{\PaliGlossA{aṅguttara nikāya 10}}\\
\begin{addmargin}[1em]{2em}
\setstretch{.5}
{\PaliGlossB{Numbered Discourses 10}}\\
\end{addmargin}
\end{absolutelynopagebreak}

\begin{absolutelynopagebreak}
\setstretch{.7}
{\PaliGlossA{5. akkosavagga}}\\
\begin{addmargin}[1em]{2em}
\setstretch{.5}
{\PaliGlossB{5. Abuse}}\\
\end{addmargin}
\end{absolutelynopagebreak}

\begin{absolutelynopagebreak}
\setstretch{.7}
{\PaliGlossA{50. bhaṇḍanasutta}}\\
\begin{addmargin}[1em]{2em}
\setstretch{.5}
{\PaliGlossB{50. Arguments}}\\
\end{addmargin}
\end{absolutelynopagebreak}

\begin{absolutelynopagebreak}
\setstretch{.7}
{\PaliGlossA{ekaṃ samayaṃ bhagavā sāvatthiyaṃ viharati jetavane anāthapiṇḍikassa ārāme.}}\\
\begin{addmargin}[1em]{2em}
\setstretch{.5}
{\PaliGlossB{At one time the Buddha was staying near Sāvatthī in Jeta’s Grove, Anāthapiṇḍika’s monastery.}}\\
\end{addmargin}
\end{absolutelynopagebreak}

\begin{absolutelynopagebreak}
\setstretch{.7}
{\PaliGlossA{tena kho pana samayena sambahulā bhikkhū pacchābhattaṃ piṇḍapātapaṭikkantā upaṭṭhānasālāyaṃ sannisinnā sannipatitā bhaṇḍanajātā kalahajātā vivādāpannā aññamaññaṃ mukhasattīhi vitudantā viharanti.}}\\
\begin{addmargin}[1em]{2em}
\setstretch{.5}
{\PaliGlossB{Now at that time, after the meal, on return from alms-round, several mendicants sat together in the meeting hall. They were arguing, quarreling, and disputing, wounding each other with barbed words.}}\\
\end{addmargin}
\end{absolutelynopagebreak}

\begin{absolutelynopagebreak}
\setstretch{.7}
{\PaliGlossA{atha kho bhagavā sāyanhasamayaṃ paṭisallānā vuṭṭhito yena upaṭṭhānasālā tenupasaṅkami; upasaṅkamitvā paññatte āsane nisīdi.}}\\
\begin{addmargin}[1em]{2em}
\setstretch{.5}
{\PaliGlossB{Then in the late afternoon, the Buddha came out of retreat and went to the assembly hall. He sat down on the seat spread out,}}\\
\end{addmargin}
\end{absolutelynopagebreak}

\begin{absolutelynopagebreak}
\setstretch{.7}
{\PaliGlossA{nisajja kho bhagavā bhikkhū āmantesi:}}\\
\begin{addmargin}[1em]{2em}
\setstretch{.5}
{\PaliGlossB{and addressed the mendicants:}}\\
\end{addmargin}
\end{absolutelynopagebreak}

\begin{absolutelynopagebreak}
\setstretch{.7}
{\PaliGlossA{“kāya nuttha, bhikkhave, etarahi kathāya sannisinnā sannipatitā, kā ca pana vo antarākathā vippakatā”ti?}}\\
\begin{addmargin}[1em]{2em}
\setstretch{.5}
{\PaliGlossB{“Mendicants, what were you sitting talking about just now? What conversation was unfinished?”}}\\
\end{addmargin}
\end{absolutelynopagebreak}

\begin{absolutelynopagebreak}
\setstretch{.7}
{\PaliGlossA{“idha mayaṃ, bhante, pacchābhattaṃ piṇḍapātapaṭikkantā upaṭṭhānasālāyaṃ sannisinnā sannipatitā bhaṇḍanajātā kalahajātā vivādāpannā aññamaññaṃ mukhasattīhi vitudantā viharāmā”ti.}}\\
\begin{addmargin}[1em]{2em}
\setstretch{.5}
{\PaliGlossB{“Sir, after the meal, on return from alms-round, we sat together in the meeting hall, arguing, quarreling, and disputing, wounding each other with barbed words.”}}\\
\end{addmargin}
\end{absolutelynopagebreak}

\begin{absolutelynopagebreak}
\setstretch{.7}
{\PaliGlossA{“na kho panetaṃ, bhikkhave, tumhākaṃ patirūpaṃ kulaputtānaṃ saddhāya agārasmā anagāriyaṃ pabbajitānaṃ, yaṃ tumhe bhaṇḍanajātā kalahajātā vivādāpannā aññamaññaṃ mukhasattīhi vitudantā vihareyyātha.}}\\
\begin{addmargin}[1em]{2em}
\setstretch{.5}
{\PaliGlossB{“Mendicants, this is not appropriate for you gentlemen who have gone forth in faith from the lay life to homelessness.}}\\
\end{addmargin}
\end{absolutelynopagebreak}

\begin{absolutelynopagebreak}
\setstretch{.7}
{\PaliGlossA{dasayime, bhikkhave, dhammā sāraṇīyā piyakaraṇā garukaraṇā saṅgahāya avivādāya sāmaggiyā ekībhāvāya saṃvattanti.}}\\
\begin{addmargin}[1em]{2em}
\setstretch{.5}
{\PaliGlossB{There are ten warm-hearted qualities that make for fondness and respect, conducing to inclusion, harmony, and unity, without quarreling.}}\\
\end{addmargin}
\end{absolutelynopagebreak}

\begin{absolutelynopagebreak}
\setstretch{.7}
{\PaliGlossA{katame dasa?}}\\
\begin{addmargin}[1em]{2em}
\setstretch{.5}
{\PaliGlossB{What ten?}}\\
\end{addmargin}
\end{absolutelynopagebreak}

\begin{absolutelynopagebreak}
\setstretch{.7}
{\PaliGlossA{idha, bhikkhave, bhikkhu sīlavā hoti, pātimokkhasaṃvarasaṃvuto viharati ācāragocarasampanno aṇumattesu vajjesu bhayadassāvī, samādāya sikkhati sikkhāpadesu.}}\\
\begin{addmargin}[1em]{2em}
\setstretch{.5}
{\PaliGlossB{Firstly, a mendicant is ethical, restrained in the monastic code, conducting themselves well and seeking alms in suitable places. Seeing danger in the slightest fault, they keep the rules they’ve undertaken.}}\\
\end{addmargin}
\end{absolutelynopagebreak}

\begin{absolutelynopagebreak}
\setstretch{.7}
{\PaliGlossA{yampi, bhikkhave, bhikkhu sīlavā hoti … pe … samādāya sikkhati sikkhāpadesu,}}\\
\begin{addmargin}[1em]{2em}
\setstretch{.5}
{\PaliGlossB{When a mendicant is ethical,}}\\
\end{addmargin}
\end{absolutelynopagebreak}

\begin{absolutelynopagebreak}
\setstretch{.7}
{\PaliGlossA{ayampi dhammo sāraṇīyo piyakaraṇo garukaraṇo saṅgahāya avivādāya sāmaggiyā ekībhāvāya saṃvattati. (1)}}\\
\begin{addmargin}[1em]{2em}
\setstretch{.5}
{\PaliGlossB{this warm-hearted quality makes for fondness and respect, conducing to inclusion, harmony, and unity, without quarreling.}}\\
\end{addmargin}
\end{absolutelynopagebreak}

\begin{absolutelynopagebreak}
\setstretch{.7}
{\PaliGlossA{puna caparaṃ, bhikkhave, bhikkhu bahussuto hoti sutadharo sutasannicayo, ye te dhammā ādikalyāṇā majjhekalyāṇā pariyosānakalyāṇā sātthaṃ sabyañjanaṃ kevalaparipuṇṇaṃ parisuddhaṃ brahmacariyaṃ abhivadanti, tathārūpāssa dhammā bahussutā honti dhātā vacasā paricitā manasānupekkhitā diṭṭhiyā suppaṭividdhā.}}\\
\begin{addmargin}[1em]{2em}
\setstretch{.5}
{\PaliGlossB{Furthermore, a mendicant is very learned, remembering and keeping what they’ve learned. These teachings are good in the beginning, good in the middle, and good in the end, meaningful and well-phrased, describing a spiritual practice that’s entirely full and pure. They are very learned in such teachings, remembering them, reinforcing them by recitation, mentally scrutinizing them, and comprehending them theoretically. …}}\\
\end{addmargin}
\end{absolutelynopagebreak}

\begin{absolutelynopagebreak}
\setstretch{.7}
{\PaliGlossA{yampi, bhikkhave, bhikkhu bahussuto hoti … pe … diṭṭhiyā suppaṭividdhā,}}\\
\begin{addmargin}[1em]{2em}
\setstretch{.5}
{\PaliGlossB{    -}}\\
\end{addmargin}
\end{absolutelynopagebreak}

\begin{absolutelynopagebreak}
\setstretch{.7}
{\PaliGlossA{ayampi dhammo sāraṇīyo piyakaraṇo garukaraṇo saṅgahāya avivādāya sāmaggiyā ekībhāvāya saṃvattati. (2)}}\\
\begin{addmargin}[1em]{2em}
\setstretch{.5}
{\PaliGlossB{    -}}\\
\end{addmargin}
\end{absolutelynopagebreak}

\begin{absolutelynopagebreak}
\setstretch{.7}
{\PaliGlossA{puna caparaṃ, bhikkhave, bhikkhu kalyāṇamitto hoti kalyāṇasahāyo kalyāṇasampavaṅko.}}\\
\begin{addmargin}[1em]{2em}
\setstretch{.5}
{\PaliGlossB{Furthermore, a mendicant has good friends, companions, and associates. …}}\\
\end{addmargin}
\end{absolutelynopagebreak}

\begin{absolutelynopagebreak}
\setstretch{.7}
{\PaliGlossA{yampi, bhikkhave, bhikkhu kalyāṇamitto hoti kalyāṇasahāyo kalyāṇasampavaṅko,}}\\
\begin{addmargin}[1em]{2em}
\setstretch{.5}
{\PaliGlossB{    -}}\\
\end{addmargin}
\end{absolutelynopagebreak}

\begin{absolutelynopagebreak}
\setstretch{.7}
{\PaliGlossA{ayampi dhammo sāraṇīyo piyakaraṇo garukaraṇo saṅgahāya avivādāya sāmaggiyā ekībhāvāya saṃvattati. (3)}}\\
\begin{addmargin}[1em]{2em}
\setstretch{.5}
{\PaliGlossB{    -}}\\
\end{addmargin}
\end{absolutelynopagebreak}

\begin{absolutelynopagebreak}
\setstretch{.7}
{\PaliGlossA{puna caparaṃ, bhikkhave, bhikkhu suvaco hoti sovacassakaraṇehi dhammehi samannāgato khamo padakkhiṇaggāhī anusāsaniṃ.}}\\
\begin{addmargin}[1em]{2em}
\setstretch{.5}
{\PaliGlossB{Furthermore, a mendicant is easy to admonish, having qualities that make them easy to admonish. They’re patient, and take instruction respectfully. …}}\\
\end{addmargin}
\end{absolutelynopagebreak}

\begin{absolutelynopagebreak}
\setstretch{.7}
{\PaliGlossA{yampi, bhikkhave, bhikkhu suvaco hoti sovacassakaraṇehi dhammehi samannāgato khamo padakkhiṇaggāhī anusāsaniṃ,}}\\
\begin{addmargin}[1em]{2em}
\setstretch{.5}
{\PaliGlossB{    -}}\\
\end{addmargin}
\end{absolutelynopagebreak}

\begin{absolutelynopagebreak}
\setstretch{.7}
{\PaliGlossA{ayampi dhammo sāraṇīyo piyakaraṇo garukaraṇo saṅgahāya avivādāya sāmaggiyā ekībhāvāya saṃvattati. (4)}}\\
\begin{addmargin}[1em]{2em}
\setstretch{.5}
{\PaliGlossB{    -}}\\
\end{addmargin}
\end{absolutelynopagebreak}

\begin{absolutelynopagebreak}
\setstretch{.7}
{\PaliGlossA{puna caparaṃ, bhikkhave, bhikkhu yāni tāni sabrahmacārīnaṃ uccāvacāni kiṅkaraṇīyāni—tattha dakkho hoti analaso, tatrūpāyāya vīmaṃsāya samannāgato alaṃ kātuṃ alaṃ saṃvidhātuṃ.}}\\
\begin{addmargin}[1em]{2em}
\setstretch{.5}
{\PaliGlossB{Furthermore, a mendicant is deft and tireless in a diverse spectrum of duties for their spiritual companions, understanding how to go about things in order to complete and organize the work. …}}\\
\end{addmargin}
\end{absolutelynopagebreak}

\begin{absolutelynopagebreak}
\setstretch{.7}
{\PaliGlossA{yampi, bhikkhave, bhikkhu yāni tāni sabrahmacārīnaṃ uccāvacāni kiṅkaraṇīyāni—tattha dakkho hoti analaso tatrūpāyāya vīmaṃsāya samannāgato alaṃ kātuṃ alaṃ saṃvidhātuṃ,}}\\
\begin{addmargin}[1em]{2em}
\setstretch{.5}
{\PaliGlossB{    -}}\\
\end{addmargin}
\end{absolutelynopagebreak}

\begin{absolutelynopagebreak}
\setstretch{.7}
{\PaliGlossA{ayampi dhammo sāraṇīyo piyakaraṇo garukaraṇo saṅgahāya avivādāya sāmaggiyā ekībhāvāya saṃvattati. (5)}}\\
\begin{addmargin}[1em]{2em}
\setstretch{.5}
{\PaliGlossB{    -}}\\
\end{addmargin}
\end{absolutelynopagebreak}

\begin{absolutelynopagebreak}
\setstretch{.7}
{\PaliGlossA{puna caparaṃ, bhikkhave, bhikkhu dhammakāmo hoti piyasamudāhāro, abhidhamme abhivinaye uḷārapāmojjo.}}\\
\begin{addmargin}[1em]{2em}
\setstretch{.5}
{\PaliGlossB{Furthermore, a mendicant loves the teachings and is a delight to converse with, being full of joy in the teaching and training. …}}\\
\end{addmargin}
\end{absolutelynopagebreak}

\begin{absolutelynopagebreak}
\setstretch{.7}
{\PaliGlossA{yampi, bhikkhave, bhikkhu dhammakāmo hoti piyasamudāhāro, abhidhamme abhivinaye uḷārapāmojjo,}}\\
\begin{addmargin}[1em]{2em}
\setstretch{.5}
{\PaliGlossB{    -}}\\
\end{addmargin}
\end{absolutelynopagebreak}

\begin{absolutelynopagebreak}
\setstretch{.7}
{\PaliGlossA{ayampi dhammo sāraṇīyo piyakaraṇo garukaraṇo saṅgahāya avivādāya sāmaggiyā ekībhāvāya saṃvattati. (6)}}\\
\begin{addmargin}[1em]{2em}
\setstretch{.5}
{\PaliGlossB{    -}}\\
\end{addmargin}
\end{absolutelynopagebreak}

\begin{absolutelynopagebreak}
\setstretch{.7}
{\PaliGlossA{puna caparaṃ, bhikkhave, bhikkhu āraddhavīriyo viharati akusalānaṃ dhammānaṃ pahānāya kusalānaṃ dhammānaṃ upasampadāya, thāmavā daḷhaparakkamo anikkhittadhuro kusalesu dhammesu.}}\\
\begin{addmargin}[1em]{2em}
\setstretch{.5}
{\PaliGlossB{Furthermore, a mendicant lives with energy roused up for giving up unskillful qualities and embracing skillful qualities. They are strong, staunchly vigorous, not slacking off when it comes to developing skillful qualities. …}}\\
\end{addmargin}
\end{absolutelynopagebreak}

\begin{absolutelynopagebreak}
\setstretch{.7}
{\PaliGlossA{yampi, bhikkhave, bhikkhu āraddhavīriyo viharati akusalānaṃ dhammānaṃ pahānāya kusalānaṃ dhammānaṃ upasampadāya, thāmavā daḷhaparakkamo anikkhittadhuro kusalesu dhammesu,}}\\
\begin{addmargin}[1em]{2em}
\setstretch{.5}
{\PaliGlossB{    -}}\\
\end{addmargin}
\end{absolutelynopagebreak}

\begin{absolutelynopagebreak}
\setstretch{.7}
{\PaliGlossA{ayampi dhammo sāraṇīyo piyakaraṇo garukaraṇo saṅgahāya avivādāya sāmaggiyā ekībhāvāya saṃvattati. (7)}}\\
\begin{addmargin}[1em]{2em}
\setstretch{.5}
{\PaliGlossB{    -}}\\
\end{addmargin}
\end{absolutelynopagebreak}

\begin{absolutelynopagebreak}
\setstretch{.7}
{\PaliGlossA{puna caparaṃ, bhikkhave, bhikkhu santuṭṭho hoti itarītaracīvarapiṇḍapātasenāsanagilānapaccayabhesajjaparikkhārena.}}\\
\begin{addmargin}[1em]{2em}
\setstretch{.5}
{\PaliGlossB{Furthermore, a mendicant is content with any kind of robes, alms-food, lodgings, and medicines and supplies for the sick. …}}\\
\end{addmargin}
\end{absolutelynopagebreak}

\begin{absolutelynopagebreak}
\setstretch{.7}
{\PaliGlossA{yampi, bhikkhave, bhikkhu santuṭṭho hoti itarītaracīvarapiṇḍapātasenāsanagilānapaccayabhesajjaparikkhārena,}}\\
\begin{addmargin}[1em]{2em}
\setstretch{.5}
{\PaliGlossB{    -}}\\
\end{addmargin}
\end{absolutelynopagebreak}

\begin{absolutelynopagebreak}
\setstretch{.7}
{\PaliGlossA{ayampi dhammo sāraṇīyo … pe … saṃvattati. (8)}}\\
\begin{addmargin}[1em]{2em}
\setstretch{.5}
{\PaliGlossB{    -}}\\
\end{addmargin}
\end{absolutelynopagebreak}

\begin{absolutelynopagebreak}
\setstretch{.7}
{\PaliGlossA{puna caparaṃ, bhikkhave, bhikkhu satimā hoti, paramena satinepakkena samannāgato, cirakatampi cirabhāsitampi saritā anussaritā.}}\\
\begin{addmargin}[1em]{2em}
\setstretch{.5}
{\PaliGlossB{Furthermore, a mendicant is mindful. They have utmost mindfulness and alertness, and can remember and recall what was said and done long ago. …}}\\
\end{addmargin}
\end{absolutelynopagebreak}

\begin{absolutelynopagebreak}
\setstretch{.7}
{\PaliGlossA{yampi, bhikkhave, bhikkhu satimā hoti, paramena satinepakkena samannāgato, cirakatampi cirabhāsitampi saritā anussaritā,}}\\
\begin{addmargin}[1em]{2em}
\setstretch{.5}
{\PaliGlossB{    -}}\\
\end{addmargin}
\end{absolutelynopagebreak}

\begin{absolutelynopagebreak}
\setstretch{.7}
{\PaliGlossA{ayampi dhammo sāraṇīyo … pe … saṃvattati. (9)}}\\
\begin{addmargin}[1em]{2em}
\setstretch{.5}
{\PaliGlossB{    -}}\\
\end{addmargin}
\end{absolutelynopagebreak}

\begin{absolutelynopagebreak}
\setstretch{.7}
{\PaliGlossA{puna caparaṃ, bhikkhave, bhikkhu paññavā hoti, udayatthagāminiyā paññāya samannāgato ariyāya nibbedhikāya sammā dukkhakkhayagāminiyā.}}\\
\begin{addmargin}[1em]{2em}
\setstretch{.5}
{\PaliGlossB{Furthermore, a mendicant is wise. They have the wisdom of arising and passing away which is noble, penetrative, and leads to the complete ending of suffering.}}\\
\end{addmargin}
\end{absolutelynopagebreak}

\begin{absolutelynopagebreak}
\setstretch{.7}
{\PaliGlossA{yampi, bhikkhave, bhikkhu paññavā hoti, udayatthagāminiyā paññāya samannāgato ariyāya nibbedhikāya sammā dukkhakkhayagāminiyā,}}\\
\begin{addmargin}[1em]{2em}
\setstretch{.5}
{\PaliGlossB{When a mendicant is wise,}}\\
\end{addmargin}
\end{absolutelynopagebreak}

\begin{absolutelynopagebreak}
\setstretch{.7}
{\PaliGlossA{ayampi dhammo sāraṇīyo … pe … saṃvattati. (10)}}\\
\begin{addmargin}[1em]{2em}
\setstretch{.5}
{\PaliGlossB{this warm-hearted quality makes for fondness and respect, helping the Saṅgha to live in harmony and unity, without quarreling.}}\\
\end{addmargin}
\end{absolutelynopagebreak}

\begin{absolutelynopagebreak}
\setstretch{.7}
{\PaliGlossA{ime kho, bhikkhave, dasa dhammā sāraṇīyā piyakaraṇā garukaraṇā saṅgahāya avivādāya sāmaggiyā ekībhāvāya saṃvattantī”ti.}}\\
\begin{addmargin}[1em]{2em}
\setstretch{.5}
{\PaliGlossB{These ten warm-hearted qualities make for fondness and respect, conducing to inclusion, harmony, and unity, without quarreling.”}}\\
\end{addmargin}
\end{absolutelynopagebreak}

\begin{absolutelynopagebreak}
\setstretch{.7}
{\PaliGlossA{dasamaṃ.}}\\
\begin{addmargin}[1em]{2em}
\setstretch{.5}
{\PaliGlossB{    -}}\\
\end{addmargin}
\end{absolutelynopagebreak}

\begin{absolutelynopagebreak}
\setstretch{.7}
{\PaliGlossA{akkosavaggo pañcamo.}}\\
\begin{addmargin}[1em]{2em}
\setstretch{.5}
{\PaliGlossB{    -}}\\
\end{addmargin}
\end{absolutelynopagebreak}

\begin{absolutelynopagebreak}
\setstretch{.7}
{\PaliGlossA{vivādā dve ca mūlāni,}}\\
\begin{addmargin}[1em]{2em}
\setstretch{.5}
{\PaliGlossB{    -}}\\
\end{addmargin}
\end{absolutelynopagebreak}

\begin{absolutelynopagebreak}
\setstretch{.7}
{\PaliGlossA{kusinārapavesane;}}\\
\begin{addmargin}[1em]{2em}
\setstretch{.5}
{\PaliGlossB{    -}}\\
\end{addmargin}
\end{absolutelynopagebreak}

\begin{absolutelynopagebreak}
\setstretch{.7}
{\PaliGlossA{sakko mahāli abhiṇhaṃ,}}\\
\begin{addmargin}[1em]{2em}
\setstretch{.5}
{\PaliGlossB{    -}}\\
\end{addmargin}
\end{absolutelynopagebreak}

\begin{absolutelynopagebreak}
\setstretch{.7}
{\PaliGlossA{sarīraṭṭhā ca bhaṇḍanāti.}}\\
\begin{addmargin}[1em]{2em}
\setstretch{.5}
{\PaliGlossB{    -}}\\
\end{addmargin}
\end{absolutelynopagebreak}

\begin{absolutelynopagebreak}
\setstretch{.7}
{\PaliGlossA{paṭhamo paṇṇāsako samatto.}}\\
\begin{addmargin}[1em]{2em}
\setstretch{.5}
{\PaliGlossB{    -}}\\
\end{addmargin}
\end{absolutelynopagebreak}
