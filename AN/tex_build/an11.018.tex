
\begin{absolutelynopagebreak}
\setstretch{.7}
{\PaliGlossA{aṅguttara nikāya 11}}\\
\begin{addmargin}[1em]{2em}
\setstretch{.5}
{\PaliGlossB{Numbered Discourses 11}}\\
\end{addmargin}
\end{absolutelynopagebreak}

\begin{absolutelynopagebreak}
\setstretch{.7}
{\PaliGlossA{2. anussativagga}}\\
\begin{addmargin}[1em]{2em}
\setstretch{.5}
{\PaliGlossB{2. Recollection}}\\
\end{addmargin}
\end{absolutelynopagebreak}

\begin{absolutelynopagebreak}
\setstretch{.7}
{\PaliGlossA{18. paṭhamasamādhisutta}}\\
\begin{addmargin}[1em]{2em}
\setstretch{.5}
{\PaliGlossB{18. Immersion (1st)}}\\
\end{addmargin}
\end{absolutelynopagebreak}

\begin{absolutelynopagebreak}
\setstretch{.7}
{\PaliGlossA{atha kho sambahulā bhikkhū yena bhagavā tenupasaṅkamiṃsu; upasaṅkamitvā bhagavantaṃ abhivādetvā ekamantaṃ nisīdiṃsu. ekamantaṃ nisinnā kho te bhikkhū bhagavantaṃ etadavocuṃ:}}\\
\begin{addmargin}[1em]{2em}
\setstretch{.5}
{\PaliGlossB{And then several mendicants went up to the Buddha, bowed, sat down to one side, and said to him:}}\\
\end{addmargin}
\end{absolutelynopagebreak}

\begin{absolutelynopagebreak}
\setstretch{.7}
{\PaliGlossA{“siyā nu kho, bhante, bhikkhuno tathārūpo samādhipaṭilābho yathā neva pathaviyaṃ pathavisaññī assa, na āpasmiṃ āposaññī assa, na tejasmiṃ tejosaññī assa, na vāyasmiṃ vāyosaññī assa, na ākāsānañcāyatane ākāsānañcāyatanasaññī assa, na viññāṇañcāyatane viññāṇañcāyatanasaññī assa, na ākiñcaññāyatane ākiñcaññāyatanasaññī assa, na nevasaññānāsaññāyatane nevasaññānāsaññāyatanasaññī assa, na idhaloke idhalokasaññī assa, na paraloke paralokasaññī assa, yampidaṃ diṭṭhaṃ sutaṃ mutaṃ viññātaṃ pattaṃ pariyesitaṃ anuvicaritaṃ manasā tatrāpi na saññī assa; saññī ca pana assā”ti?}}\\
\begin{addmargin}[1em]{2em}
\setstretch{.5}
{\PaliGlossB{“Could it be, sir, that a mendicant might gain a state of immersion like this? They wouldn’t perceive earth in earth, water in water, fire in fire, or air in air. And they wouldn’t perceive the dimension of infinite space in the dimension of infinite space, the dimension of infinite consciousness in the dimension of infinite consciousness, the dimension of nothingness in the dimension of nothingness, or the dimension of neither perception nor non-perception in the dimension of neither perception nor non-perception. They wouldn’t perceive this world in this world, or the other world in the other world. And they wouldn’t perceive what is seen, heard, thought, known, attained, sought, or explored by the mind. And yet they would still perceive.”}}\\
\end{addmargin}
\end{absolutelynopagebreak}

\begin{absolutelynopagebreak}
\setstretch{.7}
{\PaliGlossA{“siyā, bhikkhave, bhikkhuno tathārūpo samādhipaṭilābho yathā neva pathaviyaṃ pathavisaññī assa … pe … yampidaṃ diṭṭhaṃ sutaṃ mutaṃ viññātaṃ pattaṃ pariyesitaṃ anuvicaritaṃ manasā tatrāpi na saññī assa; saññī ca pana assā”ti.}}\\
\begin{addmargin}[1em]{2em}
\setstretch{.5}
{\PaliGlossB{“It could be, mendicants.”}}\\
\end{addmargin}
\end{absolutelynopagebreak}

\begin{absolutelynopagebreak}
\setstretch{.7}
{\PaliGlossA{“yathā kathaṃ pana, bhante, siyā bhikkhuno tathārūpo samādhipaṭilābho yathā neva pathaviyaṃ pathavisaññī assa … pe … yampidaṃ diṭṭhaṃ sutaṃ mutaṃ viññātaṃ pattaṃ pariyesitaṃ anuvicaritaṃ manasā tatrāpi na saññī assa; saññī ca pana assā”ti?}}\\
\begin{addmargin}[1em]{2em}
\setstretch{.5}
{\PaliGlossB{“But how could this be?”}}\\
\end{addmargin}
\end{absolutelynopagebreak}

\begin{absolutelynopagebreak}
\setstretch{.7}
{\PaliGlossA{“idha, bhikkhave, bhikkhu evaṃsaññī hoti:}}\\
\begin{addmargin}[1em]{2em}
\setstretch{.5}
{\PaliGlossB{“It’s when a mendicant perceives:}}\\
\end{addmargin}
\end{absolutelynopagebreak}

\begin{absolutelynopagebreak}
\setstretch{.7}
{\PaliGlossA{‘etaṃ santaṃ etaṃ paṇītaṃ, yadidaṃ sabbasaṅkhārasamatho sabbūpadhipaṭinissaggo taṇhākkhayo virāgo nirodho nibbānan’ti.}}\\
\begin{addmargin}[1em]{2em}
\setstretch{.5}
{\PaliGlossB{‘This is peaceful; this is sublime—that is, the stilling of all activities, the letting go of all attachments, the ending of craving, fading away, cessation, extinguishment.’}}\\
\end{addmargin}
\end{absolutelynopagebreak}

\begin{absolutelynopagebreak}
\setstretch{.7}
{\PaliGlossA{evaṃ kho, bhikkhave, siyā bhikkhuno tathārūpo samādhipaṭilābho yathā neva pathaviyaṃ pathavisaññī assa, na āpasmiṃ āposaññī assa, na tejasmiṃ tejosaññī assa, na vāyasmiṃ vāyosaññī assa, na ākāsānañcāyatane ākāsānañcāyatanasaññī assa, na viññāṇañcāyatane viññāṇañcāyatanasaññī assa, na ākiñcaññāyatane ākiñcaññāyatanasaññī assa, na nevasaññānāsaññāyatane nevasaññānāsaññāyatanasaññī assa, na idhaloke idhalokasaññī assa, na paraloke paralokasaññī assa, yampidaṃ diṭṭhaṃ sutaṃ mutaṃ viññātaṃ pattaṃ pariyesitaṃ anuvicaritaṃ manasā tatrāpi na saññī assa; saññī ca pana assā”ti.}}\\
\begin{addmargin}[1em]{2em}
\setstretch{.5}
{\PaliGlossB{That’s how a mendicant might gain a state of immersion like this. They wouldn’t perceive earth in earth, water in water, fire in fire, or air in air. And they wouldn’t perceive the dimension of infinite space in the dimension of infinite space, the dimension of infinite consciousness in the dimension of infinite consciousness, the dimension of nothingness in the dimension of nothingness, or the dimension of neither perception nor non-perception in the dimension of neither perception nor non-perception. They wouldn’t perceive this world in this world, or the other world in the other world. And they wouldn’t perceive what is seen, heard, thought, known, attained, sought, or explored by the mind. And yet they would still perceive.”}}\\
\end{addmargin}
\end{absolutelynopagebreak}

\begin{absolutelynopagebreak}
\setstretch{.7}
{\PaliGlossA{aṭṭhamaṃ.}}\\
\begin{addmargin}[1em]{2em}
\setstretch{.5}
{\PaliGlossB{    -}}\\
\end{addmargin}
\end{absolutelynopagebreak}
