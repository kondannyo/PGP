
\begin{absolutelynopagebreak}
\setstretch{.7}
{\PaliGlossA{aṅguttara nikāya 10}}\\
\begin{addmargin}[1em]{2em}
\setstretch{.5}
{\PaliGlossB{Numbered Discourses 10}}\\
\end{addmargin}
\end{absolutelynopagebreak}

\begin{absolutelynopagebreak}
\setstretch{.7}
{\PaliGlossA{2. nāthavagga}}\\
\begin{addmargin}[1em]{2em}
\setstretch{.5}
{\PaliGlossB{2. A Protector}}\\
\end{addmargin}
\end{absolutelynopagebreak}

\begin{absolutelynopagebreak}
\setstretch{.7}
{\PaliGlossA{18. dutiyanāthasutta}}\\
\begin{addmargin}[1em]{2em}
\setstretch{.5}
{\PaliGlossB{18. A Protector (2nd)}}\\
\end{addmargin}
\end{absolutelynopagebreak}

\begin{absolutelynopagebreak}
\setstretch{.7}
{\PaliGlossA{evaṃ me sutaṃ—}}\\
\begin{addmargin}[1em]{2em}
\setstretch{.5}
{\PaliGlossB{So I have heard.}}\\
\end{addmargin}
\end{absolutelynopagebreak}

\begin{absolutelynopagebreak}
\setstretch{.7}
{\PaliGlossA{ekaṃ samayaṃ bhagavā sāvatthiyaṃ viharati jetavane anāthapiṇḍikassa ārāme.}}\\
\begin{addmargin}[1em]{2em}
\setstretch{.5}
{\PaliGlossB{At one time the Buddha was staying near Sāvatthī in Jeta’s Grove, Anāthapiṇḍika’s monastery.}}\\
\end{addmargin}
\end{absolutelynopagebreak}

\begin{absolutelynopagebreak}
\setstretch{.7}
{\PaliGlossA{tatra kho bhagavā bhikkhū āmantesi:}}\\
\begin{addmargin}[1em]{2em}
\setstretch{.5}
{\PaliGlossB{There the Buddha addressed the mendicants,}}\\
\end{addmargin}
\end{absolutelynopagebreak}

\begin{absolutelynopagebreak}
\setstretch{.7}
{\PaliGlossA{“bhikkhavo”ti.}}\\
\begin{addmargin}[1em]{2em}
\setstretch{.5}
{\PaliGlossB{“Mendicants!”}}\\
\end{addmargin}
\end{absolutelynopagebreak}

\begin{absolutelynopagebreak}
\setstretch{.7}
{\PaliGlossA{“bhadante”ti te bhikkhū bhagavato paccassosuṃ.}}\\
\begin{addmargin}[1em]{2em}
\setstretch{.5}
{\PaliGlossB{“Venerable sir,” they replied.}}\\
\end{addmargin}
\end{absolutelynopagebreak}

\begin{absolutelynopagebreak}
\setstretch{.7}
{\PaliGlossA{bhagavā etadavoca:}}\\
\begin{addmargin}[1em]{2em}
\setstretch{.5}
{\PaliGlossB{The Buddha said this:}}\\
\end{addmargin}
\end{absolutelynopagebreak}

\begin{absolutelynopagebreak}
\setstretch{.7}
{\PaliGlossA{“sanāthā, bhikkhave, viharatha, mā anāthā.}}\\
\begin{addmargin}[1em]{2em}
\setstretch{.5}
{\PaliGlossB{“Mendicants, you should live with a protector, not without one.}}\\
\end{addmargin}
\end{absolutelynopagebreak}

\begin{absolutelynopagebreak}
\setstretch{.7}
{\PaliGlossA{dukkhaṃ, bhikkhave, anātho viharati.}}\\
\begin{addmargin}[1em]{2em}
\setstretch{.5}
{\PaliGlossB{Living without a protector is suffering.}}\\
\end{addmargin}
\end{absolutelynopagebreak}

\begin{absolutelynopagebreak}
\setstretch{.7}
{\PaliGlossA{dasayime, bhikkhave, nāthakaraṇā dhammā.}}\\
\begin{addmargin}[1em]{2em}
\setstretch{.5}
{\PaliGlossB{There are ten qualities that serve as protector.}}\\
\end{addmargin}
\end{absolutelynopagebreak}

\begin{absolutelynopagebreak}
\setstretch{.7}
{\PaliGlossA{katame dasa?}}\\
\begin{addmargin}[1em]{2em}
\setstretch{.5}
{\PaliGlossB{What ten?}}\\
\end{addmargin}
\end{absolutelynopagebreak}

\begin{absolutelynopagebreak}
\setstretch{.7}
{\PaliGlossA{idha, bhikkhave, bhikkhu sīlavā hoti … pe … samādāya sikkhati sikkhāpadesu.}}\\
\begin{addmargin}[1em]{2em}
\setstretch{.5}
{\PaliGlossB{Firstly, a mendicant is ethical, restrained in the code of conduct, with good behavior and supporters. Seeing danger in the slightest fault, they keep the rules they’ve undertaken.}}\\
\end{addmargin}
\end{absolutelynopagebreak}

\begin{absolutelynopagebreak}
\setstretch{.7}
{\PaliGlossA{‘sīlavā vatāyaṃ bhikkhu pātimokkhasaṃvarasaṃvuto viharati ācāragocarasampanno aṇumattesu vajjesu bhayadassāvī, samādāya sikkhati sikkhāpadesū’ti therāpi naṃ bhikkhū vattabbaṃ anusāsitabbaṃ maññanti, majjhimāpi bhikkhū … navāpi bhikkhū vattabbaṃ anusāsitabbaṃ maññanti.}}\\
\begin{addmargin}[1em]{2em}
\setstretch{.5}
{\PaliGlossB{Knowing this, the mendicants—whether senior, middle, or junior—think that mendicant is worth advising and instructing.}}\\
\end{addmargin}
\end{absolutelynopagebreak}

\begin{absolutelynopagebreak}
\setstretch{.7}
{\PaliGlossA{tassa therānukampitassa majjhimānukampitassa navānukampitassa vuddhiyeva pāṭikaṅkhā kusalesu dhammesu, no parihāni.}}\\
\begin{addmargin}[1em]{2em}
\setstretch{.5}
{\PaliGlossB{Being treated with such kindness by the senior, middle, and junior mendicants, that mendicant can expect only growth, not decline.}}\\
\end{addmargin}
\end{absolutelynopagebreak}

\begin{absolutelynopagebreak}
\setstretch{.7}
{\PaliGlossA{ayampi dhammo nāthakaraṇo. (1)}}\\
\begin{addmargin}[1em]{2em}
\setstretch{.5}
{\PaliGlossB{This is a quality that serves as protector.}}\\
\end{addmargin}
\end{absolutelynopagebreak}

\begin{absolutelynopagebreak}
\setstretch{.7}
{\PaliGlossA{puna caparaṃ, bhikkhave, bhikkhu bahussuto hoti … pe … diṭṭhiyā suppaṭividdhā.}}\\
\begin{addmargin}[1em]{2em}
\setstretch{.5}
{\PaliGlossB{Furthermore, a mendicant is very learned, remembering and keeping what they’ve learned. These teachings are good in the beginning, good in the middle, and good in the end, meaningful and well-phrased, describing a spiritual practice that’s entirely full and pure. They are very learned in such teachings, remembering them, reinforcing them by recitation, mentally scrutinizing them, and comprehending them theoretically.}}\\
\end{addmargin}
\end{absolutelynopagebreak}

\begin{absolutelynopagebreak}
\setstretch{.7}
{\PaliGlossA{‘bahussuto vatāyaṃ bhikkhu sutadharo sutasannicayo, ye te dhammā ādikalyāṇā majjhekalyāṇā pariyosānakalyāṇā sātthaṃ sabyañjanaṃ kevalaparipuṇṇaṃ parisuddhaṃ brahmacariyaṃ abhivadanti, tathārūpāssa dhammā bahussutā honti dhātā vacasā paricitā manasānupekkhitā diṭṭhiyā suppaṭividdhā’ti therāpi naṃ bhikkhū vattabbaṃ anusāsitabbaṃ maññanti, majjhimāpi bhikkhū … navāpi bhikkhū vattabbaṃ anusāsitabbaṃ maññanti.}}\\
\begin{addmargin}[1em]{2em}
\setstretch{.5}
{\PaliGlossB{Knowing this, the mendicants—whether senior, middle, or junior—think that mendicant is worth advising and instructing.}}\\
\end{addmargin}
\end{absolutelynopagebreak}

\begin{absolutelynopagebreak}
\setstretch{.7}
{\PaliGlossA{tassa therānukampitassa majjhimānukampitassa navānukampitassa vuddhiyeva pāṭikaṅkhā kusalesu dhammesu, no parihāni.}}\\
\begin{addmargin}[1em]{2em}
\setstretch{.5}
{\PaliGlossB{Being treated with such kindness by the senior, middle, and junior mendicants, that mendicant can expect only growth, not decline.}}\\
\end{addmargin}
\end{absolutelynopagebreak}

\begin{absolutelynopagebreak}
\setstretch{.7}
{\PaliGlossA{ayampi dhammo nāthakaraṇo. (2)}}\\
\begin{addmargin}[1em]{2em}
\setstretch{.5}
{\PaliGlossB{This too is a quality that serves as protector.}}\\
\end{addmargin}
\end{absolutelynopagebreak}

\begin{absolutelynopagebreak}
\setstretch{.7}
{\PaliGlossA{puna caparaṃ, bhikkhave, bhikkhu kalyāṇamitto hoti kalyāṇasahāyo kalyāṇasampavaṅko.}}\\
\begin{addmargin}[1em]{2em}
\setstretch{.5}
{\PaliGlossB{Furthermore, a mendicant has good friends, companions, and associates.}}\\
\end{addmargin}
\end{absolutelynopagebreak}

\begin{absolutelynopagebreak}
\setstretch{.7}
{\PaliGlossA{‘kalyāṇamitto vatāyaṃ bhikkhu kalyāṇasahāyo kalyāṇasampavaṅko’ti therāpi naṃ bhikkhū vattabbaṃ anusāsitabbaṃ maññanti, majjhimāpi bhikkhū … navāpi bhikkhū vattabbaṃ anusāsitabbaṃ maññanti.}}\\
\begin{addmargin}[1em]{2em}
\setstretch{.5}
{\PaliGlossB{Knowing this, the mendicants—whether senior, middle, or junior—think that mendicant is worth advising and instructing.}}\\
\end{addmargin}
\end{absolutelynopagebreak}

\begin{absolutelynopagebreak}
\setstretch{.7}
{\PaliGlossA{tassa therānukampitassa majjhimānukampitassa navānukampitassa vuddhiyeva pāṭikaṅkhā kusalesu dhammesu, no parihāni.}}\\
\begin{addmargin}[1em]{2em}
\setstretch{.5}
{\PaliGlossB{Being treated with such kindness by the senior, middle, and junior mendicants, that mendicant can expect only growth, not decline.}}\\
\end{addmargin}
\end{absolutelynopagebreak}

\begin{absolutelynopagebreak}
\setstretch{.7}
{\PaliGlossA{ayampi dhammo nāthakaraṇo. (3)}}\\
\begin{addmargin}[1em]{2em}
\setstretch{.5}
{\PaliGlossB{This too is a quality that serves as protector.}}\\
\end{addmargin}
\end{absolutelynopagebreak}

\begin{absolutelynopagebreak}
\setstretch{.7}
{\PaliGlossA{puna caparaṃ, bhikkhave, bhikkhu suvaco hoti sovacassakaraṇehi dhammehi samannāgato, khamo padakkhiṇaggāhī anusāsaniṃ.}}\\
\begin{addmargin}[1em]{2em}
\setstretch{.5}
{\PaliGlossB{Furthermore, a mendicant is easy to admonish, having qualities that make them easy to admonish. They’re patient, and take instruction respectfully.}}\\
\end{addmargin}
\end{absolutelynopagebreak}

\begin{absolutelynopagebreak}
\setstretch{.7}
{\PaliGlossA{‘suvaco vatāyaṃ bhikkhu sovacassakaraṇehi dhammehi samannāgato, khamo padakkhiṇaggāhī anusāsanin’ti therāpi naṃ bhikkhū vattabbaṃ anusāsitabbaṃ maññanti, majjhimāpi bhikkhū … navāpi bhikkhū vattabbaṃ anusāsitabbaṃ maññanti.}}\\
\begin{addmargin}[1em]{2em}
\setstretch{.5}
{\PaliGlossB{Knowing this, the mendicants—whether senior, middle, or junior—think that mendicant is worth advising and instructing.}}\\
\end{addmargin}
\end{absolutelynopagebreak}

\begin{absolutelynopagebreak}
\setstretch{.7}
{\PaliGlossA{tassa therānukampitassa majjhimānukampitassa navānukampitassa vuddhiyeva pāṭikaṅkhā kusalesu dhammesu, no parihāni.}}\\
\begin{addmargin}[1em]{2em}
\setstretch{.5}
{\PaliGlossB{Being treated with such kindness by the senior, middle, and junior mendicants, that mendicant can expect only growth, not decline.}}\\
\end{addmargin}
\end{absolutelynopagebreak}

\begin{absolutelynopagebreak}
\setstretch{.7}
{\PaliGlossA{ayampi dhammo nāthakaraṇo. (4)}}\\
\begin{addmargin}[1em]{2em}
\setstretch{.5}
{\PaliGlossB{This too is a quality that serves as protector.}}\\
\end{addmargin}
\end{absolutelynopagebreak}

\begin{absolutelynopagebreak}
\setstretch{.7}
{\PaliGlossA{puna caparaṃ, bhikkhave, bhikkhu yāni tāni sabrahmacārīnaṃ uccāvacāni kiṅkaraṇīyāni, tattha dakkho hoti analaso, tatrūpāyāya vīmaṃsāya samannāgato, alaṃ kātuṃ alaṃ saṃvidhātuṃ.}}\\
\begin{addmargin}[1em]{2em}
\setstretch{.5}
{\PaliGlossB{Furthermore, a mendicant is deft and tireless in a diverse spectrum of duties for their spiritual companions, understanding how to go about things in order to complete and organize the work.}}\\
\end{addmargin}
\end{absolutelynopagebreak}

\begin{absolutelynopagebreak}
\setstretch{.7}
{\PaliGlossA{‘yāni tāni sabrahmacārīnaṃ uccāvacāni kiṅkaraṇīyāni, tattha dakkho vatāyaṃ bhikkhu analaso, tatrūpāyāya vīmaṃsāya samannāgato, alaṃ kātuṃ alaṃ saṃvidhātun’ti therāpi naṃ bhikkhū vattabbaṃ anusāsitabbaṃ maññanti, majjhimāpi bhikkhū … navāpi bhikkhū vattabbaṃ anusāsitabbaṃ maññanti.}}\\
\begin{addmargin}[1em]{2em}
\setstretch{.5}
{\PaliGlossB{Knowing this, the mendicants—whether senior, middle, or junior—think that mendicant is worth advising and instructing.}}\\
\end{addmargin}
\end{absolutelynopagebreak}

\begin{absolutelynopagebreak}
\setstretch{.7}
{\PaliGlossA{tassa therānukampitassa majjhimānukampitassa navānukampitassa vuddhiyeva pāṭikaṅkhā kusalesu dhammesu, no parihāni.}}\\
\begin{addmargin}[1em]{2em}
\setstretch{.5}
{\PaliGlossB{Being treated with such kindness by the senior, middle, and junior mendicants, that mendicant can expect only growth, not decline.}}\\
\end{addmargin}
\end{absolutelynopagebreak}

\begin{absolutelynopagebreak}
\setstretch{.7}
{\PaliGlossA{ayampi dhammo nāthakaraṇo. (5)}}\\
\begin{addmargin}[1em]{2em}
\setstretch{.5}
{\PaliGlossB{This too is a quality that serves as protector.}}\\
\end{addmargin}
\end{absolutelynopagebreak}

\begin{absolutelynopagebreak}
\setstretch{.7}
{\PaliGlossA{puna caparaṃ, bhikkhave, bhikkhu dhammakāmo hoti piyasamudāhāro, abhidhamme abhivinaye uḷārapāmojjo.}}\\
\begin{addmargin}[1em]{2em}
\setstretch{.5}
{\PaliGlossB{Furthermore, a mendicant loves the teachings and is a delight to converse with, being full of joy in the teaching and training.}}\\
\end{addmargin}
\end{absolutelynopagebreak}

\begin{absolutelynopagebreak}
\setstretch{.7}
{\PaliGlossA{‘dhammakāmo vatāyaṃ bhikkhu piyasamudāhāro, abhidhamme abhivinaye uḷārapāmojjo’ti therāpi naṃ bhikkhū vattabbaṃ anusāsitabbaṃ maññanti, majjhimāpi bhikkhū … navāpi bhikkhū vattabbaṃ anusāsitabbaṃ maññanti.}}\\
\begin{addmargin}[1em]{2em}
\setstretch{.5}
{\PaliGlossB{Knowing this, the mendicants—whether senior, middle, or junior—think that mendicant is worth advising and instructing.}}\\
\end{addmargin}
\end{absolutelynopagebreak}

\begin{absolutelynopagebreak}
\setstretch{.7}
{\PaliGlossA{tassa therānukampitassa majjhimānukampitassa navānukampitassa vuddhiyeva pāṭikaṅkhā kusalesu dhammesu, no parihāni.}}\\
\begin{addmargin}[1em]{2em}
\setstretch{.5}
{\PaliGlossB{Being treated with such kindness by the senior, middle, and junior mendicants, that mendicant can expect only growth, not decline.}}\\
\end{addmargin}
\end{absolutelynopagebreak}

\begin{absolutelynopagebreak}
\setstretch{.7}
{\PaliGlossA{ayampi dhammo nāthakaraṇo. (6)}}\\
\begin{addmargin}[1em]{2em}
\setstretch{.5}
{\PaliGlossB{This too is a quality that serves as protector.}}\\
\end{addmargin}
\end{absolutelynopagebreak}

\begin{absolutelynopagebreak}
\setstretch{.7}
{\PaliGlossA{puna caparaṃ, bhikkhave, bhikkhu āraddhavīriyo viharati akusalānaṃ dhammānaṃ pahānāya, kusalānaṃ dhammānaṃ upasampadāya, thāmavā daḷhaparakkamo anikkhittadhuro kusalesu dhammesu ‘āraddhavīriyo vatāyaṃ bhikkhu viharati akusalānaṃ dhammānaṃ pahānāya, kusalānaṃ dhammānaṃ upasampadāya, thāmavā daḷhaparakkamo anikkhittadhuro kusalesu dhammesū’ti therāpi naṃ bhikkhū vattabbaṃ anusāsitabbaṃ maññanti, majjhimāpi bhikkhū …}}\\
\begin{addmargin}[1em]{2em}
\setstretch{.5}
{\PaliGlossB{Furthermore, a mendicant lives with energy roused up for giving up unskillful qualities and embracing skillful qualities. They are strong, staunchly vigorous, not slacking off when it comes to developing skillful qualities.}}\\
\end{addmargin}
\end{absolutelynopagebreak}

\begin{absolutelynopagebreak}
\setstretch{.7}
{\PaliGlossA{navāpi bhikkhū vattabbaṃ anusāsitabbaṃ maññanti.}}\\
\begin{addmargin}[1em]{2em}
\setstretch{.5}
{\PaliGlossB{Knowing this, the mendicants—whether senior, middle, or junior—think that mendicant is worth advising and instructing.}}\\
\end{addmargin}
\end{absolutelynopagebreak}

\begin{absolutelynopagebreak}
\setstretch{.7}
{\PaliGlossA{tassa therānukampitassa majjhimānukampitassa navānukampitassa vuddhiyeva pāṭikaṅkhā kusalesu dhammesu, no parihāni.}}\\
\begin{addmargin}[1em]{2em}
\setstretch{.5}
{\PaliGlossB{Being treated with such kindness by the senior, middle, and junior mendicants, that mendicant can expect only growth, not decline.}}\\
\end{addmargin}
\end{absolutelynopagebreak}

\begin{absolutelynopagebreak}
\setstretch{.7}
{\PaliGlossA{ayampi dhammo nāthakaraṇo. (7)}}\\
\begin{addmargin}[1em]{2em}
\setstretch{.5}
{\PaliGlossB{This too is a quality that serves as protector.}}\\
\end{addmargin}
\end{absolutelynopagebreak}

\begin{absolutelynopagebreak}
\setstretch{.7}
{\PaliGlossA{puna caparaṃ, bhikkhave, bhikkhu santuṭṭho hoti itarītaracīvarapiṇḍapātasenāsanagilānapaccayabhesajjaparikkhārena.}}\\
\begin{addmargin}[1em]{2em}
\setstretch{.5}
{\PaliGlossB{Furthermore, a mendicant is content with any kind of robes, alms-food, lodgings, and medicines and supplies for the sick.}}\\
\end{addmargin}
\end{absolutelynopagebreak}

\begin{absolutelynopagebreak}
\setstretch{.7}
{\PaliGlossA{‘santuṭṭho vatāyaṃ bhikkhu itarītaracīvarapiṇḍapātasenāsanagilānapaccayabhesajjaparikkhārenā’ti therāpi naṃ bhikkhū vattabbaṃ anusāsitabbaṃ maññanti, majjhimāpi bhikkhū … navāpi bhikkhū vattabbaṃ anusāsitabbaṃ maññanti.}}\\
\begin{addmargin}[1em]{2em}
\setstretch{.5}
{\PaliGlossB{Knowing this, the mendicants—whether senior, middle, or junior—think that mendicant is worth advising and instructing.}}\\
\end{addmargin}
\end{absolutelynopagebreak}

\begin{absolutelynopagebreak}
\setstretch{.7}
{\PaliGlossA{tassa therānukampitassa majjhimānukampitassa navānukampitassa vuddhiyeva pāṭikaṅkhā kusalesu dhammesu, no parihāni.}}\\
\begin{addmargin}[1em]{2em}
\setstretch{.5}
{\PaliGlossB{Being treated with such kindness by the senior, middle, and junior mendicants, that mendicant can expect only growth, not decline.}}\\
\end{addmargin}
\end{absolutelynopagebreak}

\begin{absolutelynopagebreak}
\setstretch{.7}
{\PaliGlossA{ayampi dhammo nāthakaraṇo. (8)}}\\
\begin{addmargin}[1em]{2em}
\setstretch{.5}
{\PaliGlossB{This too is a quality that serves as protector.}}\\
\end{addmargin}
\end{absolutelynopagebreak}

\begin{absolutelynopagebreak}
\setstretch{.7}
{\PaliGlossA{puna caparaṃ, bhikkhave, bhikkhu satimā hoti paramena satinepakkena samannāgato, cirakatampi cirabhāsitampi saritā anussaritā.}}\\
\begin{addmargin}[1em]{2em}
\setstretch{.5}
{\PaliGlossB{Furthermore, a mendicant is mindful. They have utmost mindfulness and alertness, and can remember and recall what was said and done long ago.}}\\
\end{addmargin}
\end{absolutelynopagebreak}

\begin{absolutelynopagebreak}
\setstretch{.7}
{\PaliGlossA{‘satimā vatāyaṃ bhikkhu paramena satinepakkena samannāgato, cirakatampi cirabhāsitampi saritā anussaritā’ti therāpi naṃ bhikkhū vattabbaṃ anusāsitabbaṃ maññanti, majjhimāpi bhikkhū … navāpi bhikkhū vattabbaṃ anusāsitabbaṃ maññanti.}}\\
\begin{addmargin}[1em]{2em}
\setstretch{.5}
{\PaliGlossB{Knowing this, the mendicants—whether senior, middle, or junior—think that mendicant is worth advising and instructing.}}\\
\end{addmargin}
\end{absolutelynopagebreak}

\begin{absolutelynopagebreak}
\setstretch{.7}
{\PaliGlossA{tassa therānukampitassa majjhimānukampitassa navānukampitassa vuddhiyeva pāṭikaṅkhā kusalesu dhammesu, no parihāni.}}\\
\begin{addmargin}[1em]{2em}
\setstretch{.5}
{\PaliGlossB{Being treated with such kindness by the senior, middle, and junior mendicants, that mendicant can expect only growth, not decline.}}\\
\end{addmargin}
\end{absolutelynopagebreak}

\begin{absolutelynopagebreak}
\setstretch{.7}
{\PaliGlossA{ayampi dhammo nāthakaraṇo. (9)}}\\
\begin{addmargin}[1em]{2em}
\setstretch{.5}
{\PaliGlossB{This too is a quality that serves as protector.}}\\
\end{addmargin}
\end{absolutelynopagebreak}

\begin{absolutelynopagebreak}
\setstretch{.7}
{\PaliGlossA{puna caparaṃ, bhikkhave, bhikkhu paññavā hoti udayatthagāminiyā paññāya samannāgato ariyāya nibbedhikāya sammā dukkhakkhayagāminiyā.}}\\
\begin{addmargin}[1em]{2em}
\setstretch{.5}
{\PaliGlossB{Furthermore, a mendicant is wise. They have the wisdom of arising and passing away which is noble, penetrative, and leads to the complete ending of suffering.}}\\
\end{addmargin}
\end{absolutelynopagebreak}

\begin{absolutelynopagebreak}
\setstretch{.7}
{\PaliGlossA{‘paññavā vatāyaṃ bhikkhu udayatthagāminiyā paññāya samannāgato ariyāya nibbedhikāya sammā dukkhakkhayagāminiyā’ti therāpi naṃ bhikkhū vattabbaṃ anusāsitabbaṃ maññanti, majjhimāpi bhikkhū … navāpi bhikkhū vattabbaṃ anusāsitabbaṃ maññanti.}}\\
\begin{addmargin}[1em]{2em}
\setstretch{.5}
{\PaliGlossB{Knowing this, the mendicants—whether senior, middle, or junior—think that mendicant is worth advising and instructing.}}\\
\end{addmargin}
\end{absolutelynopagebreak}

\begin{absolutelynopagebreak}
\setstretch{.7}
{\PaliGlossA{tassa therānukampitassa … pe … no parihāni.}}\\
\begin{addmargin}[1em]{2em}
\setstretch{.5}
{\PaliGlossB{Being treated with such kindness by the senior, middle, and junior mendicants, that mendicant can expect only growth, not decline.}}\\
\end{addmargin}
\end{absolutelynopagebreak}

\begin{absolutelynopagebreak}
\setstretch{.7}
{\PaliGlossA{ayampi dhammo nāthakaraṇo. (10)}}\\
\begin{addmargin}[1em]{2em}
\setstretch{.5}
{\PaliGlossB{This too is a quality that serves as protector.}}\\
\end{addmargin}
\end{absolutelynopagebreak}

\begin{absolutelynopagebreak}
\setstretch{.7}
{\PaliGlossA{sanāthā, bhikkhave, viharatha, mā anāthā.}}\\
\begin{addmargin}[1em]{2em}
\setstretch{.5}
{\PaliGlossB{You should live with a protector, not without one.}}\\
\end{addmargin}
\end{absolutelynopagebreak}

\begin{absolutelynopagebreak}
\setstretch{.7}
{\PaliGlossA{dukkhaṃ, bhikkhave, anātho viharati.}}\\
\begin{addmargin}[1em]{2em}
\setstretch{.5}
{\PaliGlossB{Living without a protector is suffering.}}\\
\end{addmargin}
\end{absolutelynopagebreak}

\begin{absolutelynopagebreak}
\setstretch{.7}
{\PaliGlossA{ime kho, bhikkhave, dasa nāthakaraṇā dhammā”ti.}}\\
\begin{addmargin}[1em]{2em}
\setstretch{.5}
{\PaliGlossB{These are the ten qualities that serve as protector.”}}\\
\end{addmargin}
\end{absolutelynopagebreak}

\begin{absolutelynopagebreak}
\setstretch{.7}
{\PaliGlossA{idamavoca bhagavā.}}\\
\begin{addmargin}[1em]{2em}
\setstretch{.5}
{\PaliGlossB{That is what the Buddha said.}}\\
\end{addmargin}
\end{absolutelynopagebreak}

\begin{absolutelynopagebreak}
\setstretch{.7}
{\PaliGlossA{attamanā te bhikkhū bhagavato bhāsitaṃ abhinandunti.}}\\
\begin{addmargin}[1em]{2em}
\setstretch{.5}
{\PaliGlossB{Satisfied, the mendicants were happy with what the Buddha said.}}\\
\end{addmargin}
\end{absolutelynopagebreak}

\begin{absolutelynopagebreak}
\setstretch{.7}
{\PaliGlossA{aṭṭhamaṃ.}}\\
\begin{addmargin}[1em]{2em}
\setstretch{.5}
{\PaliGlossB{    -}}\\
\end{addmargin}
\end{absolutelynopagebreak}
