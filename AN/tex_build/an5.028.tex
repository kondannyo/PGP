
\begin{absolutelynopagebreak}
\setstretch{.7}
{\PaliGlossA{aṅguttara nikāya 5}}\\
\begin{addmargin}[1em]{2em}
\setstretch{.5}
{\PaliGlossB{Numbered Discourses 5}}\\
\end{addmargin}
\end{absolutelynopagebreak}

\begin{absolutelynopagebreak}
\setstretch{.7}
{\PaliGlossA{3. pañcaṅgikavagga}}\\
\begin{addmargin}[1em]{2em}
\setstretch{.5}
{\PaliGlossB{3. With Five Factors}}\\
\end{addmargin}
\end{absolutelynopagebreak}

\begin{absolutelynopagebreak}
\setstretch{.7}
{\PaliGlossA{28. pañcaṅgikasutta}}\\
\begin{addmargin}[1em]{2em}
\setstretch{.5}
{\PaliGlossB{28. With Five Factors}}\\
\end{addmargin}
\end{absolutelynopagebreak}

\begin{absolutelynopagebreak}
\setstretch{.7}
{\PaliGlossA{“ariyassa, bhikkhave, pañcaṅgikassa sammāsamādhissa bhāvanaṃ desessāmi.}}\\
\begin{addmargin}[1em]{2em}
\setstretch{.5}
{\PaliGlossB{“Mendicants, I will teach you how to develop noble right immersion with five factors.}}\\
\end{addmargin}
\end{absolutelynopagebreak}

\begin{absolutelynopagebreak}
\setstretch{.7}
{\PaliGlossA{taṃ suṇātha, sādhukaṃ manasi karotha, bhāsissāmī”ti.}}\\
\begin{addmargin}[1em]{2em}
\setstretch{.5}
{\PaliGlossB{Listen and pay close attention, I will speak.”}}\\
\end{addmargin}
\end{absolutelynopagebreak}

\begin{absolutelynopagebreak}
\setstretch{.7}
{\PaliGlossA{“evaṃ, bhante”ti kho te bhikkhū bhagavato paccassosuṃ.}}\\
\begin{addmargin}[1em]{2em}
\setstretch{.5}
{\PaliGlossB{“Yes, sir,” they replied.}}\\
\end{addmargin}
\end{absolutelynopagebreak}

\begin{absolutelynopagebreak}
\setstretch{.7}
{\PaliGlossA{bhagavā etadavoca:}}\\
\begin{addmargin}[1em]{2em}
\setstretch{.5}
{\PaliGlossB{The Buddha said this:}}\\
\end{addmargin}
\end{absolutelynopagebreak}

\begin{absolutelynopagebreak}
\setstretch{.7}
{\PaliGlossA{“katamā ca, bhikkhave, ariyassa pañcaṅgikassa sammāsamādhissa bhāvanā?}}\\
\begin{addmargin}[1em]{2em}
\setstretch{.5}
{\PaliGlossB{“And how do you develop noble right immersion with five factors?}}\\
\end{addmargin}
\end{absolutelynopagebreak}

\begin{absolutelynopagebreak}
\setstretch{.7}
{\PaliGlossA{idha, bhikkhave, bhikkhu vivicceva kāmehi … pe … paṭhamaṃ jhānaṃ upasampajja viharati.}}\\
\begin{addmargin}[1em]{2em}
\setstretch{.5}
{\PaliGlossB{Firstly, a mendicant, quite secluded from sensual pleasures, secluded from unskillful qualities, enters and remains in the first absorption. It has the rapture and bliss born of seclusion, while placing the mind and keeping it connected.}}\\
\end{addmargin}
\end{absolutelynopagebreak}

\begin{absolutelynopagebreak}
\setstretch{.7}
{\PaliGlossA{so imameva kāyaṃ vivekajena pītisukhena abhisandeti parisandeti paripūreti parippharati; nāssa kiñci sabbāvato kāyassa vivekajena pītisukhena apphuṭaṃ hoti.}}\\
\begin{addmargin}[1em]{2em}
\setstretch{.5}
{\PaliGlossB{They drench, steep, fill, and spread their body with rapture and bliss born of seclusion. There’s no part of the body that’s not spread with rapture and bliss born of seclusion.}}\\
\end{addmargin}
\end{absolutelynopagebreak}

\begin{absolutelynopagebreak}
\setstretch{.7}
{\PaliGlossA{seyyathāpi, bhikkhave, dakkho nhāpako vā nhāpakantevāsī vā kaṃsathāle nhānīyacuṇṇāni ākiritvā udakena paripphosakaṃ paripphosakaṃ sanneyya. sāyaṃ nhānīyapiṇḍi snehānugatā snehaparetā santarabāhirā phuṭā snehena, na ca paggharinī.}}\\
\begin{addmargin}[1em]{2em}
\setstretch{.5}
{\PaliGlossB{It’s like when a deft bathroom attendant or their apprentice pours bath powder into a bronze dish, sprinkling it little by little with water. They knead it until the ball of bath powder is soaked and saturated with moisture, spread through inside and out; yet no moisture oozes out.}}\\
\end{addmargin}
\end{absolutelynopagebreak}

\begin{absolutelynopagebreak}
\setstretch{.7}
{\PaliGlossA{evamevaṃ kho, bhikkhave, bhikkhu imameva kāyaṃ vivekajena pītisukhena abhisandeti parisandeti paripūreti parippharati;}}\\
\begin{addmargin}[1em]{2em}
\setstretch{.5}
{\PaliGlossB{In the same way, a mendicant drenches, steeps, fills, and spreads their body with rapture and bliss born of seclusion.}}\\
\end{addmargin}
\end{absolutelynopagebreak}

\begin{absolutelynopagebreak}
\setstretch{.7}
{\PaliGlossA{nāssa kiñci sabbāvato kāyassa vivekajena pītisukhena apphuṭaṃ hoti.}}\\
\begin{addmargin}[1em]{2em}
\setstretch{.5}
{\PaliGlossB{There’s no part of the body that’s not spread with rapture and bliss born of seclusion.}}\\
\end{addmargin}
\end{absolutelynopagebreak}

\begin{absolutelynopagebreak}
\setstretch{.7}
{\PaliGlossA{ariyassa, bhikkhave, pañcaṅgikassa sammāsamādhissa ayaṃ paṭhamā bhāvanā. (1)}}\\
\begin{addmargin}[1em]{2em}
\setstretch{.5}
{\PaliGlossB{This is the first way to develop noble right immersion with five factors.}}\\
\end{addmargin}
\end{absolutelynopagebreak}

\begin{absolutelynopagebreak}
\setstretch{.7}
{\PaliGlossA{puna caparaṃ, bhikkhave, bhikkhu vitakkavicārānaṃ vūpasamā … pe … dutiyaṃ jhānaṃ upasampajja viharati.}}\\
\begin{addmargin}[1em]{2em}
\setstretch{.5}
{\PaliGlossB{Furthermore, as the placing of the mind and keeping it connected are stilled, a mendicant enters and remains in the second absorption. It has the rapture and bliss born of immersion, with internal clarity and confidence, and unified mind, without placing the mind and keeping it connected.}}\\
\end{addmargin}
\end{absolutelynopagebreak}

\begin{absolutelynopagebreak}
\setstretch{.7}
{\PaliGlossA{so imameva kāyaṃ samādhijena pītisukhena abhisandeti parisandeti paripūreti parippharati;}}\\
\begin{addmargin}[1em]{2em}
\setstretch{.5}
{\PaliGlossB{They drench, steep, fill, and spread their body with rapture and bliss born of immersion.}}\\
\end{addmargin}
\end{absolutelynopagebreak}

\begin{absolutelynopagebreak}
\setstretch{.7}
{\PaliGlossA{nāssa kiñci sabbāvato kāyassa samādhijena pītisukhena apphuṭaṃ hoti.}}\\
\begin{addmargin}[1em]{2em}
\setstretch{.5}
{\PaliGlossB{There’s no part of the body that’s not spread with rapture and bliss born of immersion.}}\\
\end{addmargin}
\end{absolutelynopagebreak}

\begin{absolutelynopagebreak}
\setstretch{.7}
{\PaliGlossA{seyyathāpi, bhikkhave, udakarahado gambhīro ubbhidodako. tassa nevassa puratthimāya disāya udakassa āyamukhaṃ, na pacchimāya disāya udakassa āyamukhaṃ, na uttarāya disāya udakassa āyamukhaṃ, na dakkhiṇāya disāya udakassa āyamukhaṃ, devo ca kālena kālaṃ sammā dhāraṃ nānuppaveccheyya.}}\\
\begin{addmargin}[1em]{2em}
\setstretch{.5}
{\PaliGlossB{It’s like a deep lake fed by spring water. There’s no inlet to the east, west, north, or south, and no rainfall to replenish it from time to time.}}\\
\end{addmargin}
\end{absolutelynopagebreak}

\begin{absolutelynopagebreak}
\setstretch{.7}
{\PaliGlossA{atha kho tamhāva udakarahadā sītā vāridhārā ubbhijjitvā tameva udakarahadaṃ sītena vārinā abhisandeyya parisandeyya paripūreyya paripphareyya; nāssa kiñci sabbāvato udakarahadassa sītena vārinā apphuṭaṃ assa.}}\\
\begin{addmargin}[1em]{2em}
\setstretch{.5}
{\PaliGlossB{But the stream of cool water welling up in the lake drenches, steeps, fills, and spreads throughout the lake. There’s no part of the lake that’s not spread through with cool water.}}\\
\end{addmargin}
\end{absolutelynopagebreak}

\begin{absolutelynopagebreak}
\setstretch{.7}
{\PaliGlossA{evamevaṃ kho, bhikkhave, bhikkhu imameva kāyaṃ samādhijena pītisukhena abhisandeti parisandeti paripūreti parippharati;}}\\
\begin{addmargin}[1em]{2em}
\setstretch{.5}
{\PaliGlossB{In the same way, a mendicant drenches, steeps, fills, and spreads their body with rapture and bliss born of immersion.}}\\
\end{addmargin}
\end{absolutelynopagebreak}

\begin{absolutelynopagebreak}
\setstretch{.7}
{\PaliGlossA{nāssa kiñci sabbāvato kāyassa samādhijena pītisukhena apphuṭaṃ hoti.}}\\
\begin{addmargin}[1em]{2em}
\setstretch{.5}
{\PaliGlossB{There’s no part of the body that’s not spread with rapture and bliss born of immersion.}}\\
\end{addmargin}
\end{absolutelynopagebreak}

\begin{absolutelynopagebreak}
\setstretch{.7}
{\PaliGlossA{ariyassa, bhikkhave, pañcaṅgikassa sammāsamādhissa ayaṃ dutiyā bhāvanā. (2)}}\\
\begin{addmargin}[1em]{2em}
\setstretch{.5}
{\PaliGlossB{This is the second way to develop noble right immersion with five factors.}}\\
\end{addmargin}
\end{absolutelynopagebreak}

\begin{absolutelynopagebreak}
\setstretch{.7}
{\PaliGlossA{puna caparaṃ, bhikkhave, bhikkhu pītiyā ca virāgā … pe … tatiyaṃ jhānaṃ upasampajja viharati.}}\\
\begin{addmargin}[1em]{2em}
\setstretch{.5}
{\PaliGlossB{Furthermore, with the fading away of rapture, a mendicant enters and remains in the third absorption. They meditate with equanimity, mindful and aware, personally experiencing the bliss of which the noble ones declare, ‘Equanimous and mindful, one meditates in bliss.’}}\\
\end{addmargin}
\end{absolutelynopagebreak}

\begin{absolutelynopagebreak}
\setstretch{.7}
{\PaliGlossA{so imameva kāyaṃ nippītikena sukhena abhisandeti parisandeti paripūreti parippharati;}}\\
\begin{addmargin}[1em]{2em}
\setstretch{.5}
{\PaliGlossB{They drench, steep, fill, and spread their body with bliss free of rapture.}}\\
\end{addmargin}
\end{absolutelynopagebreak}

\begin{absolutelynopagebreak}
\setstretch{.7}
{\PaliGlossA{nāssa kiñci sabbāvato kāyassa nippītikena sukhena apphuṭaṃ hoti.}}\\
\begin{addmargin}[1em]{2em}
\setstretch{.5}
{\PaliGlossB{There’s no part of the body that’s not spread with bliss free of rapture.}}\\
\end{addmargin}
\end{absolutelynopagebreak}

\begin{absolutelynopagebreak}
\setstretch{.7}
{\PaliGlossA{seyyathāpi, bhikkhave, uppaliniyaṃ vā paduminiyaṃ vā puṇḍarīkiniyaṃ vā appekaccāni uppalāni vā padumāni vā puṇḍarīkāni vā udake jātāni udake saṃvaḍḍhāni udakānuggatāni anto nimuggaposīni.}}\\
\begin{addmargin}[1em]{2em}
\setstretch{.5}
{\PaliGlossB{It’s like a pool with blue water lilies, or pink or white lotuses. Some of them sprout and grow in the water without rising above it, thriving underwater.}}\\
\end{addmargin}
\end{absolutelynopagebreak}

\begin{absolutelynopagebreak}
\setstretch{.7}
{\PaliGlossA{tāni yāva caggā yāva ca mūlā sītena vārinā abhisannāni parisannāni paripūrāni paripphuṭāni;}}\\
\begin{addmargin}[1em]{2em}
\setstretch{.5}
{\PaliGlossB{From the tip to the root they’re drenched, steeped, filled, and soaked with cool water.}}\\
\end{addmargin}
\end{absolutelynopagebreak}

\begin{absolutelynopagebreak}
\setstretch{.7}
{\PaliGlossA{nāssa kiñci sabbāvataṃ uppalānaṃ vā padumānaṃ vā puṇḍarīkānaṃ vā sītena vārinā apphuṭaṃ assa.}}\\
\begin{addmargin}[1em]{2em}
\setstretch{.5}
{\PaliGlossB{There’s no part of them that’s not spread through with cool water.}}\\
\end{addmargin}
\end{absolutelynopagebreak}

\begin{absolutelynopagebreak}
\setstretch{.7}
{\PaliGlossA{evamevaṃ kho, bhikkhave, bhikkhu imameva kāyaṃ nippītikena sukhena abhisandeti parisandeti paripūreti parippharati;}}\\
\begin{addmargin}[1em]{2em}
\setstretch{.5}
{\PaliGlossB{In the same way, a mendicant drenches, steeps, fills, and spreads their body with bliss free of rapture.}}\\
\end{addmargin}
\end{absolutelynopagebreak}

\begin{absolutelynopagebreak}
\setstretch{.7}
{\PaliGlossA{nāssa kiñci sabbāvato kāyassa nippītikena sukhena apphuṭaṃ hoti.}}\\
\begin{addmargin}[1em]{2em}
\setstretch{.5}
{\PaliGlossB{There’s no part of the body that’s not spread with bliss free of rapture.}}\\
\end{addmargin}
\end{absolutelynopagebreak}

\begin{absolutelynopagebreak}
\setstretch{.7}
{\PaliGlossA{ariyassa, bhikkhave, pañcaṅgikassa sammāsamādhissa ayaṃ tatiyā bhāvanā. (3)}}\\
\begin{addmargin}[1em]{2em}
\setstretch{.5}
{\PaliGlossB{This is the third way to develop noble right immersion with five factors.}}\\
\end{addmargin}
\end{absolutelynopagebreak}

\begin{absolutelynopagebreak}
\setstretch{.7}
{\PaliGlossA{puna caparaṃ, bhikkhave, bhikkhu sukhassa ca pahānā … pe … catutthaṃ jhānaṃ upasampajja viharati.}}\\
\begin{addmargin}[1em]{2em}
\setstretch{.5}
{\PaliGlossB{Furthermore, giving up pleasure and pain, and ending former happiness and sadness, a mendicant enters and remains in the fourth absorption. It is without pleasure or pain, with pure equanimity and mindfulness.}}\\
\end{addmargin}
\end{absolutelynopagebreak}

\begin{absolutelynopagebreak}
\setstretch{.7}
{\PaliGlossA{so imameva kāyaṃ parisuddhena cetasā pariyodātena pharitvā nisinno hoti;}}\\
\begin{addmargin}[1em]{2em}
\setstretch{.5}
{\PaliGlossB{They sit spreading their body through with pure bright mind.}}\\
\end{addmargin}
\end{absolutelynopagebreak}

\begin{absolutelynopagebreak}
\setstretch{.7}
{\PaliGlossA{nāssa kiñci sabbāvato kāyassa parisuddhena cetasā pariyodātena apphuṭaṃ hoti.}}\\
\begin{addmargin}[1em]{2em}
\setstretch{.5}
{\PaliGlossB{There’s no part of the body that’s not spread with pure bright mind.}}\\
\end{addmargin}
\end{absolutelynopagebreak}

\begin{absolutelynopagebreak}
\setstretch{.7}
{\PaliGlossA{seyyathāpi, bhikkhave, puriso odātena vatthena sasīsaṃ pārupitvā nisinno assa;}}\\
\begin{addmargin}[1em]{2em}
\setstretch{.5}
{\PaliGlossB{It’s like someone sitting wrapped from head to foot with white cloth.}}\\
\end{addmargin}
\end{absolutelynopagebreak}

\begin{absolutelynopagebreak}
\setstretch{.7}
{\PaliGlossA{nāssa kiñci sabbāvato kāyassa odātena vatthena apphuṭaṃ assa.}}\\
\begin{addmargin}[1em]{2em}
\setstretch{.5}
{\PaliGlossB{There’s no part of the body that’s not spread over with white cloth.}}\\
\end{addmargin}
\end{absolutelynopagebreak}

\begin{absolutelynopagebreak}
\setstretch{.7}
{\PaliGlossA{evamevaṃ kho, bhikkhave, bhikkhu imameva kāyaṃ parisuddhena cetasā pariyodātena pharitvā nisinno hoti;}}\\
\begin{addmargin}[1em]{2em}
\setstretch{.5}
{\PaliGlossB{In the same way, they sit spreading their body through with pure bright mind.}}\\
\end{addmargin}
\end{absolutelynopagebreak}

\begin{absolutelynopagebreak}
\setstretch{.7}
{\PaliGlossA{nāssa kiñci sabbāvato kāyassa parisuddhena cetasā pariyodātena apphuṭaṃ hoti.}}\\
\begin{addmargin}[1em]{2em}
\setstretch{.5}
{\PaliGlossB{There’s no part of the body that’s not spread with pure bright mind.}}\\
\end{addmargin}
\end{absolutelynopagebreak}

\begin{absolutelynopagebreak}
\setstretch{.7}
{\PaliGlossA{ariyassa, bhikkhave, pañcaṅgikassa sammāsamādhissa ayaṃ catutthā bhāvanā. (4)}}\\
\begin{addmargin}[1em]{2em}
\setstretch{.5}
{\PaliGlossB{This is the fourth way to develop noble right immersion with five factors.}}\\
\end{addmargin}
\end{absolutelynopagebreak}

\begin{absolutelynopagebreak}
\setstretch{.7}
{\PaliGlossA{puna caparaṃ, bhikkhave, bhikkhuno paccavekkhaṇānimittaṃ suggahitaṃ hoti sumanasikataṃ sūpadhāritaṃ suppaṭividdhaṃ paññāya.}}\\
\begin{addmargin}[1em]{2em}
\setstretch{.5}
{\PaliGlossB{Furthermore, the meditation that is a foundation for reviewing is properly grasped, attended, borne in mind, and comprehended with wisdom by a mendicant.}}\\
\end{addmargin}
\end{absolutelynopagebreak}

\begin{absolutelynopagebreak}
\setstretch{.7}
{\PaliGlossA{seyyathāpi, bhikkhave, aññova aññaṃ paccavekkheyya, ṭhito vā nisinnaṃ paccavekkheyya, nisinno vā nipannaṃ paccavekkheyya.}}\\
\begin{addmargin}[1em]{2em}
\setstretch{.5}
{\PaliGlossB{It’s like when someone views someone else. Someone standing might view someone sitting, or someone sitting might view someone lying down.}}\\
\end{addmargin}
\end{absolutelynopagebreak}

\begin{absolutelynopagebreak}
\setstretch{.7}
{\PaliGlossA{evamevaṃ kho, bhikkhave, bhikkhuno paccavekkhaṇānimittaṃ suggahitaṃ hoti sumanasikataṃ sūpadhāritaṃ suppaṭividdhaṃ paññāya.}}\\
\begin{addmargin}[1em]{2em}
\setstretch{.5}
{\PaliGlossB{In the same way, the meditation that is a foundation for reviewing is properly grasped, attended, borne in mind, and comprehended with wisdom by a mendicant.}}\\
\end{addmargin}
\end{absolutelynopagebreak}

\begin{absolutelynopagebreak}
\setstretch{.7}
{\PaliGlossA{ariyassa, bhikkhave, pañcaṅgikassa sammāsamādhissa ayaṃ pañcamā bhāvanā. (5)}}\\
\begin{addmargin}[1em]{2em}
\setstretch{.5}
{\PaliGlossB{This is the fifth way to develop noble right immersion with five factors.}}\\
\end{addmargin}
\end{absolutelynopagebreak}

\begin{absolutelynopagebreak}
\setstretch{.7}
{\PaliGlossA{evaṃ bhāvite kho, bhikkhave, bhikkhu ariye pañcaṅgike sammāsamādhimhi evaṃ bahulīkate yassa yassa abhiññāsacchikaraṇīyassa dhammassa cittaṃ abhininnāmeti abhiññāsacchikiriyāya, tatra tatreva sakkhibhabbataṃ pāpuṇāti sati sati āyatane.}}\\
\begin{addmargin}[1em]{2em}
\setstretch{.5}
{\PaliGlossB{When the noble right immersion with five factors is cultivated in this way, a mendicant becomes capable of realizing anything that can be realized by insight to which they extend the mind, in each and every case.}}\\
\end{addmargin}
\end{absolutelynopagebreak}

\begin{absolutelynopagebreak}
\setstretch{.7}
{\PaliGlossA{seyyathāpi, bhikkhave, udakamaṇiko ādhāre ṭhapito pūro udakassa samatittiko kākapeyyo.}}\\
\begin{addmargin}[1em]{2em}
\setstretch{.5}
{\PaliGlossB{Suppose a water jar was placed on a stand, full to the brim so a crow could drink from it.}}\\
\end{addmargin}
\end{absolutelynopagebreak}

\begin{absolutelynopagebreak}
\setstretch{.7}
{\PaliGlossA{tamenaṃ balavā puriso yato yato āvajjeyya, āgaccheyya udakan”ti?}}\\
\begin{addmargin}[1em]{2em}
\setstretch{.5}
{\PaliGlossB{If a strong man was to tip it any which way, would water pour out?”}}\\
\end{addmargin}
\end{absolutelynopagebreak}

\begin{absolutelynopagebreak}
\setstretch{.7}
{\PaliGlossA{“evaṃ, bhante”.}}\\
\begin{addmargin}[1em]{2em}
\setstretch{.5}
{\PaliGlossB{“Yes, sir.”}}\\
\end{addmargin}
\end{absolutelynopagebreak}

\begin{absolutelynopagebreak}
\setstretch{.7}
{\PaliGlossA{“evamevaṃ kho, bhikkhave, bhikkhu evaṃ bhāvite ariye pañcaṅgike sammāsamādhimhi evaṃ bahulīkate yassa yassa abhiññāsacchikaraṇīyassa dhammassa cittaṃ abhininnāmeti abhiññāsacchikiriyāya, tatra tatreva sakkhibhabbataṃ pāpuṇāti sati sati āyatane.}}\\
\begin{addmargin}[1em]{2em}
\setstretch{.5}
{\PaliGlossB{“In the same way, when noble right immersion with five factors is cultivated in this way, a mendicant becomes capable of realizing anything that can be realized by insight to which they extend the mind, in each and every case.}}\\
\end{addmargin}
\end{absolutelynopagebreak}

\begin{absolutelynopagebreak}
\setstretch{.7}
{\PaliGlossA{seyyathāpi, bhikkhave, same bhūmibhāge pokkharaṇī caturaṃsā ālibaddhā pūrā udakassa samatittikā kākapeyyā.}}\\
\begin{addmargin}[1em]{2em}
\setstretch{.5}
{\PaliGlossB{Suppose there was a square, walled lotus pond on level ground, full to the brim so a crow could drink from it.}}\\
\end{addmargin}
\end{absolutelynopagebreak}

\begin{absolutelynopagebreak}
\setstretch{.7}
{\PaliGlossA{tamenaṃ balavā puriso yato yato āliṃ muñceyya, āgaccheyya udakan”ti?}}\\
\begin{addmargin}[1em]{2em}
\setstretch{.5}
{\PaliGlossB{If a strong man was to open the wall on any side, would water pour out?”}}\\
\end{addmargin}
\end{absolutelynopagebreak}

\begin{absolutelynopagebreak}
\setstretch{.7}
{\PaliGlossA{“evaṃ, bhante”.}}\\
\begin{addmargin}[1em]{2em}
\setstretch{.5}
{\PaliGlossB{“Yes, sir.”}}\\
\end{addmargin}
\end{absolutelynopagebreak}

\begin{absolutelynopagebreak}
\setstretch{.7}
{\PaliGlossA{“evamevaṃ kho, bhikkhave, bhikkhu evaṃ bhāvite ariye pañcaṅgike sammāsamādhimhi evaṃ bahulīkate yassa yassa abhiññāsacchikaraṇīyassa dhammassa … pe … sati sati āyatane.}}\\
\begin{addmargin}[1em]{2em}
\setstretch{.5}
{\PaliGlossB{“In the same way, when noble right immersion with five factors is cultivated in this way, a mendicant becomes capable of realizing anything that can be realized by insight to which they extend the mind, in each and every case.}}\\
\end{addmargin}
\end{absolutelynopagebreak}

\begin{absolutelynopagebreak}
\setstretch{.7}
{\PaliGlossA{seyyathāpi, bhikkhave, subhūmiyaṃ catumahāpathe ājaññaratho yutto assa ṭhito odhastapatodo.}}\\
\begin{addmargin}[1em]{2em}
\setstretch{.5}
{\PaliGlossB{Suppose a chariot stood harnessed to thoroughbreds at a level crossroads, with a goad ready.}}\\
\end{addmargin}
\end{absolutelynopagebreak}

\begin{absolutelynopagebreak}
\setstretch{.7}
{\PaliGlossA{tamenaṃ dakkho yoggācariyo assadammasārathi abhiruhitvā vāmena hatthena rasmiyo gahetvā dakkhiṇena hatthena patodaṃ gahetvā yenicchakaṃ yadicchakaṃ sāreyyapi paccāsāreyyapi.}}\\
\begin{addmargin}[1em]{2em}
\setstretch{.5}
{\PaliGlossB{Then a deft horse trainer, a master charioteer, might mount the chariot, taking the reins in his right hand and goad in the left. He’d drive out and back wherever he wishes, whenever he wishes.}}\\
\end{addmargin}
\end{absolutelynopagebreak}

\begin{absolutelynopagebreak}
\setstretch{.7}
{\PaliGlossA{evamevaṃ kho, bhikkhave, bhikkhu evaṃ bhāvite ariye pañcaṅgike sammāsamādhimhi evaṃ bahulīkate yassa yassa abhiññāsacchikaraṇīyassa dhammassa cittaṃ abhininnāmeti abhiññāsacchikiriyāya,}}\\
\begin{addmargin}[1em]{2em}
\setstretch{.5}
{\PaliGlossB{In the same way, when noble right immersion with five factors is cultivated in this way,}}\\
\end{addmargin}
\end{absolutelynopagebreak}

\begin{absolutelynopagebreak}
\setstretch{.7}
{\PaliGlossA{tatra tatreva sakkhibhabbataṃ pāpuṇāti sati sati āyatane.}}\\
\begin{addmargin}[1em]{2em}
\setstretch{.5}
{\PaliGlossB{a mendicant becomes capable of realizing anything that can be realized by insight to which they extend the mind, in each and every case.}}\\
\end{addmargin}
\end{absolutelynopagebreak}

\begin{absolutelynopagebreak}
\setstretch{.7}
{\PaliGlossA{so sace ākaṅkhati: ‘anekavihitaṃ iddhividhaṃ paccanubhaveyyaṃ—ekopi hutvā bahudhā assaṃ … pe … yāva brahmalokāpi kāyena vasaṃ vatteyyan’ti,}}\\
\begin{addmargin}[1em]{2em}
\setstretch{.5}
{\PaliGlossB{If you wish: ‘May I wield the many kinds of psychic power: multiplying myself and becoming one again … controlling the body as far as the Brahmā realm.’}}\\
\end{addmargin}
\end{absolutelynopagebreak}

\begin{absolutelynopagebreak}
\setstretch{.7}
{\PaliGlossA{tatra tatreva sakkhibhabbataṃ pāpuṇāti sati sati āyatane.}}\\
\begin{addmargin}[1em]{2em}
\setstretch{.5}
{\PaliGlossB{You’re capable of realizing it, in each and every case.}}\\
\end{addmargin}
\end{absolutelynopagebreak}

\begin{absolutelynopagebreak}
\setstretch{.7}
{\PaliGlossA{so sace ākaṅkhati: ‘dibbāya sotadhātuyā visuddhāya atikkantamānusikāya ubho sadde suṇeyyaṃ—dibbe ca mānuse ca ye dūre santike cā’ti,}}\\
\begin{addmargin}[1em]{2em}
\setstretch{.5}
{\PaliGlossB{If you wish: ‘With clairaudience that is purified and superhuman, may I hear both kinds of sounds, human and divine, whether near or far.’}}\\
\end{addmargin}
\end{absolutelynopagebreak}

\begin{absolutelynopagebreak}
\setstretch{.7}
{\PaliGlossA{tatra tatreva sakkhibhabbataṃ pāpuṇāti sati sati āyatane.}}\\
\begin{addmargin}[1em]{2em}
\setstretch{.5}
{\PaliGlossB{You’re capable of realizing it, in each and every case.}}\\
\end{addmargin}
\end{absolutelynopagebreak}

\begin{absolutelynopagebreak}
\setstretch{.7}
{\PaliGlossA{so sace ākaṅkhati: ‘parasattānaṃ parapuggalānaṃ cetasā ceto paricca pajāneyyaṃ—sarāgaṃ vā cittaṃ sarāgaṃ cittanti pajāneyyaṃ, vītarāgaṃ vā cittaṃ vītarāgaṃ cittanti pajāneyyaṃ, sadosaṃ vā cittaṃ … vītadosaṃ vā cittaṃ … samohaṃ vā cittaṃ … vītamohaṃ vā cittaṃ … saṅkhittaṃ vā cittaṃ … vikkhittaṃ vā cittaṃ … mahaggataṃ vā cittaṃ … amahaggataṃ vā cittaṃ … sauttaraṃ vā cittaṃ … anuttaraṃ vā cittaṃ … samāhitaṃ vā cittaṃ … asamāhitaṃ vā cittaṃ … vimuttaṃ vā cittaṃ … avimuttaṃ vā cittaṃ avimuttaṃ cittanti pajāneyyan’ti,}}\\
\begin{addmargin}[1em]{2em}
\setstretch{.5}
{\PaliGlossB{If you wish: ‘May I understand the minds of other beings and individuals, having comprehended them with my mind. May I understand mind with greed as “mind with greed”, and mind without greed as “mind without greed”; mind with hate as “mind with hate”, and mind without hate as “mind without hate”; mind with delusion as “mind with delusion”, and mind without delusion as “mind without delusion”; constricted mind as “constricted mind”, and scattered mind as “scattered mind”; expansive mind as “expansive mind”, and unexpansive mind as “unexpansive mind”; mind that is not supreme as “mind that is not supreme”, and mind that is supreme as “mind that is supreme”; mind immersed in samādhi as “mind immersed in samādhi”, and mind not immersed in samādhi as “mind not immersed in samādhi”; freed mind as “freed mind”, and unfreed mind as “unfreed mind”.’}}\\
\end{addmargin}
\end{absolutelynopagebreak}

\begin{absolutelynopagebreak}
\setstretch{.7}
{\PaliGlossA{tatra tatreva sakkhibhabbataṃ pāpuṇāti sati sati āyatane.}}\\
\begin{addmargin}[1em]{2em}
\setstretch{.5}
{\PaliGlossB{You’re capable of realizing it, in each and every case.}}\\
\end{addmargin}
\end{absolutelynopagebreak}

\begin{absolutelynopagebreak}
\setstretch{.7}
{\PaliGlossA{so sace ākaṅkhati: ‘anekavihitaṃ pubbenivāsaṃ anussareyyaṃ, seyyathidaṃ—ekampi jātiṃ, dvepi jātiyo … pe … iti sākāraṃ sauddesaṃ anekavihitaṃ pubbenivāsaṃ anussareyyan’ti,}}\\
\begin{addmargin}[1em]{2em}
\setstretch{.5}
{\PaliGlossB{If you wish: ‘May I recollect many kinds of past lives, with features and details.’}}\\
\end{addmargin}
\end{absolutelynopagebreak}

\begin{absolutelynopagebreak}
\setstretch{.7}
{\PaliGlossA{tatra tatreva sakkhibhabbataṃ pāpuṇāti sati sati āyatane.}}\\
\begin{addmargin}[1em]{2em}
\setstretch{.5}
{\PaliGlossB{You’re capable of realizing it, in each and every case.}}\\
\end{addmargin}
\end{absolutelynopagebreak}

\begin{absolutelynopagebreak}
\setstretch{.7}
{\PaliGlossA{so sace ākaṅkhati: ‘dibbena cakkhunā visuddhena atikkantamānusakena … pe … yathākammūpage satte pajāneyyan’ti,}}\\
\begin{addmargin}[1em]{2em}
\setstretch{.5}
{\PaliGlossB{If you wish: ‘With clairvoyance that is purified and superhuman, may I see sentient beings passing away and being reborn according to their deeds.’}}\\
\end{addmargin}
\end{absolutelynopagebreak}

\begin{absolutelynopagebreak}
\setstretch{.7}
{\PaliGlossA{tatra tatreva sakkhibhabbataṃ pāpuṇāti sati sati āyatane.}}\\
\begin{addmargin}[1em]{2em}
\setstretch{.5}
{\PaliGlossB{You’re capable of realizing it, in each and every case.}}\\
\end{addmargin}
\end{absolutelynopagebreak}

\begin{absolutelynopagebreak}
\setstretch{.7}
{\PaliGlossA{so sace ākaṅkhati: ‘āsavānaṃ khayā anāsavaṃ cetovimuttiṃ paññāvimuttiṃ diṭṭheva dhamme sayaṃ abhiññā sacchikatvā upasampajja vihareyyan’ti,}}\\
\begin{addmargin}[1em]{2em}
\setstretch{.5}
{\PaliGlossB{If you wish: ‘May I realize the undefiled freedom of heart and freedom by wisdom in this very life, and live having realized it with my own insight due to the ending of defilements.’}}\\
\end{addmargin}
\end{absolutelynopagebreak}

\begin{absolutelynopagebreak}
\setstretch{.7}
{\PaliGlossA{tatra tatreva sakkhibhabbataṃ pāpuṇāti sati sati āyatane”ti.}}\\
\begin{addmargin}[1em]{2em}
\setstretch{.5}
{\PaliGlossB{You’re capable of realizing it, in each and every case.”}}\\
\end{addmargin}
\end{absolutelynopagebreak}

\begin{absolutelynopagebreak}
\setstretch{.7}
{\PaliGlossA{aṭṭhamaṃ.}}\\
\begin{addmargin}[1em]{2em}
\setstretch{.5}
{\PaliGlossB{    -}}\\
\end{addmargin}
\end{absolutelynopagebreak}
