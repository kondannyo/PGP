
\begin{absolutelynopagebreak}
\setstretch{.7}
{\PaliGlossA{aṅguttara nikāya 10}}\\
\begin{addmargin}[1em]{2em}
\setstretch{.5}
{\PaliGlossB{Numbered Discourses 10}}\\
\end{addmargin}
\end{absolutelynopagebreak}

\begin{absolutelynopagebreak}
\setstretch{.7}
{\PaliGlossA{4. upālivagga}}\\
\begin{addmargin}[1em]{2em}
\setstretch{.5}
{\PaliGlossB{4. With Upāli}}\\
\end{addmargin}
\end{absolutelynopagebreak}

\begin{absolutelynopagebreak}
\setstretch{.7}
{\PaliGlossA{37. saṅghabhedasutta}}\\
\begin{addmargin}[1em]{2em}
\setstretch{.5}
{\PaliGlossB{37. Schism in the Saṅgha}}\\
\end{addmargin}
\end{absolutelynopagebreak}

\begin{absolutelynopagebreak}
\setstretch{.7}
{\PaliGlossA{“‘saṃghabhedo, saṃghabhedo’ti, bhante, vuccati.}}\\
\begin{addmargin}[1em]{2em}
\setstretch{.5}
{\PaliGlossB{“Sir, they speak of ‘schism in the Saṅgha’.}}\\
\end{addmargin}
\end{absolutelynopagebreak}

\begin{absolutelynopagebreak}
\setstretch{.7}
{\PaliGlossA{kittāvatā nu kho, bhante, saṃgho bhinno hotī”ti?}}\\
\begin{addmargin}[1em]{2em}
\setstretch{.5}
{\PaliGlossB{How is schism in the Saṅgha defined?”}}\\
\end{addmargin}
\end{absolutelynopagebreak}

\begin{absolutelynopagebreak}
\setstretch{.7}
{\PaliGlossA{“idhupāli, bhikkhū adhammaṃ dhammoti dīpenti, dhammaṃ adhammoti dīpenti, avinayaṃ vinayoti dīpenti, vinayaṃ avinayoti dīpenti, abhāsitaṃ alapitaṃ tathāgatena bhāsitaṃ lapitaṃ tathāgatenāti dīpenti, bhāsitaṃ lapitaṃ tathāgatena abhāsitaṃ alapitaṃ tathāgatenāti dīpenti, anāciṇṇaṃ tathāgatena āciṇṇaṃ tathāgatenāti dīpenti, āciṇṇaṃ tathāgatena anāciṇṇaṃ tathāgatenāti dīpenti, apaññattaṃ tathāgatena paññattaṃ tathāgatenāti dīpenti, paññattaṃ tathāgatena apaññattaṃ tathāgatenāti dīpenti.}}\\
\begin{addmargin}[1em]{2em}
\setstretch{.5}
{\PaliGlossB{“Upāli, it’s when a mendicant explains what is not the teaching as the teaching, and what is the teaching as not the teaching. They explain what is not the training as the training, and what is the training as not the training. They explain what was not spoken and stated by the Realized One as spoken and stated by the Realized One, and what was spoken and stated by the Realized One as not spoken and stated by the Realized One. They explain what was not practiced by the Realized One as practiced by the Realized One, and what was practiced by the Realized One as not practiced by the Realized One. They explain what was not prescribed by the Realized One as prescribed by the Realized One, and what was prescribed by the Realized One as not prescribed by the Realized One.}}\\
\end{addmargin}
\end{absolutelynopagebreak}

\begin{absolutelynopagebreak}
\setstretch{.7}
{\PaliGlossA{te imehi dasahi vatthūhi avakassanti apakassanti āveni kammāni karonti āveni pātimokkhaṃ uddisanti.}}\\
\begin{addmargin}[1em]{2em}
\setstretch{.5}
{\PaliGlossB{On these ten grounds they split off and go their own way. They perform legal acts autonomously and recite the monastic code autonomously.}}\\
\end{addmargin}
\end{absolutelynopagebreak}

\begin{absolutelynopagebreak}
\setstretch{.7}
{\PaliGlossA{ettāvatā kho, upāli, saṃgho bhinno hotī”ti.}}\\
\begin{addmargin}[1em]{2em}
\setstretch{.5}
{\PaliGlossB{That is how schism in the Saṅgha is defined.”}}\\
\end{addmargin}
\end{absolutelynopagebreak}

\begin{absolutelynopagebreak}
\setstretch{.7}
{\PaliGlossA{sattamaṃ.}}\\
\begin{addmargin}[1em]{2em}
\setstretch{.5}
{\PaliGlossB{    -}}\\
\end{addmargin}
\end{absolutelynopagebreak}
