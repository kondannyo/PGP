
\begin{absolutelynopagebreak}
\setstretch{.7}
{\PaliGlossA{aṅguttara nikāya 6}}\\
\begin{addmargin}[1em]{2em}
\setstretch{.5}
{\PaliGlossB{Numbered Discourses 6}}\\
\end{addmargin}
\end{absolutelynopagebreak}

\begin{absolutelynopagebreak}
\setstretch{.7}
{\PaliGlossA{6. mahāvagga}}\\
\begin{addmargin}[1em]{2em}
\setstretch{.5}
{\PaliGlossB{6. The Great Chapter}}\\
\end{addmargin}
\end{absolutelynopagebreak}

\begin{absolutelynopagebreak}
\setstretch{.7}
{\PaliGlossA{56. phaggunasutta}}\\
\begin{addmargin}[1em]{2em}
\setstretch{.5}
{\PaliGlossB{56. With Phagguṇa}}\\
\end{addmargin}
\end{absolutelynopagebreak}

\begin{absolutelynopagebreak}
\setstretch{.7}
{\PaliGlossA{tena kho pana samayena āyasmā phagguno ābādhiko hoti dukkhito bāḷhagilāno.}}\\
\begin{addmargin}[1em]{2em}
\setstretch{.5}
{\PaliGlossB{Now at that time Venerable Phagguṇa was sick, suffering, gravely ill.}}\\
\end{addmargin}
\end{absolutelynopagebreak}

\begin{absolutelynopagebreak}
\setstretch{.7}
{\PaliGlossA{atha kho āyasmā ānando yena bhagavā tenupasaṅkami; upasaṅkamitvā bhagavantaṃ abhivādetvā ekamantaṃ nisīdi. ekamantaṃ nisinno kho āyasmā ānando bhagavantaṃ etadavoca:}}\\
\begin{addmargin}[1em]{2em}
\setstretch{.5}
{\PaliGlossB{Then Venerable Ānanda went up to the Buddha, bowed, sat down to one side, and said to him:}}\\
\end{addmargin}
\end{absolutelynopagebreak}

\begin{absolutelynopagebreak}
\setstretch{.7}
{\PaliGlossA{“āyasmā, bhante, phagguno ābādhiko dukkhito bāḷhagilāno.}}\\
\begin{addmargin}[1em]{2em}
\setstretch{.5}
{\PaliGlossB{“Sir, Venerable Phagguṇa is sick.}}\\
\end{addmargin}
\end{absolutelynopagebreak}

\begin{absolutelynopagebreak}
\setstretch{.7}
{\PaliGlossA{sādhu, bhante, bhagavā yenāyasmā phagguno tenupasaṅkamatu anukampaṃ upādāyā”ti.}}\\
\begin{addmargin}[1em]{2em}
\setstretch{.5}
{\PaliGlossB{Sir, please go to Venerable Phagguṇa out of compassion.”}}\\
\end{addmargin}
\end{absolutelynopagebreak}

\begin{absolutelynopagebreak}
\setstretch{.7}
{\PaliGlossA{adhivāsesi bhagavā tuṇhībhāvena.}}\\
\begin{addmargin}[1em]{2em}
\setstretch{.5}
{\PaliGlossB{The Buddha consented in silence.}}\\
\end{addmargin}
\end{absolutelynopagebreak}

\begin{absolutelynopagebreak}
\setstretch{.7}
{\PaliGlossA{atha kho bhagavā sāyanhasamayaṃ paṭisallānā vuṭṭhito yenāyasmā phagguno tenupasaṅkami.}}\\
\begin{addmargin}[1em]{2em}
\setstretch{.5}
{\PaliGlossB{Then in the late afternoon, the Buddha came out of retreat and went to Venerable Phagguṇa.}}\\
\end{addmargin}
\end{absolutelynopagebreak}

\begin{absolutelynopagebreak}
\setstretch{.7}
{\PaliGlossA{addasā kho āyasmā phagguno bhagavantaṃ dūratova āgacchantaṃ.}}\\
\begin{addmargin}[1em]{2em}
\setstretch{.5}
{\PaliGlossB{Venerable Phagguṇa saw the Buddha coming off in the distance}}\\
\end{addmargin}
\end{absolutelynopagebreak}

\begin{absolutelynopagebreak}
\setstretch{.7}
{\PaliGlossA{disvāna mañcake samadhosi.}}\\
\begin{addmargin}[1em]{2em}
\setstretch{.5}
{\PaliGlossB{and tried to rise on his cot.}}\\
\end{addmargin}
\end{absolutelynopagebreak}

\begin{absolutelynopagebreak}
\setstretch{.7}
{\PaliGlossA{atha kho bhagavā āyasmantaṃ phaggunaṃ etadavoca:}}\\
\begin{addmargin}[1em]{2em}
\setstretch{.5}
{\PaliGlossB{The Buddha said to him,}}\\
\end{addmargin}
\end{absolutelynopagebreak}

\begin{absolutelynopagebreak}
\setstretch{.7}
{\PaliGlossA{“alaṃ, phagguna, mā tvaṃ mañcake samadhosi.}}\\
\begin{addmargin}[1em]{2em}
\setstretch{.5}
{\PaliGlossB{“It’s all right, Phagguṇa, don’t get up.}}\\
\end{addmargin}
\end{absolutelynopagebreak}

\begin{absolutelynopagebreak}
\setstretch{.7}
{\PaliGlossA{santimāni āsanāni parehi paññattāni, tatthāhaṃ nisīdissāmī”ti.}}\\
\begin{addmargin}[1em]{2em}
\setstretch{.5}
{\PaliGlossB{There are some seats spread out by others, I will sit there.”}}\\
\end{addmargin}
\end{absolutelynopagebreak}

\begin{absolutelynopagebreak}
\setstretch{.7}
{\PaliGlossA{nisīdi bhagavā paññatte āsane.}}\\
\begin{addmargin}[1em]{2em}
\setstretch{.5}
{\PaliGlossB{He sat on the seat spread out}}\\
\end{addmargin}
\end{absolutelynopagebreak}

\begin{absolutelynopagebreak}
\setstretch{.7}
{\PaliGlossA{nisajja kho bhagavā āyasmantaṃ phaggunaṃ etadavoca:}}\\
\begin{addmargin}[1em]{2em}
\setstretch{.5}
{\PaliGlossB{and said to Venerable Phagguṇa:}}\\
\end{addmargin}
\end{absolutelynopagebreak}

\begin{absolutelynopagebreak}
\setstretch{.7}
{\PaliGlossA{“kacci te, phagguna, khamanīyaṃ kacci yāpanīyaṃ? kacci te dukkhā vedanā paṭikkamanti, no abhikkamanti; paṭikkamosānaṃ paññāyati, no abhikkamo”ti?}}\\
\begin{addmargin}[1em]{2em}
\setstretch{.5}
{\PaliGlossB{“Phagguṇa, I hope you’re keeping well; I hope you’re alright. And I hope the pain is fading, not growing, that its fading is evident, not its growing.”}}\\
\end{addmargin}
\end{absolutelynopagebreak}

\begin{absolutelynopagebreak}
\setstretch{.7}
{\PaliGlossA{“na me, bhante, khamanīyaṃ na yāpanīyaṃ. bāḷhā me dukkhā vedanā abhikkamanti, no paṭikkamanti; abhikkamosānaṃ paññāyati, no paṭikkamo.}}\\
\begin{addmargin}[1em]{2em}
\setstretch{.5}
{\PaliGlossB{“Sir, I’m not keeping well, I’m not alright. The pain is terrible and growing, not fading; its growing is evident, not its fading.}}\\
\end{addmargin}
\end{absolutelynopagebreak}

\begin{absolutelynopagebreak}
\setstretch{.7}
{\PaliGlossA{seyyathāpi, bhante, balavā puriso tiṇhena sikharena muddhani abhimattheyya; evamevaṃ kho me, bhante, adhimattā vātā muddhani ūhananti.}}\\
\begin{addmargin}[1em]{2em}
\setstretch{.5}
{\PaliGlossB{The winds piercing my head are so severe, it feels like a strong man drilling into my head with a sharp point.}}\\
\end{addmargin}
\end{absolutelynopagebreak}

\begin{absolutelynopagebreak}
\setstretch{.7}
{\PaliGlossA{na me, bhante, khamanīyaṃ na yāpanīyaṃ. bāḷhā me dukkhā vedanā abhikkamanti, no paṭikkamanti; abhikkamosānaṃ paññāyati, no paṭikkamo.}}\\
\begin{addmargin}[1em]{2em}
\setstretch{.5}
{\PaliGlossB{I’m not keeping well.}}\\
\end{addmargin}
\end{absolutelynopagebreak}

\begin{absolutelynopagebreak}
\setstretch{.7}
{\PaliGlossA{seyyathāpi, bhante, balavā puriso daḷhena varattakkhaṇḍena sīsaveṭhanaṃ dadeyya; evamevaṃ kho me, bhante, adhimattā sīse sīsavedanā.}}\\
\begin{addmargin}[1em]{2em}
\setstretch{.5}
{\PaliGlossB{The pain in my head is so severe, it feels like a strong man tightening a tough leather strap around my head.}}\\
\end{addmargin}
\end{absolutelynopagebreak}

\begin{absolutelynopagebreak}
\setstretch{.7}
{\PaliGlossA{na me, bhante, khamanīyaṃ na yāpanīyaṃ. bāḷhā me dukkhā vedanā abhikkamanti, no paṭikkamanti; abhikkamosānaṃ paññāyati, no paṭikkamo.}}\\
\begin{addmargin}[1em]{2em}
\setstretch{.5}
{\PaliGlossB{I’m not keeping well.}}\\
\end{addmargin}
\end{absolutelynopagebreak}

\begin{absolutelynopagebreak}
\setstretch{.7}
{\PaliGlossA{seyyathāpi, bhante, dakkho goghātako vā goghātakantevāsī vā tiṇhena govikantanena kucchiṃ parikanteyya; evamevaṃ kho me, bhante, adhimattā vātā kucchiṃ parikantanti.}}\\
\begin{addmargin}[1em]{2em}
\setstretch{.5}
{\PaliGlossB{The winds piercing my belly are so severe, it feels like a deft butcher or their apprentice is slicing my belly open with a meat cleaver.}}\\
\end{addmargin}
\end{absolutelynopagebreak}

\begin{absolutelynopagebreak}
\setstretch{.7}
{\PaliGlossA{na me, bhante, khamanīyaṃ na yāpanīyaṃ. bāḷhā me dukkhā vedanā abhikkamanti, no paṭikkamanti; abhikkamosānaṃ paññāyati, no paṭikkamo.}}\\
\begin{addmargin}[1em]{2em}
\setstretch{.5}
{\PaliGlossB{I’m not keeping well.}}\\
\end{addmargin}
\end{absolutelynopagebreak}

\begin{absolutelynopagebreak}
\setstretch{.7}
{\PaliGlossA{seyyathāpi, bhante, dve balavanto purisā dubbalataraṃ purisaṃ nānābāhāsu gahetvā aṅgārakāsuyā santāpeyyuṃ samparitāpeyyuṃ; evamevaṃ kho me, bhante, adhimatto kāyasmiṃ ḍāho.}}\\
\begin{addmargin}[1em]{2em}
\setstretch{.5}
{\PaliGlossB{The burning in my body is so severe, it feels like two strong men grabbing a weaker man by the arms to burn and scorch him on a pit of glowing coals.}}\\
\end{addmargin}
\end{absolutelynopagebreak}

\begin{absolutelynopagebreak}
\setstretch{.7}
{\PaliGlossA{na me, bhante, khamanīyaṃ na yāpanīyaṃ. bāḷhā me dukkhā vedanā abhikkamanti, no paṭikkamanti; abhikkamosānaṃ paññāyati, no paṭikkamo”ti.}}\\
\begin{addmargin}[1em]{2em}
\setstretch{.5}
{\PaliGlossB{I’m not keeping well, I’m not alright. The pain is terrible and growing, not fading; its growing is evident, not its fading.”}}\\
\end{addmargin}
\end{absolutelynopagebreak}

\begin{absolutelynopagebreak}
\setstretch{.7}
{\PaliGlossA{atha kho bhagavā āyasmantaṃ phaggunaṃ dhammiyā kathāya sandassetvā samādapetvā samuttejetvā sampahaṃsetvā uṭṭhāyāsanā pakkāmi.}}\\
\begin{addmargin}[1em]{2em}
\setstretch{.5}
{\PaliGlossB{Then the Buddha educated, encouraged, fired up, and inspired Venerable Phagguṇa with a Dhamma talk, after which he got up from his seat and left.}}\\
\end{addmargin}
\end{absolutelynopagebreak}

\begin{absolutelynopagebreak}
\setstretch{.7}
{\PaliGlossA{atha kho āyasmā phagguno acirapakkantassa bhagavato kālamakāsi.}}\\
\begin{addmargin}[1em]{2em}
\setstretch{.5}
{\PaliGlossB{Not long after the Buddha left, Venerable Phagguṇa passed away.}}\\
\end{addmargin}
\end{absolutelynopagebreak}

\begin{absolutelynopagebreak}
\setstretch{.7}
{\PaliGlossA{tamhi cassa samaye maraṇakāle indriyāni vippasīdiṃsu.}}\\
\begin{addmargin}[1em]{2em}
\setstretch{.5}
{\PaliGlossB{At the time of his death, his faculties were bright and clear.}}\\
\end{addmargin}
\end{absolutelynopagebreak}

\begin{absolutelynopagebreak}
\setstretch{.7}
{\PaliGlossA{atha kho āyasmā ānando yena bhagavā tenupasaṅkami; upasaṅkamitvā bhagavantaṃ abhivādetvā ekamantaṃ nisīdi. ekamantaṃ nisinno kho āyasmā ānando bhagavantaṃ etadavoca:}}\\
\begin{addmargin}[1em]{2em}
\setstretch{.5}
{\PaliGlossB{Then Venerable Ānanda went up to the Buddha, bowed, sat down to one side, and said to him,}}\\
\end{addmargin}
\end{absolutelynopagebreak}

\begin{absolutelynopagebreak}
\setstretch{.7}
{\PaliGlossA{“āyasmā, bhante, phagguno acirapakkantassa bhagavato kālamakāsi.}}\\
\begin{addmargin}[1em]{2em}
\setstretch{.5}
{\PaliGlossB{“Sir, soon after the Buddha left, Venerable Phagguṇa died.}}\\
\end{addmargin}
\end{absolutelynopagebreak}

\begin{absolutelynopagebreak}
\setstretch{.7}
{\PaliGlossA{tamhi cassa samaye maraṇakāle indriyāni vippasīdiṃsū”ti.}}\\
\begin{addmargin}[1em]{2em}
\setstretch{.5}
{\PaliGlossB{At the time of his death, his faculties were bright and clear.”}}\\
\end{addmargin}
\end{absolutelynopagebreak}

\begin{absolutelynopagebreak}
\setstretch{.7}
{\PaliGlossA{“kiṃ hānanda, phaggunassa bhikkhuno indriyāni na vippasīdissanti.}}\\
\begin{addmargin}[1em]{2em}
\setstretch{.5}
{\PaliGlossB{“And why shouldn’t his faculties be bright and clear?}}\\
\end{addmargin}
\end{absolutelynopagebreak}

\begin{absolutelynopagebreak}
\setstretch{.7}
{\PaliGlossA{phaggunassa, ānanda, bhikkhuno pañcahi orambhāgiyehi saṃyojanehi cittaṃ avimuttaṃ ahosi.}}\\
\begin{addmargin}[1em]{2em}
\setstretch{.5}
{\PaliGlossB{The mendicant Phagguṇa’s mind was not freed from the five lower fetters.}}\\
\end{addmargin}
\end{absolutelynopagebreak}

\begin{absolutelynopagebreak}
\setstretch{.7}
{\PaliGlossA{tassa taṃ dhammadesanaṃ sutvā pañcahi orambhāgiyehi saṃyojanehi cittaṃ vimuttaṃ.}}\\
\begin{addmargin}[1em]{2em}
\setstretch{.5}
{\PaliGlossB{But when he heard that teaching his mind was freed from them.}}\\
\end{addmargin}
\end{absolutelynopagebreak}

\begin{absolutelynopagebreak}
\setstretch{.7}
{\PaliGlossA{chayime, ānanda, ānisaṃsā kālena dhammassavane kālena atthupaparikkhāya.}}\\
\begin{addmargin}[1em]{2em}
\setstretch{.5}
{\PaliGlossB{Ānanda, there are these six benefits to hearing the teaching at the right time and examining the meaning at the right time.}}\\
\end{addmargin}
\end{absolutelynopagebreak}

\begin{absolutelynopagebreak}
\setstretch{.7}
{\PaliGlossA{katame cha?}}\\
\begin{addmargin}[1em]{2em}
\setstretch{.5}
{\PaliGlossB{What six?}}\\
\end{addmargin}
\end{absolutelynopagebreak}

\begin{absolutelynopagebreak}
\setstretch{.7}
{\PaliGlossA{idhānanda, bhikkhuno pañcahi orambhāgiyehi saṃyojanehi cittaṃ avimuttaṃ hoti.}}\\
\begin{addmargin}[1em]{2em}
\setstretch{.5}
{\PaliGlossB{Firstly, take the case of a mendicant whose mind is not freed from the five lower fetters.}}\\
\end{addmargin}
\end{absolutelynopagebreak}

\begin{absolutelynopagebreak}
\setstretch{.7}
{\PaliGlossA{so tamhi samaye maraṇakāle labhati tathāgataṃ dassanāya.}}\\
\begin{addmargin}[1em]{2em}
\setstretch{.5}
{\PaliGlossB{At the time of death they get to see the Realized One.}}\\
\end{addmargin}
\end{absolutelynopagebreak}

\begin{absolutelynopagebreak}
\setstretch{.7}
{\PaliGlossA{tassa tathāgato dhammaṃ deseti ādikalyāṇaṃ majjhekalyāṇaṃ pariyosānakalyāṇaṃ sātthaṃ sabyañjanaṃ, kevalaparipuṇṇaṃ parisuddhaṃ brahmacariyaṃ pakāseti.}}\\
\begin{addmargin}[1em]{2em}
\setstretch{.5}
{\PaliGlossB{The Realized One teaches them Dhamma that’s good in the beginning, good in the middle, and good in the end, meaningful and well-phrased. And he reveals a spiritual practice that’s entirely full and pure.}}\\
\end{addmargin}
\end{absolutelynopagebreak}

\begin{absolutelynopagebreak}
\setstretch{.7}
{\PaliGlossA{tassa taṃ dhammadesanaṃ sutvā pañcahi orambhāgiyehi saṃyojanehi cittaṃ vimuccati.}}\\
\begin{addmargin}[1em]{2em}
\setstretch{.5}
{\PaliGlossB{When they hear that teaching their mind is freed from the five lower fetters.}}\\
\end{addmargin}
\end{absolutelynopagebreak}

\begin{absolutelynopagebreak}
\setstretch{.7}
{\PaliGlossA{ayaṃ, ānanda, paṭhamo ānisaṃso kālena dhammassavane. (1)}}\\
\begin{addmargin}[1em]{2em}
\setstretch{.5}
{\PaliGlossB{This is the first benefit of listening to the teaching.}}\\
\end{addmargin}
\end{absolutelynopagebreak}

\begin{absolutelynopagebreak}
\setstretch{.7}
{\PaliGlossA{puna caparaṃ, ānanda, bhikkhuno pañcahi orambhāgiyehi saṃyojanehi cittaṃ avimuttaṃ hoti.}}\\
\begin{addmargin}[1em]{2em}
\setstretch{.5}
{\PaliGlossB{Next, take the case of another mendicant whose mind is not freed from the five lower fetters.}}\\
\end{addmargin}
\end{absolutelynopagebreak}

\begin{absolutelynopagebreak}
\setstretch{.7}
{\PaliGlossA{so tamhi samaye maraṇakāle na heva kho labhati tathāgataṃ dassanāya, api ca kho tathāgatasāvakaṃ labhati dassanāya.}}\\
\begin{addmargin}[1em]{2em}
\setstretch{.5}
{\PaliGlossB{At the time of death they don’t get to see the Realized One, but they get to see a Realized One’s disciple.}}\\
\end{addmargin}
\end{absolutelynopagebreak}

\begin{absolutelynopagebreak}
\setstretch{.7}
{\PaliGlossA{tassa tathāgatasāvako dhammaṃ deseti ādikalyāṇaṃ majjhekalyāṇaṃ pariyosānakalyāṇaṃ sātthaṃ sabyañjanaṃ, kevalaparipuṇṇaṃ parisuddhaṃ brahmacariyaṃ pakāseti.}}\\
\begin{addmargin}[1em]{2em}
\setstretch{.5}
{\PaliGlossB{The Realized One’s disciple teaches them Dhamma …}}\\
\end{addmargin}
\end{absolutelynopagebreak}

\begin{absolutelynopagebreak}
\setstretch{.7}
{\PaliGlossA{tassa taṃ dhammadesanaṃ sutvā pañcahi orambhāgiyehi saṃyojanehi cittaṃ vimuccati.}}\\
\begin{addmargin}[1em]{2em}
\setstretch{.5}
{\PaliGlossB{When they hear that teaching their mind is freed from the five lower fetters.}}\\
\end{addmargin}
\end{absolutelynopagebreak}

\begin{absolutelynopagebreak}
\setstretch{.7}
{\PaliGlossA{ayaṃ, ānanda, dutiyo ānisaṃso kālena dhammassavane. (2)}}\\
\begin{addmargin}[1em]{2em}
\setstretch{.5}
{\PaliGlossB{This is the second benefit of listening to the teaching.}}\\
\end{addmargin}
\end{absolutelynopagebreak}

\begin{absolutelynopagebreak}
\setstretch{.7}
{\PaliGlossA{puna caparaṃ, ānanda, bhikkhuno pañcahi orambhāgiyehi saṃyojanehi cittaṃ avimuttaṃ hoti.}}\\
\begin{addmargin}[1em]{2em}
\setstretch{.5}
{\PaliGlossB{Next, take the case of another mendicant whose mind is not freed from the five lower fetters.}}\\
\end{addmargin}
\end{absolutelynopagebreak}

\begin{absolutelynopagebreak}
\setstretch{.7}
{\PaliGlossA{so tamhi samaye maraṇakāle na heva kho labhati tathāgataṃ dassanāya, napi tathāgatasāvakaṃ labhati dassanāya;}}\\
\begin{addmargin}[1em]{2em}
\setstretch{.5}
{\PaliGlossB{At the time of death they don’t get to see the Realized One, or to see a Realized One’s disciple.}}\\
\end{addmargin}
\end{absolutelynopagebreak}

\begin{absolutelynopagebreak}
\setstretch{.7}
{\PaliGlossA{api ca kho yathāsutaṃ yathāpariyattaṃ dhammaṃ cetasā anuvitakketi anuvicāreti manasānupekkhati.}}\\
\begin{addmargin}[1em]{2em}
\setstretch{.5}
{\PaliGlossB{But they think about and consider the teaching in their heart, examining it with the mind as they learned and memorized it.}}\\
\end{addmargin}
\end{absolutelynopagebreak}

\begin{absolutelynopagebreak}
\setstretch{.7}
{\PaliGlossA{tassa yathāsutaṃ yathāpariyattaṃ dhammaṃ cetasā anuvitakkayato anuvicārayato manasānupekkhato pañcahi orambhāgiyehi saṃyojanehi cittaṃ vimuccati.}}\\
\begin{addmargin}[1em]{2em}
\setstretch{.5}
{\PaliGlossB{As they do so their mind is freed from the five lower fetters.}}\\
\end{addmargin}
\end{absolutelynopagebreak}

\begin{absolutelynopagebreak}
\setstretch{.7}
{\PaliGlossA{ayaṃ, ānanda, tatiyo ānisaṃso kālena atthupaparikkhāya. (3)}}\\
\begin{addmargin}[1em]{2em}
\setstretch{.5}
{\PaliGlossB{This is the third benefit of listening to the teaching.}}\\
\end{addmargin}
\end{absolutelynopagebreak}

\begin{absolutelynopagebreak}
\setstretch{.7}
{\PaliGlossA{idhānanda, bhikkhuno pañcahi orambhāgiyehi saṃyojanehi cittaṃ vimuttaṃ hoti, anuttare ca kho upadhisaṅkhaye cittaṃ avimuttaṃ hoti.}}\\
\begin{addmargin}[1em]{2em}
\setstretch{.5}
{\PaliGlossB{Next, take the case of a mendicant whose mind is freed from the five lower fetters, but not with the supreme ending of attachments.}}\\
\end{addmargin}
\end{absolutelynopagebreak}

\begin{absolutelynopagebreak}
\setstretch{.7}
{\PaliGlossA{so tamhi samaye maraṇakāle labhati tathāgataṃ dassanāya.}}\\
\begin{addmargin}[1em]{2em}
\setstretch{.5}
{\PaliGlossB{At the time of death they get to see the Realized One.}}\\
\end{addmargin}
\end{absolutelynopagebreak}

\begin{absolutelynopagebreak}
\setstretch{.7}
{\PaliGlossA{tassa tathāgato dhammaṃ deseti ādikalyāṇaṃ majjhekalyāṇaṃ … pe … brahmacariyaṃ pakāseti.}}\\
\begin{addmargin}[1em]{2em}
\setstretch{.5}
{\PaliGlossB{The Realized One teaches them Dhamma …}}\\
\end{addmargin}
\end{absolutelynopagebreak}

\begin{absolutelynopagebreak}
\setstretch{.7}
{\PaliGlossA{tassa taṃ dhammadesanaṃ sutvā anuttare upadhisaṅkhaye cittaṃ vimuccati.}}\\
\begin{addmargin}[1em]{2em}
\setstretch{.5}
{\PaliGlossB{When they hear that teaching their mind is freed with the supreme ending of attachments.}}\\
\end{addmargin}
\end{absolutelynopagebreak}

\begin{absolutelynopagebreak}
\setstretch{.7}
{\PaliGlossA{ayaṃ, ānanda, catuttho ānisaṃso kālena dhammassavane. (4)}}\\
\begin{addmargin}[1em]{2em}
\setstretch{.5}
{\PaliGlossB{This is the fourth benefit of listening to the teaching.}}\\
\end{addmargin}
\end{absolutelynopagebreak}

\begin{absolutelynopagebreak}
\setstretch{.7}
{\PaliGlossA{puna caparaṃ, ānanda, bhikkhuno pañcahi orambhāgiyehi saṃyojanehi cittaṃ vimuttaṃ hoti, anuttare ca kho upadhisaṅkhaye cittaṃ avimuttaṃ hoti.}}\\
\begin{addmargin}[1em]{2em}
\setstretch{.5}
{\PaliGlossB{Next, take the case of another mendicant whose mind is freed from the five lower fetters, but not with the supreme ending of attachments.}}\\
\end{addmargin}
\end{absolutelynopagebreak}

\begin{absolutelynopagebreak}
\setstretch{.7}
{\PaliGlossA{so tamhi samaye maraṇakāle na heva kho labhati tathāgataṃ dassanāya, api ca kho tathāgatasāvakaṃ labhati dassanāya.}}\\
\begin{addmargin}[1em]{2em}
\setstretch{.5}
{\PaliGlossB{At the time of death they don’t get to see the Realized One, but they get to see a Realized One’s disciple.}}\\
\end{addmargin}
\end{absolutelynopagebreak}

\begin{absolutelynopagebreak}
\setstretch{.7}
{\PaliGlossA{tassa tathāgatasāvako dhammaṃ deseti ādikalyāṇaṃ … pe … parisuddhaṃ brahmacariyaṃ pakāseti.}}\\
\begin{addmargin}[1em]{2em}
\setstretch{.5}
{\PaliGlossB{The Realized One’s disciple teaches them Dhamma …}}\\
\end{addmargin}
\end{absolutelynopagebreak}

\begin{absolutelynopagebreak}
\setstretch{.7}
{\PaliGlossA{tassa taṃ dhammadesanaṃ sutvā anuttare upadhisaṅkhaye cittaṃ vimuccati.}}\\
\begin{addmargin}[1em]{2em}
\setstretch{.5}
{\PaliGlossB{When they hear that teaching their mind is freed with the supreme ending of attachments.}}\\
\end{addmargin}
\end{absolutelynopagebreak}

\begin{absolutelynopagebreak}
\setstretch{.7}
{\PaliGlossA{ayaṃ, ānanda, pañcamo ānisaṃso kālena dhammassavane. (5)}}\\
\begin{addmargin}[1em]{2em}
\setstretch{.5}
{\PaliGlossB{This is the fifth benefit of listening to the teaching.}}\\
\end{addmargin}
\end{absolutelynopagebreak}

\begin{absolutelynopagebreak}
\setstretch{.7}
{\PaliGlossA{puna caparaṃ, ānanda, bhikkhuno pañcahi orambhāgiyehi saṃyojanehi cittaṃ vimuttaṃ hoti, anuttare ca kho upadhisaṅkhaye cittaṃ avimuttaṃ hoti.}}\\
\begin{addmargin}[1em]{2em}
\setstretch{.5}
{\PaliGlossB{Next, take the case of another mendicant whose mind is freed from the five lower fetters, but not with the supreme ending of attachments.}}\\
\end{addmargin}
\end{absolutelynopagebreak}

\begin{absolutelynopagebreak}
\setstretch{.7}
{\PaliGlossA{so tamhi samaye maraṇakāle na heva kho labhati tathāgataṃ dassanāya, napi tathāgatasāvakaṃ labhati dassanāya;}}\\
\begin{addmargin}[1em]{2em}
\setstretch{.5}
{\PaliGlossB{At the time of death they don’t get to see the Realized One, or to see a Realized One’s disciple.}}\\
\end{addmargin}
\end{absolutelynopagebreak}

\begin{absolutelynopagebreak}
\setstretch{.7}
{\PaliGlossA{api ca kho yathāsutaṃ yathāpariyattaṃ dhammaṃ cetasā anuvitakketi anuvicāreti manasānupekkhati.}}\\
\begin{addmargin}[1em]{2em}
\setstretch{.5}
{\PaliGlossB{But they think about and consider the teaching in their heart, examining it with the mind as they learned and memorized it.}}\\
\end{addmargin}
\end{absolutelynopagebreak}

\begin{absolutelynopagebreak}
\setstretch{.7}
{\PaliGlossA{tassa yathāsutaṃ yathāpariyattaṃ dhammaṃ cetasā anuvitakkayato anuvicārayato manasānupekkhato anuttare upadhisaṅkhaye cittaṃ vimuccati.}}\\
\begin{addmargin}[1em]{2em}
\setstretch{.5}
{\PaliGlossB{As they do so their mind is freed with the supreme ending of attachments.}}\\
\end{addmargin}
\end{absolutelynopagebreak}

\begin{absolutelynopagebreak}
\setstretch{.7}
{\PaliGlossA{ayaṃ, ānanda, chaṭṭho ānisaṃso kālena atthupaparikkhāya. (6)}}\\
\begin{addmargin}[1em]{2em}
\setstretch{.5}
{\PaliGlossB{This is the sixth benefit of listening to the teaching.}}\\
\end{addmargin}
\end{absolutelynopagebreak}

\begin{absolutelynopagebreak}
\setstretch{.7}
{\PaliGlossA{ime kho, ānanda, cha ānisaṃsā kālena dhammassavane kālena atthupaparikkhāyā”ti.}}\\
\begin{addmargin}[1em]{2em}
\setstretch{.5}
{\PaliGlossB{These are the six benefits to hearing the teaching at the right time and examining the meaning at the right time.”}}\\
\end{addmargin}
\end{absolutelynopagebreak}

\begin{absolutelynopagebreak}
\setstretch{.7}
{\PaliGlossA{dutiyaṃ.}}\\
\begin{addmargin}[1em]{2em}
\setstretch{.5}
{\PaliGlossB{    -}}\\
\end{addmargin}
\end{absolutelynopagebreak}
