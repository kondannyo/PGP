
\begin{absolutelynopagebreak}
\setstretch{.7}
{\PaliGlossA{aṅguttara nikāya 3}}\\
\begin{addmargin}[1em]{2em}
\setstretch{.5}
{\PaliGlossB{Numbered Discourses 3}}\\
\end{addmargin}
\end{absolutelynopagebreak}

\begin{absolutelynopagebreak}
\setstretch{.7}
{\PaliGlossA{3. puggalavagga}}\\
\begin{addmargin}[1em]{2em}
\setstretch{.5}
{\PaliGlossB{3. Persons}}\\
\end{addmargin}
\end{absolutelynopagebreak}

\begin{absolutelynopagebreak}
\setstretch{.7}
{\PaliGlossA{25. vajirūpamasutta}}\\
\begin{addmargin}[1em]{2em}
\setstretch{.5}
{\PaliGlossB{25. Like Diamond}}\\
\end{addmargin}
\end{absolutelynopagebreak}

\begin{absolutelynopagebreak}
\setstretch{.7}
{\PaliGlossA{“tayome, bhikkhave, puggalā santo saṃvijjamānā lokasmiṃ.}}\\
\begin{addmargin}[1em]{2em}
\setstretch{.5}
{\PaliGlossB{“These three people are found in the world.}}\\
\end{addmargin}
\end{absolutelynopagebreak}

\begin{absolutelynopagebreak}
\setstretch{.7}
{\PaliGlossA{katame tayo?}}\\
\begin{addmargin}[1em]{2em}
\setstretch{.5}
{\PaliGlossB{What three?}}\\
\end{addmargin}
\end{absolutelynopagebreak}

\begin{absolutelynopagebreak}
\setstretch{.7}
{\PaliGlossA{arukūpamacitto puggalo, vijjūpamacitto puggalo, vajirūpamacitto puggalo.}}\\
\begin{addmargin}[1em]{2em}
\setstretch{.5}
{\PaliGlossB{A person with a mind like an open sore, a person with a mind like lightning, and a person with a mind like diamond.}}\\
\end{addmargin}
\end{absolutelynopagebreak}

\begin{absolutelynopagebreak}
\setstretch{.7}
{\PaliGlossA{katamo ca, bhikkhave, arukūpamacitto puggalo?}}\\
\begin{addmargin}[1em]{2em}
\setstretch{.5}
{\PaliGlossB{And who has a mind like an open sore?}}\\
\end{addmargin}
\end{absolutelynopagebreak}

\begin{absolutelynopagebreak}
\setstretch{.7}
{\PaliGlossA{idha, bhikkhave, ekacco puggalo kodhano hoti upāyāsabahulo}}\\
\begin{addmargin}[1em]{2em}
\setstretch{.5}
{\PaliGlossB{It’s someone who is irritable and bad-tempered.}}\\
\end{addmargin}
\end{absolutelynopagebreak}

\begin{absolutelynopagebreak}
\setstretch{.7}
{\PaliGlossA{appampi vutto samāno abhisajjati kuppati byāpajjati patitthīyati kopañca dosañca appaccayañca pātukaroti.}}\\
\begin{addmargin}[1em]{2em}
\setstretch{.5}
{\PaliGlossB{Even when lightly criticized they lose their temper, becoming annoyed, hostile, and hard-hearted, and they display annoyance, hate, and bitterness.}}\\
\end{addmargin}
\end{absolutelynopagebreak}

\begin{absolutelynopagebreak}
\setstretch{.7}
{\PaliGlossA{seyyathāpi, bhikkhave, duṭṭhāruko kaṭṭhena vā kaṭhalāya vā ghaṭṭito bhiyyoso mattāya āsavaṃ deti;}}\\
\begin{addmargin}[1em]{2em}
\setstretch{.5}
{\PaliGlossB{They’re like a festering sore, which, when you hit it with a stick or a stone, discharges even more.}}\\
\end{addmargin}
\end{absolutelynopagebreak}

\begin{absolutelynopagebreak}
\setstretch{.7}
{\PaliGlossA{evamevaṃ kho, bhikkhave, idhekacco puggalo kodhano hoti upāyāsabahulo}}\\
\begin{addmargin}[1em]{2em}
\setstretch{.5}
{\PaliGlossB{In the same way, someone is irritable and bad-tempered.}}\\
\end{addmargin}
\end{absolutelynopagebreak}

\begin{absolutelynopagebreak}
\setstretch{.7}
{\PaliGlossA{appampi vutto samāno abhisajjati kuppati byāpajjati patitthīyati kopañca dosañca appaccayañca pātukaroti.}}\\
\begin{addmargin}[1em]{2em}
\setstretch{.5}
{\PaliGlossB{Even when lightly criticized they lose their temper, becoming annoyed, hostile, and hard-hearted, and they display annoyance, hate, and bitterness.}}\\
\end{addmargin}
\end{absolutelynopagebreak}

\begin{absolutelynopagebreak}
\setstretch{.7}
{\PaliGlossA{ayaṃ vuccati, bhikkhave, arukūpamacitto puggalo.}}\\
\begin{addmargin}[1em]{2em}
\setstretch{.5}
{\PaliGlossB{This is called a person with a mind like an open sore.}}\\
\end{addmargin}
\end{absolutelynopagebreak}

\begin{absolutelynopagebreak}
\setstretch{.7}
{\PaliGlossA{katamo ca, bhikkhave, vijjūpamacitto puggalo?}}\\
\begin{addmargin}[1em]{2em}
\setstretch{.5}
{\PaliGlossB{And who has a mind like lightning?}}\\
\end{addmargin}
\end{absolutelynopagebreak}

\begin{absolutelynopagebreak}
\setstretch{.7}
{\PaliGlossA{idha, bhikkhave, ekacco puggalo ‘idaṃ dukkhan’ti yathābhūtaṃ pajānāti, ‘ayaṃ dukkhasamudayo’ti yathābhūtaṃ pajānāti, ‘ayaṃ dukkhanirodho’ti yathābhūtaṃ pajānāti, ‘ayaṃ dukkhanirodhagāminī paṭipadā’ti yathābhūtaṃ pajānāti.}}\\
\begin{addmargin}[1em]{2em}
\setstretch{.5}
{\PaliGlossB{It’s someone who truly understands: ‘This is suffering’ … ‘This is the origin of suffering’ … ‘This is the cessation of suffering’ … ‘This is the practice that leads to the cessation of suffering’.}}\\
\end{addmargin}
\end{absolutelynopagebreak}

\begin{absolutelynopagebreak}
\setstretch{.7}
{\PaliGlossA{seyyathāpi, bhikkhave, cakkhumā puriso rattandhakāratimisāyaṃ vijjantarikāya rūpāni passeyya;}}\\
\begin{addmargin}[1em]{2em}
\setstretch{.5}
{\PaliGlossB{They’re like a person with keen eyes in the dark of the night, who sees by a flash of lightning.}}\\
\end{addmargin}
\end{absolutelynopagebreak}

\begin{absolutelynopagebreak}
\setstretch{.7}
{\PaliGlossA{evamevaṃ kho, bhikkhave, idhekacco puggalo ‘idaṃ dukkhan’ti yathābhūtaṃ pajānāti … pe … ‘ayaṃ dukkhanirodhagāminī paṭipadā’ti yathābhūtaṃ pajānāti.}}\\
\begin{addmargin}[1em]{2em}
\setstretch{.5}
{\PaliGlossB{In the same way, someone truly understands: ‘This is suffering’ … ‘This is the origin of suffering’ … ‘This is the cessation of suffering’ … ‘This is the practice that leads to the cessation of suffering’.}}\\
\end{addmargin}
\end{absolutelynopagebreak}

\begin{absolutelynopagebreak}
\setstretch{.7}
{\PaliGlossA{ayaṃ vuccati, bhikkhave, vijjūpamacitto puggalo.}}\\
\begin{addmargin}[1em]{2em}
\setstretch{.5}
{\PaliGlossB{This is called a person with a mind like lightning.}}\\
\end{addmargin}
\end{absolutelynopagebreak}

\begin{absolutelynopagebreak}
\setstretch{.7}
{\PaliGlossA{katamo ca, bhikkhave, vajirūpamacitto puggalo?}}\\
\begin{addmargin}[1em]{2em}
\setstretch{.5}
{\PaliGlossB{And who has a mind like diamond?}}\\
\end{addmargin}
\end{absolutelynopagebreak}

\begin{absolutelynopagebreak}
\setstretch{.7}
{\PaliGlossA{idha, bhikkhave, ekacco puggalo āsavānaṃ khayā anāsavaṃ cetovimuttiṃ paññāvimuttiṃ diṭṭheva dhamme sayaṃ abhiññā sacchikatvā upasampajja viharati.}}\\
\begin{addmargin}[1em]{2em}
\setstretch{.5}
{\PaliGlossB{It’s someone who realizes the undefiled freedom of heart and freedom by wisdom in this very life. And they live having realized it with their own insight due to the ending of defilements.}}\\
\end{addmargin}
\end{absolutelynopagebreak}

\begin{absolutelynopagebreak}
\setstretch{.7}
{\PaliGlossA{seyyathāpi, bhikkhave, vajirassa natthi kiñci abhejjaṃ maṇi vā pāsāṇo vā;}}\\
\begin{addmargin}[1em]{2em}
\setstretch{.5}
{\PaliGlossB{It’s like a diamond, which can’t be cut by anything at all, not even a gem or a stone.}}\\
\end{addmargin}
\end{absolutelynopagebreak}

\begin{absolutelynopagebreak}
\setstretch{.7}
{\PaliGlossA{evamevaṃ kho, bhikkhave, idhekacco puggalo āsavānaṃ khayā … pe … upasampajja viharati.}}\\
\begin{addmargin}[1em]{2em}
\setstretch{.5}
{\PaliGlossB{In the same way, someone realizes the undefiled freedom of heart and freedom by wisdom in this very life. …}}\\
\end{addmargin}
\end{absolutelynopagebreak}

\begin{absolutelynopagebreak}
\setstretch{.7}
{\PaliGlossA{ayaṃ vuccati, bhikkhave, vajirūpamacitto puggalo.}}\\
\begin{addmargin}[1em]{2em}
\setstretch{.5}
{\PaliGlossB{This is called a person with a mind like diamond.}}\\
\end{addmargin}
\end{absolutelynopagebreak}

\begin{absolutelynopagebreak}
\setstretch{.7}
{\PaliGlossA{ime kho, bhikkhave, tayo puggalā santo saṃvijjamānā lokasmin”ti.}}\\
\begin{addmargin}[1em]{2em}
\setstretch{.5}
{\PaliGlossB{These are the three people found in the world.”}}\\
\end{addmargin}
\end{absolutelynopagebreak}

\begin{absolutelynopagebreak}
\setstretch{.7}
{\PaliGlossA{pañcamaṃ.}}\\
\begin{addmargin}[1em]{2em}
\setstretch{.5}
{\PaliGlossB{    -}}\\
\end{addmargin}
\end{absolutelynopagebreak}
