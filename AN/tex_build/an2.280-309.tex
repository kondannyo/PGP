
\begin{absolutelynopagebreak}
\setstretch{.7}
{\PaliGlossA{aṅguttara nikāya 2}}\\
\begin{addmargin}[1em]{2em}
\setstretch{.5}
{\PaliGlossB{Numbered Discourses 2}}\\
\end{addmargin}
\end{absolutelynopagebreak}

\begin{absolutelynopagebreak}
\setstretch{.7}
{\PaliGlossA{18. vinayapeyyāla}}\\
\begin{addmargin}[1em]{2em}
\setstretch{.5}
{\PaliGlossB{18. Abbreviated Texts Beginning with the Training}}\\
\end{addmargin}
\end{absolutelynopagebreak}

\begin{absolutelynopagebreak}
\setstretch{.7}
{\PaliGlossA{280}}\\
\begin{addmargin}[1em]{2em}
\setstretch{.5}
{\PaliGlossB{280}}\\
\end{addmargin}
\end{absolutelynopagebreak}

\begin{absolutelynopagebreak}
\setstretch{.7}
{\PaliGlossA{“dveme, bhikkhave, atthavase paṭicca tathāgatena sāvakānaṃ sikkhāpadaṃ paññattaṃ.}}\\
\begin{addmargin}[1em]{2em}
\setstretch{.5}
{\PaliGlossB{“For two reasons the Realized One laid down training rules for his disciples.}}\\
\end{addmargin}
\end{absolutelynopagebreak}

\begin{absolutelynopagebreak}
\setstretch{.7}
{\PaliGlossA{katame dve?}}\\
\begin{addmargin}[1em]{2em}
\setstretch{.5}
{\PaliGlossB{What two?}}\\
\end{addmargin}
\end{absolutelynopagebreak}

\begin{absolutelynopagebreak}
\setstretch{.7}
{\PaliGlossA{saṃghasuṭṭhutāya saṃghaphāsutāya … dummaṅkūnaṃ puggalānaṃ niggahāya, pesalānaṃ bhikkhūnaṃ phāsuvihārāya … diṭṭhadhammikānaṃ āsavānaṃ saṃvarāya, samparāyikānaṃ āsavānaṃ paṭighātāya … diṭṭhadhammikānaṃ verānaṃ saṃvarāya, samparāyikānaṃ verānaṃ paṭighātāya … diṭṭhadhammikānaṃ vajjānaṃ saṃvarāya, samparāyikānaṃ vajjānaṃ paṭighātāya … diṭṭhadhammikānaṃ bhayānaṃ saṃvarāya, samparāyikānaṃ bhayānaṃ paṭighātāya … diṭṭhadhammikānaṃ akusalānaṃ dhammānaṃ saṃvarāya, samparāyikānaṃ akusalānaṃ dhammānaṃ paṭighātāya … gihīnaṃ anukampāya, pāpicchānaṃ bhikkhūnaṃ pakkhupacchedāya … appasannānaṃ pasādāya, pasannānaṃ bhiyyobhāvāya … saddhammaṭṭhitiyā vinayānuggahāya.}}\\
\begin{addmargin}[1em]{2em}
\setstretch{.5}
{\PaliGlossB{For the well-being and comfort of the Saṅgha … For keeping difficult persons in check and for the comfort of good-hearted mendicants … For restraining defilements that affect the present life and protecting against defilements that affect lives to come … For restraining threats to the present life and protecting against threats to lives to come … For restraining faults that affect the present life and protecting against faults that affect lives to come … For restraining hazards that affect the present life and protecting against hazards that affect lives to come … For restraining unskillful qualities that affect the present life and protecting against unskillful qualities that affect lives to come … Out of sympathy for laypeople and for breaking up factions of mendicants with wicked desires … For inspiring confidence in those without it, and increasing confidence in those who have it … For the continuation of the true teaching and the support of the training.}}\\
\end{addmargin}
\end{absolutelynopagebreak}

\begin{absolutelynopagebreak}
\setstretch{.7}
{\PaliGlossA{ime kho, bhikkhave, dve atthavase paṭicca tathāgatena sāvakānaṃ sikkhāpadaṃ paññattan”ti.}}\\
\begin{addmargin}[1em]{2em}
\setstretch{.5}
{\PaliGlossB{These are the two reasons why the Realized One laid down training rules for his disciples.”}}\\
\end{addmargin}
\end{absolutelynopagebreak}

\begin{absolutelynopagebreak}
\setstretch{.7}
{\PaliGlossA{281–309}}\\
\begin{addmargin}[1em]{2em}
\setstretch{.5}
{\PaliGlossB{281–309}}\\
\end{addmargin}
\end{absolutelynopagebreak}

\begin{absolutelynopagebreak}
\setstretch{.7}
{\PaliGlossA{“dveme, bhikkhave, atthavase paṭicca tathāgatena sāvakānaṃ pātimokkhaṃ paññattaṃ … pe … pātimokkhuddeso paññatto … pātimokkhaṭṭhapanaṃ paññattaṃ … pavāraṇā paññattā … pavāraṇaṭṭhapanaṃ paññattaṃ … tajjanīyakammaṃ paññattaṃ … niyassakammaṃ paññattaṃ … pabbājanīyakammaṃ paññattaṃ … paṭisāraṇīyakammaṃ paññattaṃ … ukkhepanīyakammaṃ paññattaṃ … parivāsadānaṃ paññattaṃ … mūlāyapaṭikassanaṃ paññattaṃ … mānattadānaṃ paññattaṃ … abbhānaṃ paññattaṃ … osāraṇīyaṃ paññattaṃ … nissāraṇīyaṃ paññattaṃ … upasampadā paññattā … ñattikammaṃ paññattaṃ … ñattidutiyakammaṃ paññattaṃ … ñatticatutthakammaṃ paññattaṃ … apaññatte paññattaṃ … paññatte anupaññattaṃ … sammukhāvinayo paññatto … sativinayo paññatto … amūḷhavinayo paññatto … paṭiññātakaraṇaṃ paññattaṃ … yebhuyyasikā paññattā … tassapāpiyasikā paññattā … tiṇavatthārako paññatto.}}\\
\begin{addmargin}[1em]{2em}
\setstretch{.5}
{\PaliGlossB{“For two reasons the Realized One laid down for his disciples the monastic code … the recitation of the monastic code … the suspension of the recitation of the monastic code … the invitation to admonish … the setting aside of the invitation to admonish … the disciplinary act of censure … placing under dependence … banishment … reconciliation … debarment … probation … being sent back to the beginning … penance … reinstatement … restoration … removal … ordination … an act with a motion … an act with a motion and one announcement … an act with a motion and three announcements … laying down what was not previously laid down … amending what was laid down … the settling of a disciplinary matter in the presence of those concerned … the settling of a disciplinary matter by accurate recollection … the settling of a disciplinary matter due to recovery from madness … the settling of a disciplinary matter due to the acknowledgement of the offense … the settling of a disciplinary matter by the decision of a majority … the settling of a disciplinary matter by a verdict of aggravated misconduct … the settling of a disciplinary matter by covering over with grass.}}\\
\end{addmargin}
\end{absolutelynopagebreak}

\begin{absolutelynopagebreak}
\setstretch{.7}
{\PaliGlossA{katame dve?}}\\
\begin{addmargin}[1em]{2em}
\setstretch{.5}
{\PaliGlossB{What two?}}\\
\end{addmargin}
\end{absolutelynopagebreak}

\begin{absolutelynopagebreak}
\setstretch{.7}
{\PaliGlossA{saṅghasuṭṭhutāya, saṅghaphāsutāya … dummaṅkūnaṃ puggalānaṃ niggahāya, pesalānaṃ bhikkhūnaṃ phāsuvihārāya … diṭṭhadhammikānaṃ āsavānaṃ saṃvarāya, samparāyikānaṃ āsavānaṃ paṭighātāya … diṭṭhadhammikānaṃ verānaṃ saṃvarāya, samparāyikānaṃ verānaṃ paṭighātāya … diṭṭhadhammikānaṃ vajjānaṃ saṃvarāya, samparāyikānaṃ vajjānaṃ paṭighātāya … diṭṭhadhammikānaṃ bhayānaṃ saṃvarāya, samparāyikānaṃ bhayānaṃ paṭighātāya … diṭṭhadhammikānaṃ akusalānaṃ dhammānaṃ saṃvarāya, samparāyikānaṃ akusalānaṃ dhammānaṃ paṭighātāya … gihīnaṃ anukampāya, pāpicchānaṃ bhikkhūnaṃ pakkhupacchedāya … appasannānaṃ pasādāya, pasannānaṃ bhiyyobhāvāya … saddhammaṭṭhitiyā, vinayānuggahāya.}}\\
\begin{addmargin}[1em]{2em}
\setstretch{.5}
{\PaliGlossB{For the well-being and comfort of the Saṅgha … For keeping difficult persons in check and for the comfort of good-hearted mendicants … For restraining defilements that affect the present life and protecting against defilements that affect lives to come … For restraining threats to the present life and protecting against threats to lives to come … For restraining faults that affect the present life and protecting against faults that affect lives to come … For restraining hazards that affect the present life and protecting against hazards that affect lives to come … For restraining unskillful qualities that affect the present life and protecting against unskillful qualities that affect lives to come … Out of sympathy for laypeople and for breaking up factions of mendicants with wicked desires … For inspiring confidence in those without it, and increasing confidence in those who have it … For the continuation of the true teaching and the support of the training.}}\\
\end{addmargin}
\end{absolutelynopagebreak}

\begin{absolutelynopagebreak}
\setstretch{.7}
{\PaliGlossA{ime kho, bhikkhave, dve atthavase paṭicca tathāgatena sāvakānaṃ tiṇavatthārako paññatto”ti. (11–300.)}}\\
\begin{addmargin}[1em]{2em}
\setstretch{.5}
{\PaliGlossB{These are the two reasons why the Realized One laid down the settlement of a disciplinary matter by covering over with grass for his disciples.”}}\\
\end{addmargin}
\end{absolutelynopagebreak}

\begin{absolutelynopagebreak}
\setstretch{.7}
{\PaliGlossA{vinayapeyyālaṃ niṭṭhitaṃ.}}\\
\begin{addmargin}[1em]{2em}
\setstretch{.5}
{\PaliGlossB{    -}}\\
\end{addmargin}
\end{absolutelynopagebreak}
