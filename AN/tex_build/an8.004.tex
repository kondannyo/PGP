
\begin{absolutelynopagebreak}
\setstretch{.7}
{\PaliGlossA{aṅguttara nikāya 8}}\\
\begin{addmargin}[1em]{2em}
\setstretch{.5}
{\PaliGlossB{Numbered Discourses 8}}\\
\end{addmargin}
\end{absolutelynopagebreak}

\begin{absolutelynopagebreak}
\setstretch{.7}
{\PaliGlossA{1. mettāvagga}}\\
\begin{addmargin}[1em]{2em}
\setstretch{.5}
{\PaliGlossB{1. Love}}\\
\end{addmargin}
\end{absolutelynopagebreak}

\begin{absolutelynopagebreak}
\setstretch{.7}
{\PaliGlossA{4. dutiyaappiyasutta}}\\
\begin{addmargin}[1em]{2em}
\setstretch{.5}
{\PaliGlossB{4. Disliked (2nd)}}\\
\end{addmargin}
\end{absolutelynopagebreak}

\begin{absolutelynopagebreak}
\setstretch{.7}
{\PaliGlossA{“aṭṭhahi, bhikkhave, dhammehi samannāgato bhikkhu sabrahmacārīnaṃ appiyo ca hoti amanāpo ca agaru ca abhāvanīyo ca.}}\\
\begin{addmargin}[1em]{2em}
\setstretch{.5}
{\PaliGlossB{“Mendicants, a mendicant with eight qualities is disliked and disapproved by their spiritual companions, not respected or admired.}}\\
\end{addmargin}
\end{absolutelynopagebreak}

\begin{absolutelynopagebreak}
\setstretch{.7}
{\PaliGlossA{katamehi aṭṭhahi?}}\\
\begin{addmargin}[1em]{2em}
\setstretch{.5}
{\PaliGlossB{What eight?}}\\
\end{addmargin}
\end{absolutelynopagebreak}

\begin{absolutelynopagebreak}
\setstretch{.7}
{\PaliGlossA{idha, bhikkhave, bhikkhu lābhakāmo ca hoti, sakkārakāmo ca, anavaññattikāmo ca, akālaññū ca, amattaññū ca, asuci ca, bahubhāṇī ca, akkosakaparibhāsako ca sabrahmacārīnaṃ.}}\\
\begin{addmargin}[1em]{2em}
\setstretch{.5}
{\PaliGlossB{It’s when a mendicant desires material possessions, honor, and to be looked up to. They know neither moderation nor the proper time. Their conduct is impure, they talk a lot, and they insult and abuse their spiritual companions.}}\\
\end{addmargin}
\end{absolutelynopagebreak}

\begin{absolutelynopagebreak}
\setstretch{.7}
{\PaliGlossA{imehi kho, bhikkhave, aṭṭhahi dhammehi samannāgato bhikkhu sabrahmacārīnaṃ appiyo ca hoti amanāpo ca agaru ca abhāvanīyo ca.}}\\
\begin{addmargin}[1em]{2em}
\setstretch{.5}
{\PaliGlossB{A mendicant with these eight qualities is disliked and disapproved by their spiritual companions, not respected or admired.}}\\
\end{addmargin}
\end{absolutelynopagebreak}

\begin{absolutelynopagebreak}
\setstretch{.7}
{\PaliGlossA{aṭṭhahi, bhikkhave, dhammehi samannāgato bhikkhu sabrahmacārīnaṃ piyo ca hoti manāpo ca garu ca bhāvanīyo ca.}}\\
\begin{addmargin}[1em]{2em}
\setstretch{.5}
{\PaliGlossB{A mendicant with eight qualities is liked and approved by their spiritual companions, and respected and admired.}}\\
\end{addmargin}
\end{absolutelynopagebreak}

\begin{absolutelynopagebreak}
\setstretch{.7}
{\PaliGlossA{katamehi aṭṭhahi?}}\\
\begin{addmargin}[1em]{2em}
\setstretch{.5}
{\PaliGlossB{What eight?}}\\
\end{addmargin}
\end{absolutelynopagebreak}

\begin{absolutelynopagebreak}
\setstretch{.7}
{\PaliGlossA{idha, bhikkhave, bhikkhu na lābhakāmo ca hoti, na sakkārakāmo ca, na anavaññattikāmo ca, kālaññū ca, mattaññū ca, suci ca, na bahubhāṇī ca, anakkosakaparibhāsako ca sabrahmacārīnaṃ.}}\\
\begin{addmargin}[1em]{2em}
\setstretch{.5}
{\PaliGlossB{It’s when a mendicant doesn’t desire material possessions, honor, and to be looked up to. They know moderation and the proper time. Their conduct is pure, they don’t talk a lot, and they don’t insult and abuse their spiritual companions.}}\\
\end{addmargin}
\end{absolutelynopagebreak}

\begin{absolutelynopagebreak}
\setstretch{.7}
{\PaliGlossA{imehi kho, bhikkhave, aṭṭhahi dhammehi samannāgato bhikkhu sabrahmacārīnaṃ piyo ca hoti manāpo ca garu ca bhāvanīyo cā”ti.}}\\
\begin{addmargin}[1em]{2em}
\setstretch{.5}
{\PaliGlossB{A mendicant with these eight qualities is liked and approved by their spiritual companions, and respected and admired.”}}\\
\end{addmargin}
\end{absolutelynopagebreak}

\begin{absolutelynopagebreak}
\setstretch{.7}
{\PaliGlossA{catutthaṃ.}}\\
\begin{addmargin}[1em]{2em}
\setstretch{.5}
{\PaliGlossB{    -}}\\
\end{addmargin}
\end{absolutelynopagebreak}
