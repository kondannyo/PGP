
\begin{absolutelynopagebreak}
\setstretch{.7}
{\PaliGlossA{aṅguttara nikāya 4}}\\
\begin{addmargin}[1em]{2em}
\setstretch{.5}
{\PaliGlossB{Numbered Discourses 4}}\\
\end{addmargin}
\end{absolutelynopagebreak}

\begin{absolutelynopagebreak}
\setstretch{.7}
{\PaliGlossA{3. uruvelavagga}}\\
\begin{addmargin}[1em]{2em}
\setstretch{.5}
{\PaliGlossB{3. At Uruvelā}}\\
\end{addmargin}
\end{absolutelynopagebreak}

\begin{absolutelynopagebreak}
\setstretch{.7}
{\PaliGlossA{28. ariyavaṃsasutta}}\\
\begin{addmargin}[1em]{2em}
\setstretch{.5}
{\PaliGlossB{28. The Noble Traditions}}\\
\end{addmargin}
\end{absolutelynopagebreak}

\begin{absolutelynopagebreak}
\setstretch{.7}
{\PaliGlossA{“cattārome, bhikkhave, ariyavaṃsā aggaññā rattaññā vaṃsaññā porāṇā asaṃkiṇṇā asaṃkiṇṇapubbā, na saṅkīyanti na saṅkīyissanti, appaṭikuṭṭhā samaṇehi brāhmaṇehi viññūhi.}}\\
\begin{addmargin}[1em]{2em}
\setstretch{.5}
{\PaliGlossB{“Mendicants, these four noble traditions are original, long-standing, traditional, and ancient. They are uncorrupted, as they have been since the beginning. They’re not being corrupted now, nor will they be. Sensible ascetics and brahmins don’t look down on them.}}\\
\end{addmargin}
\end{absolutelynopagebreak}

\begin{absolutelynopagebreak}
\setstretch{.7}
{\PaliGlossA{katame cattāro?}}\\
\begin{addmargin}[1em]{2em}
\setstretch{.5}
{\PaliGlossB{What four?}}\\
\end{addmargin}
\end{absolutelynopagebreak}

\begin{absolutelynopagebreak}
\setstretch{.7}
{\PaliGlossA{idha, bhikkhave, bhikkhu santuṭṭho hoti itarītarena cīvarena, itarītaracīvarasantuṭṭhiyā ca vaṇṇavādī, na ca cīvarahetu anesanaṃ appatirūpaṃ āpajjati, aladdhā ca cīvaraṃ na paritassati, laddhā ca cīvaraṃ agadhito amucchito anajjhosanno ādīnavadassāvī nissaraṇapañño paribhuñjati;}}\\
\begin{addmargin}[1em]{2em}
\setstretch{.5}
{\PaliGlossB{Firstly, a mendicant is content with any kind of robe, and praises such contentment. They don’t try to get hold of a robe in an improper way. They don’t get upset if they don’t get a robe. And if they do get a robe, they use it untied, uninfatuated, unattached, seeing the drawback, and understanding the escape.}}\\
\end{addmargin}
\end{absolutelynopagebreak}

\begin{absolutelynopagebreak}
\setstretch{.7}
{\PaliGlossA{tāya ca pana itarītaracīvarasantuṭṭhiyā nevattānukkaṃseti, no paraṃ vambheti.}}\\
\begin{addmargin}[1em]{2em}
\setstretch{.5}
{\PaliGlossB{But they don’t glorify themselves or put others down on account of their contentment.}}\\
\end{addmargin}
\end{absolutelynopagebreak}

\begin{absolutelynopagebreak}
\setstretch{.7}
{\PaliGlossA{yo hi tattha dakkho analaso sampajāno patissato, ayaṃ vuccati, bhikkhave, bhikkhu porāṇe aggaññe ariyavaṃse ṭhito.}}\\
\begin{addmargin}[1em]{2em}
\setstretch{.5}
{\PaliGlossB{A mendicant who is deft, tireless, aware, and mindful in this is said to stand in the ancient, original noble tradition.}}\\
\end{addmargin}
\end{absolutelynopagebreak}

\begin{absolutelynopagebreak}
\setstretch{.7}
{\PaliGlossA{puna caparaṃ, bhikkhave, bhikkhu santuṭṭho hoti itarītarena piṇḍapātena, itarītarapiṇḍapātasantuṭṭhiyā ca vaṇṇavādī, na ca piṇḍapātahetu anesanaṃ appatirūpaṃ āpajjati, aladdhā ca piṇḍapātaṃ na paritassati, laddhā ca piṇḍapātaṃ agadhito amucchito anajjhosanno ādīnavadassāvī nissaraṇapañño paribhuñjati;}}\\
\begin{addmargin}[1em]{2em}
\setstretch{.5}
{\PaliGlossB{Furthermore, a mendicant is content with any kind of alms-food …}}\\
\end{addmargin}
\end{absolutelynopagebreak}

\begin{absolutelynopagebreak}
\setstretch{.7}
{\PaliGlossA{tāya ca pana itarītarapiṇḍapātasantuṭṭhiyā nevattānukkaṃseti, no paraṃ vambheti.}}\\
\begin{addmargin}[1em]{2em}
\setstretch{.5}
{\PaliGlossB{    -}}\\
\end{addmargin}
\end{absolutelynopagebreak}

\begin{absolutelynopagebreak}
\setstretch{.7}
{\PaliGlossA{yo hi tattha dakkho analaso sampajāno patissato, ayaṃ vuccati, bhikkhave, bhikkhu porāṇe aggaññe ariyavaṃse ṭhito.}}\\
\begin{addmargin}[1em]{2em}
\setstretch{.5}
{\PaliGlossB{    -}}\\
\end{addmargin}
\end{absolutelynopagebreak}

\begin{absolutelynopagebreak}
\setstretch{.7}
{\PaliGlossA{puna caparaṃ, bhikkhave, bhikkhu santuṭṭho hoti itarītarena senāsanena, itarītarasenāsanasantuṭṭhiyā ca vaṇṇavādī, na ca senāsanahetu anesanaṃ appatirūpaṃ āpajjati, aladdhā ca senāsanaṃ na paritassati, laddhā ca senāsanaṃ agadhito amucchito anajjhosanno ādīnavadassāvī nissaraṇapañño paribhuñjati;}}\\
\begin{addmargin}[1em]{2em}
\setstretch{.5}
{\PaliGlossB{Furthermore, a mendicant is content with any kind of lodgings …}}\\
\end{addmargin}
\end{absolutelynopagebreak}

\begin{absolutelynopagebreak}
\setstretch{.7}
{\PaliGlossA{tāya ca pana itarītarasenāsanasantuṭṭhiyā nevattānukkaṃseti, no paraṃ vambheti.}}\\
\begin{addmargin}[1em]{2em}
\setstretch{.5}
{\PaliGlossB{    -}}\\
\end{addmargin}
\end{absolutelynopagebreak}

\begin{absolutelynopagebreak}
\setstretch{.7}
{\PaliGlossA{yo hi tattha dakkho analaso sampajāno patissato, ayaṃ vuccati, bhikkhave, bhikkhu porāṇe aggaññe ariyavaṃse ṭhito.}}\\
\begin{addmargin}[1em]{2em}
\setstretch{.5}
{\PaliGlossB{    -}}\\
\end{addmargin}
\end{absolutelynopagebreak}

\begin{absolutelynopagebreak}
\setstretch{.7}
{\PaliGlossA{puna caparaṃ, bhikkhave, bhikkhu bhāvanārāmo hoti bhāvanārato, pahānārāmo hoti pahānarato;}}\\
\begin{addmargin}[1em]{2em}
\setstretch{.5}
{\PaliGlossB{Furthermore, a mendicant enjoys meditation and loves to meditate. They enjoy giving up and love to give up.}}\\
\end{addmargin}
\end{absolutelynopagebreak}

\begin{absolutelynopagebreak}
\setstretch{.7}
{\PaliGlossA{tāya ca pana bhāvanārāmatāya bhāvanāratiyā pahānārāmatāya pahānaratiyā nevattānukkaṃseti, no paraṃ vambheti.}}\\
\begin{addmargin}[1em]{2em}
\setstretch{.5}
{\PaliGlossB{But they don’t glorify themselves or put down others on account of their love for meditation and giving up.}}\\
\end{addmargin}
\end{absolutelynopagebreak}

\begin{absolutelynopagebreak}
\setstretch{.7}
{\PaliGlossA{yo hi tattha dakkho analaso sampajāno patissato, ayaṃ vuccati, bhikkhave, bhikkhu porāṇe aggaññe ariyavaṃse ṭhito.}}\\
\begin{addmargin}[1em]{2em}
\setstretch{.5}
{\PaliGlossB{A mendicant who is deft, tireless, aware, and mindful in this is said to stand in the ancient, original noble tradition.}}\\
\end{addmargin}
\end{absolutelynopagebreak}

\begin{absolutelynopagebreak}
\setstretch{.7}
{\PaliGlossA{ime kho, bhikkhave, cattāro ariyavaṃsā aggaññā rattaññā vaṃsaññā porāṇā asaṅkiṇṇā asaṅkiṇṇapubbā, na saṅkīyanti na saṅkīyissanti, appaṭikuṭṭhā samaṇehi brāhmaṇehi viññūhi.}}\\
\begin{addmargin}[1em]{2em}
\setstretch{.5}
{\PaliGlossB{These four noble traditions are original, long-standing, traditional, and ancient. They are uncorrupted, as they have been since the beginning. They’re not being corrupted now nor will they be. Sensible ascetics and brahmins don’t look down on them.}}\\
\end{addmargin}
\end{absolutelynopagebreak}

\begin{absolutelynopagebreak}
\setstretch{.7}
{\PaliGlossA{imehi ca pana, bhikkhave, catūhi ariyavaṃsehi samannāgato bhikkhu puratthimāya cepi disāya viharati sveva aratiṃ sahati, na taṃ arati sahati;}}\\
\begin{addmargin}[1em]{2em}
\setstretch{.5}
{\PaliGlossB{When a mendicant has these four noble traditions, if they live in the east they prevail over discontent, and discontent doesn’t prevail over them.}}\\
\end{addmargin}
\end{absolutelynopagebreak}

\begin{absolutelynopagebreak}
\setstretch{.7}
{\PaliGlossA{pacchimāya cepi disāya viharati sveva aratiṃ sahati, na taṃ arati sahati;}}\\
\begin{addmargin}[1em]{2em}
\setstretch{.5}
{\PaliGlossB{If they live in the west …}}\\
\end{addmargin}
\end{absolutelynopagebreak}

\begin{absolutelynopagebreak}
\setstretch{.7}
{\PaliGlossA{uttarāya cepi disāya viharati sveva aratiṃ sahati, na taṃ arati sahati;}}\\
\begin{addmargin}[1em]{2em}
\setstretch{.5}
{\PaliGlossB{the north …}}\\
\end{addmargin}
\end{absolutelynopagebreak}

\begin{absolutelynopagebreak}
\setstretch{.7}
{\PaliGlossA{dakkhiṇāya cepi disāya viharati sveva aratiṃ sahati, na taṃ arati sahati.}}\\
\begin{addmargin}[1em]{2em}
\setstretch{.5}
{\PaliGlossB{the south, they prevail over discontent, and discontent doesn’t prevail over them.}}\\
\end{addmargin}
\end{absolutelynopagebreak}

\begin{absolutelynopagebreak}
\setstretch{.7}
{\PaliGlossA{taṃ kissa hetu?}}\\
\begin{addmargin}[1em]{2em}
\setstretch{.5}
{\PaliGlossB{Why is that?}}\\
\end{addmargin}
\end{absolutelynopagebreak}

\begin{absolutelynopagebreak}
\setstretch{.7}
{\PaliGlossA{aratiratisaho hi, bhikkhave, dhīroti.}}\\
\begin{addmargin}[1em]{2em}
\setstretch{.5}
{\PaliGlossB{Because a wise one prevails over desire and discontent.}}\\
\end{addmargin}
\end{absolutelynopagebreak}

\begin{absolutelynopagebreak}
\setstretch{.7}
{\PaliGlossA{nārati sahati dhīraṃ,}}\\
\begin{addmargin}[1em]{2em}
\setstretch{.5}
{\PaliGlossB{Dissatisfaction doesn’t prevail over a wise one;}}\\
\end{addmargin}
\end{absolutelynopagebreak}

\begin{absolutelynopagebreak}
\setstretch{.7}
{\PaliGlossA{nārati dhīraṃ sahati;}}\\
\begin{addmargin}[1em]{2em}
\setstretch{.5}
{\PaliGlossB{for the wise one is not beaten by discontent.}}\\
\end{addmargin}
\end{absolutelynopagebreak}

\begin{absolutelynopagebreak}
\setstretch{.7}
{\PaliGlossA{dhīrova aratiṃ sahati,}}\\
\begin{addmargin}[1em]{2em}
\setstretch{.5}
{\PaliGlossB{A wise one prevails over discontent,}}\\
\end{addmargin}
\end{absolutelynopagebreak}

\begin{absolutelynopagebreak}
\setstretch{.7}
{\PaliGlossA{dhīro hi aratissaho.}}\\
\begin{addmargin}[1em]{2em}
\setstretch{.5}
{\PaliGlossB{for the wise one is a beater of discontent.}}\\
\end{addmargin}
\end{absolutelynopagebreak}

\begin{absolutelynopagebreak}
\setstretch{.7}
{\PaliGlossA{sabbakammavihāyīnaṃ,}}\\
\begin{addmargin}[1em]{2em}
\setstretch{.5}
{\PaliGlossB{Who can hold back the dispeller,}}\\
\end{addmargin}
\end{absolutelynopagebreak}

\begin{absolutelynopagebreak}
\setstretch{.7}
{\PaliGlossA{panuṇṇaṃ ko nivāraye;}}\\
\begin{addmargin}[1em]{2em}
\setstretch{.5}
{\PaliGlossB{who’s thrown away all karma?}}\\
\end{addmargin}
\end{absolutelynopagebreak}

\begin{absolutelynopagebreak}
\setstretch{.7}
{\PaliGlossA{nekkhaṃ jambonadasseva,}}\\
\begin{addmargin}[1em]{2em}
\setstretch{.5}
{\PaliGlossB{They’re like a coin of mountain gold:}}\\
\end{addmargin}
\end{absolutelynopagebreak}

\begin{absolutelynopagebreak}
\setstretch{.7}
{\PaliGlossA{ko taṃ ninditumarahati;}}\\
\begin{addmargin}[1em]{2em}
\setstretch{.5}
{\PaliGlossB{who is worthy of criticizing them?}}\\
\end{addmargin}
\end{absolutelynopagebreak}

\begin{absolutelynopagebreak}
\setstretch{.7}
{\PaliGlossA{devāpi naṃ pasaṃsanti,}}\\
\begin{addmargin}[1em]{2em}
\setstretch{.5}
{\PaliGlossB{Even the gods praise them,}}\\
\end{addmargin}
\end{absolutelynopagebreak}

\begin{absolutelynopagebreak}
\setstretch{.7}
{\PaliGlossA{brahmunāpi pasaṃsito”ti.}}\\
\begin{addmargin}[1em]{2em}
\setstretch{.5}
{\PaliGlossB{and by Brahmā, too, they’re praised.”}}\\
\end{addmargin}
\end{absolutelynopagebreak}

\begin{absolutelynopagebreak}
\setstretch{.7}
{\PaliGlossA{aṭṭhamaṃ.}}\\
\begin{addmargin}[1em]{2em}
\setstretch{.5}
{\PaliGlossB{    -}}\\
\end{addmargin}
\end{absolutelynopagebreak}
