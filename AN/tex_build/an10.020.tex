
\begin{absolutelynopagebreak}
\setstretch{.7}
{\PaliGlossA{aṅguttara nikāya 10}}\\
\begin{addmargin}[1em]{2em}
\setstretch{.5}
{\PaliGlossB{Numbered Discourses 10}}\\
\end{addmargin}
\end{absolutelynopagebreak}

\begin{absolutelynopagebreak}
\setstretch{.7}
{\PaliGlossA{2. nāthavagga}}\\
\begin{addmargin}[1em]{2em}
\setstretch{.5}
{\PaliGlossB{2. A Protector}}\\
\end{addmargin}
\end{absolutelynopagebreak}

\begin{absolutelynopagebreak}
\setstretch{.7}
{\PaliGlossA{20. dutiyaariyāvāsasutta}}\\
\begin{addmargin}[1em]{2em}
\setstretch{.5}
{\PaliGlossB{20. Abodes of the Noble Ones (2nd)}}\\
\end{addmargin}
\end{absolutelynopagebreak}

\begin{absolutelynopagebreak}
\setstretch{.7}
{\PaliGlossA{ekaṃ samayaṃ bhagavā kurūsu viharati kammāsadhammaṃ nāma kurūnaṃ nigamo.}}\\
\begin{addmargin}[1em]{2em}
\setstretch{.5}
{\PaliGlossB{At one time the Buddha was staying in the land of the Kurus, near the Kuru town named Kammāsadamma.}}\\
\end{addmargin}
\end{absolutelynopagebreak}

\begin{absolutelynopagebreak}
\setstretch{.7}
{\PaliGlossA{tatra kho bhagavā bhikkhū āmantesi … pe ….}}\\
\begin{addmargin}[1em]{2em}
\setstretch{.5}
{\PaliGlossB{There the Buddha addressed the mendicants:}}\\
\end{addmargin}
\end{absolutelynopagebreak}

\begin{absolutelynopagebreak}
\setstretch{.7}
{\PaliGlossA{“dasayime, bhikkhave, ariyāvāsā, ye ariyā āvasiṃsu vā āvasanti vā āvasissanti vā.}}\\
\begin{addmargin}[1em]{2em}
\setstretch{.5}
{\PaliGlossB{“There are these ten noble abodes in which the noble ones of the past, present, and future abide.}}\\
\end{addmargin}
\end{absolutelynopagebreak}

\begin{absolutelynopagebreak}
\setstretch{.7}
{\PaliGlossA{katame dasa?}}\\
\begin{addmargin}[1em]{2em}
\setstretch{.5}
{\PaliGlossB{What ten?}}\\
\end{addmargin}
\end{absolutelynopagebreak}

\begin{absolutelynopagebreak}
\setstretch{.7}
{\PaliGlossA{idha, bhikkhave, bhikkhu pañcaṅgavippahīno hoti, chaḷaṅgasamannāgato, ekārakkho, caturāpasseno, paṇunnapaccekasacco, samavayasaṭṭhesano, anāvilasaṅkappo, passaddhakāyasaṅkhāro, suvimuttacitto, suvimuttapañño.}}\\
\begin{addmargin}[1em]{2em}
\setstretch{.5}
{\PaliGlossB{A mendicant has given up five factors, possesses six factors, has a single guard, has four supports, has eliminated idiosyncratic interpretations of the truth, has totally given up searching, has unsullied intentions, has stilled the physical process, and is well freed in mind and well freed by wisdom.}}\\
\end{addmargin}
\end{absolutelynopagebreak}

\begin{absolutelynopagebreak}
\setstretch{.7}
{\PaliGlossA{kathañca, bhikkhave, bhikkhu pañcaṅgavippahīno hoti?}}\\
\begin{addmargin}[1em]{2em}
\setstretch{.5}
{\PaliGlossB{And how has a mendicant given up five factors?}}\\
\end{addmargin}
\end{absolutelynopagebreak}

\begin{absolutelynopagebreak}
\setstretch{.7}
{\PaliGlossA{idha, bhikkhave, bhikkhuno kāmacchando pahīno hoti, byāpādo pahīno hoti, thinamiddhaṃ pahīnaṃ hoti, uddhaccakukkuccaṃ pahīnaṃ hoti, vicikicchā pahīnā hoti.}}\\
\begin{addmargin}[1em]{2em}
\setstretch{.5}
{\PaliGlossB{It’s when a mendicant has given up sensual desire, ill will, dullness and drowsiness, restlessness and remorse, and doubt.}}\\
\end{addmargin}
\end{absolutelynopagebreak}

\begin{absolutelynopagebreak}
\setstretch{.7}
{\PaliGlossA{evaṃ kho, bhikkhave, bhikkhu pañcaṅgavippahīno hoti. (1)}}\\
\begin{addmargin}[1em]{2em}
\setstretch{.5}
{\PaliGlossB{That’s how a mendicant has given up five factors.}}\\
\end{addmargin}
\end{absolutelynopagebreak}

\begin{absolutelynopagebreak}
\setstretch{.7}
{\PaliGlossA{kathañca, bhikkhave, bhikkhu chaḷaṅgasamannāgato hoti?}}\\
\begin{addmargin}[1em]{2em}
\setstretch{.5}
{\PaliGlossB{And how does a mendicant possess six factors?}}\\
\end{addmargin}
\end{absolutelynopagebreak}

\begin{absolutelynopagebreak}
\setstretch{.7}
{\PaliGlossA{idha, bhikkhave, bhikkhu cakkhunā rūpaṃ disvā neva sumano hoti na dummano, upekkhako viharati sato sampajāno.}}\\
\begin{addmargin}[1em]{2em}
\setstretch{.5}
{\PaliGlossB{It’s when a mendicant, seeing a sight with their eyes, is neither happy nor sad. They remain equanimous, mindful and aware.}}\\
\end{addmargin}
\end{absolutelynopagebreak}

\begin{absolutelynopagebreak}
\setstretch{.7}
{\PaliGlossA{sotena saddaṃ sutvā …}}\\
\begin{addmargin}[1em]{2em}
\setstretch{.5}
{\PaliGlossB{Hearing a sound with their ears …}}\\
\end{addmargin}
\end{absolutelynopagebreak}

\begin{absolutelynopagebreak}
\setstretch{.7}
{\PaliGlossA{ghānena gandhaṃ ghāyitvā …}}\\
\begin{addmargin}[1em]{2em}
\setstretch{.5}
{\PaliGlossB{Smelling an odor with their nose …}}\\
\end{addmargin}
\end{absolutelynopagebreak}

\begin{absolutelynopagebreak}
\setstretch{.7}
{\PaliGlossA{jivhāya rasaṃ sāyitvā …}}\\
\begin{addmargin}[1em]{2em}
\setstretch{.5}
{\PaliGlossB{Tasting a flavor with their tongue …}}\\
\end{addmargin}
\end{absolutelynopagebreak}

\begin{absolutelynopagebreak}
\setstretch{.7}
{\PaliGlossA{kāyena phoṭṭhabbaṃ phusitvā …}}\\
\begin{addmargin}[1em]{2em}
\setstretch{.5}
{\PaliGlossB{Feeling a touch with their body …}}\\
\end{addmargin}
\end{absolutelynopagebreak}

\begin{absolutelynopagebreak}
\setstretch{.7}
{\PaliGlossA{manasā dhammaṃ viññāya neva sumano hoti na dummano, upekkhako viharati sato sampajāno.}}\\
\begin{addmargin}[1em]{2em}
\setstretch{.5}
{\PaliGlossB{Knowing a thought with their mind, they’re neither happy nor sad. They remain equanimous, mindful and aware.}}\\
\end{addmargin}
\end{absolutelynopagebreak}

\begin{absolutelynopagebreak}
\setstretch{.7}
{\PaliGlossA{evaṃ kho, bhikkhave, bhikkhu chaḷaṅgasamannāgato hoti. (2)}}\\
\begin{addmargin}[1em]{2em}
\setstretch{.5}
{\PaliGlossB{That’s how a mendicant possesses six factors.}}\\
\end{addmargin}
\end{absolutelynopagebreak}

\begin{absolutelynopagebreak}
\setstretch{.7}
{\PaliGlossA{kathañca, bhikkhave, bhikkhu ekārakkho hoti?}}\\
\begin{addmargin}[1em]{2em}
\setstretch{.5}
{\PaliGlossB{And how does a mendicant have a single guard?}}\\
\end{addmargin}
\end{absolutelynopagebreak}

\begin{absolutelynopagebreak}
\setstretch{.7}
{\PaliGlossA{idha, bhikkhave, bhikkhu satārakkhena cetasā samannāgato hoti.}}\\
\begin{addmargin}[1em]{2em}
\setstretch{.5}
{\PaliGlossB{It’s when a mendicant’s heart is guarded by mindfulness.}}\\
\end{addmargin}
\end{absolutelynopagebreak}

\begin{absolutelynopagebreak}
\setstretch{.7}
{\PaliGlossA{evaṃ kho, bhikkhave, bhikkhu ekārakkho hoti. (3)}}\\
\begin{addmargin}[1em]{2em}
\setstretch{.5}
{\PaliGlossB{That’s how a mendicant has a single guard.}}\\
\end{addmargin}
\end{absolutelynopagebreak}

\begin{absolutelynopagebreak}
\setstretch{.7}
{\PaliGlossA{kathañca, bhikkhave, bhikkhu caturāpasseno hoti?}}\\
\begin{addmargin}[1em]{2em}
\setstretch{.5}
{\PaliGlossB{And how does a mendicant have four supports?}}\\
\end{addmargin}
\end{absolutelynopagebreak}

\begin{absolutelynopagebreak}
\setstretch{.7}
{\PaliGlossA{idha, bhikkhave, bhikkhu saṅkhāyekaṃ paṭisevati, saṅkhāyekaṃ adhivāseti, saṅkhāyekaṃ parivajjeti, saṅkhāyekaṃ vinodeti.}}\\
\begin{addmargin}[1em]{2em}
\setstretch{.5}
{\PaliGlossB{After reflection, a mendicant uses some things, endures some things, avoids some things, and gets rid of some things.}}\\
\end{addmargin}
\end{absolutelynopagebreak}

\begin{absolutelynopagebreak}
\setstretch{.7}
{\PaliGlossA{evaṃ kho, bhikkhave, bhikkhu caturāpasseno hoti. (4)}}\\
\begin{addmargin}[1em]{2em}
\setstretch{.5}
{\PaliGlossB{That’s how a mendicant has four supports.}}\\
\end{addmargin}
\end{absolutelynopagebreak}

\begin{absolutelynopagebreak}
\setstretch{.7}
{\PaliGlossA{kathañca, bhikkhave, bhikkhu paṇunnapaccekasacco hoti?}}\\
\begin{addmargin}[1em]{2em}
\setstretch{.5}
{\PaliGlossB{And how has a mendicant eliminated idiosyncratic interpretations of the truth?}}\\
\end{addmargin}
\end{absolutelynopagebreak}

\begin{absolutelynopagebreak}
\setstretch{.7}
{\PaliGlossA{idha, bhikkhave, bhikkhuno yāni tāni puthusamaṇabrāhmaṇānaṃ puthupaccekasaccāni, seyyathidaṃ:}}\\
\begin{addmargin}[1em]{2em}
\setstretch{.5}
{\PaliGlossB{Different ascetics and brahmins have different idiosyncratic interpretations of the truth. For example:}}\\
\end{addmargin}
\end{absolutelynopagebreak}

\begin{absolutelynopagebreak}
\setstretch{.7}
{\PaliGlossA{‘sassato loko’ti vā, ‘asassato loko’ti vā, ‘antavā loko’ti vā, ‘anantavā loko’ti vā, ‘taṃ jīvaṃ taṃ sarīran’ti vā, ‘aññaṃ jīvaṃ aññaṃ sarīran’ti vā, ‘hoti tathāgato paraṃ maraṇā’ti vā, ‘na hoti tathāgato paraṃ maraṇā’ti vā, ‘hoti ca na ca hoti tathāgato paraṃ maraṇā’ti vā, ‘neva hoti na na hoti tathāgato paraṃ maraṇā’ti vā,}}\\
\begin{addmargin}[1em]{2em}
\setstretch{.5}
{\PaliGlossB{the world is eternal, or not eternal, or finite, or infinite; the soul and the body are the same thing, or they are different things; after death, a Realized One exists, or doesn’t exist, or both exists and doesn’t exist, or neither exists nor doesn’t exist.}}\\
\end{addmargin}
\end{absolutelynopagebreak}

\begin{absolutelynopagebreak}
\setstretch{.7}
{\PaliGlossA{sabbāni tāni nunnāni honti paṇunnāni cattāni vantāni muttāni pahīnāni paṭinissaṭṭhāni.}}\\
\begin{addmargin}[1em]{2em}
\setstretch{.5}
{\PaliGlossB{A mendicant has dispelled, eliminated, thrown out, rejected, let go of, given up, and relinquished all these.}}\\
\end{addmargin}
\end{absolutelynopagebreak}

\begin{absolutelynopagebreak}
\setstretch{.7}
{\PaliGlossA{evaṃ kho, bhikkhave, bhikkhu paṇunnapaccekasacco hoti. (5)}}\\
\begin{addmargin}[1em]{2em}
\setstretch{.5}
{\PaliGlossB{That’s how a mendicant has eliminated idiosyncratic interpretations of the truth.}}\\
\end{addmargin}
\end{absolutelynopagebreak}

\begin{absolutelynopagebreak}
\setstretch{.7}
{\PaliGlossA{kathañca, bhikkhave, bhikkhu samavayasaṭṭhesano hoti?}}\\
\begin{addmargin}[1em]{2em}
\setstretch{.5}
{\PaliGlossB{And how has a mendicant totally given up searching?}}\\
\end{addmargin}
\end{absolutelynopagebreak}

\begin{absolutelynopagebreak}
\setstretch{.7}
{\PaliGlossA{idha, bhikkhave, bhikkhuno kāmesanā pahīnā hoti, bhavesanā pahīnā hoti, brahmacariyesanā paṭippassaddhā.}}\\
\begin{addmargin}[1em]{2em}
\setstretch{.5}
{\PaliGlossB{It’s when they’ve given up searching for sensual pleasures, for continued existence, and for a spiritual life.}}\\
\end{addmargin}
\end{absolutelynopagebreak}

\begin{absolutelynopagebreak}
\setstretch{.7}
{\PaliGlossA{evaṃ kho, bhikkhave, bhikkhu samavayasaṭṭhesano hoti. (6)}}\\
\begin{addmargin}[1em]{2em}
\setstretch{.5}
{\PaliGlossB{That’s how a mendicant has totally given up searching.}}\\
\end{addmargin}
\end{absolutelynopagebreak}

\begin{absolutelynopagebreak}
\setstretch{.7}
{\PaliGlossA{kathañca, bhikkhave, bhikkhu anāvilasaṅkappo hoti?}}\\
\begin{addmargin}[1em]{2em}
\setstretch{.5}
{\PaliGlossB{And how does a mendicant have unsullied intentions?}}\\
\end{addmargin}
\end{absolutelynopagebreak}

\begin{absolutelynopagebreak}
\setstretch{.7}
{\PaliGlossA{idha, bhikkhave, bhikkhuno kāmasaṅkappo pahīno hoti, byāpādasaṅkappo pahīno hoti, vihiṃsāsaṅkappo pahīno hoti.}}\\
\begin{addmargin}[1em]{2em}
\setstretch{.5}
{\PaliGlossB{It’s when a mendicant has given up intentions of sensuality, malice, and cruelty.}}\\
\end{addmargin}
\end{absolutelynopagebreak}

\begin{absolutelynopagebreak}
\setstretch{.7}
{\PaliGlossA{evaṃ kho, bhikkhave, bhikkhu anāvilasaṅkappo hoti. (7)}}\\
\begin{addmargin}[1em]{2em}
\setstretch{.5}
{\PaliGlossB{That’s how a mendicant has unsullied intentions.}}\\
\end{addmargin}
\end{absolutelynopagebreak}

\begin{absolutelynopagebreak}
\setstretch{.7}
{\PaliGlossA{kathañca, bhikkhave, bhikkhu passaddhakāyasaṅkhāro hoti?}}\\
\begin{addmargin}[1em]{2em}
\setstretch{.5}
{\PaliGlossB{And how has a mendicant stilled the physical process?}}\\
\end{addmargin}
\end{absolutelynopagebreak}

\begin{absolutelynopagebreak}
\setstretch{.7}
{\PaliGlossA{idha, bhikkhave, bhikkhu sukhassa ca pahānā dukkhassa ca pahānā pubbeva somanassadomanassānaṃ atthaṅgamā adukkhamasukhaṃ upekkhāsatipārisuddhiṃ catutthaṃ jhānaṃ upasampajja viharati.}}\\
\begin{addmargin}[1em]{2em}
\setstretch{.5}
{\PaliGlossB{It’s when, giving up pleasure and pain, and ending former happiness and sadness, they enter and remain in the fourth absorption, without pleasure or pain, with pure equanimity and mindfulness.}}\\
\end{addmargin}
\end{absolutelynopagebreak}

\begin{absolutelynopagebreak}
\setstretch{.7}
{\PaliGlossA{evaṃ kho, bhikkhave, bhikkhu passaddhakāyasaṅkhāro hoti. (8)}}\\
\begin{addmargin}[1em]{2em}
\setstretch{.5}
{\PaliGlossB{That’s how a mendicant has stilled the physical process.}}\\
\end{addmargin}
\end{absolutelynopagebreak}

\begin{absolutelynopagebreak}
\setstretch{.7}
{\PaliGlossA{kathañca, bhikkhave, bhikkhu suvimuttacitto hoti?}}\\
\begin{addmargin}[1em]{2em}
\setstretch{.5}
{\PaliGlossB{And how is a mendicant well freed in mind?}}\\
\end{addmargin}
\end{absolutelynopagebreak}

\begin{absolutelynopagebreak}
\setstretch{.7}
{\PaliGlossA{idha, bhikkhave, bhikkhuno rāgā cittaṃ vimuttaṃ hoti, dosā cittaṃ vimuttaṃ hoti, mohā cittaṃ vimuttaṃ hoti.}}\\
\begin{addmargin}[1em]{2em}
\setstretch{.5}
{\PaliGlossB{It’s when a mendicant’s mind is freed from greed, hate, and delusion.}}\\
\end{addmargin}
\end{absolutelynopagebreak}

\begin{absolutelynopagebreak}
\setstretch{.7}
{\PaliGlossA{evaṃ kho, bhikkhave, bhikkhu suvimuttacitto hoti. (9)}}\\
\begin{addmargin}[1em]{2em}
\setstretch{.5}
{\PaliGlossB{That’s how a mendicant is well freed in mind.}}\\
\end{addmargin}
\end{absolutelynopagebreak}

\begin{absolutelynopagebreak}
\setstretch{.7}
{\PaliGlossA{kathañca, bhikkhave, bhikkhu suvimuttapañño hoti?}}\\
\begin{addmargin}[1em]{2em}
\setstretch{.5}
{\PaliGlossB{And how is a mendicant well freed by wisdom?}}\\
\end{addmargin}
\end{absolutelynopagebreak}

\begin{absolutelynopagebreak}
\setstretch{.7}
{\PaliGlossA{idha, bhikkhave, bhikkhu ‘rāgo me pahīno ucchinnamūlo tālāvatthukato anabhāvaṅkato āyatiṃ anuppādadhammo’ti pajānāti, doso me pahīno … pe … ‘moho me pahīno ucchinnamūlo tālāvatthukato anabhāvaṅkato āyatiṃ anuppādadhammo’ti pajānāti.}}\\
\begin{addmargin}[1em]{2em}
\setstretch{.5}
{\PaliGlossB{It’s when a mendicant understands: ‘I’ve given up greed, hate, and delusion, cut them off at the root, made them like a palm stump, obliterated them, so they’re unable to arise in the future.’}}\\
\end{addmargin}
\end{absolutelynopagebreak}

\begin{absolutelynopagebreak}
\setstretch{.7}
{\PaliGlossA{evaṃ kho, bhikkhave, bhikkhu suvimuttapañño hoti. (10)}}\\
\begin{addmargin}[1em]{2em}
\setstretch{.5}
{\PaliGlossB{That’s how a mendicant’s mind is well freed by wisdom.}}\\
\end{addmargin}
\end{absolutelynopagebreak}

\begin{absolutelynopagebreak}
\setstretch{.7}
{\PaliGlossA{ye hi keci, bhikkhave, atītamaddhānaṃ ariyā ariyāvāse āvasiṃsu, sabbe te imeva dasa ariyāvāse āvasiṃsu;}}\\
\begin{addmargin}[1em]{2em}
\setstretch{.5}
{\PaliGlossB{Mendicants, whether in the past, future, or present, all the noble ones of noble abodes abide in these same ten noble abodes.}}\\
\end{addmargin}
\end{absolutelynopagebreak}

\begin{absolutelynopagebreak}
\setstretch{.7}
{\PaliGlossA{ye hi keci, bhikkhave, anāgatamaddhānaṃ ariyā ariyāvāse āvasissanti, sabbe te imeva dasa ariyāvāse āvasissanti;}}\\
\begin{addmargin}[1em]{2em}
\setstretch{.5}
{\PaliGlossB{    -}}\\
\end{addmargin}
\end{absolutelynopagebreak}

\begin{absolutelynopagebreak}
\setstretch{.7}
{\PaliGlossA{ye hi keci, bhikkhave, etarahi ariyā ariyāvāse āvasanti, sabbe te imeva dasa ariyāvāse āvasanti.}}\\
\begin{addmargin}[1em]{2em}
\setstretch{.5}
{\PaliGlossB{    -}}\\
\end{addmargin}
\end{absolutelynopagebreak}

\begin{absolutelynopagebreak}
\setstretch{.7}
{\PaliGlossA{ime kho, bhikkhave, dasa ariyāvāsā, ye ariyā āvasiṃsu vā āvasanti vā āvasissanti vā”ti.}}\\
\begin{addmargin}[1em]{2em}
\setstretch{.5}
{\PaliGlossB{These are the ten noble abodes in which the noble ones of the past, present, and future abide.”}}\\
\end{addmargin}
\end{absolutelynopagebreak}

\begin{absolutelynopagebreak}
\setstretch{.7}
{\PaliGlossA{dasamaṃ.}}\\
\begin{addmargin}[1em]{2em}
\setstretch{.5}
{\PaliGlossB{    -}}\\
\end{addmargin}
\end{absolutelynopagebreak}

\begin{absolutelynopagebreak}
\setstretch{.7}
{\PaliGlossA{nāthavaggo dutiyo.}}\\
\begin{addmargin}[1em]{2em}
\setstretch{.5}
{\PaliGlossB{    -}}\\
\end{addmargin}
\end{absolutelynopagebreak}

\begin{absolutelynopagebreak}
\setstretch{.7}
{\PaliGlossA{senāsanañca pañcaṅgaṃ,}}\\
\begin{addmargin}[1em]{2em}
\setstretch{.5}
{\PaliGlossB{    -}}\\
\end{addmargin}
\end{absolutelynopagebreak}

\begin{absolutelynopagebreak}
\setstretch{.7}
{\PaliGlossA{saṃyojanākhilena ca;}}\\
\begin{addmargin}[1em]{2em}
\setstretch{.5}
{\PaliGlossB{    -}}\\
\end{addmargin}
\end{absolutelynopagebreak}

\begin{absolutelynopagebreak}
\setstretch{.7}
{\PaliGlossA{appamādo āhuneyyo,}}\\
\begin{addmargin}[1em]{2em}
\setstretch{.5}
{\PaliGlossB{    -}}\\
\end{addmargin}
\end{absolutelynopagebreak}

\begin{absolutelynopagebreak}
\setstretch{.7}
{\PaliGlossA{dve nāthā dve ariyāvāsāti.}}\\
\begin{addmargin}[1em]{2em}
\setstretch{.5}
{\PaliGlossB{    -}}\\
\end{addmargin}
\end{absolutelynopagebreak}
