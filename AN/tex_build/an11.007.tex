
\begin{absolutelynopagebreak}
\setstretch{.7}
{\PaliGlossA{aṅguttara nikāya 11}}\\
\begin{addmargin}[1em]{2em}
\setstretch{.5}
{\PaliGlossB{Numbered Discourses 11}}\\
\end{addmargin}
\end{absolutelynopagebreak}

\begin{absolutelynopagebreak}
\setstretch{.7}
{\PaliGlossA{1. nissayavagga}}\\
\begin{addmargin}[1em]{2em}
\setstretch{.5}
{\PaliGlossB{1. Dependence}}\\
\end{addmargin}
\end{absolutelynopagebreak}

\begin{absolutelynopagebreak}
\setstretch{.7}
{\PaliGlossA{7. saññāsutta}}\\
\begin{addmargin}[1em]{2em}
\setstretch{.5}
{\PaliGlossB{7. Percipient}}\\
\end{addmargin}
\end{absolutelynopagebreak}

\begin{absolutelynopagebreak}
\setstretch{.7}
{\PaliGlossA{atha kho āyasmā ānando yena bhagavā tenupasaṅkami; upasaṅkamitvā bhagavantaṃ abhivādetvā ekamantaṃ nisīdi. ekamantaṃ nisinno kho āyasmā ānando bhagavantaṃ etadavoca:}}\\
\begin{addmargin}[1em]{2em}
\setstretch{.5}
{\PaliGlossB{Then Venerable Ānanda went up to the Buddha, bowed, sat down to one side, and said to him:}}\\
\end{addmargin}
\end{absolutelynopagebreak}

\begin{absolutelynopagebreak}
\setstretch{.7}
{\PaliGlossA{“siyā nu kho, bhante, bhikkhuno tathārūpo samādhipaṭilābho yathā neva pathaviyaṃ pathavisaññī assa, na āpasmiṃ āposaññī assa, na tejasmiṃ tejosaññī assa, na vāyasmiṃ vāyosaññī assa, na ākāsānañcāyatane ākāsānañcāyatanasaññī assa, na viññāṇañcāyatane viññāṇañcāyatanasaññī assa, na ākiñcaññāyatane ākiñcaññāyatanasaññī assa, na nevasaññānāsaññāyatane nevasaññānāsaññāyatanasaññī assa, na idhaloke idhalokasaññī assa, na paraloke paralokasaññī assa, yampidaṃ diṭṭhaṃ sutaṃ mutaṃ viññātaṃ pattaṃ pariyesitaṃ anuvicaritaṃ manasā, tatrāpi na saññī assa; saññī ca pana assā”ti?}}\\
\begin{addmargin}[1em]{2em}
\setstretch{.5}
{\PaliGlossB{“Could it be, sir, that a mendicant might gain a state of immersion like this? They wouldn’t perceive earth in earth, water in water, fire in fire, or air in air. And they wouldn’t perceive the dimension of infinite space in the dimension of infinite space, the dimension of infinite consciousness in the dimension of infinite consciousness, the dimension of nothingness in the dimension of nothingness, or the dimension of neither perception nor non-perception in the dimension of neither perception nor non-perception. They wouldn’t perceive this world in this world, or the other world in the other world. And they wouldn’t perceive what is seen, heard, thought, known, attained, sought, or explored by the mind. And yet they would still perceive.”}}\\
\end{addmargin}
\end{absolutelynopagebreak}

\begin{absolutelynopagebreak}
\setstretch{.7}
{\PaliGlossA{“siyā, ānanda, bhikkhuno tathārūpo samādhipaṭilābho yathā neva pathaviyaṃ pathavisaññī assa, na āpasmiṃ āposaññī assa, na tejasmiṃ tejosaññī assa, na vāyasmiṃ vāyosaññī assa, na ākāsānañcāyatane ākāsānañcāyatanasaññī assa, na viññāṇañcāyatane viññāṇañcāyatanasaññī assa, na ākiñcaññāyatane ākiñcaññāyatanasaññī assa, na nevasaññānāsaññāyatane nevasaññānāsaññāyatanasaññī assa, na idhaloke idhalokasaññī assa, na paraloke paralokasaññī assa, yampidaṃ diṭṭhaṃ sutaṃ mutaṃ viññātaṃ pattaṃ pariyesitaṃ anuvicaritaṃ manasā, tatrāpi na saññī assa; saññī ca pana assā”ti.}}\\
\begin{addmargin}[1em]{2em}
\setstretch{.5}
{\PaliGlossB{“It could be, Ānanda, that a mendicant might gain a state of immersion like this. They wouldn’t perceive earth in earth, water in water, fire in fire, or air in air. And they wouldn’t perceive the dimension of infinite space in the dimension of infinite space, the dimension of infinite consciousness in the dimension of infinite consciousness, the dimension of nothingness in the dimension of nothingness, or the dimension of neither perception nor non-perception in the dimension of neither perception nor non-perception. They wouldn’t perceive this world in this world, or the other world in the other world. And they wouldn’t perceive what is seen, heard, thought, known, attained, sought, or explored by the mind. And yet they would still perceive.”}}\\
\end{addmargin}
\end{absolutelynopagebreak}

\begin{absolutelynopagebreak}
\setstretch{.7}
{\PaliGlossA{“yathā kathaṃ pana, bhante, siyā bhikkhuno tathārūpo samādhipaṭilābho yathā neva pathaviyaṃ pathavisaññī assa, na āpasmiṃ āposaññī assa, na tejasmiṃ tejosaññī assa, na vāyasmiṃ vāyosaññī assa, na ākāsānañcāyatane ākāsānañcāyatanasaññī assa, na viññāṇañcāyatane viññāṇañcāyatanasaññī assa, na ākiñcaññāyatane ākiñcaññāyatanasaññī assa, na nevasaññānāsaññāyatane nevasaññānāsaññāyatanasaññī assa, na idhaloke idhalokasaññī assa, na paraloke paralokasaññī assa, yampidaṃ diṭṭhaṃ sutaṃ mutaṃ viññātaṃ pattaṃ pariyesitaṃ anuvicaritaṃ manasā, tatrāpi na saññī assa, saññī ca pana assā”ti.}}\\
\begin{addmargin}[1em]{2em}
\setstretch{.5}
{\PaliGlossB{“But how could this be, sir?”}}\\
\end{addmargin}
\end{absolutelynopagebreak}

\begin{absolutelynopagebreak}
\setstretch{.7}
{\PaliGlossA{“idhānanda, bhikkhu evaṃsaññī hoti:}}\\
\begin{addmargin}[1em]{2em}
\setstretch{.5}
{\PaliGlossB{“Ānanda, it’s when a mendicant perceives:}}\\
\end{addmargin}
\end{absolutelynopagebreak}

\begin{absolutelynopagebreak}
\setstretch{.7}
{\PaliGlossA{‘etaṃ santaṃ etaṃ paṇītaṃ, yadidaṃ sabbasaṅkhārasamatho sabbūpadhipaṭinissaggo taṇhākkhayo virāgo nirodho nibbānan’ti.}}\\
\begin{addmargin}[1em]{2em}
\setstretch{.5}
{\PaliGlossB{‘This is peaceful; this is sublime—that is, the stilling of all activities, the letting go of all attachments, the ending of craving, fading away, cessation, extinguishment.’}}\\
\end{addmargin}
\end{absolutelynopagebreak}

\begin{absolutelynopagebreak}
\setstretch{.7}
{\PaliGlossA{evaṃ kho, ānanda, siyā bhikkhuno tathārūpo samādhipaṭilābho yathā neva pathaviyaṃ pathavisaññī assa, na āpasmiṃ āposaññī assa, na tejasmiṃ tejosaññī assa, na vāyasmiṃ vāyosaññī assa, na ākāsānañcāyatane ākāsānañcāyatanasaññī assa, na viññāṇañcāyatane viññāṇañcāyatanasaññī assa, na ākiñcaññāyatane ākiñcaññāyatanasaññī assa, na nevasaññānāsaññāyatane nevasaññānāsaññāyatanasaññī assa, na idhaloke idhalokasaññī assa, na paraloke paralokasaññī assa, yampidaṃ diṭṭhaṃ sutaṃ mutaṃ viññātaṃ pattaṃ pariyesitaṃ anuvicaritaṃ manasā, tatrāpi na saññī assa, saññī ca pana assā”ti.}}\\
\begin{addmargin}[1em]{2em}
\setstretch{.5}
{\PaliGlossB{That’s how a mendicant might gain a state of immersion like this. They wouldn’t perceive earth in earth, water in water, fire in fire, or air in air. And they wouldn’t perceive the dimension of infinite space in the dimension of infinite space, the dimension of infinite consciousness in the dimension of infinite consciousness, the dimension of nothingness in the dimension of nothingness, or the dimension of neither perception nor non-perception in the dimension of neither perception nor non-perception. They wouldn’t perceive this world in this world, or the other world in the other world. And they wouldn’t perceive what is seen, heard, thought, known, attained, sought, or explored by the mind. And yet they would still perceive.”}}\\
\end{addmargin}
\end{absolutelynopagebreak}

\begin{absolutelynopagebreak}
\setstretch{.7}
{\PaliGlossA{atha kho āyasmā ānando bhagavato bhāsitaṃ abhinanditvā anumoditvā uṭṭhāyāsanā bhagavantaṃ abhivādetvā padakkhiṇaṃ katvā yenāyasmā sāriputto tenupasaṅkami; upasaṅkamitvā āyasmatā sāriputtena saddhiṃ sammodi.}}\\
\begin{addmargin}[1em]{2em}
\setstretch{.5}
{\PaliGlossB{And then Ānanda approved and agreed with what the Buddha said. He got up from his seat, bowed, and respectfully circled the Buddha, keeping him on his right. Then he went up to Venerable Sāriputta, and exchanged greetings with him.}}\\
\end{addmargin}
\end{absolutelynopagebreak}

\begin{absolutelynopagebreak}
\setstretch{.7}
{\PaliGlossA{sammodanīyaṃ kathaṃ sāraṇīyaṃ vītisāretvā ekamantaṃ nisīdi. ekamantaṃ nisinno kho āyasmā ānando āyasmantaṃ sāriputtaṃ etadavoca:}}\\
\begin{addmargin}[1em]{2em}
\setstretch{.5}
{\PaliGlossB{When the greetings and polite conversation were over, he sat down to one side and said to Sāriputta:}}\\
\end{addmargin}
\end{absolutelynopagebreak}

\begin{absolutelynopagebreak}
\setstretch{.7}
{\PaliGlossA{“siyā nu kho, āvuso sāriputta, bhikkhuno tathārūpo samādhipaṭilābho yathā neva pathaviyaṃ pathavisaññī assa … pe … yampidaṃ diṭṭhaṃ sutaṃ mutaṃ viññātaṃ pattaṃ pariyesitaṃ anuvicaritaṃ manasā, tatrāpi na saññī assa, saññī pana assā”ti.}}\\
\begin{addmargin}[1em]{2em}
\setstretch{.5}
{\PaliGlossB{“Could it be, reverend Sāriputta, that a mendicant might gain a state of immersion like this? They wouldn’t perceive earth in earth … And they wouldn’t perceive what is seen, heard, thought, known, attained, sought, or explored by the mind. And yet they would still perceive.”}}\\
\end{addmargin}
\end{absolutelynopagebreak}

\begin{absolutelynopagebreak}
\setstretch{.7}
{\PaliGlossA{“siyā, āvuso ānanda, bhikkhuno tathārūpo samādhipaṭilābho yathā neva pathaviyaṃ pathavisaññī assa … pe … yampidaṃ diṭṭhaṃ sutaṃ mutaṃ viññātaṃ pattaṃ pariyesitaṃ anuvicaritaṃ manasā, tatrāpi na saññī assa, saññī ca pana assā”ti.}}\\
\begin{addmargin}[1em]{2em}
\setstretch{.5}
{\PaliGlossB{“It could be, Reverend Ānanda.”}}\\
\end{addmargin}
\end{absolutelynopagebreak}

\begin{absolutelynopagebreak}
\setstretch{.7}
{\PaliGlossA{“yathā kathaṃ panāvuso sāriputta, siyā bhikkhuno tathārūpo samādhipaṭilābho yathā neva pathaviyaṃ pathavisaññī assa … pe … yampidaṃ diṭṭhaṃ sutaṃ mutaṃ viññātaṃ pattaṃ pariyesitaṃ anuvicaritaṃ manasā, tatrāpi na saññī assa, saññī ca pana assā”ti?}}\\
\begin{addmargin}[1em]{2em}
\setstretch{.5}
{\PaliGlossB{“But how could this be?”}}\\
\end{addmargin}
\end{absolutelynopagebreak}

\begin{absolutelynopagebreak}
\setstretch{.7}
{\PaliGlossA{“idha, āvuso ānanda, bhikkhu evaṃsaññī hoti:}}\\
\begin{addmargin}[1em]{2em}
\setstretch{.5}
{\PaliGlossB{“Ānanda, it’s when a mendicant perceives:}}\\
\end{addmargin}
\end{absolutelynopagebreak}

\begin{absolutelynopagebreak}
\setstretch{.7}
{\PaliGlossA{‘etaṃ santaṃ etaṃ paṇītaṃ, yadidaṃ sabbasaṅkhārasamatho sabbūpadhipaṭinissaggo taṇhākkhayo virāgo nirodho nibbānan’ti.}}\\
\begin{addmargin}[1em]{2em}
\setstretch{.5}
{\PaliGlossB{‘This is peaceful; this is sublime—that is, the stilling of all activities, the letting go of all attachments, the ending of craving, fading away, cessation, extinguishment.’}}\\
\end{addmargin}
\end{absolutelynopagebreak}

\begin{absolutelynopagebreak}
\setstretch{.7}
{\PaliGlossA{evaṃ kho, āvuso ānanda, siyā bhikkhuno tathārūpo samādhipaṭilābho yathā neva pathaviyaṃ pathavisaññī assa … pe … yampidaṃ diṭṭhaṃ sutaṃ mutaṃ viññātaṃ pattaṃ pariyesitaṃ anuvicaritaṃ manasā, tatrāpi na saññī assa, saññī ca pana assā”ti.}}\\
\begin{addmargin}[1em]{2em}
\setstretch{.5}
{\PaliGlossB{That’s how a mendicant might gain a state of immersion like this. They wouldn’t perceive earth in earth … And they wouldn’t perceive what is seen, heard, thought, known, attained, sought, or explored by the mind. And yet they would still perceive.”}}\\
\end{addmargin}
\end{absolutelynopagebreak}

\begin{absolutelynopagebreak}
\setstretch{.7}
{\PaliGlossA{“acchariyaṃ, āvuso, abbhutaṃ, āvuso.}}\\
\begin{addmargin}[1em]{2em}
\setstretch{.5}
{\PaliGlossB{“It’s incredible, it’s amazing!}}\\
\end{addmargin}
\end{absolutelynopagebreak}

\begin{absolutelynopagebreak}
\setstretch{.7}
{\PaliGlossA{yatra hi nāma satthu ceva sāvakassa ca atthena attho byañjanena byañjanaṃ saṃsandissati samessati na viggayhissati, yadidaṃ aggapadasmiṃ.}}\\
\begin{addmargin}[1em]{2em}
\setstretch{.5}
{\PaliGlossB{How the meaning and the phrasing of the teacher and the disciple fit together and agree without conflict when it comes to the chief matter!}}\\
\end{addmargin}
\end{absolutelynopagebreak}

\begin{absolutelynopagebreak}
\setstretch{.7}
{\PaliGlossA{idānāhaṃ, āvuso, bhagavantaṃ upasaṅkamitvā etamatthaṃ apucchiṃ.}}\\
\begin{addmargin}[1em]{2em}
\setstretch{.5}
{\PaliGlossB{Just now I went to the Buddha and asked him about this matter.}}\\
\end{addmargin}
\end{absolutelynopagebreak}

\begin{absolutelynopagebreak}
\setstretch{.7}
{\PaliGlossA{bhagavāpi me etehi akkharehi etehi padehi etehi byañjanehi etamatthaṃ byākāsi, seyyathāpi āyasmā sāriputto.}}\\
\begin{addmargin}[1em]{2em}
\setstretch{.5}
{\PaliGlossB{And the Buddha explained it to me in this manner, with these words and phrases, just like Venerable Sāriputta.}}\\
\end{addmargin}
\end{absolutelynopagebreak}

\begin{absolutelynopagebreak}
\setstretch{.7}
{\PaliGlossA{acchariyaṃ, āvuso, abbhutaṃ, āvuso.}}\\
\begin{addmargin}[1em]{2em}
\setstretch{.5}
{\PaliGlossB{It’s incredible, it’s amazing!}}\\
\end{addmargin}
\end{absolutelynopagebreak}

\begin{absolutelynopagebreak}
\setstretch{.7}
{\PaliGlossA{yatra hi nāma satthu ceva sāvakassa ca atthena attho byañjanena byañjanaṃ saṃsandissati samessati na viggayhissati, yadidaṃ aggapadasmin”ti.}}\\
\begin{addmargin}[1em]{2em}
\setstretch{.5}
{\PaliGlossB{How the meaning and the phrasing of the teacher and the disciple fit together and agree without conflict when it comes to the chief matter!”}}\\
\end{addmargin}
\end{absolutelynopagebreak}

\begin{absolutelynopagebreak}
\setstretch{.7}
{\PaliGlossA{sattamaṃ.}}\\
\begin{addmargin}[1em]{2em}
\setstretch{.5}
{\PaliGlossB{    -}}\\
\end{addmargin}
\end{absolutelynopagebreak}
