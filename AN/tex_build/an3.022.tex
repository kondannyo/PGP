
\begin{absolutelynopagebreak}
\setstretch{.7}
{\PaliGlossA{aṅguttara nikāya 3}}\\
\begin{addmargin}[1em]{2em}
\setstretch{.5}
{\PaliGlossB{Numbered Discourses 3}}\\
\end{addmargin}
\end{absolutelynopagebreak}

\begin{absolutelynopagebreak}
\setstretch{.7}
{\PaliGlossA{3. puggalavagga}}\\
\begin{addmargin}[1em]{2em}
\setstretch{.5}
{\PaliGlossB{3. Persons}}\\
\end{addmargin}
\end{absolutelynopagebreak}

\begin{absolutelynopagebreak}
\setstretch{.7}
{\PaliGlossA{22. gilānasutta}}\\
\begin{addmargin}[1em]{2em}
\setstretch{.5}
{\PaliGlossB{22. Patients}}\\
\end{addmargin}
\end{absolutelynopagebreak}

\begin{absolutelynopagebreak}
\setstretch{.7}
{\PaliGlossA{“tayome, bhikkhave, gilānā santo saṃvijjamānā lokasmiṃ.}}\\
\begin{addmargin}[1em]{2em}
\setstretch{.5}
{\PaliGlossB{“These three patients are found in the world.}}\\
\end{addmargin}
\end{absolutelynopagebreak}

\begin{absolutelynopagebreak}
\setstretch{.7}
{\PaliGlossA{katame tayo?}}\\
\begin{addmargin}[1em]{2em}
\setstretch{.5}
{\PaliGlossB{What three?}}\\
\end{addmargin}
\end{absolutelynopagebreak}

\begin{absolutelynopagebreak}
\setstretch{.7}
{\PaliGlossA{idha, bhikkhave, ekacco gilāno labhanto vā sappāyāni bhojanāni alabhanto vā sappāyāni bhojanāni, labhanto vā sappāyāni bhesajjāni alabhanto vā sappāyāni bhesajjāni, labhanto vā patirūpaṃ upaṭṭhākaṃ alabhanto vā patirūpaṃ upaṭṭhākaṃ neva vuṭṭhāti tamhā ābādhā.}}\\
\begin{addmargin}[1em]{2em}
\setstretch{.5}
{\PaliGlossB{In some cases a patient won’t recover from an illness, regardless of whether or not they get suitable food and medicines, and a capable carer.}}\\
\end{addmargin}
\end{absolutelynopagebreak}

\begin{absolutelynopagebreak}
\setstretch{.7}
{\PaliGlossA{idha pana, bhikkhave, ekacco gilāno labhanto vā sappāyāni bhojanāni alabhanto vā sappāyāni bhojanāni, labhanto vā sappāyāni bhesajjāni alabhanto vā sappāyāni bhesajjāni, labhanto vā patirūpaṃ upaṭṭhākaṃ alabhanto vā patirūpaṃ upaṭṭhākaṃ vuṭṭhāti tamhā ābādhā.}}\\
\begin{addmargin}[1em]{2em}
\setstretch{.5}
{\PaliGlossB{In some cases a patient will recover from an illness, regardless of whether or not they get suitable food and medicines, and a capable carer.}}\\
\end{addmargin}
\end{absolutelynopagebreak}

\begin{absolutelynopagebreak}
\setstretch{.7}
{\PaliGlossA{idha pana, bhikkhave, ekacco gilāno labhantova sappāyāni bhojanāni no alabhanto, labhantova sappāyāni bhesajjāni no alabhanto, labhantova patirūpaṃ upaṭṭhākaṃ no alabhanto vuṭṭhāti tamhā ābādhā.}}\\
\begin{addmargin}[1em]{2em}
\setstretch{.5}
{\PaliGlossB{In some cases a patient can recover from an illness, but only if they get suitable food and medicines, and a capable carer, and not if they don’t get these things.}}\\
\end{addmargin}
\end{absolutelynopagebreak}

\begin{absolutelynopagebreak}
\setstretch{.7}
{\PaliGlossA{tatra, bhikkhave, yvāyaṃ gilāno labhantova sappāyāni bhojanāni no alabhanto, labhantova sappāyāni bhesajjāni no alabhanto, labhantova patirūpaṃ upaṭṭhākaṃ no alabhanto vuṭṭhāti tamhā ābādhā, imaṃ kho, bhikkhave, gilānaṃ paṭicca gilānabhattaṃ anuññātaṃ gilānabhesajjaṃ anuññātaṃ gilānupaṭṭhāko anuññāto.}}\\
\begin{addmargin}[1em]{2em}
\setstretch{.5}
{\PaliGlossB{Now, it’s for the sake of the last patient—who will recover only if they get suitable food and medicines, and a capable carer—that food, medicines, and a carer are prescribed.}}\\
\end{addmargin}
\end{absolutelynopagebreak}

\begin{absolutelynopagebreak}
\setstretch{.7}
{\PaliGlossA{imañca pana, bhikkhave, gilānaṃ paṭicca aññepi gilānā upaṭṭhātabbā.}}\\
\begin{addmargin}[1em]{2em}
\setstretch{.5}
{\PaliGlossB{But also, for the sake of this patient, the other patients should be looked after.}}\\
\end{addmargin}
\end{absolutelynopagebreak}

\begin{absolutelynopagebreak}
\setstretch{.7}
{\PaliGlossA{ime kho, bhikkhave, tayo gilānā santo saṃvijjamānā lokasmiṃ.}}\\
\begin{addmargin}[1em]{2em}
\setstretch{.5}
{\PaliGlossB{These are the three kinds of patients found in the world.}}\\
\end{addmargin}
\end{absolutelynopagebreak}

\begin{absolutelynopagebreak}
\setstretch{.7}
{\PaliGlossA{evamevaṃ kho, bhikkhave, tayome gilānūpamā puggalā santo saṃvijjamānā lokasmiṃ.}}\\
\begin{addmargin}[1em]{2em}
\setstretch{.5}
{\PaliGlossB{In the same way, these three people similar to patients are found among the mendicants.}}\\
\end{addmargin}
\end{absolutelynopagebreak}

\begin{absolutelynopagebreak}
\setstretch{.7}
{\PaliGlossA{katame tayo?}}\\
\begin{addmargin}[1em]{2em}
\setstretch{.5}
{\PaliGlossB{What three?}}\\
\end{addmargin}
\end{absolutelynopagebreak}

\begin{absolutelynopagebreak}
\setstretch{.7}
{\PaliGlossA{idha, bhikkhave, ekacco puggalo labhanto vā tathāgataṃ dassanāya alabhanto vā tathāgataṃ dassanāya, labhanto vā tathāgatappaveditaṃ dhammavinayaṃ savanāya alabhanto vā tathāgatappaveditaṃ dhammavinayaṃ savanāya neva okkamati niyāmaṃ kusalesu dhammesu sammattaṃ.}}\\
\begin{addmargin}[1em]{2em}
\setstretch{.5}
{\PaliGlossB{Some people don’t enter the sure path with regards to skillful qualities, regardless of whether or not they get to see a Realized One, and to hear the teaching and training that he proclaims.}}\\
\end{addmargin}
\end{absolutelynopagebreak}

\begin{absolutelynopagebreak}
\setstretch{.7}
{\PaliGlossA{idha, pana, bhikkhave, ekacco puggalo labhanto vā tathāgataṃ dassanāya alabhanto vā tathāgataṃ dassanāya, labhanto vā tathāgatappaveditaṃ dhammavinayaṃ savanāya alabhanto vā tathāgatappaveditaṃ dhammavinayaṃ savanāya okkamati niyāmaṃ kusalesu dhammesu sammattaṃ.}}\\
\begin{addmargin}[1em]{2em}
\setstretch{.5}
{\PaliGlossB{Some people do enter the sure path with regards to skillful qualities, regardless of whether or not they get to see a Realized One, and to hear the teaching and training that he proclaims.}}\\
\end{addmargin}
\end{absolutelynopagebreak}

\begin{absolutelynopagebreak}
\setstretch{.7}
{\PaliGlossA{idha pana, bhikkhave, ekacco puggalo labhantova tathāgataṃ dassanāya no alabhanto, labhantova tathāgatappaveditaṃ dhammavinayaṃ savanāya no alabhanto okkamati niyāmaṃ kusalesu dhammesu sammattaṃ.}}\\
\begin{addmargin}[1em]{2em}
\setstretch{.5}
{\PaliGlossB{Some people can enter the sure path with regards to skillful qualities, but only if they get to see a Realized One, and to hear the teaching and training that he proclaims, and not when they don’t get those things.}}\\
\end{addmargin}
\end{absolutelynopagebreak}

\begin{absolutelynopagebreak}
\setstretch{.7}
{\PaliGlossA{tatra, bhikkhave, yvāyaṃ puggalo labhantova tathāgataṃ dassanāya no alabhanto, labhantova tathāgatappaveditaṃ dhammavinayaṃ savanāya no alabhanto okkamati niyāmaṃ kusalesu dhammesu sammattaṃ, imaṃ kho, bhikkhave, puggalaṃ paṭicca dhammadesanā anuññātā.}}\\
\begin{addmargin}[1em]{2em}
\setstretch{.5}
{\PaliGlossB{Now, it’s for the sake of this last person that teaching the Dhamma is prescribed, that is, the one who can enter the sure path with regards to skillful qualities, but only if they get to see a Realized One, and to hear the teaching and training that he proclaims.}}\\
\end{addmargin}
\end{absolutelynopagebreak}

\begin{absolutelynopagebreak}
\setstretch{.7}
{\PaliGlossA{imañca pana, bhikkhave, puggalaṃ paṭicca aññesampi dhammo desetabbo.}}\\
\begin{addmargin}[1em]{2em}
\setstretch{.5}
{\PaliGlossB{But also, for the sake of this person, the other people should be taught Dhamma.}}\\
\end{addmargin}
\end{absolutelynopagebreak}

\begin{absolutelynopagebreak}
\setstretch{.7}
{\PaliGlossA{ime kho, bhikkhave, tayo gilānūpamā puggalā santo saṃvijjamānā lokasmin”ti.}}\\
\begin{addmargin}[1em]{2em}
\setstretch{.5}
{\PaliGlossB{These are the three people similar to patients found in the world.”}}\\
\end{addmargin}
\end{absolutelynopagebreak}

\begin{absolutelynopagebreak}
\setstretch{.7}
{\PaliGlossA{dutiyaṃ.}}\\
\begin{addmargin}[1em]{2em}
\setstretch{.5}
{\PaliGlossB{    -}}\\
\end{addmargin}
\end{absolutelynopagebreak}
