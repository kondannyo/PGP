
\begin{absolutelynopagebreak}
\setstretch{.7}
{\PaliGlossA{aṅguttara nikāya 3}}\\
\begin{addmargin}[1em]{2em}
\setstretch{.5}
{\PaliGlossB{Numbered Discourses 3}}\\
\end{addmargin}
\end{absolutelynopagebreak}

\begin{absolutelynopagebreak}
\setstretch{.7}
{\PaliGlossA{2. rathakāravagga}}\\
\begin{addmargin}[1em]{2em}
\setstretch{.5}
{\PaliGlossB{2. The Chariot-maker}}\\
\end{addmargin}
\end{absolutelynopagebreak}

\begin{absolutelynopagebreak}
\setstretch{.7}
{\PaliGlossA{11. ñātasutta}}\\
\begin{addmargin}[1em]{2em}
\setstretch{.5}
{\PaliGlossB{11. Well-known}}\\
\end{addmargin}
\end{absolutelynopagebreak}

\begin{absolutelynopagebreak}
\setstretch{.7}
{\PaliGlossA{“tīhi, bhikkhave, dhammehi samannāgato ñāto bhikkhu bahujanaahitāya paṭipanno hoti bahujanadukkhāya, bahuno janassa anatthāya ahitāya dukkhāya devamanussānaṃ.}}\\
\begin{addmargin}[1em]{2em}
\setstretch{.5}
{\PaliGlossB{“Mendicants, a well-known mendicant who has three qualities is acting for the hurt and unhappiness of the people, for the harm, hurt, and suffering of gods and humans.}}\\
\end{addmargin}
\end{absolutelynopagebreak}

\begin{absolutelynopagebreak}
\setstretch{.7}
{\PaliGlossA{katamehi tīhi?}}\\
\begin{addmargin}[1em]{2em}
\setstretch{.5}
{\PaliGlossB{What three?}}\\
\end{addmargin}
\end{absolutelynopagebreak}

\begin{absolutelynopagebreak}
\setstretch{.7}
{\PaliGlossA{ananulomike kāyakamme samādapeti, ananulomike vacīkamme samādapeti, ananulomikesu dhammesu samādapeti.}}\\
\begin{addmargin}[1em]{2em}
\setstretch{.5}
{\PaliGlossB{They encourage deeds of body and speech, as well as principles, that don’t reinforce good qualities.}}\\
\end{addmargin}
\end{absolutelynopagebreak}

\begin{absolutelynopagebreak}
\setstretch{.7}
{\PaliGlossA{imehi kho, bhikkhave, tīhi dhammehi samannāgato ñāto bhikkhu bahujanaahitāya paṭipanno hoti bahujanadukkhāya, bahuno janassa anatthāya ahitāya dukkhāya devamanussānaṃ.}}\\
\begin{addmargin}[1em]{2em}
\setstretch{.5}
{\PaliGlossB{A well-known mendicant who has these three qualities is acting for the hurt and unhappiness of the people, for the harm, hurt, and suffering of gods and humans.}}\\
\end{addmargin}
\end{absolutelynopagebreak}

\begin{absolutelynopagebreak}
\setstretch{.7}
{\PaliGlossA{tīhi, bhikkhave, dhammehi samannāgato ñāto bhikkhu bahujanahitāya paṭipanno hoti bahujanasukhāya, bahuno janassa atthāya hitāya sukhāya devamanussānaṃ.}}\\
\begin{addmargin}[1em]{2em}
\setstretch{.5}
{\PaliGlossB{A well-known mendicant who has three qualities is acting for the welfare and happiness of the people, for the benefit, welfare, and happiness of gods and humans.}}\\
\end{addmargin}
\end{absolutelynopagebreak}

\begin{absolutelynopagebreak}
\setstretch{.7}
{\PaliGlossA{katamehi tīhi?}}\\
\begin{addmargin}[1em]{2em}
\setstretch{.5}
{\PaliGlossB{What three?}}\\
\end{addmargin}
\end{absolutelynopagebreak}

\begin{absolutelynopagebreak}
\setstretch{.7}
{\PaliGlossA{anulomike kāyakamme samādapeti, anulomike vacīkamme samādapeti, anulomikesu dhammesu samādapeti.}}\\
\begin{addmargin}[1em]{2em}
\setstretch{.5}
{\PaliGlossB{They encourage deeds of body and speech, as well as principles, that reinforce good qualities.}}\\
\end{addmargin}
\end{absolutelynopagebreak}

\begin{absolutelynopagebreak}
\setstretch{.7}
{\PaliGlossA{imehi kho, bhikkhave, tīhi dhammehi samannāgato ñāto bhikkhu bahujanahitāya paṭipanno hoti bahujanasukhāya, bahuno janassa atthāya hitāya sukhāya devamanussānan”ti.}}\\
\begin{addmargin}[1em]{2em}
\setstretch{.5}
{\PaliGlossB{A well-known mendicant who has these three qualities is acting for the welfare and happiness of the people, for the benefit, welfare, and happiness of gods and humans.”}}\\
\end{addmargin}
\end{absolutelynopagebreak}

\begin{absolutelynopagebreak}
\setstretch{.7}
{\PaliGlossA{paṭhamaṃ.}}\\
\begin{addmargin}[1em]{2em}
\setstretch{.5}
{\PaliGlossB{    -}}\\
\end{addmargin}
\end{absolutelynopagebreak}
