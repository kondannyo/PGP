
\begin{absolutelynopagebreak}
\setstretch{.7}
{\PaliGlossA{aṅguttara nikāya 10}}\\
\begin{addmargin}[1em]{2em}
\setstretch{.5}
{\PaliGlossB{Numbered Discourses 10}}\\
\end{addmargin}
\end{absolutelynopagebreak}

\begin{absolutelynopagebreak}
\setstretch{.7}
{\PaliGlossA{5. akkosavagga}}\\
\begin{addmargin}[1em]{2em}
\setstretch{.5}
{\PaliGlossB{5. Abuse}}\\
\end{addmargin}
\end{absolutelynopagebreak}

\begin{absolutelynopagebreak}
\setstretch{.7}
{\PaliGlossA{41. vivādasutta}}\\
\begin{addmargin}[1em]{2em}
\setstretch{.5}
{\PaliGlossB{41. Arguments}}\\
\end{addmargin}
\end{absolutelynopagebreak}

\begin{absolutelynopagebreak}
\setstretch{.7}
{\PaliGlossA{atha kho āyasmā upāli yena bhagavā tenupasaṅkami; upasaṅkamitvā bhagavantaṃ abhivādetvā ekamantaṃ nisīdi. ekamantaṃ nisinno kho āyasmā upāli bhagavantaṃ etadavoca:}}\\
\begin{addmargin}[1em]{2em}
\setstretch{.5}
{\PaliGlossB{Then Venerable Upāli went up to the Buddha, bowed, sat down to one side, and said to him:}}\\
\end{addmargin}
\end{absolutelynopagebreak}

\begin{absolutelynopagebreak}
\setstretch{.7}
{\PaliGlossA{“ko nu kho, bhante, hetu ko paccayo, yena saṃghe bhaṇḍanakalahaviggahavivādā uppajjanti, bhikkhū ca na phāsu viharantī”ti?}}\\
\begin{addmargin}[1em]{2em}
\setstretch{.5}
{\PaliGlossB{“What is the cause, sir, what is the reason, why arguments, quarrels, and disputes arise in the Saṅgha, and the mendicants don’t live happily?”}}\\
\end{addmargin}
\end{absolutelynopagebreak}

\begin{absolutelynopagebreak}
\setstretch{.7}
{\PaliGlossA{“idhupāli, bhikkhū adhammaṃ dhammoti dīpenti, dhammaṃ adhammoti dīpenti, avinayaṃ vinayoti dīpenti, vinayaṃ avinayoti dīpenti, abhāsitaṃ alapitaṃ tathāgatena bhāsitaṃ lapitaṃ tathāgatenāti dīpenti, bhāsitaṃ lapitaṃ tathāgatena abhāsitaṃ alapitaṃ tathāgatenāti dīpenti, anāciṇṇaṃ tathāgatena āciṇṇaṃ tathāgatenāti dīpenti, āciṇṇaṃ tathāgatena anāciṇṇaṃ tathāgatenāti dīpenti, apaññattaṃ tathāgatena paññattaṃ tathāgatenāti dīpenti, paññattaṃ tathāgatena apaññattaṃ tathāgatenāti dīpenti.}}\\
\begin{addmargin}[1em]{2em}
\setstretch{.5}
{\PaliGlossB{“Upāli, it’s when a mendicant explains what is not the teaching as the teaching, and what is the teaching as not the teaching. They explain what is not the training as the training, and what is the training as not the training. They explain what was not spoken and stated by the Realized One as spoken and stated by the Realized One, and what was spoken and stated by the Realized One as not spoken and stated by the Realized One. They explain what was not practiced by the Realized One as practiced by the Realized One, and what was practiced by the Realized One as not practiced by the Realized One. They explain what was not prescribed by the Realized One as prescribed by the Realized One, and what was prescribed by the Realized One as not prescribed by the Realized One.}}\\
\end{addmargin}
\end{absolutelynopagebreak}

\begin{absolutelynopagebreak}
\setstretch{.7}
{\PaliGlossA{ayaṃ kho, upāli, hetu ayaṃ paccayo, yena saṃghe bhaṇḍanakalahaviggahavivādā uppajjanti, bhikkhū ca na phāsu viharantī”ti.}}\\
\begin{addmargin}[1em]{2em}
\setstretch{.5}
{\PaliGlossB{This is the cause, this is the reason why arguments, quarrels, and disputes arise in the Saṅgha, and the mendicants don’t live happily.”}}\\
\end{addmargin}
\end{absolutelynopagebreak}

\begin{absolutelynopagebreak}
\setstretch{.7}
{\PaliGlossA{paṭhamaṃ.}}\\
\begin{addmargin}[1em]{2em}
\setstretch{.5}
{\PaliGlossB{    -}}\\
\end{addmargin}
\end{absolutelynopagebreak}
