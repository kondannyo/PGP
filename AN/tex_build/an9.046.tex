
\begin{absolutelynopagebreak}
\setstretch{.7}
{\PaliGlossA{aṅguttara nikāya 9}}\\
\begin{addmargin}[1em]{2em}
\setstretch{.5}
{\PaliGlossB{Numbered Discourses 9}}\\
\end{addmargin}
\end{absolutelynopagebreak}

\begin{absolutelynopagebreak}
\setstretch{.7}
{\PaliGlossA{5. sāmaññavagga}}\\
\begin{addmargin}[1em]{2em}
\setstretch{.5}
{\PaliGlossB{5. Similarity}}\\
\end{addmargin}
\end{absolutelynopagebreak}

\begin{absolutelynopagebreak}
\setstretch{.7}
{\PaliGlossA{46. sandiṭṭhikadhammasutta}}\\
\begin{addmargin}[1em]{2em}
\setstretch{.5}
{\PaliGlossB{46. In This Very Life}}\\
\end{addmargin}
\end{absolutelynopagebreak}

\begin{absolutelynopagebreak}
\setstretch{.7}
{\PaliGlossA{“‘sandiṭṭhiko dhammo, sandiṭṭhiko dhammo’ti, āvuso, vuccati.}}\\
\begin{addmargin}[1em]{2em}
\setstretch{.5}
{\PaliGlossB{“Reverend, they speak of ‘a teaching visible in this very life’.}}\\
\end{addmargin}
\end{absolutelynopagebreak}

\begin{absolutelynopagebreak}
\setstretch{.7}
{\PaliGlossA{kittāvatā nu kho, āvuso, sandiṭṭhiko dhammo vutto bhagavatā”ti?}}\\
\begin{addmargin}[1em]{2em}
\setstretch{.5}
{\PaliGlossB{In what way did the Buddha speak of a teaching visible in this very life?”}}\\
\end{addmargin}
\end{absolutelynopagebreak}

\begin{absolutelynopagebreak}
\setstretch{.7}
{\PaliGlossA{“idhāvuso, bhikkhu vivicceva kāmehi … pe … paṭhamaṃ jhānaṃ upasampajja viharati.}}\\
\begin{addmargin}[1em]{2em}
\setstretch{.5}
{\PaliGlossB{“First, take a mendicant who, quite secluded from sensual pleasures … enters and remains in the first absorption.}}\\
\end{addmargin}
\end{absolutelynopagebreak}

\begin{absolutelynopagebreak}
\setstretch{.7}
{\PaliGlossA{ettāvatāpi kho, āvuso, sandiṭṭhiko dhammo vutto bhagavatā pariyāyena … pe ….}}\\
\begin{addmargin}[1em]{2em}
\setstretch{.5}
{\PaliGlossB{To this extent the Buddha spoke of the teaching visible in this very life in a qualified sense. …}}\\
\end{addmargin}
\end{absolutelynopagebreak}

\begin{absolutelynopagebreak}
\setstretch{.7}
{\PaliGlossA{puna caparaṃ, āvuso, bhikkhu sabbaso nevasaññānāsaññāyatanaṃ samatikkamma saññāvedayitanirodhaṃ upasampajja viharati, paññāya cassa disvā āsavā parikkhīṇā honti.}}\\
\begin{addmargin}[1em]{2em}
\setstretch{.5}
{\PaliGlossB{Furthermore, take a mendicant who, going totally beyond the dimension of neither perception nor non-perception, enters and remains in the cessation of perception and feeling. And, having seen with wisdom, their defilements come to an end.}}\\
\end{addmargin}
\end{absolutelynopagebreak}

\begin{absolutelynopagebreak}
\setstretch{.7}
{\PaliGlossA{ettāvatāpi kho, āvuso, sandiṭṭhiko dhammo vutto bhagavatā nippariyāyenā”ti.}}\\
\begin{addmargin}[1em]{2em}
\setstretch{.5}
{\PaliGlossB{To this extent the Buddha spoke of the teaching visible in this very life in a definitive sense.”}}\\
\end{addmargin}
\end{absolutelynopagebreak}

\begin{absolutelynopagebreak}
\setstretch{.7}
{\PaliGlossA{pañcamaṃ.}}\\
\begin{addmargin}[1em]{2em}
\setstretch{.5}
{\PaliGlossB{    -}}\\
\end{addmargin}
\end{absolutelynopagebreak}
