
\begin{absolutelynopagebreak}
\setstretch{.7}
{\PaliGlossA{aṅguttara nikāya 6}}\\
\begin{addmargin}[1em]{2em}
\setstretch{.5}
{\PaliGlossB{Numbered Discourses 6}}\\
\end{addmargin}
\end{absolutelynopagebreak}

\begin{absolutelynopagebreak}
\setstretch{.7}
{\PaliGlossA{10. ānisaṃsavagga}}\\
\begin{addmargin}[1em]{2em}
\setstretch{.5}
{\PaliGlossB{10. Benefit}}\\
\end{addmargin}
\end{absolutelynopagebreak}

\begin{absolutelynopagebreak}
\setstretch{.7}
{\PaliGlossA{105. bhavasutta}}\\
\begin{addmargin}[1em]{2em}
\setstretch{.5}
{\PaliGlossB{105. States of Existence}}\\
\end{addmargin}
\end{absolutelynopagebreak}

\begin{absolutelynopagebreak}
\setstretch{.7}
{\PaliGlossA{“tayome, bhikkhave, bhavā pahātabbā, tīsu sikkhāsu sikkhitabbaṃ.}}\\
\begin{addmargin}[1em]{2em}
\setstretch{.5}
{\PaliGlossB{“Mendicants, you should give up these three states of existence. And you should train in three trainings.}}\\
\end{addmargin}
\end{absolutelynopagebreak}

\begin{absolutelynopagebreak}
\setstretch{.7}
{\PaliGlossA{katame tayo bhavā pahātabbā?}}\\
\begin{addmargin}[1em]{2em}
\setstretch{.5}
{\PaliGlossB{What are the three states of existence you should give up?}}\\
\end{addmargin}
\end{absolutelynopagebreak}

\begin{absolutelynopagebreak}
\setstretch{.7}
{\PaliGlossA{kāmabhavo, rūpabhavo, arūpabhavo—}}\\
\begin{addmargin}[1em]{2em}
\setstretch{.5}
{\PaliGlossB{Existence in the sensual realm, the realm of luminous form, and the formless realm.}}\\
\end{addmargin}
\end{absolutelynopagebreak}

\begin{absolutelynopagebreak}
\setstretch{.7}
{\PaliGlossA{ime tayo bhavā pahātabbā.}}\\
\begin{addmargin}[1em]{2em}
\setstretch{.5}
{\PaliGlossB{These are the three states of existence you should give up.}}\\
\end{addmargin}
\end{absolutelynopagebreak}

\begin{absolutelynopagebreak}
\setstretch{.7}
{\PaliGlossA{katamāsu tīsu sikkhāsu sikkhitabbaṃ?}}\\
\begin{addmargin}[1em]{2em}
\setstretch{.5}
{\PaliGlossB{What are the three trainings you should train in?}}\\
\end{addmargin}
\end{absolutelynopagebreak}

\begin{absolutelynopagebreak}
\setstretch{.7}
{\PaliGlossA{adhisīlasikkhāya, adhicittasikkhāya, adhipaññāsikkhāya—}}\\
\begin{addmargin}[1em]{2em}
\setstretch{.5}
{\PaliGlossB{The training in the higher ethics, the higher mind, and the higher wisdom.}}\\
\end{addmargin}
\end{absolutelynopagebreak}

\begin{absolutelynopagebreak}
\setstretch{.7}
{\PaliGlossA{imāsu tīsu sikkhāsu sikkhitabbaṃ.}}\\
\begin{addmargin}[1em]{2em}
\setstretch{.5}
{\PaliGlossB{These are the three trainings you should train in.}}\\
\end{addmargin}
\end{absolutelynopagebreak}

\begin{absolutelynopagebreak}
\setstretch{.7}
{\PaliGlossA{yato kho, bhikkhave, bhikkhuno ime tayo bhavā pahīnā honti, imāsu ca tīsu sikkhāsu sikkhitasikkho hoti—}}\\
\begin{addmargin}[1em]{2em}
\setstretch{.5}
{\PaliGlossB{When a mendicant has given up these three states of existence and has trained in these three trainings}}\\
\end{addmargin}
\end{absolutelynopagebreak}

\begin{absolutelynopagebreak}
\setstretch{.7}
{\PaliGlossA{ayaṃ vuccati, bhikkhave, bhikkhu acchecchi taṇhaṃ, vivattayi saṃyojanaṃ, sammā mānābhisamayā antamakāsi dukkhassā”ti.}}\\
\begin{addmargin}[1em]{2em}
\setstretch{.5}
{\PaliGlossB{they’re called a mendicant who has cut off craving, untied the fetters, and by rightly comprehending conceit has made an end of suffering.”}}\\
\end{addmargin}
\end{absolutelynopagebreak}

\begin{absolutelynopagebreak}
\setstretch{.7}
{\PaliGlossA{dasamaṃ.}}\\
\begin{addmargin}[1em]{2em}
\setstretch{.5}
{\PaliGlossB{    -}}\\
\end{addmargin}
\end{absolutelynopagebreak}
