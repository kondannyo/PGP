
\begin{absolutelynopagebreak}
\setstretch{.7}
{\PaliGlossA{aṅguttara nikāya 5}}\\
\begin{addmargin}[1em]{2em}
\setstretch{.5}
{\PaliGlossB{Numbered Discourses 5}}\\
\end{addmargin}
\end{absolutelynopagebreak}

\begin{absolutelynopagebreak}
\setstretch{.7}
{\PaliGlossA{1. sekhabalavagga}}\\
\begin{addmargin}[1em]{2em}
\setstretch{.5}
{\PaliGlossB{1. Powers of a Trainee}}\\
\end{addmargin}
\end{absolutelynopagebreak}

\begin{absolutelynopagebreak}
\setstretch{.7}
{\PaliGlossA{3. dukkhasutta}}\\
\begin{addmargin}[1em]{2em}
\setstretch{.5}
{\PaliGlossB{3. Suffering}}\\
\end{addmargin}
\end{absolutelynopagebreak}

\begin{absolutelynopagebreak}
\setstretch{.7}
{\PaliGlossA{“pañcahi, bhikkhave, dhammehi samannāgato bhikkhu diṭṭheva dhamme dukkhaṃ viharati savighātaṃ saupāyāsaṃ sapariḷāhaṃ, kāyassa ca bhedā paraṃ maraṇā duggati pāṭikaṅkhā.}}\\
\begin{addmargin}[1em]{2em}
\setstretch{.5}
{\PaliGlossB{“Mendicants, when a mendicant has five qualities they live unhappily in the present life—with distress, anguish, and fever—and when the body breaks up, after death, they can expect a bad rebirth.}}\\
\end{addmargin}
\end{absolutelynopagebreak}

\begin{absolutelynopagebreak}
\setstretch{.7}
{\PaliGlossA{katamehi pañcahi?}}\\
\begin{addmargin}[1em]{2em}
\setstretch{.5}
{\PaliGlossB{What five?}}\\
\end{addmargin}
\end{absolutelynopagebreak}

\begin{absolutelynopagebreak}
\setstretch{.7}
{\PaliGlossA{idha, bhikkhave, bhikkhu assaddho hoti, ahiriko hoti, anottappī hoti, kusīto hoti, duppañño hoti.}}\\
\begin{addmargin}[1em]{2em}
\setstretch{.5}
{\PaliGlossB{It’s when a mendicant is faithless, shameless, imprudent, lazy, and witless.}}\\
\end{addmargin}
\end{absolutelynopagebreak}

\begin{absolutelynopagebreak}
\setstretch{.7}
{\PaliGlossA{imehi kho, bhikkhave, pañcahi dhammehi samannāgato bhikkhu diṭṭheva dhamme dukkhaṃ viharati savighātaṃ saupāyāsaṃ sapariḷāhaṃ, kāyassa ca bhedā paraṃ maraṇā duggati pāṭikaṅkhā.}}\\
\begin{addmargin}[1em]{2em}
\setstretch{.5}
{\PaliGlossB{When a mendicant has these five qualities they live unhappily in the present life—with distress, anguish, and fever—and when the body breaks up, after death, they can expect a bad rebirth.}}\\
\end{addmargin}
\end{absolutelynopagebreak}

\begin{absolutelynopagebreak}
\setstretch{.7}
{\PaliGlossA{pañcahi, bhikkhave, dhammehi samannāgato bhikkhu diṭṭheva dhamme sukhaṃ viharati avighātaṃ anupāyāsaṃ apariḷāhaṃ, kāyassa ca bhedā paraṃ maraṇā sugati pāṭikaṅkhā.}}\\
\begin{addmargin}[1em]{2em}
\setstretch{.5}
{\PaliGlossB{When a mendicant has five qualities they live happily in the present life—without distress, anguish, or fever—and when the body breaks up, after death, they can expect a good rebirth.}}\\
\end{addmargin}
\end{absolutelynopagebreak}

\begin{absolutelynopagebreak}
\setstretch{.7}
{\PaliGlossA{katamehi pañcahi?}}\\
\begin{addmargin}[1em]{2em}
\setstretch{.5}
{\PaliGlossB{What five?}}\\
\end{addmargin}
\end{absolutelynopagebreak}

\begin{absolutelynopagebreak}
\setstretch{.7}
{\PaliGlossA{idha, bhikkhave, bhikkhu saddho hoti, hirīmā hoti, ottappī hoti, āraddhavīriyo hoti, paññavā hoti.}}\\
\begin{addmargin}[1em]{2em}
\setstretch{.5}
{\PaliGlossB{It’s when a mendicant is faithful, conscientious, prudent, energetic, and wise.}}\\
\end{addmargin}
\end{absolutelynopagebreak}

\begin{absolutelynopagebreak}
\setstretch{.7}
{\PaliGlossA{imehi kho, bhikkhave, pañcahi dhammehi samannāgato bhikkhu diṭṭheva dhamme sukhaṃ viharati avighātaṃ anupāyāsaṃ apariḷāhaṃ, kāyassa ca bhedā paraṃ maraṇā sugati pāṭikaṅkhā”ti.}}\\
\begin{addmargin}[1em]{2em}
\setstretch{.5}
{\PaliGlossB{When a mendicant has these five qualities they live happily in the present life—without distress, anguish, or fever—and when the body breaks up, after death, they can expect a good rebirth.”}}\\
\end{addmargin}
\end{absolutelynopagebreak}

\begin{absolutelynopagebreak}
\setstretch{.7}
{\PaliGlossA{tatiyaṃ.}}\\
\begin{addmargin}[1em]{2em}
\setstretch{.5}
{\PaliGlossB{    -}}\\
\end{addmargin}
\end{absolutelynopagebreak}
