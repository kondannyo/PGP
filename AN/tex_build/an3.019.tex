
\begin{absolutelynopagebreak}
\setstretch{.7}
{\PaliGlossA{aṅguttara nikāya 3}}\\
\begin{addmargin}[1em]{2em}
\setstretch{.5}
{\PaliGlossB{Numbered Discourses 3}}\\
\end{addmargin}
\end{absolutelynopagebreak}

\begin{absolutelynopagebreak}
\setstretch{.7}
{\PaliGlossA{2. rathakāravagga}}\\
\begin{addmargin}[1em]{2em}
\setstretch{.5}
{\PaliGlossB{2. The Chariot-maker}}\\
\end{addmargin}
\end{absolutelynopagebreak}

\begin{absolutelynopagebreak}
\setstretch{.7}
{\PaliGlossA{19. paṭhamapāpaṇikasutta}}\\
\begin{addmargin}[1em]{2em}
\setstretch{.5}
{\PaliGlossB{19. A Shopkeeper (1st)}}\\
\end{addmargin}
\end{absolutelynopagebreak}

\begin{absolutelynopagebreak}
\setstretch{.7}
{\PaliGlossA{“tīhi, bhikkhave, aṅgehi samannāgato pāpaṇiko abhabbo anadhigataṃ vā bhogaṃ adhigantuṃ, adhigataṃ vā bhogaṃ phātiṃ kātuṃ.}}\\
\begin{addmargin}[1em]{2em}
\setstretch{.5}
{\PaliGlossB{“Mendicants, a shopkeeper who has three factors is unable to acquire more wealth or to increase the wealth they’ve already acquired.}}\\
\end{addmargin}
\end{absolutelynopagebreak}

\begin{absolutelynopagebreak}
\setstretch{.7}
{\PaliGlossA{katamehi tīhi?}}\\
\begin{addmargin}[1em]{2em}
\setstretch{.5}
{\PaliGlossB{What three?}}\\
\end{addmargin}
\end{absolutelynopagebreak}

\begin{absolutelynopagebreak}
\setstretch{.7}
{\PaliGlossA{idha, bhikkhave, pāpaṇiko pubbaṇhasamayaṃ na sakkaccaṃ kammantaṃ adhiṭṭhāti, majjhanhikasamayaṃ na sakkaccaṃ kammantaṃ adhiṭṭhāti, sāyanhasamayaṃ na sakkaccaṃ kammantaṃ adhiṭṭhāti.}}\\
\begin{addmargin}[1em]{2em}
\setstretch{.5}
{\PaliGlossB{It’s when a shopkeeper doesn’t carefully apply themselves to their work in the morning, at midday, and in the afternoon.}}\\
\end{addmargin}
\end{absolutelynopagebreak}

\begin{absolutelynopagebreak}
\setstretch{.7}
{\PaliGlossA{imehi kho, bhikkhave, tīhi aṅgehi samannāgato pāpaṇiko abhabbo anadhigataṃ vā bhogaṃ adhigantuṃ, adhigataṃ vā bhogaṃ phātiṃ kātuṃ.}}\\
\begin{addmargin}[1em]{2em}
\setstretch{.5}
{\PaliGlossB{A shopkeeper who has these three factors is unable to acquire more wealth or to increase the wealth they’ve already acquired.}}\\
\end{addmargin}
\end{absolutelynopagebreak}

\begin{absolutelynopagebreak}
\setstretch{.7}
{\PaliGlossA{evamevaṃ kho, bhikkhave, tīhi dhammehi samannāgato bhikkhu abhabbo anadhigataṃ vā kusalaṃ dhammaṃ adhigantuṃ, adhigataṃ vā kusalaṃ dhammaṃ phātiṃ kātuṃ.}}\\
\begin{addmargin}[1em]{2em}
\setstretch{.5}
{\PaliGlossB{In the same way, a mendicant who has three factors is unable to acquire more skillful qualities or to increase the skillful qualities they’ve already acquired.}}\\
\end{addmargin}
\end{absolutelynopagebreak}

\begin{absolutelynopagebreak}
\setstretch{.7}
{\PaliGlossA{katamehi tīhi?}}\\
\begin{addmargin}[1em]{2em}
\setstretch{.5}
{\PaliGlossB{What three?}}\\
\end{addmargin}
\end{absolutelynopagebreak}

\begin{absolutelynopagebreak}
\setstretch{.7}
{\PaliGlossA{idha, bhikkhave, bhikkhu pubbaṇhasamayaṃ na sakkaccaṃ samādhinimittaṃ adhiṭṭhāti, majjhanhikasamayaṃ na sakkaccaṃ samādhinimittaṃ adhiṭṭhāti, sāyanhasamayaṃ na sakkaccaṃ samādhinimittaṃ adhiṭṭhāti.}}\\
\begin{addmargin}[1em]{2em}
\setstretch{.5}
{\PaliGlossB{It’s when a mendicant doesn’t carefully apply themselves to a meditation subject as a foundation of immersion in the morning, at midday, and in the afternoon.}}\\
\end{addmargin}
\end{absolutelynopagebreak}

\begin{absolutelynopagebreak}
\setstretch{.7}
{\PaliGlossA{imehi kho, bhikkhave, tīhi dhammehi samannāgato bhikkhu abhabbo anadhigataṃ vā kusalaṃ dhammaṃ adhigantuṃ, adhigataṃ vā kusalaṃ dhammaṃ phātiṃ kātuṃ.}}\\
\begin{addmargin}[1em]{2em}
\setstretch{.5}
{\PaliGlossB{A mendicant who has these three factors is unable to acquire more skillful qualities or to increase the skillful qualities they’ve already acquired.}}\\
\end{addmargin}
\end{absolutelynopagebreak}

\begin{absolutelynopagebreak}
\setstretch{.7}
{\PaliGlossA{tīhi, bhikkhave, aṅgehi samannāgato pāpaṇiko bhabbo anadhigataṃ vā bhogaṃ adhigantuṃ, adhigataṃ vā bhogaṃ phātiṃ kātuṃ.}}\\
\begin{addmargin}[1em]{2em}
\setstretch{.5}
{\PaliGlossB{A shopkeeper who has three factors is able to acquire more wealth or to increase the wealth they’ve already acquired.}}\\
\end{addmargin}
\end{absolutelynopagebreak}

\begin{absolutelynopagebreak}
\setstretch{.7}
{\PaliGlossA{katamehi tīhi?}}\\
\begin{addmargin}[1em]{2em}
\setstretch{.5}
{\PaliGlossB{What three?}}\\
\end{addmargin}
\end{absolutelynopagebreak}

\begin{absolutelynopagebreak}
\setstretch{.7}
{\PaliGlossA{idha, bhikkhave, pāpaṇiko pubbaṇhasamayaṃ sakkaccaṃ kammantaṃ adhiṭṭhāti, majjhanhikasamayaṃ … pe … sāyanhasamayaṃ sakkaccaṃ kammantaṃ adhiṭṭhāti.}}\\
\begin{addmargin}[1em]{2em}
\setstretch{.5}
{\PaliGlossB{It’s when a shopkeeper carefully applies themselves to their work in the morning, at midday, and in the afternoon.}}\\
\end{addmargin}
\end{absolutelynopagebreak}

\begin{absolutelynopagebreak}
\setstretch{.7}
{\PaliGlossA{imehi kho, bhikkhave, tīhi aṅgehi samannāgato pāpaṇiko bhabbo anadhigataṃ vā bhogaṃ adhigantuṃ, adhigataṃ vā bhogaṃ phātiṃ kātuṃ.}}\\
\begin{addmargin}[1em]{2em}
\setstretch{.5}
{\PaliGlossB{A shopkeeper who has these three factors is able to acquire more wealth or to increase the wealth they’ve already acquired.}}\\
\end{addmargin}
\end{absolutelynopagebreak}

\begin{absolutelynopagebreak}
\setstretch{.7}
{\PaliGlossA{evamevaṃ kho, bhikkhave, tīhi dhammehi samannāgato bhikkhu bhabbo anadhigataṃ vā kusalaṃ dhammaṃ adhigantuṃ, adhigataṃ vā kusalaṃ dhammaṃ phātiṃ kātuṃ.}}\\
\begin{addmargin}[1em]{2em}
\setstretch{.5}
{\PaliGlossB{In the same way, a mendicant who has three factors is able to acquire more skillful qualities or to increase the skillful qualities they’ve already acquired.}}\\
\end{addmargin}
\end{absolutelynopagebreak}

\begin{absolutelynopagebreak}
\setstretch{.7}
{\PaliGlossA{katamehi tīhi?}}\\
\begin{addmargin}[1em]{2em}
\setstretch{.5}
{\PaliGlossB{What three?}}\\
\end{addmargin}
\end{absolutelynopagebreak}

\begin{absolutelynopagebreak}
\setstretch{.7}
{\PaliGlossA{idha, bhikkhave, bhikkhu pubbaṇhasamayaṃ sakkaccaṃ samādhinimittaṃ adhiṭṭhāti, majjhanhikasamayaṃ … pe … sāyanhasamayaṃ sakkaccaṃ samādhinimittaṃ adhiṭṭhāti.}}\\
\begin{addmargin}[1em]{2em}
\setstretch{.5}
{\PaliGlossB{It’s when a mendicant carefully applies themselves to a meditation subject as a foundation of immersion in the morning, at midday, and in the afternoon.}}\\
\end{addmargin}
\end{absolutelynopagebreak}

\begin{absolutelynopagebreak}
\setstretch{.7}
{\PaliGlossA{imehi kho, bhikkhave, tīhi dhammehi samannāgato bhikkhu bhabbo anadhigataṃ vā kusalaṃ dhammaṃ adhigantuṃ, adhigataṃ vā kusalaṃ dhammaṃ phātiṃ kātun”ti.}}\\
\begin{addmargin}[1em]{2em}
\setstretch{.5}
{\PaliGlossB{A mendicant who has these three factors is able to acquire more skillful qualities or to increase the skillful qualities they’ve already acquired.”}}\\
\end{addmargin}
\end{absolutelynopagebreak}

\begin{absolutelynopagebreak}
\setstretch{.7}
{\PaliGlossA{navamaṃ.}}\\
\begin{addmargin}[1em]{2em}
\setstretch{.5}
{\PaliGlossB{    -}}\\
\end{addmargin}
\end{absolutelynopagebreak}
