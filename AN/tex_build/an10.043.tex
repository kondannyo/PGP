
\begin{absolutelynopagebreak}
\setstretch{.7}
{\PaliGlossA{aṅguttara nikāya 10}}\\
\begin{addmargin}[1em]{2em}
\setstretch{.5}
{\PaliGlossB{Numbered Discourses 10}}\\
\end{addmargin}
\end{absolutelynopagebreak}

\begin{absolutelynopagebreak}
\setstretch{.7}
{\PaliGlossA{5. akkosavagga}}\\
\begin{addmargin}[1em]{2em}
\setstretch{.5}
{\PaliGlossB{5. Abuse}}\\
\end{addmargin}
\end{absolutelynopagebreak}

\begin{absolutelynopagebreak}
\setstretch{.7}
{\PaliGlossA{43. dutiyavivādamūlasutta}}\\
\begin{addmargin}[1em]{2em}
\setstretch{.5}
{\PaliGlossB{43. Roots of Arguments (2nd)}}\\
\end{addmargin}
\end{absolutelynopagebreak}

\begin{absolutelynopagebreak}
\setstretch{.7}
{\PaliGlossA{“kati nu kho, bhante, vivādamūlānī”ti?}}\\
\begin{addmargin}[1em]{2em}
\setstretch{.5}
{\PaliGlossB{“Sir, how many roots of arguments are there?”}}\\
\end{addmargin}
\end{absolutelynopagebreak}

\begin{absolutelynopagebreak}
\setstretch{.7}
{\PaliGlossA{“dasa kho, upāli, vivādamūlāni.}}\\
\begin{addmargin}[1em]{2em}
\setstretch{.5}
{\PaliGlossB{“Upāli, there are ten roots of arguments.}}\\
\end{addmargin}
\end{absolutelynopagebreak}

\begin{absolutelynopagebreak}
\setstretch{.7}
{\PaliGlossA{katamāni dasa?}}\\
\begin{addmargin}[1em]{2em}
\setstretch{.5}
{\PaliGlossB{What ten?}}\\
\end{addmargin}
\end{absolutelynopagebreak}

\begin{absolutelynopagebreak}
\setstretch{.7}
{\PaliGlossA{idhupāli, bhikkhū anāpattiṃ āpattīti dīpenti, āpattiṃ anāpattīti dīpenti, lahukaṃ āpattiṃ garukāpattīti dīpenti, garukaṃ āpattiṃ lahukāpattīti dīpenti, duṭṭhullaṃ āpattiṃ aduṭṭhullāpattīti dīpenti, aduṭṭhullaṃ āpattiṃ duṭṭhullāpattīti dīpenti, sāvasesaṃ āpattiṃ anavasesāpattīti dīpenti, anavasesaṃ āpattiṃ sāvasesāpattīti dīpenti, sappaṭikammaṃ āpattiṃ appaṭikammāpattīti dīpenti, appaṭikammaṃ āpattiṃ sappaṭikammāpattīti dīpenti.}}\\
\begin{addmargin}[1em]{2em}
\setstretch{.5}
{\PaliGlossB{It’s when a mendicant explains what is not an offense as an offense, and what is an offense as not an offense. They explain a light offense as a serious offense, and a serious offense as a light offense. They explain an offense committed with corrupt intention as an offense not committed with corrupt intention, and an offense not committed with corrupt intention as an offense committed with corrupt intention. They explain an offense requiring rehabilitation as an offense not requiring rehabilitation, and an offense not requiring rehabilitation as an offense requiring rehabilitation. They explain an offense with redress as an offense without redress, and an offense without redress as an offense with redress.}}\\
\end{addmargin}
\end{absolutelynopagebreak}

\begin{absolutelynopagebreak}
\setstretch{.7}
{\PaliGlossA{imāni kho, upāli, dasa vivādamūlānī”ti.}}\\
\begin{addmargin}[1em]{2em}
\setstretch{.5}
{\PaliGlossB{These are the ten roots of arguments.”}}\\
\end{addmargin}
\end{absolutelynopagebreak}

\begin{absolutelynopagebreak}
\setstretch{.7}
{\PaliGlossA{tatiyaṃ.}}\\
\begin{addmargin}[1em]{2em}
\setstretch{.5}
{\PaliGlossB{    -}}\\
\end{addmargin}
\end{absolutelynopagebreak}
