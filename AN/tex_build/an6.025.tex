
\begin{absolutelynopagebreak}
\setstretch{.7}
{\PaliGlossA{aṅguttara nikāya 6}}\\
\begin{addmargin}[1em]{2em}
\setstretch{.5}
{\PaliGlossB{Numbered Discourses 6}}\\
\end{addmargin}
\end{absolutelynopagebreak}

\begin{absolutelynopagebreak}
\setstretch{.7}
{\PaliGlossA{3. anuttariyavagga}}\\
\begin{addmargin}[1em]{2em}
\setstretch{.5}
{\PaliGlossB{3. Unsurpassable}}\\
\end{addmargin}
\end{absolutelynopagebreak}

\begin{absolutelynopagebreak}
\setstretch{.7}
{\PaliGlossA{25. anussatiṭṭhānasutta}}\\
\begin{addmargin}[1em]{2em}
\setstretch{.5}
{\PaliGlossB{25. Topics for Recollection}}\\
\end{addmargin}
\end{absolutelynopagebreak}

\begin{absolutelynopagebreak}
\setstretch{.7}
{\PaliGlossA{“chayimāni, bhikkhave, anussatiṭṭhānāni.}}\\
\begin{addmargin}[1em]{2em}
\setstretch{.5}
{\PaliGlossB{“Mendicants, there are these six topics for recollection.}}\\
\end{addmargin}
\end{absolutelynopagebreak}

\begin{absolutelynopagebreak}
\setstretch{.7}
{\PaliGlossA{katamāni cha?}}\\
\begin{addmargin}[1em]{2em}
\setstretch{.5}
{\PaliGlossB{What six?}}\\
\end{addmargin}
\end{absolutelynopagebreak}

\begin{absolutelynopagebreak}
\setstretch{.7}
{\PaliGlossA{idha, bhikkhave, ariyasāvako tathāgataṃ anussarati:}}\\
\begin{addmargin}[1em]{2em}
\setstretch{.5}
{\PaliGlossB{Firstly, a noble disciple recollects the Realized One:}}\\
\end{addmargin}
\end{absolutelynopagebreak}

\begin{absolutelynopagebreak}
\setstretch{.7}
{\PaliGlossA{‘itipi so bhagavā … pe … satthā devamanussānaṃ buddho bhagavā’ti.}}\\
\begin{addmargin}[1em]{2em}
\setstretch{.5}
{\PaliGlossB{‘That Blessed One is perfected, a fully awakened Buddha, accomplished in knowledge and conduct, holy, knower of the world, supreme guide for those who wish to train, teacher of gods and humans, awakened, blessed.’}}\\
\end{addmargin}
\end{absolutelynopagebreak}

\begin{absolutelynopagebreak}
\setstretch{.7}
{\PaliGlossA{yasmiṃ, bhikkhave, samaye ariyasāvako tathāgataṃ anussarati, nevassa tasmiṃ samaye rāgapariyuṭṭhitaṃ cittaṃ hoti, na dosapariyuṭṭhitaṃ cittaṃ hoti, na mohapariyuṭṭhitaṃ cittaṃ hoti;}}\\
\begin{addmargin}[1em]{2em}
\setstretch{.5}
{\PaliGlossB{When a noble disciple recollects the Realized One their mind is not full of greed, hate, and delusion.}}\\
\end{addmargin}
\end{absolutelynopagebreak}

\begin{absolutelynopagebreak}
\setstretch{.7}
{\PaliGlossA{ujugatamevassa tasmiṃ samaye cittaṃ hoti, nikkhantaṃ muttaṃ vuṭṭhitaṃ gedhamhā.}}\\
\begin{addmargin}[1em]{2em}
\setstretch{.5}
{\PaliGlossB{At that time their mind is unswerving. They’ve left behind greed; they’re free of it and have risen above it.}}\\
\end{addmargin}
\end{absolutelynopagebreak}

\begin{absolutelynopagebreak}
\setstretch{.7}
{\PaliGlossA{‘gedho’ti kho, bhikkhave, pañcannetaṃ kāmaguṇānaṃ adhivacanaṃ.}}\\
\begin{addmargin}[1em]{2em}
\setstretch{.5}
{\PaliGlossB{‘Greed’ is a term for the five kinds of sensual stimulation.}}\\
\end{addmargin}
\end{absolutelynopagebreak}

\begin{absolutelynopagebreak}
\setstretch{.7}
{\PaliGlossA{idampi kho, bhikkhave, ārammaṇaṃ karitvā evam’idhekacce sattā visujjhanti. (1)}}\\
\begin{addmargin}[1em]{2em}
\setstretch{.5}
{\PaliGlossB{Relying on this some sentient beings are purified.}}\\
\end{addmargin}
\end{absolutelynopagebreak}

\begin{absolutelynopagebreak}
\setstretch{.7}
{\PaliGlossA{puna caparaṃ, bhikkhave, ariyasāvako dhammaṃ anussarati:}}\\
\begin{addmargin}[1em]{2em}
\setstretch{.5}
{\PaliGlossB{Furthermore, a noble disciple recollects the teaching:}}\\
\end{addmargin}
\end{absolutelynopagebreak}

\begin{absolutelynopagebreak}
\setstretch{.7}
{\PaliGlossA{‘svākkhāto bhagavatā dhammo … pe … paccattaṃ veditabbo viññūhī’ti.}}\\
\begin{addmargin}[1em]{2em}
\setstretch{.5}
{\PaliGlossB{‘The teaching is well explained by the Buddha—visible in this very life, immediately effective, inviting inspection, relevant, so that sensible people can know it for themselves.’}}\\
\end{addmargin}
\end{absolutelynopagebreak}

\begin{absolutelynopagebreak}
\setstretch{.7}
{\PaliGlossA{yasmiṃ, bhikkhave, samaye ariyasāvako dhammaṃ anussarati, nevassa tasmiṃ samaye rāgapariyuṭṭhitaṃ cittaṃ hoti, na dosapariyuṭṭhitaṃ cittaṃ hoti, na mohapariyuṭṭhitaṃ cittaṃ hoti;}}\\
\begin{addmargin}[1em]{2em}
\setstretch{.5}
{\PaliGlossB{When a noble disciple recollects the teaching their mind is not full of greed, hate, and delusion. …}}\\
\end{addmargin}
\end{absolutelynopagebreak}

\begin{absolutelynopagebreak}
\setstretch{.7}
{\PaliGlossA{ujugatamevassa tasmiṃ samaye cittaṃ hoti, nikkhantaṃ muttaṃ vuṭṭhitaṃ gedhamhā.}}\\
\begin{addmargin}[1em]{2em}
\setstretch{.5}
{\PaliGlossB{    -}}\\
\end{addmargin}
\end{absolutelynopagebreak}

\begin{absolutelynopagebreak}
\setstretch{.7}
{\PaliGlossA{‘gedho’ti kho, bhikkhave, pañcannetaṃ kāmaguṇānaṃ adhivacanaṃ.}}\\
\begin{addmargin}[1em]{2em}
\setstretch{.5}
{\PaliGlossB{    -}}\\
\end{addmargin}
\end{absolutelynopagebreak}

\begin{absolutelynopagebreak}
\setstretch{.7}
{\PaliGlossA{idampi kho, bhikkhave, ārammaṇaṃ karitvā evam’idhekacce sattā visujjhanti. (2)}}\\
\begin{addmargin}[1em]{2em}
\setstretch{.5}
{\PaliGlossB{    -}}\\
\end{addmargin}
\end{absolutelynopagebreak}

\begin{absolutelynopagebreak}
\setstretch{.7}
{\PaliGlossA{puna caparaṃ, bhikkhave, ariyasāvako saṅghaṃ anussarati:}}\\
\begin{addmargin}[1em]{2em}
\setstretch{.5}
{\PaliGlossB{Furthermore, a noble disciple recollects the Saṅgha:}}\\
\end{addmargin}
\end{absolutelynopagebreak}

\begin{absolutelynopagebreak}
\setstretch{.7}
{\PaliGlossA{‘suppaṭipanno bhagavato sāvakasaṅgho … pe … anuttaraṃ puññakkhettaṃ lokassā’ti.}}\\
\begin{addmargin}[1em]{2em}
\setstretch{.5}
{\PaliGlossB{‘The Saṅgha of the Buddha’s disciples is practicing the way that’s good, straightforward, methodical, and proper. It consists of the four pairs, the eight individuals. This is the Saṅgha of the Buddha’s disciples that is worthy of offerings dedicated to the gods, worthy of hospitality, worthy of a religious donation, worthy of greeting with joined palms, and is the supreme field of merit for the world.’}}\\
\end{addmargin}
\end{absolutelynopagebreak}

\begin{absolutelynopagebreak}
\setstretch{.7}
{\PaliGlossA{yasmiṃ, bhikkhave, samaye ariyasāvako saṅghaṃ anussarati, nevassa tasmiṃ samaye rāgapariyuṭṭhitaṃ cittaṃ hoti, na dosapariyuṭṭhitaṃ cittaṃ hoti, na mohapariyuṭṭhitaṃ cittaṃ hoti;}}\\
\begin{addmargin}[1em]{2em}
\setstretch{.5}
{\PaliGlossB{When a noble disciple recollects the Saṅgha their mind is not full of greed, hate, and delusion. …}}\\
\end{addmargin}
\end{absolutelynopagebreak}

\begin{absolutelynopagebreak}
\setstretch{.7}
{\PaliGlossA{ujugatamevassa tasmiṃ samaye cittaṃ hoti, nikkhantaṃ muttaṃ vuṭṭhitaṃ gedhamhā.}}\\
\begin{addmargin}[1em]{2em}
\setstretch{.5}
{\PaliGlossB{    -}}\\
\end{addmargin}
\end{absolutelynopagebreak}

\begin{absolutelynopagebreak}
\setstretch{.7}
{\PaliGlossA{‘gedho’ti kho, bhikkhave, pañcannetaṃ kāmaguṇānaṃ adhivacanaṃ.}}\\
\begin{addmargin}[1em]{2em}
\setstretch{.5}
{\PaliGlossB{    -}}\\
\end{addmargin}
\end{absolutelynopagebreak}

\begin{absolutelynopagebreak}
\setstretch{.7}
{\PaliGlossA{idampi kho, bhikkhave, ārammaṇaṃ karitvā evam’idhekacce sattā visujjhanti. (3)}}\\
\begin{addmargin}[1em]{2em}
\setstretch{.5}
{\PaliGlossB{    -}}\\
\end{addmargin}
\end{absolutelynopagebreak}

\begin{absolutelynopagebreak}
\setstretch{.7}
{\PaliGlossA{puna caparaṃ, bhikkhave, ariyasāvako attano sīlāni anussarati akhaṇḍāni … pe … samādhisaṃvattanikāni.}}\\
\begin{addmargin}[1em]{2em}
\setstretch{.5}
{\PaliGlossB{Furthermore, a noble disciple recollects their own ethical precepts, which are unbroken, impeccable, spotless, and unmarred, liberating, praised by sensible people, not mistaken, and leading to immersion.}}\\
\end{addmargin}
\end{absolutelynopagebreak}

\begin{absolutelynopagebreak}
\setstretch{.7}
{\PaliGlossA{yasmiṃ, bhikkhave, samaye ariyasāvako sīlaṃ anussarati, nevassa tasmiṃ samaye rāgapariyuṭṭhitaṃ cittaṃ hoti, na dosapariyuṭṭhitaṃ cittaṃ hoti, na mohapariyuṭṭhitaṃ cittaṃ hoti;}}\\
\begin{addmargin}[1em]{2em}
\setstretch{.5}
{\PaliGlossB{When a noble disciple recollects their ethical precepts their mind is not full of greed, hate, and delusion. …}}\\
\end{addmargin}
\end{absolutelynopagebreak}

\begin{absolutelynopagebreak}
\setstretch{.7}
{\PaliGlossA{ujugatamevassa tasmiṃ samaye cittaṃ hoti, nikkhantaṃ muttaṃ vuṭṭhitaṃ gedhamhā.}}\\
\begin{addmargin}[1em]{2em}
\setstretch{.5}
{\PaliGlossB{    -}}\\
\end{addmargin}
\end{absolutelynopagebreak}

\begin{absolutelynopagebreak}
\setstretch{.7}
{\PaliGlossA{‘gedho’ti kho, bhikkhave, pañcannetaṃ kāmaguṇānaṃ adhivacanaṃ.}}\\
\begin{addmargin}[1em]{2em}
\setstretch{.5}
{\PaliGlossB{    -}}\\
\end{addmargin}
\end{absolutelynopagebreak}

\begin{absolutelynopagebreak}
\setstretch{.7}
{\PaliGlossA{idampi kho, bhikkhave, ārammaṇaṃ karitvā evam’idhekacce sattā visujjhanti. (4)}}\\
\begin{addmargin}[1em]{2em}
\setstretch{.5}
{\PaliGlossB{    -}}\\
\end{addmargin}
\end{absolutelynopagebreak}

\begin{absolutelynopagebreak}
\setstretch{.7}
{\PaliGlossA{puna caparaṃ, bhikkhave, ariyasāvako attano cāgaṃ anussarati:}}\\
\begin{addmargin}[1em]{2em}
\setstretch{.5}
{\PaliGlossB{Furthermore, a noble disciple recollects their own generosity:}}\\
\end{addmargin}
\end{absolutelynopagebreak}

\begin{absolutelynopagebreak}
\setstretch{.7}
{\PaliGlossA{‘lābhā vata me. suladdhaṃ vata me … pe …}}\\
\begin{addmargin}[1em]{2em}
\setstretch{.5}
{\PaliGlossB{‘I’m so fortunate, so very fortunate!}}\\
\end{addmargin}
\end{absolutelynopagebreak}

\begin{absolutelynopagebreak}
\setstretch{.7}
{\PaliGlossA{yācayogo dānasaṃvibhāgarato’ti.}}\\
\begin{addmargin}[1em]{2em}
\setstretch{.5}
{\PaliGlossB{Among people full of the stain of stinginess I live at home rid of the stain of stinginess, freely generous, open-handed, loving to let go, committed to charity, loving to give and to share.’}}\\
\end{addmargin}
\end{absolutelynopagebreak}

\begin{absolutelynopagebreak}
\setstretch{.7}
{\PaliGlossA{… pe …}}\\
\begin{addmargin}[1em]{2em}
\setstretch{.5}
{\PaliGlossB{When a noble disciple recollects their generosity their mind is not full of greed, hate, and delusion. …}}\\
\end{addmargin}
\end{absolutelynopagebreak}

\begin{absolutelynopagebreak}
\setstretch{.7}
{\PaliGlossA{evam’idhekacce sattā visujjhanti. (5)}}\\
\begin{addmargin}[1em]{2em}
\setstretch{.5}
{\PaliGlossB{    -}}\\
\end{addmargin}
\end{absolutelynopagebreak}

\begin{absolutelynopagebreak}
\setstretch{.7}
{\PaliGlossA{puna caparaṃ, bhikkhave, ariyasāvako devatā anussarati:}}\\
\begin{addmargin}[1em]{2em}
\setstretch{.5}
{\PaliGlossB{Furthermore, a noble disciple recollects the deities:}}\\
\end{addmargin}
\end{absolutelynopagebreak}

\begin{absolutelynopagebreak}
\setstretch{.7}
{\PaliGlossA{‘santi devā cātumahārājikā, santi devā tāvatiṃsā, santi devā yāmā, santi devā tusitā, santi devā nimmānaratino, santi devā paranimmitavasavattino, santi devā brahmakāyikā, santi devā tatuttari.}}\\
\begin{addmargin}[1em]{2em}
\setstretch{.5}
{\PaliGlossB{‘There are the Gods of the Four Great Kings, the Gods of the Thirty-Three, the Gods of Yama, the Joyful Gods, the Gods Who Love to Create, the Gods Who Control the Creations of Others, the Gods of Brahmā’s Host, and gods even higher than these.}}\\
\end{addmargin}
\end{absolutelynopagebreak}

\begin{absolutelynopagebreak}
\setstretch{.7}
{\PaliGlossA{yathārūpāya saddhāya samannāgatā tā devatā ito cutā tattha upapannā;}}\\
\begin{addmargin}[1em]{2em}
\setstretch{.5}
{\PaliGlossB{When those deities passed away from here, they were reborn there because of their faith, ethics, learning, generosity, and wisdom.}}\\
\end{addmargin}
\end{absolutelynopagebreak}

\begin{absolutelynopagebreak}
\setstretch{.7}
{\PaliGlossA{mayhampi tathārūpā saddhā saṃvijjati. yathārūpena sīlena … sutena … cāgena … paññāya samannāgatā tā devatā ito cutā tattha upapannā; mayhampi tathārūpā paññā saṃvijjatī’ti.}}\\
\begin{addmargin}[1em]{2em}
\setstretch{.5}
{\PaliGlossB{I, too, have the same kind of faith, ethics, learning, generosity, and wisdom.’}}\\
\end{addmargin}
\end{absolutelynopagebreak}

\begin{absolutelynopagebreak}
\setstretch{.7}
{\PaliGlossA{yasmiṃ, bhikkhave, samaye ariyasāvako attano ca tāsañca devatānaṃ saddhañca sīlañca sutañca cāgañca paññañca anussarati nevassa tasmiṃ samaye rāgapariyuṭṭhitaṃ cittaṃ hoti, na dosapariyuṭṭhitaṃ cittaṃ hoti, na mohapariyuṭṭhitaṃ cittaṃ hoti;}}\\
\begin{addmargin}[1em]{2em}
\setstretch{.5}
{\PaliGlossB{When a noble disciple recollects the faith, ethics, learning, generosity, and wisdom of both themselves and the deities their mind is not full of greed, hate, and delusion.}}\\
\end{addmargin}
\end{absolutelynopagebreak}

\begin{absolutelynopagebreak}
\setstretch{.7}
{\PaliGlossA{ujugatamevassa tasmiṃ samaye cittaṃ hoti, nikkhantaṃ muttaṃ vuṭṭhitaṃ gedhamhā.}}\\
\begin{addmargin}[1em]{2em}
\setstretch{.5}
{\PaliGlossB{At that time their mind is unswerving. They’ve left behind greed; they’re free of it and have risen above it.}}\\
\end{addmargin}
\end{absolutelynopagebreak}

\begin{absolutelynopagebreak}
\setstretch{.7}
{\PaliGlossA{‘gedho’ti kho, bhikkhave, pañcannetaṃ kāmaguṇānaṃ adhivacanaṃ.}}\\
\begin{addmargin}[1em]{2em}
\setstretch{.5}
{\PaliGlossB{‘Greed’ is a term for the five kinds of sensual stimulation.}}\\
\end{addmargin}
\end{absolutelynopagebreak}

\begin{absolutelynopagebreak}
\setstretch{.7}
{\PaliGlossA{idampi kho, bhikkhave, ārammaṇaṃ karitvā evam’idhekacce sattā visujjhanti. (6)}}\\
\begin{addmargin}[1em]{2em}
\setstretch{.5}
{\PaliGlossB{Relying on this some sentient beings are purified.}}\\
\end{addmargin}
\end{absolutelynopagebreak}

\begin{absolutelynopagebreak}
\setstretch{.7}
{\PaliGlossA{imāni kho, bhikkhave, cha anussatiṭṭhānānī”ti.}}\\
\begin{addmargin}[1em]{2em}
\setstretch{.5}
{\PaliGlossB{These are the six topics for recollection.”}}\\
\end{addmargin}
\end{absolutelynopagebreak}

\begin{absolutelynopagebreak}
\setstretch{.7}
{\PaliGlossA{pañcamaṃ.}}\\
\begin{addmargin}[1em]{2em}
\setstretch{.5}
{\PaliGlossB{    -}}\\
\end{addmargin}
\end{absolutelynopagebreak}
