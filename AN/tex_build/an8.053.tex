
\begin{absolutelynopagebreak}
\setstretch{.7}
{\PaliGlossA{aṅguttara nikāya 8}}\\
\begin{addmargin}[1em]{2em}
\setstretch{.5}
{\PaliGlossB{Numbered Discourses 8}}\\
\end{addmargin}
\end{absolutelynopagebreak}

\begin{absolutelynopagebreak}
\setstretch{.7}
{\PaliGlossA{6. gotamīvagga}}\\
\begin{addmargin}[1em]{2em}
\setstretch{.5}
{\PaliGlossB{6. Gotamī}}\\
\end{addmargin}
\end{absolutelynopagebreak}

\begin{absolutelynopagebreak}
\setstretch{.7}
{\PaliGlossA{53. saṅkhittasutta}}\\
\begin{addmargin}[1em]{2em}
\setstretch{.5}
{\PaliGlossB{53. Brief Advice to Gotamī}}\\
\end{addmargin}
\end{absolutelynopagebreak}

\begin{absolutelynopagebreak}
\setstretch{.7}
{\PaliGlossA{ekaṃ samayaṃ bhagavā vesāliyaṃ viharati mahāvane kūṭāgārasālāyaṃ.}}\\
\begin{addmargin}[1em]{2em}
\setstretch{.5}
{\PaliGlossB{At one time the Buddha was staying near Vesālī, at the Great Wood, in the hall with the peaked roof.}}\\
\end{addmargin}
\end{absolutelynopagebreak}

\begin{absolutelynopagebreak}
\setstretch{.7}
{\PaliGlossA{atha kho mahāpajāpatī gotamī yena bhagavā tenupasaṅkami; upasaṅkamitvā bhagavantaṃ abhivādetvā ekamantaṃ aṭṭhāsi. ekamantaṃ ṭhitā kho sā mahāpajāpatī gotamī bhagavantaṃ etadavoca:}}\\
\begin{addmargin}[1em]{2em}
\setstretch{.5}
{\PaliGlossB{Then Mahāpajāpatī Gotamī went up to the Buddha, bowed, stood to one side, and said to him:}}\\
\end{addmargin}
\end{absolutelynopagebreak}

\begin{absolutelynopagebreak}
\setstretch{.7}
{\PaliGlossA{“sādhu me, bhante, bhagavā saṅkhittena dhammaṃ desetu, yamahaṃ bhagavato dhammaṃ sutvā ekā vūpakaṭṭhā appamattā ātāpinī pahitattā vihareyyan”ti.}}\\
\begin{addmargin}[1em]{2em}
\setstretch{.5}
{\PaliGlossB{“Sir, may the Buddha please teach me Dhamma in brief. When I’ve heard it, I’ll live alone, withdrawn, diligent, keen, and resolute.”}}\\
\end{addmargin}
\end{absolutelynopagebreak}

\begin{absolutelynopagebreak}
\setstretch{.7}
{\PaliGlossA{“ye kho tvaṃ, gotami, dhamme jāneyyāsi:}}\\
\begin{addmargin}[1em]{2em}
\setstretch{.5}
{\PaliGlossB{“Gotamī, you might know that certain things}}\\
\end{addmargin}
\end{absolutelynopagebreak}

\begin{absolutelynopagebreak}
\setstretch{.7}
{\PaliGlossA{‘ime dhammā sarāgāya saṃvattanti, no virāgāya;}}\\
\begin{addmargin}[1em]{2em}
\setstretch{.5}
{\PaliGlossB{lead to passion, not dispassion;}}\\
\end{addmargin}
\end{absolutelynopagebreak}

\begin{absolutelynopagebreak}
\setstretch{.7}
{\PaliGlossA{saṃyogāya saṃvattanti, no visaṃyogāya;}}\\
\begin{addmargin}[1em]{2em}
\setstretch{.5}
{\PaliGlossB{to being fettered, not to being unfettered;}}\\
\end{addmargin}
\end{absolutelynopagebreak}

\begin{absolutelynopagebreak}
\setstretch{.7}
{\PaliGlossA{ācayāya saṃvattanti, no apacayāya;}}\\
\begin{addmargin}[1em]{2em}
\setstretch{.5}
{\PaliGlossB{to accumulation, not dispersal;}}\\
\end{addmargin}
\end{absolutelynopagebreak}

\begin{absolutelynopagebreak}
\setstretch{.7}
{\PaliGlossA{mahicchatāya saṃvattanti, no appicchatāya;}}\\
\begin{addmargin}[1em]{2em}
\setstretch{.5}
{\PaliGlossB{to more desires, not fewer;}}\\
\end{addmargin}
\end{absolutelynopagebreak}

\begin{absolutelynopagebreak}
\setstretch{.7}
{\PaliGlossA{asantuṭṭhiyā saṃvattanti, no santuṭṭhiyā;}}\\
\begin{addmargin}[1em]{2em}
\setstretch{.5}
{\PaliGlossB{to lack of contentment, not contentment;}}\\
\end{addmargin}
\end{absolutelynopagebreak}

\begin{absolutelynopagebreak}
\setstretch{.7}
{\PaliGlossA{saṅgaṇikāya saṃvattanti, no pavivekāya;}}\\
\begin{addmargin}[1em]{2em}
\setstretch{.5}
{\PaliGlossB{to crowding, not seclusion;}}\\
\end{addmargin}
\end{absolutelynopagebreak}

\begin{absolutelynopagebreak}
\setstretch{.7}
{\PaliGlossA{kosajjāya saṃvattanti, no vīriyārambhāya;}}\\
\begin{addmargin}[1em]{2em}
\setstretch{.5}
{\PaliGlossB{to laziness, not energy;}}\\
\end{addmargin}
\end{absolutelynopagebreak}

\begin{absolutelynopagebreak}
\setstretch{.7}
{\PaliGlossA{dubbharatāya saṃvattanti, no subharatāyā’ti,}}\\
\begin{addmargin}[1em]{2em}
\setstretch{.5}
{\PaliGlossB{to being hard to look after, not being easy to look after.}}\\
\end{addmargin}
\end{absolutelynopagebreak}

\begin{absolutelynopagebreak}
\setstretch{.7}
{\PaliGlossA{ekaṃsena, gotami, dhāreyyāsi:}}\\
\begin{addmargin}[1em]{2em}
\setstretch{.5}
{\PaliGlossB{You should definitely bear in mind that these things are}}\\
\end{addmargin}
\end{absolutelynopagebreak}

\begin{absolutelynopagebreak}
\setstretch{.7}
{\PaliGlossA{‘neso dhammo, neso vinayo, netaṃ satthusāsanan’ti.}}\\
\begin{addmargin}[1em]{2em}
\setstretch{.5}
{\PaliGlossB{not the teaching, not the training, and not the Teacher’s instructions.}}\\
\end{addmargin}
\end{absolutelynopagebreak}

\begin{absolutelynopagebreak}
\setstretch{.7}
{\PaliGlossA{ye ca kho tvaṃ, gotami, dhamme jāneyyāsi:}}\\
\begin{addmargin}[1em]{2em}
\setstretch{.5}
{\PaliGlossB{You might know that certain things}}\\
\end{addmargin}
\end{absolutelynopagebreak}

\begin{absolutelynopagebreak}
\setstretch{.7}
{\PaliGlossA{‘ime dhammā virāgāya saṃvattanti, no sarāgāya;}}\\
\begin{addmargin}[1em]{2em}
\setstretch{.5}
{\PaliGlossB{lead to dispassion, not passion;}}\\
\end{addmargin}
\end{absolutelynopagebreak}

\begin{absolutelynopagebreak}
\setstretch{.7}
{\PaliGlossA{visaṃyogāya saṃvattanti, no saṃyogāya;}}\\
\begin{addmargin}[1em]{2em}
\setstretch{.5}
{\PaliGlossB{to being unfettered, not to being fettered;}}\\
\end{addmargin}
\end{absolutelynopagebreak}

\begin{absolutelynopagebreak}
\setstretch{.7}
{\PaliGlossA{apacayāya saṃvattanti, no ācayāya;}}\\
\begin{addmargin}[1em]{2em}
\setstretch{.5}
{\PaliGlossB{to dispersal, not accumulation;}}\\
\end{addmargin}
\end{absolutelynopagebreak}

\begin{absolutelynopagebreak}
\setstretch{.7}
{\PaliGlossA{appicchatāya saṃvattanti, no mahicchatāya;}}\\
\begin{addmargin}[1em]{2em}
\setstretch{.5}
{\PaliGlossB{to fewer desires, not more;}}\\
\end{addmargin}
\end{absolutelynopagebreak}

\begin{absolutelynopagebreak}
\setstretch{.7}
{\PaliGlossA{santuṭṭhiyā saṃvattanti, no asantuṭṭhiyā;}}\\
\begin{addmargin}[1em]{2em}
\setstretch{.5}
{\PaliGlossB{to contentment, not lack of contentment;}}\\
\end{addmargin}
\end{absolutelynopagebreak}

\begin{absolutelynopagebreak}
\setstretch{.7}
{\PaliGlossA{pavivekāya saṃvattanti, no saṅgaṇikāya;}}\\
\begin{addmargin}[1em]{2em}
\setstretch{.5}
{\PaliGlossB{to seclusion, not crowding;}}\\
\end{addmargin}
\end{absolutelynopagebreak}

\begin{absolutelynopagebreak}
\setstretch{.7}
{\PaliGlossA{vīriyārambhāya saṃvattanti, no kosajjāya;}}\\
\begin{addmargin}[1em]{2em}
\setstretch{.5}
{\PaliGlossB{to energy, not laziness;}}\\
\end{addmargin}
\end{absolutelynopagebreak}

\begin{absolutelynopagebreak}
\setstretch{.7}
{\PaliGlossA{subharatāya saṃvattanti, no dubbharatāyā’ti,}}\\
\begin{addmargin}[1em]{2em}
\setstretch{.5}
{\PaliGlossB{to being easy to look after, not being hard to look after.}}\\
\end{addmargin}
\end{absolutelynopagebreak}

\begin{absolutelynopagebreak}
\setstretch{.7}
{\PaliGlossA{ekaṃsena, gotami, dhāreyyāsi:}}\\
\begin{addmargin}[1em]{2em}
\setstretch{.5}
{\PaliGlossB{You should definitely bear in mind that these things are}}\\
\end{addmargin}
\end{absolutelynopagebreak}

\begin{absolutelynopagebreak}
\setstretch{.7}
{\PaliGlossA{‘eso dhammo, eso vinayo, etaṃ satthusāsanan’”ti.}}\\
\begin{addmargin}[1em]{2em}
\setstretch{.5}
{\PaliGlossB{the teaching, the training, and the Teacher’s instructions.”}}\\
\end{addmargin}
\end{absolutelynopagebreak}

\begin{absolutelynopagebreak}
\setstretch{.7}
{\PaliGlossA{tatiyaṃ.}}\\
\begin{addmargin}[1em]{2em}
\setstretch{.5}
{\PaliGlossB{    -}}\\
\end{addmargin}
\end{absolutelynopagebreak}
