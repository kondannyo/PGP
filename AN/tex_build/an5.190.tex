
\begin{absolutelynopagebreak}
\setstretch{.7}
{\PaliGlossA{aṅguttara nikāya 5}}\\
\begin{addmargin}[1em]{2em}
\setstretch{.5}
{\PaliGlossB{Numbered Discourses 5}}\\
\end{addmargin}
\end{absolutelynopagebreak}

\begin{absolutelynopagebreak}
\setstretch{.7}
{\PaliGlossA{19. araññavagga}}\\
\begin{addmargin}[1em]{2em}
\setstretch{.5}
{\PaliGlossB{19. Wilderness Dwellers}}\\
\end{addmargin}
\end{absolutelynopagebreak}

\begin{absolutelynopagebreak}
\setstretch{.7}
{\PaliGlossA{190. pattapiṇḍikasutta}}\\
\begin{addmargin}[1em]{2em}
\setstretch{.5}
{\PaliGlossB{190. Those Who Eat Only From the Almsbowl}}\\
\end{addmargin}
\end{absolutelynopagebreak}

\begin{absolutelynopagebreak}
\setstretch{.7}
{\PaliGlossA{“pañcime, bhikkhave, pattapiṇḍikā.}}\\
\begin{addmargin}[1em]{2em}
\setstretch{.5}
{\PaliGlossB{“Mendicants, there are these five kinds of people who eat only from the almsbowl.}}\\
\end{addmargin}
\end{absolutelynopagebreak}

\begin{absolutelynopagebreak}
\setstretch{.7}
{\PaliGlossA{katame pañca?}}\\
\begin{addmargin}[1em]{2em}
\setstretch{.5}
{\PaliGlossB{What five?}}\\
\end{addmargin}
\end{absolutelynopagebreak}

\begin{absolutelynopagebreak}
\setstretch{.7}
{\PaliGlossA{mandattā momūhattā pattapiṇḍiko hoti, pāpiccho icchāpakato pattapiṇḍiko hoti, ummādā cittakkhepā pattapiṇḍiko hoti, ‘vaṇṇitaṃ buddhehi buddhasāvakehī’ti pattapiṇḍiko hoti, appicchataṃyeva nissāya santuṭṭhiṃyeva nissāya sallekhaṃyeva nissāya pavivekaṃyeva nissāya idamatthitaṃyeva nissāya pattapiṇḍiko hoti.}}\\
\begin{addmargin}[1em]{2em}
\setstretch{.5}
{\PaliGlossB{A person may eat only from the almsbowl because of stupidity and folly. Or because of wicked desires, being naturally full of desires. Or because of madness and mental disorder. Or because it is praised by the Buddhas and their disciples. Or for the sake of having few wishes, for the sake of contentment, self-effacement, seclusion, and simplicity.}}\\
\end{addmargin}
\end{absolutelynopagebreak}

\begin{absolutelynopagebreak}
\setstretch{.7}
{\PaliGlossA{ime kho, bhikkhave, pañca pattapiṇḍikā.}}\\
\begin{addmargin}[1em]{2em}
\setstretch{.5}
{\PaliGlossB{These are the five kinds of people who eat only from the almsbowl.}}\\
\end{addmargin}
\end{absolutelynopagebreak}

\begin{absolutelynopagebreak}
\setstretch{.7}
{\PaliGlossA{imesaṃ kho, bhikkhave, pañcannaṃ pattapiṇḍikānaṃ yvāyaṃ pattapiṇḍiko appicchataṃyeva nissāya santuṭṭhiṃyeva nissāya sallekhaṃyeva nissāya pavivekaṃyeva nissāya idamatthitaṃyeva nissāya pattapiṇḍiko hoti, ayaṃ imesaṃ pañcannaṃ pattapiṇḍikānaṃ aggo ca seṭṭho ca mokkho ca uttamo ca pavaro ca.}}\\
\begin{addmargin}[1em]{2em}
\setstretch{.5}
{\PaliGlossB{But the person who eats only from the almsbowl for the sake of having few wishes is the foremost, best, chief, highest, and finest of the five.}}\\
\end{addmargin}
\end{absolutelynopagebreak}

\begin{absolutelynopagebreak}
\setstretch{.7}
{\PaliGlossA{seyyathāpi, bhikkhave, gavā khīraṃ, khīramhā dadhi, dadhimhā navanītaṃ, navanītamhā sappi, sappimhā sappimaṇḍo, sappimaṇḍo tattha aggamakkhāyati;}}\\
\begin{addmargin}[1em]{2em}
\setstretch{.5}
{\PaliGlossB{From a cow comes milk, from milk comes curds, from curds come butter, from butter comes ghee, and from ghee comes cream of ghee. And the cream of ghee is said to be the best of these.}}\\
\end{addmargin}
\end{absolutelynopagebreak}

\begin{absolutelynopagebreak}
\setstretch{.7}
{\PaliGlossA{evamevaṃ kho, bhikkhave, imesaṃ pañcannaṃ pattapiṇḍikānaṃ yvāyaṃ pattapiṇḍiko appicchataṃyeva nissāya santuṭṭhiṃyeva nissāya sallekhaṃyeva nissāya pavivekaṃyeva nissāya idamatthitaṃyeva nissāya pattapiṇḍiko hoti, ayaṃ imesaṃ pañcannaṃ pattapiṇḍikānaṃ aggo ca seṭṭho ca mokkho ca uttamo ca pavaro cā”ti.}}\\
\begin{addmargin}[1em]{2em}
\setstretch{.5}
{\PaliGlossB{In the same way, the person who eats only from the almsbowl for the sake of having few wishes is the foremost, best, chief, highest, and finest of the five.”}}\\
\end{addmargin}
\end{absolutelynopagebreak}

\begin{absolutelynopagebreak}
\setstretch{.7}
{\PaliGlossA{dasamaṃ.}}\\
\begin{addmargin}[1em]{2em}
\setstretch{.5}
{\PaliGlossB{    -}}\\
\end{addmargin}
\end{absolutelynopagebreak}

\begin{absolutelynopagebreak}
\setstretch{.7}
{\PaliGlossA{araññavaggo catuttho.}}\\
\begin{addmargin}[1em]{2em}
\setstretch{.5}
{\PaliGlossB{    -}}\\
\end{addmargin}
\end{absolutelynopagebreak}

\begin{absolutelynopagebreak}
\setstretch{.7}
{\PaliGlossA{araññaṃ cīvaraṃ rukkha,}}\\
\begin{addmargin}[1em]{2em}
\setstretch{.5}
{\PaliGlossB{    -}}\\
\end{addmargin}
\end{absolutelynopagebreak}

\begin{absolutelynopagebreak}
\setstretch{.7}
{\PaliGlossA{susānaṃ abbhokāsikaṃ;}}\\
\begin{addmargin}[1em]{2em}
\setstretch{.5}
{\PaliGlossB{    -}}\\
\end{addmargin}
\end{absolutelynopagebreak}

\begin{absolutelynopagebreak}
\setstretch{.7}
{\PaliGlossA{nesajjaṃ santhataṃ ekāsanikaṃ,}}\\
\begin{addmargin}[1em]{2em}
\setstretch{.5}
{\PaliGlossB{    -}}\\
\end{addmargin}
\end{absolutelynopagebreak}

\begin{absolutelynopagebreak}
\setstretch{.7}
{\PaliGlossA{khalupacchāpiṇḍikena cāti.}}\\
\begin{addmargin}[1em]{2em}
\setstretch{.5}
{\PaliGlossB{    -}}\\
\end{addmargin}
\end{absolutelynopagebreak}
