
\begin{absolutelynopagebreak}
\setstretch{.7}
{\PaliGlossA{aṅguttara nikāya 3}}\\
\begin{addmargin}[1em]{2em}
\setstretch{.5}
{\PaliGlossB{Numbered Discourses 3}}\\
\end{addmargin}
\end{absolutelynopagebreak}

\begin{absolutelynopagebreak}
\setstretch{.7}
{\PaliGlossA{8. ānandavagga}}\\
\begin{addmargin}[1em]{2em}
\setstretch{.5}
{\PaliGlossB{8. Ānanda}}\\
\end{addmargin}
\end{absolutelynopagebreak}

\begin{absolutelynopagebreak}
\setstretch{.7}
{\PaliGlossA{71. channasutta}}\\
\begin{addmargin}[1em]{2em}
\setstretch{.5}
{\PaliGlossB{71. With Channa}}\\
\end{addmargin}
\end{absolutelynopagebreak}

\begin{absolutelynopagebreak}
\setstretch{.7}
{\PaliGlossA{ekaṃ samayaṃ bhagavā sāvatthiyaṃ viharati jetavane anāthapiṇḍikassa ārāme.}}\\
\begin{addmargin}[1em]{2em}
\setstretch{.5}
{\PaliGlossB{At one time the Buddha was staying near Sāvatthī in Jeta’s Grove, Anāthapiṇḍika’s monastery.}}\\
\end{addmargin}
\end{absolutelynopagebreak}

\begin{absolutelynopagebreak}
\setstretch{.7}
{\PaliGlossA{atha kho channo paribbājako yenāyasmā ānando tenupasaṅkami; upasaṅkamitvā āyasmatā ānandena saddhiṃ sammodi.}}\\
\begin{addmargin}[1em]{2em}
\setstretch{.5}
{\PaliGlossB{Then the wanderer Channa went up to Venerable Ānanda and exchanged greetings with him.}}\\
\end{addmargin}
\end{absolutelynopagebreak}

\begin{absolutelynopagebreak}
\setstretch{.7}
{\PaliGlossA{sammodanīyaṃ kathaṃ sāraṇīyaṃ vītisāretvā ekamantaṃ nisīdi. ekamantaṃ nisinno kho channo paribbājako āyasmantaṃ ānandaṃ etadavoca:}}\\
\begin{addmargin}[1em]{2em}
\setstretch{.5}
{\PaliGlossB{When the greetings and polite conversation were over, he sat down to one side and said to Ānanda:}}\\
\end{addmargin}
\end{absolutelynopagebreak}

\begin{absolutelynopagebreak}
\setstretch{.7}
{\PaliGlossA{“tumhepi, āvuso ānanda, rāgassa pahānaṃ paññāpetha, dosassa pahānaṃ paññāpetha, mohassa pahānaṃ paññāpethā”ti.}}\\
\begin{addmargin}[1em]{2em}
\setstretch{.5}
{\PaliGlossB{“Reverend Ānanda, do you advocate giving up greed, hate, and delusion?”}}\\
\end{addmargin}
\end{absolutelynopagebreak}

\begin{absolutelynopagebreak}
\setstretch{.7}
{\PaliGlossA{“mayaṃ kho, āvuso, rāgassa pahānaṃ paññāpema, dosassa pahānaṃ paññāpema, mohassa pahānaṃ paññapemā”ti.}}\\
\begin{addmargin}[1em]{2em}
\setstretch{.5}
{\PaliGlossB{“We do, reverend.”}}\\
\end{addmargin}
\end{absolutelynopagebreak}

\begin{absolutelynopagebreak}
\setstretch{.7}
{\PaliGlossA{“kiṃ pana tumhe, āvuso, rāge ādīnavaṃ disvā rāgassa pahānaṃ paññāpetha, kiṃ dose ādīnavaṃ disvā dosassa pahānaṃ paññāpetha, kiṃ mohe ādīnavaṃ disvā mohassa pahānaṃ paññāpethā”ti?}}\\
\begin{addmargin}[1em]{2em}
\setstretch{.5}
{\PaliGlossB{“But what drawbacks have you seen, Reverend Ānanda, that you advocate giving up greed, hate, and delusion?”}}\\
\end{addmargin}
\end{absolutelynopagebreak}

\begin{absolutelynopagebreak}
\setstretch{.7}
{\PaliGlossA{“ratto kho, āvuso, rāgena abhibhūto pariyādinnacitto attabyābādhāyapi ceteti, parabyābādhāyapi ceteti, ubhayabyābādhāyapi ceteti, cetasikampi dukkhaṃ domanassaṃ paṭisaṃvedeti;}}\\
\begin{addmargin}[1em]{2em}
\setstretch{.5}
{\PaliGlossB{“A greedy person, overcome by greed, intends to hurt themselves, hurt others, and hurt both. They experience mental pain and sadness.}}\\
\end{addmargin}
\end{absolutelynopagebreak}

\begin{absolutelynopagebreak}
\setstretch{.7}
{\PaliGlossA{rāge pahīne nevattabyābādhāyapi ceteti, na parabyābādhāyapi ceteti, na ubhayabyābādhāyapi ceteti, na cetasikaṃ dukkhaṃ domanassaṃ paṭisaṃvedeti.}}\\
\begin{addmargin}[1em]{2em}
\setstretch{.5}
{\PaliGlossB{When greed has been given up, they don’t intend to hurt themselves, hurt others, and hurt both. They don’t experience mental pain and sadness.}}\\
\end{addmargin}
\end{absolutelynopagebreak}

\begin{absolutelynopagebreak}
\setstretch{.7}
{\PaliGlossA{ratto kho, āvuso, rāgena abhibhūto pariyādinnacitto kāyena duccaritaṃ carati, vācāya duccaritaṃ carati, manasā duccaritaṃ carati;}}\\
\begin{addmargin}[1em]{2em}
\setstretch{.5}
{\PaliGlossB{A greedy person does bad things by way of body, speech, and mind.}}\\
\end{addmargin}
\end{absolutelynopagebreak}

\begin{absolutelynopagebreak}
\setstretch{.7}
{\PaliGlossA{rāge pahīne neva kāyena duccaritaṃ carati, na vācāya duccaritaṃ carati, na manasā duccaritaṃ carati.}}\\
\begin{addmargin}[1em]{2em}
\setstretch{.5}
{\PaliGlossB{When greed has been given up, they don’t do bad things by way of body, speech, and mind.}}\\
\end{addmargin}
\end{absolutelynopagebreak}

\begin{absolutelynopagebreak}
\setstretch{.7}
{\PaliGlossA{ratto kho, āvuso, rāgena abhibhūto pariyādinnacitto attatthampi yathābhūtaṃ nappajānāti, paratthampi yathābhūtaṃ nappajānāti, ubhayatthampi yathābhūtaṃ nappajānāti;}}\\
\begin{addmargin}[1em]{2em}
\setstretch{.5}
{\PaliGlossB{A greedy person doesn’t truly understand what’s for their own good, the good of another, or the good of both.}}\\
\end{addmargin}
\end{absolutelynopagebreak}

\begin{absolutelynopagebreak}
\setstretch{.7}
{\PaliGlossA{rāge pahīne attatthampi yathābhūtaṃ pajānāti, paratthampi yathābhūtaṃ pajānāti, ubhayatthampi yathābhūtaṃ pajānāti.}}\\
\begin{addmargin}[1em]{2em}
\setstretch{.5}
{\PaliGlossB{When greed has been given up, they truly understand what’s for their own good, the good of another, or the good of both.}}\\
\end{addmargin}
\end{absolutelynopagebreak}

\begin{absolutelynopagebreak}
\setstretch{.7}
{\PaliGlossA{rāgo kho, āvuso, andhakaraṇo acakkhukaraṇo aññāṇakaraṇo paññānirodhiko vighātapakkhiko anibbānasaṃvattaniko.}}\\
\begin{addmargin}[1em]{2em}
\setstretch{.5}
{\PaliGlossB{Greed is a destroyer of sight, vision, and knowledge. It blocks wisdom, it’s on the side of anguish, and it doesn’t lead to extinguishment.}}\\
\end{addmargin}
\end{absolutelynopagebreak}

\begin{absolutelynopagebreak}
\setstretch{.7}
{\PaliGlossA{duṭṭho kho, āvuso, dosena … pe …}}\\
\begin{addmargin}[1em]{2em}
\setstretch{.5}
{\PaliGlossB{A hateful person, overcome by hate, intends to hurt themselves, hurt others, and hurt both. …}}\\
\end{addmargin}
\end{absolutelynopagebreak}

\begin{absolutelynopagebreak}
\setstretch{.7}
{\PaliGlossA{mūḷho kho, āvuso, mohena abhibhūto pariyādinnacitto attabyābādhāyapi ceteti, parabyābādhāyapi ceteti, ubhayabyābādhāyapi ceteti, cetasikampi dukkhaṃ domanassaṃ paṭisaṃvedeti;}}\\
\begin{addmargin}[1em]{2em}
\setstretch{.5}
{\PaliGlossB{A deluded person, overcome by delusion, intends to hurt themselves, hurt others, and hurt both. They experience mental pain and sadness.}}\\
\end{addmargin}
\end{absolutelynopagebreak}

\begin{absolutelynopagebreak}
\setstretch{.7}
{\PaliGlossA{mohe pahīne nevattabyābādhāyapi ceteti, na parabyābādhāyapi ceteti, na ubhayabyābādhāyapi ceteti, na cetasikaṃ dukkhaṃ domanassaṃ paṭisaṃvedeti.}}\\
\begin{addmargin}[1em]{2em}
\setstretch{.5}
{\PaliGlossB{When delusion has been given up, they don’t intend to hurt themselves, hurt others, and hurt both. They don’t experience mental pain and sadness.}}\\
\end{addmargin}
\end{absolutelynopagebreak}

\begin{absolutelynopagebreak}
\setstretch{.7}
{\PaliGlossA{mūḷho kho, āvuso, mohena abhibhūto pariyādinnacitto kāyena duccaritaṃ carati, vācāya duccaritaṃ carati, manasā duccaritaṃ carati;}}\\
\begin{addmargin}[1em]{2em}
\setstretch{.5}
{\PaliGlossB{A deluded person does bad things by way of body, speech, and mind.}}\\
\end{addmargin}
\end{absolutelynopagebreak}

\begin{absolutelynopagebreak}
\setstretch{.7}
{\PaliGlossA{mohe pahīne neva kāyena duccaritaṃ carati, na vācāya duccaritaṃ carati, na manasā duccaritaṃ carati.}}\\
\begin{addmargin}[1em]{2em}
\setstretch{.5}
{\PaliGlossB{When delusion has been given up, they don’t do bad things by way of body, speech, and mind.}}\\
\end{addmargin}
\end{absolutelynopagebreak}

\begin{absolutelynopagebreak}
\setstretch{.7}
{\PaliGlossA{mūḷho kho, āvuso, mohena abhibhūto pariyādinnacitto attatthampi yathābhūtaṃ nappajānāti, paratthampi yathābhūtaṃ nappajānāti, ubhayatthampi yathābhūtaṃ nappajānāti;}}\\
\begin{addmargin}[1em]{2em}
\setstretch{.5}
{\PaliGlossB{A deluded person doesn’t truly understand what’s for their own good, the good of another, or the good of both.}}\\
\end{addmargin}
\end{absolutelynopagebreak}

\begin{absolutelynopagebreak}
\setstretch{.7}
{\PaliGlossA{mohe pahīne attatthampi yathābhūtaṃ pajānāti, paratthampi yathābhūtaṃ pajānāti, ubhayatthampi yathābhūtaṃ pajānāti.}}\\
\begin{addmargin}[1em]{2em}
\setstretch{.5}
{\PaliGlossB{When delusion has been given up, they truly understand what’s for their own good, the good of another, or the good of both.}}\\
\end{addmargin}
\end{absolutelynopagebreak}

\begin{absolutelynopagebreak}
\setstretch{.7}
{\PaliGlossA{moho kho, āvuso, andhakaraṇo acakkhukaraṇo aññāṇakaraṇo paññānirodhiko vighātapakkhiko anibbānasaṃvattaniko.}}\\
\begin{addmargin}[1em]{2em}
\setstretch{.5}
{\PaliGlossB{Delusion is a destroyer of sight, vision, and knowledge; it blocks wisdom, it’s on the side of anguish, and it doesn’t lead to extinguishment.}}\\
\end{addmargin}
\end{absolutelynopagebreak}

\begin{absolutelynopagebreak}
\setstretch{.7}
{\PaliGlossA{idaṃ kho mayaṃ, āvuso, rāge ādīnavaṃ disvā rāgassa pahānaṃ paññāpema.}}\\
\begin{addmargin}[1em]{2em}
\setstretch{.5}
{\PaliGlossB{This is the drawback we’ve seen in greed, hate, and delusion, and this is why we advocate giving them up.”}}\\
\end{addmargin}
\end{absolutelynopagebreak}

\begin{absolutelynopagebreak}
\setstretch{.7}
{\PaliGlossA{idaṃ dose ādīnavaṃ disvā dosassa pahānaṃ paññāpema.}}\\
\begin{addmargin}[1em]{2em}
\setstretch{.5}
{\PaliGlossB{    -}}\\
\end{addmargin}
\end{absolutelynopagebreak}

\begin{absolutelynopagebreak}
\setstretch{.7}
{\PaliGlossA{idaṃ mohe ādīnavaṃ disvā mohassa pahānaṃ paññāpemā”ti.}}\\
\begin{addmargin}[1em]{2em}
\setstretch{.5}
{\PaliGlossB{    -}}\\
\end{addmargin}
\end{absolutelynopagebreak}

\begin{absolutelynopagebreak}
\setstretch{.7}
{\PaliGlossA{“atthi panāvuso, maggo atthi paṭipadā etassa rāgassa dosassa mohassa pahānāyā”ti?}}\\
\begin{addmargin}[1em]{2em}
\setstretch{.5}
{\PaliGlossB{“But, reverend, is there a path and a practice for giving up that greed, hate, and delusion?”}}\\
\end{addmargin}
\end{absolutelynopagebreak}

\begin{absolutelynopagebreak}
\setstretch{.7}
{\PaliGlossA{“atthāvuso, maggo atthi paṭipadā etassa rāgassa dosassa mohassa pahānāyā”ti.}}\\
\begin{addmargin}[1em]{2em}
\setstretch{.5}
{\PaliGlossB{“There is, reverend, a path and a practice for giving up that greed, hate, and delusion.”}}\\
\end{addmargin}
\end{absolutelynopagebreak}

\begin{absolutelynopagebreak}
\setstretch{.7}
{\PaliGlossA{“katamo panāvuso, maggo katamā paṭipadā etassa rāgassa dosassa mohassa pahānāyā”ti?}}\\
\begin{addmargin}[1em]{2em}
\setstretch{.5}
{\PaliGlossB{“Well, what is it?”}}\\
\end{addmargin}
\end{absolutelynopagebreak}

\begin{absolutelynopagebreak}
\setstretch{.7}
{\PaliGlossA{“ayameva ariyo aṭṭhaṅgiko maggo, seyyathidaṃ—}}\\
\begin{addmargin}[1em]{2em}
\setstretch{.5}
{\PaliGlossB{“It is simply this noble eightfold path, that is:}}\\
\end{addmargin}
\end{absolutelynopagebreak}

\begin{absolutelynopagebreak}
\setstretch{.7}
{\PaliGlossA{sammādiṭṭhi … pe … sammāsamādhi.}}\\
\begin{addmargin}[1em]{2em}
\setstretch{.5}
{\PaliGlossB{right view, right thought, right speech, right action, right livelihood, right effort, right mindfulness, and right immersion.}}\\
\end{addmargin}
\end{absolutelynopagebreak}

\begin{absolutelynopagebreak}
\setstretch{.7}
{\PaliGlossA{ayaṃ kho, āvuso, maggo ayaṃ paṭipadā etassa rāgassa dosassa mohassa pahānāyā”ti.}}\\
\begin{addmargin}[1em]{2em}
\setstretch{.5}
{\PaliGlossB{This is the path, this is the practice for giving up that greed, hate, and delusion.”}}\\
\end{addmargin}
\end{absolutelynopagebreak}

\begin{absolutelynopagebreak}
\setstretch{.7}
{\PaliGlossA{“bhaddako kho, āvuso, maggo bhaddikā paṭipadā etassa rāgassa dosassa mohassa pahānāya.}}\\
\begin{addmargin}[1em]{2em}
\setstretch{.5}
{\PaliGlossB{“This is a fine path, a fine practice, for giving up that greed, hate, and delusion.}}\\
\end{addmargin}
\end{absolutelynopagebreak}

\begin{absolutelynopagebreak}
\setstretch{.7}
{\PaliGlossA{alañca panāvuso ānanda, appamādāyā”ti.}}\\
\begin{addmargin}[1em]{2em}
\setstretch{.5}
{\PaliGlossB{Just this much is enough to be diligent.”}}\\
\end{addmargin}
\end{absolutelynopagebreak}

\begin{absolutelynopagebreak}
\setstretch{.7}
{\PaliGlossA{paṭhamaṃ.}}\\
\begin{addmargin}[1em]{2em}
\setstretch{.5}
{\PaliGlossB{    -}}\\
\end{addmargin}
\end{absolutelynopagebreak}
