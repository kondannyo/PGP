
\begin{absolutelynopagebreak}
\setstretch{.7}
{\PaliGlossA{aṅguttara nikāya 10}}\\
\begin{addmargin}[1em]{2em}
\setstretch{.5}
{\PaliGlossB{Numbered Discourses 10}}\\
\end{addmargin}
\end{absolutelynopagebreak}

\begin{absolutelynopagebreak}
\setstretch{.7}
{\PaliGlossA{10. upālivagga}}\\
\begin{addmargin}[1em]{2em}
\setstretch{.5}
{\PaliGlossB{10. With Upāli}}\\
\end{addmargin}
\end{absolutelynopagebreak}

\begin{absolutelynopagebreak}
\setstretch{.7}
{\PaliGlossA{91. kāmabhogīsutta}}\\
\begin{addmargin}[1em]{2em}
\setstretch{.5}
{\PaliGlossB{91. Pleasure Seekers}}\\
\end{addmargin}
\end{absolutelynopagebreak}

\begin{absolutelynopagebreak}
\setstretch{.7}
{\PaliGlossA{ekaṃ samayaṃ bhagavā sāvatthiyaṃ viharati jetavane anāthapiṇḍikassa ārāme.}}\\
\begin{addmargin}[1em]{2em}
\setstretch{.5}
{\PaliGlossB{At one time the Buddha was staying near Sāvatthī in Jeta’s Grove, Anāthapiṇḍika’s monastery.}}\\
\end{addmargin}
\end{absolutelynopagebreak}

\begin{absolutelynopagebreak}
\setstretch{.7}
{\PaliGlossA{atha kho anāthapiṇḍiko gahapati yena bhagavā tenupasaṅkami; upasaṅkamitvā bhagavantaṃ abhivādetvā ekamantaṃ nisīdi. ekamantaṃ nisinnaṃ kho anāthapiṇḍikaṃ gahapatiṃ bhagavā etadavoca:}}\\
\begin{addmargin}[1em]{2em}
\setstretch{.5}
{\PaliGlossB{Then the householder Anāthapiṇḍika went up to the Buddha, bowed, and sat down to one side. Seated to one side, the Buddha said to the householder Anāthapiṇḍika:}}\\
\end{addmargin}
\end{absolutelynopagebreak}

\begin{absolutelynopagebreak}
\setstretch{.7}
{\PaliGlossA{“dasayime, gahapati, kāmabhogī santo saṃvijjamānā lokasmiṃ.}}\\
\begin{addmargin}[1em]{2em}
\setstretch{.5}
{\PaliGlossB{“These ten pleasure seekers are found in the world.}}\\
\end{addmargin}
\end{absolutelynopagebreak}

\begin{absolutelynopagebreak}
\setstretch{.7}
{\PaliGlossA{katame dasa?}}\\
\begin{addmargin}[1em]{2em}
\setstretch{.5}
{\PaliGlossB{What ten?}}\\
\end{addmargin}
\end{absolutelynopagebreak}

\begin{absolutelynopagebreak}
\setstretch{.7}
{\PaliGlossA{idha, gahapati, ekacco kāmabhogī adhammena bhoge pariyesati sāhasena;}}\\
\begin{addmargin}[1em]{2em}
\setstretch{.5}
{\PaliGlossB{First, a pleasure seeker seeks wealth using illegitimate, coercive means.}}\\
\end{addmargin}
\end{absolutelynopagebreak}

\begin{absolutelynopagebreak}
\setstretch{.7}
{\PaliGlossA{adhammena bhoge pariyesitvā sāhasena na attānaṃ sukheti na pīṇeti na saṃvibhajati na puññāni karoti. (1)}}\\
\begin{addmargin}[1em]{2em}
\setstretch{.5}
{\PaliGlossB{They don’t make themselves happy and pleased, nor share it and make merit.}}\\
\end{addmargin}
\end{absolutelynopagebreak}

\begin{absolutelynopagebreak}
\setstretch{.7}
{\PaliGlossA{idha pana, gahapati, ekacco kāmabhogī adhammena bhoge pariyesati sāhasena;}}\\
\begin{addmargin}[1em]{2em}
\setstretch{.5}
{\PaliGlossB{Next, a pleasure seeker seeks wealth using illegitimate, coercive means.}}\\
\end{addmargin}
\end{absolutelynopagebreak}

\begin{absolutelynopagebreak}
\setstretch{.7}
{\PaliGlossA{adhammena bhoge pariyesitvā sāhasena attānaṃ sukheti pīṇeti, na saṃvibhajati na puññāni karoti. (2)}}\\
\begin{addmargin}[1em]{2em}
\setstretch{.5}
{\PaliGlossB{They make themselves happy and pleased, but don’t share it and make merit.}}\\
\end{addmargin}
\end{absolutelynopagebreak}

\begin{absolutelynopagebreak}
\setstretch{.7}
{\PaliGlossA{idha pana, gahapati, ekacco kāmabhogī adhammena bhoge pariyesati sāhasena;}}\\
\begin{addmargin}[1em]{2em}
\setstretch{.5}
{\PaliGlossB{Next, a pleasure seeker seeks wealth using illegitimate, coercive means.}}\\
\end{addmargin}
\end{absolutelynopagebreak}

\begin{absolutelynopagebreak}
\setstretch{.7}
{\PaliGlossA{adhammena bhoge pariyesitvā sāhasena attānaṃ sukheti pīṇeti saṃvibhajati puññāni karoti. (3)}}\\
\begin{addmargin}[1em]{2em}
\setstretch{.5}
{\PaliGlossB{They make themselves happy and pleased, and they share it and make merit.}}\\
\end{addmargin}
\end{absolutelynopagebreak}

\begin{absolutelynopagebreak}
\setstretch{.7}
{\PaliGlossA{idha pana, gahapati, ekacco kāmabhogī dhammādhammena bhoge pariyesati sāhasenapi asāhasenapi;}}\\
\begin{addmargin}[1em]{2em}
\setstretch{.5}
{\PaliGlossB{Next, a pleasure seeker seeks wealth using means both legitimate and illegitimate, and coercive and non-coercive.}}\\
\end{addmargin}
\end{absolutelynopagebreak}

\begin{absolutelynopagebreak}
\setstretch{.7}
{\PaliGlossA{dhammādhammena bhoge pariyesitvā sāhasenapi asāhasenapi na attānaṃ sukheti na pīṇeti na saṃvibhajati na puññāni karoti. (4)}}\\
\begin{addmargin}[1em]{2em}
\setstretch{.5}
{\PaliGlossB{They don’t make themselves happy and pleased, nor share it and make merit.}}\\
\end{addmargin}
\end{absolutelynopagebreak}

\begin{absolutelynopagebreak}
\setstretch{.7}
{\PaliGlossA{idha pana, gahapati, ekacco kāmabhogī dhammādhammena bhoge pariyesati sāhasenapi asāhasenapi;}}\\
\begin{addmargin}[1em]{2em}
\setstretch{.5}
{\PaliGlossB{Next, a pleasure seeker seeks wealth using means both legitimate and illegitimate, and coercive and non-coercive.}}\\
\end{addmargin}
\end{absolutelynopagebreak}

\begin{absolutelynopagebreak}
\setstretch{.7}
{\PaliGlossA{dhammādhammena bhoge pariyesitvā sāhasenapi asāhasenapi attānaṃ sukheti pīṇeti, na saṃvibhajati na puññāni karoti. (5)}}\\
\begin{addmargin}[1em]{2em}
\setstretch{.5}
{\PaliGlossB{They make themselves happy and pleased, but don't share it and make merit.}}\\
\end{addmargin}
\end{absolutelynopagebreak}

\begin{absolutelynopagebreak}
\setstretch{.7}
{\PaliGlossA{idha pana, gahapati, ekacco kāmabhogī dhammādhammena bhoge pariyesati sāhasenapi asāhasenapi;}}\\
\begin{addmargin}[1em]{2em}
\setstretch{.5}
{\PaliGlossB{Next, a pleasure seeker seeks wealth using means both legitimate and illegitimate, and coercive and non-coercive.}}\\
\end{addmargin}
\end{absolutelynopagebreak}

\begin{absolutelynopagebreak}
\setstretch{.7}
{\PaliGlossA{dhammādhammena bhoge pariyesitvā sāhasenapi asāhasenapi attānaṃ sukheti pīṇeti saṃvibhajati puññāni karoti. (6)}}\\
\begin{addmargin}[1em]{2em}
\setstretch{.5}
{\PaliGlossB{They make themselves happy and pleased, and they share it and make merit.}}\\
\end{addmargin}
\end{absolutelynopagebreak}

\begin{absolutelynopagebreak}
\setstretch{.7}
{\PaliGlossA{idha pana, gahapati, ekacco kāmabhogī dhammena bhoge pariyesati asāhasena;}}\\
\begin{addmargin}[1em]{2em}
\setstretch{.5}
{\PaliGlossB{Next, a pleasure seeker seeks wealth using legitimate, non-coercive means.}}\\
\end{addmargin}
\end{absolutelynopagebreak}

\begin{absolutelynopagebreak}
\setstretch{.7}
{\PaliGlossA{dhammena bhoge pariyesitvā asāhasena na attānaṃ sukheti na pīṇeti na saṃvibhajati na puññāni karoti. (7)}}\\
\begin{addmargin}[1em]{2em}
\setstretch{.5}
{\PaliGlossB{They don’t make themselves happy and pleased, nor share it and make merit.}}\\
\end{addmargin}
\end{absolutelynopagebreak}

\begin{absolutelynopagebreak}
\setstretch{.7}
{\PaliGlossA{idha pana, gahapati, ekacco kāmabhogī dhammena bhoge pariyesati asāhasena;}}\\
\begin{addmargin}[1em]{2em}
\setstretch{.5}
{\PaliGlossB{Next, a pleasure seeker seeks wealth using legitimate, non-coercive means.}}\\
\end{addmargin}
\end{absolutelynopagebreak}

\begin{absolutelynopagebreak}
\setstretch{.7}
{\PaliGlossA{dhammena bhoge pariyesitvā asāhasena attānaṃ sukheti pīṇeti, na saṃvibhajati na puññāni karoti. (8)}}\\
\begin{addmargin}[1em]{2em}
\setstretch{.5}
{\PaliGlossB{They make themselves happy and pleased, but don’t share it and make merit.}}\\
\end{addmargin}
\end{absolutelynopagebreak}

\begin{absolutelynopagebreak}
\setstretch{.7}
{\PaliGlossA{idha pana, gahapati, ekacco kāmabhogī dhammena bhoge pariyesati asāhasena;}}\\
\begin{addmargin}[1em]{2em}
\setstretch{.5}
{\PaliGlossB{Next, a pleasure seeker seeks wealth using legitimate, non-coercive means.}}\\
\end{addmargin}
\end{absolutelynopagebreak}

\begin{absolutelynopagebreak}
\setstretch{.7}
{\PaliGlossA{dhammena bhoge pariyesitvā asāhasena attānaṃ sukheti pīṇeti saṃvibhajati puññāni karoti.}}\\
\begin{addmargin}[1em]{2em}
\setstretch{.5}
{\PaliGlossB{They make themselves happy and pleased, and they share it and make merit.}}\\
\end{addmargin}
\end{absolutelynopagebreak}

\begin{absolutelynopagebreak}
\setstretch{.7}
{\PaliGlossA{te ca bhoge gathito mucchito ajjhosanno anādīnavadassāvī anissaraṇapañño paribhuñjati. (9)}}\\
\begin{addmargin}[1em]{2em}
\setstretch{.5}
{\PaliGlossB{But they enjoy that wealth tied, infatuated, attached, blind to the drawbacks, and not understanding the escape.}}\\
\end{addmargin}
\end{absolutelynopagebreak}

\begin{absolutelynopagebreak}
\setstretch{.7}
{\PaliGlossA{idha pana, gahapati, ekacco kāmabhogī dhammena bhoge pariyesati asāhasena;}}\\
\begin{addmargin}[1em]{2em}
\setstretch{.5}
{\PaliGlossB{Next, a pleasure seeker seeks wealth using legitimate, non-coercive means.}}\\
\end{addmargin}
\end{absolutelynopagebreak}

\begin{absolutelynopagebreak}
\setstretch{.7}
{\PaliGlossA{dhammena bhoge pariyesitvā asāhasena attānaṃ sukheti pīṇeti saṃvibhajati puññāni karoti.}}\\
\begin{addmargin}[1em]{2em}
\setstretch{.5}
{\PaliGlossB{They make themselves happy and pleased, and they share it and make merit.}}\\
\end{addmargin}
\end{absolutelynopagebreak}

\begin{absolutelynopagebreak}
\setstretch{.7}
{\PaliGlossA{te ca bhoge agathito amucchito anajjhosanno ādīnavadassāvī nissaraṇapañño paribhuñjati. (10)}}\\
\begin{addmargin}[1em]{2em}
\setstretch{.5}
{\PaliGlossB{And they enjoy that wealth untied, uninfatuated, unattached, seeing the drawbacks, and understanding the escape.}}\\
\end{addmargin}
\end{absolutelynopagebreak}

\begin{absolutelynopagebreak}
\setstretch{.7}
{\PaliGlossA{tatra, gahapati, yvāyaṃ kāmabhogī adhammena bhoge pariyesati sāhasena, adhammena bhoge pariyesitvā sāhasena na attānaṃ sukheti na pīṇeti na saṃvibhajati na puññāni karoti, ayaṃ, gahapati, kāmabhogī tīhi ṭhānehi gārayho.}}\\
\begin{addmargin}[1em]{2em}
\setstretch{.5}
{\PaliGlossB{Now, consider the pleasure seeker who seeks wealth using illegitimate, coercive means, and who doesn’t make themselves happy and pleased, nor share it and make merit. They may be criticized on three grounds.}}\\
\end{addmargin}
\end{absolutelynopagebreak}

\begin{absolutelynopagebreak}
\setstretch{.7}
{\PaliGlossA{‘adhammena bhoge pariyesati sāhasenā’ti, iminā paṭhamena ṭhānena gārayho.}}\\
\begin{addmargin}[1em]{2em}
\setstretch{.5}
{\PaliGlossB{They seek for wealth using illegitimate, coercive means. This is the first ground for criticism.}}\\
\end{addmargin}
\end{absolutelynopagebreak}

\begin{absolutelynopagebreak}
\setstretch{.7}
{\PaliGlossA{‘na attānaṃ sukheti na pīṇetī’ti, iminā dutiyena ṭhānena gārayho.}}\\
\begin{addmargin}[1em]{2em}
\setstretch{.5}
{\PaliGlossB{They don’t make themselves happy and pleased. This is the second ground for criticism.}}\\
\end{addmargin}
\end{absolutelynopagebreak}

\begin{absolutelynopagebreak}
\setstretch{.7}
{\PaliGlossA{‘na saṃvibhajati na puññāni karotī’ti, iminā tatiyena ṭhānena gārayho.}}\\
\begin{addmargin}[1em]{2em}
\setstretch{.5}
{\PaliGlossB{They don’t share it and make merit. This is the third ground for criticism.}}\\
\end{addmargin}
\end{absolutelynopagebreak}

\begin{absolutelynopagebreak}
\setstretch{.7}
{\PaliGlossA{ayaṃ, gahapati, kāmabhogī imehi tīhi ṭhānehi gārayho. (1)}}\\
\begin{addmargin}[1em]{2em}
\setstretch{.5}
{\PaliGlossB{This pleasure seeker may be criticized on these three grounds.}}\\
\end{addmargin}
\end{absolutelynopagebreak}

\begin{absolutelynopagebreak}
\setstretch{.7}
{\PaliGlossA{tatra, gahapati, yvāyaṃ kāmabhogī adhammena bhoge pariyesati sāhasena, adhammena bhoge pariyesitvā sāhasena attānaṃ sukheti pīṇeti na saṃvibhajati na puññāni karoti, ayaṃ, gahapati, kāmabhogī dvīhi ṭhānehi gārayho ekena ṭhānena pāsaṃso.}}\\
\begin{addmargin}[1em]{2em}
\setstretch{.5}
{\PaliGlossB{Now, consider the pleasure seeker who seeks wealth using illegitimate, coercive means, and who makes themselves happy and pleased, but doesn’t share it and make merit. They may be criticized on two grounds, and praised on one.}}\\
\end{addmargin}
\end{absolutelynopagebreak}

\begin{absolutelynopagebreak}
\setstretch{.7}
{\PaliGlossA{‘adhammena bhoge pariyesati sāhasenā’ti, iminā paṭhamena ṭhānena gārayho.}}\\
\begin{addmargin}[1em]{2em}
\setstretch{.5}
{\PaliGlossB{They seek for wealth using illegitimate, coercive means. This is the first ground for criticism.}}\\
\end{addmargin}
\end{absolutelynopagebreak}

\begin{absolutelynopagebreak}
\setstretch{.7}
{\PaliGlossA{‘attānaṃ sukheti pīṇetī’ti, iminā ekena ṭhānena pāsaṃso.}}\\
\begin{addmargin}[1em]{2em}
\setstretch{.5}
{\PaliGlossB{They make themselves happy and pleased. This is the one ground for praise.}}\\
\end{addmargin}
\end{absolutelynopagebreak}

\begin{absolutelynopagebreak}
\setstretch{.7}
{\PaliGlossA{‘na saṃvibhajati na puññāni karotī’ti iminā dutiyena ṭhānena gārayho.}}\\
\begin{addmargin}[1em]{2em}
\setstretch{.5}
{\PaliGlossB{They don’t share it and make merit. This is the second ground for criticism.}}\\
\end{addmargin}
\end{absolutelynopagebreak}

\begin{absolutelynopagebreak}
\setstretch{.7}
{\PaliGlossA{ayaṃ, gahapati, kāmabhogī imehi dvīhi ṭhānehi gārayho iminā ekena ṭhānena pāsaṃso. (2)}}\\
\begin{addmargin}[1em]{2em}
\setstretch{.5}
{\PaliGlossB{This pleasure seeker may be criticized on these two grounds, and praised on this one.}}\\
\end{addmargin}
\end{absolutelynopagebreak}

\begin{absolutelynopagebreak}
\setstretch{.7}
{\PaliGlossA{tatra, gahapati, yvāyaṃ kāmabhogī adhammena bhoge pariyesati sāhasena, adhammena bhoge pariyesitvā sāhasena attānaṃ sukheti pīṇeti saṃvibhajati puññāni karoti, ayaṃ, gahapati, kāmabhogī ekena ṭhānena gārayho dvīhi ṭhānehi pāsaṃso.}}\\
\begin{addmargin}[1em]{2em}
\setstretch{.5}
{\PaliGlossB{Now, consider the pleasure seeker who seeks wealth using illegitimate, coercive means, and who makes themselves happy and pleased, and shares it and makes merit. They may be criticized on one ground, and praised on two.}}\\
\end{addmargin}
\end{absolutelynopagebreak}

\begin{absolutelynopagebreak}
\setstretch{.7}
{\PaliGlossA{‘adhammena bhoge pariyesati sāhasenā’ti, iminā ekena ṭhānena gārayho.}}\\
\begin{addmargin}[1em]{2em}
\setstretch{.5}
{\PaliGlossB{They seek for wealth using illegitimate, coercive means. This is the one ground for criticism.}}\\
\end{addmargin}
\end{absolutelynopagebreak}

\begin{absolutelynopagebreak}
\setstretch{.7}
{\PaliGlossA{‘attānaṃ sukheti pīṇetī’ti, iminā paṭhamena ṭhānena pāsaṃso.}}\\
\begin{addmargin}[1em]{2em}
\setstretch{.5}
{\PaliGlossB{They make themselves happy and pleased. This is the first ground for praise.}}\\
\end{addmargin}
\end{absolutelynopagebreak}

\begin{absolutelynopagebreak}
\setstretch{.7}
{\PaliGlossA{‘saṃvibhajati puññāni karotī’ti, iminā dutiyena ṭhānena pāsaṃso.}}\\
\begin{addmargin}[1em]{2em}
\setstretch{.5}
{\PaliGlossB{They share it and make merit. This is the second ground for praise.}}\\
\end{addmargin}
\end{absolutelynopagebreak}

\begin{absolutelynopagebreak}
\setstretch{.7}
{\PaliGlossA{ayaṃ, gahapati, kāmabhogī iminā ekena ṭhānena gārayho, imehi dvīhi ṭhānehi pāsaṃso. (3)}}\\
\begin{addmargin}[1em]{2em}
\setstretch{.5}
{\PaliGlossB{This pleasure seeker may be criticized on this one ground, and praised on these two.}}\\
\end{addmargin}
\end{absolutelynopagebreak}

\begin{absolutelynopagebreak}
\setstretch{.7}
{\PaliGlossA{tatra, gahapati, yvāyaṃ kāmabhogī dhammādhammena bhoge pariyesati sāhasenapi asāhasenapi, dhammādhammena bhoge pariyesitvā sāhasenapi asāhasenapi na attānaṃ sukheti na pīṇeti na saṃvibhajati na puññāni karoti, ayaṃ, gahapati, kāmabhogī ekena ṭhānena pāsaṃso tīhi ṭhānehi gārayho.}}\\
\begin{addmargin}[1em]{2em}
\setstretch{.5}
{\PaliGlossB{Now, consider the pleasure seeker who seeks wealth using means both legitimate and illegitimate, and coercive and non-coercive, and who doesn’t make themselves happy and pleased, nor share it and make merit. They may be praised on one ground, and criticized on three.}}\\
\end{addmargin}
\end{absolutelynopagebreak}

\begin{absolutelynopagebreak}
\setstretch{.7}
{\PaliGlossA{‘dhammena bhoge pariyesati asāhasenā’ti, iminā ekena ṭhānena pāsaṃso.}}\\
\begin{addmargin}[1em]{2em}
\setstretch{.5}
{\PaliGlossB{They seek for wealth using legitimate, non-coercive means. This is the one ground for praise.}}\\
\end{addmargin}
\end{absolutelynopagebreak}

\begin{absolutelynopagebreak}
\setstretch{.7}
{\PaliGlossA{‘adhammena bhoge pariyesati sāhasenā’ti, iminā paṭhamena ṭhānena gārayho.}}\\
\begin{addmargin}[1em]{2em}
\setstretch{.5}
{\PaliGlossB{They seek for wealth using illegitimate, coercive means. This is the first ground for criticism.}}\\
\end{addmargin}
\end{absolutelynopagebreak}

\begin{absolutelynopagebreak}
\setstretch{.7}
{\PaliGlossA{‘na attānaṃ sukheti na pīṇetī’ti, iminā dutiyena ṭhānena gārayho.}}\\
\begin{addmargin}[1em]{2em}
\setstretch{.5}
{\PaliGlossB{They don’t make themselves happy and pleased. This is the second ground for criticism.}}\\
\end{addmargin}
\end{absolutelynopagebreak}

\begin{absolutelynopagebreak}
\setstretch{.7}
{\PaliGlossA{‘na saṃvibhajati na puññāni karotī’ti, iminā tatiyena ṭhānena gārayho.}}\\
\begin{addmargin}[1em]{2em}
\setstretch{.5}
{\PaliGlossB{They don’t share it and make merit. This is the third ground for criticism.}}\\
\end{addmargin}
\end{absolutelynopagebreak}

\begin{absolutelynopagebreak}
\setstretch{.7}
{\PaliGlossA{ayaṃ, gahapati, kāmabhogī iminā ekena ṭhānena pāsaṃso imehi tīhi ṭhānehi gārayho. (4)}}\\
\begin{addmargin}[1em]{2em}
\setstretch{.5}
{\PaliGlossB{This pleasure seeker may be praised on this one ground, and criticized on these three.}}\\
\end{addmargin}
\end{absolutelynopagebreak}

\begin{absolutelynopagebreak}
\setstretch{.7}
{\PaliGlossA{tatra, gahapati, yvāyaṃ kāmabhogī dhammādhammena bhoge pariyesati sāhasenapi asāhasenapi, dhammādhammena bhoge pariyesitvā sāhasenapi asāhasenapi attānaṃ sukheti pīṇeti na saṃvibhajati na puññāni karoti, ayaṃ, gahapati, kāmabhogī dvīhi ṭhānehi pāsaṃso dvīhi ṭhānehi gārayho.}}\\
\begin{addmargin}[1em]{2em}
\setstretch{.5}
{\PaliGlossB{Now, consider the pleasure seeker who seeks wealth using means both legitimate and illegitimate, and coercive and non-coercive, and who makes themselves happy and pleased, but doesn’t share it and make merit. They may be praised on two grounds, and criticized on two.}}\\
\end{addmargin}
\end{absolutelynopagebreak}

\begin{absolutelynopagebreak}
\setstretch{.7}
{\PaliGlossA{‘dhammena bhoge pariyesati asāhasenā’ti, iminā paṭhamena ṭhānena pāsaṃso.}}\\
\begin{addmargin}[1em]{2em}
\setstretch{.5}
{\PaliGlossB{They seek for wealth using legitimate, non-coercive means. This is the first ground for praise.}}\\
\end{addmargin}
\end{absolutelynopagebreak}

\begin{absolutelynopagebreak}
\setstretch{.7}
{\PaliGlossA{‘adhammena bhoge pariyesati sāhasenā’ti, iminā paṭhamena ṭhānena gārayho.}}\\
\begin{addmargin}[1em]{2em}
\setstretch{.5}
{\PaliGlossB{They seek for wealth using illegitimate, coercive means. This is the first ground for criticism.}}\\
\end{addmargin}
\end{absolutelynopagebreak}

\begin{absolutelynopagebreak}
\setstretch{.7}
{\PaliGlossA{‘attānaṃ sukheti pīṇetī’ti, iminā dutiyena ṭhānena pāsaṃso.}}\\
\begin{addmargin}[1em]{2em}
\setstretch{.5}
{\PaliGlossB{They make themselves happy and pleased. This is the second ground for praise.}}\\
\end{addmargin}
\end{absolutelynopagebreak}

\begin{absolutelynopagebreak}
\setstretch{.7}
{\PaliGlossA{‘na saṃvibhajati na puññāni karotī’ti, iminā dutiyena ṭhānena gārayho.}}\\
\begin{addmargin}[1em]{2em}
\setstretch{.5}
{\PaliGlossB{They don’t share it and make merit. This is the second ground for criticism.}}\\
\end{addmargin}
\end{absolutelynopagebreak}

\begin{absolutelynopagebreak}
\setstretch{.7}
{\PaliGlossA{ayaṃ, gahapati, kāmabhogī imehi dvīhi ṭhānehi pāsaṃso imehi dvīhi ṭhānehi gārayho. (5)}}\\
\begin{addmargin}[1em]{2em}
\setstretch{.5}
{\PaliGlossB{This pleasure seeker may be praised on these two grounds, and criticized on these two.}}\\
\end{addmargin}
\end{absolutelynopagebreak}

\begin{absolutelynopagebreak}
\setstretch{.7}
{\PaliGlossA{tatra, gahapati, yvāyaṃ kāmabhogī dhammādhammena bhoge pariyesati sāhasenapi asāhasenapi, dhammādhammena bhoge pariyesitvā sāhasenapi asāhasenapi attānaṃ sukheti pīṇeti saṃvibhajati puññāni karoti, ayaṃ, gahapati, kāmabhogī tīhi ṭhānehi pāsaṃso ekena ṭhānena gārayho.}}\\
\begin{addmargin}[1em]{2em}
\setstretch{.5}
{\PaliGlossB{Now, consider the pleasure seeker who seeks wealth using means both legitimate and illegitimate, and coercive and non-coercive, and who makes themselves happy and pleased, and shares it and make merit. They may be praised on three grounds, and criticized on one.}}\\
\end{addmargin}
\end{absolutelynopagebreak}

\begin{absolutelynopagebreak}
\setstretch{.7}
{\PaliGlossA{‘dhammena bhoge pariyesati asāhasenā’ti, iminā paṭhamena ṭhānena pāsaṃso.}}\\
\begin{addmargin}[1em]{2em}
\setstretch{.5}
{\PaliGlossB{They seek for wealth using legitimate, non-coercive means. This is the first ground for praise.}}\\
\end{addmargin}
\end{absolutelynopagebreak}

\begin{absolutelynopagebreak}
\setstretch{.7}
{\PaliGlossA{‘adhammena bhoge pariyesati sāhasenā’ti, iminā ekena ṭhānena gārayho.}}\\
\begin{addmargin}[1em]{2em}
\setstretch{.5}
{\PaliGlossB{They seek for wealth using illegitimate, coercive means. This is the one ground for criticism.}}\\
\end{addmargin}
\end{absolutelynopagebreak}

\begin{absolutelynopagebreak}
\setstretch{.7}
{\PaliGlossA{‘attānaṃ sukheti pīṇetī’ti, iminā dutiyena ṭhānena pāsaṃso.}}\\
\begin{addmargin}[1em]{2em}
\setstretch{.5}
{\PaliGlossB{They make themselves happy and pleased. This is the second ground for praise.}}\\
\end{addmargin}
\end{absolutelynopagebreak}

\begin{absolutelynopagebreak}
\setstretch{.7}
{\PaliGlossA{‘saṃvibhajati puññāni karotī’ti, iminā tatiyena ṭhānena pāsaṃso.}}\\
\begin{addmargin}[1em]{2em}
\setstretch{.5}
{\PaliGlossB{They share it and make merit. This is the third ground for praise.}}\\
\end{addmargin}
\end{absolutelynopagebreak}

\begin{absolutelynopagebreak}
\setstretch{.7}
{\PaliGlossA{ayaṃ, gahapati, kāmabhogī imehi tīhi ṭhānehi pāsaṃso iminā ekena ṭhānena gārayho. (6)}}\\
\begin{addmargin}[1em]{2em}
\setstretch{.5}
{\PaliGlossB{This pleasure seeker may be praised on these three grounds, and criticized on this one.}}\\
\end{addmargin}
\end{absolutelynopagebreak}

\begin{absolutelynopagebreak}
\setstretch{.7}
{\PaliGlossA{tatra, gahapati, yvāyaṃ kāmabhogī dhammena bhoge pariyesati asāhasena, dhammena bhoge pariyesitvā asāhasena na attānaṃ sukheti na pīṇeti na saṃvibhajati na puññāni karoti, ayaṃ, gahapati, kāmabhogī ekena ṭhānena pāsaṃso dvīhi ṭhānehi gārayho.}}\\
\begin{addmargin}[1em]{2em}
\setstretch{.5}
{\PaliGlossB{Now, consider the pleasure seeker who seeks wealth using legitimate, non-coercive means, and who doesn’t make themselves happy and pleased, nor share it and make merit. They may be praised on one ground and criticized on two.}}\\
\end{addmargin}
\end{absolutelynopagebreak}

\begin{absolutelynopagebreak}
\setstretch{.7}
{\PaliGlossA{‘dhammena bhoge pariyesati asāhasenā’ti, iminā ekena ṭhānena pāsaṃso.}}\\
\begin{addmargin}[1em]{2em}
\setstretch{.5}
{\PaliGlossB{They seek for wealth using legitimate, non-coercive means. This is the one ground for praise.}}\\
\end{addmargin}
\end{absolutelynopagebreak}

\begin{absolutelynopagebreak}
\setstretch{.7}
{\PaliGlossA{‘na attānaṃ sukheti na pīṇetī’ti, iminā paṭhamena ṭhānena gārayho.}}\\
\begin{addmargin}[1em]{2em}
\setstretch{.5}
{\PaliGlossB{They don’t make themselves happy and pleased. This is the first ground for criticism.}}\\
\end{addmargin}
\end{absolutelynopagebreak}

\begin{absolutelynopagebreak}
\setstretch{.7}
{\PaliGlossA{‘na saṃvibhajati na puññāni karotī’ti, iminā dutiyena ṭhānena gārayho.}}\\
\begin{addmargin}[1em]{2em}
\setstretch{.5}
{\PaliGlossB{They don’t share it and make merit. This is the second ground for criticism.}}\\
\end{addmargin}
\end{absolutelynopagebreak}

\begin{absolutelynopagebreak}
\setstretch{.7}
{\PaliGlossA{ayaṃ, gahapati, kāmabhogī iminā ekena ṭhānena pāsaṃso imehi dvīhi ṭhānehi gārayho. (7)}}\\
\begin{addmargin}[1em]{2em}
\setstretch{.5}
{\PaliGlossB{This pleasure seeker may be praised on this one ground, and criticized on these two.}}\\
\end{addmargin}
\end{absolutelynopagebreak}

\begin{absolutelynopagebreak}
\setstretch{.7}
{\PaliGlossA{tatra, gahapati, yvāyaṃ kāmabhogī dhammena bhoge pariyesati asāhasena, dhammena bhoge pariyesitvā asāhasena attānaṃ sukheti pīṇeti na saṃvibhajati na puññāni karoti, ayaṃ, gahapati, kāmabhogī dvīhi ṭhānehi pāsaṃso ekena ṭhānena gārayho.}}\\
\begin{addmargin}[1em]{2em}
\setstretch{.5}
{\PaliGlossB{Now, consider the pleasure seeker who seeks wealth using legitimate, non-coercive means, and who makes themselves happy and pleased, but doesn’t share it and make merit. They may be praised on two grounds and criticized on one.}}\\
\end{addmargin}
\end{absolutelynopagebreak}

\begin{absolutelynopagebreak}
\setstretch{.7}
{\PaliGlossA{‘dhammena bhoge pariyesati asāhasenā’ti, iminā paṭhamena ṭhānena pāsaṃso.}}\\
\begin{addmargin}[1em]{2em}
\setstretch{.5}
{\PaliGlossB{They seek for wealth using legitimate, non-coercive means. This is the first ground for praise.}}\\
\end{addmargin}
\end{absolutelynopagebreak}

\begin{absolutelynopagebreak}
\setstretch{.7}
{\PaliGlossA{‘attānaṃ sukheti pīṇetī’ti, iminā dutiyena ṭhānena pāsaṃso.}}\\
\begin{addmargin}[1em]{2em}
\setstretch{.5}
{\PaliGlossB{They make themselves happy and pleased. This is the second ground for praise.}}\\
\end{addmargin}
\end{absolutelynopagebreak}

\begin{absolutelynopagebreak}
\setstretch{.7}
{\PaliGlossA{‘na saṃvibhajati na puññāni karotī’ti iminā ekena ṭhānena gārayho.}}\\
\begin{addmargin}[1em]{2em}
\setstretch{.5}
{\PaliGlossB{They don’t share it and make merit. This is the one ground for criticism.}}\\
\end{addmargin}
\end{absolutelynopagebreak}

\begin{absolutelynopagebreak}
\setstretch{.7}
{\PaliGlossA{ayaṃ, gahapati, kāmabhogī imehi dvīhi ṭhānehi pāsaṃso iminā ekena ṭhānena gārayho. (8)}}\\
\begin{addmargin}[1em]{2em}
\setstretch{.5}
{\PaliGlossB{This pleasure seeker may be praised on these two grounds, and criticized on this one.}}\\
\end{addmargin}
\end{absolutelynopagebreak}

\begin{absolutelynopagebreak}
\setstretch{.7}
{\PaliGlossA{tatra, gahapati yvāyaṃ kāmabhogī dhammena bhoge pariyesati asāhasena, dhammena bhoge pariyesitvā asāhasena attānaṃ sukheti pīṇeti saṃvibhajati puññāni karoti, te ca bhoge gathito mucchito ajjhosanno anādīnavadassāvī anissaraṇapañño paribhuñjati, ayaṃ, gahapati, kāmabhogī tīhi ṭhānehi pāsaṃso ekena ṭhānena gārayho.}}\\
\begin{addmargin}[1em]{2em}
\setstretch{.5}
{\PaliGlossB{Now, consider the pleasure seeker who seeks wealth using legitimate, non-coercive means, and who makes themselves happy and pleased, and shares it and makes merit. But they enjoy that wealth tied, infatuated, attached, blind to the drawbacks, and not understanding the escape. They may be praised on three grounds and criticized on one.}}\\
\end{addmargin}
\end{absolutelynopagebreak}

\begin{absolutelynopagebreak}
\setstretch{.7}
{\PaliGlossA{‘dhammena bhoge pariyesati asāhasenā’ti, iminā paṭhamena ṭhānena pāsaṃso.}}\\
\begin{addmargin}[1em]{2em}
\setstretch{.5}
{\PaliGlossB{They seek for wealth using legitimate, non-coercive means. This is the first ground for praise.}}\\
\end{addmargin}
\end{absolutelynopagebreak}

\begin{absolutelynopagebreak}
\setstretch{.7}
{\PaliGlossA{‘attānaṃ sukheti pīṇetī’ti, iminā dutiyena ṭhānena pāsaṃso.}}\\
\begin{addmargin}[1em]{2em}
\setstretch{.5}
{\PaliGlossB{They make themselves happy and pleased. This is the second ground for praise.}}\\
\end{addmargin}
\end{absolutelynopagebreak}

\begin{absolutelynopagebreak}
\setstretch{.7}
{\PaliGlossA{‘saṃvibhajati puññāni karotī’ti, iminā tatiyena ṭhānena pāsaṃso.}}\\
\begin{addmargin}[1em]{2em}
\setstretch{.5}
{\PaliGlossB{They share it and make merit. This is the third ground for praise.}}\\
\end{addmargin}
\end{absolutelynopagebreak}

\begin{absolutelynopagebreak}
\setstretch{.7}
{\PaliGlossA{‘te ca bhoge gathito mucchito ajjhosanno anādīnavadassāvī anissaraṇapañño paribhuñjatī’ti, iminā ekena ṭhānena gārayho.}}\\
\begin{addmargin}[1em]{2em}
\setstretch{.5}
{\PaliGlossB{They enjoy that wealth tied, infatuated, attached, blind to the drawbacks, and not understanding the escape. This is the one ground for criticism.}}\\
\end{addmargin}
\end{absolutelynopagebreak}

\begin{absolutelynopagebreak}
\setstretch{.7}
{\PaliGlossA{ayaṃ, gahapati, kāmabhogī imehi tīhi ṭhānehi pāsaṃso iminā ekena ṭhānena gārayho. (9)}}\\
\begin{addmargin}[1em]{2em}
\setstretch{.5}
{\PaliGlossB{This pleasure seeker may be praised on these three grounds, and criticized on this one.}}\\
\end{addmargin}
\end{absolutelynopagebreak}

\begin{absolutelynopagebreak}
\setstretch{.7}
{\PaliGlossA{tatra, gahapati, yvāyaṃ kāmabhogī dhammena bhoge pariyesati asāhasena, dhammena bhoge pariyesitvā asāhasena attānaṃ sukheti pīṇeti saṃvibhajati puññāni karoti, te ca bhoge agathito amucchito anajjhosanno ādīnavadassāvī nissaraṇapañño paribhuñjati, ayaṃ, gahapati, kāmabhogī catūhi ṭhānehi pāsaṃso.}}\\
\begin{addmargin}[1em]{2em}
\setstretch{.5}
{\PaliGlossB{Now, consider the pleasure seeker who seeks wealth using legitimate, non-coercive means, and who makes themselves happy and pleased, and shares it and makes merit. And they enjoy that wealth untied, uninfatuated, unattached, seeing the drawbacks, and understanding the escape. They may be praised on four grounds.}}\\
\end{addmargin}
\end{absolutelynopagebreak}

\begin{absolutelynopagebreak}
\setstretch{.7}
{\PaliGlossA{‘dhammena bhoge pariyesati asāhasenā’ti, iminā paṭhamena ṭhānena pāsaṃso.}}\\
\begin{addmargin}[1em]{2em}
\setstretch{.5}
{\PaliGlossB{They seek for wealth using legitimate, non-coercive means. This is the first ground for praise.}}\\
\end{addmargin}
\end{absolutelynopagebreak}

\begin{absolutelynopagebreak}
\setstretch{.7}
{\PaliGlossA{‘attānaṃ sukheti pīṇetī’ti, iminā dutiyena ṭhānena pāsaṃso.}}\\
\begin{addmargin}[1em]{2em}
\setstretch{.5}
{\PaliGlossB{They make themselves happy and pleased. This is the second ground for praise.}}\\
\end{addmargin}
\end{absolutelynopagebreak}

\begin{absolutelynopagebreak}
\setstretch{.7}
{\PaliGlossA{‘saṃvibhajati puññāni karotī’ti, iminā tatiyena ṭhānena pāsaṃso.}}\\
\begin{addmargin}[1em]{2em}
\setstretch{.5}
{\PaliGlossB{They share it and make merit. This is the third ground for praise.}}\\
\end{addmargin}
\end{absolutelynopagebreak}

\begin{absolutelynopagebreak}
\setstretch{.7}
{\PaliGlossA{‘te ca bhoge agathito amucchito anajjhosanno ādīnavadassāvī nissaraṇapañño paribhuñjatī’ti, iminā catutthena ṭhānena pāsaṃso.}}\\
\begin{addmargin}[1em]{2em}
\setstretch{.5}
{\PaliGlossB{They enjoy that wealth untied, uninfatuated, unattached, seeing the drawbacks, and understanding the escape. This is the fourth ground for praise.}}\\
\end{addmargin}
\end{absolutelynopagebreak}

\begin{absolutelynopagebreak}
\setstretch{.7}
{\PaliGlossA{ayaṃ, gahapati, kāmabhogī imehi catūhi ṭhānehi pāsaṃso. (10)}}\\
\begin{addmargin}[1em]{2em}
\setstretch{.5}
{\PaliGlossB{This pleasure seeker may be praised on these four grounds.}}\\
\end{addmargin}
\end{absolutelynopagebreak}

\begin{absolutelynopagebreak}
\setstretch{.7}
{\PaliGlossA{ime kho, gahapati, dasa kāmabhogī santo saṃvijjamānā lokasmiṃ.}}\\
\begin{addmargin}[1em]{2em}
\setstretch{.5}
{\PaliGlossB{These are the ten pleasure seekers found in the world.}}\\
\end{addmargin}
\end{absolutelynopagebreak}

\begin{absolutelynopagebreak}
\setstretch{.7}
{\PaliGlossA{imesaṃ kho, gahapati, dasannaṃ kāmabhogīnaṃ yvāyaṃ kāmabhogī dhammena bhoge pariyesati asāhasena, dhammena bhoge pariyesitvā asāhasena attānaṃ sukheti pīṇeti saṃvibhajati puññāni karoti, te ca bhoge agathito amucchito anajjhosanno ādīnavadassāvī nissaraṇapañño paribhuñjati, ayaṃ imesaṃ dasannaṃ kāmabhogīnaṃ aggo ca seṭṭho ca pāmokkho ca uttamo ca pavaro ca.}}\\
\begin{addmargin}[1em]{2em}
\setstretch{.5}
{\PaliGlossB{The pleasure seeker who seeks wealth using legitimate, non-coercive means, who makes themselves happy and pleased, and shares it and makes merit, and who uses that wealth untied, uninfatuated, unattached, seeing the drawbacks, and understanding the escape is the foremost, best, chief, highest, and finest of the ten.}}\\
\end{addmargin}
\end{absolutelynopagebreak}

\begin{absolutelynopagebreak}
\setstretch{.7}
{\PaliGlossA{seyyathāpi, gahapati, gavā khīraṃ, khīramhā dadhi, dadhimhā navanītaṃ, navanītamhā sappi, sappimhā sappimaṇḍo. sappimaṇḍo tattha aggamakkhāyati.}}\\
\begin{addmargin}[1em]{2em}
\setstretch{.5}
{\PaliGlossB{From a cow comes milk, from milk comes curds, from curds come butter, from butter comes ghee, and from ghee comes cream of ghee. And the cream of ghee is said to be the best of these.}}\\
\end{addmargin}
\end{absolutelynopagebreak}

\begin{absolutelynopagebreak}
\setstretch{.7}
{\PaliGlossA{evamevaṃ kho, gahapati, imesaṃ dasannaṃ kāmabhogīnaṃ yvāyaṃ kāmabhogī dhammena bhoge pariyesati asāhasena, dhammena bhoge pariyesitvā asāhasena attānaṃ sukheti pīṇeti saṃvibhajati puññāni karoti, te ca bhoge agathito amucchito anajjhosanno ādīnavadassāvī nissaraṇapañño paribhuñjati, ayaṃ imesaṃ dasannaṃ kāmabhogīnaṃ aggo ca seṭṭho ca pāmokkho ca uttamo ca pavaro cā”ti.}}\\
\begin{addmargin}[1em]{2em}
\setstretch{.5}
{\PaliGlossB{In the same way, the pleasure seeker who seeks wealth using legitimate, non-coercive means, who makes themselves happy and pleased, and shares it and makes merit, and who uses that wealth untied, uninfatuated, unattached, seeing the drawbacks, and understanding the escape is the foremost, best, chief, highest, and finest of the ten.”}}\\
\end{addmargin}
\end{absolutelynopagebreak}

\begin{absolutelynopagebreak}
\setstretch{.7}
{\PaliGlossA{paṭhamaṃ.}}\\
\begin{addmargin}[1em]{2em}
\setstretch{.5}
{\PaliGlossB{    -}}\\
\end{addmargin}
\end{absolutelynopagebreak}
