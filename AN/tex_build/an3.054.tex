
\begin{absolutelynopagebreak}
\setstretch{.7}
{\PaliGlossA{aṅguttara nikāya 3}}\\
\begin{addmargin}[1em]{2em}
\setstretch{.5}
{\PaliGlossB{Numbered Discourses 3}}\\
\end{addmargin}
\end{absolutelynopagebreak}

\begin{absolutelynopagebreak}
\setstretch{.7}
{\PaliGlossA{6. brāhmaṇavagga}}\\
\begin{addmargin}[1em]{2em}
\setstretch{.5}
{\PaliGlossB{6. Brahmins}}\\
\end{addmargin}
\end{absolutelynopagebreak}

\begin{absolutelynopagebreak}
\setstretch{.7}
{\PaliGlossA{54. paribbājakasutta}}\\
\begin{addmargin}[1em]{2em}
\setstretch{.5}
{\PaliGlossB{54. A Wanderer}}\\
\end{addmargin}
\end{absolutelynopagebreak}

\begin{absolutelynopagebreak}
\setstretch{.7}
{\PaliGlossA{atha kho aññataro brāhmaṇaparibbājako yena bhagavā tenupasaṅkami; upasaṅkamitvā … pe … ekamantaṃ nisinno kho so brāhmaṇaparibbājako bhagavantaṃ etadavoca:}}\\
\begin{addmargin}[1em]{2em}
\setstretch{.5}
{\PaliGlossB{Then a brahmin wanderer went up to the Buddha … Seated to one side he said to the Buddha:}}\\
\end{addmargin}
\end{absolutelynopagebreak}

\begin{absolutelynopagebreak}
\setstretch{.7}
{\PaliGlossA{“‘sandiṭṭhiko dhammo sandiṭṭhiko dhammo’ti, bho gotama, vuccati.}}\\
\begin{addmargin}[1em]{2em}
\setstretch{.5}
{\PaliGlossB{“Master Gotama, they speak of ‘a teaching visible in this very life’.}}\\
\end{addmargin}
\end{absolutelynopagebreak}

\begin{absolutelynopagebreak}
\setstretch{.7}
{\PaliGlossA{kittāvatā nu kho, bho gotama, sandiṭṭhiko dhammo hoti akāliko ehipassiko opaneyyiko paccattaṃ veditabbo viññūhī”ti?}}\\
\begin{addmargin}[1em]{2em}
\setstretch{.5}
{\PaliGlossB{In what way is the teaching visible in this very life, immediately effective, inviting inspection, relevant, so that sensible people can know it for themselves?”}}\\
\end{addmargin}
\end{absolutelynopagebreak}

\begin{absolutelynopagebreak}
\setstretch{.7}
{\PaliGlossA{“ratto kho, brāhmaṇa, rāgena abhibhūto pariyādinnacitto attabyābādhāyapi ceteti, parabyābādhāyapi ceteti, ubhayabyābādhāyapi ceteti, cetasikampi dukkhaṃ domanassaṃ paṭisaṃvedeti.}}\\
\begin{addmargin}[1em]{2em}
\setstretch{.5}
{\PaliGlossB{“A greedy person, overcome by greed, intends to hurt themselves, hurt others, and hurt both. They experience mental pain and sadness.}}\\
\end{addmargin}
\end{absolutelynopagebreak}

\begin{absolutelynopagebreak}
\setstretch{.7}
{\PaliGlossA{rāge pahīne nevattabyābādhāyapi ceteti, na parabyābādhāyapi ceteti, na ubhayabyābādhāyapi ceteti, na cetasikampi dukkhaṃ domanassaṃ paṭisaṃvedeti.}}\\
\begin{addmargin}[1em]{2em}
\setstretch{.5}
{\PaliGlossB{When greed has been given up, they don’t intend to hurt themselves, hurt others, and hurt both. They don’t experience mental pain and sadness.}}\\
\end{addmargin}
\end{absolutelynopagebreak}

\begin{absolutelynopagebreak}
\setstretch{.7}
{\PaliGlossA{ratto kho, brāhmaṇa, rāgena abhibhūto pariyādinnacitto kāyena duccaritaṃ carati, vācāya duccaritaṃ carati, manasā duccaritaṃ carati.}}\\
\begin{addmargin}[1em]{2em}
\setstretch{.5}
{\PaliGlossB{A greedy person does bad things by way of body, speech, and mind.}}\\
\end{addmargin}
\end{absolutelynopagebreak}

\begin{absolutelynopagebreak}
\setstretch{.7}
{\PaliGlossA{rāge pahīne neva kāyena duccaritaṃ carati, na vācāya duccaritaṃ carati, na manasā duccaritaṃ carati.}}\\
\begin{addmargin}[1em]{2em}
\setstretch{.5}
{\PaliGlossB{When greed has been given up, they don’t do bad things by way of body, speech, and mind.}}\\
\end{addmargin}
\end{absolutelynopagebreak}

\begin{absolutelynopagebreak}
\setstretch{.7}
{\PaliGlossA{ratto kho, brāhmaṇa, rāgena abhibhūto pariyādinnacitto attatthampi yathābhūtaṃ nappajānāti, paratthampi yathābhūtaṃ nappajānāti, ubhayatthampi yathābhūtaṃ nappajānāti.}}\\
\begin{addmargin}[1em]{2em}
\setstretch{.5}
{\PaliGlossB{A greedy person doesn’t truly understand what’s for their own good, the good of another, or the good of both.}}\\
\end{addmargin}
\end{absolutelynopagebreak}

\begin{absolutelynopagebreak}
\setstretch{.7}
{\PaliGlossA{rāge pahīne attatthampi yathābhūtaṃ pajānāti, paratthampi yathābhūtaṃ pajānāti, ubhayatthampi yathābhūtaṃ pajānāti.}}\\
\begin{addmargin}[1em]{2em}
\setstretch{.5}
{\PaliGlossB{When greed has been given up, they truly understand what’s for their own good, the good of another, or the good of both.}}\\
\end{addmargin}
\end{absolutelynopagebreak}

\begin{absolutelynopagebreak}
\setstretch{.7}
{\PaliGlossA{evampi kho, brāhmaṇa, sandiṭṭhiko dhammo hoti … pe ….}}\\
\begin{addmargin}[1em]{2em}
\setstretch{.5}
{\PaliGlossB{This is how the teaching is visible in this very life, immediately effective, inviting inspection, relevant, so that sensible people can know it for themselves.}}\\
\end{addmargin}
\end{absolutelynopagebreak}

\begin{absolutelynopagebreak}
\setstretch{.7}
{\PaliGlossA{duṭṭho kho, brāhmaṇa, dosena … pe …}}\\
\begin{addmargin}[1em]{2em}
\setstretch{.5}
{\PaliGlossB{A hateful person …}}\\
\end{addmargin}
\end{absolutelynopagebreak}

\begin{absolutelynopagebreak}
\setstretch{.7}
{\PaliGlossA{mūḷho kho, brāhmaṇa, mohena abhibhūto pariyādinnacitto attabyābādhāyapi ceteti, parabyābādhāyapi ceteti, ubhayabyābādhāyapi ceteti, cetasikampi dukkhaṃ domanassaṃ paṭisaṃvedeti.}}\\
\begin{addmargin}[1em]{2em}
\setstretch{.5}
{\PaliGlossB{A deluded person, overcome by delusion, intends to hurt themselves, hurt others, and hurt both. They experience mental pain and sadness.}}\\
\end{addmargin}
\end{absolutelynopagebreak}

\begin{absolutelynopagebreak}
\setstretch{.7}
{\PaliGlossA{mohe pahīne nevattabyābādhāyapi ceteti, na parabyābādhāyapi ceteti, na ubhayabyābādhāyapi ceteti, na cetasikaṃ dukkhaṃ domanassaṃ paṭisaṃvedeti.}}\\
\begin{addmargin}[1em]{2em}
\setstretch{.5}
{\PaliGlossB{When delusion has been given up, they don’t intend to hurt themselves, hurt others, and hurt both. They don’t experience mental pain and sadness.}}\\
\end{addmargin}
\end{absolutelynopagebreak}

\begin{absolutelynopagebreak}
\setstretch{.7}
{\PaliGlossA{mūḷho kho, brāhmaṇa, mohena abhibhūto pariyādinnacitto, kāyena duccaritaṃ carati, vācāya duccaritaṃ carati, manasā duccaritaṃ carati.}}\\
\begin{addmargin}[1em]{2em}
\setstretch{.5}
{\PaliGlossB{A deluded person does bad things by way of body, speech, and mind.}}\\
\end{addmargin}
\end{absolutelynopagebreak}

\begin{absolutelynopagebreak}
\setstretch{.7}
{\PaliGlossA{mohe pahīne neva kāyena duccaritaṃ carati, na vācāya duccaritaṃ carati, na manasā duccaritaṃ carati.}}\\
\begin{addmargin}[1em]{2em}
\setstretch{.5}
{\PaliGlossB{When delusion has been given up, they don’t do bad things by way of body, speech, and mind.}}\\
\end{addmargin}
\end{absolutelynopagebreak}

\begin{absolutelynopagebreak}
\setstretch{.7}
{\PaliGlossA{mūḷho kho, brāhmaṇa, mohena abhibhūto pariyādinnacitto attatthampi yathābhūtaṃ nappajānāti, paratthampi yathābhūtaṃ nappajānāti, ubhayatthampi yathābhūtaṃ nappajānāti.}}\\
\begin{addmargin}[1em]{2em}
\setstretch{.5}
{\PaliGlossB{A deluded person doesn’t truly understand what’s for their own good, the good of another, or the good of both.}}\\
\end{addmargin}
\end{absolutelynopagebreak}

\begin{absolutelynopagebreak}
\setstretch{.7}
{\PaliGlossA{mohe pahīne attatthampi yathābhūtaṃ pajānāti, paratthampi yathābhūtaṃ pajānāti, ubhayatthampi yathābhūtaṃ pajānāti.}}\\
\begin{addmargin}[1em]{2em}
\setstretch{.5}
{\PaliGlossB{When delusion has been given up, they truly understand what’s for their own good, the good of another, or the good of both.}}\\
\end{addmargin}
\end{absolutelynopagebreak}

\begin{absolutelynopagebreak}
\setstretch{.7}
{\PaliGlossA{evaṃ kho, brāhmaṇa, sandiṭṭhiko dhammo hoti akāliko ehipassiko opaneyyiko paccattaṃ veditabbo viññūhī”ti.}}\\
\begin{addmargin}[1em]{2em}
\setstretch{.5}
{\PaliGlossB{This, too, is how the teaching is visible in this very life, immediately effective, inviting inspection, relevant, so that sensible people can know it for themselves.”}}\\
\end{addmargin}
\end{absolutelynopagebreak}

\begin{absolutelynopagebreak}
\setstretch{.7}
{\PaliGlossA{“abhikkantaṃ, bho gotama … pe …}}\\
\begin{addmargin}[1em]{2em}
\setstretch{.5}
{\PaliGlossB{“Excellent, Master Gotama! Excellent! …}}\\
\end{addmargin}
\end{absolutelynopagebreak}

\begin{absolutelynopagebreak}
\setstretch{.7}
{\PaliGlossA{upāsakaṃ maṃ bhavaṃ gotamo dhāretu ajjatagge pāṇupetaṃ saraṇaṃ gatan”ti.}}\\
\begin{addmargin}[1em]{2em}
\setstretch{.5}
{\PaliGlossB{From this day forth, may Master Gotama remember me as a lay follower who has gone for refuge for life.”}}\\
\end{addmargin}
\end{absolutelynopagebreak}

\begin{absolutelynopagebreak}
\setstretch{.7}
{\PaliGlossA{catutthaṃ.}}\\
\begin{addmargin}[1em]{2em}
\setstretch{.5}
{\PaliGlossB{    -}}\\
\end{addmargin}
\end{absolutelynopagebreak}
