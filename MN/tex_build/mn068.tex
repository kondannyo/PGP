
\vskip 0.05in
\begin{absolutelynopagebreak}
\setstretch{.7}
{\PaliGlossA{Majjhima Nikāya 68}}\\
\begin{addmargin}[1em]{2em}
\setstretch{.5}
{\PaliGlossB{Middle Discourses 68}}\\
\end{addmargin}
\end{absolutelynopagebreak}

\begin{absolutelynopagebreak}
\setstretch{.7}
{\PaliGlossA{Naḷakapānasutta}}\\
\begin{addmargin}[1em]{2em}
\setstretch{.5}
{\PaliGlossB{At Naḷakapāna}}\\
\end{addmargin}
\end{absolutelynopagebreak}

\vskip 0.05in
\begin{absolutelynopagebreak}
\setstretch{.7}
{\PaliGlossA{1. Evaṃ me sutaṃ—}}\\
\begin{addmargin}[1em]{2em}
\setstretch{.5}
{\PaliGlossB{So I have heard.}}\\
\end{addmargin}
\end{absolutelynopagebreak}

\begin{absolutelynopagebreak}
\setstretch{.7}
{\PaliGlossA{ekaṃ samayaṃ bhagavā kosalesu viharati naḷakapāne palāsavane.}}\\
\begin{addmargin}[1em]{2em}
\setstretch{.5}
{\PaliGlossB{At one time the Buddha was staying in the land of the Kosalans near Naḷakapāna in the Parrot Tree grove.}}\\
\end{addmargin}
\end{absolutelynopagebreak}

\vskip 0.05in
\begin{absolutelynopagebreak}
\setstretch{.7}
{\PaliGlossA{2. Tena kho pana samayena sambahulā abhiññātā abhiññātā kulaputtā bhagavantaṃ uddissa saddhā agārasmā anagāriyaṃ pabbajitā honti—}}\\
\begin{addmargin}[1em]{2em}
\setstretch{.5}
{\PaliGlossB{Now at that time several very well-known gentlemen had gone forth from the lay life to homelessness out of faith in the Buddha—}}\\
\end{addmargin}
\end{absolutelynopagebreak}

\begin{absolutelynopagebreak}
\setstretch{.7}
{\PaliGlossA{āyasmā ca anuruddho, āyasmā ca bhaddiyo, āyasmā ca kimilo, āyasmā ca bhagu, āyasmā ca koṇḍañño, āyasmā ca revato, āyasmā ca ānando, aññe ca abhiññātā abhiññātā kulaputtā.}}\\
\begin{addmargin}[1em]{2em}
\setstretch{.5}
{\PaliGlossB{The venerables Anuruddha, Bhaddiya, Kimbila, Bhagu, Koṇḍañña, Revata, Ānanda, and other very well-known gentlemen.}}\\
\end{addmargin}
\end{absolutelynopagebreak}

\vskip 0.05in
\begin{absolutelynopagebreak}
\setstretch{.7}
{\PaliGlossA{3. Tena kho pana samayena bhagavā bhikkhusaṃghaparivuto abbhokāse nisinno hoti.}}\\
\begin{addmargin}[1em]{2em}
\setstretch{.5}
{\PaliGlossB{Now at that time the Buddha was sitting in the open, surrounded by the mendicant Saṅgha.}}\\
\end{addmargin}
\end{absolutelynopagebreak}

\begin{absolutelynopagebreak}
\setstretch{.7}
{\PaliGlossA{Atha kho bhagavā te kulaputte ārabbha bhikkhū āmantesi:}}\\
\begin{addmargin}[1em]{2em}
\setstretch{.5}
{\PaliGlossB{Then the Buddha spoke to the mendicants about those gentlemen:}}\\
\end{addmargin}
\end{absolutelynopagebreak}

\begin{absolutelynopagebreak}
\setstretch{.7}
{\PaliGlossA{“ye te, bhikkhave, kulaputtā mamaṃ uddissa saddhā agārasmā anagāriyaṃ pabbajitā, kacci te, bhikkhave, bhikkhū abhiratā brahmacariye”ti?}}\\
\begin{addmargin}[1em]{2em}
\setstretch{.5}
{\PaliGlossB{“Mendicants, those gentlemen who have gone forth from the lay life to homelessness out of faith in me—I trust they’re satisfied with the spiritual life?”}}\\
\end{addmargin}
\end{absolutelynopagebreak}

\begin{absolutelynopagebreak}
\setstretch{.7}
{\PaliGlossA{Evaṃ vutte, te bhikkhū tuṇhī ahesuṃ.}}\\
\begin{addmargin}[1em]{2em}
\setstretch{.5}
{\PaliGlossB{When this was said, the mendicants kept silent.}}\\
\end{addmargin}
\end{absolutelynopagebreak}

\begin{absolutelynopagebreak}
\setstretch{.7}
{\PaliGlossA{Dutiyampi kho bhagavā te kulaputte ārabbha bhikkhū āmantesi:}}\\
\begin{addmargin}[1em]{2em}
\setstretch{.5}
{\PaliGlossB{For a second and a third time the Buddha asked the same question.}}\\
\end{addmargin}
\end{absolutelynopagebreak}

\begin{absolutelynopagebreak}
\setstretch{.7}
{\PaliGlossA{“ye te, bhikkhave, kulaputtā mamaṃ uddissa saddhā agārasmā anagāriyaṃ pabbajitā, kacci te, bhikkhave, bhikkhū abhiratā brahmacariye”ti?}}\\
\begin{addmargin}[1em]{2em}
\setstretch{.5}
{\PaliGlossB{    -}}\\
\end{addmargin}
\end{absolutelynopagebreak}

\begin{absolutelynopagebreak}
\setstretch{.7}
{\PaliGlossA{Dutiyampi kho te bhikkhū tuṇhī ahesuṃ.}}\\
\begin{addmargin}[1em]{2em}
\setstretch{.5}
{\PaliGlossB{    -}}\\
\end{addmargin}
\end{absolutelynopagebreak}

\begin{absolutelynopagebreak}
\setstretch{.7}
{\PaliGlossA{Tatiyampi kho bhagavā te kulaputte ārabbha bhikkhū āmantesi:}}\\
\begin{addmargin}[1em]{2em}
\setstretch{.5}
{\PaliGlossB{    -}}\\
\end{addmargin}
\end{absolutelynopagebreak}

\begin{absolutelynopagebreak}
\setstretch{.7}
{\PaliGlossA{“ye te, bhikkhave, kulaputtā mamaṃ uddissa saddhā agārasmā anagāriyaṃ pabbajitā, kacci te, bhikkhave, bhikkhū abhiratā brahmacariye”ti?}}\\
\begin{addmargin}[1em]{2em}
\setstretch{.5}
{\PaliGlossB{    -}}\\
\end{addmargin}
\end{absolutelynopagebreak}

\begin{absolutelynopagebreak}
\setstretch{.7}
{\PaliGlossA{Tatiyampi kho te bhikkhū tuṇhī ahesuṃ.}}\\
\begin{addmargin}[1em]{2em}
\setstretch{.5}
{\PaliGlossB{For a third time, the mendicants kept silent.}}\\
\end{addmargin}
\end{absolutelynopagebreak}

\vskip 0.05in
\begin{absolutelynopagebreak}
\setstretch{.7}
{\PaliGlossA{4. Atha kho bhagavato etadahosi:}}\\
\begin{addmargin}[1em]{2em}
\setstretch{.5}
{\PaliGlossB{Then it occurred to the Buddha,}}\\
\end{addmargin}
\end{absolutelynopagebreak}

\begin{absolutelynopagebreak}
\setstretch{.7}
{\PaliGlossA{“yannūnāhaṃ te kulaputte puccheyyan”ti.}}\\
\begin{addmargin}[1em]{2em}
\setstretch{.5}
{\PaliGlossB{“Why don’t I question just those gentlemen?”}}\\
\end{addmargin}
\end{absolutelynopagebreak}

\begin{absolutelynopagebreak}
\setstretch{.7}
{\PaliGlossA{Atha kho bhagavā āyasmantaṃ anuruddhaṃ āmantesi:}}\\
\begin{addmargin}[1em]{2em}
\setstretch{.5}
{\PaliGlossB{Then the Buddha said to Venerable Anuruddha,}}\\
\end{addmargin}
\end{absolutelynopagebreak}

\begin{absolutelynopagebreak}
\setstretch{.7}
{\PaliGlossA{“kacci tumhe, anuruddhā, abhiratā brahmacariye”ti?}}\\
\begin{addmargin}[1em]{2em}
\setstretch{.5}
{\PaliGlossB{“Anuruddha and friends, I hope you’re satisfied with the spiritual life?”}}\\
\end{addmargin}
\end{absolutelynopagebreak}

\begin{absolutelynopagebreak}
\setstretch{.7}
{\PaliGlossA{“Taggha mayaṃ, bhante, abhiratā brahmacariye”ti.}}\\
\begin{addmargin}[1em]{2em}
\setstretch{.5}
{\PaliGlossB{“Indeed, sir, we are satisfied with the spiritual life.”}}\\
\end{addmargin}
\end{absolutelynopagebreak}

\vskip 0.05in
\begin{absolutelynopagebreak}
\setstretch{.7}
{\PaliGlossA{5. “Sādhu sādhu, anuruddhā.}}\\
\begin{addmargin}[1em]{2em}
\setstretch{.5}
{\PaliGlossB{“Good, good, Anuruddha and friends!}}\\
\end{addmargin}
\end{absolutelynopagebreak}

\begin{absolutelynopagebreak}
\setstretch{.7}
{\PaliGlossA{Etaṃ kho, anuruddhā, tumhākaṃ patirūpaṃ kulaputtānaṃ saddhā agārasmā anagāriyaṃ pabbajitānaṃ yaṃ tumhe abhirameyyātha brahmacariye.}}\\
\begin{addmargin}[1em]{2em}
\setstretch{.5}
{\PaliGlossB{It’s appropriate for gentlemen like yourselves, who have gone forth in faith from the lay life to homelessness, to be satisfied with the spiritual life.}}\\
\end{addmargin}
\end{absolutelynopagebreak}

\begin{absolutelynopagebreak}
\setstretch{.7}
{\PaliGlossA{Yena tumhe, anuruddhā, bhadrena yobbanena samannāgatā paṭhamena vayasā susukāḷakesā kāme paribhuñjeyyātha tena tumhe, anuruddhā, bhadrenapi yobbanena samannāgatā paṭhamena vayasā susukāḷakesā agārasmā anagāriyaṃ pabbajitā.}}\\
\begin{addmargin}[1em]{2em}
\setstretch{.5}
{\PaliGlossB{Since you’re blessed with youth, in the prime of life, black-haired, you could have enjoyed sensual pleasures; yet you have gone forth from the lay life to homelessness.}}\\
\end{addmargin}
\end{absolutelynopagebreak}

\begin{absolutelynopagebreak}
\setstretch{.7}
{\PaliGlossA{Te ca kho pana tumhe, anuruddhā, neva rājābhinītā agārasmā anagāriyaṃ pabbajitā, na corābhinītā agārasmā anagāriyaṃ pabbajitā, na iṇaṭṭā agārasmā anagāriyaṃ pabbajitā, na bhayaṭṭā agārasmā anagāriyaṃ pabbajitā, nājīvikāpakatā agārasmā anagāriyaṃ pabbajitā.}}\\
\begin{addmargin}[1em]{2em}
\setstretch{.5}
{\PaliGlossB{But you didn’t go forth because you were forced to by kings or bandits, or because you’re in debt or threatened, or to earn a living.}}\\
\end{addmargin}
\end{absolutelynopagebreak}

\begin{absolutelynopagebreak}
\setstretch{.7}
{\PaliGlossA{Api ca khomhi otiṇṇo jātiyā jarāya maraṇena sokehi paridevehi dukkhehi domanassehi upāyāsehi, dukkhotiṇṇo dukkhapareto;}}\\
\begin{addmargin}[1em]{2em}
\setstretch{.5}
{\PaliGlossB{Rather, didn’t you go forth thinking: ‘I’m swamped by rebirth, old age, and death; by sorrow, lamentation, pain, sadness, and distress. I’m swamped by suffering, mired in suffering.}}\\
\end{addmargin}
\end{absolutelynopagebreak}

\begin{absolutelynopagebreak}
\setstretch{.7}
{\PaliGlossA{appeva nāma imassa kevalassa dukkhakkhandhassa antakiriyā paññāyethāti—}}\\
\begin{addmargin}[1em]{2em}
\setstretch{.5}
{\PaliGlossB{Hopefully I can find an end to this entire mass of suffering’?”}}\\
\end{addmargin}
\end{absolutelynopagebreak}

\begin{absolutelynopagebreak}
\setstretch{.7}
{\PaliGlossA{nanu tumhe, anuruddhā, evaṃ saddhā agārasmā anagāriyaṃ pabbajitā”ti?}}\\
\begin{addmargin}[1em]{2em}
\setstretch{.5}
{\PaliGlossB{    -}}\\
\end{addmargin}
\end{absolutelynopagebreak}

\begin{absolutelynopagebreak}
\setstretch{.7}
{\PaliGlossA{“Evaṃ, bhante”.}}\\
\begin{addmargin}[1em]{2em}
\setstretch{.5}
{\PaliGlossB{“Yes, sir.”}}\\
\end{addmargin}
\end{absolutelynopagebreak}

\vskip 0.05in
\begin{absolutelynopagebreak}
\setstretch{.7}
{\PaliGlossA{6. “Evaṃ pabbajitena ca pana, anuruddhā, kulaputtena kimassa karaṇīyaṃ?}}\\
\begin{addmargin}[1em]{2em}
\setstretch{.5}
{\PaliGlossB{“But, Anuruddha and friends, when a gentleman has gone forth like this, what should they do?}}\\
\end{addmargin}
\end{absolutelynopagebreak}

\begin{absolutelynopagebreak}
\setstretch{.7}
{\PaliGlossA{Vivekaṃ, anuruddhā, kāmehi vivekaṃ akusalehi dhammehi pītisukhaṃ nādhigacchati aññaṃ vā tato santataraṃ, tassa abhijjhāpi cittaṃ pariyādāya tiṭṭhati, byāpādopi cittaṃ pariyādāya tiṭṭhati, thinamiddhampi cittaṃ pariyādāya tiṭṭhati uddhaccakukkuccampi cittaṃ pariyādāya tiṭṭhati, vicikicchāpi cittaṃ pariyādāya tiṭṭhati, aratīpi cittaṃ pariyādāya tiṭṭhati, tandīpi cittaṃ pariyādāya tiṭṭhati.}}\\
\begin{addmargin}[1em]{2em}
\setstretch{.5}
{\PaliGlossB{Take someone who doesn’t achieve the rapture and bliss that are secluded from sensual pleasures and unskillful qualities, or something even more peaceful than that. Their mind is still occupied by desire, ill will, dullness and drowsiness, restlessness and remorse, doubt, discontent, and sloth.}}\\
\end{addmargin}
\end{absolutelynopagebreak}

\begin{absolutelynopagebreak}
\setstretch{.7}
{\PaliGlossA{Vivekaṃ, anuruddhā, kāmehi vivekaṃ akusalehi dhammehi pītisukhaṃ nādhigacchati aññaṃ vā tato santataraṃ.}}\\
\begin{addmargin}[1em]{2em}
\setstretch{.5}
{\PaliGlossB{That’s someone who doesn’t achieve the rapture and bliss that are secluded from sensual pleasures and unskillful qualities, or something even more peaceful than that.}}\\
\end{addmargin}
\end{absolutelynopagebreak}

\begin{absolutelynopagebreak}
\setstretch{.7}
{\PaliGlossA{Vivekaṃ, anuruddhā, kāmehi vivekaṃ akusalehi dhammehi pītisukhaṃ adhigacchati aññaṃ vā tato santataraṃ, tassa abhijjhāpi cittaṃ na pariyādāya tiṭṭhati, byāpādopi cittaṃ na pariyādāya tiṭṭhati, thinamiddhampi cittaṃ na pariyādāya tiṭṭhati, uddhaccakukkuccampi cittaṃ na pariyādāya tiṭṭhati, vicikicchāpi cittaṃ na pariyādāya tiṭṭhati, aratīpi cittaṃ na pariyādāya tiṭṭhati, tandīpi cittaṃ na pariyādāya tiṭṭhati.}}\\
\begin{addmargin}[1em]{2em}
\setstretch{.5}
{\PaliGlossB{Take someone who does achieve the rapture and bliss that are secluded from sensual pleasures and unskillful qualities, or something even more peaceful than that. Their mind is not occupied by desire, ill will, dullness and drowsiness, restlessness and remorse, doubt, discontent, and sloth.}}\\
\end{addmargin}
\end{absolutelynopagebreak}

\begin{absolutelynopagebreak}
\setstretch{.7}
{\PaliGlossA{Vivekaṃ, anuruddhā, kāmehi vivekaṃ akusalehi dhammehi pītisukhaṃ adhigacchati aññaṃ vā tato santataraṃ.}}\\
\begin{addmargin}[1em]{2em}
\setstretch{.5}
{\PaliGlossB{That’s someone who does achieve the rapture and bliss that are secluded from sensual pleasures and unskillful qualities, or something even more peaceful than that.}}\\
\end{addmargin}
\end{absolutelynopagebreak}

\vskip 0.05in
\begin{absolutelynopagebreak}
\setstretch{.7}
{\PaliGlossA{7. Kinti vo, anuruddhā, mayi hoti:}}\\
\begin{addmargin}[1em]{2em}
\setstretch{.5}
{\PaliGlossB{Is this what you think of me?}}\\
\end{addmargin}
\end{absolutelynopagebreak}

\begin{absolutelynopagebreak}
\setstretch{.7}
{\PaliGlossA{‘ye āsavā saṃkilesikā ponobbhavikā sadarā dukkhavipākā āyatiṃ jātijarāmaraṇiyā, appahīnā te tathāgatassa;}}\\
\begin{addmargin}[1em]{2em}
\setstretch{.5}
{\PaliGlossB{‘The Realized One has not given up the defilements, the corruptions that lead to future lives and are hurtful, resulting in suffering and future rebirth, old age, and death.}}\\
\end{addmargin}
\end{absolutelynopagebreak}

\begin{absolutelynopagebreak}
\setstretch{.7}
{\PaliGlossA{tasmā tathāgato saṅkhāyekaṃ paṭisevati, saṅkhāyekaṃ adhivāseti, saṅkhāyekaṃ parivajjeti, saṅkhāyekaṃ vinodetī’”ti?}}\\
\begin{addmargin}[1em]{2em}
\setstretch{.5}
{\PaliGlossB{That’s why, after reflection, he uses some things, endures some things, avoids some things, and gets rid of some things.’”}}\\
\end{addmargin}
\end{absolutelynopagebreak}

\begin{absolutelynopagebreak}
\setstretch{.7}
{\PaliGlossA{“Na kho no, bhante, bhagavati evaṃ hoti:}}\\
\begin{addmargin}[1em]{2em}
\setstretch{.5}
{\PaliGlossB{“No sir, we don’t think of you that way.}}\\
\end{addmargin}
\end{absolutelynopagebreak}

\begin{absolutelynopagebreak}
\setstretch{.7}
{\PaliGlossA{‘ye āsavā saṅkilesikā ponobbhavikā sadarā dukkhavipākā āyatiṃ jātijarāmaraṇiyā, appahīnā te tathāgatassa;}}\\
\begin{addmargin}[1em]{2em}
\setstretch{.5}
{\PaliGlossB{    -}}\\
\end{addmargin}
\end{absolutelynopagebreak}

\begin{absolutelynopagebreak}
\setstretch{.7}
{\PaliGlossA{tasmā tathāgato saṅkhāyekaṃ paṭisevati, saṅkhāyekaṃ adhivāseti, saṅkhāyekaṃ parivajjeti, saṅkhāyekaṃ vinodetī’ti.}}\\
\begin{addmargin}[1em]{2em}
\setstretch{.5}
{\PaliGlossB{    -}}\\
\end{addmargin}
\end{absolutelynopagebreak}

\begin{absolutelynopagebreak}
\setstretch{.7}
{\PaliGlossA{Evaṃ kho no, bhante, bhagavati hoti:}}\\
\begin{addmargin}[1em]{2em}
\setstretch{.5}
{\PaliGlossB{We think of you this way:}}\\
\end{addmargin}
\end{absolutelynopagebreak}

\begin{absolutelynopagebreak}
\setstretch{.7}
{\PaliGlossA{‘ye āsavā saṅkilesikā ponobbhavikā sadarā dukkhavipākā āyatiṃ jātijarāmaraṇiyā, pahīnā te tathāgatassa;}}\\
\begin{addmargin}[1em]{2em}
\setstretch{.5}
{\PaliGlossB{‘The Realized One has given up the defilements, the corruptions that lead to future lives and are hurtful, resulting in suffering and future rebirth, old age, and death.}}\\
\end{addmargin}
\end{absolutelynopagebreak}

\begin{absolutelynopagebreak}
\setstretch{.7}
{\PaliGlossA{tasmā tathāgato saṅkhāyekaṃ paṭisevati, saṅkhāyekaṃ adhivāseti, saṅkhāyekaṃ parivajjeti, saṅkhāyekaṃ vinodetī’”ti.}}\\
\begin{addmargin}[1em]{2em}
\setstretch{.5}
{\PaliGlossB{That’s why, after reflection, he uses some things, endures some things, avoids some things, and gets rid of some things.’”}}\\
\end{addmargin}
\end{absolutelynopagebreak}

\begin{absolutelynopagebreak}
\setstretch{.7}
{\PaliGlossA{“Sādhu sādhu, anuruddhā.}}\\
\begin{addmargin}[1em]{2em}
\setstretch{.5}
{\PaliGlossB{“Good, good, Anuruddha and friends!}}\\
\end{addmargin}
\end{absolutelynopagebreak}

\begin{absolutelynopagebreak}
\setstretch{.7}
{\PaliGlossA{Tathāgatassa, anuruddhā, ye āsavā saṅkilesikā ponobbhavikā sadarā dukkhavipākā āyatiṃ jātijarāmaraṇiyā, pahīnā te ucchinnamūlā tālāvatthukatā anabhāvaṅkatā āyatiṃ anuppādadhammā.}}\\
\begin{addmargin}[1em]{2em}
\setstretch{.5}
{\PaliGlossB{The Realized One has given up the defilements, the corruptions that lead to future lives and are hurtful, resulting in suffering and future rebirth, old age, and death. He has cut them off at the root, made them like a palm stump, obliterated them so they are unable to arise in the future.}}\\
\end{addmargin}
\end{absolutelynopagebreak}

\begin{absolutelynopagebreak}
\setstretch{.7}
{\PaliGlossA{Seyyathāpi, anuruddhā, tālo matthakacchinno abhabbo punavirūḷhiyā;}}\\
\begin{addmargin}[1em]{2em}
\setstretch{.5}
{\PaliGlossB{Just as a palm tree with its crown cut off is incapable of further growth,}}\\
\end{addmargin}
\end{absolutelynopagebreak}

\begin{absolutelynopagebreak}
\setstretch{.7}
{\PaliGlossA{evameva kho, anuruddhā, tathāgatassa ye āsavā saṅkilesikā ponobbhavikā sadarā dukkhavipākā āyatiṃ jātijarāmaraṇiyā, pahīnā te ucchinnamūlā tālāvatthukatā anabhāvaṅkatā āyatiṃ anuppādadhammā;}}\\
\begin{addmargin}[1em]{2em}
\setstretch{.5}
{\PaliGlossB{in the same way, the Realized One has given up the defilements so they are unable to arise in the future.}}\\
\end{addmargin}
\end{absolutelynopagebreak}

\begin{absolutelynopagebreak}
\setstretch{.7}
{\PaliGlossA{tasmā tathāgato saṅkhāyekaṃ paṭisevati, saṅkhāyekaṃ adhivāseti, saṅkhāyekaṃ parivajjeti, saṅkhāyekaṃ vinodeti.}}\\
\begin{addmargin}[1em]{2em}
\setstretch{.5}
{\PaliGlossB{That’s why, after reflection, he uses some things, endures some things, avoids some things, and gets rid of some things.}}\\
\end{addmargin}
\end{absolutelynopagebreak}

\vskip 0.05in
\begin{absolutelynopagebreak}
\setstretch{.7}
{\PaliGlossA{8. Taṃ kiṃ maññasi, anuruddhā,}}\\
\begin{addmargin}[1em]{2em}
\setstretch{.5}
{\PaliGlossB{What do you think, Anuruddha and friends?}}\\
\end{addmargin}
\end{absolutelynopagebreak}

\begin{absolutelynopagebreak}
\setstretch{.7}
{\PaliGlossA{kaṃ atthavasaṃ sampassamāno tathāgato sāvake abbhatīte kālaṅkate upapattīsu byākaroti:}}\\
\begin{addmargin}[1em]{2em}
\setstretch{.5}
{\PaliGlossB{What advantage does the Realized One see in declaring the rebirth of his disciples who have passed away:}}\\
\end{addmargin}
\end{absolutelynopagebreak}

\begin{absolutelynopagebreak}
\setstretch{.7}
{\PaliGlossA{‘asu amutra upapanno; asu amutra upapanno’”ti?}}\\
\begin{addmargin}[1em]{2em}
\setstretch{.5}
{\PaliGlossB{‘This one is reborn here, while that one is reborn there’?”}}\\
\end{addmargin}
\end{absolutelynopagebreak}

\begin{absolutelynopagebreak}
\setstretch{.7}
{\PaliGlossA{“Bhagavaṃmūlakā no, bhante, dhammā bhagavaṃnettikā bhagavaṃpaṭisaraṇā. Sādhu vata, bhante, bhagavantaṃyeva paṭibhātu etassa bhāsitassa attho. Bhagavato sutvā bhikkhū dhāressantī”ti.}}\\
\begin{addmargin}[1em]{2em}
\setstretch{.5}
{\PaliGlossB{“Our teachings are rooted in the Buddha. He is our guide and our refuge. Sir, may the Buddha himself please clarify the meaning of this. The mendicants will listen and remember it.”}}\\
\end{addmargin}
\end{absolutelynopagebreak}

\vskip 0.05in
\begin{absolutelynopagebreak}
\setstretch{.7}
{\PaliGlossA{9. “Na kho, anuruddhā, tathāgato janakuhanatthaṃ na janalapanatthaṃ na lābhasakkārasilokānisaṃsatthaṃ na ‘iti maṃ jano jānātū’ti sāvake abbhatīte kālaṅkate upapattīsu byākaroti:}}\\
\begin{addmargin}[1em]{2em}
\setstretch{.5}
{\PaliGlossB{“The Realized One does not declare such things for the sake of deceiving people or flattering them, nor for the benefit of possessions, honor, or popularity, nor thinking, ‘So let people know about me!’}}\\
\end{addmargin}
\end{absolutelynopagebreak}

\begin{absolutelynopagebreak}
\setstretch{.7}
{\PaliGlossA{‘asu amutra upapanno, asu amutra upapanno’ti.}}\\
\begin{addmargin}[1em]{2em}
\setstretch{.5}
{\PaliGlossB{    -}}\\
\end{addmargin}
\end{absolutelynopagebreak}

\begin{absolutelynopagebreak}
\setstretch{.7}
{\PaliGlossA{Santi ca kho, anuruddhā, kulaputtā saddhā uḷāravedā uḷārapāmojjā.}}\\
\begin{addmargin}[1em]{2em}
\setstretch{.5}
{\PaliGlossB{Rather, there are gentlemen of faith who are full of sublime joy and gladness.}}\\
\end{addmargin}
\end{absolutelynopagebreak}

\begin{absolutelynopagebreak}
\setstretch{.7}
{\PaliGlossA{Te taṃ sutvā tadatthāya cittaṃ upasaṃharanti.}}\\
\begin{addmargin}[1em]{2em}
\setstretch{.5}
{\PaliGlossB{When they hear that, they apply their minds to that end.}}\\
\end{addmargin}
\end{absolutelynopagebreak}

\begin{absolutelynopagebreak}
\setstretch{.7}
{\PaliGlossA{Tesaṃ taṃ, anuruddhā, hoti dīgharattaṃ hitāya sukhāya.}}\\
\begin{addmargin}[1em]{2em}
\setstretch{.5}
{\PaliGlossB{That is for their lasting welfare and happiness.}}\\
\end{addmargin}
\end{absolutelynopagebreak}

\vskip 0.05in
\begin{absolutelynopagebreak}
\setstretch{.7}
{\PaliGlossA{10. Idhānuruddhā, bhikkhu suṇāti:}}\\
\begin{addmargin}[1em]{2em}
\setstretch{.5}
{\PaliGlossB{Take a monk who hears this:}}\\
\end{addmargin}
\end{absolutelynopagebreak}

\begin{absolutelynopagebreak}
\setstretch{.7}
{\PaliGlossA{‘itthannāmo bhikkhu kālaṅkato;}}\\
\begin{addmargin}[1em]{2em}
\setstretch{.5}
{\PaliGlossB{‘The monk named so-and-so has passed away.}}\\
\end{addmargin}
\end{absolutelynopagebreak}

\begin{absolutelynopagebreak}
\setstretch{.7}
{\PaliGlossA{so bhagavatā byākato—}}\\
\begin{addmargin}[1em]{2em}
\setstretch{.5}
{\PaliGlossB{The Buddha has declared that,}}\\
\end{addmargin}
\end{absolutelynopagebreak}

\begin{absolutelynopagebreak}
\setstretch{.7}
{\PaliGlossA{aññāya saṇṭhahī’ti.}}\\
\begin{addmargin}[1em]{2em}
\setstretch{.5}
{\PaliGlossB{he was enlightened.’}}\\
\end{addmargin}
\end{absolutelynopagebreak}

\begin{absolutelynopagebreak}
\setstretch{.7}
{\PaliGlossA{So kho panassa āyasmā sāmaṃ diṭṭho vā hoti anussavassuto vā:}}\\
\begin{addmargin}[1em]{2em}
\setstretch{.5}
{\PaliGlossB{And he’s either seen for himself, or heard from someone else, that that venerable}}\\
\end{addmargin}
\end{absolutelynopagebreak}

\begin{absolutelynopagebreak}
\setstretch{.7}
{\PaliGlossA{‘evaṃsīlo so āyasmā ahosi itipi, evaṃdhammo so āyasmā ahosi itipi, evaṃpañño so āyasmā ahosi itipi, evaṃvihārī so āyasmā ahosi itipi, evaṃvimutto so āyasmā ahosi itipī’ti.}}\\
\begin{addmargin}[1em]{2em}
\setstretch{.5}
{\PaliGlossB{had such ethics, such qualities, such wisdom, such meditation, or such freedom.}}\\
\end{addmargin}
\end{absolutelynopagebreak}

\begin{absolutelynopagebreak}
\setstretch{.7}
{\PaliGlossA{So tassa saddhañca sīlañca sutañca cāgañca paññañca anussaranto tadatthāya cittaṃ upasaṃharati.}}\\
\begin{addmargin}[1em]{2em}
\setstretch{.5}
{\PaliGlossB{Recollecting that monk’s faith, ethics, learning, generosity, and wisdom, he applies his mind to that end.}}\\
\end{addmargin}
\end{absolutelynopagebreak}

\begin{absolutelynopagebreak}
\setstretch{.7}
{\PaliGlossA{Evampi kho, anuruddhā, bhikkhuno phāsuvihāro hoti.}}\\
\begin{addmargin}[1em]{2em}
\setstretch{.5}
{\PaliGlossB{That’s how a monk lives at ease.}}\\
\end{addmargin}
\end{absolutelynopagebreak}

\vskip 0.05in
\begin{absolutelynopagebreak}
\setstretch{.7}
{\PaliGlossA{11. Idhānuruddhā, bhikkhu suṇāti:}}\\
\begin{addmargin}[1em]{2em}
\setstretch{.5}
{\PaliGlossB{Take a monk who hears this:}}\\
\end{addmargin}
\end{absolutelynopagebreak}

\begin{absolutelynopagebreak}
\setstretch{.7}
{\PaliGlossA{‘itthannāmo bhikkhu kālaṅkato;}}\\
\begin{addmargin}[1em]{2em}
\setstretch{.5}
{\PaliGlossB{‘The monk named so-and-so has passed away.}}\\
\end{addmargin}
\end{absolutelynopagebreak}

\begin{absolutelynopagebreak}
\setstretch{.7}
{\PaliGlossA{so bhagavatā byākato—}}\\
\begin{addmargin}[1em]{2em}
\setstretch{.5}
{\PaliGlossB{The Buddha has declared that,}}\\
\end{addmargin}
\end{absolutelynopagebreak}

\begin{absolutelynopagebreak}
\setstretch{.7}
{\PaliGlossA{pañcannaṃ orambhāgiyānaṃ saṃyojanānaṃ parikkhayā opapātiko tattha parinibbāyī anāvattidhammo tasmā lokā’ti.}}\\
\begin{addmargin}[1em]{2em}
\setstretch{.5}
{\PaliGlossB{with the ending of the five lower fetters, he’s been reborn spontaneously and will become extinguished there, not liable to return from that world.’}}\\
\end{addmargin}
\end{absolutelynopagebreak}

\begin{absolutelynopagebreak}
\setstretch{.7}
{\PaliGlossA{So kho panassa āyasmā sāmaṃ diṭṭho vā hoti anussavassuto vā:}}\\
\begin{addmargin}[1em]{2em}
\setstretch{.5}
{\PaliGlossB{And he’s either seen for himself, or heard from someone else, that that venerable}}\\
\end{addmargin}
\end{absolutelynopagebreak}

\begin{absolutelynopagebreak}
\setstretch{.7}
{\PaliGlossA{‘evaṃsīlo so āyasmā ahosi itipi, evaṃdhammo … pe … evaṃpañño … evaṃvihārī … evaṃvimutto so āyasmā ahosi itipī’ti.}}\\
\begin{addmargin}[1em]{2em}
\setstretch{.5}
{\PaliGlossB{had such ethics, such qualities, such wisdom, such meditation, or such freedom.}}\\
\end{addmargin}
\end{absolutelynopagebreak}

\begin{absolutelynopagebreak}
\setstretch{.7}
{\PaliGlossA{So tassa saddhañca sīlañca sutañca cāgañca paññañca anussaranto tadatthāya cittaṃ upasaṃharati.}}\\
\begin{addmargin}[1em]{2em}
\setstretch{.5}
{\PaliGlossB{Recollecting that monk’s faith, ethics, learning, generosity, and wisdom, he applies his mind to that end.}}\\
\end{addmargin}
\end{absolutelynopagebreak}

\begin{absolutelynopagebreak}
\setstretch{.7}
{\PaliGlossA{Evampi kho, anuruddhā, bhikkhuno phāsuvihāro hoti.}}\\
\begin{addmargin}[1em]{2em}
\setstretch{.5}
{\PaliGlossB{That too is how a monk lives at ease.}}\\
\end{addmargin}
\end{absolutelynopagebreak}

\vskip 0.05in
\begin{absolutelynopagebreak}
\setstretch{.7}
{\PaliGlossA{12. Idhānuruddhā, bhikkhu suṇāti:}}\\
\begin{addmargin}[1em]{2em}
\setstretch{.5}
{\PaliGlossB{Take a monk who hears this:}}\\
\end{addmargin}
\end{absolutelynopagebreak}

\begin{absolutelynopagebreak}
\setstretch{.7}
{\PaliGlossA{‘itthannāmo bhikkhu kālaṅkato;}}\\
\begin{addmargin}[1em]{2em}
\setstretch{.5}
{\PaliGlossB{‘The monk named so-and-so has passed away.}}\\
\end{addmargin}
\end{absolutelynopagebreak}

\begin{absolutelynopagebreak}
\setstretch{.7}
{\PaliGlossA{so bhagavatā byākato—}}\\
\begin{addmargin}[1em]{2em}
\setstretch{.5}
{\PaliGlossB{The Buddha has declared that,}}\\
\end{addmargin}
\end{absolutelynopagebreak}

\begin{absolutelynopagebreak}
\setstretch{.7}
{\PaliGlossA{tiṇṇaṃ saṃyojanānaṃ parikkhayā rāgadosamohānaṃ tanuttā sakadāgāmī sakideva imaṃ lokaṃ āgantvā dukkhassantaṃ karissatī’ti.}}\\
\begin{addmargin}[1em]{2em}
\setstretch{.5}
{\PaliGlossB{with the ending of three fetters, and the weakening of greed, hate, and delusion, he’s a once-returner. He’ll come back to this world once only, then make an end of suffering.’}}\\
\end{addmargin}
\end{absolutelynopagebreak}

\begin{absolutelynopagebreak}
\setstretch{.7}
{\PaliGlossA{So kho panassa āyasmā sāmaṃ diṭṭho vā hoti anussavassuto vā:}}\\
\begin{addmargin}[1em]{2em}
\setstretch{.5}
{\PaliGlossB{And he’s either seen for himself, or heard from someone else, that that venerable}}\\
\end{addmargin}
\end{absolutelynopagebreak}

\begin{absolutelynopagebreak}
\setstretch{.7}
{\PaliGlossA{‘evaṃsīlo so āyasmā ahosi itipi, evaṃdhammo … pe … evaṃpañño … evaṃvihārī … evaṃvimutto so āyasmā ahosi itipī’ti.}}\\
\begin{addmargin}[1em]{2em}
\setstretch{.5}
{\PaliGlossB{had such ethics, such qualities, such wisdom, such meditation, or such freedom.}}\\
\end{addmargin}
\end{absolutelynopagebreak}

\begin{absolutelynopagebreak}
\setstretch{.7}
{\PaliGlossA{So tassa saddhañca sīlañca sutañca cāgañca paññañca anussaranto tadatthāya cittaṃ upasaṃharati.}}\\
\begin{addmargin}[1em]{2em}
\setstretch{.5}
{\PaliGlossB{Recollecting that monk’s faith, ethics, learning, generosity, and wisdom, he applies his mind to that end.}}\\
\end{addmargin}
\end{absolutelynopagebreak}

\begin{absolutelynopagebreak}
\setstretch{.7}
{\PaliGlossA{Evampi kho, anuruddhā, bhikkhuno phāsuvihāro hoti.}}\\
\begin{addmargin}[1em]{2em}
\setstretch{.5}
{\PaliGlossB{That too is how a monk lives at ease.}}\\
\end{addmargin}
\end{absolutelynopagebreak}

\vskip 0.05in
\begin{absolutelynopagebreak}
\setstretch{.7}
{\PaliGlossA{13. Idhānuruddhā, bhikkhu suṇāti:}}\\
\begin{addmargin}[1em]{2em}
\setstretch{.5}
{\PaliGlossB{Take a monk who hears this:}}\\
\end{addmargin}
\end{absolutelynopagebreak}

\begin{absolutelynopagebreak}
\setstretch{.7}
{\PaliGlossA{‘itthannāmo bhikkhu kālaṅkato;}}\\
\begin{addmargin}[1em]{2em}
\setstretch{.5}
{\PaliGlossB{‘The monk named so-and-so has passed away.}}\\
\end{addmargin}
\end{absolutelynopagebreak}

\begin{absolutelynopagebreak}
\setstretch{.7}
{\PaliGlossA{so bhagavatā byākato—}}\\
\begin{addmargin}[1em]{2em}
\setstretch{.5}
{\PaliGlossB{The Buddha has declared that,}}\\
\end{addmargin}
\end{absolutelynopagebreak}

\begin{absolutelynopagebreak}
\setstretch{.7}
{\PaliGlossA{tiṇṇaṃ saṃyojanānaṃ parikkhayā sotāpanno avinipātadhammo niyato sambodhiparāyaṇo’ti.}}\\
\begin{addmargin}[1em]{2em}
\setstretch{.5}
{\PaliGlossB{with the ending of three fetters he’s a stream-enterer, not liable to be reborn in the underworld, bound for awakening.’}}\\
\end{addmargin}
\end{absolutelynopagebreak}

\begin{absolutelynopagebreak}
\setstretch{.7}
{\PaliGlossA{So kho panassa āyasmā sāmaṃ diṭṭho vā hoti anussavassuto vā:}}\\
\begin{addmargin}[1em]{2em}
\setstretch{.5}
{\PaliGlossB{And he’s either seen for himself, or heard from someone else, that that venerable}}\\
\end{addmargin}
\end{absolutelynopagebreak}

\begin{absolutelynopagebreak}
\setstretch{.7}
{\PaliGlossA{‘evaṃsīlo so āyasmā ahosi itipi, evaṃdhammo … pe … evaṃpañño … evaṃvihārī … evaṃvimutto so āyasmā ahosi itipī’ti.}}\\
\begin{addmargin}[1em]{2em}
\setstretch{.5}
{\PaliGlossB{had such ethics, such qualities, such wisdom, such meditation, or such freedom.}}\\
\end{addmargin}
\end{absolutelynopagebreak}

\begin{absolutelynopagebreak}
\setstretch{.7}
{\PaliGlossA{So tassa saddhañca sīlañca sutañca cāgañca paññañca anussaranto tadatthāya cittaṃ upasaṃharati.}}\\
\begin{addmargin}[1em]{2em}
\setstretch{.5}
{\PaliGlossB{Recollecting that monk’s faith, ethics, learning, generosity, and wisdom, he applies his mind to that end.}}\\
\end{addmargin}
\end{absolutelynopagebreak}

\begin{absolutelynopagebreak}
\setstretch{.7}
{\PaliGlossA{Evampi kho, anuruddhā, bhikkhuno phāsuvihāro hoti.}}\\
\begin{addmargin}[1em]{2em}
\setstretch{.5}
{\PaliGlossB{That too is how a monk lives at ease.}}\\
\end{addmargin}
\end{absolutelynopagebreak}

\vskip 0.05in
\begin{absolutelynopagebreak}
\setstretch{.7}
{\PaliGlossA{14. Idhānuruddhā, bhikkhunī suṇāti:}}\\
\begin{addmargin}[1em]{2em}
\setstretch{.5}
{\PaliGlossB{Take a nun who hears this:}}\\
\end{addmargin}
\end{absolutelynopagebreak}

\begin{absolutelynopagebreak}
\setstretch{.7}
{\PaliGlossA{‘itthannāmā bhikkhunī kālaṅkatā;}}\\
\begin{addmargin}[1em]{2em}
\setstretch{.5}
{\PaliGlossB{‘The nun named so-and-so has passed away.}}\\
\end{addmargin}
\end{absolutelynopagebreak}

\begin{absolutelynopagebreak}
\setstretch{.7}
{\PaliGlossA{sā bhagavatā byākatā—}}\\
\begin{addmargin}[1em]{2em}
\setstretch{.5}
{\PaliGlossB{The Buddha has declared that,}}\\
\end{addmargin}
\end{absolutelynopagebreak}

\begin{absolutelynopagebreak}
\setstretch{.7}
{\PaliGlossA{aññāya saṇṭhahī’ti.}}\\
\begin{addmargin}[1em]{2em}
\setstretch{.5}
{\PaliGlossB{she was enlightened.’}}\\
\end{addmargin}
\end{absolutelynopagebreak}

\begin{absolutelynopagebreak}
\setstretch{.7}
{\PaliGlossA{Sā kho panassā bhaginī sāmaṃ diṭṭhā vā hoti anussavassutā vā:}}\\
\begin{addmargin}[1em]{2em}
\setstretch{.5}
{\PaliGlossB{And she’s either seen for herself, or heard from someone else, that that sister}}\\
\end{addmargin}
\end{absolutelynopagebreak}

\begin{absolutelynopagebreak}
\setstretch{.7}
{\PaliGlossA{‘evaṃsīlā sā bhaginī ahosi itipi, evaṃdhammā sā bhaginī ahosi itipi, evaṃpaññā sā bhaginī ahosi itipi, evaṃvihārinī sā bhaginī ahosi itipi, evaṃvimuttā sā bhaginī ahosi itipī’ti.}}\\
\begin{addmargin}[1em]{2em}
\setstretch{.5}
{\PaliGlossB{had such ethics, such qualities, such wisdom, such meditation, or such freedom.}}\\
\end{addmargin}
\end{absolutelynopagebreak}

\begin{absolutelynopagebreak}
\setstretch{.7}
{\PaliGlossA{Sā tassā saddhañca sīlañca sutañca cāgañca paññañca anussarantī tadatthāya cittaṃ upasaṃharati.}}\\
\begin{addmargin}[1em]{2em}
\setstretch{.5}
{\PaliGlossB{Recollecting that nun’s faith, ethics, learning, generosity, and wisdom, she applies her mind to that end.}}\\
\end{addmargin}
\end{absolutelynopagebreak}

\begin{absolutelynopagebreak}
\setstretch{.7}
{\PaliGlossA{Evampi kho, anuruddhā, bhikkhuniyā phāsuvihāro hoti.}}\\
\begin{addmargin}[1em]{2em}
\setstretch{.5}
{\PaliGlossB{That’s how a nun lives at ease.}}\\
\end{addmargin}
\end{absolutelynopagebreak}

\vskip 0.05in
\begin{absolutelynopagebreak}
\setstretch{.7}
{\PaliGlossA{15. Idhānuruddhā, bhikkhunī suṇāti:}}\\
\begin{addmargin}[1em]{2em}
\setstretch{.5}
{\PaliGlossB{Take a nun who hears this:}}\\
\end{addmargin}
\end{absolutelynopagebreak}

\begin{absolutelynopagebreak}
\setstretch{.7}
{\PaliGlossA{‘itthannāmā bhikkhunī kālaṅkatā;}}\\
\begin{addmargin}[1em]{2em}
\setstretch{.5}
{\PaliGlossB{‘The nun named so-and-so has passed away.}}\\
\end{addmargin}
\end{absolutelynopagebreak}

\begin{absolutelynopagebreak}
\setstretch{.7}
{\PaliGlossA{sā bhagavatā byākatā—}}\\
\begin{addmargin}[1em]{2em}
\setstretch{.5}
{\PaliGlossB{The Buddha has declared that,}}\\
\end{addmargin}
\end{absolutelynopagebreak}

\begin{absolutelynopagebreak}
\setstretch{.7}
{\PaliGlossA{pañcannaṃ orambhāgiyānaṃ saṃyojanānaṃ parikkhayā opapātikā tattha parinibbāyinī anāvattidhammā tasmā lokā’ti.}}\\
\begin{addmargin}[1em]{2em}
\setstretch{.5}
{\PaliGlossB{with the ending of the five lower fetters, she’s been reborn spontaneously and will become extinguished there, not liable to return from that world.’}}\\
\end{addmargin}
\end{absolutelynopagebreak}

\begin{absolutelynopagebreak}
\setstretch{.7}
{\PaliGlossA{Sā kho panassā bhaginī sāmaṃ diṭṭhā vā hoti anussavassutā vā:}}\\
\begin{addmargin}[1em]{2em}
\setstretch{.5}
{\PaliGlossB{And she’s either seen for herself, or heard from someone else, that that sister}}\\
\end{addmargin}
\end{absolutelynopagebreak}

\begin{absolutelynopagebreak}
\setstretch{.7}
{\PaliGlossA{‘evaṃsīlā sā bhaginī ahosi itipi, evaṃdhammā … pe … evaṃpaññā … evaṃvihārinī … evaṃvimuttā sā bhaginī ahosi itipī’ti.}}\\
\begin{addmargin}[1em]{2em}
\setstretch{.5}
{\PaliGlossB{had such ethics, such qualities, such wisdom, such meditation, or such freedom.}}\\
\end{addmargin}
\end{absolutelynopagebreak}

\begin{absolutelynopagebreak}
\setstretch{.7}
{\PaliGlossA{Sā tassā saddhañca sīlañca sutañca cāgañca paññañca anussarantī tadatthāya cittaṃ upasaṃharati.}}\\
\begin{addmargin}[1em]{2em}
\setstretch{.5}
{\PaliGlossB{Recollecting that nun’s faith, ethics, learning, generosity, and wisdom, she applies her mind to that end.}}\\
\end{addmargin}
\end{absolutelynopagebreak}

\begin{absolutelynopagebreak}
\setstretch{.7}
{\PaliGlossA{Evampi kho, anuruddhā, bhikkhuniyā phāsuvihāro hoti.}}\\
\begin{addmargin}[1em]{2em}
\setstretch{.5}
{\PaliGlossB{That too is how a nun lives at ease.}}\\
\end{addmargin}
\end{absolutelynopagebreak}

\vskip 0.05in
\begin{absolutelynopagebreak}
\setstretch{.7}
{\PaliGlossA{16. Idhānuruddhā, bhikkhunī suṇāti:}}\\
\begin{addmargin}[1em]{2em}
\setstretch{.5}
{\PaliGlossB{Take a nun who hears this:}}\\
\end{addmargin}
\end{absolutelynopagebreak}

\begin{absolutelynopagebreak}
\setstretch{.7}
{\PaliGlossA{‘itthannāmā bhikkhunī kālaṅkatā;}}\\
\begin{addmargin}[1em]{2em}
\setstretch{.5}
{\PaliGlossB{‘The nun named so-and-so has passed away.}}\\
\end{addmargin}
\end{absolutelynopagebreak}

\begin{absolutelynopagebreak}
\setstretch{.7}
{\PaliGlossA{sā bhagavatā byākatā—}}\\
\begin{addmargin}[1em]{2em}
\setstretch{.5}
{\PaliGlossB{The Buddha has declared that,}}\\
\end{addmargin}
\end{absolutelynopagebreak}

\begin{absolutelynopagebreak}
\setstretch{.7}
{\PaliGlossA{tiṇṇaṃ saṃyojanānaṃ parikkhayā rāgadosamohānaṃ tanuttā sakadāgāminī sakideva imaṃ lokaṃ āgantvā dukkhassantaṃ karissatī’ti.}}\\
\begin{addmargin}[1em]{2em}
\setstretch{.5}
{\PaliGlossB{with the ending of three fetters, and the weakening of greed, hate, and delusion, she’s a once-returner. She’ll come back to this world once only, then make an end of suffering.’}}\\
\end{addmargin}
\end{absolutelynopagebreak}

\begin{absolutelynopagebreak}
\setstretch{.7}
{\PaliGlossA{Sā kho panassā bhaginī sāmaṃ diṭṭhā vā hoti anussavassutā vā:}}\\
\begin{addmargin}[1em]{2em}
\setstretch{.5}
{\PaliGlossB{And she’s either seen for herself, or heard from someone else, that that sister}}\\
\end{addmargin}
\end{absolutelynopagebreak}

\begin{absolutelynopagebreak}
\setstretch{.7}
{\PaliGlossA{‘evaṃsīlā sā bhaginī ahosi itipi, evaṃdhammā … pe … evaṃpaññā … evaṃvihārinī … evaṃvimuttā sā bhaginī ahosi itipī’ti.}}\\
\begin{addmargin}[1em]{2em}
\setstretch{.5}
{\PaliGlossB{had such ethics, such qualities, such wisdom, such meditation, or such freedom.}}\\
\end{addmargin}
\end{absolutelynopagebreak}

\begin{absolutelynopagebreak}
\setstretch{.7}
{\PaliGlossA{Sā tassā saddhañca sīlañca sutañca cāgañca paññañca anussarantī tadatthāya cittaṃ upasaṃharati.}}\\
\begin{addmargin}[1em]{2em}
\setstretch{.5}
{\PaliGlossB{Recollecting that nun’s faith, ethics, learning, generosity, and wisdom, she applies her mind to that end.}}\\
\end{addmargin}
\end{absolutelynopagebreak}

\begin{absolutelynopagebreak}
\setstretch{.7}
{\PaliGlossA{Evampi kho, anuruddhā, bhikkhuniyā phāsuvihāro hoti.}}\\
\begin{addmargin}[1em]{2em}
\setstretch{.5}
{\PaliGlossB{That too is how a nun lives at ease.}}\\
\end{addmargin}
\end{absolutelynopagebreak}

\vskip 0.05in
\begin{absolutelynopagebreak}
\setstretch{.7}
{\PaliGlossA{17. Idhānuruddhā, bhikkhunī suṇāti:}}\\
\begin{addmargin}[1em]{2em}
\setstretch{.5}
{\PaliGlossB{Take a nun who hears this:}}\\
\end{addmargin}
\end{absolutelynopagebreak}

\begin{absolutelynopagebreak}
\setstretch{.7}
{\PaliGlossA{‘itthannāmā bhikkhunī kālaṅkatā;}}\\
\begin{addmargin}[1em]{2em}
\setstretch{.5}
{\PaliGlossB{‘The nun named so-and-so has passed away.}}\\
\end{addmargin}
\end{absolutelynopagebreak}

\begin{absolutelynopagebreak}
\setstretch{.7}
{\PaliGlossA{sā bhagavatā byākatā—}}\\
\begin{addmargin}[1em]{2em}
\setstretch{.5}
{\PaliGlossB{The Buddha has declared that,}}\\
\end{addmargin}
\end{absolutelynopagebreak}

\begin{absolutelynopagebreak}
\setstretch{.7}
{\PaliGlossA{tiṇṇaṃ saṃyojanānaṃ parikkhayā sotāpannā avinipātadhammā niyatā sambodhiparāyaṇā’ti.}}\\
\begin{addmargin}[1em]{2em}
\setstretch{.5}
{\PaliGlossB{with the ending of three fetters she’s a stream-enterer, not liable to be reborn in the underworld, bound for awakening.’}}\\
\end{addmargin}
\end{absolutelynopagebreak}

\begin{absolutelynopagebreak}
\setstretch{.7}
{\PaliGlossA{Sā kho panassā bhaginī sāmaṃ diṭṭhā vā hoti anussavassutā vā:}}\\
\begin{addmargin}[1em]{2em}
\setstretch{.5}
{\PaliGlossB{And she’s either seen for herself, or heard from someone else, that that sister}}\\
\end{addmargin}
\end{absolutelynopagebreak}

\begin{absolutelynopagebreak}
\setstretch{.7}
{\PaliGlossA{‘evaṃsīlā sā bhaginī ahosi itipi, evaṃdhammā … evaṃpaññā … evaṃvihārinī … evaṃvimuttā sā bhaginī ahosi itipī’ti.}}\\
\begin{addmargin}[1em]{2em}
\setstretch{.5}
{\PaliGlossB{had such ethics, such qualities, such wisdom, such meditation, or such freedom.}}\\
\end{addmargin}
\end{absolutelynopagebreak}

\begin{absolutelynopagebreak}
\setstretch{.7}
{\PaliGlossA{Sā tassā saddhañca sīlañca sutañca cāgañca paññañca anussarantī tadatthāya cittaṃ upasaṃharati.}}\\
\begin{addmargin}[1em]{2em}
\setstretch{.5}
{\PaliGlossB{Recollecting that nun’s faith, ethics, learning, generosity, and wisdom, she applies her mind to that end.}}\\
\end{addmargin}
\end{absolutelynopagebreak}

\begin{absolutelynopagebreak}
\setstretch{.7}
{\PaliGlossA{Evampi kho, anuruddhā, bhikkhuniyā phāsuvihāro hoti.}}\\
\begin{addmargin}[1em]{2em}
\setstretch{.5}
{\PaliGlossB{That too is how a nun lives at ease.}}\\
\end{addmargin}
\end{absolutelynopagebreak}

\vskip 0.05in
\begin{absolutelynopagebreak}
\setstretch{.7}
{\PaliGlossA{18. Idhānuruddhā, upāsako suṇāti:}}\\
\begin{addmargin}[1em]{2em}
\setstretch{.5}
{\PaliGlossB{Take a layman who hears this:}}\\
\end{addmargin}
\end{absolutelynopagebreak}

\begin{absolutelynopagebreak}
\setstretch{.7}
{\PaliGlossA{‘itthannāmo upāsako kālaṅkato;}}\\
\begin{addmargin}[1em]{2em}
\setstretch{.5}
{\PaliGlossB{‘The layman named so-and-so has passed away.}}\\
\end{addmargin}
\end{absolutelynopagebreak}

\begin{absolutelynopagebreak}
\setstretch{.7}
{\PaliGlossA{so bhagavatā byākato—}}\\
\begin{addmargin}[1em]{2em}
\setstretch{.5}
{\PaliGlossB{The Buddha has declared that,}}\\
\end{addmargin}
\end{absolutelynopagebreak}

\begin{absolutelynopagebreak}
\setstretch{.7}
{\PaliGlossA{pañcannaṃ orambhāgiyānaṃ saṃyojanānaṃ parikkhayā opapātiko tattha parinibbāyī anāvattidhammo tasmā lokā’ti.}}\\
\begin{addmargin}[1em]{2em}
\setstretch{.5}
{\PaliGlossB{with the ending of the five lower fetters, he’s been reborn spontaneously and will become extinguished there, not liable to return from that world.’}}\\
\end{addmargin}
\end{absolutelynopagebreak}

\begin{absolutelynopagebreak}
\setstretch{.7}
{\PaliGlossA{So kho panassa āyasmā sāmaṃ diṭṭho vā hoti anussavassuto vā:}}\\
\begin{addmargin}[1em]{2em}
\setstretch{.5}
{\PaliGlossB{And he’s either seen for himself, or heard from someone else, that that venerable}}\\
\end{addmargin}
\end{absolutelynopagebreak}

\begin{absolutelynopagebreak}
\setstretch{.7}
{\PaliGlossA{‘evaṃsīlo so āyasmā ahosi itipi, evaṃdhammo so āyasmā ahosi itipi, evaṃpañño so āyasmā ahosi itipi, evaṃvihārī so āyasmā ahosi itipi, evaṃvimutto so āyasmā ahosi itipī’ti.}}\\
\begin{addmargin}[1em]{2em}
\setstretch{.5}
{\PaliGlossB{had such ethics, such qualities, such wisdom, such meditation, or such freedom.}}\\
\end{addmargin}
\end{absolutelynopagebreak}

\begin{absolutelynopagebreak}
\setstretch{.7}
{\PaliGlossA{So tassa saddhañca sutañca cāgañca paññañca anussaranto tadatthāya cittaṃ upasaṃharati.}}\\
\begin{addmargin}[1em]{2em}
\setstretch{.5}
{\PaliGlossB{Recollecting that layman’s faith, ethics, learning, generosity, and wisdom, he applies his mind to that end.}}\\
\end{addmargin}
\end{absolutelynopagebreak}

\begin{absolutelynopagebreak}
\setstretch{.7}
{\PaliGlossA{Evampi kho, anuruddhā, upāsakassa phāsuvihāro hoti.}}\\
\begin{addmargin}[1em]{2em}
\setstretch{.5}
{\PaliGlossB{That’s how a layman lives at ease.}}\\
\end{addmargin}
\end{absolutelynopagebreak}

\vskip 0.05in
\begin{absolutelynopagebreak}
\setstretch{.7}
{\PaliGlossA{19. Idhānuruddhā, upāsako suṇāti:}}\\
\begin{addmargin}[1em]{2em}
\setstretch{.5}
{\PaliGlossB{Take a layman who hears this:}}\\
\end{addmargin}
\end{absolutelynopagebreak}

\begin{absolutelynopagebreak}
\setstretch{.7}
{\PaliGlossA{‘itthannāmo upāsako kālaṅkato;}}\\
\begin{addmargin}[1em]{2em}
\setstretch{.5}
{\PaliGlossB{‘The layman named so-and-so has passed away.}}\\
\end{addmargin}
\end{absolutelynopagebreak}

\begin{absolutelynopagebreak}
\setstretch{.7}
{\PaliGlossA{so bhagavatā byākato—}}\\
\begin{addmargin}[1em]{2em}
\setstretch{.5}
{\PaliGlossB{The Buddha has declared that,}}\\
\end{addmargin}
\end{absolutelynopagebreak}

\begin{absolutelynopagebreak}
\setstretch{.7}
{\PaliGlossA{tiṇṇaṃ saṃyojanānaṃ parikkhayā rāgadosamohānaṃ tanuttā sakadāgāmī sakideva imaṃ lokaṃ āgantvā dukkhassantaṃ karissatī’ti.}}\\
\begin{addmargin}[1em]{2em}
\setstretch{.5}
{\PaliGlossB{with the ending of three fetters, and the weakening of greed, hate, and delusion, he’s a once-returner. He’ll come back to this world once only, then make an end of suffering.’}}\\
\end{addmargin}
\end{absolutelynopagebreak}

\begin{absolutelynopagebreak}
\setstretch{.7}
{\PaliGlossA{So kho panassa āyasmā sāmaṃ diṭṭho vā hoti anussavassuto vā:}}\\
\begin{addmargin}[1em]{2em}
\setstretch{.5}
{\PaliGlossB{And he’s either seen for himself, or heard from someone else, that that venerable}}\\
\end{addmargin}
\end{absolutelynopagebreak}

\begin{absolutelynopagebreak}
\setstretch{.7}
{\PaliGlossA{‘evaṃsīlo so āyasmā ahosi itipi, evaṃdhammo … evaṃpañño … evaṃvihārī … evaṃvimutto so āyasmā ahosi itipī’ti.}}\\
\begin{addmargin}[1em]{2em}
\setstretch{.5}
{\PaliGlossB{had such ethics, such qualities, such wisdom, such meditation, or such freedom.}}\\
\end{addmargin}
\end{absolutelynopagebreak}

\begin{absolutelynopagebreak}
\setstretch{.7}
{\PaliGlossA{So tassa saddhañca sīlañca sutañca cāgañca paññañca anussaranto tadatthāya cittaṃ upasaṃharati.}}\\
\begin{addmargin}[1em]{2em}
\setstretch{.5}
{\PaliGlossB{Recollecting that layman’s faith, ethics, learning, generosity, and wisdom, he applies his mind to that end.}}\\
\end{addmargin}
\end{absolutelynopagebreak}

\begin{absolutelynopagebreak}
\setstretch{.7}
{\PaliGlossA{Evampi kho, anuruddhā, upāsakassa phāsuvihāro hoti.}}\\
\begin{addmargin}[1em]{2em}
\setstretch{.5}
{\PaliGlossB{That too is how a layman lives at ease.}}\\
\end{addmargin}
\end{absolutelynopagebreak}

\vskip 0.05in
\begin{absolutelynopagebreak}
\setstretch{.7}
{\PaliGlossA{20. Idhānuruddhā, upāsako suṇāti:}}\\
\begin{addmargin}[1em]{2em}
\setstretch{.5}
{\PaliGlossB{Take a layman who hears this:}}\\
\end{addmargin}
\end{absolutelynopagebreak}

\begin{absolutelynopagebreak}
\setstretch{.7}
{\PaliGlossA{‘itthannāmo upāsako kālaṅkato;}}\\
\begin{addmargin}[1em]{2em}
\setstretch{.5}
{\PaliGlossB{‘The layman named so-and-so has passed away.}}\\
\end{addmargin}
\end{absolutelynopagebreak}

\begin{absolutelynopagebreak}
\setstretch{.7}
{\PaliGlossA{so bhagavatā byākato—}}\\
\begin{addmargin}[1em]{2em}
\setstretch{.5}
{\PaliGlossB{The Buddha has declared that,}}\\
\end{addmargin}
\end{absolutelynopagebreak}

\begin{absolutelynopagebreak}
\setstretch{.7}
{\PaliGlossA{tiṇṇaṃ saṃyojanānaṃ parikkhayā sotāpanno avinipātadhammo niyato sambodhiparāyaṇo’ti.}}\\
\begin{addmargin}[1em]{2em}
\setstretch{.5}
{\PaliGlossB{with the ending of three fetters he’s a stream-enterer, not liable to be reborn in the underworld, bound for awakening.’}}\\
\end{addmargin}
\end{absolutelynopagebreak}

\begin{absolutelynopagebreak}
\setstretch{.7}
{\PaliGlossA{So kho panassa āyasmā sāmaṃ diṭṭho vā hoti anussavassuto vā:}}\\
\begin{addmargin}[1em]{2em}
\setstretch{.5}
{\PaliGlossB{And he’s either seen for himself, or heard from someone else, that that venerable}}\\
\end{addmargin}
\end{absolutelynopagebreak}

\begin{absolutelynopagebreak}
\setstretch{.7}
{\PaliGlossA{‘evaṃsīlo so āyasmā ahosi itipi, evaṃdhammo … evaṃpañño … evaṃvihārī … evaṃvimutto so āyasmā ahosi itipī’ti.}}\\
\begin{addmargin}[1em]{2em}
\setstretch{.5}
{\PaliGlossB{had such ethics, such qualities, such wisdom, such meditation, or such freedom.}}\\
\end{addmargin}
\end{absolutelynopagebreak}

\begin{absolutelynopagebreak}
\setstretch{.7}
{\PaliGlossA{So tassa saddhañca sīlañca sutañca cāgañca paññañca anussaranto tadatthāya cittaṃ upasaṃharati.}}\\
\begin{addmargin}[1em]{2em}
\setstretch{.5}
{\PaliGlossB{Recollecting that layman’s faith, ethics, learning, generosity, and wisdom, he applies his mind to that end.}}\\
\end{addmargin}
\end{absolutelynopagebreak}

\begin{absolutelynopagebreak}
\setstretch{.7}
{\PaliGlossA{Evampi kho, anuruddhā upāsakassa phāsuvihāro hoti.}}\\
\begin{addmargin}[1em]{2em}
\setstretch{.5}
{\PaliGlossB{That too is how a layman lives at ease.}}\\
\end{addmargin}
\end{absolutelynopagebreak}

\vskip 0.05in
\begin{absolutelynopagebreak}
\setstretch{.7}
{\PaliGlossA{21. Idhānuruddhā, upāsikā suṇāti:}}\\
\begin{addmargin}[1em]{2em}
\setstretch{.5}
{\PaliGlossB{Take a laywoman who hears this:}}\\
\end{addmargin}
\end{absolutelynopagebreak}

\begin{absolutelynopagebreak}
\setstretch{.7}
{\PaliGlossA{‘itthannāmā upāsikā kālaṅkatā;}}\\
\begin{addmargin}[1em]{2em}
\setstretch{.5}
{\PaliGlossB{‘The laywoman named so-and-so has passed away.}}\\
\end{addmargin}
\end{absolutelynopagebreak}

\begin{absolutelynopagebreak}
\setstretch{.7}
{\PaliGlossA{sā bhagavatā byākatā—}}\\
\begin{addmargin}[1em]{2em}
\setstretch{.5}
{\PaliGlossB{The Buddha has declared that,}}\\
\end{addmargin}
\end{absolutelynopagebreak}

\begin{absolutelynopagebreak}
\setstretch{.7}
{\PaliGlossA{pañcannaṃ orambhāgiyānaṃ saṃyojanānaṃ parikkhayā opapātikā tattha parinibbāyinī anāvattidhammā tasmā lokā’ti.}}\\
\begin{addmargin}[1em]{2em}
\setstretch{.5}
{\PaliGlossB{with the ending of the five lower fetters, she’s been reborn spontaneously and will become extinguished there, not liable to return from that world.’}}\\
\end{addmargin}
\end{absolutelynopagebreak}

\begin{absolutelynopagebreak}
\setstretch{.7}
{\PaliGlossA{Sā kho panassā bhaginī sāmaṃ diṭṭhā vā hoti anussavassutā vā:}}\\
\begin{addmargin}[1em]{2em}
\setstretch{.5}
{\PaliGlossB{And she’s either seen for herself, or heard from someone else, that that sister}}\\
\end{addmargin}
\end{absolutelynopagebreak}

\begin{absolutelynopagebreak}
\setstretch{.7}
{\PaliGlossA{‘evaṃsīlā sā bhaginī ahosi itipi, evaṃdhammā … evaṃpaññā … evaṃvihārinī … evaṃvimuttā sā bhaginī ahosi itipī’ti.}}\\
\begin{addmargin}[1em]{2em}
\setstretch{.5}
{\PaliGlossB{had such ethics, such qualities, such wisdom, such meditation, or such freedom.}}\\
\end{addmargin}
\end{absolutelynopagebreak}

\begin{absolutelynopagebreak}
\setstretch{.7}
{\PaliGlossA{Sā tassā saddhañca sīlañca sutañca cāgañca paññañca anussarantī tadatthāya cittaṃ upasaṃharati.}}\\
\begin{addmargin}[1em]{2em}
\setstretch{.5}
{\PaliGlossB{Recollecting that laywoman’s faith, ethics, learning, generosity, and wisdom, she applies her mind to that end.}}\\
\end{addmargin}
\end{absolutelynopagebreak}

\begin{absolutelynopagebreak}
\setstretch{.7}
{\PaliGlossA{Evampi kho, anuruddhā, upāsikāya phāsuvihāro hoti.}}\\
\begin{addmargin}[1em]{2em}
\setstretch{.5}
{\PaliGlossB{That’s how a laywoman lives at ease.}}\\
\end{addmargin}
\end{absolutelynopagebreak}

\vskip 0.05in
\begin{absolutelynopagebreak}
\setstretch{.7}
{\PaliGlossA{22. Idhānuruddhā, upāsikā suṇāti:}}\\
\begin{addmargin}[1em]{2em}
\setstretch{.5}
{\PaliGlossB{Take a laywoman who hears this:}}\\
\end{addmargin}
\end{absolutelynopagebreak}

\begin{absolutelynopagebreak}
\setstretch{.7}
{\PaliGlossA{‘itthannāmā upāsikā kālaṅkatā;}}\\
\begin{addmargin}[1em]{2em}
\setstretch{.5}
{\PaliGlossB{‘The laywoman named so-and-so has passed away.}}\\
\end{addmargin}
\end{absolutelynopagebreak}

\begin{absolutelynopagebreak}
\setstretch{.7}
{\PaliGlossA{sā bhagavatā byākatā—}}\\
\begin{addmargin}[1em]{2em}
\setstretch{.5}
{\PaliGlossB{The Buddha has declared that,}}\\
\end{addmargin}
\end{absolutelynopagebreak}

\begin{absolutelynopagebreak}
\setstretch{.7}
{\PaliGlossA{tiṇṇaṃ saṃyojanānaṃ parikkhayā rāgadosamohānaṃ tanuttā sakadāgāminī sakideva imaṃ lokaṃ āgantvā dukkhassantaṃ karissatī’ti.}}\\
\begin{addmargin}[1em]{2em}
\setstretch{.5}
{\PaliGlossB{with the ending of three fetters, and the weakening of greed, hate, and delusion, she’s a once-returner. She’ll come back to this world once only, then make an end of suffering.’}}\\
\end{addmargin}
\end{absolutelynopagebreak}

\begin{absolutelynopagebreak}
\setstretch{.7}
{\PaliGlossA{Sā kho panassā bhaginī sāmaṃ diṭṭhā vā hoti anussavassutā vā:}}\\
\begin{addmargin}[1em]{2em}
\setstretch{.5}
{\PaliGlossB{And she’s either seen for herself, or heard from someone else, that that sister}}\\
\end{addmargin}
\end{absolutelynopagebreak}

\begin{absolutelynopagebreak}
\setstretch{.7}
{\PaliGlossA{‘evaṃsīlā sā bhaginī ahosi itipi, evaṃdhammā … evaṃpaññā … evaṃvihārinī … evaṃvimuttā sā bhaginī ahosi itipī’ti.}}\\
\begin{addmargin}[1em]{2em}
\setstretch{.5}
{\PaliGlossB{had such ethics, such qualities, such wisdom, such meditation, or such freedom.}}\\
\end{addmargin}
\end{absolutelynopagebreak}

\begin{absolutelynopagebreak}
\setstretch{.7}
{\PaliGlossA{Sā tassā saddhañca sīlañca sutañca cāgañca paññañca anussarantī tadatthāya cittaṃ upasaṃharati.}}\\
\begin{addmargin}[1em]{2em}
\setstretch{.5}
{\PaliGlossB{Recollecting that laywoman’s faith, ethics, learning, generosity, and wisdom, she applies her mind to that end.}}\\
\end{addmargin}
\end{absolutelynopagebreak}

\begin{absolutelynopagebreak}
\setstretch{.7}
{\PaliGlossA{Evampi kho, anuruddhā, upāsikāya phāsuvihāro hoti.}}\\
\begin{addmargin}[1em]{2em}
\setstretch{.5}
{\PaliGlossB{That too is how a laywoman lives at ease.}}\\
\end{addmargin}
\end{absolutelynopagebreak}

\vskip 0.05in
\begin{absolutelynopagebreak}
\setstretch{.7}
{\PaliGlossA{23. Idhānuruddhā, upāsikā suṇāti:}}\\
\begin{addmargin}[1em]{2em}
\setstretch{.5}
{\PaliGlossB{Take a laywoman who hears this:}}\\
\end{addmargin}
\end{absolutelynopagebreak}

\begin{absolutelynopagebreak}
\setstretch{.7}
{\PaliGlossA{‘itthannāmā upāsikā kālaṅkatā;}}\\
\begin{addmargin}[1em]{2em}
\setstretch{.5}
{\PaliGlossB{‘The laywoman named so-and-so has passed away.}}\\
\end{addmargin}
\end{absolutelynopagebreak}

\begin{absolutelynopagebreak}
\setstretch{.7}
{\PaliGlossA{sā bhagavatā byākatā—}}\\
\begin{addmargin}[1em]{2em}
\setstretch{.5}
{\PaliGlossB{The Buddha has declared that,}}\\
\end{addmargin}
\end{absolutelynopagebreak}

\begin{absolutelynopagebreak}
\setstretch{.7}
{\PaliGlossA{tiṇṇaṃ saṃyojanānaṃ parikkhayā sotāpannā avinipātadhammā niyatā sambodhiparāyaṇā’ti.}}\\
\begin{addmargin}[1em]{2em}
\setstretch{.5}
{\PaliGlossB{with the ending of three fetters she’s a stream-enterer, not liable to be reborn in the underworld, bound for awakening.’}}\\
\end{addmargin}
\end{absolutelynopagebreak}

\begin{absolutelynopagebreak}
\setstretch{.7}
{\PaliGlossA{Sā kho panassā bhaginī sāmaṃ diṭṭhā vā hoti anussavassutā vā:}}\\
\begin{addmargin}[1em]{2em}
\setstretch{.5}
{\PaliGlossB{And she’s either seen for herself, or heard from someone else, that that sister}}\\
\end{addmargin}
\end{absolutelynopagebreak}

\begin{absolutelynopagebreak}
\setstretch{.7}
{\PaliGlossA{‘evaṃsīlā sā bhaginī ahosi itipi, evaṃdhammā sā bhaginī ahosi itipi, evaṃpaññā sā bhaginī ahosi itipi, evaṃvihārinī sā bhaginī ahosi itipi, evaṃvimuttā sā bhaginī ahosi itipī’ti.}}\\
\begin{addmargin}[1em]{2em}
\setstretch{.5}
{\PaliGlossB{had such ethics, such qualities, such wisdom, such meditation, or such freedom.}}\\
\end{addmargin}
\end{absolutelynopagebreak}

\begin{absolutelynopagebreak}
\setstretch{.7}
{\PaliGlossA{Sā tassā saddhañca sīlañca sutañca cāgañca paññañca anussarantī tadatthāya cittaṃ upasaṃharati.}}\\
\begin{addmargin}[1em]{2em}
\setstretch{.5}
{\PaliGlossB{Recollecting that laywoman’s faith, ethics, learning, generosity, and wisdom, she applies her mind to that end.}}\\
\end{addmargin}
\end{absolutelynopagebreak}

\begin{absolutelynopagebreak}
\setstretch{.7}
{\PaliGlossA{Evampi kho, anuruddhā, upāsikāya phāsuvihāro hoti.}}\\
\begin{addmargin}[1em]{2em}
\setstretch{.5}
{\PaliGlossB{That too is how a laywoman lives at ease.}}\\
\end{addmargin}
\end{absolutelynopagebreak}

\vskip 0.05in
\begin{absolutelynopagebreak}
\setstretch{.7}
{\PaliGlossA{28. Iti kho, anuruddhā, tathāgato na janakuhanatthaṃ na janalapanatthaṃ na lābhasakkārasilokānisaṃsatthaṃ na ‘iti maṃ jano jānātū’ti sāvake abbhatīte kālaṅkate upapattīsu byākaroti:}}\\
\begin{addmargin}[1em]{2em}
\setstretch{.5}
{\PaliGlossB{So it’s not for the sake of deceiving people or flattering them, nor for the benefit of possessions, honor, or popularity, nor thinking, ‘So let people know about me!’ that the Realized One declares the rebirth of his disciples who have passed away:}}\\
\end{addmargin}
\end{absolutelynopagebreak}

\begin{absolutelynopagebreak}
\setstretch{.7}
{\PaliGlossA{‘asu amutra upapanno, asu amutra upapanno’ti.}}\\
\begin{addmargin}[1em]{2em}
\setstretch{.5}
{\PaliGlossB{‘This one is reborn here, while that one is reborn there.’}}\\
\end{addmargin}
\end{absolutelynopagebreak}

\begin{absolutelynopagebreak}
\setstretch{.7}
{\PaliGlossA{Santi ca kho, anuruddhā, kulaputtā saddhā uḷāravedā uḷārapāmojjā.}}\\
\begin{addmargin}[1em]{2em}
\setstretch{.5}
{\PaliGlossB{Rather, there are gentlemen of faith who are full of joy and gladness.}}\\
\end{addmargin}
\end{absolutelynopagebreak}

\begin{absolutelynopagebreak}
\setstretch{.7}
{\PaliGlossA{Te taṃ sutvā tadatthāya cittaṃ upasaṃharanti.}}\\
\begin{addmargin}[1em]{2em}
\setstretch{.5}
{\PaliGlossB{When they hear that, they apply their minds to that end.}}\\
\end{addmargin}
\end{absolutelynopagebreak}

\begin{absolutelynopagebreak}
\setstretch{.7}
{\PaliGlossA{Tesaṃ taṃ, anuruddhā, hoti dīgharattaṃ hitāya sukhāyā”ti.}}\\
\begin{addmargin}[1em]{2em}
\setstretch{.5}
{\PaliGlossB{That is for their lasting welfare and happiness.”}}\\
\end{addmargin}
\end{absolutelynopagebreak}

\begin{absolutelynopagebreak}
\setstretch{.7}
{\PaliGlossA{Idamavoca bhagavā.}}\\
\begin{addmargin}[1em]{2em}
\setstretch{.5}
{\PaliGlossB{That is what the Buddha said.}}\\
\end{addmargin}
\end{absolutelynopagebreak}

\begin{absolutelynopagebreak}
\setstretch{.7}
{\PaliGlossA{Attamano āyasmā anuruddho bhagavato bhāsitaṃ abhinandīti.}}\\
\begin{addmargin}[1em]{2em}
\setstretch{.5}
{\PaliGlossB{Satisfied, Venerable Anuruddha and friends were happy with what the Buddha said.}}\\
\end{addmargin}
\end{absolutelynopagebreak}

\begin{absolutelynopagebreak}
\setstretch{.7}
{\PaliGlossA{Naḷakapānasuttaṃ niṭṭhitaṃ aṭṭhamaṃ.}}\\
\begin{addmargin}[1em]{2em}
\setstretch{.5}
{\PaliGlossB{    -}}\\
\end{addmargin}
\end{absolutelynopagebreak}
