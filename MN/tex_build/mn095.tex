
\begin{absolutelynopagebreak}
\setstretch{.7}
{\PaliGlossA{Majjhima Nikāya 95}}\\
\begin{addmargin}[1em]{2em}
\setstretch{.5}
{\PaliGlossB{Middle Discourses 95}}\\
\end{addmargin}
\end{absolutelynopagebreak}

\begin{absolutelynopagebreak}
\setstretch{.7}
{\PaliGlossA{Caṅkīsutta}}\\
\begin{addmargin}[1em]{2em}
\setstretch{.5}
{\PaliGlossB{With Caṅkī}}\\
\end{addmargin}
\end{absolutelynopagebreak}

\vskip 0.05in
\begin{absolutelynopagebreak}
\setstretch{.7}
{\PaliGlossA{Evaṃ me sutaṃ—}}\\
\begin{addmargin}[1em]{2em}
\setstretch{.5}
{\PaliGlossB{So I have heard.}}\\
\end{addmargin}
\end{absolutelynopagebreak}

\begin{absolutelynopagebreak}
\setstretch{.7}
{\PaliGlossA{ekaṃ samayaṃ bhagavā kosalesu cārikaṃ caramāno mahatā bhikkhusaṅghena saddhiṃ yena opāsādaṃ nāma kosalānaṃ brāhmaṇagāmo tadavasari.}}\\
\begin{addmargin}[1em]{2em}
\setstretch{.5}
{\PaliGlossB{At one time the Buddha was wandering in the land of the Kosalans together with a large Saṅgha of mendicants when he arrived at a village of the Kosalan brahmins named Opāsāda.}}\\
\end{addmargin}
\end{absolutelynopagebreak}

\begin{absolutelynopagebreak}
\setstretch{.7}
{\PaliGlossA{Tatra sudaṃ bhagavā opāsāde viharati uttarena opāsādaṃ devavane sālavane.}}\\
\begin{addmargin}[1em]{2em}
\setstretch{.5}
{\PaliGlossB{He stayed in a sal grove to the north of Opāsāda called the “Gods’ Grove”.}}\\
\end{addmargin}
\end{absolutelynopagebreak}

\vskip 0.05in
\begin{absolutelynopagebreak}
\setstretch{.7}
{\PaliGlossA{Tena kho pana samayena caṅkī brāhmaṇo opāsādaṃ ajjhāvasati sattussadaṃ satiṇakaṭṭhodakaṃ sadhaññaṃ rājabhoggaṃ raññā pasenadinā kosalena dinnaṃ rājadāyaṃ brahmadeyyaṃ.}}\\
\begin{addmargin}[1em]{2em}
\setstretch{.5}
{\PaliGlossB{Now at that time the brahmin Caṅkī was living in Opāsāda. It was a crown property given by King Pasenadi of Kosala, teeming with living creatures, full of hay, wood, water, and grain, a royal endowment of the highest quality.}}\\
\end{addmargin}
\end{absolutelynopagebreak}

\vskip 0.05in
\begin{absolutelynopagebreak}
\setstretch{.7}
{\PaliGlossA{Assosuṃ kho opāsādakā brāhmaṇagahapatikā:}}\\
\begin{addmargin}[1em]{2em}
\setstretch{.5}
{\PaliGlossB{The brahmins and householders of Opāsāda heard:}}\\
\end{addmargin}
\end{absolutelynopagebreak}

\begin{absolutelynopagebreak}
\setstretch{.7}
{\PaliGlossA{“samaṇo khalu, bho, gotamo sakyaputto sakyakulā pabbajito kosalesu cārikaṃ caramāno mahatā bhikkhusaṅghena saddhiṃ opāsādaṃ anuppatto, opāsāde viharati uttarena opāsādaṃ devavane sālavane.}}\\
\begin{addmargin}[1em]{2em}
\setstretch{.5}
{\PaliGlossB{“It seems the ascetic Gotama—a Sakyan, gone forth from a Sakyan family—has arrived at Opāsāda together with a large Saṅgha of mendicants. He is staying in the God’s Grove to the north.}}\\
\end{addmargin}
\end{absolutelynopagebreak}

\begin{absolutelynopagebreak}
\setstretch{.7}
{\PaliGlossA{Taṃ kho pana bhavantaṃ gotamaṃ evaṃ kalyāṇo kittisaddo abbhuggato:}}\\
\begin{addmargin}[1em]{2em}
\setstretch{.5}
{\PaliGlossB{He has this good reputation:}}\\
\end{addmargin}
\end{absolutelynopagebreak}

\begin{absolutelynopagebreak}
\setstretch{.7}
{\PaliGlossA{‘itipi so bhagavā arahaṃ sammāsambuddho vijjācaraṇasampanno sugato lokavidū anuttaro purisadammasārathi satthā devamanussānaṃ buddho bhagavā’ti.}}\\
\begin{addmargin}[1em]{2em}
\setstretch{.5}
{\PaliGlossB{‘That Blessed One is perfected, a fully awakened Buddha, accomplished in knowledge and conduct, holy, knower of the world, supreme guide for those who wish to train, teacher of gods and humans, awakened, blessed.’}}\\
\end{addmargin}
\end{absolutelynopagebreak}

\begin{absolutelynopagebreak}
\setstretch{.7}
{\PaliGlossA{So imaṃ lokaṃ sadevakaṃ samārakaṃ sabrahmakaṃ sassamaṇabrāhmaṇiṃ pajaṃ sadevamanussaṃ sayaṃ abhiññā sacchikatvā pavedeti.}}\\
\begin{addmargin}[1em]{2em}
\setstretch{.5}
{\PaliGlossB{He has realized with his own insight this world—with its gods, Māras and Brahmās, this population with its ascetics and brahmins, gods and humans—and he makes it known to others.}}\\
\end{addmargin}
\end{absolutelynopagebreak}

\begin{absolutelynopagebreak}
\setstretch{.7}
{\PaliGlossA{So dhammaṃ deseti ādikalyāṇaṃ majjhekalyāṇaṃ pariyosānakalyāṇaṃ sātthaṃ sabyañjanaṃ, kevalaparipuṇṇaṃ parisuddhaṃ brahmacariyaṃ pakāseti.}}\\
\begin{addmargin}[1em]{2em}
\setstretch{.5}
{\PaliGlossB{He teaches Dhamma that’s good in the beginning, good in the middle, and good in the end, meaningful and well-phrased. And he reveals a spiritual practice that’s entirely full and pure.}}\\
\end{addmargin}
\end{absolutelynopagebreak}

\begin{absolutelynopagebreak}
\setstretch{.7}
{\PaliGlossA{Sādhu kho pana tathārūpānaṃ arahataṃ dassanaṃ hotī”ti.}}\\
\begin{addmargin}[1em]{2em}
\setstretch{.5}
{\PaliGlossB{It’s good to see such perfected ones.”}}\\
\end{addmargin}
\end{absolutelynopagebreak}

\vskip 0.05in
\begin{absolutelynopagebreak}
\setstretch{.7}
{\PaliGlossA{Atha kho opāsādakā brāhmaṇagahapatikā opāsādā nikkhamitvā saṅghasaṅghī gaṇībhūtā uttarenamukhā gacchanti yena devavanaṃ sālavanaṃ.}}\\
\begin{addmargin}[1em]{2em}
\setstretch{.5}
{\PaliGlossB{Then, having departed Opāsāda, they formed into companies and headed north to the God’s Grove.}}\\
\end{addmargin}
\end{absolutelynopagebreak}

\vskip 0.05in
\begin{absolutelynopagebreak}
\setstretch{.7}
{\PaliGlossA{Tena kho pana samayena caṅkī brāhmaṇo uparipāsāde divāseyyaṃ upagato.}}\\
\begin{addmargin}[1em]{2em}
\setstretch{.5}
{\PaliGlossB{Now at that time the brahmin Caṅkī had retired to the upper floor of his stilt longhouse for his midday nap.}}\\
\end{addmargin}
\end{absolutelynopagebreak}

\begin{absolutelynopagebreak}
\setstretch{.7}
{\PaliGlossA{Addasā kho caṅkī brāhmaṇo opāsādake brāhmaṇagahapatike opāsādā nikkhamitvā saṅghasaṅghī gaṇībhūte uttarenamukhaṃ yena devavanaṃ sālavanaṃ tenupasaṅkamante.}}\\
\begin{addmargin}[1em]{2em}
\setstretch{.5}
{\PaliGlossB{He saw the brahmins and householders heading for the God’s Grove,}}\\
\end{addmargin}
\end{absolutelynopagebreak}

\begin{absolutelynopagebreak}
\setstretch{.7}
{\PaliGlossA{Disvā khattaṃ āmantesi:}}\\
\begin{addmargin}[1em]{2em}
\setstretch{.5}
{\PaliGlossB{and addressed his steward,}}\\
\end{addmargin}
\end{absolutelynopagebreak}

\begin{absolutelynopagebreak}
\setstretch{.7}
{\PaliGlossA{“kiṃ nu kho, bho khatte, opāsādakā brāhmaṇagahapatikā opāsādā nikkhamitvā saṅghasaṅghī gaṇībhūtā uttarenamukhā gacchanti yena devavanaṃ sālavanan”ti?}}\\
\begin{addmargin}[1em]{2em}
\setstretch{.5}
{\PaliGlossB{“My steward, why are the brahmins and householders heading north for the God’s Grove?”}}\\
\end{addmargin}
\end{absolutelynopagebreak}

\vskip 0.05in
\begin{absolutelynopagebreak}
\setstretch{.7}
{\PaliGlossA{“Atthi, bho caṅkī, samaṇo gotamo sakyaputto sakyakulā pabbajito kosalesu cārikaṃ caramāno mahatā bhikkhusaṅghena saddhiṃ opāsādaṃ anuppatto, opāsāde viharati uttarena opāsādaṃ devavane sālavane.}}\\
\begin{addmargin}[1em]{2em}
\setstretch{.5}
{\PaliGlossB{“The ascetic Gotama has arrived at Opāsāda together with a large Saṅgha of mendicants. He is staying in the God’s Grove to the north.}}\\
\end{addmargin}
\end{absolutelynopagebreak}

\begin{absolutelynopagebreak}
\setstretch{.7}
{\PaliGlossA{Taṃ kho pana bhavantaṃ gotamaṃ evaṃ kalyāṇo kittisaddo abbhuggato:}}\\
\begin{addmargin}[1em]{2em}
\setstretch{.5}
{\PaliGlossB{He has this good reputation:}}\\
\end{addmargin}
\end{absolutelynopagebreak}

\begin{absolutelynopagebreak}
\setstretch{.7}
{\PaliGlossA{‘itipi so bhagavā arahaṃ sammāsambuddho vijjācaraṇasampanno sugato lokavidū anuttaro purisadammasārathi satthā devamanussānaṃ buddho bhagavā’ti.}}\\
\begin{addmargin}[1em]{2em}
\setstretch{.5}
{\PaliGlossB{‘That Blessed One is perfected, a fully awakened Buddha, accomplished in knowledge and conduct, holy, knower of the world, supreme guide for those who wish to train, teacher of gods and humans, awakened, blessed.’}}\\
\end{addmargin}
\end{absolutelynopagebreak}

\begin{absolutelynopagebreak}
\setstretch{.7}
{\PaliGlossA{Tamete bhavantaṃ gotamaṃ dassanāya gacchantī”ti.}}\\
\begin{addmargin}[1em]{2em}
\setstretch{.5}
{\PaliGlossB{They’re going to see that Master Gotama.”}}\\
\end{addmargin}
\end{absolutelynopagebreak}

\begin{absolutelynopagebreak}
\setstretch{.7}
{\PaliGlossA{“Tena hi, bho khatte, yena opāsādakā brāhmaṇagahapatikā tenupasaṅkama; upasaṅkamitvā opāsādake brāhmaṇagahapatike evaṃ vadehi:}}\\
\begin{addmargin}[1em]{2em}
\setstretch{.5}
{\PaliGlossB{“Well then, go to the brahmins and householders and say to them:}}\\
\end{addmargin}
\end{absolutelynopagebreak}

\begin{absolutelynopagebreak}
\setstretch{.7}
{\PaliGlossA{‘caṅkī, bho, brāhmaṇo evamāha—}}\\
\begin{addmargin}[1em]{2em}
\setstretch{.5}
{\PaliGlossB{“Sirs, the brahmin Caṅkī asks}}\\
\end{addmargin}
\end{absolutelynopagebreak}

\begin{absolutelynopagebreak}
\setstretch{.7}
{\PaliGlossA{āgamentu kira bhonto, caṅkīpi brāhmaṇo samaṇaṃ gotamaṃ dassanāya upasaṅkamissatī’”ti.}}\\
\begin{addmargin}[1em]{2em}
\setstretch{.5}
{\PaliGlossB{you to wait, as he will also go to see the ascetic Gotama.”}}\\
\end{addmargin}
\end{absolutelynopagebreak}

\begin{absolutelynopagebreak}
\setstretch{.7}
{\PaliGlossA{“Evaṃ, bho”ti kho so khatto caṅkissa brāhmaṇassa paṭissutvā yena opāsādakā brāhmaṇagahapatikā tenupasaṅkami; upasaṅkamitvā opāsādake brāhmaṇagahapatike etadavoca:}}\\
\begin{addmargin}[1em]{2em}
\setstretch{.5}
{\PaliGlossB{“Yes, sir,” replied the steward, and did as he was asked.}}\\
\end{addmargin}
\end{absolutelynopagebreak}

\begin{absolutelynopagebreak}
\setstretch{.7}
{\PaliGlossA{“caṅkī, bho, brāhmaṇo evamāha:}}\\
\begin{addmargin}[1em]{2em}
\setstretch{.5}
{\PaliGlossB{    -}}\\
\end{addmargin}
\end{absolutelynopagebreak}

\begin{absolutelynopagebreak}
\setstretch{.7}
{\PaliGlossA{‘āgamentu kira bhonto, caṅkīpi brāhmaṇo samaṇaṃ gotamaṃ dassanāya upasaṅkamissatī’”ti.}}\\
\begin{addmargin}[1em]{2em}
\setstretch{.5}
{\PaliGlossB{    -}}\\
\end{addmargin}
\end{absolutelynopagebreak}

\vskip 0.05in
\begin{absolutelynopagebreak}
\setstretch{.7}
{\PaliGlossA{Tena kho pana samayena nānāverajjakānaṃ brāhmaṇānaṃ pañcamattāni brāhmaṇasatāni opāsāde paṭivasanti kenacideva karaṇīyena.}}\\
\begin{addmargin}[1em]{2em}
\setstretch{.5}
{\PaliGlossB{Now at that time around five hundred brahmins from abroad were residing in Opāsāda on some business.}}\\
\end{addmargin}
\end{absolutelynopagebreak}

\begin{absolutelynopagebreak}
\setstretch{.7}
{\PaliGlossA{Assosuṃ kho te brāhmaṇā:}}\\
\begin{addmargin}[1em]{2em}
\setstretch{.5}
{\PaliGlossB{They heard that}}\\
\end{addmargin}
\end{absolutelynopagebreak}

\begin{absolutelynopagebreak}
\setstretch{.7}
{\PaliGlossA{“caṅkī kira brāhmaṇo samaṇaṃ gotamaṃ dassanāya upasaṅkamissatī”ti.}}\\
\begin{addmargin}[1em]{2em}
\setstretch{.5}
{\PaliGlossB{the brahmin Caṅkī was going to see the ascetic Gotama.}}\\
\end{addmargin}
\end{absolutelynopagebreak}

\begin{absolutelynopagebreak}
\setstretch{.7}
{\PaliGlossA{Atha kho te brāhmaṇā yena caṅkī brāhmaṇo tenupasaṅkamiṃsu; upasaṅkamitvā caṅkiṃ brāhmaṇaṃ etadavocuṃ:}}\\
\begin{addmargin}[1em]{2em}
\setstretch{.5}
{\PaliGlossB{They approached Caṅkī and said to him,}}\\
\end{addmargin}
\end{absolutelynopagebreak}

\begin{absolutelynopagebreak}
\setstretch{.7}
{\PaliGlossA{“saccaṃ kira bhavaṃ caṅkī samaṇaṃ gotamaṃ dassanāya upasaṅkamissatī”ti?}}\\
\begin{addmargin}[1em]{2em}
\setstretch{.5}
{\PaliGlossB{“Is it really true that you are going to see the ascetic Gotama?”}}\\
\end{addmargin}
\end{absolutelynopagebreak}

\begin{absolutelynopagebreak}
\setstretch{.7}
{\PaliGlossA{“Evaṃ kho me, bho, hoti:}}\\
\begin{addmargin}[1em]{2em}
\setstretch{.5}
{\PaliGlossB{“Yes, gentlemen, it is true.”}}\\
\end{addmargin}
\end{absolutelynopagebreak}

\begin{absolutelynopagebreak}
\setstretch{.7}
{\PaliGlossA{‘ahaṃ samaṇaṃ gotamaṃ dassanāya upasaṅkamissāmī’”ti.}}\\
\begin{addmargin}[1em]{2em}
\setstretch{.5}
{\PaliGlossB{    -}}\\
\end{addmargin}
\end{absolutelynopagebreak}

\vskip 0.05in
\begin{absolutelynopagebreak}
\setstretch{.7}
{\PaliGlossA{“Mā bhavaṃ caṅkī samaṇaṃ gotamaṃ dassanāya upasaṅkami.}}\\
\begin{addmargin}[1em]{2em}
\setstretch{.5}
{\PaliGlossB{“Please don’t!}}\\
\end{addmargin}
\end{absolutelynopagebreak}

\begin{absolutelynopagebreak}
\setstretch{.7}
{\PaliGlossA{Na arahati bhavaṃ caṅkī samaṇaṃ gotamaṃ dassanāya upasaṅkamituṃ;}}\\
\begin{addmargin}[1em]{2em}
\setstretch{.5}
{\PaliGlossB{It’s not appropriate for you to go to see the ascetic Gotama;}}\\
\end{addmargin}
\end{absolutelynopagebreak}

\begin{absolutelynopagebreak}
\setstretch{.7}
{\PaliGlossA{samaṇotveva gotamo arahati bhavantaṃ caṅkiṃ dassanāya upasaṅkamituṃ.}}\\
\begin{addmargin}[1em]{2em}
\setstretch{.5}
{\PaliGlossB{it’s appropriate that he comes to see you.}}\\
\end{addmargin}
\end{absolutelynopagebreak}

\begin{absolutelynopagebreak}
\setstretch{.7}
{\PaliGlossA{Bhavañhi caṅkī ubhato sujāto mātito ca pitito ca saṃsuddhagahaṇiko yāva sattamā pitāmahayugā akkhitto anupakkuṭṭho jātivādena.}}\\
\begin{addmargin}[1em]{2em}
\setstretch{.5}
{\PaliGlossB{You are well born on both your mother’s and father’s side, of pure descent, irrefutable and impeccable in questions of ancestry back to the seventh paternal generation.}}\\
\end{addmargin}
\end{absolutelynopagebreak}

\begin{absolutelynopagebreak}
\setstretch{.7}
{\PaliGlossA{Yampi bhavaṃ caṅkī ubhato sujāto mātito ca pitito ca saṃsuddhagahaṇiko yāva sattamā pitāmahayugā akkhitto anupakkuṭṭho jātivādena, imināpaṅgena na arahati bhavaṃ caṅkī samaṇaṃ gotamaṃ dassanāya upasaṅkamituṃ;}}\\
\begin{addmargin}[1em]{2em}
\setstretch{.5}
{\PaliGlossB{For this reason it’s not appropriate for you to go to see the ascetic Gotama;}}\\
\end{addmargin}
\end{absolutelynopagebreak}

\begin{absolutelynopagebreak}
\setstretch{.7}
{\PaliGlossA{samaṇotveva gotamo arahati bhavantaṃ caṅkiṃ dassanāya upasaṅkamituṃ.}}\\
\begin{addmargin}[1em]{2em}
\setstretch{.5}
{\PaliGlossB{it’s appropriate that he comes to see you.}}\\
\end{addmargin}
\end{absolutelynopagebreak}

\begin{absolutelynopagebreak}
\setstretch{.7}
{\PaliGlossA{Bhavañhi caṅkī aḍḍho mahaddhano mahābhogo … pe …}}\\
\begin{addmargin}[1em]{2em}
\setstretch{.5}
{\PaliGlossB{You’re rich, affluent, and wealthy. …}}\\
\end{addmargin}
\end{absolutelynopagebreak}

\begin{absolutelynopagebreak}
\setstretch{.7}
{\PaliGlossA{bhavañhi caṅkī tiṇṇaṃ vedānaṃ pāragū sanighaṇḍukeṭubhānaṃ sākkharappabhedānaṃ itihāsapañcamānaṃ, padako, veyyākaraṇo, lokāyatamahāpurisalakkhaṇesu anavayo … pe …}}\\
\begin{addmargin}[1em]{2em}
\setstretch{.5}
{\PaliGlossB{You recite and remember the hymns, and have mastered the three Vedas, together with their vocabularies, ritual, phonology and etymology, and the testament as fifth. You know philology and grammar, and are well versed in cosmology and the marks of a great man. …}}\\
\end{addmargin}
\end{absolutelynopagebreak}

\begin{absolutelynopagebreak}
\setstretch{.7}
{\PaliGlossA{bhavañhi caṅkī abhirūpo dassanīyo pāsādiko paramāya vaṇṇapokkharatāya samannāgato brahmavaṇṇī brahmavacchasī akhuddāvakāso dassanāya … pe …}}\\
\begin{addmargin}[1em]{2em}
\setstretch{.5}
{\PaliGlossB{You are attractive, good-looking, lovely, of surpassing beauty. You are magnificent, splendid, remarkable to behold. …}}\\
\end{addmargin}
\end{absolutelynopagebreak}

\begin{absolutelynopagebreak}
\setstretch{.7}
{\PaliGlossA{bhavañhi caṅkī sīlavā vuddhasīlī vuddhasīlena samannāgato … pe …}}\\
\begin{addmargin}[1em]{2em}
\setstretch{.5}
{\PaliGlossB{You are ethical, mature in ethical conduct. …}}\\
\end{addmargin}
\end{absolutelynopagebreak}

\begin{absolutelynopagebreak}
\setstretch{.7}
{\PaliGlossA{bhavañhi caṅkī kalyāṇavāco kalyāṇavākkaraṇo poriyā vācāya samannāgato vissaṭṭhāya anelagalāya atthassa viññāpaniyā … pe …}}\\
\begin{addmargin}[1em]{2em}
\setstretch{.5}
{\PaliGlossB{You’re a good speaker, with a polished, clear, and articulate voice that expresses the meaning. …}}\\
\end{addmargin}
\end{absolutelynopagebreak}

\begin{absolutelynopagebreak}
\setstretch{.7}
{\PaliGlossA{bhavañhi caṅkī bahūnaṃ ācariyapācariyo, tīṇi māṇavakasatāni mante vāceti … pe …}}\\
\begin{addmargin}[1em]{2em}
\setstretch{.5}
{\PaliGlossB{You teach the teachers of many, and teach three hundred students to recite the hymns. …}}\\
\end{addmargin}
\end{absolutelynopagebreak}

\begin{absolutelynopagebreak}
\setstretch{.7}
{\PaliGlossA{bhavañhi caṅkī rañño pasenadissa kosalassa sakkato garukato mānito pūjito apacito … pe …}}\\
\begin{addmargin}[1em]{2em}
\setstretch{.5}
{\PaliGlossB{You’re honored, respected, revered, venerated, and esteemed by King Pasenadi of Kosala}}\\
\end{addmargin}
\end{absolutelynopagebreak}

\begin{absolutelynopagebreak}
\setstretch{.7}
{\PaliGlossA{bhavañhi caṅkī brāhmaṇassa pokkharasātissa sakkato garukato mānito pūjito apacito … pe …}}\\
\begin{addmargin}[1em]{2em}
\setstretch{.5}
{\PaliGlossB{and the brahmin Pokkharasāti. …}}\\
\end{addmargin}
\end{absolutelynopagebreak}

\begin{absolutelynopagebreak}
\setstretch{.7}
{\PaliGlossA{bhavañhi caṅkī opāsādaṃ ajjhāvasati sattussadaṃ satiṇakaṭṭhodakaṃ sadhaññaṃ rājabhoggaṃ raññā pasenadinā kosalena dinnaṃ rājadāyaṃ brahmadeyyaṃ.}}\\
\begin{addmargin}[1em]{2em}
\setstretch{.5}
{\PaliGlossB{You live in Opāsāda, a crown property given by King Pasenadi of Kosala, teeming with living creatures, full of hay, wood, water, and grain, a royal endowment of the highest quality.}}\\
\end{addmargin}
\end{absolutelynopagebreak}

\begin{absolutelynopagebreak}
\setstretch{.7}
{\PaliGlossA{Yampi bhavaṃ caṅkī opāsādaṃ ajjhāvasati sattussadaṃ satiṇakaṭṭhodakaṃ sadhaññaṃ rājabhoggaṃ raññā pasenadinā kosalena dinnaṃ rājadāyaṃ brahmadeyyaṃ, imināpaṅgena na arahati bhavaṃ caṅkī samaṇaṃ gotamaṃ dassanāya upasaṅkamituṃ;}}\\
\begin{addmargin}[1em]{2em}
\setstretch{.5}
{\PaliGlossB{For all these reasons it’s not appropriate for you to go to see the ascetic Gotama;}}\\
\end{addmargin}
\end{absolutelynopagebreak}

\begin{absolutelynopagebreak}
\setstretch{.7}
{\PaliGlossA{samaṇotveva gotamo arahati bhavantaṃ caṅkiṃ dassanāya upasaṅkamitun”ti.}}\\
\begin{addmargin}[1em]{2em}
\setstretch{.5}
{\PaliGlossB{it’s appropriate that he comes to see you.”}}\\
\end{addmargin}
\end{absolutelynopagebreak}

\vskip 0.05in
\begin{absolutelynopagebreak}
\setstretch{.7}
{\PaliGlossA{Evaṃ vutte, caṅkī brāhmaṇo te brāhmaṇe etadavoca:}}\\
\begin{addmargin}[1em]{2em}
\setstretch{.5}
{\PaliGlossB{When they had spoken, Caṅkī said to those brahmins:}}\\
\end{addmargin}
\end{absolutelynopagebreak}

\begin{absolutelynopagebreak}
\setstretch{.7}
{\PaliGlossA{“tena hi, bho, mamapi suṇātha, yathā mayameva arahāma taṃ samaṇaṃ gotamaṃ dassanāya upasaṅkamituṃ;}}\\
\begin{addmargin}[1em]{2em}
\setstretch{.5}
{\PaliGlossB{“Well then, gentlemen, listen to why it’s appropriate for me to go to see the ascetic Gotama,}}\\
\end{addmargin}
\end{absolutelynopagebreak}

\begin{absolutelynopagebreak}
\setstretch{.7}
{\PaliGlossA{na tveva arahati so bhavaṃ gotamo amhākaṃ dassanāya upasaṅkamituṃ.}}\\
\begin{addmargin}[1em]{2em}
\setstretch{.5}
{\PaliGlossB{and it’s not appropriate for him to come to see me.}}\\
\end{addmargin}
\end{absolutelynopagebreak}

\begin{absolutelynopagebreak}
\setstretch{.7}
{\PaliGlossA{Samaṇo khalu, bho, gotamo ubhato sujāto mātito ca pitito ca saṃsuddhagahaṇiko yāva sattamā pitāmahayugā akkhitto anupakkuṭṭho jātivādena.}}\\
\begin{addmargin}[1em]{2em}
\setstretch{.5}
{\PaliGlossB{He is well born on both his mother’s and father’s side, of pure descent, irrefutable and impeccable in questions of ancestry back to the seventh paternal generation.}}\\
\end{addmargin}
\end{absolutelynopagebreak}

\begin{absolutelynopagebreak}
\setstretch{.7}
{\PaliGlossA{Yampi, bho, samaṇo gotamo ubhato sujāto mātito ca pitito ca saṃsuddhagahaṇiko yāva sattamā pitāmahayugā akkhitto anupakkuṭṭho jātivādena, imināpaṅgena na arahati so bhavaṃ gotamo amhākaṃ dassanāya upasaṅkamituṃ;}}\\
\begin{addmargin}[1em]{2em}
\setstretch{.5}
{\PaliGlossB{For this reason it’s not appropriate for the ascetic Gotama to come to see me;}}\\
\end{addmargin}
\end{absolutelynopagebreak}

\begin{absolutelynopagebreak}
\setstretch{.7}
{\PaliGlossA{atha kho mayameva arahāma taṃ bhavantaṃ gotamaṃ dassanāya upasaṅkamituṃ.}}\\
\begin{addmargin}[1em]{2em}
\setstretch{.5}
{\PaliGlossB{rather, it’s appropriate for me to go to see him.}}\\
\end{addmargin}
\end{absolutelynopagebreak}

\begin{absolutelynopagebreak}
\setstretch{.7}
{\PaliGlossA{Samaṇo khalu, bho, gotamo pahūtaṃ hiraññasuvaṇṇaṃ ohāya pabbajito bhūmigatañca vehāsaṭṭhañca … pe …}}\\
\begin{addmargin}[1em]{2em}
\setstretch{.5}
{\PaliGlossB{When he went forth he abandoned abundant gold coin and bullion stored in dungeons and towers. …}}\\
\end{addmargin}
\end{absolutelynopagebreak}

\begin{absolutelynopagebreak}
\setstretch{.7}
{\PaliGlossA{Samaṇo khalu, bho, gotamo daharova samāno yuvā susukāḷakeso bhadrena yobbanena samannāgato paṭhamena vayasā agārasmā anagāriyaṃ pabbajito … pe …}}\\
\begin{addmargin}[1em]{2em}
\setstretch{.5}
{\PaliGlossB{He went forth from the lay life to homelessness while still a youth, young, black-haired, blessed with youth, in the prime of life. …}}\\
\end{addmargin}
\end{absolutelynopagebreak}

\begin{absolutelynopagebreak}
\setstretch{.7}
{\PaliGlossA{Samaṇo khalu, bho, gotamo akāmakānaṃ mātāpitūnaṃ assumukhānaṃ rudantānaṃ kesamassuṃ ohāretvā kāsāyāni vatthāni acchādetvā agārasmā anagāriyaṃ pabbajito … pe …}}\\
\begin{addmargin}[1em]{2em}
\setstretch{.5}
{\PaliGlossB{Though his mother and father wished otherwise, weeping with tearful faces, he shaved off his hair and beard, dressed in ocher robes, and went forth from the lay life to homelessness. …}}\\
\end{addmargin}
\end{absolutelynopagebreak}

\begin{absolutelynopagebreak}
\setstretch{.7}
{\PaliGlossA{Samaṇo khalu, bho, gotamo abhirūpo dassanīyo pāsādiko paramāya vaṇṇapokkharatāya samannāgato brahmavaṇṇī brahmavacchasī akhuddāvakāso dassanāya … pe …}}\\
\begin{addmargin}[1em]{2em}
\setstretch{.5}
{\PaliGlossB{He is attractive, good-looking, lovely, of surpassing beauty. He is magnificent, splendid, remarkable to behold. …}}\\
\end{addmargin}
\end{absolutelynopagebreak}

\begin{absolutelynopagebreak}
\setstretch{.7}
{\PaliGlossA{Samaṇo khalu, bho, gotamo sīlavā ariyasīlī kusalasīlī kusalena sīlena samannāgato … pe …}}\\
\begin{addmargin}[1em]{2em}
\setstretch{.5}
{\PaliGlossB{He is ethical, possessing ethical conduct that is noble and skillful. …}}\\
\end{addmargin}
\end{absolutelynopagebreak}

\begin{absolutelynopagebreak}
\setstretch{.7}
{\PaliGlossA{Samaṇo khalu, bho, gotamo kalyāṇavāco kalyāṇavākkaraṇo poriyā vācāya samannāgato vissaṭṭhāya anelagalāya atthassa viññāpaniyā … pe …}}\\
\begin{addmargin}[1em]{2em}
\setstretch{.5}
{\PaliGlossB{He’s a good speaker, with a polished, clear, and articulate voice that expresses the meaning. …}}\\
\end{addmargin}
\end{absolutelynopagebreak}

\begin{absolutelynopagebreak}
\setstretch{.7}
{\PaliGlossA{Samaṇo khalu, bho, gotamo bahūnaṃ ācariyapācariyo … pe …}}\\
\begin{addmargin}[1em]{2em}
\setstretch{.5}
{\PaliGlossB{He’s a teacher of teachers. …}}\\
\end{addmargin}
\end{absolutelynopagebreak}

\begin{absolutelynopagebreak}
\setstretch{.7}
{\PaliGlossA{Samaṇo khalu, bho, gotamo khīṇakāmarāgo vigatacāpallo … pe …}}\\
\begin{addmargin}[1em]{2em}
\setstretch{.5}
{\PaliGlossB{He has ended sensual desire, and is rid of caprice. …}}\\
\end{addmargin}
\end{absolutelynopagebreak}

\begin{absolutelynopagebreak}
\setstretch{.7}
{\PaliGlossA{Samaṇo khalu, bho, gotamo kammavādī kiriyavādī apāpapurekkhāro brahmaññāya pajāya … pe …}}\\
\begin{addmargin}[1em]{2em}
\setstretch{.5}
{\PaliGlossB{He teaches the efficacy of deeds and action. He doesn’t wish any harm upon the community of brahmins. …}}\\
\end{addmargin}
\end{absolutelynopagebreak}

\begin{absolutelynopagebreak}
\setstretch{.7}
{\PaliGlossA{Samaṇo khalu, bho, gotamo uccā kulā pabbajito asambhinnā khattiyakulā … pe …}}\\
\begin{addmargin}[1em]{2em}
\setstretch{.5}
{\PaliGlossB{He went forth from an eminent family of unbroken aristocratic lineage. …}}\\
\end{addmargin}
\end{absolutelynopagebreak}

\begin{absolutelynopagebreak}
\setstretch{.7}
{\PaliGlossA{Samaṇo khalu, bho, gotamo aḍḍhā kulā pabbajito mahaddhanā mahābhogā … pe …}}\\
\begin{addmargin}[1em]{2em}
\setstretch{.5}
{\PaliGlossB{He went forth from a rich, affluent, and wealthy family. …}}\\
\end{addmargin}
\end{absolutelynopagebreak}

\begin{absolutelynopagebreak}
\setstretch{.7}
{\PaliGlossA{Samaṇaṃ khalu, bho, gotamaṃ tiroraṭṭhā tirojanapadā saṃpucchituṃ āgacchanti … pe …}}\\
\begin{addmargin}[1em]{2em}
\setstretch{.5}
{\PaliGlossB{People come from distant lands and distant countries to question him. …}}\\
\end{addmargin}
\end{absolutelynopagebreak}

\begin{absolutelynopagebreak}
\setstretch{.7}
{\PaliGlossA{Samaṇaṃ khalu, bho, gotamaṃ anekāni devatāsahassāni pāṇehi saraṇaṃ gatāni … pe …}}\\
\begin{addmargin}[1em]{2em}
\setstretch{.5}
{\PaliGlossB{Many thousands of deities have gone for refuge for life to him. …}}\\
\end{addmargin}
\end{absolutelynopagebreak}

\begin{absolutelynopagebreak}
\setstretch{.7}
{\PaliGlossA{Samaṇaṃ khalu, bho, gotamaṃ evaṃ kalyāṇo kittisaddo abbhuggato:}}\\
\begin{addmargin}[1em]{2em}
\setstretch{.5}
{\PaliGlossB{He has this good reputation:}}\\
\end{addmargin}
\end{absolutelynopagebreak}

\begin{absolutelynopagebreak}
\setstretch{.7}
{\PaliGlossA{‘itipi so bhagavā arahaṃ sammāsambuddho vijjācaraṇasampanno sugato lokavidū anuttaro purisadammasārathi satthā devamanussānaṃ buddho bhagavā’ti … pe …}}\\
\begin{addmargin}[1em]{2em}
\setstretch{.5}
{\PaliGlossB{‘That Blessed One is perfected, a fully awakened Buddha, accomplished in knowledge and conduct, holy, knower of the world, supreme guide for those who wish to train, teacher of gods and humans, awakened, blessed.’ …}}\\
\end{addmargin}
\end{absolutelynopagebreak}

\begin{absolutelynopagebreak}
\setstretch{.7}
{\PaliGlossA{Samaṇo khalu, bho, gotamo dvattiṃsamahāpurisalakkhaṇehi samannāgato … pe …}}\\
\begin{addmargin}[1em]{2em}
\setstretch{.5}
{\PaliGlossB{He has the thirty-two marks of a great man. …}}\\
\end{addmargin}
\end{absolutelynopagebreak}

\begin{absolutelynopagebreak}
\setstretch{.7}
{\PaliGlossA{Samaṇaṃ khalu, bho, gotamaṃ rājā māgadho seniyo bimbisāro saputtadāro pāṇehi saraṇaṃ gato … pe …}}\\
\begin{addmargin}[1em]{2em}
\setstretch{.5}
{\PaliGlossB{King Seniya Bimbisāra of Magadha and his wives and children have gone for refuge for life to the ascetic Gotama. …}}\\
\end{addmargin}
\end{absolutelynopagebreak}

\begin{absolutelynopagebreak}
\setstretch{.7}
{\PaliGlossA{Samaṇaṃ khalu, bho, gotamaṃ rājā pasenadi kosalo saputtadāro pāṇehi saraṇaṃ gato … pe …}}\\
\begin{addmargin}[1em]{2em}
\setstretch{.5}
{\PaliGlossB{King Pasenadi of Kosala and his wives and children have gone for refuge for life to the ascetic Gotama. …}}\\
\end{addmargin}
\end{absolutelynopagebreak}

\begin{absolutelynopagebreak}
\setstretch{.7}
{\PaliGlossA{Samaṇaṃ khalu, bho, gotamaṃ brāhmaṇo pokkharasāti saputtadāro pāṇehi saraṇaṃ gato … pe …}}\\
\begin{addmargin}[1em]{2em}
\setstretch{.5}
{\PaliGlossB{The brahmin Pokkharasāti and his wives and children have gone for refuge for life to the ascetic Gotama. …}}\\
\end{addmargin}
\end{absolutelynopagebreak}

\begin{absolutelynopagebreak}
\setstretch{.7}
{\PaliGlossA{Samaṇo khalu, bho, gotamo opāsādaṃ anuppatto opāsāde viharati uttarena opāsādaṃ devavane sālavane.}}\\
\begin{addmargin}[1em]{2em}
\setstretch{.5}
{\PaliGlossB{The ascetic Gotama has arrived to stay in the God’s Grove to the north of Opāsāda.}}\\
\end{addmargin}
\end{absolutelynopagebreak}

\begin{absolutelynopagebreak}
\setstretch{.7}
{\PaliGlossA{Ye kho te samaṇā vā brāhmaṇā vā amhākaṃ gāmakkhettaṃ āgacchanti, atithī no te honti.}}\\
\begin{addmargin}[1em]{2em}
\setstretch{.5}
{\PaliGlossB{Any ascetic or brahmin who comes to stay in our village district is our guest,}}\\
\end{addmargin}
\end{absolutelynopagebreak}

\begin{absolutelynopagebreak}
\setstretch{.7}
{\PaliGlossA{Atithī kho panamhehi sakkātabbā garukātabbā mānetabbā pūjetabbā.}}\\
\begin{addmargin}[1em]{2em}
\setstretch{.5}
{\PaliGlossB{and should be honored and respected as such.}}\\
\end{addmargin}
\end{absolutelynopagebreak}

\begin{absolutelynopagebreak}
\setstretch{.7}
{\PaliGlossA{Yampi samaṇo gotamo opāsādaṃ anuppatto opāsāde viharati uttarena opāsādaṃ devavane sālavane, atithimhākaṃ samaṇo gotamo.}}\\
\begin{addmargin}[1em]{2em}
\setstretch{.5}
{\PaliGlossB{    -}}\\
\end{addmargin}
\end{absolutelynopagebreak}

\begin{absolutelynopagebreak}
\setstretch{.7}
{\PaliGlossA{Atithi kho panamhehi sakkātabbo garukātabbo mānetabbo pūjetabbo.}}\\
\begin{addmargin}[1em]{2em}
\setstretch{.5}
{\PaliGlossB{    -}}\\
\end{addmargin}
\end{absolutelynopagebreak}

\begin{absolutelynopagebreak}
\setstretch{.7}
{\PaliGlossA{Imināpaṅgena na arahati so bhavaṃ gotamo amhākaṃ dassanāya upasaṅkamituṃ;}}\\
\begin{addmargin}[1em]{2em}
\setstretch{.5}
{\PaliGlossB{For this reason, too, it’s not appropriate for Master Gotama to come to see me,}}\\
\end{addmargin}
\end{absolutelynopagebreak}

\begin{absolutelynopagebreak}
\setstretch{.7}
{\PaliGlossA{atha kho mayameva arahāma taṃ bhavantaṃ gotamaṃ dassanāya upasaṅkamituṃ.}}\\
\begin{addmargin}[1em]{2em}
\setstretch{.5}
{\PaliGlossB{rather, it’s appropriate for me to go to see him.}}\\
\end{addmargin}
\end{absolutelynopagebreak}

\begin{absolutelynopagebreak}
\setstretch{.7}
{\PaliGlossA{Ettake kho ahaṃ, bho, tassa bhoto gotamassa vaṇṇe pariyāpuṇāmi, no ca kho so bhavaṃ gotamo ettakavaṇṇo;}}\\
\begin{addmargin}[1em]{2em}
\setstretch{.5}
{\PaliGlossB{This is the extent of Master Gotama’s praise that I have learned. But his praises are not confined to this,}}\\
\end{addmargin}
\end{absolutelynopagebreak}

\begin{absolutelynopagebreak}
\setstretch{.7}
{\PaliGlossA{aparimāṇavaṇṇo hi so bhavaṃ gotamo.}}\\
\begin{addmargin}[1em]{2em}
\setstretch{.5}
{\PaliGlossB{for the praise of Master Gotama is limitless.}}\\
\end{addmargin}
\end{absolutelynopagebreak}

\begin{absolutelynopagebreak}
\setstretch{.7}
{\PaliGlossA{Ekamekenapi tena aṅgena samannāgato na arahati, so bhavaṃ gotamo amhākaṃ dassanāya upasaṅkamituṃ;}}\\
\begin{addmargin}[1em]{2em}
\setstretch{.5}
{\PaliGlossB{The possession of even a single one of these factors makes it inappropriate for Master Gotama to come to see me,}}\\
\end{addmargin}
\end{absolutelynopagebreak}

\begin{absolutelynopagebreak}
\setstretch{.7}
{\PaliGlossA{atha kho mayameva arahāma taṃ bhavantaṃ gotamaṃ dassanāya upasaṅkamitunti.}}\\
\begin{addmargin}[1em]{2em}
\setstretch{.5}
{\PaliGlossB{rather, it’s appropriate for me to go to see him.}}\\
\end{addmargin}
\end{absolutelynopagebreak}

\begin{absolutelynopagebreak}
\setstretch{.7}
{\PaliGlossA{Tena hi, bho, sabbeva mayaṃ samaṇaṃ gotamaṃ dassanāya upasaṅkamissāmā”ti.}}\\
\begin{addmargin}[1em]{2em}
\setstretch{.5}
{\PaliGlossB{Well then, gentlemen, let’s all go to see the ascetic Gotama.”}}\\
\end{addmargin}
\end{absolutelynopagebreak}

\vskip 0.05in
\begin{absolutelynopagebreak}
\setstretch{.7}
{\PaliGlossA{Atha kho caṅkī brāhmaṇo mahatā brāhmaṇagaṇena saddhiṃ yena bhagavā tenupasaṅkami; upasaṅkamitvā bhagavatā saddhiṃ sammodi.}}\\
\begin{addmargin}[1em]{2em}
\setstretch{.5}
{\PaliGlossB{Then Caṅkī together with a large group of brahmins went to the Buddha and exchanged greetings with him.}}\\
\end{addmargin}
\end{absolutelynopagebreak}

\begin{absolutelynopagebreak}
\setstretch{.7}
{\PaliGlossA{Sammodanīyaṃ kathaṃ sāraṇīyaṃ vītisāretvā ekamantaṃ nisīdi.}}\\
\begin{addmargin}[1em]{2em}
\setstretch{.5}
{\PaliGlossB{When the greetings and polite conversation were over, he sat down to one side.}}\\
\end{addmargin}
\end{absolutelynopagebreak}

\vskip 0.05in
\begin{absolutelynopagebreak}
\setstretch{.7}
{\PaliGlossA{Tena kho pana samayena bhagavā vuddhehi vuddhehi brāhmaṇehi saddhiṃ kiñci kiñci kathaṃ sāraṇīyaṃ vītisāretvā nisinno hoti.}}\\
\begin{addmargin}[1em]{2em}
\setstretch{.5}
{\PaliGlossB{Now at that time the Buddha was sitting engaged in some polite conversation together with some very senior brahmins.}}\\
\end{addmargin}
\end{absolutelynopagebreak}

\begin{absolutelynopagebreak}
\setstretch{.7}
{\PaliGlossA{Tena kho pana samayena kāpaṭiko nāma māṇavo daharo vuttasiro soḷasavassuddesiko jātiyā, tiṇṇaṃ vedānaṃ pāragū sanighaṇḍukeṭubhānaṃ sākkharappabhedānaṃ itihāsapañcamānaṃ, padako, veyyākaraṇo, lokāyatamahāpurisalakkhaṇesu anavayo tassaṃ parisāyaṃ nisinno hoti.}}\\
\begin{addmargin}[1em]{2em}
\setstretch{.5}
{\PaliGlossB{And the brahmin student Kāpaṭika was sitting in that assembly. He was young, newly tonsured; he was sixteen years old. He had mastered the three Vedas, together with their vocabularies, ritual, phonology and etymology, and the testament as fifth. He knew philology and grammar, and was well versed in cosmology and the marks of a great man.}}\\
\end{addmargin}
\end{absolutelynopagebreak}

\begin{absolutelynopagebreak}
\setstretch{.7}
{\PaliGlossA{So vuddhānaṃ vuddhānaṃ brāhmaṇānaṃ bhagavatā saddhiṃ mantayamānānaṃ antarantarā kathaṃ opāteti.}}\\
\begin{addmargin}[1em]{2em}
\setstretch{.5}
{\PaliGlossB{While the senior brahmins were conversing together with the Buddha, he interrupted.}}\\
\end{addmargin}
\end{absolutelynopagebreak}

\begin{absolutelynopagebreak}
\setstretch{.7}
{\PaliGlossA{Atha kho bhagavā kāpaṭikaṃ māṇavaṃ apasādeti:}}\\
\begin{addmargin}[1em]{2em}
\setstretch{.5}
{\PaliGlossB{Then the Buddha rebuked Kāpaṭika,}}\\
\end{addmargin}
\end{absolutelynopagebreak}

\begin{absolutelynopagebreak}
\setstretch{.7}
{\PaliGlossA{“māyasmā bhāradvājo vuddhānaṃ vuddhānaṃ brāhmaṇānaṃ mantayamānānaṃ antarantarā kathaṃ opātetu.}}\\
\begin{addmargin}[1em]{2em}
\setstretch{.5}
{\PaliGlossB{“Venerable Bhāradvāja, don’t interrupt the senior brahmins.}}\\
\end{addmargin}
\end{absolutelynopagebreak}

\begin{absolutelynopagebreak}
\setstretch{.7}
{\PaliGlossA{Kathāpariyosānaṃ āyasmā bhāradvājo āgametū”ti.}}\\
\begin{addmargin}[1em]{2em}
\setstretch{.5}
{\PaliGlossB{Wait until they’ve finished speaking.”}}\\
\end{addmargin}
\end{absolutelynopagebreak}

\begin{absolutelynopagebreak}
\setstretch{.7}
{\PaliGlossA{Evaṃ vutte, caṅkī brāhmaṇo bhagavantaṃ etadavoca:}}\\
\begin{addmargin}[1em]{2em}
\setstretch{.5}
{\PaliGlossB{When he had spoken, Caṅkī said to the Buddha,}}\\
\end{addmargin}
\end{absolutelynopagebreak}

\begin{absolutelynopagebreak}
\setstretch{.7}
{\PaliGlossA{“mā bhavaṃ gotamo kāpaṭikaṃ māṇavaṃ apasādesi.}}\\
\begin{addmargin}[1em]{2em}
\setstretch{.5}
{\PaliGlossB{“Master Gotama, don’t rebuke the student Kāpaṭika.}}\\
\end{addmargin}
\end{absolutelynopagebreak}

\begin{absolutelynopagebreak}
\setstretch{.7}
{\PaliGlossA{kulaputto ca kāpaṭiko māṇavo, bahussuto ca kāpaṭiko māṇavo, paṇḍito ca kāpaṭiko māṇavo, kalyāṇavākkaraṇo ca kāpaṭiko māṇavo, pahoti ca kāpaṭiko māṇavo bhotā gotamena saddhiṃ asmiṃ vacane paṭimantetun”ti.}}\\
\begin{addmargin}[1em]{2em}
\setstretch{.5}
{\PaliGlossB{He’s a gentleman, learned, astute, a good speaker. He’s capable of having a dialogue with Master Gotama about this.”}}\\
\end{addmargin}
\end{absolutelynopagebreak}

\vskip 0.05in
\begin{absolutelynopagebreak}
\setstretch{.7}
{\PaliGlossA{Atha kho bhagavato etadahosi:}}\\
\begin{addmargin}[1em]{2em}
\setstretch{.5}
{\PaliGlossB{Then it occurred to the Buddha,}}\\
\end{addmargin}
\end{absolutelynopagebreak}

\begin{absolutelynopagebreak}
\setstretch{.7}
{\PaliGlossA{“addhā kho kāpaṭikassa māṇavassa tevijjake pāvacane kathā bhavissati.}}\\
\begin{addmargin}[1em]{2em}
\setstretch{.5}
{\PaliGlossB{“Clearly the student Kāpaṭika will talk about the scriptural heritage of the three Vedas.}}\\
\end{addmargin}
\end{absolutelynopagebreak}

\begin{absolutelynopagebreak}
\setstretch{.7}
{\PaliGlossA{Tathā hi naṃ brāhmaṇā sampurekkharontī”ti.}}\\
\begin{addmargin}[1em]{2em}
\setstretch{.5}
{\PaliGlossB{That’s why they put him at the front.”}}\\
\end{addmargin}
\end{absolutelynopagebreak}

\begin{absolutelynopagebreak}
\setstretch{.7}
{\PaliGlossA{Atha kho kāpaṭikassa māṇavassa etadahosi:}}\\
\begin{addmargin}[1em]{2em}
\setstretch{.5}
{\PaliGlossB{Then Kāpaṭika thought,}}\\
\end{addmargin}
\end{absolutelynopagebreak}

\begin{absolutelynopagebreak}
\setstretch{.7}
{\PaliGlossA{“yadā me samaṇo gotamo cakkhuṃ upasaṃharissati, athāhaṃ samaṇaṃ gotamaṃ pañhaṃ pucchissāmī”ti.}}\\
\begin{addmargin}[1em]{2em}
\setstretch{.5}
{\PaliGlossB{“When the ascetic Gotama looks at me, I’ll ask him a question.”}}\\
\end{addmargin}
\end{absolutelynopagebreak}

\begin{absolutelynopagebreak}
\setstretch{.7}
{\PaliGlossA{Atha kho bhagavā kāpaṭikassa māṇavassa cetasā cetoparivitakkamaññāya yena kāpaṭiko māṇavo tena cakkhūni upasaṃhāsi.}}\\
\begin{addmargin}[1em]{2em}
\setstretch{.5}
{\PaliGlossB{Then the Buddha, knowing what Kāpaṭika was thinking, looked at him.}}\\
\end{addmargin}
\end{absolutelynopagebreak}

\begin{absolutelynopagebreak}
\setstretch{.7}
{\PaliGlossA{Atha kho kāpaṭikassa māṇavassa etadahosi:}}\\
\begin{addmargin}[1em]{2em}
\setstretch{.5}
{\PaliGlossB{Then Kāpaṭika thought,}}\\
\end{addmargin}
\end{absolutelynopagebreak}

\begin{absolutelynopagebreak}
\setstretch{.7}
{\PaliGlossA{“samannāharati kho maṃ samaṇo gotamo.}}\\
\begin{addmargin}[1em]{2em}
\setstretch{.5}
{\PaliGlossB{“The ascetic Gotama is engaging with me.}}\\
\end{addmargin}
\end{absolutelynopagebreak}

\begin{absolutelynopagebreak}
\setstretch{.7}
{\PaliGlossA{Yannūnāhaṃ samaṇaṃ gotamaṃ pañhaṃ puccheyyan”ti.}}\\
\begin{addmargin}[1em]{2em}
\setstretch{.5}
{\PaliGlossB{Why don’t I ask him a question?”}}\\
\end{addmargin}
\end{absolutelynopagebreak}

\begin{absolutelynopagebreak}
\setstretch{.7}
{\PaliGlossA{Atha kho kāpaṭiko māṇavo bhagavantaṃ etadavoca:}}\\
\begin{addmargin}[1em]{2em}
\setstretch{.5}
{\PaliGlossB{Then he said,}}\\
\end{addmargin}
\end{absolutelynopagebreak}

\begin{absolutelynopagebreak}
\setstretch{.7}
{\PaliGlossA{“yadidaṃ, bho gotama, brāhmaṇānaṃ porāṇaṃ mantapadaṃ itihitihaparamparāya piṭakasampadāya, tattha ca brāhmaṇā ekaṃsena niṭṭhaṃ gacchanti:}}\\
\begin{addmargin}[1em]{2em}
\setstretch{.5}
{\PaliGlossB{“Master Gotama, regarding that which by the lineage of testament and by canonical authority is the traditional hymnal of the brahmins, the brahmins come to the definite conclusion:}}\\
\end{addmargin}
\end{absolutelynopagebreak}

\begin{absolutelynopagebreak}
\setstretch{.7}
{\PaliGlossA{‘idameva saccaṃ, moghamaññan’ti.}}\\
\begin{addmargin}[1em]{2em}
\setstretch{.5}
{\PaliGlossB{‘This is the only truth, other ideas are silly.’}}\\
\end{addmargin}
\end{absolutelynopagebreak}

\begin{absolutelynopagebreak}
\setstretch{.7}
{\PaliGlossA{Idha bhavaṃ gotamo kimāhā”ti?}}\\
\begin{addmargin}[1em]{2em}
\setstretch{.5}
{\PaliGlossB{What do you say about this?”}}\\
\end{addmargin}
\end{absolutelynopagebreak}

\vskip 0.05in
\begin{absolutelynopagebreak}
\setstretch{.7}
{\PaliGlossA{“Kiṃ pana, bhāradvāja, atthi koci brāhmaṇānaṃ ekabrāhmaṇopi yo evamāha:}}\\
\begin{addmargin}[1em]{2em}
\setstretch{.5}
{\PaliGlossB{“Well, Bhāradvāja, is there even a single one of the brahmins who says this:}}\\
\end{addmargin}
\end{absolutelynopagebreak}

\begin{absolutelynopagebreak}
\setstretch{.7}
{\PaliGlossA{‘ahametaṃ jānāmi, ahametaṃ passāmi.}}\\
\begin{addmargin}[1em]{2em}
\setstretch{.5}
{\PaliGlossB{‘I know this, I see this:}}\\
\end{addmargin}
\end{absolutelynopagebreak}

\begin{absolutelynopagebreak}
\setstretch{.7}
{\PaliGlossA{Idameva saccaṃ, moghamaññan’”ti?}}\\
\begin{addmargin}[1em]{2em}
\setstretch{.5}
{\PaliGlossB{this is the only truth, other ideas are silly’?”}}\\
\end{addmargin}
\end{absolutelynopagebreak}

\begin{absolutelynopagebreak}
\setstretch{.7}
{\PaliGlossA{“No hidaṃ, bho gotama”.}}\\
\begin{addmargin}[1em]{2em}
\setstretch{.5}
{\PaliGlossB{“No, Master Gotama.”}}\\
\end{addmargin}
\end{absolutelynopagebreak}

\begin{absolutelynopagebreak}
\setstretch{.7}
{\PaliGlossA{“Kiṃ pana, bhāradvāja, atthi koci brāhmaṇānaṃ ekācariyopi, ekācariyapācariyopi, yāva sattamā ācariyamahayugāpi, yo evamāha:}}\\
\begin{addmargin}[1em]{2em}
\setstretch{.5}
{\PaliGlossB{“Well, is there even a single teacher of the brahmins, or a teacher’s teacher, or anyone back to the seventh generation of teachers, who says this:}}\\
\end{addmargin}
\end{absolutelynopagebreak}

\begin{absolutelynopagebreak}
\setstretch{.7}
{\PaliGlossA{‘ahametaṃ jānāmi, ahametaṃ passāmi.}}\\
\begin{addmargin}[1em]{2em}
\setstretch{.5}
{\PaliGlossB{‘I know this, I see this:}}\\
\end{addmargin}
\end{absolutelynopagebreak}

\begin{absolutelynopagebreak}
\setstretch{.7}
{\PaliGlossA{Idameva saccaṃ, moghamaññan’”ti?}}\\
\begin{addmargin}[1em]{2em}
\setstretch{.5}
{\PaliGlossB{this is the only truth, other ideas are silly’?”}}\\
\end{addmargin}
\end{absolutelynopagebreak}

\begin{absolutelynopagebreak}
\setstretch{.7}
{\PaliGlossA{“No hidaṃ, bho gotama”.}}\\
\begin{addmargin}[1em]{2em}
\setstretch{.5}
{\PaliGlossB{“No, Master Gotama.”}}\\
\end{addmargin}
\end{absolutelynopagebreak}

\begin{absolutelynopagebreak}
\setstretch{.7}
{\PaliGlossA{“Kiṃ pana, bhāradvāja, yepi te brāhmaṇānaṃ pubbakā isayo mantānaṃ kattāro mantānaṃ pavattāro yesamidaṃ etarahi brāhmaṇā porāṇaṃ mantapadaṃ gītaṃ pavuttaṃ samihitaṃ tadanugāyanti tadanubhāsanti bhāsitamanubhāsanti vācitamanuvācenti seyyathidaṃ—aṭṭhako vāmako vāmadevo vessāmitto yamataggi aṅgīraso bhāradvājo vāseṭṭho kassapo bhagu,}}\\
\begin{addmargin}[1em]{2em}
\setstretch{.5}
{\PaliGlossB{“Well, what of the ancient hermits of the brahmins, namely Aṭṭhaka, Vāmaka, Vāmadeva, Vessāmitta, Yamadaggi, Aṅgīrasa, Bhāradvāja, Vāseṭṭha, Kassapa, and Bhagu? They were the authors and propagators of the hymns. Their hymnal was sung and propagated and compiled in ancient times; and these days, brahmins continue to sing and chant it, chanting what was chanted and teaching what was taught.}}\\
\end{addmargin}
\end{absolutelynopagebreak}

\begin{absolutelynopagebreak}
\setstretch{.7}
{\PaliGlossA{tepi evamāhaṃsu:}}\\
\begin{addmargin}[1em]{2em}
\setstretch{.5}
{\PaliGlossB{Did even they say:}}\\
\end{addmargin}
\end{absolutelynopagebreak}

\begin{absolutelynopagebreak}
\setstretch{.7}
{\PaliGlossA{‘mayametaṃ jānāma, mayametaṃ passāma.}}\\
\begin{addmargin}[1em]{2em}
\setstretch{.5}
{\PaliGlossB{‘We know this, we see this:}}\\
\end{addmargin}
\end{absolutelynopagebreak}

\begin{absolutelynopagebreak}
\setstretch{.7}
{\PaliGlossA{Idameva saccaṃ, moghamaññan’”ti?}}\\
\begin{addmargin}[1em]{2em}
\setstretch{.5}
{\PaliGlossB{this is the only truth, other ideas are silly’?”}}\\
\end{addmargin}
\end{absolutelynopagebreak}

\begin{absolutelynopagebreak}
\setstretch{.7}
{\PaliGlossA{“No hidaṃ, bho gotama”.}}\\
\begin{addmargin}[1em]{2em}
\setstretch{.5}
{\PaliGlossB{“No, Master Gotama.”}}\\
\end{addmargin}
\end{absolutelynopagebreak}

\begin{absolutelynopagebreak}
\setstretch{.7}
{\PaliGlossA{“Iti kira, bhāradvāja, natthi koci brāhmaṇānaṃ ekabrāhmaṇopi yo evamāha:}}\\
\begin{addmargin}[1em]{2em}
\setstretch{.5}
{\PaliGlossB{“So, Bhāradvāja, it seems that there is not a single one of the brahmins,}}\\
\end{addmargin}
\end{absolutelynopagebreak}

\begin{absolutelynopagebreak}
\setstretch{.7}
{\PaliGlossA{‘ahametaṃ jānāmi, ahametaṃ passāmi.}}\\
\begin{addmargin}[1em]{2em}
\setstretch{.5}
{\PaliGlossB{    -}}\\
\end{addmargin}
\end{absolutelynopagebreak}

\begin{absolutelynopagebreak}
\setstretch{.7}
{\PaliGlossA{Idameva saccaṃ, moghamaññan’ti;}}\\
\begin{addmargin}[1em]{2em}
\setstretch{.5}
{\PaliGlossB{    -}}\\
\end{addmargin}
\end{absolutelynopagebreak}

\begin{absolutelynopagebreak}
\setstretch{.7}
{\PaliGlossA{natthi koci brāhmaṇānaṃ ekācariyopi ekācariyapācariyopi, yāva sattamā ācariyamahayugāpi, yo evamāha:}}\\
\begin{addmargin}[1em]{2em}
\setstretch{.5}
{\PaliGlossB{not even anyone back to the seventh generation of teachers,}}\\
\end{addmargin}
\end{absolutelynopagebreak}

\begin{absolutelynopagebreak}
\setstretch{.7}
{\PaliGlossA{‘ahametaṃ jānāmi, ahametaṃ passāmi.}}\\
\begin{addmargin}[1em]{2em}
\setstretch{.5}
{\PaliGlossB{    -}}\\
\end{addmargin}
\end{absolutelynopagebreak}

\begin{absolutelynopagebreak}
\setstretch{.7}
{\PaliGlossA{Idameva saccaṃ, moghamaññan’ti;}}\\
\begin{addmargin}[1em]{2em}
\setstretch{.5}
{\PaliGlossB{    -}}\\
\end{addmargin}
\end{absolutelynopagebreak}

\begin{absolutelynopagebreak}
\setstretch{.7}
{\PaliGlossA{yepi te brāhmaṇānaṃ pubbakā isayo mantānaṃ kattāro mantānaṃ pavattāro yesamidaṃ etarahi brāhmaṇā porāṇaṃ mantapadaṃ gītaṃ pavuttaṃ samihitaṃ tadanugāyanti tadanubhāsanti bhāsitamanubhāsanti vācitamanuvācenti seyyathidaṃ—aṭṭhako vāmako vāmadevo vessāmitto yamataggi aṅgīraso bhāradvājo vāseṭṭho kassapo bhagu, tepi na evamāhaṃsu:}}\\
\begin{addmargin}[1em]{2em}
\setstretch{.5}
{\PaliGlossB{nor even the ancient hermits of the brahmins who say:}}\\
\end{addmargin}
\end{absolutelynopagebreak}

\begin{absolutelynopagebreak}
\setstretch{.7}
{\PaliGlossA{‘mayametaṃ jānāma, mayametaṃ passāma.}}\\
\begin{addmargin}[1em]{2em}
\setstretch{.5}
{\PaliGlossB{‘We know this, we see this:}}\\
\end{addmargin}
\end{absolutelynopagebreak}

\begin{absolutelynopagebreak}
\setstretch{.7}
{\PaliGlossA{Idameva saccaṃ, moghamaññan’ti.}}\\
\begin{addmargin}[1em]{2em}
\setstretch{.5}
{\PaliGlossB{this is the only truth, other ideas are silly.’}}\\
\end{addmargin}
\end{absolutelynopagebreak}

\begin{absolutelynopagebreak}
\setstretch{.7}
{\PaliGlossA{Seyyathāpi, bhāradvāja, andhaveṇi paramparāsaṃsattā purimopi na passati majjhimopi na passati pacchimopi na passati;}}\\
\begin{addmargin}[1em]{2em}
\setstretch{.5}
{\PaliGlossB{Suppose there was a queue of blind men, each holding the one in front: the first one does not see, the middle one does not see, and the last one does not see.}}\\
\end{addmargin}
\end{absolutelynopagebreak}

\begin{absolutelynopagebreak}
\setstretch{.7}
{\PaliGlossA{evameva kho, bhāradvāja, andhaveṇūpamaṃ maññe brāhmaṇānaṃ bhāsitaṃ sampajjati—purimopi na passati majjhimopi na passati pacchimopi na passati.}}\\
\begin{addmargin}[1em]{2em}
\setstretch{.5}
{\PaliGlossB{In the same way, it seems to me that the brahmins’ statement turns out to be like a queue of blind men: the first one does not see, the middle one does not see, and the last one does not see.}}\\
\end{addmargin}
\end{absolutelynopagebreak}

\begin{absolutelynopagebreak}
\setstretch{.7}
{\PaliGlossA{Taṃ kiṃ maññasi, bhāradvāja,}}\\
\begin{addmargin}[1em]{2em}
\setstretch{.5}
{\PaliGlossB{What do you think, Bhāradvāja?}}\\
\end{addmargin}
\end{absolutelynopagebreak}

\begin{absolutelynopagebreak}
\setstretch{.7}
{\PaliGlossA{nanu evaṃ sante brāhmaṇānaṃ amūlikā saddhā sampajjatī”ti?}}\\
\begin{addmargin}[1em]{2em}
\setstretch{.5}
{\PaliGlossB{This being so, doesn’t the brahmins’ faith turn out to be baseless?”}}\\
\end{addmargin}
\end{absolutelynopagebreak}

\vskip 0.05in
\begin{absolutelynopagebreak}
\setstretch{.7}
{\PaliGlossA{“Na khvettha, bho gotama, brāhmaṇā saddhāyeva payirupāsanti, anussavāpettha brāhmaṇā payirupāsantī”ti.}}\\
\begin{addmargin}[1em]{2em}
\setstretch{.5}
{\PaliGlossB{“The brahmins don’t just honor this because of faith, but also because of oral transmission.”}}\\
\end{addmargin}
\end{absolutelynopagebreak}

\begin{absolutelynopagebreak}
\setstretch{.7}
{\PaliGlossA{“Pubbeva kho tvaṃ, bhāradvāja, saddhaṃ agamāsi, anussavaṃ idāni vadesi.}}\\
\begin{addmargin}[1em]{2em}
\setstretch{.5}
{\PaliGlossB{“First you relied on faith, now you speak of oral tradition.}}\\
\end{addmargin}
\end{absolutelynopagebreak}

\begin{absolutelynopagebreak}
\setstretch{.7}
{\PaliGlossA{Pañca kho ime, bhāradvāja, dhammā diṭṭheva dhamme dvedhā vipākā.}}\\
\begin{addmargin}[1em]{2em}
\setstretch{.5}
{\PaliGlossB{These five things can be seen to turn out in two different ways.}}\\
\end{addmargin}
\end{absolutelynopagebreak}

\begin{absolutelynopagebreak}
\setstretch{.7}
{\PaliGlossA{Katame pañca?}}\\
\begin{addmargin}[1em]{2em}
\setstretch{.5}
{\PaliGlossB{What five?}}\\
\end{addmargin}
\end{absolutelynopagebreak}

\begin{absolutelynopagebreak}
\setstretch{.7}
{\PaliGlossA{Saddhā, ruci, anussavo, ākāraparivitakko, diṭṭhinijjhānakkhanti—}}\\
\begin{addmargin}[1em]{2em}
\setstretch{.5}
{\PaliGlossB{Faith, preference, oral tradition, reasoned contemplation, and acceptance of a view after consideration.}}\\
\end{addmargin}
\end{absolutelynopagebreak}

\begin{absolutelynopagebreak}
\setstretch{.7}
{\PaliGlossA{ime kho, bhāradvāja, pañca dhammā diṭṭheva dhamme dvedhā vipākā.}}\\
\begin{addmargin}[1em]{2em}
\setstretch{.5}
{\PaliGlossB{    -}}\\
\end{addmargin}
\end{absolutelynopagebreak}

\begin{absolutelynopagebreak}
\setstretch{.7}
{\PaliGlossA{Api ca, bhāradvāja, susaddahitaṃyeva hoti, tañca hoti rittaṃ tucchaṃ musā;}}\\
\begin{addmargin}[1em]{2em}
\setstretch{.5}
{\PaliGlossB{Even though you have full faith in something, it may be void, hollow, and false.}}\\
\end{addmargin}
\end{absolutelynopagebreak}

\begin{absolutelynopagebreak}
\setstretch{.7}
{\PaliGlossA{no cepi susaddahitaṃ hoti, tañca hoti bhūtaṃ tacchaṃ anaññathā.}}\\
\begin{addmargin}[1em]{2em}
\setstretch{.5}
{\PaliGlossB{And even if you don’t have full faith in something, it may be true and real, not otherwise.}}\\
\end{addmargin}
\end{absolutelynopagebreak}

\begin{absolutelynopagebreak}
\setstretch{.7}
{\PaliGlossA{Api ca, bhāradvāja, surucitaṃyeva hoti … pe …}}\\
\begin{addmargin}[1em]{2em}
\setstretch{.5}
{\PaliGlossB{Even though you have a strong preference for something …}}\\
\end{addmargin}
\end{absolutelynopagebreak}

\begin{absolutelynopagebreak}
\setstretch{.7}
{\PaliGlossA{svānussutaṃyeva hoti … pe …}}\\
\begin{addmargin}[1em]{2em}
\setstretch{.5}
{\PaliGlossB{something may be accurately transmitted …}}\\
\end{addmargin}
\end{absolutelynopagebreak}

\begin{absolutelynopagebreak}
\setstretch{.7}
{\PaliGlossA{suparivitakkitaṃyeva hoti … pe …}}\\
\begin{addmargin}[1em]{2em}
\setstretch{.5}
{\PaliGlossB{something may be well contemplated …}}\\
\end{addmargin}
\end{absolutelynopagebreak}

\begin{absolutelynopagebreak}
\setstretch{.7}
{\PaliGlossA{sunijjhāyitaṃyeva hoti, tañca hoti rittaṃ tucchaṃ musā;}}\\
\begin{addmargin}[1em]{2em}
\setstretch{.5}
{\PaliGlossB{something may be well considered, it may be void, hollow, and false.}}\\
\end{addmargin}
\end{absolutelynopagebreak}

\begin{absolutelynopagebreak}
\setstretch{.7}
{\PaliGlossA{no cepi sunijjhāyitaṃ hoti, tañca hoti bhūtaṃ tacchaṃ anaññathā.}}\\
\begin{addmargin}[1em]{2em}
\setstretch{.5}
{\PaliGlossB{And even if something is not well considered, it may be true and real, not otherwise.}}\\
\end{addmargin}
\end{absolutelynopagebreak}

\begin{absolutelynopagebreak}
\setstretch{.7}
{\PaliGlossA{Saccamanurakkhatā, bhāradvāja, viññunā purisena nālamettha ekaṃsena niṭṭhaṃ gantuṃ:}}\\
\begin{addmargin}[1em]{2em}
\setstretch{.5}
{\PaliGlossB{For a sensible person who is preserving truth this is not sufficient to come to the definite conclusion:}}\\
\end{addmargin}
\end{absolutelynopagebreak}

\begin{absolutelynopagebreak}
\setstretch{.7}
{\PaliGlossA{‘idameva saccaṃ, moghamaññan’”ti.}}\\
\begin{addmargin}[1em]{2em}
\setstretch{.5}
{\PaliGlossB{‘This is the only truth, other ideas are silly.’”}}\\
\end{addmargin}
\end{absolutelynopagebreak}

\vskip 0.05in
\begin{absolutelynopagebreak}
\setstretch{.7}
{\PaliGlossA{“Kittāvatā pana, bho gotama, saccānurakkhaṇā hoti, kittāvatā saccamanurakkhati?}}\\
\begin{addmargin}[1em]{2em}
\setstretch{.5}
{\PaliGlossB{“But Master Gotama, how do you define the preservation of truth?”}}\\
\end{addmargin}
\end{absolutelynopagebreak}

\begin{absolutelynopagebreak}
\setstretch{.7}
{\PaliGlossA{Saccānurakkhaṇaṃ mayaṃ bhavantaṃ gotamaṃ pucchāmā”ti.}}\\
\begin{addmargin}[1em]{2em}
\setstretch{.5}
{\PaliGlossB{    -}}\\
\end{addmargin}
\end{absolutelynopagebreak}

\begin{absolutelynopagebreak}
\setstretch{.7}
{\PaliGlossA{“Saddhā cepi, bhāradvāja, purisassa hoti;}}\\
\begin{addmargin}[1em]{2em}
\setstretch{.5}
{\PaliGlossB{“If a person has faith,}}\\
\end{addmargin}
\end{absolutelynopagebreak}

\begin{absolutelynopagebreak}
\setstretch{.7}
{\PaliGlossA{‘evaṃ me saddhā’ti—}}\\
\begin{addmargin}[1em]{2em}
\setstretch{.5}
{\PaliGlossB{they preserve truth by saying, ‘Such is my faith.’}}\\
\end{addmargin}
\end{absolutelynopagebreak}

\begin{absolutelynopagebreak}
\setstretch{.7}
{\PaliGlossA{iti vadaṃ saccamanurakkhati, na tveva tāva ekaṃsena niṭṭhaṃ gacchati:}}\\
\begin{addmargin}[1em]{2em}
\setstretch{.5}
{\PaliGlossB{But they don’t yet come to the definite conclusion:}}\\
\end{addmargin}
\end{absolutelynopagebreak}

\begin{absolutelynopagebreak}
\setstretch{.7}
{\PaliGlossA{‘idameva saccaṃ, moghamaññan’ti ().}}\\
\begin{addmargin}[1em]{2em}
\setstretch{.5}
{\PaliGlossB{‘This is the only truth, other ideas are silly.’}}\\
\end{addmargin}
\end{absolutelynopagebreak}

\begin{absolutelynopagebreak}
\setstretch{.7}
{\PaliGlossA{Ruci cepi, bhāradvāja, purisassa hoti … pe …}}\\
\begin{addmargin}[1em]{2em}
\setstretch{.5}
{\PaliGlossB{If a person has a preference …}}\\
\end{addmargin}
\end{absolutelynopagebreak}

\begin{absolutelynopagebreak}
\setstretch{.7}
{\PaliGlossA{anussavo cepi, bhāradvāja, purisassa hoti … pe …}}\\
\begin{addmargin}[1em]{2em}
\setstretch{.5}
{\PaliGlossB{or has received an oral transmission …}}\\
\end{addmargin}
\end{absolutelynopagebreak}

\begin{absolutelynopagebreak}
\setstretch{.7}
{\PaliGlossA{ākāraparivitakko cepi, bhāradvāja, purisassa hoti … pe …}}\\
\begin{addmargin}[1em]{2em}
\setstretch{.5}
{\PaliGlossB{or has a reasoned reflection about something …}}\\
\end{addmargin}
\end{absolutelynopagebreak}

\begin{absolutelynopagebreak}
\setstretch{.7}
{\PaliGlossA{diṭṭhinijjhānakkhanti cepi, bhāradvāja, purisassa hoti;}}\\
\begin{addmargin}[1em]{2em}
\setstretch{.5}
{\PaliGlossB{or has accepted a view after contemplation,}}\\
\end{addmargin}
\end{absolutelynopagebreak}

\begin{absolutelynopagebreak}
\setstretch{.7}
{\PaliGlossA{‘evaṃ me diṭṭhinijjhānakkhantī’ti—}}\\
\begin{addmargin}[1em]{2em}
\setstretch{.5}
{\PaliGlossB{they preserve truth by saying, ‘Such is the view I have accepted after contemplation.’}}\\
\end{addmargin}
\end{absolutelynopagebreak}

\begin{absolutelynopagebreak}
\setstretch{.7}
{\PaliGlossA{iti vadaṃ saccamanurakkhati, na tveva tāva ekaṃsena niṭṭhaṃ gacchati:}}\\
\begin{addmargin}[1em]{2em}
\setstretch{.5}
{\PaliGlossB{But they don’t yet come to the definite conclusion:}}\\
\end{addmargin}
\end{absolutelynopagebreak}

\begin{absolutelynopagebreak}
\setstretch{.7}
{\PaliGlossA{‘idameva saccaṃ, moghamaññan’ti.}}\\
\begin{addmargin}[1em]{2em}
\setstretch{.5}
{\PaliGlossB{‘This is the only truth, other ideas are silly.’}}\\
\end{addmargin}
\end{absolutelynopagebreak}

\begin{absolutelynopagebreak}
\setstretch{.7}
{\PaliGlossA{Ettāvatā kho, bhāradvāja, saccānurakkhaṇā hoti, ettāvatā saccamanurakkhati, ettāvatā ca mayaṃ saccānurakkhaṇaṃ paññapema;}}\\
\begin{addmargin}[1em]{2em}
\setstretch{.5}
{\PaliGlossB{That’s how the preservation of truth is defined, Bhāradvāja. I describe the preservation of truth as defined in this way.}}\\
\end{addmargin}
\end{absolutelynopagebreak}

\begin{absolutelynopagebreak}
\setstretch{.7}
{\PaliGlossA{na tveva tāva saccānubodho hotī”ti.}}\\
\begin{addmargin}[1em]{2em}
\setstretch{.5}
{\PaliGlossB{But this is not yet the awakening to the truth.”}}\\
\end{addmargin}
\end{absolutelynopagebreak}

\vskip 0.05in
\begin{absolutelynopagebreak}
\setstretch{.7}
{\PaliGlossA{“Ettāvatā, bho gotama, saccānurakkhaṇā hoti, ettāvatā saccamanurakkhati, ettāvatā ca mayaṃ saccānurakkhaṇaṃ pekkhāma.}}\\
\begin{addmargin}[1em]{2em}
\setstretch{.5}
{\PaliGlossB{“That’s how the preservation of truth is defined, Master Gotama. We regard the preservation of truth as defined in this way.}}\\
\end{addmargin}
\end{absolutelynopagebreak}

\begin{absolutelynopagebreak}
\setstretch{.7}
{\PaliGlossA{Kittāvatā pana, bho gotama, saccānubodho hoti, kittāvatā saccamanubujjhati?}}\\
\begin{addmargin}[1em]{2em}
\setstretch{.5}
{\PaliGlossB{But Master Gotama, how do you define awakening to the truth?”}}\\
\end{addmargin}
\end{absolutelynopagebreak}

\begin{absolutelynopagebreak}
\setstretch{.7}
{\PaliGlossA{Saccānubodhaṃ mayaṃ bhavantaṃ gotamaṃ pucchāmā”ti.}}\\
\begin{addmargin}[1em]{2em}
\setstretch{.5}
{\PaliGlossB{    -}}\\
\end{addmargin}
\end{absolutelynopagebreak}

\vskip 0.05in
\begin{absolutelynopagebreak}
\setstretch{.7}
{\PaliGlossA{“Idha, bhāradvāja, bhikkhu aññataraṃ gāmaṃ vā nigamaṃ vā upanissāya viharati.}}\\
\begin{addmargin}[1em]{2em}
\setstretch{.5}
{\PaliGlossB{“Bhāradvāja, take the case of a mendicant living supported by a town or village.}}\\
\end{addmargin}
\end{absolutelynopagebreak}

\begin{absolutelynopagebreak}
\setstretch{.7}
{\PaliGlossA{Tamenaṃ gahapati vā gahapatiputto vā upasaṅkamitvā tīsu dhammesu samannesati—}}\\
\begin{addmargin}[1em]{2em}
\setstretch{.5}
{\PaliGlossB{A householder or their child approaches and scrutinizes them for three kinds of things:}}\\
\end{addmargin}
\end{absolutelynopagebreak}

\begin{absolutelynopagebreak}
\setstretch{.7}
{\PaliGlossA{lobhanīyesu dhammesu, dosanīyesu dhammesu, mohanīyesu dhammesu.}}\\
\begin{addmargin}[1em]{2em}
\setstretch{.5}
{\PaliGlossB{things that arouse greed, things that provoke hate, and things that promote delusion.}}\\
\end{addmargin}
\end{absolutelynopagebreak}

\begin{absolutelynopagebreak}
\setstretch{.7}
{\PaliGlossA{Atthi nu kho imassāyasmato tathārūpā lobhanīyā dhammā yathārūpehi lobhanīyehi dhammehi pariyādinnacitto ajānaṃ vā vadeyya—}}\\
\begin{addmargin}[1em]{2em}
\setstretch{.5}
{\PaliGlossB{‘Does this venerable have any qualities that arouse greed? Such qualities that, were their mind to be overwhelmed by them, they might say}}\\
\end{addmargin}
\end{absolutelynopagebreak}

\begin{absolutelynopagebreak}
\setstretch{.7}
{\PaliGlossA{jānāmīti, apassaṃ vā vadeyya—}}\\
\begin{addmargin}[1em]{2em}
\setstretch{.5}
{\PaliGlossB{that they know, even though they don’t know, or that they see, even though they don’t see;}}\\
\end{addmargin}
\end{absolutelynopagebreak}

\begin{absolutelynopagebreak}
\setstretch{.7}
{\PaliGlossA{passāmīti, paraṃ vā tadatthāya samādapeyya yaṃ paresaṃ assa dīgharattaṃ ahitāya dukkhāyāti?}}\\
\begin{addmargin}[1em]{2em}
\setstretch{.5}
{\PaliGlossB{or that they might encourage others to do what is for their lasting harm and suffering?’}}\\
\end{addmargin}
\end{absolutelynopagebreak}

\begin{absolutelynopagebreak}
\setstretch{.7}
{\PaliGlossA{Tamenaṃ samannesamāno evaṃ jānāti:}}\\
\begin{addmargin}[1em]{2em}
\setstretch{.5}
{\PaliGlossB{Scrutinizing them they find:}}\\
\end{addmargin}
\end{absolutelynopagebreak}

\begin{absolutelynopagebreak}
\setstretch{.7}
{\PaliGlossA{‘natthi kho imassāyasmato tathārūpā lobhanīyā dhammā yathārūpehi lobhanīyehi dhammehi pariyādinnacitto ajānaṃ vā vadeyya—}}\\
\begin{addmargin}[1em]{2em}
\setstretch{.5}
{\PaliGlossB{‘This venerable has no such qualities that arouse greed.}}\\
\end{addmargin}
\end{absolutelynopagebreak}

\begin{absolutelynopagebreak}
\setstretch{.7}
{\PaliGlossA{jānāmīti, apassaṃ vā vadeyya—}}\\
\begin{addmargin}[1em]{2em}
\setstretch{.5}
{\PaliGlossB{    -}}\\
\end{addmargin}
\end{absolutelynopagebreak}

\begin{absolutelynopagebreak}
\setstretch{.7}
{\PaliGlossA{passāmīti, paraṃ vā tadatthāya samādapeyya yaṃ paresaṃ assa dīgharattaṃ ahitāya dukkhāya.}}\\
\begin{addmargin}[1em]{2em}
\setstretch{.5}
{\PaliGlossB{    -}}\\
\end{addmargin}
\end{absolutelynopagebreak}

\begin{absolutelynopagebreak}
\setstretch{.7}
{\PaliGlossA{Tathārūpo kho panimassāyasmato kāyasamācāro tathārūpo vacīsamācāro yathā taṃ aluddhassa.}}\\
\begin{addmargin}[1em]{2em}
\setstretch{.5}
{\PaliGlossB{Rather, that venerable has bodily and verbal behavior like that of someone without greed.}}\\
\end{addmargin}
\end{absolutelynopagebreak}

\begin{absolutelynopagebreak}
\setstretch{.7}
{\PaliGlossA{Yaṃ kho pana ayamāyasmā dhammaṃ deseti, gambhīro so dhammo duddaso duranubodho santo paṇīto atakkāvacaro nipuṇo paṇḍitavedanīyo;}}\\
\begin{addmargin}[1em]{2em}
\setstretch{.5}
{\PaliGlossB{And the principle that they teach is deep, hard to see, hard to understand, peaceful, sublime, beyond the scope of reason, subtle, comprehensible to the astute.}}\\
\end{addmargin}
\end{absolutelynopagebreak}

\begin{absolutelynopagebreak}
\setstretch{.7}
{\PaliGlossA{na so dhammo sudesiyo luddhenā’ti.}}\\
\begin{addmargin}[1em]{2em}
\setstretch{.5}
{\PaliGlossB{It’s not easy for someone with greed to teach this.’}}\\
\end{addmargin}
\end{absolutelynopagebreak}

\vskip 0.05in
\begin{absolutelynopagebreak}
\setstretch{.7}
{\PaliGlossA{Yato naṃ samannesamāno visuddhaṃ lobhanīyehi dhammehi samanupassati tato naṃ uttari samannesati dosanīyesu dhammesu.}}\\
\begin{addmargin}[1em]{2em}
\setstretch{.5}
{\PaliGlossB{Scrutinizing them in this way they see that they are purified of qualities that arouse greed. Next, they search them for qualities that provoke hate.}}\\
\end{addmargin}
\end{absolutelynopagebreak}

\begin{absolutelynopagebreak}
\setstretch{.7}
{\PaliGlossA{Atthi nu kho imassāyasmato tathārūpā dosanīyā dhammā yathārūpehi dosanīyehi dhammehi pariyādinnacitto ajānaṃ vā vadeyya—}}\\
\begin{addmargin}[1em]{2em}
\setstretch{.5}
{\PaliGlossB{‘Does this venerable have any qualities that provoke hate? Such qualities that, were their mind to be overwhelmed by them, they might say}}\\
\end{addmargin}
\end{absolutelynopagebreak}

\begin{absolutelynopagebreak}
\setstretch{.7}
{\PaliGlossA{jānāmīti, apassaṃ vā vadeyya—}}\\
\begin{addmargin}[1em]{2em}
\setstretch{.5}
{\PaliGlossB{that they know, even though they don’t know, or that they see, even though they don’t see;}}\\
\end{addmargin}
\end{absolutelynopagebreak}

\begin{absolutelynopagebreak}
\setstretch{.7}
{\PaliGlossA{passāmīti, paraṃ vā tadatthāya samādapeyya yaṃ paresaṃ assa dīgharattaṃ ahitāya dukkhāyāti?}}\\
\begin{addmargin}[1em]{2em}
\setstretch{.5}
{\PaliGlossB{or that they might encourage others to do what is for their lasting harm and suffering?’}}\\
\end{addmargin}
\end{absolutelynopagebreak}

\begin{absolutelynopagebreak}
\setstretch{.7}
{\PaliGlossA{Tamenaṃ samannesamāno evaṃ jānāti:}}\\
\begin{addmargin}[1em]{2em}
\setstretch{.5}
{\PaliGlossB{Scrutinizing them they find:}}\\
\end{addmargin}
\end{absolutelynopagebreak}

\begin{absolutelynopagebreak}
\setstretch{.7}
{\PaliGlossA{‘natthi kho imassāyasmato tathārūpā dosanīyā dhammā yathārūpehi dosanīyehi dhammehi pariyādinnacitto ajānaṃ vā vadeyya—}}\\
\begin{addmargin}[1em]{2em}
\setstretch{.5}
{\PaliGlossB{‘This venerable has no such qualities that provoke hate.}}\\
\end{addmargin}
\end{absolutelynopagebreak}

\begin{absolutelynopagebreak}
\setstretch{.7}
{\PaliGlossA{jānāmīti, apassaṃ vā vadeyya—}}\\
\begin{addmargin}[1em]{2em}
\setstretch{.5}
{\PaliGlossB{    -}}\\
\end{addmargin}
\end{absolutelynopagebreak}

\begin{absolutelynopagebreak}
\setstretch{.7}
{\PaliGlossA{passāmīti, paraṃ vā tadatthāya samādapeyya yaṃ paresaṃ assa dīgharattaṃ ahitāya dukkhāya.}}\\
\begin{addmargin}[1em]{2em}
\setstretch{.5}
{\PaliGlossB{    -}}\\
\end{addmargin}
\end{absolutelynopagebreak}

\begin{absolutelynopagebreak}
\setstretch{.7}
{\PaliGlossA{Tathārūpo kho panimassāyasmato kāyasamācāro tathārūpo vacīsamācāro yathā taṃ aduṭṭhassa.}}\\
\begin{addmargin}[1em]{2em}
\setstretch{.5}
{\PaliGlossB{Rather, that venerable has bodily and verbal behavior like that of someone without hate.}}\\
\end{addmargin}
\end{absolutelynopagebreak}

\begin{absolutelynopagebreak}
\setstretch{.7}
{\PaliGlossA{Yaṃ kho pana ayamāyasmā dhammaṃ deseti, gambhīro so dhammo duddaso duranubodho santo paṇīto atakkāvacaro nipuṇo paṇḍitavedanīyo;}}\\
\begin{addmargin}[1em]{2em}
\setstretch{.5}
{\PaliGlossB{And the principle that they teach is deep, hard to see, hard to understand, peaceful, sublime, beyond the scope of reason, subtle, comprehensible to the astute.}}\\
\end{addmargin}
\end{absolutelynopagebreak}

\begin{absolutelynopagebreak}
\setstretch{.7}
{\PaliGlossA{na so dhammo sudesiyo duṭṭhenā’ti.}}\\
\begin{addmargin}[1em]{2em}
\setstretch{.5}
{\PaliGlossB{It’s not easy for someone with hate to teach this.’}}\\
\end{addmargin}
\end{absolutelynopagebreak}

\vskip 0.05in
\begin{absolutelynopagebreak}
\setstretch{.7}
{\PaliGlossA{Yato naṃ samannesamāno visuddhaṃ dosanīyehi dhammehi samanupassati, tato naṃ uttari samannesati mohanīyesu dhammesu.}}\\
\begin{addmargin}[1em]{2em}
\setstretch{.5}
{\PaliGlossB{Scrutinizing them in this way they see that they are purified of qualities that provoke hate. Next, they scrutinize them for qualities that promote delusion.}}\\
\end{addmargin}
\end{absolutelynopagebreak}

\begin{absolutelynopagebreak}
\setstretch{.7}
{\PaliGlossA{Atthi nu kho imassāyasmato tathārūpā mohanīyā dhammā yathārūpehi mohanīyehi dhammehi pariyādinnacitto ajānaṃ vā vadeyya—}}\\
\begin{addmargin}[1em]{2em}
\setstretch{.5}
{\PaliGlossB{‘Does this venerable have any qualities that promote delusion? Such qualities that, were their mind to be overwhelmed by them, they might say}}\\
\end{addmargin}
\end{absolutelynopagebreak}

\begin{absolutelynopagebreak}
\setstretch{.7}
{\PaliGlossA{jānāmīti, apassaṃ vā vadeyya—}}\\
\begin{addmargin}[1em]{2em}
\setstretch{.5}
{\PaliGlossB{that they know, even though they don’t know, or that they see, even though they don’t see;}}\\
\end{addmargin}
\end{absolutelynopagebreak}

\begin{absolutelynopagebreak}
\setstretch{.7}
{\PaliGlossA{passāmīti, paraṃ vā tadatthāya samādapeyya yaṃ paresaṃ assa dīgharattaṃ ahitāya dukkhāyāti?}}\\
\begin{addmargin}[1em]{2em}
\setstretch{.5}
{\PaliGlossB{or that they might encourage others to do what is for their lasting harm and suffering?’}}\\
\end{addmargin}
\end{absolutelynopagebreak}

\begin{absolutelynopagebreak}
\setstretch{.7}
{\PaliGlossA{Tamenaṃ samannesamāno evaṃ jānāti:}}\\
\begin{addmargin}[1em]{2em}
\setstretch{.5}
{\PaliGlossB{Scrutinizing them they find:}}\\
\end{addmargin}
\end{absolutelynopagebreak}

\begin{absolutelynopagebreak}
\setstretch{.7}
{\PaliGlossA{‘natthi kho imassāyasmato tathārūpā mohanīyā dhammā yathārūpehi mohanīyehi dhammehi pariyādinnacitto ajānaṃ vā vadeyya—}}\\
\begin{addmargin}[1em]{2em}
\setstretch{.5}
{\PaliGlossB{‘This venerable has no such qualities that promote delusion.}}\\
\end{addmargin}
\end{absolutelynopagebreak}

\begin{absolutelynopagebreak}
\setstretch{.7}
{\PaliGlossA{jānāmīti, apassaṃ vā vadeyya—}}\\
\begin{addmargin}[1em]{2em}
\setstretch{.5}
{\PaliGlossB{    -}}\\
\end{addmargin}
\end{absolutelynopagebreak}

\begin{absolutelynopagebreak}
\setstretch{.7}
{\PaliGlossA{passāmīti, paraṃ vā tadatthāya samādapeyya yaṃ paresaṃ assa dīgharattaṃ ahitāya dukkhāya.}}\\
\begin{addmargin}[1em]{2em}
\setstretch{.5}
{\PaliGlossB{    -}}\\
\end{addmargin}
\end{absolutelynopagebreak}

\begin{absolutelynopagebreak}
\setstretch{.7}
{\PaliGlossA{Tathārūpo kho panimassāyasmato kāyasamācāro tathārūpo vacīsamācāro yathā taṃ amūḷhassa.}}\\
\begin{addmargin}[1em]{2em}
\setstretch{.5}
{\PaliGlossB{Rather, that venerable has bodily and verbal behavior like that of someone without delusion.}}\\
\end{addmargin}
\end{absolutelynopagebreak}

\begin{absolutelynopagebreak}
\setstretch{.7}
{\PaliGlossA{Yaṃ kho pana ayamāyasmā dhammaṃ deseti, gambhīro so dhammo duddaso duranubodho santo paṇīto atakkāvacaro nipuṇo paṇḍitavedanīyo;}}\\
\begin{addmargin}[1em]{2em}
\setstretch{.5}
{\PaliGlossB{And the principle that they teach is deep, hard to see, hard to understand, peaceful, sublime, beyond the scope of reason, subtle, comprehensible to the astute.}}\\
\end{addmargin}
\end{absolutelynopagebreak}

\begin{absolutelynopagebreak}
\setstretch{.7}
{\PaliGlossA{na so dhammo sudesiyo mūḷhenā’ti.}}\\
\begin{addmargin}[1em]{2em}
\setstretch{.5}
{\PaliGlossB{It’s not easy for someone with delusion to teach this.’}}\\
\end{addmargin}
\end{absolutelynopagebreak}

\vskip 0.05in
\begin{absolutelynopagebreak}
\setstretch{.7}
{\PaliGlossA{Yato naṃ samannesamāno visuddhaṃ mohanīyehi dhammehi samanupassati;}}\\
\begin{addmargin}[1em]{2em}
\setstretch{.5}
{\PaliGlossB{Scrutinizing them in this way they see that they are purified of qualities that promote delusion.}}\\
\end{addmargin}
\end{absolutelynopagebreak}

\begin{absolutelynopagebreak}
\setstretch{.7}
{\PaliGlossA{atha tamhi saddhaṃ niveseti, saddhājāto upasaṅkamati, upasaṅkamanto payirupāsati, payirupāsanto sotaṃ odahati, ohitasoto dhammaṃ suṇāti, sutvā dhammaṃ dhāreti, dhatānaṃ dhammānaṃ atthaṃ upaparikkhati, atthaṃ upaparikkhato dhammā nijjhānaṃ khamanti, dhammanijjhānakkhantiyā sati chando jāyati, chandajāto ussahati, ussahitvā tuleti, tulayitvā padahati, pahitatto samāno kāyena ceva paramasaccaṃ sacchikaroti paññāya ca naṃ ativijjha passati.}}\\
\begin{addmargin}[1em]{2em}
\setstretch{.5}
{\PaliGlossB{Next, they place faith in them. When faith has arisen they approach the teacher. They pay homage, lend an ear, hear the teachings, remember the teachings, reflect on their meaning, and accept them after consideration. Then enthusiasm springs up; they make an effort, weigh up, and persevere. Persevering, they directly realize the ultimate truth, and see it with penetrating wisdom.}}\\
\end{addmargin}
\end{absolutelynopagebreak}

\begin{absolutelynopagebreak}
\setstretch{.7}
{\PaliGlossA{Ettāvatā kho, bhāradvāja, saccānubodho hoti, ettāvatā saccamanubujjhati, ettāvatā ca mayaṃ saccānubodhaṃ paññapema;}}\\
\begin{addmargin}[1em]{2em}
\setstretch{.5}
{\PaliGlossB{That’s how the awakening to truth is defined, Bhāradvāja. I describe the awakening to truth as defined in this way.}}\\
\end{addmargin}
\end{absolutelynopagebreak}

\begin{absolutelynopagebreak}
\setstretch{.7}
{\PaliGlossA{na tveva tāva saccānuppatti hotī”ti.}}\\
\begin{addmargin}[1em]{2em}
\setstretch{.5}
{\PaliGlossB{But this is not yet the arrival at the truth.”}}\\
\end{addmargin}
\end{absolutelynopagebreak}

\vskip 0.05in
\begin{absolutelynopagebreak}
\setstretch{.7}
{\PaliGlossA{“Ettāvatā, bho gotama, saccānubodho hoti, ettāvatā saccamanubujjhati, ettāvatā ca mayaṃ saccānubodhaṃ pekkhāma.}}\\
\begin{addmargin}[1em]{2em}
\setstretch{.5}
{\PaliGlossB{“That’s how the awakening to truth is defined, Master Gotama. I regard the awakening to truth as defined in this way.}}\\
\end{addmargin}
\end{absolutelynopagebreak}

\begin{absolutelynopagebreak}
\setstretch{.7}
{\PaliGlossA{Kittāvatā pana, bho gotama, saccānuppatti hoti, kittāvatā saccamanupāpuṇāti?}}\\
\begin{addmargin}[1em]{2em}
\setstretch{.5}
{\PaliGlossB{But Master Gotama, how do you define the arrival at the truth?”}}\\
\end{addmargin}
\end{absolutelynopagebreak}

\begin{absolutelynopagebreak}
\setstretch{.7}
{\PaliGlossA{Saccānuppattiṃ mayaṃ bhavantaṃ gotamaṃ pucchāmā”ti.}}\\
\begin{addmargin}[1em]{2em}
\setstretch{.5}
{\PaliGlossB{    -}}\\
\end{addmargin}
\end{absolutelynopagebreak}

\begin{absolutelynopagebreak}
\setstretch{.7}
{\PaliGlossA{“Tesaṃyeva, bhāradvāja, dhammānaṃ āsevanā bhāvanā bahulīkammaṃ saccānuppatti hoti.}}\\
\begin{addmargin}[1em]{2em}
\setstretch{.5}
{\PaliGlossB{“By the cultivation, development, and making much of these very same things there is the arrival at the truth.}}\\
\end{addmargin}
\end{absolutelynopagebreak}

\begin{absolutelynopagebreak}
\setstretch{.7}
{\PaliGlossA{Ettāvatā kho, bhāradvāja, saccānuppatti hoti, ettāvatā saccamanupāpuṇāti, ettāvatā ca mayaṃ saccānuppattiṃ paññapemā”ti.}}\\
\begin{addmargin}[1em]{2em}
\setstretch{.5}
{\PaliGlossB{That’s how the arrival at the truth is defined, Bhāradvāja. I describe the arrival at the truth as defined in this way.”}}\\
\end{addmargin}
\end{absolutelynopagebreak}

\vskip 0.05in
\begin{absolutelynopagebreak}
\setstretch{.7}
{\PaliGlossA{“Ettāvatā, bho gotama, saccānuppatti hoti, ettāvatā saccamanupāpuṇāti, ettāvatā ca mayaṃ saccānuppattiṃ pekkhāma.}}\\
\begin{addmargin}[1em]{2em}
\setstretch{.5}
{\PaliGlossB{“That’s how the arrival at the truth is defined, Master Gotama. I regard the arrival at the truth as defined in this way.}}\\
\end{addmargin}
\end{absolutelynopagebreak}

\begin{absolutelynopagebreak}
\setstretch{.7}
{\PaliGlossA{Saccānuppattiyā pana, bho gotama, katamo dhammo bahukāro?}}\\
\begin{addmargin}[1em]{2em}
\setstretch{.5}
{\PaliGlossB{But what quality is helpful for arriving at the truth?”}}\\
\end{addmargin}
\end{absolutelynopagebreak}

\begin{absolutelynopagebreak}
\setstretch{.7}
{\PaliGlossA{Saccānuppattiyā bahukāraṃ dhammaṃ mayaṃ bhavantaṃ gotamaṃ pucchāmā”ti.}}\\
\begin{addmargin}[1em]{2em}
\setstretch{.5}
{\PaliGlossB{    -}}\\
\end{addmargin}
\end{absolutelynopagebreak}

\begin{absolutelynopagebreak}
\setstretch{.7}
{\PaliGlossA{“Saccānuppattiyā kho, bhāradvāja, padhānaṃ bahukāraṃ.}}\\
\begin{addmargin}[1em]{2em}
\setstretch{.5}
{\PaliGlossB{“Striving is helpful for arriving at the truth.}}\\
\end{addmargin}
\end{absolutelynopagebreak}

\begin{absolutelynopagebreak}
\setstretch{.7}
{\PaliGlossA{No cetaṃ padaheyya, nayidaṃ saccamanupāpuṇeyya.}}\\
\begin{addmargin}[1em]{2em}
\setstretch{.5}
{\PaliGlossB{If you don’t strive, you won’t arrive at the truth.}}\\
\end{addmargin}
\end{absolutelynopagebreak}

\begin{absolutelynopagebreak}
\setstretch{.7}
{\PaliGlossA{Yasmā ca kho padahati tasmā saccamanupāpuṇāti.}}\\
\begin{addmargin}[1em]{2em}
\setstretch{.5}
{\PaliGlossB{You arrive at the truth because you strive.}}\\
\end{addmargin}
\end{absolutelynopagebreak}

\begin{absolutelynopagebreak}
\setstretch{.7}
{\PaliGlossA{Tasmā saccānuppattiyā padhānaṃ bahukāran”ti.}}\\
\begin{addmargin}[1em]{2em}
\setstretch{.5}
{\PaliGlossB{That’s why striving is helpful for arriving at the truth.”}}\\
\end{addmargin}
\end{absolutelynopagebreak}

\vskip 0.05in
\begin{absolutelynopagebreak}
\setstretch{.7}
{\PaliGlossA{“Padhānassa pana, bho gotama, katamo dhammo bahukāro?}}\\
\begin{addmargin}[1em]{2em}
\setstretch{.5}
{\PaliGlossB{“But what quality is helpful for striving?”}}\\
\end{addmargin}
\end{absolutelynopagebreak}

\begin{absolutelynopagebreak}
\setstretch{.7}
{\PaliGlossA{Padhānassa bahukāraṃ dhammaṃ mayaṃ bhavantaṃ gotamaṃ pucchāmā”ti.}}\\
\begin{addmargin}[1em]{2em}
\setstretch{.5}
{\PaliGlossB{    -}}\\
\end{addmargin}
\end{absolutelynopagebreak}

\begin{absolutelynopagebreak}
\setstretch{.7}
{\PaliGlossA{“Padhānassa kho, bhāradvāja, tulanā bahukārā.}}\\
\begin{addmargin}[1em]{2em}
\setstretch{.5}
{\PaliGlossB{“Weighing up the teachings is helpful for striving …}}\\
\end{addmargin}
\end{absolutelynopagebreak}

\begin{absolutelynopagebreak}
\setstretch{.7}
{\PaliGlossA{No cetaṃ tuleyya, nayidaṃ padaheyya.}}\\
\begin{addmargin}[1em]{2em}
\setstretch{.5}
{\PaliGlossB{    -}}\\
\end{addmargin}
\end{absolutelynopagebreak}

\begin{absolutelynopagebreak}
\setstretch{.7}
{\PaliGlossA{Yasmā ca kho tuleti tasmā padahati.}}\\
\begin{addmargin}[1em]{2em}
\setstretch{.5}
{\PaliGlossB{    -}}\\
\end{addmargin}
\end{absolutelynopagebreak}

\begin{absolutelynopagebreak}
\setstretch{.7}
{\PaliGlossA{Tasmā padhānassa tulanā bahukārā”ti.}}\\
\begin{addmargin}[1em]{2em}
\setstretch{.5}
{\PaliGlossB{    -}}\\
\end{addmargin}
\end{absolutelynopagebreak}

\vskip 0.05in
\begin{absolutelynopagebreak}
\setstretch{.7}
{\PaliGlossA{“Tulanāya pana, bho gotama, katamo dhammo bahukāro?}}\\
\begin{addmargin}[1em]{2em}
\setstretch{.5}
{\PaliGlossB{    -}}\\
\end{addmargin}
\end{absolutelynopagebreak}

\begin{absolutelynopagebreak}
\setstretch{.7}
{\PaliGlossA{Tulanāya bahukāraṃ dhammaṃ mayaṃ bhavantaṃ gotamaṃ pucchāmā”ti.}}\\
\begin{addmargin}[1em]{2em}
\setstretch{.5}
{\PaliGlossB{    -}}\\
\end{addmargin}
\end{absolutelynopagebreak}

\begin{absolutelynopagebreak}
\setstretch{.7}
{\PaliGlossA{“Tulanāya kho, bhāradvāja, ussāho bahukāro.}}\\
\begin{addmargin}[1em]{2em}
\setstretch{.5}
{\PaliGlossB{Making an effort is helpful for weighing up the teachings …}}\\
\end{addmargin}
\end{absolutelynopagebreak}

\begin{absolutelynopagebreak}
\setstretch{.7}
{\PaliGlossA{No cetaṃ ussaheyya, nayidaṃ tuleyya.}}\\
\begin{addmargin}[1em]{2em}
\setstretch{.5}
{\PaliGlossB{    -}}\\
\end{addmargin}
\end{absolutelynopagebreak}

\begin{absolutelynopagebreak}
\setstretch{.7}
{\PaliGlossA{Yasmā ca kho ussahati tasmā tuleti.}}\\
\begin{addmargin}[1em]{2em}
\setstretch{.5}
{\PaliGlossB{    -}}\\
\end{addmargin}
\end{absolutelynopagebreak}

\begin{absolutelynopagebreak}
\setstretch{.7}
{\PaliGlossA{Tasmā tulanāya ussāho bahukāro”ti.}}\\
\begin{addmargin}[1em]{2em}
\setstretch{.5}
{\PaliGlossB{    -}}\\
\end{addmargin}
\end{absolutelynopagebreak}

\vskip 0.05in
\begin{absolutelynopagebreak}
\setstretch{.7}
{\PaliGlossA{“Ussāhassa pana, bho gotama, katamo dhammo bahukāro?}}\\
\begin{addmargin}[1em]{2em}
\setstretch{.5}
{\PaliGlossB{    -}}\\
\end{addmargin}
\end{absolutelynopagebreak}

\begin{absolutelynopagebreak}
\setstretch{.7}
{\PaliGlossA{Ussāhassa bahukāraṃ dhammaṃ mayaṃ bhavantaṃ gotamaṃ pucchāmā”ti.}}\\
\begin{addmargin}[1em]{2em}
\setstretch{.5}
{\PaliGlossB{    -}}\\
\end{addmargin}
\end{absolutelynopagebreak}

\begin{absolutelynopagebreak}
\setstretch{.7}
{\PaliGlossA{“Ussāhassa kho, bhāradvāja, chando bahukāro.}}\\
\begin{addmargin}[1em]{2em}
\setstretch{.5}
{\PaliGlossB{Enthusiasm is helpful for making an effort …}}\\
\end{addmargin}
\end{absolutelynopagebreak}

\begin{absolutelynopagebreak}
\setstretch{.7}
{\PaliGlossA{No cetaṃ chando jāyetha, nayidaṃ ussaheyya.}}\\
\begin{addmargin}[1em]{2em}
\setstretch{.5}
{\PaliGlossB{    -}}\\
\end{addmargin}
\end{absolutelynopagebreak}

\begin{absolutelynopagebreak}
\setstretch{.7}
{\PaliGlossA{Yasmā ca kho chando jāyati tasmā ussahati.}}\\
\begin{addmargin}[1em]{2em}
\setstretch{.5}
{\PaliGlossB{    -}}\\
\end{addmargin}
\end{absolutelynopagebreak}

\begin{absolutelynopagebreak}
\setstretch{.7}
{\PaliGlossA{Tasmā ussāhassa chando bahukāro”ti.}}\\
\begin{addmargin}[1em]{2em}
\setstretch{.5}
{\PaliGlossB{    -}}\\
\end{addmargin}
\end{absolutelynopagebreak}

\vskip 0.05in
\begin{absolutelynopagebreak}
\setstretch{.7}
{\PaliGlossA{“Chandassa pana, bho gotama, katamo dhammo bahukāro?}}\\
\begin{addmargin}[1em]{2em}
\setstretch{.5}
{\PaliGlossB{    -}}\\
\end{addmargin}
\end{absolutelynopagebreak}

\begin{absolutelynopagebreak}
\setstretch{.7}
{\PaliGlossA{Chandassa bahukāraṃ dhammaṃ mayaṃ bhavantaṃ gotamaṃ pucchāmā”ti.}}\\
\begin{addmargin}[1em]{2em}
\setstretch{.5}
{\PaliGlossB{    -}}\\
\end{addmargin}
\end{absolutelynopagebreak}

\begin{absolutelynopagebreak}
\setstretch{.7}
{\PaliGlossA{“Chandassa kho, bhāradvāja, dhammanijjhānakkhanti bahukārā.}}\\
\begin{addmargin}[1em]{2em}
\setstretch{.5}
{\PaliGlossB{Acceptance of the teachings after consideration is helpful for enthusiasm …}}\\
\end{addmargin}
\end{absolutelynopagebreak}

\begin{absolutelynopagebreak}
\setstretch{.7}
{\PaliGlossA{No cete dhammā nijjhānaṃ khameyyuṃ, nayidaṃ chando jāyetha.}}\\
\begin{addmargin}[1em]{2em}
\setstretch{.5}
{\PaliGlossB{    -}}\\
\end{addmargin}
\end{absolutelynopagebreak}

\begin{absolutelynopagebreak}
\setstretch{.7}
{\PaliGlossA{Yasmā ca kho dhammā nijjhānaṃ khamanti tasmā chando jāyati.}}\\
\begin{addmargin}[1em]{2em}
\setstretch{.5}
{\PaliGlossB{    -}}\\
\end{addmargin}
\end{absolutelynopagebreak}

\begin{absolutelynopagebreak}
\setstretch{.7}
{\PaliGlossA{Tasmā chandassa dhammanijjhānakkhanti bahukārā”ti.}}\\
\begin{addmargin}[1em]{2em}
\setstretch{.5}
{\PaliGlossB{    -}}\\
\end{addmargin}
\end{absolutelynopagebreak}

\vskip 0.05in
\begin{absolutelynopagebreak}
\setstretch{.7}
{\PaliGlossA{“Dhammanijjhānakkhantiyā pana, bho gotama, katamo dhammo bahukāro?}}\\
\begin{addmargin}[1em]{2em}
\setstretch{.5}
{\PaliGlossB{    -}}\\
\end{addmargin}
\end{absolutelynopagebreak}

\begin{absolutelynopagebreak}
\setstretch{.7}
{\PaliGlossA{Dhammanijjhānakkhantiyā bahukāraṃ dhammaṃ mayaṃ bhavantaṃ gotamaṃ pucchāmā”ti.}}\\
\begin{addmargin}[1em]{2em}
\setstretch{.5}
{\PaliGlossB{    -}}\\
\end{addmargin}
\end{absolutelynopagebreak}

\begin{absolutelynopagebreak}
\setstretch{.7}
{\PaliGlossA{“Dhammanijjhānakkhantiyā kho, bhāradvāja, atthūpaparikkhā bahukārā.}}\\
\begin{addmargin}[1em]{2em}
\setstretch{.5}
{\PaliGlossB{Reflecting on the meaning of the teachings is helpful for accepting them after consideration …}}\\
\end{addmargin}
\end{absolutelynopagebreak}

\begin{absolutelynopagebreak}
\setstretch{.7}
{\PaliGlossA{No cetaṃ atthaṃ upaparikkheyya, nayidaṃ dhammā nijjhānaṃ khameyyuṃ.}}\\
\begin{addmargin}[1em]{2em}
\setstretch{.5}
{\PaliGlossB{    -}}\\
\end{addmargin}
\end{absolutelynopagebreak}

\begin{absolutelynopagebreak}
\setstretch{.7}
{\PaliGlossA{Yasmā ca kho atthaṃ upaparikkhati tasmā dhammā nijjhānaṃ khamanti.}}\\
\begin{addmargin}[1em]{2em}
\setstretch{.5}
{\PaliGlossB{    -}}\\
\end{addmargin}
\end{absolutelynopagebreak}

\begin{absolutelynopagebreak}
\setstretch{.7}
{\PaliGlossA{Tasmā dhammanijjhānakkhantiyā atthūpaparikkhā bahukārā”ti.}}\\
\begin{addmargin}[1em]{2em}
\setstretch{.5}
{\PaliGlossB{    -}}\\
\end{addmargin}
\end{absolutelynopagebreak}

\vskip 0.05in
\begin{absolutelynopagebreak}
\setstretch{.7}
{\PaliGlossA{“Atthūpaparikkhāya pana, bho gotama, katamo dhammo bahukāro?}}\\
\begin{addmargin}[1em]{2em}
\setstretch{.5}
{\PaliGlossB{    -}}\\
\end{addmargin}
\end{absolutelynopagebreak}

\begin{absolutelynopagebreak}
\setstretch{.7}
{\PaliGlossA{Atthūpaparikkhāya bahukāraṃ dhammaṃ mayaṃ bhavantaṃ gotamaṃ pucchāmā”ti.}}\\
\begin{addmargin}[1em]{2em}
\setstretch{.5}
{\PaliGlossB{    -}}\\
\end{addmargin}
\end{absolutelynopagebreak}

\begin{absolutelynopagebreak}
\setstretch{.7}
{\PaliGlossA{“Atthūpaparikkhāya kho, bhāradvāja, dhammadhāraṇā bahukārā.}}\\
\begin{addmargin}[1em]{2em}
\setstretch{.5}
{\PaliGlossB{Remembering the teachings is helpful for reflecting on their meaning …}}\\
\end{addmargin}
\end{absolutelynopagebreak}

\begin{absolutelynopagebreak}
\setstretch{.7}
{\PaliGlossA{No cetaṃ dhammaṃ dhāreyya, nayidaṃ atthaṃ upaparikkheyya.}}\\
\begin{addmargin}[1em]{2em}
\setstretch{.5}
{\PaliGlossB{    -}}\\
\end{addmargin}
\end{absolutelynopagebreak}

\begin{absolutelynopagebreak}
\setstretch{.7}
{\PaliGlossA{Yasmā ca kho dhammaṃ dhāreti tasmā atthaṃ upaparikkhati.}}\\
\begin{addmargin}[1em]{2em}
\setstretch{.5}
{\PaliGlossB{    -}}\\
\end{addmargin}
\end{absolutelynopagebreak}

\begin{absolutelynopagebreak}
\setstretch{.7}
{\PaliGlossA{Tasmā atthūpaparikkhāya dhammadhāraṇā bahukārā”ti.}}\\
\begin{addmargin}[1em]{2em}
\setstretch{.5}
{\PaliGlossB{    -}}\\
\end{addmargin}
\end{absolutelynopagebreak}

\vskip 0.05in
\begin{absolutelynopagebreak}
\setstretch{.7}
{\PaliGlossA{“Dhammadhāraṇāya pana, bho gotama, katamo dhammo bahukāro?}}\\
\begin{addmargin}[1em]{2em}
\setstretch{.5}
{\PaliGlossB{    -}}\\
\end{addmargin}
\end{absolutelynopagebreak}

\begin{absolutelynopagebreak}
\setstretch{.7}
{\PaliGlossA{Dhammadhāraṇāya bahukāraṃ dhammaṃ mayaṃ bhavantaṃ gotamaṃ pucchāmā”ti.}}\\
\begin{addmargin}[1em]{2em}
\setstretch{.5}
{\PaliGlossB{    -}}\\
\end{addmargin}
\end{absolutelynopagebreak}

\begin{absolutelynopagebreak}
\setstretch{.7}
{\PaliGlossA{“Dhammadhāraṇāya kho, bhāradvāja, dhammassavanaṃ bahukāraṃ.}}\\
\begin{addmargin}[1em]{2em}
\setstretch{.5}
{\PaliGlossB{Hearing the teachings is helpful for remembering the teachings …}}\\
\end{addmargin}
\end{absolutelynopagebreak}

\begin{absolutelynopagebreak}
\setstretch{.7}
{\PaliGlossA{No cetaṃ dhammaṃ suṇeyya, nayidaṃ dhammaṃ dhāreyya.}}\\
\begin{addmargin}[1em]{2em}
\setstretch{.5}
{\PaliGlossB{    -}}\\
\end{addmargin}
\end{absolutelynopagebreak}

\begin{absolutelynopagebreak}
\setstretch{.7}
{\PaliGlossA{Yasmā ca kho dhammaṃ suṇāti tasmā dhammaṃ dhāreti.}}\\
\begin{addmargin}[1em]{2em}
\setstretch{.5}
{\PaliGlossB{    -}}\\
\end{addmargin}
\end{absolutelynopagebreak}

\begin{absolutelynopagebreak}
\setstretch{.7}
{\PaliGlossA{Tasmā dhammadhāraṇāya dhammassavanaṃ bahukāran”ti.}}\\
\begin{addmargin}[1em]{2em}
\setstretch{.5}
{\PaliGlossB{    -}}\\
\end{addmargin}
\end{absolutelynopagebreak}

\vskip 0.05in
\begin{absolutelynopagebreak}
\setstretch{.7}
{\PaliGlossA{“Dhammassavanassa pana, bho gotama, katamo dhammo bahukāro?}}\\
\begin{addmargin}[1em]{2em}
\setstretch{.5}
{\PaliGlossB{    -}}\\
\end{addmargin}
\end{absolutelynopagebreak}

\begin{absolutelynopagebreak}
\setstretch{.7}
{\PaliGlossA{Dhammassavanassa bahukāraṃ dhammaṃ mayaṃ bhavantaṃ gotamaṃ pucchāmā”ti.}}\\
\begin{addmargin}[1em]{2em}
\setstretch{.5}
{\PaliGlossB{    -}}\\
\end{addmargin}
\end{absolutelynopagebreak}

\begin{absolutelynopagebreak}
\setstretch{.7}
{\PaliGlossA{“Dhammassavanassa kho, bhāradvāja, sotāvadhānaṃ bahukāraṃ.}}\\
\begin{addmargin}[1em]{2em}
\setstretch{.5}
{\PaliGlossB{Listening is helpful for hearing the teachings …}}\\
\end{addmargin}
\end{absolutelynopagebreak}

\begin{absolutelynopagebreak}
\setstretch{.7}
{\PaliGlossA{No cetaṃ sotaṃ odaheyya, nayidaṃ dhammaṃ suṇeyya.}}\\
\begin{addmargin}[1em]{2em}
\setstretch{.5}
{\PaliGlossB{    -}}\\
\end{addmargin}
\end{absolutelynopagebreak}

\begin{absolutelynopagebreak}
\setstretch{.7}
{\PaliGlossA{Yasmā ca kho sotaṃ odahati tasmā dhammaṃ suṇāti.}}\\
\begin{addmargin}[1em]{2em}
\setstretch{.5}
{\PaliGlossB{    -}}\\
\end{addmargin}
\end{absolutelynopagebreak}

\begin{absolutelynopagebreak}
\setstretch{.7}
{\PaliGlossA{Tasmā dhammassavanassa sotāvadhānaṃ bahukāran”ti.}}\\
\begin{addmargin}[1em]{2em}
\setstretch{.5}
{\PaliGlossB{    -}}\\
\end{addmargin}
\end{absolutelynopagebreak}

\vskip 0.05in
\begin{absolutelynopagebreak}
\setstretch{.7}
{\PaliGlossA{“Sotāvadhānassa pana, bho gotama, katamo dhammo bahukāro?}}\\
\begin{addmargin}[1em]{2em}
\setstretch{.5}
{\PaliGlossB{    -}}\\
\end{addmargin}
\end{absolutelynopagebreak}

\begin{absolutelynopagebreak}
\setstretch{.7}
{\PaliGlossA{Sotāvadhānassa bahukāraṃ dhammaṃ mayaṃ bhavantaṃ gotamaṃ pucchāmā”ti.}}\\
\begin{addmargin}[1em]{2em}
\setstretch{.5}
{\PaliGlossB{    -}}\\
\end{addmargin}
\end{absolutelynopagebreak}

\begin{absolutelynopagebreak}
\setstretch{.7}
{\PaliGlossA{“Sotāvadhānassa kho, bhāradvāja, payirupāsanā bahukārā.}}\\
\begin{addmargin}[1em]{2em}
\setstretch{.5}
{\PaliGlossB{Paying homage is helpful for listening …}}\\
\end{addmargin}
\end{absolutelynopagebreak}

\begin{absolutelynopagebreak}
\setstretch{.7}
{\PaliGlossA{No cetaṃ payirupāseyya, nayidaṃ sotaṃ odaheyya.}}\\
\begin{addmargin}[1em]{2em}
\setstretch{.5}
{\PaliGlossB{    -}}\\
\end{addmargin}
\end{absolutelynopagebreak}

\begin{absolutelynopagebreak}
\setstretch{.7}
{\PaliGlossA{Yasmā ca kho payirupāsati tasmā sotaṃ odahati.}}\\
\begin{addmargin}[1em]{2em}
\setstretch{.5}
{\PaliGlossB{    -}}\\
\end{addmargin}
\end{absolutelynopagebreak}

\begin{absolutelynopagebreak}
\setstretch{.7}
{\PaliGlossA{Tasmā sotāvadhānassa payirupāsanā bahukārā”ti.}}\\
\begin{addmargin}[1em]{2em}
\setstretch{.5}
{\PaliGlossB{    -}}\\
\end{addmargin}
\end{absolutelynopagebreak}

\vskip 0.05in
\begin{absolutelynopagebreak}
\setstretch{.7}
{\PaliGlossA{“Payirupāsanāya pana, bho gotama, katamo dhammo bahukāro?}}\\
\begin{addmargin}[1em]{2em}
\setstretch{.5}
{\PaliGlossB{    -}}\\
\end{addmargin}
\end{absolutelynopagebreak}

\begin{absolutelynopagebreak}
\setstretch{.7}
{\PaliGlossA{Payirupāsanāya bahukāraṃ dhammaṃ mayaṃ bhavantaṃ gotamaṃ pucchāmā”ti.}}\\
\begin{addmargin}[1em]{2em}
\setstretch{.5}
{\PaliGlossB{    -}}\\
\end{addmargin}
\end{absolutelynopagebreak}

\begin{absolutelynopagebreak}
\setstretch{.7}
{\PaliGlossA{“Payirupāsanāya kho, bhāradvāja, upasaṅkamanaṃ bahukāraṃ.}}\\
\begin{addmargin}[1em]{2em}
\setstretch{.5}
{\PaliGlossB{Approaching is helpful for paying homage …}}\\
\end{addmargin}
\end{absolutelynopagebreak}

\begin{absolutelynopagebreak}
\setstretch{.7}
{\PaliGlossA{No cetaṃ upasaṅkameyya, nayidaṃ payirupāseyya.}}\\
\begin{addmargin}[1em]{2em}
\setstretch{.5}
{\PaliGlossB{    -}}\\
\end{addmargin}
\end{absolutelynopagebreak}

\begin{absolutelynopagebreak}
\setstretch{.7}
{\PaliGlossA{Yasmā ca kho upasaṅkamati tasmā payirupāsati.}}\\
\begin{addmargin}[1em]{2em}
\setstretch{.5}
{\PaliGlossB{    -}}\\
\end{addmargin}
\end{absolutelynopagebreak}

\begin{absolutelynopagebreak}
\setstretch{.7}
{\PaliGlossA{Tasmā payirupāsanāya upasaṅkamanaṃ bahukāran”ti.}}\\
\begin{addmargin}[1em]{2em}
\setstretch{.5}
{\PaliGlossB{    -}}\\
\end{addmargin}
\end{absolutelynopagebreak}

\vskip 0.05in
\begin{absolutelynopagebreak}
\setstretch{.7}
{\PaliGlossA{“Upasaṅkamanassa pana, bho gotama, katamo dhammo bahukāro?}}\\
\begin{addmargin}[1em]{2em}
\setstretch{.5}
{\PaliGlossB{    -}}\\
\end{addmargin}
\end{absolutelynopagebreak}

\begin{absolutelynopagebreak}
\setstretch{.7}
{\PaliGlossA{Upasaṅkamanassa bahukāraṃ dhammaṃ mayaṃ bhavantaṃ gotamaṃ pucchāmā”ti.}}\\
\begin{addmargin}[1em]{2em}
\setstretch{.5}
{\PaliGlossB{    -}}\\
\end{addmargin}
\end{absolutelynopagebreak}

\begin{absolutelynopagebreak}
\setstretch{.7}
{\PaliGlossA{“Upasaṅkamanassa kho, bhāradvāja, saddhā bahukārā.}}\\
\begin{addmargin}[1em]{2em}
\setstretch{.5}
{\PaliGlossB{Faith is helpful for approaching a teacher.}}\\
\end{addmargin}
\end{absolutelynopagebreak}

\begin{absolutelynopagebreak}
\setstretch{.7}
{\PaliGlossA{No cetaṃ saddhā jāyetha, nayidaṃ upasaṅkameyya.}}\\
\begin{addmargin}[1em]{2em}
\setstretch{.5}
{\PaliGlossB{If you don’t give rise to faith, you won’t approach a teacher.}}\\
\end{addmargin}
\end{absolutelynopagebreak}

\begin{absolutelynopagebreak}
\setstretch{.7}
{\PaliGlossA{Yasmā ca kho saddhā jāyati tasmā upasaṅkamati.}}\\
\begin{addmargin}[1em]{2em}
\setstretch{.5}
{\PaliGlossB{You approach a teacher because you have faith.}}\\
\end{addmargin}
\end{absolutelynopagebreak}

\begin{absolutelynopagebreak}
\setstretch{.7}
{\PaliGlossA{Tasmā upasaṅkamanassa saddhā bahukārā”ti.}}\\
\begin{addmargin}[1em]{2em}
\setstretch{.5}
{\PaliGlossB{That’s why faith is helpful for approaching a teacher.”}}\\
\end{addmargin}
\end{absolutelynopagebreak}

\vskip 0.05in
\begin{absolutelynopagebreak}
\setstretch{.7}
{\PaliGlossA{“Saccānurakkhaṇaṃ mayaṃ bhavantaṃ gotamaṃ apucchimha, saccānurakkhaṇaṃ bhavaṃ gotamo byākāsi;}}\\
\begin{addmargin}[1em]{2em}
\setstretch{.5}
{\PaliGlossB{“I’ve asked Master Gotama about the preservation of truth, and he has answered me.}}\\
\end{addmargin}
\end{absolutelynopagebreak}

\begin{absolutelynopagebreak}
\setstretch{.7}
{\PaliGlossA{tañca panamhākaṃ ruccati ceva khamati ca tena camha attamanā.}}\\
\begin{addmargin}[1em]{2em}
\setstretch{.5}
{\PaliGlossB{I like and accept this, and am satisfied with it.}}\\
\end{addmargin}
\end{absolutelynopagebreak}

\begin{absolutelynopagebreak}
\setstretch{.7}
{\PaliGlossA{Saccānubodhaṃ mayaṃ bhavantaṃ gotamaṃ apucchimha, saccānubodhaṃ bhavaṃ gotamo byākāsi;}}\\
\begin{addmargin}[1em]{2em}
\setstretch{.5}
{\PaliGlossB{I’ve asked Master Gotama about awakening to the truth, and he has answered me.}}\\
\end{addmargin}
\end{absolutelynopagebreak}

\begin{absolutelynopagebreak}
\setstretch{.7}
{\PaliGlossA{tañca panamhākaṃ ruccati ceva khamati ca tena camha attamanā.}}\\
\begin{addmargin}[1em]{2em}
\setstretch{.5}
{\PaliGlossB{I like and accept this, and am satisfied with it.}}\\
\end{addmargin}
\end{absolutelynopagebreak}

\begin{absolutelynopagebreak}
\setstretch{.7}
{\PaliGlossA{Saccānuppattiṃ mayaṃ bhavantaṃ gotamaṃ apucchimha, saccānuppattiṃ bhavaṃ gotamo byākāsi;}}\\
\begin{addmargin}[1em]{2em}
\setstretch{.5}
{\PaliGlossB{I’ve asked Master Gotama about the arrival at the truth, and he has answered me.}}\\
\end{addmargin}
\end{absolutelynopagebreak}

\begin{absolutelynopagebreak}
\setstretch{.7}
{\PaliGlossA{tañca panamhākaṃ ruccati ceva khamati ca tena camha attamanā.}}\\
\begin{addmargin}[1em]{2em}
\setstretch{.5}
{\PaliGlossB{I like and accept this, and am satisfied with it.}}\\
\end{addmargin}
\end{absolutelynopagebreak}

\begin{absolutelynopagebreak}
\setstretch{.7}
{\PaliGlossA{Saccānuppattiyā bahukāraṃ dhammaṃ mayaṃ bhavantaṃ gotamaṃ apucchimha, saccānuppattiyā bahukāraṃ dhammaṃ bhavaṃ gotamo byākāsi;}}\\
\begin{addmargin}[1em]{2em}
\setstretch{.5}
{\PaliGlossB{I’ve asked Master Gotama about the things that are helpful for the arrival at the truth, and he has answered me.}}\\
\end{addmargin}
\end{absolutelynopagebreak}

\begin{absolutelynopagebreak}
\setstretch{.7}
{\PaliGlossA{tañca panamhākaṃ ruccati ceva khamati ca tena camha attamanā.}}\\
\begin{addmargin}[1em]{2em}
\setstretch{.5}
{\PaliGlossB{I like and accept this, and am satisfied with it.}}\\
\end{addmargin}
\end{absolutelynopagebreak}

\begin{absolutelynopagebreak}
\setstretch{.7}
{\PaliGlossA{Yaṃyadeva ca mayaṃ bhavantaṃ gotamaṃ apucchimha taṃtadeva bhavaṃ gotamo byākāsi;}}\\
\begin{addmargin}[1em]{2em}
\setstretch{.5}
{\PaliGlossB{Whatever I have asked Master Gotama about he has answered me.}}\\
\end{addmargin}
\end{absolutelynopagebreak}

\begin{absolutelynopagebreak}
\setstretch{.7}
{\PaliGlossA{tañca panamhākaṃ ruccati ceva khamati ca tena camha attamanā.}}\\
\begin{addmargin}[1em]{2em}
\setstretch{.5}
{\PaliGlossB{I like and accept this, and am satisfied with it.}}\\
\end{addmargin}
\end{absolutelynopagebreak}

\begin{absolutelynopagebreak}
\setstretch{.7}
{\PaliGlossA{Mayañhi, bho gotama, pubbe evaṃ jānāma:}}\\
\begin{addmargin}[1em]{2em}
\setstretch{.5}
{\PaliGlossB{Master Gotama, I used to think this:}}\\
\end{addmargin}
\end{absolutelynopagebreak}

\begin{absolutelynopagebreak}
\setstretch{.7}
{\PaliGlossA{‘ke ca muṇḍakā samaṇakā ibbhā kaṇhā bandhupādāpaccā, ke ca dhammassa aññātāro’ti?}}\\
\begin{addmargin}[1em]{2em}
\setstretch{.5}
{\PaliGlossB{‘Who are these shavelings, fake ascetics, riffraff, black spawn from the feet of our Kinsman to be counted alongside those who understand the teaching?’}}\\
\end{addmargin}
\end{absolutelynopagebreak}

\begin{absolutelynopagebreak}
\setstretch{.7}
{\PaliGlossA{Ajanesi vata me bhavaṃ gotamo samaṇesu samaṇapemaṃ, samaṇesu samaṇapasādaṃ, samaṇesu samaṇagāravaṃ.}}\\
\begin{addmargin}[1em]{2em}
\setstretch{.5}
{\PaliGlossB{The Buddha has inspired me to have love, confidence, and respect for ascetics!}}\\
\end{addmargin}
\end{absolutelynopagebreak}

\vskip 0.05in
\begin{absolutelynopagebreak}
\setstretch{.7}
{\PaliGlossA{Abhikkantaṃ, bho gotama … pe …}}\\
\begin{addmargin}[1em]{2em}
\setstretch{.5}
{\PaliGlossB{Excellent, Master Gotama! …}}\\
\end{addmargin}
\end{absolutelynopagebreak}

\begin{absolutelynopagebreak}
\setstretch{.7}
{\PaliGlossA{upāsakaṃ maṃ bhavaṃ gotamo dhāretu ajjatagge pāṇupetaṃ saraṇaṃ gatan”ti.}}\\
\begin{addmargin}[1em]{2em}
\setstretch{.5}
{\PaliGlossB{From this day forth, may Master Gotama remember me as a lay follower who has gone for refuge for life.”}}\\
\end{addmargin}
\end{absolutelynopagebreak}

\begin{absolutelynopagebreak}
\setstretch{.7}
{\PaliGlossA{Caṅkīsuttaṃ niṭṭhitaṃ pañcamaṃ.}}\\
\begin{addmargin}[1em]{2em}
\setstretch{.5}
{\PaliGlossB{    -}}\\
\end{addmargin}
\end{absolutelynopagebreak}
