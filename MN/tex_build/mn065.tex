
\begin{absolutelynopagebreak}
\setstretch{.7}
{\PaliGlossA{Majjhima Nikāya 65}}\\
\begin{addmargin}[1em]{2em}
\setstretch{.5}
{\PaliGlossB{Middle Discourses 65}}\\
\end{addmargin}
\end{absolutelynopagebreak}

\begin{absolutelynopagebreak}
\setstretch{.7}
{\PaliGlossA{Bhaddālisutta}}\\
\begin{addmargin}[1em]{2em}
\setstretch{.5}
{\PaliGlossB{With Bhaddāli}}\\
\end{addmargin}
\end{absolutelynopagebreak}

\vskip 0.05in
\begin{absolutelynopagebreak}
\setstretch{.7}
{\PaliGlossA{Evaṃ me sutaṃ—}}\\
\begin{addmargin}[1em]{2em}
\setstretch{.5}
{\PaliGlossB{So I have heard.}}\\
\end{addmargin}
\end{absolutelynopagebreak}

\begin{absolutelynopagebreak}
\setstretch{.7}
{\PaliGlossA{ekaṃ samayaṃ bhagavā sāvatthiyaṃ viharati jetavane anāthapiṇḍikassa ārāme.}}\\
\begin{addmargin}[1em]{2em}
\setstretch{.5}
{\PaliGlossB{At one time the Buddha was staying near Sāvatthī in Jeta’s Grove, Anāthapiṇḍika’s monastery.}}\\
\end{addmargin}
\end{absolutelynopagebreak}

\begin{absolutelynopagebreak}
\setstretch{.7}
{\PaliGlossA{Tatra kho bhagavā bhikkhū āmantesi:}}\\
\begin{addmargin}[1em]{2em}
\setstretch{.5}
{\PaliGlossB{There the Buddha addressed the mendicants,}}\\
\end{addmargin}
\end{absolutelynopagebreak}

\begin{absolutelynopagebreak}
\setstretch{.7}
{\PaliGlossA{“bhikkhavo”ti.}}\\
\begin{addmargin}[1em]{2em}
\setstretch{.5}
{\PaliGlossB{“Mendicants!”}}\\
\end{addmargin}
\end{absolutelynopagebreak}

\begin{absolutelynopagebreak}
\setstretch{.7}
{\PaliGlossA{“Bhadante”ti te bhikkhū bhagavato paccassosuṃ.}}\\
\begin{addmargin}[1em]{2em}
\setstretch{.5}
{\PaliGlossB{“Venerable sir,” they replied.}}\\
\end{addmargin}
\end{absolutelynopagebreak}

\begin{absolutelynopagebreak}
\setstretch{.7}
{\PaliGlossA{Bhagavā etadavoca:}}\\
\begin{addmargin}[1em]{2em}
\setstretch{.5}
{\PaliGlossB{The Buddha said this:}}\\
\end{addmargin}
\end{absolutelynopagebreak}

\vskip 0.05in
\begin{absolutelynopagebreak}
\setstretch{.7}
{\PaliGlossA{“Ahaṃ kho, bhikkhave, ekāsanabhojanaṃ bhuñjāmi;}}\\
\begin{addmargin}[1em]{2em}
\setstretch{.5}
{\PaliGlossB{“Mendicants, I eat my food in one sitting per day.}}\\
\end{addmargin}
\end{absolutelynopagebreak}

\begin{absolutelynopagebreak}
\setstretch{.7}
{\PaliGlossA{ekāsanabhojanaṃ kho, ahaṃ, bhikkhave, bhuñjamāno appābādhatañca sañjānāmi appātaṅkatañca lahuṭṭhānañca balañca phāsuvihārañca.}}\\
\begin{addmargin}[1em]{2em}
\setstretch{.5}
{\PaliGlossB{Doing so, I find that I’m healthy and well, nimble, strong, and living comfortably.}}\\
\end{addmargin}
\end{absolutelynopagebreak}

\begin{absolutelynopagebreak}
\setstretch{.7}
{\PaliGlossA{Etha, tumhepi, bhikkhave, ekāsanabhojanaṃ bhuñjatha;}}\\
\begin{addmargin}[1em]{2em}
\setstretch{.5}
{\PaliGlossB{You too should eat your food in one sitting per day.}}\\
\end{addmargin}
\end{absolutelynopagebreak}

\begin{absolutelynopagebreak}
\setstretch{.7}
{\PaliGlossA{ekāsanabhojanaṃ kho, bhikkhave, tumhepi bhuñjamānā appābādhatañca sañjānissatha appātaṅkatañca lahuṭṭhānañca balañca phāsuvihārañcā”ti.}}\\
\begin{addmargin}[1em]{2em}
\setstretch{.5}
{\PaliGlossB{Doing so, you’ll find that you’re healthy and well, nimble, strong, and living comfortably.”}}\\
\end{addmargin}
\end{absolutelynopagebreak}

\vskip 0.05in
\begin{absolutelynopagebreak}
\setstretch{.7}
{\PaliGlossA{Evaṃ vutte, āyasmā bhaddāli bhagavantaṃ etadavoca:}}\\
\begin{addmargin}[1em]{2em}
\setstretch{.5}
{\PaliGlossB{When he said this, Venerable Bhaddāli said to the Buddha,}}\\
\end{addmargin}
\end{absolutelynopagebreak}

\begin{absolutelynopagebreak}
\setstretch{.7}
{\PaliGlossA{“ahaṃ kho, bhante, na ussahāmi ekāsanabhojanaṃ bhuñjituṃ;}}\\
\begin{addmargin}[1em]{2em}
\setstretch{.5}
{\PaliGlossB{“Sir, I’m not going to try to eat my food in one sitting per day.}}\\
\end{addmargin}
\end{absolutelynopagebreak}

\begin{absolutelynopagebreak}
\setstretch{.7}
{\PaliGlossA{ekāsanabhojanañhi me, bhante, bhuñjato siyā kukkuccaṃ, siyā vippaṭisāro”ti.}}\\
\begin{addmargin}[1em]{2em}
\setstretch{.5}
{\PaliGlossB{For when eating once a day I might feel remorse and regret.”}}\\
\end{addmargin}
\end{absolutelynopagebreak}

\vskip 0.05in
\begin{absolutelynopagebreak}
\setstretch{.7}
{\PaliGlossA{“Tena hi tvaṃ, bhaddāli, yattha nimantito assasi tattha ekadesaṃ bhuñjitvā ekadesaṃ nīharitvāpi bhuñjeyyāsi.}}\\
\begin{addmargin}[1em]{2em}
\setstretch{.5}
{\PaliGlossB{“Well then, Bhaddāli, eat one part of the meal in the place where you’re invited, and bring the rest back to eat.}}\\
\end{addmargin}
\end{absolutelynopagebreak}

\begin{absolutelynopagebreak}
\setstretch{.7}
{\PaliGlossA{Evampi kho tvaṃ, bhaddāli, bhuñjamāno ekāsano yāpessasī”ti.}}\\
\begin{addmargin}[1em]{2em}
\setstretch{.5}
{\PaliGlossB{Eating this way, too, you will sustain yourself.”}}\\
\end{addmargin}
\end{absolutelynopagebreak}

\begin{absolutelynopagebreak}
\setstretch{.7}
{\PaliGlossA{“Evampi kho ahaṃ, bhante, na ussahāmi bhuñjituṃ;}}\\
\begin{addmargin}[1em]{2em}
\setstretch{.5}
{\PaliGlossB{“Sir, I’m not going to try to eat that way, either.}}\\
\end{addmargin}
\end{absolutelynopagebreak}

\begin{absolutelynopagebreak}
\setstretch{.7}
{\PaliGlossA{evampi hi me, bhante, bhuñjato siyā kukkuccaṃ, siyā vippaṭisāro”ti.}}\\
\begin{addmargin}[1em]{2em}
\setstretch{.5}
{\PaliGlossB{For when eating that way I might also feel remorse and regret.”}}\\
\end{addmargin}
\end{absolutelynopagebreak}

\begin{absolutelynopagebreak}
\setstretch{.7}
{\PaliGlossA{Atha kho āyasmā bhaddāli bhagavatā sikkhāpade paññāpiyamāne bhikkhusaṃghe sikkhaṃ samādiyamāne anussāhaṃ pavedesi.}}\\
\begin{addmargin}[1em]{2em}
\setstretch{.5}
{\PaliGlossB{Then, as this rule was being laid down by the Buddha and the Saṅgha was undertaking it, Bhaddāli announced he would not try to keep it.}}\\
\end{addmargin}
\end{absolutelynopagebreak}

\begin{absolutelynopagebreak}
\setstretch{.7}
{\PaliGlossA{Atha kho āyasmā bhaddāli sabbaṃ taṃ temāsaṃ na bhagavato sammukhībhāvaṃ adāsi, yathā taṃ satthusāsane sikkhāya aparipūrakārī.}}\\
\begin{addmargin}[1em]{2em}
\setstretch{.5}
{\PaliGlossB{Then for the whole of that three months Bhaddāli did not present himself in the presence of the Buddha, as happens when someone doesn’t fulfill the training according to the Teacher’s instructions.}}\\
\end{addmargin}
\end{absolutelynopagebreak}

\vskip 0.05in
\begin{absolutelynopagebreak}
\setstretch{.7}
{\PaliGlossA{Tena kho pana samayena sambahulā bhikkhū bhagavato cīvarakammaṃ karonti—}}\\
\begin{addmargin}[1em]{2em}
\setstretch{.5}
{\PaliGlossB{At that time several mendicants were making a robe for the Buddha, thinking that}}\\
\end{addmargin}
\end{absolutelynopagebreak}

\begin{absolutelynopagebreak}
\setstretch{.7}
{\PaliGlossA{niṭṭhitacīvaro bhagavā temāsaccayena cārikaṃ pakkamissatīti.}}\\
\begin{addmargin}[1em]{2em}
\setstretch{.5}
{\PaliGlossB{when his robe was finished and the three months of the rains residence had passed the Buddha would set out wandering.}}\\
\end{addmargin}
\end{absolutelynopagebreak}

\vskip 0.05in
\begin{absolutelynopagebreak}
\setstretch{.7}
{\PaliGlossA{Atha kho āyasmā bhaddāli yena te bhikkhū tenupasaṅkami; upasaṅkamitvā tehi bhikkhūhi saddhiṃ sammodi.}}\\
\begin{addmargin}[1em]{2em}
\setstretch{.5}
{\PaliGlossB{Then Bhaddāli went up to those mendicants, and exchanged greetings with them.}}\\
\end{addmargin}
\end{absolutelynopagebreak}

\begin{absolutelynopagebreak}
\setstretch{.7}
{\PaliGlossA{Sammodanīyaṃ kathaṃ sāraṇīyaṃ vītisāretvā ekamantaṃ nisīdi. Ekamantaṃ nisinnaṃ kho āyasmantaṃ bhaddāliṃ te bhikkhū etadavocuṃ:}}\\
\begin{addmargin}[1em]{2em}
\setstretch{.5}
{\PaliGlossB{When the greetings and polite conversation were over, he sat down to one side. The mendicants said to Bhaddāli,}}\\
\end{addmargin}
\end{absolutelynopagebreak}

\begin{absolutelynopagebreak}
\setstretch{.7}
{\PaliGlossA{“idaṃ kho, āvuso bhaddāli, bhagavato cīvarakammaṃ karīyati.}}\\
\begin{addmargin}[1em]{2em}
\setstretch{.5}
{\PaliGlossB{“Reverend Bhaddāli, this robe is being made for the Buddha.}}\\
\end{addmargin}
\end{absolutelynopagebreak}

\begin{absolutelynopagebreak}
\setstretch{.7}
{\PaliGlossA{Niṭṭhitacīvaro bhagavā temāsaccayena cārikaṃ pakkamissati.}}\\
\begin{addmargin}[1em]{2em}
\setstretch{.5}
{\PaliGlossB{When it’s finished and the three months of the rains residence have passed the Buddha will set out wandering.}}\\
\end{addmargin}
\end{absolutelynopagebreak}

\begin{absolutelynopagebreak}
\setstretch{.7}
{\PaliGlossA{Iṅghāvuso bhaddāli, etaṃ dosakaṃ sādhukaṃ manasi karohi, mā te pacchā dukkarataraṃ ahosī”ti.}}\\
\begin{addmargin}[1em]{2em}
\setstretch{.5}
{\PaliGlossB{Come on, Bhaddāli, learn your lesson. Don’t make it hard for yourself later on.”}}\\
\end{addmargin}
\end{absolutelynopagebreak}

\vskip 0.05in
\begin{absolutelynopagebreak}
\setstretch{.7}
{\PaliGlossA{“Evamāvuso”ti kho āyasmā bhaddāli tesaṃ bhikkhūnaṃ paṭissutvā yena bhagavā tenupasaṅkami; upasaṅkamitvā bhagavantaṃ abhivādetvā ekamantaṃ nisīdi. Ekamantaṃ nisinno kho āyasmā bhaddāli bhagavantaṃ etadavoca:}}\\
\begin{addmargin}[1em]{2em}
\setstretch{.5}
{\PaliGlossB{“Yes, reverends,” Bhaddāli replied. He went to the Buddha, bowed, sat down to one side, and said to him,}}\\
\end{addmargin}
\end{absolutelynopagebreak}

\begin{absolutelynopagebreak}
\setstretch{.7}
{\PaliGlossA{“accayo maṃ, bhante, accagamā yathābālaṃ yathāmūḷhaṃ yathāakusalaṃ, yohaṃ bhagavatā sikkhāpade paññāpiyamāne bhikkhusaṃghe sikkhaṃ samādiyamāne anussāhaṃ pavedesiṃ.}}\\
\begin{addmargin}[1em]{2em}
\setstretch{.5}
{\PaliGlossB{“I have made a mistake, sir. It was foolish, stupid, and unskillful of me that, as this rule was being laid down by the Buddha and the Saṅgha was undertaking it, I announced I would not try to keep it.}}\\
\end{addmargin}
\end{absolutelynopagebreak}

\begin{absolutelynopagebreak}
\setstretch{.7}
{\PaliGlossA{Tassa me, bhante, bhagavā accayaṃ accayato paṭiggaṇhātu āyatiṃ saṃvarāyā”ti.}}\\
\begin{addmargin}[1em]{2em}
\setstretch{.5}
{\PaliGlossB{Please, sir, accept my mistake for what it is, so I will restrain myself in future.”}}\\
\end{addmargin}
\end{absolutelynopagebreak}

\vskip 0.05in
\begin{absolutelynopagebreak}
\setstretch{.7}
{\PaliGlossA{“Taggha tvaṃ, bhaddāli, accayo accagamā yathābālaṃ yathāmūḷhaṃ yathāakusalaṃ, yaṃ tvaṃ mayā sikkhāpade paññāpiyamāne bhikkhusaṅghe sikkhaṃ samādiyamāne anussāhaṃ pavedesi.}}\\
\begin{addmargin}[1em]{2em}
\setstretch{.5}
{\PaliGlossB{“Indeed, Bhaddāli, you made a mistake. It was foolish, stupid, and unskillful of you that, as this rule was being laid down by the Buddha and the Saṅgha was undertaking it, you announced you would not try to keep it.}}\\
\end{addmargin}
\end{absolutelynopagebreak}

\vskip 0.05in
\begin{absolutelynopagebreak}
\setstretch{.7}
{\PaliGlossA{Samayopi kho te, bhaddāli, appaṭividdho ahosi:}}\\
\begin{addmargin}[1em]{2em}
\setstretch{.5}
{\PaliGlossB{And you didn’t realize this situation:}}\\
\end{addmargin}
\end{absolutelynopagebreak}

\begin{absolutelynopagebreak}
\setstretch{.7}
{\PaliGlossA{‘bhagavā kho sāvatthiyaṃ viharati, bhagavāpi maṃ jānissati—}}\\
\begin{addmargin}[1em]{2em}
\setstretch{.5}
{\PaliGlossB{‘The Buddha is staying in Sāvatthī, and he’ll know me}}\\
\end{addmargin}
\end{absolutelynopagebreak}

\begin{absolutelynopagebreak}
\setstretch{.7}
{\PaliGlossA{bhaddāli nāma bhikkhu satthusāsane sikkhāya aparipūrakārī’ti.}}\\
\begin{addmargin}[1em]{2em}
\setstretch{.5}
{\PaliGlossB{as the mendicant named Bhaddāli who doesn’t fulfill the training according to the Teacher’s instructions.’}}\\
\end{addmargin}
\end{absolutelynopagebreak}

\begin{absolutelynopagebreak}
\setstretch{.7}
{\PaliGlossA{Ayampi kho te, bhaddāli, samayo appaṭividdho ahosi.}}\\
\begin{addmargin}[1em]{2em}
\setstretch{.5}
{\PaliGlossB{    -}}\\
\end{addmargin}
\end{absolutelynopagebreak}

\begin{absolutelynopagebreak}
\setstretch{.7}
{\PaliGlossA{Samayopi kho te, bhaddāli, appaṭividdho ahosi:}}\\
\begin{addmargin}[1em]{2em}
\setstretch{.5}
{\PaliGlossB{And you didn’t realize this situation:}}\\
\end{addmargin}
\end{absolutelynopagebreak}

\begin{absolutelynopagebreak}
\setstretch{.7}
{\PaliGlossA{‘sambahulā kho bhikkhū sāvatthiyaṃ vassaṃ upagatā, tepi maṃ jānissanti—}}\\
\begin{addmargin}[1em]{2em}
\setstretch{.5}
{\PaliGlossB{‘Several monks have commenced the rains retreat in Sāvatthī …}}\\
\end{addmargin}
\end{absolutelynopagebreak}

\begin{absolutelynopagebreak}
\setstretch{.7}
{\PaliGlossA{bhaddāli nāma bhikkhu satthusāsane sikkhāya aparipūrakārī’ti.}}\\
\begin{addmargin}[1em]{2em}
\setstretch{.5}
{\PaliGlossB{    -}}\\
\end{addmargin}
\end{absolutelynopagebreak}

\begin{absolutelynopagebreak}
\setstretch{.7}
{\PaliGlossA{Ayampi kho te, bhaddāli, samayo appaṭividdho ahosi.}}\\
\begin{addmargin}[1em]{2em}
\setstretch{.5}
{\PaliGlossB{    -}}\\
\end{addmargin}
\end{absolutelynopagebreak}

\begin{absolutelynopagebreak}
\setstretch{.7}
{\PaliGlossA{Samayopi kho te, bhaddāli, appaṭividdho ahosi:}}\\
\begin{addmargin}[1em]{2em}
\setstretch{.5}
{\PaliGlossB{    -}}\\
\end{addmargin}
\end{absolutelynopagebreak}

\begin{absolutelynopagebreak}
\setstretch{.7}
{\PaliGlossA{‘sambahulā kho bhikkhuniyo sāvatthiyaṃ vassaṃ upagatā, tāpi maṃ jānissanti—}}\\
\begin{addmargin}[1em]{2em}
\setstretch{.5}
{\PaliGlossB{several nuns have commenced the rains retreat in Sāvatthī …}}\\
\end{addmargin}
\end{absolutelynopagebreak}

\begin{absolutelynopagebreak}
\setstretch{.7}
{\PaliGlossA{bhaddāli nāma bhikkhu satthusāsane sikkhāya aparipūrakārī’ti.}}\\
\begin{addmargin}[1em]{2em}
\setstretch{.5}
{\PaliGlossB{    -}}\\
\end{addmargin}
\end{absolutelynopagebreak}

\begin{absolutelynopagebreak}
\setstretch{.7}
{\PaliGlossA{Ayampi kho te, bhaddāli, samayo appaṭividdho ahosi.}}\\
\begin{addmargin}[1em]{2em}
\setstretch{.5}
{\PaliGlossB{    -}}\\
\end{addmargin}
\end{absolutelynopagebreak}

\begin{absolutelynopagebreak}
\setstretch{.7}
{\PaliGlossA{Samayopi kho te, bhaddāli, appaṭividdho ahosi:}}\\
\begin{addmargin}[1em]{2em}
\setstretch{.5}
{\PaliGlossB{    -}}\\
\end{addmargin}
\end{absolutelynopagebreak}

\begin{absolutelynopagebreak}
\setstretch{.7}
{\PaliGlossA{‘sambahulā kho upāsakā sāvatthiyaṃ paṭivasanti, tepi maṃ jānissanti—}}\\
\begin{addmargin}[1em]{2em}
\setstretch{.5}
{\PaliGlossB{several laymen reside in Sāvatthī …}}\\
\end{addmargin}
\end{absolutelynopagebreak}

\begin{absolutelynopagebreak}
\setstretch{.7}
{\PaliGlossA{bhaddāli nāma bhikkhu satthusāsane sikkhāya aparipūrakārī’ti.}}\\
\begin{addmargin}[1em]{2em}
\setstretch{.5}
{\PaliGlossB{    -}}\\
\end{addmargin}
\end{absolutelynopagebreak}

\begin{absolutelynopagebreak}
\setstretch{.7}
{\PaliGlossA{Ayampi kho te, bhaddāli, samayo appaṭividdho ahosi.}}\\
\begin{addmargin}[1em]{2em}
\setstretch{.5}
{\PaliGlossB{    -}}\\
\end{addmargin}
\end{absolutelynopagebreak}

\begin{absolutelynopagebreak}
\setstretch{.7}
{\PaliGlossA{Samayopi kho te, bhaddāli, appaṭividdho ahosi:}}\\
\begin{addmargin}[1em]{2em}
\setstretch{.5}
{\PaliGlossB{    -}}\\
\end{addmargin}
\end{absolutelynopagebreak}

\begin{absolutelynopagebreak}
\setstretch{.7}
{\PaliGlossA{‘sambahulā kho upāsikā sāvatthiyaṃ paṭivasanti, tāpi maṃ jānissanti—}}\\
\begin{addmargin}[1em]{2em}
\setstretch{.5}
{\PaliGlossB{several laywomen reside in Sāvatthī, and they’ll know me}}\\
\end{addmargin}
\end{absolutelynopagebreak}

\begin{absolutelynopagebreak}
\setstretch{.7}
{\PaliGlossA{bhaddāli nāma bhikkhu satthusāsane sikkhāya aparipūrakārī’ti.}}\\
\begin{addmargin}[1em]{2em}
\setstretch{.5}
{\PaliGlossB{as the mendicant named Bhaddāli who doesn’t fulfill the training according to the Teacher’s instructions. …}}\\
\end{addmargin}
\end{absolutelynopagebreak}

\begin{absolutelynopagebreak}
\setstretch{.7}
{\PaliGlossA{Ayampi kho te, bhaddāli, samayo appaṭividdho ahosi.}}\\
\begin{addmargin}[1em]{2em}
\setstretch{.5}
{\PaliGlossB{    -}}\\
\end{addmargin}
\end{absolutelynopagebreak}

\begin{absolutelynopagebreak}
\setstretch{.7}
{\PaliGlossA{Samayopi kho te, bhaddāli, appaṭividdho ahosi:}}\\
\begin{addmargin}[1em]{2em}
\setstretch{.5}
{\PaliGlossB{    -}}\\
\end{addmargin}
\end{absolutelynopagebreak}

\begin{absolutelynopagebreak}
\setstretch{.7}
{\PaliGlossA{‘sambahulā kho nānātitthiyā samaṇabrāhmaṇā sāvatthiyaṃ vassaṃ upagatā, tepi maṃ jānissanti—}}\\
\begin{addmargin}[1em]{2em}
\setstretch{.5}
{\PaliGlossB{Several ascetics and brahmins who follow various other paths have commenced the rains retreat in Sāvatthī, and they’ll know me}}\\
\end{addmargin}
\end{absolutelynopagebreak}

\begin{absolutelynopagebreak}
\setstretch{.7}
{\PaliGlossA{bhaddāli nāma bhikkhu samaṇassa gotamassa sāvako theraññataro bhikkhu sāsane sikkhāya aparipūrakārī’ti.}}\\
\begin{addmargin}[1em]{2em}
\setstretch{.5}
{\PaliGlossB{as the mendicant named Bhaddāli, one of the senior disciples of Gotama, who doesn’t fulfill the training according to the Teacher’s instructions.’}}\\
\end{addmargin}
\end{absolutelynopagebreak}

\begin{absolutelynopagebreak}
\setstretch{.7}
{\PaliGlossA{Ayampi kho te, bhaddāli, samayo appaṭividdho ahosī”ti.}}\\
\begin{addmargin}[1em]{2em}
\setstretch{.5}
{\PaliGlossB{You also didn’t realize this situation.”}}\\
\end{addmargin}
\end{absolutelynopagebreak}

\vskip 0.05in
\begin{absolutelynopagebreak}
\setstretch{.7}
{\PaliGlossA{“Accayo maṃ, bhante, accagamā yathābālaṃ yathāmūḷhaṃ yathāakusalaṃ, yohaṃ bhagavatā sikkhāpade paññāpiyamāne bhikkhusaṅghe sikkhaṃ samādiyamāne anussāhaṃ pavedesiṃ.}}\\
\begin{addmargin}[1em]{2em}
\setstretch{.5}
{\PaliGlossB{“I made a mistake, sir. It was foolish, stupid, and unskillful of me that, as this rule was being laid down by the Buddha and the Saṅgha was undertaking it, I announced I would not try to keep it.}}\\
\end{addmargin}
\end{absolutelynopagebreak}

\begin{absolutelynopagebreak}
\setstretch{.7}
{\PaliGlossA{Tassa me, bhante, bhagavā accayaṃ accayato paṭiggaṇhātu āyatiṃ saṃvarāyā”ti.}}\\
\begin{addmargin}[1em]{2em}
\setstretch{.5}
{\PaliGlossB{Please, sir, accept my mistake for what it is, so I will restrain myself in future.”}}\\
\end{addmargin}
\end{absolutelynopagebreak}

\begin{absolutelynopagebreak}
\setstretch{.7}
{\PaliGlossA{“Taggha tvaṃ, bhaddāli, accayo accagamā yathābālaṃ yathāmūḷhaṃ yathāakusalaṃ, yaṃ tvaṃ mayā sikkhāpade paññāpiyamāne bhikkhusaṅghe sikkhaṃ samādiyamāne anussāhaṃ pavedesi.}}\\
\begin{addmargin}[1em]{2em}
\setstretch{.5}
{\PaliGlossB{“Indeed, Bhaddāli, you made a mistake. It was foolish, stupid, and unskillful of you that, as this rule was being laid down by the Buddha and the Saṅgha was undertaking it, you announced you would not try to keep it.}}\\
\end{addmargin}
\end{absolutelynopagebreak}

\vskip 0.05in
\begin{absolutelynopagebreak}
\setstretch{.7}
{\PaliGlossA{Taṃ kiṃ maññasi, bhaddāli,}}\\
\begin{addmargin}[1em]{2em}
\setstretch{.5}
{\PaliGlossB{What do you think, Bhaddāli?}}\\
\end{addmargin}
\end{absolutelynopagebreak}

\begin{absolutelynopagebreak}
\setstretch{.7}
{\PaliGlossA{idhassa bhikkhu ubhatobhāgavimutto, tamahaṃ evaṃ vadeyyaṃ:}}\\
\begin{addmargin}[1em]{2em}
\setstretch{.5}
{\PaliGlossB{Suppose I was to say this to a mendicant who is freed both ways:}}\\
\end{addmargin}
\end{absolutelynopagebreak}

\begin{absolutelynopagebreak}
\setstretch{.7}
{\PaliGlossA{‘ehi me tvaṃ, bhikkhu, paṅke saṅkamo hohī’ti, api nu kho so saṅkameyya vā aññena vā kāyaṃ sannāmeyya, ‘no’ti vā vadeyyā”ti?}}\\
\begin{addmargin}[1em]{2em}
\setstretch{.5}
{\PaliGlossB{‘Please, mendicant, be a bridge for me to cross over the mud.’ Would they cross over themselves, or struggle to get out of it, or just say no?”}}\\
\end{addmargin}
\end{absolutelynopagebreak}

\begin{absolutelynopagebreak}
\setstretch{.7}
{\PaliGlossA{“No hetaṃ, bhante”.}}\\
\begin{addmargin}[1em]{2em}
\setstretch{.5}
{\PaliGlossB{“No, sir.”}}\\
\end{addmargin}
\end{absolutelynopagebreak}

\begin{absolutelynopagebreak}
\setstretch{.7}
{\PaliGlossA{“Taṃ kiṃ maññasi, bhaddāli,}}\\
\begin{addmargin}[1em]{2em}
\setstretch{.5}
{\PaliGlossB{“What do you think, Bhaddāli?}}\\
\end{addmargin}
\end{absolutelynopagebreak}

\begin{absolutelynopagebreak}
\setstretch{.7}
{\PaliGlossA{idhassa bhikkhu paññāvimutto …}}\\
\begin{addmargin}[1em]{2em}
\setstretch{.5}
{\PaliGlossB{Suppose I was to say the same thing to a mendicant who is freed by wisdom,}}\\
\end{addmargin}
\end{absolutelynopagebreak}

\begin{absolutelynopagebreak}
\setstretch{.7}
{\PaliGlossA{kāyasakkhi …}}\\
\begin{addmargin}[1em]{2em}
\setstretch{.5}
{\PaliGlossB{or a personal witness,}}\\
\end{addmargin}
\end{absolutelynopagebreak}

\begin{absolutelynopagebreak}
\setstretch{.7}
{\PaliGlossA{diṭṭhippatto …}}\\
\begin{addmargin}[1em]{2em}
\setstretch{.5}
{\PaliGlossB{or attained to view,}}\\
\end{addmargin}
\end{absolutelynopagebreak}

\begin{absolutelynopagebreak}
\setstretch{.7}
{\PaliGlossA{saddhāvimutto …}}\\
\begin{addmargin}[1em]{2em}
\setstretch{.5}
{\PaliGlossB{or freed by faith,}}\\
\end{addmargin}
\end{absolutelynopagebreak}

\begin{absolutelynopagebreak}
\setstretch{.7}
{\PaliGlossA{dhammānusārī …}}\\
\begin{addmargin}[1em]{2em}
\setstretch{.5}
{\PaliGlossB{or a follower of the teachings,}}\\
\end{addmargin}
\end{absolutelynopagebreak}

\begin{absolutelynopagebreak}
\setstretch{.7}
{\PaliGlossA{saddhānusārī, tamahaṃ evaṃ vadeyyaṃ:}}\\
\begin{addmargin}[1em]{2em}
\setstretch{.5}
{\PaliGlossB{or a follower by faith:}}\\
\end{addmargin}
\end{absolutelynopagebreak}

\begin{absolutelynopagebreak}
\setstretch{.7}
{\PaliGlossA{‘ehi me tvaṃ, bhikkhu, paṅke saṅkamo hohī’ti, api nu kho so saṅkameyya vā aññena vā kāyaṃ sannāmeyya, ‘no’ti vā vadeyyā”ti?}}\\
\begin{addmargin}[1em]{2em}
\setstretch{.5}
{\PaliGlossB{‘Please, mendicant, be a bridge for me to cross over the mud.’ Would they cross over themselves, or struggle to get out of it, or just say no?”}}\\
\end{addmargin}
\end{absolutelynopagebreak}

\begin{absolutelynopagebreak}
\setstretch{.7}
{\PaliGlossA{“No hetaṃ, bhante”.}}\\
\begin{addmargin}[1em]{2em}
\setstretch{.5}
{\PaliGlossB{“No, sir.”}}\\
\end{addmargin}
\end{absolutelynopagebreak}

\vskip 0.05in
\begin{absolutelynopagebreak}
\setstretch{.7}
{\PaliGlossA{“Taṃ kiṃ maññasi, bhaddāli,}}\\
\begin{addmargin}[1em]{2em}
\setstretch{.5}
{\PaliGlossB{“What do you think, Bhaddāli?}}\\
\end{addmargin}
\end{absolutelynopagebreak}

\begin{absolutelynopagebreak}
\setstretch{.7}
{\PaliGlossA{api nu tvaṃ, bhaddāli, tasmiṃ samaye ubhatobhāgavimutto vā hosi paññāvimutto vā kāyasakkhi vā diṭṭhippatto vā saddhāvimutto vā dhammānusārī vā saddhānusārī vā”ti?}}\\
\begin{addmargin}[1em]{2em}
\setstretch{.5}
{\PaliGlossB{At that time were you freed both ways, freed by wisdom, a personal witness, attained to view, freed by faith, a follower of the teachings, or a follower by faith?”}}\\
\end{addmargin}
\end{absolutelynopagebreak}

\begin{absolutelynopagebreak}
\setstretch{.7}
{\PaliGlossA{“No hetaṃ, bhante”.}}\\
\begin{addmargin}[1em]{2em}
\setstretch{.5}
{\PaliGlossB{“No, sir.”}}\\
\end{addmargin}
\end{absolutelynopagebreak}

\begin{absolutelynopagebreak}
\setstretch{.7}
{\PaliGlossA{“Nanu tvaṃ, bhaddāli, tasmiṃ samaye ritto tuccho aparaddho”ti?}}\\
\begin{addmargin}[1em]{2em}
\setstretch{.5}
{\PaliGlossB{“Weren’t you void, hollow, and mistaken?”}}\\
\end{addmargin}
\end{absolutelynopagebreak}

\vskip 0.05in
\begin{absolutelynopagebreak}
\setstretch{.7}
{\PaliGlossA{“Evaṃ, bhante.}}\\
\begin{addmargin}[1em]{2em}
\setstretch{.5}
{\PaliGlossB{“Yes, sir.”}}\\
\end{addmargin}
\end{absolutelynopagebreak}

\begin{absolutelynopagebreak}
\setstretch{.7}
{\PaliGlossA{Accayo maṃ, bhante, accagamā yathābālaṃ yathāmūḷhaṃ yathāakusalaṃ, yohaṃ bhagavatā sikkhāpade paññāpiyamāne bhikkhusaṅghe sikkhaṃ samādiyamāne anussāhaṃ pavedesiṃ.}}\\
\begin{addmargin}[1em]{2em}
\setstretch{.5}
{\PaliGlossB{“I made a mistake, sir. …}}\\
\end{addmargin}
\end{absolutelynopagebreak}

\begin{absolutelynopagebreak}
\setstretch{.7}
{\PaliGlossA{Tassa me, bhante, bhagavā accayaṃ accayato paṭiggaṇhātu āyatiṃ saṃvarāyā”ti.}}\\
\begin{addmargin}[1em]{2em}
\setstretch{.5}
{\PaliGlossB{Please, sir, accept my mistake for what it is, so I will restrain myself in future.”}}\\
\end{addmargin}
\end{absolutelynopagebreak}

\begin{absolutelynopagebreak}
\setstretch{.7}
{\PaliGlossA{“Taggha tvaṃ, bhaddāli, accayo accagamā yathābālaṃ yathāmūḷhaṃ yathāakusalaṃ, yaṃ tvaṃ mayā sikkhāpade paññāpiyamāne bhikkhusaṅghe sikkhaṃ samādiyamāne anussāhaṃ pavedesi.}}\\
\begin{addmargin}[1em]{2em}
\setstretch{.5}
{\PaliGlossB{“Indeed, Bhaddāli, you made a mistake. …}}\\
\end{addmargin}
\end{absolutelynopagebreak}

\begin{absolutelynopagebreak}
\setstretch{.7}
{\PaliGlossA{Yato ca kho tvaṃ, bhaddāli, accayaṃ accayato disvā yathādhammaṃ paṭikarosi, taṃ te mayaṃ paṭiggaṇhāma.}}\\
\begin{addmargin}[1em]{2em}
\setstretch{.5}
{\PaliGlossB{But since you have recognized your mistake for what it is, and have dealt with it properly, I accept it.}}\\
\end{addmargin}
\end{absolutelynopagebreak}

\begin{absolutelynopagebreak}
\setstretch{.7}
{\PaliGlossA{Vuddhihesā, bhaddāli, ariyassa vinaye yo accayaṃ accayato disvā yathādhammaṃ paṭikaroti, āyatiṃ saṃvaraṃ āpajjati.}}\\
\begin{addmargin}[1em]{2em}
\setstretch{.5}
{\PaliGlossB{For it is growth in the training of the noble one to recognize a mistake for what it is, deal with it properly, and commit to restraint in the future.}}\\
\end{addmargin}
\end{absolutelynopagebreak}

\vskip 0.05in
\begin{absolutelynopagebreak}
\setstretch{.7}
{\PaliGlossA{Idha, bhaddāli, ekacco bhikkhu satthusāsane sikkhāya aparipūrakārī hoti.}}\\
\begin{addmargin}[1em]{2em}
\setstretch{.5}
{\PaliGlossB{Bhaddāli, take a mendicant who doesn’t fulfill the training according to the Teacher’s instructions.}}\\
\end{addmargin}
\end{absolutelynopagebreak}

\begin{absolutelynopagebreak}
\setstretch{.7}
{\PaliGlossA{Tassa evaṃ hoti:}}\\
\begin{addmargin}[1em]{2em}
\setstretch{.5}
{\PaliGlossB{They think,}}\\
\end{addmargin}
\end{absolutelynopagebreak}

\begin{absolutelynopagebreak}
\setstretch{.7}
{\PaliGlossA{‘yannūnāhaṃ vivittaṃ senāsanaṃ bhajeyyaṃ araññaṃ rukkhamūlaṃ pabbataṃ kandaraṃ giriguhaṃ susānaṃ vanapatthaṃ abbhokāsaṃ palālapuñjaṃ.}}\\
\begin{addmargin}[1em]{2em}
\setstretch{.5}
{\PaliGlossB{‘Why don’t I frequent a secluded lodging—a wilderness, the root of a tree, a hill, a ravine, a mountain cave, a charnel ground, a forest, the open air, a heap of straw.}}\\
\end{addmargin}
\end{absolutelynopagebreak}

\begin{absolutelynopagebreak}
\setstretch{.7}
{\PaliGlossA{Appeva nāmāhaṃ uttari manussadhammā alamariyañāṇadassanavisesaṃ sacchikareyyan’ti.}}\\
\begin{addmargin}[1em]{2em}
\setstretch{.5}
{\PaliGlossB{Hopefully I’ll realize a superhuman distinction in knowledge and vision worthy of the noble ones.’}}\\
\end{addmargin}
\end{absolutelynopagebreak}

\begin{absolutelynopagebreak}
\setstretch{.7}
{\PaliGlossA{So vivittaṃ senāsanaṃ bhajati araññaṃ rukkhamūlaṃ pabbataṃ kandaraṃ giriguhaṃ susānaṃ vanapatthaṃ abbhokāsaṃ palālapuñjaṃ.}}\\
\begin{addmargin}[1em]{2em}
\setstretch{.5}
{\PaliGlossB{So they frequent a secluded lodging.}}\\
\end{addmargin}
\end{absolutelynopagebreak}

\begin{absolutelynopagebreak}
\setstretch{.7}
{\PaliGlossA{Tassa tathāvūpakaṭṭhassa viharato satthāpi upavadati, anuviccapi viññū sabrahmacārī upavadanti, devatāpi upavadanti, attāpi attānaṃ upavadati.}}\\
\begin{addmargin}[1em]{2em}
\setstretch{.5}
{\PaliGlossB{While they’re living withdrawn, they’re reprimanded by the Teacher, by sensible spiritual companions after examination, by deities, and by themselves.}}\\
\end{addmargin}
\end{absolutelynopagebreak}

\begin{absolutelynopagebreak}
\setstretch{.7}
{\PaliGlossA{So satthārāpi upavadito, anuviccapi viññūhi sabrahmacārīhi upavadito, devatāhipi upavadito, attanāpi attānaṃ upavadito na uttari manussadhammā alamariyañāṇadassanavisesaṃ sacchikaroti.}}\\
\begin{addmargin}[1em]{2em}
\setstretch{.5}
{\PaliGlossB{Being reprimanded in this way, they don’t realize any superhuman distinction in knowledge and vision worthy of the noble ones.}}\\
\end{addmargin}
\end{absolutelynopagebreak}

\begin{absolutelynopagebreak}
\setstretch{.7}
{\PaliGlossA{Taṃ kissa hetu?}}\\
\begin{addmargin}[1em]{2em}
\setstretch{.5}
{\PaliGlossB{Why is that?}}\\
\end{addmargin}
\end{absolutelynopagebreak}

\begin{absolutelynopagebreak}
\setstretch{.7}
{\PaliGlossA{Evañhi taṃ, bhaddāli, hoti yathā taṃ satthusāsane sikkhāya aparipūrakārissa.}}\\
\begin{addmargin}[1em]{2em}
\setstretch{.5}
{\PaliGlossB{Because that’s how it is when someone doesn’t fulfill the training according to the Teacher’s instructions.}}\\
\end{addmargin}
\end{absolutelynopagebreak}

\vskip 0.05in
\begin{absolutelynopagebreak}
\setstretch{.7}
{\PaliGlossA{Idha pana, bhaddāli, ekacco bhikkhu satthusāsane sikkhāya paripūrakārī hoti.}}\\
\begin{addmargin}[1em]{2em}
\setstretch{.5}
{\PaliGlossB{But take a mendicant who does fulfill the training according to the Teacher’s instructions.}}\\
\end{addmargin}
\end{absolutelynopagebreak}

\begin{absolutelynopagebreak}
\setstretch{.7}
{\PaliGlossA{Tassa evaṃ hoti:}}\\
\begin{addmargin}[1em]{2em}
\setstretch{.5}
{\PaliGlossB{They think,}}\\
\end{addmargin}
\end{absolutelynopagebreak}

\begin{absolutelynopagebreak}
\setstretch{.7}
{\PaliGlossA{‘yannūnāhaṃ vivittaṃ senāsanaṃ bhajeyyaṃ araññaṃ rukkhamūlaṃ pabbataṃ kandaraṃ giriguhaṃ susānaṃ vanapatthaṃ abbhokāsaṃ palālapuñjaṃ.}}\\
\begin{addmargin}[1em]{2em}
\setstretch{.5}
{\PaliGlossB{‘Why don’t I frequent a secluded lodging—a wilderness, the root of a tree, a hill, a ravine, a mountain cave, a charnel ground, a forest, the open air, a heap of straw.}}\\
\end{addmargin}
\end{absolutelynopagebreak}

\begin{absolutelynopagebreak}
\setstretch{.7}
{\PaliGlossA{Appeva nāmāhaṃ uttari manussadhammā alamariyañāṇadassanavisesaṃ sacchikareyyan’ti.}}\\
\begin{addmargin}[1em]{2em}
\setstretch{.5}
{\PaliGlossB{Hopefully I’ll realize a superhuman distinction in knowledge and vision worthy of the noble ones.’}}\\
\end{addmargin}
\end{absolutelynopagebreak}

\begin{absolutelynopagebreak}
\setstretch{.7}
{\PaliGlossA{So vivittaṃ senāsanaṃ bhajati araññaṃ rukkhamūlaṃ pabbataṃ kandaraṃ giriguhaṃ susānaṃ vanapatthaṃ abbhokāsaṃ palālapuñjaṃ.}}\\
\begin{addmargin}[1em]{2em}
\setstretch{.5}
{\PaliGlossB{They frequent a secluded lodging—a wilderness, the root of a tree, a hill, a ravine, a mountain cave, a charnel ground, a forest, the open air, a heap of straw.}}\\
\end{addmargin}
\end{absolutelynopagebreak}

\begin{absolutelynopagebreak}
\setstretch{.7}
{\PaliGlossA{Tassa tathāvūpakaṭṭhassa viharato satthāpi na upavadati, anuviccapi viññū sabrahmacārī na upavadanti, devatāpi na upavadanti, attāpi attānaṃ na upavadati.}}\\
\begin{addmargin}[1em]{2em}
\setstretch{.5}
{\PaliGlossB{While they’re living withdrawn, they’re not reprimanded by the Teacher, by sensible spiritual companions after examination, by deities, or by themselves.}}\\
\end{addmargin}
\end{absolutelynopagebreak}

\begin{absolutelynopagebreak}
\setstretch{.7}
{\PaliGlossA{So satthārāpi anupavadito, anuviccapi viññūhi sabrahmacārīhi anupavadito, devatāhipi anupavadito, attanāpi attānaṃ anupavadito uttari manussadhammā alamariyañāṇadassanavisesaṃ sacchikaroti.}}\\
\begin{addmargin}[1em]{2em}
\setstretch{.5}
{\PaliGlossB{Not being reprimanded in this way, they realize a superhuman distinction in knowledge and vision worthy of the noble ones.}}\\
\end{addmargin}
\end{absolutelynopagebreak}

\vskip 0.05in
\begin{absolutelynopagebreak}
\setstretch{.7}
{\PaliGlossA{So vivicceva kāmehi vivicca akusalehi dhammehi savitakkaṃ savicāraṃ vivekajaṃ pītisukhaṃ paṭhamaṃ jhānaṃ upasampajja viharati.}}\\
\begin{addmargin}[1em]{2em}
\setstretch{.5}
{\PaliGlossB{Quite secluded from sensual pleasures, secluded from unskillful qualities, they enter and remain in the first absorption, which has the rapture and bliss born of seclusion, while placing the mind and keeping it connected.}}\\
\end{addmargin}
\end{absolutelynopagebreak}

\begin{absolutelynopagebreak}
\setstretch{.7}
{\PaliGlossA{Taṃ kissa hetu?}}\\
\begin{addmargin}[1em]{2em}
\setstretch{.5}
{\PaliGlossB{Why is that?}}\\
\end{addmargin}
\end{absolutelynopagebreak}

\begin{absolutelynopagebreak}
\setstretch{.7}
{\PaliGlossA{Evañhi taṃ, bhaddāli, hoti yathā taṃ satthusāsane sikkhāya paripūrakārissa.}}\\
\begin{addmargin}[1em]{2em}
\setstretch{.5}
{\PaliGlossB{Because that’s what happens when someone fulfills the training according to the Teacher’s instructions.}}\\
\end{addmargin}
\end{absolutelynopagebreak}

\vskip 0.05in
\begin{absolutelynopagebreak}
\setstretch{.7}
{\PaliGlossA{Puna caparaṃ, bhaddāli, bhikkhu vitakkavicārānaṃ vūpasamā ajjhattaṃ sampasādanaṃ cetaso ekodibhāvaṃ avitakkaṃ avicāraṃ samādhijaṃ pītisukhaṃ dutiyaṃ jhānaṃ upasampajja viharati.}}\\
\begin{addmargin}[1em]{2em}
\setstretch{.5}
{\PaliGlossB{Furthermore, as the placing of the mind and keeping it connected are stilled, a mendicant enters and remains in the second absorption, which has the rapture and bliss born of immersion, with internal clarity and confidence, and unified mind, without placing the mind and keeping it connected.}}\\
\end{addmargin}
\end{absolutelynopagebreak}

\begin{absolutelynopagebreak}
\setstretch{.7}
{\PaliGlossA{Taṃ kissa hetu?}}\\
\begin{addmargin}[1em]{2em}
\setstretch{.5}
{\PaliGlossB{Why is that?}}\\
\end{addmargin}
\end{absolutelynopagebreak}

\begin{absolutelynopagebreak}
\setstretch{.7}
{\PaliGlossA{Evañhi taṃ, bhaddāli, hoti yathā taṃ satthusāsane sikkhāya paripūrakārissa.}}\\
\begin{addmargin}[1em]{2em}
\setstretch{.5}
{\PaliGlossB{Because that’s what happens when someone fulfills the training according to the Teacher’s instructions.}}\\
\end{addmargin}
\end{absolutelynopagebreak}

\begin{absolutelynopagebreak}
\setstretch{.7}
{\PaliGlossA{Puna caparaṃ, bhaddāli, bhikkhu pītiyā ca virāgā upekkhako ca viharati, sato ca sampajāno sukhañca kāyena paṭisaṃvedeti, yaṃ taṃ ariyā ācikkhanti: ‘upekkhako satimā sukhavihārī’ti tatiyaṃ jhānaṃ upasampajja viharati.}}\\
\begin{addmargin}[1em]{2em}
\setstretch{.5}
{\PaliGlossB{Furthermore, with the fading away of rapture, a mendicant enters and remains in the third absorption, where they meditate with equanimity, mindful and aware, personally experiencing the bliss of which the noble ones declare, ‘Equanimous and mindful, one meditates in bliss.’}}\\
\end{addmargin}
\end{absolutelynopagebreak}

\begin{absolutelynopagebreak}
\setstretch{.7}
{\PaliGlossA{Taṃ kissa hetu?}}\\
\begin{addmargin}[1em]{2em}
\setstretch{.5}
{\PaliGlossB{Why is that?}}\\
\end{addmargin}
\end{absolutelynopagebreak}

\begin{absolutelynopagebreak}
\setstretch{.7}
{\PaliGlossA{Evañhi taṃ, bhaddāli, hoti yathā taṃ satthusāsane sikkhāya paripūrakārissa.}}\\
\begin{addmargin}[1em]{2em}
\setstretch{.5}
{\PaliGlossB{Because that’s what happens when someone fulfills the training according to the Teacher’s instructions.}}\\
\end{addmargin}
\end{absolutelynopagebreak}

\begin{absolutelynopagebreak}
\setstretch{.7}
{\PaliGlossA{Puna caparaṃ, bhaddāli, bhikkhu sukhassa ca pahānā dukkhassa ca pahānā pubbeva somanassadomanassānaṃ atthaṅgamā adukkhamasukhaṃ upekkhāsatipārisuddhiṃ catutthaṃ jhānaṃ upasampajja viharati.}}\\
\begin{addmargin}[1em]{2em}
\setstretch{.5}
{\PaliGlossB{Furthermore, giving up pleasure and pain, and ending former happiness and sadness, a mendicant enters and remains in the fourth absorption, without pleasure or pain, with pure equanimity and mindfulness.}}\\
\end{addmargin}
\end{absolutelynopagebreak}

\begin{absolutelynopagebreak}
\setstretch{.7}
{\PaliGlossA{Taṃ kissa hetu?}}\\
\begin{addmargin}[1em]{2em}
\setstretch{.5}
{\PaliGlossB{Why is that?}}\\
\end{addmargin}
\end{absolutelynopagebreak}

\begin{absolutelynopagebreak}
\setstretch{.7}
{\PaliGlossA{Evañhi taṃ, bhaddāli, hoti yathā taṃ satthusāsane sikkhāya paripūrakārissa.}}\\
\begin{addmargin}[1em]{2em}
\setstretch{.5}
{\PaliGlossB{Because that’s what happens when someone fulfills the training according to the Teacher’s instructions.}}\\
\end{addmargin}
\end{absolutelynopagebreak}

\vskip 0.05in
\begin{absolutelynopagebreak}
\setstretch{.7}
{\PaliGlossA{So evaṃ samāhite citte parisuddhe pariyodāte anaṅgaṇe vigatūpakkilese mudubhūte kammaniye ṭhite āneñjappatte pubbenivāsānussatiñāṇāya cittaṃ abhininnāmeti.}}\\
\begin{addmargin}[1em]{2em}
\setstretch{.5}
{\PaliGlossB{When their mind has become immersed in samādhi like this—purified, bright, flawless, rid of corruptions, pliable, workable, steady, and imperturbable—they extend it toward recollection of past lives.}}\\
\end{addmargin}
\end{absolutelynopagebreak}

\begin{absolutelynopagebreak}
\setstretch{.7}
{\PaliGlossA{So anekavihitaṃ pubbenivāsaṃ anussarati, seyyathidaṃ—ekampi jātiṃ dvepi jātiyo … pe … iti sākāraṃ sauddesaṃ anekavihitaṃ pubbenivāsaṃ anussarati.}}\\
\begin{addmargin}[1em]{2em}
\setstretch{.5}
{\PaliGlossB{They recollect many kinds of past lives, that is, one, two, three, four, five, ten, twenty, thirty, forty, fifty, a hundred, a thousand, a hundred thousand rebirths; many eons of the world contracting, many eons of the world expanding, many eons of the world contracting and expanding. … They recollect their many kinds of past lives, with features and details.}}\\
\end{addmargin}
\end{absolutelynopagebreak}

\begin{absolutelynopagebreak}
\setstretch{.7}
{\PaliGlossA{Taṃ kissa hetu?}}\\
\begin{addmargin}[1em]{2em}
\setstretch{.5}
{\PaliGlossB{Why is that?}}\\
\end{addmargin}
\end{absolutelynopagebreak}

\begin{absolutelynopagebreak}
\setstretch{.7}
{\PaliGlossA{Evañhi taṃ, bhaddāli, hoti yathā taṃ satthusāsane sikkhāya paripūrakārissa.}}\\
\begin{addmargin}[1em]{2em}
\setstretch{.5}
{\PaliGlossB{Because that’s what happens when someone fulfills the training according to the Teacher’s instructions.}}\\
\end{addmargin}
\end{absolutelynopagebreak}

\vskip 0.05in
\begin{absolutelynopagebreak}
\setstretch{.7}
{\PaliGlossA{So evaṃ samāhite citte parisuddhe pariyodāte anaṅgaṇe vigatūpakkilese mudubhūte kammaniye ṭhite āneñjappatte sattānaṃ cutūpapātañāṇāya cittaṃ abhininnāmeti.}}\\
\begin{addmargin}[1em]{2em}
\setstretch{.5}
{\PaliGlossB{When their mind has become immersed in samādhi like this—purified, bright, flawless, rid of corruptions, pliable, workable, steady, and imperturbable—they extend it toward knowledge of the death and rebirth of sentient beings.}}\\
\end{addmargin}
\end{absolutelynopagebreak}

\begin{absolutelynopagebreak}
\setstretch{.7}
{\PaliGlossA{So dibbena cakkhunā visuddhena atikkantamānusakena satte passati cavamāne upapajjamāne hīne paṇīte suvaṇṇe dubbaṇṇe sugate duggate yathākammūpage satte pajānāti: ‘ime vata bhonto sattā kāyaduccaritena samannāgatā … pe … vinipātaṃ nirayaṃ upapannā; ime vā pana bhonto sattā kāyasucaritena samannāgatā … pe … sugatiṃ saggaṃ lokaṃ upapannā’ti iti dibbena cakkhunā visuddhena atikkantamānusakena … pe … yathākammūpage satte pajānāti.}}\\
\begin{addmargin}[1em]{2em}
\setstretch{.5}
{\PaliGlossB{With clairvoyance that is purified and superhuman, they see sentient beings passing away and being reborn—inferior and superior, beautiful and ugly, in a good place or a bad place. They understand how sentient beings are reborn according to their deeds: ‘These dear beings did bad things by way of body, speech, and mind. … They’re reborn in the underworld, hell. These dear beings, however, did good things by way of body, speech, and mind. … they’re reborn in a good place, a heavenly realm.’ And so, with clairvoyance that is purified and superhuman … they understand how sentient beings are reborn according to their deeds.}}\\
\end{addmargin}
\end{absolutelynopagebreak}

\begin{absolutelynopagebreak}
\setstretch{.7}
{\PaliGlossA{Taṃ kissa hetu?}}\\
\begin{addmargin}[1em]{2em}
\setstretch{.5}
{\PaliGlossB{Why is that?}}\\
\end{addmargin}
\end{absolutelynopagebreak}

\begin{absolutelynopagebreak}
\setstretch{.7}
{\PaliGlossA{Evañhi taṃ, bhaddāli, hoti yathā taṃ satthusāsane sikkhāya paripūrakārissa.}}\\
\begin{addmargin}[1em]{2em}
\setstretch{.5}
{\PaliGlossB{Because that’s what happens when someone fulfills the training according to the Teacher’s instructions.}}\\
\end{addmargin}
\end{absolutelynopagebreak}

\vskip 0.05in
\begin{absolutelynopagebreak}
\setstretch{.7}
{\PaliGlossA{So evaṃ samāhite citte parisuddhe pariyodāte anaṅgaṇe vigatūpakkilese mudubhūte kammaniye ṭhite āneñjappatte āsavānaṃ khayañāṇāya cittaṃ abhininnāmeti.}}\\
\begin{addmargin}[1em]{2em}
\setstretch{.5}
{\PaliGlossB{When their mind has become immersed in samādhi like this—purified, bright, flawless, rid of corruptions, pliable, workable, steady, and imperturbable—they extend it toward knowledge of the ending of defilements.}}\\
\end{addmargin}
\end{absolutelynopagebreak}

\begin{absolutelynopagebreak}
\setstretch{.7}
{\PaliGlossA{So ‘idaṃ dukkhan’ti yathābhūtaṃ pajānāti, ‘ayaṃ dukkhasamudayo’ti yathābhūtaṃ pajānāti, ‘ayaṃ dukkhanirodho’ti yathābhūtaṃ pajānāti, ‘ayaṃ dukkhanirodhagāminī paṭipadā’ti yathābhūtaṃ pajānāti;}}\\
\begin{addmargin}[1em]{2em}
\setstretch{.5}
{\PaliGlossB{They truly understand: ‘This is suffering’ … ‘This is the origin of suffering’ … ‘This is the cessation of suffering’ … ‘This is the practice that leads to the cessation of suffering’.}}\\
\end{addmargin}
\end{absolutelynopagebreak}

\begin{absolutelynopagebreak}
\setstretch{.7}
{\PaliGlossA{‘ime āsavā’ti yathābhūtaṃ pajānāti, ‘ayaṃ āsavasamudayo’ti yathābhūtaṃ pajānāti, ‘ayaṃ āsavanirodho’ti yathābhūtaṃ pajānāti, ‘ayaṃ āsavanirodhagāminī paṭipadā’ti yathābhūtaṃ pajānāti.}}\\
\begin{addmargin}[1em]{2em}
\setstretch{.5}
{\PaliGlossB{They truly understand: ‘These are defilements’ … ‘This is the origin of defilements’ … ‘This is the cessation of defilements’ … ‘This is the practice that leads to the cessation of defilements’.}}\\
\end{addmargin}
\end{absolutelynopagebreak}

\vskip 0.05in
\begin{absolutelynopagebreak}
\setstretch{.7}
{\PaliGlossA{Tassa evaṃ jānato evaṃ passato kāmāsavāpi cittaṃ vimuccati, bhavāsavāpi cittaṃ vimuccati, avijjāsavāpi cittaṃ vimuccati.}}\\
\begin{addmargin}[1em]{2em}
\setstretch{.5}
{\PaliGlossB{Knowing and seeing like this, their mind is freed from the defilements of sensuality, desire to be reborn, and ignorance.}}\\
\end{addmargin}
\end{absolutelynopagebreak}

\begin{absolutelynopagebreak}
\setstretch{.7}
{\PaliGlossA{Vimuttasmiṃ vimuttamiti ñāṇaṃ hoti.}}\\
\begin{addmargin}[1em]{2em}
\setstretch{.5}
{\PaliGlossB{When they’re freed, they know they’re freed.}}\\
\end{addmargin}
\end{absolutelynopagebreak}

\begin{absolutelynopagebreak}
\setstretch{.7}
{\PaliGlossA{‘Khīṇā jāti, vusitaṃ brahmacariyaṃ, kataṃ karaṇīyaṃ, nāparaṃ itthattāyā’ti pajānāti.}}\\
\begin{addmargin}[1em]{2em}
\setstretch{.5}
{\PaliGlossB{They understand: ‘Rebirth is ended, the spiritual journey has been completed, what had to be done has been done, there is no return to any state of existence.’}}\\
\end{addmargin}
\end{absolutelynopagebreak}

\begin{absolutelynopagebreak}
\setstretch{.7}
{\PaliGlossA{Taṃ kissa hetu?}}\\
\begin{addmargin}[1em]{2em}
\setstretch{.5}
{\PaliGlossB{Why is that?}}\\
\end{addmargin}
\end{absolutelynopagebreak}

\begin{absolutelynopagebreak}
\setstretch{.7}
{\PaliGlossA{Evañhi taṃ, bhaddāli, hoti yathā taṃ satthusāsane sikkhāya paripūrakārissā”ti.}}\\
\begin{addmargin}[1em]{2em}
\setstretch{.5}
{\PaliGlossB{Because that’s what happens when someone fulfills the training according to the Teacher’s instructions.”}}\\
\end{addmargin}
\end{absolutelynopagebreak}

\vskip 0.05in
\begin{absolutelynopagebreak}
\setstretch{.7}
{\PaliGlossA{Evaṃ vutte, āyasmā bhaddāli bhagavantaṃ etadavoca:}}\\
\begin{addmargin}[1em]{2em}
\setstretch{.5}
{\PaliGlossB{When he said this, Venerable Bhaddāli said to the Buddha,}}\\
\end{addmargin}
\end{absolutelynopagebreak}

\begin{absolutelynopagebreak}
\setstretch{.7}
{\PaliGlossA{“ko nu kho, bhante, hetu, ko paccayo yena midhekaccaṃ bhikkhuṃ pasayha pasayha kāraṇaṃ karonti?}}\\
\begin{addmargin}[1em]{2em}
\setstretch{.5}
{\PaliGlossB{“What is the cause, sir, what is the reason why they punish some monk, repeatedly pressuring him?}}\\
\end{addmargin}
\end{absolutelynopagebreak}

\begin{absolutelynopagebreak}
\setstretch{.7}
{\PaliGlossA{Ko pana, bhante, hetu, ko paccayo yena midhekaccaṃ bhikkhuṃ no tathā pasayha pasayha kāraṇaṃ karontī”ti?}}\\
\begin{addmargin}[1em]{2em}
\setstretch{.5}
{\PaliGlossB{And what is the cause, what is the reason why they don’t similarly punish another monk, repeatedly pressuring him?”}}\\
\end{addmargin}
\end{absolutelynopagebreak}

\vskip 0.05in
\begin{absolutelynopagebreak}
\setstretch{.7}
{\PaliGlossA{“Idha, bhaddāli, ekacco bhikkhu abhiṇhāpattiko hoti āpattibahulo.}}\\
\begin{addmargin}[1em]{2em}
\setstretch{.5}
{\PaliGlossB{“Take a monk who is a frequent offender with many offenses.}}\\
\end{addmargin}
\end{absolutelynopagebreak}

\begin{absolutelynopagebreak}
\setstretch{.7}
{\PaliGlossA{So bhikkhūhi vuccamāno aññenaññaṃ paṭicarati, bahiddhā kathaṃ apanāmeti, kopañca dosañca appaccayañca pātukaroti, na sammā vattati, na lomaṃ pāteti, na netthāraṃ vattati, ‘yena saṃgho attamano hoti taṃ karomī’ti nāha.}}\\
\begin{addmargin}[1em]{2em}
\setstretch{.5}
{\PaliGlossB{When admonished by the monks, he dodges the issue, distracting the discussion with irrelevant points. He displays annoyance, hate, and bitterness. He doesn’t proceed properly, he doesn’t fall in line, he doesn’t proceed to get past it, and he doesn’t say: ‘I’ll do what pleases the Saṅgha.’}}\\
\end{addmargin}
\end{absolutelynopagebreak}

\begin{absolutelynopagebreak}
\setstretch{.7}
{\PaliGlossA{Tatra, bhaddāli, bhikkhūnaṃ evaṃ hoti:}}\\
\begin{addmargin}[1em]{2em}
\setstretch{.5}
{\PaliGlossB{In such a case, the monks say:}}\\
\end{addmargin}
\end{absolutelynopagebreak}

\begin{absolutelynopagebreak}
\setstretch{.7}
{\PaliGlossA{‘ayaṃ kho, āvuso, bhikkhu abhiṇhāpattiko āpattibahulo.}}\\
\begin{addmargin}[1em]{2em}
\setstretch{.5}
{\PaliGlossB{‘Reverends, this monk is a frequent offender, with many offenses.}}\\
\end{addmargin}
\end{absolutelynopagebreak}

\begin{absolutelynopagebreak}
\setstretch{.7}
{\PaliGlossA{So bhikkhūhi vuccamāno aññenaññaṃ paṭicarati, bahiddhā kathaṃ apanāmeti, kopañca dosañca appaccayañca pātukaroti, na sammā vattati, na lomaṃ pāteti, na netthāraṃ vattati, “yena saṃgho attamano hoti taṃ karomī”ti nāha.}}\\
\begin{addmargin}[1em]{2em}
\setstretch{.5}
{\PaliGlossB{When admonished by the monks, he dodges the issue, distracting the discussion with irrelevant points. He displays annoyance, hate, and bitterness. He doesn’t proceed properly, he doesn’t fall in line, he doesn’t proceed to get past it, and he doesn’t say: “I’ll do what pleases the Saṅgha.”}}\\
\end{addmargin}
\end{absolutelynopagebreak}

\begin{absolutelynopagebreak}
\setstretch{.7}
{\PaliGlossA{Sādhu vatāyasmanto imassa bhikkhuno tathā tathā upaparikkhatha yathāssidaṃ adhikaraṇaṃ na khippameva vūpasameyyā’ti.}}\\
\begin{addmargin}[1em]{2em}
\setstretch{.5}
{\PaliGlossB{It’d be good for the venerables to examine this monk in such a way that this disciplinary issue is not quickly settled.’}}\\
\end{addmargin}
\end{absolutelynopagebreak}

\begin{absolutelynopagebreak}
\setstretch{.7}
{\PaliGlossA{Tassa kho evaṃ, bhaddāli, bhikkhuno bhikkhū tathā tathā upaparikkhanti yathāssidaṃ adhikaraṇaṃ na khippameva vūpasammati.}}\\
\begin{addmargin}[1em]{2em}
\setstretch{.5}
{\PaliGlossB{And that’s what they do.}}\\
\end{addmargin}
\end{absolutelynopagebreak}

\vskip 0.05in
\begin{absolutelynopagebreak}
\setstretch{.7}
{\PaliGlossA{Idha pana, bhaddāli, ekacco bhikkhu abhiṇhāpattiko hoti āpattibahulo.}}\\
\begin{addmargin}[1em]{2em}
\setstretch{.5}
{\PaliGlossB{Take some other monk who is a frequent offender with many offenses.}}\\
\end{addmargin}
\end{absolutelynopagebreak}

\begin{absolutelynopagebreak}
\setstretch{.7}
{\PaliGlossA{So bhikkhūhi vuccamāno nāññenaññaṃ paṭicarati, bahiddhā kathaṃ na apanāmeti, na kopañca dosañca appaccayañca pātukaroti, sammā vattati, lomaṃ pāteti, netthāraṃ vattati, ‘yena saṅgho attamano hoti taṃ karomī’ti āha.}}\\
\begin{addmargin}[1em]{2em}
\setstretch{.5}
{\PaliGlossB{When admonished by the monks, he doesn’t dodge the issue, distracting the discussion with irrelevant points. He doesn’t display annoyance, hate, and bitterness. He proceeds properly, he falls in line, he proceeds to get past it, and he says: ‘I’ll do what pleases the Saṅgha.’}}\\
\end{addmargin}
\end{absolutelynopagebreak}

\begin{absolutelynopagebreak}
\setstretch{.7}
{\PaliGlossA{Tatra, bhaddāli, bhikkhūnaṃ evaṃ hoti:}}\\
\begin{addmargin}[1em]{2em}
\setstretch{.5}
{\PaliGlossB{In such a case, the monks say:}}\\
\end{addmargin}
\end{absolutelynopagebreak}

\begin{absolutelynopagebreak}
\setstretch{.7}
{\PaliGlossA{‘ayaṃ kho, āvuso, bhikkhu abhiṇhāpattiko āpattibahulo.}}\\
\begin{addmargin}[1em]{2em}
\setstretch{.5}
{\PaliGlossB{‘Reverends, this monk is a frequent offender, with many offenses.}}\\
\end{addmargin}
\end{absolutelynopagebreak}

\begin{absolutelynopagebreak}
\setstretch{.7}
{\PaliGlossA{So bhikkhūhi vuccamāno nāññenaññaṃ paṭicarati, bahiddhā kathaṃ na apanāmeti, na kopañca dosañca appaccayañca pātukaroti, sammā vattati, lomaṃ pāteti, netthāraṃ vattati, “yena saṅgho attamano hoti taṃ karomī”ti āha.}}\\
\begin{addmargin}[1em]{2em}
\setstretch{.5}
{\PaliGlossB{When admonished by the monks, he doesn’t dodge the issue, distracting the discussion with irrelevant points. He doesn’t display annoyance, hate, and bitterness. He proceeds properly, he falls in line, he proceeds to get past it, and he says: ‘I’ll do what pleases the Saṅgha.’}}\\
\end{addmargin}
\end{absolutelynopagebreak}

\begin{absolutelynopagebreak}
\setstretch{.7}
{\PaliGlossA{Sādhu vatāyasmanto, imassa bhikkhuno tathā tathā upaparikkhatha yathāssidaṃ adhikaraṇaṃ khippameva vūpasameyyā’ti.}}\\
\begin{addmargin}[1em]{2em}
\setstretch{.5}
{\PaliGlossB{It’d be good for the venerables to examine this monk in such a way that this disciplinary issue is quickly settled.’}}\\
\end{addmargin}
\end{absolutelynopagebreak}

\begin{absolutelynopagebreak}
\setstretch{.7}
{\PaliGlossA{Tassa kho evaṃ, bhaddāli, bhikkhuno bhikkhū tathā tathā upaparikkhanti yathāssidaṃ adhikaraṇaṃ khippameva vūpasammati.}}\\
\begin{addmargin}[1em]{2em}
\setstretch{.5}
{\PaliGlossB{And that’s what they do.}}\\
\end{addmargin}
\end{absolutelynopagebreak}

\vskip 0.05in
\begin{absolutelynopagebreak}
\setstretch{.7}
{\PaliGlossA{Idha, bhaddāli, ekacco bhikkhu adhiccāpattiko hoti anāpattibahulo.}}\\
\begin{addmargin}[1em]{2em}
\setstretch{.5}
{\PaliGlossB{Take some other monk who is an occasional offender without many offenses.}}\\
\end{addmargin}
\end{absolutelynopagebreak}

\begin{absolutelynopagebreak}
\setstretch{.7}
{\PaliGlossA{So bhikkhūhi vuccamāno aññenaññaṃ paṭicarati, bahiddhā kathaṃ apanāmeti, kopañca dosañca appaccayañca pātukaroti, na sammā vattati, na lomaṃ pāteti, na netthāraṃ vattati, ‘yena saṅgho attamano hoti taṃ karomī’ti nāha.}}\\
\begin{addmargin}[1em]{2em}
\setstretch{.5}
{\PaliGlossB{When admonished by the monks, he dodges the issue …}}\\
\end{addmargin}
\end{absolutelynopagebreak}

\begin{absolutelynopagebreak}
\setstretch{.7}
{\PaliGlossA{Tatra, bhaddāli, bhikkhūnaṃ evaṃ hoti:}}\\
\begin{addmargin}[1em]{2em}
\setstretch{.5}
{\PaliGlossB{In such a case, the monks say:}}\\
\end{addmargin}
\end{absolutelynopagebreak}

\begin{absolutelynopagebreak}
\setstretch{.7}
{\PaliGlossA{‘ayaṃ kho, āvuso, bhikkhu adhiccāpattiko anāpattibahulo.}}\\
\begin{addmargin}[1em]{2em}
\setstretch{.5}
{\PaliGlossB{‘Reverends, this monk is an occasional offender without many offenses.}}\\
\end{addmargin}
\end{absolutelynopagebreak}

\begin{absolutelynopagebreak}
\setstretch{.7}
{\PaliGlossA{So bhikkhūhi vuccamāno aññenaññaṃ paṭicarati, bahiddhā kathaṃ apanāmeti, kopañca dosañca appaccayañca pātukaroti, na sammā vattati, na lomaṃ pāteti, na netthāraṃ vattati, “yena saṅgho attamano hoti taṃ karomī”ti nāha.}}\\
\begin{addmargin}[1em]{2em}
\setstretch{.5}
{\PaliGlossB{When admonished by the monks, he dodges the issue …}}\\
\end{addmargin}
\end{absolutelynopagebreak}

\begin{absolutelynopagebreak}
\setstretch{.7}
{\PaliGlossA{Sādhu vatāyasmanto, imassa bhikkhuno tathā tathā upaparikkhatha yathāssidaṃ adhikaraṇaṃ na khippameva vūpasameyyā’ti.}}\\
\begin{addmargin}[1em]{2em}
\setstretch{.5}
{\PaliGlossB{It’d be good for the venerables to examine this monk in such a way that this disciplinary issue is not quickly settled.’}}\\
\end{addmargin}
\end{absolutelynopagebreak}

\begin{absolutelynopagebreak}
\setstretch{.7}
{\PaliGlossA{Tassa kho evaṃ, bhaddāli, bhikkhuno bhikkhū tathā tathā upaparikkhanti yathāssidaṃ adhikaraṇaṃ na khippameva vūpasammati.}}\\
\begin{addmargin}[1em]{2em}
\setstretch{.5}
{\PaliGlossB{And that’s what they do.}}\\
\end{addmargin}
\end{absolutelynopagebreak}

\vskip 0.05in
\begin{absolutelynopagebreak}
\setstretch{.7}
{\PaliGlossA{Idha pana, bhaddāli, ekacco bhikkhu adhiccāpattiko hoti anāpattibahulo.}}\\
\begin{addmargin}[1em]{2em}
\setstretch{.5}
{\PaliGlossB{Take some other monk who is an occasional offender without many offenses.}}\\
\end{addmargin}
\end{absolutelynopagebreak}

\begin{absolutelynopagebreak}
\setstretch{.7}
{\PaliGlossA{So bhikkhūhi vuccamāno nāññenaññaṃ paṭicarati, na bahiddhā kathaṃ apanāmeti, na kopañca dosañca appaccayañca pātukaroti, sammā vattati, lomaṃ pāteti, netthāraṃ vattati, ‘yena saṅgho attamano hoti taṃ karomī’ti āha.}}\\
\begin{addmargin}[1em]{2em}
\setstretch{.5}
{\PaliGlossB{When admonished by the monks, he doesn’t dodge the issue …}}\\
\end{addmargin}
\end{absolutelynopagebreak}

\begin{absolutelynopagebreak}
\setstretch{.7}
{\PaliGlossA{Tatra, bhaddāli, bhikkhūnaṃ evaṃ hoti:}}\\
\begin{addmargin}[1em]{2em}
\setstretch{.5}
{\PaliGlossB{In such a case, the monks say:}}\\
\end{addmargin}
\end{absolutelynopagebreak}

\begin{absolutelynopagebreak}
\setstretch{.7}
{\PaliGlossA{‘ayaṃ kho, āvuso, bhikkhu adhiccāpattiko anāpattibahulo.}}\\
\begin{addmargin}[1em]{2em}
\setstretch{.5}
{\PaliGlossB{‘Reverends, this monk is an occasional offender without many offenses.}}\\
\end{addmargin}
\end{absolutelynopagebreak}

\begin{absolutelynopagebreak}
\setstretch{.7}
{\PaliGlossA{So bhikkhūhi vuccamāno nāññenaññaṃ paṭicarati, na bahiddhā kathaṃ apanāmeti, na kopañca dosañca appaccayañca pātukaroti, sammā vattati, lomaṃ pāteti, netthāraṃ vattati, “yena saṅgho attamano hoti taṃ karomī”ti āha.}}\\
\begin{addmargin}[1em]{2em}
\setstretch{.5}
{\PaliGlossB{When admonished by the monks, he doesn’t dodge the issue …}}\\
\end{addmargin}
\end{absolutelynopagebreak}

\begin{absolutelynopagebreak}
\setstretch{.7}
{\PaliGlossA{Sādhu vatāyasmanto, imassa bhikkhuno tathā tathā upaparikkhatha yathāssidaṃ adhikaraṇaṃ khippameva vūpasameyyā’ti.}}\\
\begin{addmargin}[1em]{2em}
\setstretch{.5}
{\PaliGlossB{It’d be good for the venerables to examine this monk in such a way that this disciplinary issue is quickly settled.’}}\\
\end{addmargin}
\end{absolutelynopagebreak}

\begin{absolutelynopagebreak}
\setstretch{.7}
{\PaliGlossA{Tassa kho evaṃ, bhaddāli, bhikkhuno bhikkhū tathā tathā upaparikkhanti yathāssidaṃ adhikaraṇaṃ khippameva vūpasammati.}}\\
\begin{addmargin}[1em]{2em}
\setstretch{.5}
{\PaliGlossB{And that’s what they do.}}\\
\end{addmargin}
\end{absolutelynopagebreak}

\vskip 0.05in
\begin{absolutelynopagebreak}
\setstretch{.7}
{\PaliGlossA{Idha, bhaddāli, ekacco bhikkhu saddhāmattakena vahati pemamattakena.}}\\
\begin{addmargin}[1em]{2em}
\setstretch{.5}
{\PaliGlossB{Take some other monk who gets by with mere faith and love.}}\\
\end{addmargin}
\end{absolutelynopagebreak}

\begin{absolutelynopagebreak}
\setstretch{.7}
{\PaliGlossA{Tatra, bhaddāli, bhikkhūnaṃ evaṃ hoti:}}\\
\begin{addmargin}[1em]{2em}
\setstretch{.5}
{\PaliGlossB{In such a case, the monks say:}}\\
\end{addmargin}
\end{absolutelynopagebreak}

\begin{absolutelynopagebreak}
\setstretch{.7}
{\PaliGlossA{‘ayaṃ kho, āvuso, bhikkhu saddhāmattakena vahati pemamattakena.}}\\
\begin{addmargin}[1em]{2em}
\setstretch{.5}
{\PaliGlossB{‘Reverends, this monk gets by with mere faith and love.}}\\
\end{addmargin}
\end{absolutelynopagebreak}

\begin{absolutelynopagebreak}
\setstretch{.7}
{\PaliGlossA{Sace mayaṃ imaṃ bhikkhuṃ pasayha pasayha kāraṇaṃ karissāma—}}\\
\begin{addmargin}[1em]{2em}
\setstretch{.5}
{\PaliGlossB{If we punish him, repeatedly pressuring him—}}\\
\end{addmargin}
\end{absolutelynopagebreak}

\begin{absolutelynopagebreak}
\setstretch{.7}
{\PaliGlossA{mā yampissa taṃ saddhāmattakaṃ pemamattakaṃ tamhāpi parihāyī’ti.}}\\
\begin{addmargin}[1em]{2em}
\setstretch{.5}
{\PaliGlossB{no, let him not lose what little faith and love he has!’}}\\
\end{addmargin}
\end{absolutelynopagebreak}

\vskip 0.05in
\begin{absolutelynopagebreak}
\setstretch{.7}
{\PaliGlossA{Seyyathāpi, bhaddāli, purisassa ekaṃ cakkhuṃ, tassa mittāmaccā ñātisālohitā taṃ ekaṃ cakkhuṃ rakkheyyuṃ:}}\\
\begin{addmargin}[1em]{2em}
\setstretch{.5}
{\PaliGlossB{Suppose there was a person with one eye. Their friends and colleagues, relatives and kin would protect that one eye:}}\\
\end{addmargin}
\end{absolutelynopagebreak}

\begin{absolutelynopagebreak}
\setstretch{.7}
{\PaliGlossA{‘mā yampissa taṃ ekaṃ cakkhuṃ tamhāpi parihāyī’ti;}}\\
\begin{addmargin}[1em]{2em}
\setstretch{.5}
{\PaliGlossB{‘Let them not lose the one eye that they have!’}}\\
\end{addmargin}
\end{absolutelynopagebreak}

\begin{absolutelynopagebreak}
\setstretch{.7}
{\PaliGlossA{evameva kho, bhaddāli, idhekacco bhikkhu saddhāmattakena vahati pemamattakena.}}\\
\begin{addmargin}[1em]{2em}
\setstretch{.5}
{\PaliGlossB{In the same way, some monk gets by with mere faith and love.}}\\
\end{addmargin}
\end{absolutelynopagebreak}

\begin{absolutelynopagebreak}
\setstretch{.7}
{\PaliGlossA{Tatra, bhaddāli, bhikkhūnaṃ evaṃ hoti:}}\\
\begin{addmargin}[1em]{2em}
\setstretch{.5}
{\PaliGlossB{In such a case, the monks say:}}\\
\end{addmargin}
\end{absolutelynopagebreak}

\begin{absolutelynopagebreak}
\setstretch{.7}
{\PaliGlossA{‘ayaṃ kho, āvuso, bhikkhu saddhāmattakena vahati pemamattakena.}}\\
\begin{addmargin}[1em]{2em}
\setstretch{.5}
{\PaliGlossB{‘Reverends, this monk gets by with mere faith and love.}}\\
\end{addmargin}
\end{absolutelynopagebreak}

\begin{absolutelynopagebreak}
\setstretch{.7}
{\PaliGlossA{Sace mayaṃ imaṃ bhikkhuṃ pasayha pasayha kāraṇaṃ karissāma—}}\\
\begin{addmargin}[1em]{2em}
\setstretch{.5}
{\PaliGlossB{If we punish him, repeatedly pressuring him—}}\\
\end{addmargin}
\end{absolutelynopagebreak}

\begin{absolutelynopagebreak}
\setstretch{.7}
{\PaliGlossA{mā yampissa taṃ saddhāmattakaṃ pemamattakaṃ tamhāpi parihāyī’ti.}}\\
\begin{addmargin}[1em]{2em}
\setstretch{.5}
{\PaliGlossB{no, let him not lose what little faith and love he has!’}}\\
\end{addmargin}
\end{absolutelynopagebreak}

\begin{absolutelynopagebreak}
\setstretch{.7}
{\PaliGlossA{Ayaṃ kho, bhaddāli, hetu ayaṃ paccayo yena midhekaccaṃ bhikkhuṃ pasayha pasayha kāraṇaṃ karonti.}}\\
\begin{addmargin}[1em]{2em}
\setstretch{.5}
{\PaliGlossB{This is the cause, this is the reason why they punish some monk, repeatedly pressuring him.}}\\
\end{addmargin}
\end{absolutelynopagebreak}

\begin{absolutelynopagebreak}
\setstretch{.7}
{\PaliGlossA{Ayaṃ pana, bhaddāli, hetu ayaṃ paccayo, yena midhekaccaṃ bhikkhuṃ no tathā pasayha pasayha kāraṇaṃ karontī”ti.}}\\
\begin{addmargin}[1em]{2em}
\setstretch{.5}
{\PaliGlossB{And this is the cause, this is the reason why they don’t similarly punish another monk, repeatedly pressuring him.”}}\\
\end{addmargin}
\end{absolutelynopagebreak}

\vskip 0.05in
\begin{absolutelynopagebreak}
\setstretch{.7}
{\PaliGlossA{“Ko nu kho, bhante, hetu, ko paccayo yena pubbe appatarāni ceva sikkhāpadāni ahesuṃ bahutarā ca bhikkhū aññāya saṇṭhahiṃsu?}}\\
\begin{addmargin}[1em]{2em}
\setstretch{.5}
{\PaliGlossB{“What is the cause, sir, what is the reason why there used to be fewer training rules but more enlightened mendicants?}}\\
\end{addmargin}
\end{absolutelynopagebreak}

\begin{absolutelynopagebreak}
\setstretch{.7}
{\PaliGlossA{Ko pana, bhante, hetu, ko paccayo yena etarahi bahutarāni ceva sikkhāpadāni honti appatarā ca bhikkhū aññāya saṇṭhahantī”ti?}}\\
\begin{addmargin}[1em]{2em}
\setstretch{.5}
{\PaliGlossB{And what is the cause, what is the reason why these days there are more training rules and fewer enlightened mendicants?”}}\\
\end{addmargin}
\end{absolutelynopagebreak}

\vskip 0.05in
\begin{absolutelynopagebreak}
\setstretch{.7}
{\PaliGlossA{“Evametaṃ, bhaddāli, hoti sattesu hāyamānesu, saddhamme antaradhāyamāne, bahutarāni ceva sikkhāpadāni honti appatarā ca bhikkhū aññāya saṇṭhahantīti.}}\\
\begin{addmargin}[1em]{2em}
\setstretch{.5}
{\PaliGlossB{“That’s how it is, Bhaddāli. When sentient beings are in decline and the true teaching is disappearing there are more training rules and fewer enlightened mendicants.}}\\
\end{addmargin}
\end{absolutelynopagebreak}

\begin{absolutelynopagebreak}
\setstretch{.7}
{\PaliGlossA{Na tāva, bhaddāli, satthā sāvakānaṃ sikkhāpadaṃ paññāpeti yāva na idhekacce āsavaṭṭhānīyā dhammā saṅghe pātubhavanti.}}\\
\begin{addmargin}[1em]{2em}
\setstretch{.5}
{\PaliGlossB{The Teacher doesnʼt lay down training rules for disciples as long as certain defiling influences have not appeared in the Saṅgha.}}\\
\end{addmargin}
\end{absolutelynopagebreak}

\begin{absolutelynopagebreak}
\setstretch{.7}
{\PaliGlossA{Yato ca kho, bhaddāli, idhekacce āsavaṭṭhānīyā dhammā saṅghe pātubhavanti, atha satthā sāvakānaṃ sikkhāpadaṃ paññāpeti tesaṃyeva āsavaṭṭhānīyānaṃ dhammānaṃ paṭighātāya.}}\\
\begin{addmargin}[1em]{2em}
\setstretch{.5}
{\PaliGlossB{But when such defiling influences appear in the Saṅgha, the Teacher lays down training rules for disciples to protect against them.}}\\
\end{addmargin}
\end{absolutelynopagebreak}

\vskip 0.05in
\begin{absolutelynopagebreak}
\setstretch{.7}
{\PaliGlossA{Na tāva, bhaddāli, idhekacce āsavaṭṭhānīyā dhammā saṅghe pātubhavanti yāva na saṅgho mahattaṃ patto hoti.}}\\
\begin{addmargin}[1em]{2em}
\setstretch{.5}
{\PaliGlossB{And they donʼt appear until the Saṅgha has attained a great size,}}\\
\end{addmargin}
\end{absolutelynopagebreak}

\begin{absolutelynopagebreak}
\setstretch{.7}
{\PaliGlossA{Yato ca kho, bhaddāli, saṅgho mahattaṃ patto hoti, atha idhekacce āsavaṭṭhānīyā dhammā saṅghe pātubhavanti.}}\\
\begin{addmargin}[1em]{2em}
\setstretch{.5}
{\PaliGlossB{    -}}\\
\end{addmargin}
\end{absolutelynopagebreak}

\begin{absolutelynopagebreak}
\setstretch{.7}
{\PaliGlossA{Atha satthā sāvakānaṃ sikkhāpadaṃ paññāpeti tesaṃyeva āsavaṭṭhānīyānaṃ dhammānaṃ paṭighātāya.}}\\
\begin{addmargin}[1em]{2em}
\setstretch{.5}
{\PaliGlossB{    -}}\\
\end{addmargin}
\end{absolutelynopagebreak}

\begin{absolutelynopagebreak}
\setstretch{.7}
{\PaliGlossA{Na tāva, bhaddāli, idhekacce āsavaṭṭhānīyā dhammā saṅghe pātubhavanti yāva na saṅgho lābhaggaṃ patto hoti, yasaggaṃ patto hoti, bāhusaccaṃ patto hoti, rattaññutaṃ patto hoti.}}\\
\begin{addmargin}[1em]{2em}
\setstretch{.5}
{\PaliGlossB{an abundance of material support and fame, learning, and seniority.}}\\
\end{addmargin}
\end{absolutelynopagebreak}

\begin{absolutelynopagebreak}
\setstretch{.7}
{\PaliGlossA{Yato ca kho, bhaddāli, saṅgho rattaññutaṃ patto hoti, atha idhekacce āsavaṭṭhānīyā dhammā saṅghe pātubhavanti, atha satthā sāvakānaṃ sikkhāpadaṃ paññāpeti tesaṃyeva āsavaṭṭhānīyānaṃ dhammānaṃ paṭighātāya.}}\\
\begin{addmargin}[1em]{2em}
\setstretch{.5}
{\PaliGlossB{But when the Saṅgha has attained these things, then such defiling influences appear in the Saṅgha, and the Teacher lays down training rules for disciples to protect against them.}}\\
\end{addmargin}
\end{absolutelynopagebreak}

\vskip 0.05in
\begin{absolutelynopagebreak}
\setstretch{.7}
{\PaliGlossA{Appakā kho tumhe, bhaddāli, tena samayena ahuvattha yadā vo ahaṃ ājānīyasusūpamaṃ dhammapariyāyaṃ desesiṃ.}}\\
\begin{addmargin}[1em]{2em}
\setstretch{.5}
{\PaliGlossB{There were only of few of you there at the time when I taught the exposition of the teaching on the simile of the thoroughbred colt.}}\\
\end{addmargin}
\end{absolutelynopagebreak}

\begin{absolutelynopagebreak}
\setstretch{.7}
{\PaliGlossA{Taṃ sarasi bhaddālī”ti?}}\\
\begin{addmargin}[1em]{2em}
\setstretch{.5}
{\PaliGlossB{Do you remember that, Bhaddāli?”}}\\
\end{addmargin}
\end{absolutelynopagebreak}

\begin{absolutelynopagebreak}
\setstretch{.7}
{\PaliGlossA{“No hetaṃ, bhante”.}}\\
\begin{addmargin}[1em]{2em}
\setstretch{.5}
{\PaliGlossB{“No, sir.”}}\\
\end{addmargin}
\end{absolutelynopagebreak}

\begin{absolutelynopagebreak}
\setstretch{.7}
{\PaliGlossA{“Tatra, bhaddāli, kaṃ hetuṃ paccesī”ti?}}\\
\begin{addmargin}[1em]{2em}
\setstretch{.5}
{\PaliGlossB{“What do you believe the reason for that is?”}}\\
\end{addmargin}
\end{absolutelynopagebreak}

\begin{absolutelynopagebreak}
\setstretch{.7}
{\PaliGlossA{“So hi nūnāhaṃ, bhante, dīgharattaṃ satthusāsane sikkhāya aparipūrakārī ahosin”ti.}}\\
\begin{addmargin}[1em]{2em}
\setstretch{.5}
{\PaliGlossB{“Sir, it’s surely because for a long time now I haven’t fulfilled the training according to the Teacher’s instructions.”}}\\
\end{addmargin}
\end{absolutelynopagebreak}

\begin{absolutelynopagebreak}
\setstretch{.7}
{\PaliGlossA{“Na kho, bhaddāli, eseva hetu, esa paccayo.}}\\
\begin{addmargin}[1em]{2em}
\setstretch{.5}
{\PaliGlossB{“That’s not the only reason, Bhaddāli.}}\\
\end{addmargin}
\end{absolutelynopagebreak}

\begin{absolutelynopagebreak}
\setstretch{.7}
{\PaliGlossA{Api ca me tvaṃ, bhaddāli, dīgharattaṃ cetasā cetoparicca vidito:}}\\
\begin{addmargin}[1em]{2em}
\setstretch{.5}
{\PaliGlossB{Rather, for a long time I have comprehended your mind and known:}}\\
\end{addmargin}
\end{absolutelynopagebreak}

\begin{absolutelynopagebreak}
\setstretch{.7}
{\PaliGlossA{‘na cāyaṃ moghapuriso mayā dhamme desiyamāne aṭṭhiṃ katvā manasi katvā sabbacetaso samannāharitvā ohitasoto dhammaṃ suṇātī’ti.}}\\
\begin{addmargin}[1em]{2em}
\setstretch{.5}
{\PaliGlossB{‘While I’m teaching, this silly man doesn’t pay heed, pay attention, engage wholeheartedly, or lend an ear.’}}\\
\end{addmargin}
\end{absolutelynopagebreak}

\begin{absolutelynopagebreak}
\setstretch{.7}
{\PaliGlossA{Api ca te ahaṃ, bhaddāli, ājānīyasusūpamaṃ dhammapariyāyaṃ desessāmi.}}\\
\begin{addmargin}[1em]{2em}
\setstretch{.5}
{\PaliGlossB{Still, Bhaddāli, I shall teach the exposition of the teaching on the simile of the thoroughbred colt.}}\\
\end{addmargin}
\end{absolutelynopagebreak}

\begin{absolutelynopagebreak}
\setstretch{.7}
{\PaliGlossA{Taṃ suṇāhi, sādhukaṃ manasi karohi; bhāsissāmī”ti.}}\\
\begin{addmargin}[1em]{2em}
\setstretch{.5}
{\PaliGlossB{Listen and pay close attention, I will speak.”}}\\
\end{addmargin}
\end{absolutelynopagebreak}

\begin{absolutelynopagebreak}
\setstretch{.7}
{\PaliGlossA{“Evaṃ, bhante”ti kho āyasmā bhaddāli bhagavato paccassosi.}}\\
\begin{addmargin}[1em]{2em}
\setstretch{.5}
{\PaliGlossB{“Yes, sir,” Bhaddāli replied.}}\\
\end{addmargin}
\end{absolutelynopagebreak}

\begin{absolutelynopagebreak}
\setstretch{.7}
{\PaliGlossA{Bhagavā etadavoca:}}\\
\begin{addmargin}[1em]{2em}
\setstretch{.5}
{\PaliGlossB{The Buddha said this:}}\\
\end{addmargin}
\end{absolutelynopagebreak}

\vskip 0.05in
\begin{absolutelynopagebreak}
\setstretch{.7}
{\PaliGlossA{“Seyyathāpi, bhaddāli, dakkho assadamako bhadraṃ assājānīyaṃ labhitvā paṭhameneva mukhādhāne kāraṇaṃ kāreti.}}\\
\begin{addmargin}[1em]{2em}
\setstretch{.5}
{\PaliGlossB{“Suppose a deft horse trainer were to obtain a fine thoroughbred. First of all he’d make it get used to wearing the bit.}}\\
\end{addmargin}
\end{absolutelynopagebreak}

\begin{absolutelynopagebreak}
\setstretch{.7}
{\PaliGlossA{Tassa mukhādhāne kāraṇaṃ kāriyamānassa hontiyeva visūkāyitāni visevitāni vipphanditāni kānici kānici, yathā taṃ akāritapubbaṃ kāraṇaṃ kāriyamānassa.}}\\
\begin{addmargin}[1em]{2em}
\setstretch{.5}
{\PaliGlossB{Because it has not done this before, it still resorts to some tricks, dodges, and evasions.}}\\
\end{addmargin}
\end{absolutelynopagebreak}

\begin{absolutelynopagebreak}
\setstretch{.7}
{\PaliGlossA{So abhiṇhakāraṇā anupubbakāraṇā tasmiṃ ṭhāne parinibbāyati.}}\\
\begin{addmargin}[1em]{2em}
\setstretch{.5}
{\PaliGlossB{But with regular and gradual practice it quells that bad habit.}}\\
\end{addmargin}
\end{absolutelynopagebreak}

\begin{absolutelynopagebreak}
\setstretch{.7}
{\PaliGlossA{Yato kho, bhaddāli, bhadro assājānīyo abhiṇhakāraṇā anupubbakāraṇā tasmiṃ ṭhāne parinibbuto hoti, tamenaṃ assadamako uttari kāraṇaṃ kāreti yugādhāne.}}\\
\begin{addmargin}[1em]{2em}
\setstretch{.5}
{\PaliGlossB{When it has done this, the horse trainer next makes it get used to wearing the harness.}}\\
\end{addmargin}
\end{absolutelynopagebreak}

\begin{absolutelynopagebreak}
\setstretch{.7}
{\PaliGlossA{Tassa yugādhāne kāraṇaṃ kāriyamānassa hontiyeva visūkāyitāni visevitāni vipphanditāni kānici kānici, yathā taṃ akāritapubbaṃ kāraṇaṃ kāriyamānassa.}}\\
\begin{addmargin}[1em]{2em}
\setstretch{.5}
{\PaliGlossB{Because it has not done this before, it still resorts to some tricks, dodges, and evasions.}}\\
\end{addmargin}
\end{absolutelynopagebreak}

\begin{absolutelynopagebreak}
\setstretch{.7}
{\PaliGlossA{So abhiṇhakāraṇā anupubbakāraṇā tasmiṃ ṭhāne parinibbāyati.}}\\
\begin{addmargin}[1em]{2em}
\setstretch{.5}
{\PaliGlossB{But with regular and gradual practice it quells that bad habit.}}\\
\end{addmargin}
\end{absolutelynopagebreak}

\begin{absolutelynopagebreak}
\setstretch{.7}
{\PaliGlossA{Yato kho, bhaddāli, bhadro assājānīyo abhiṇhakāraṇā anupubbakāraṇā tasmiṃ ṭhāne parinibbuto hoti, tamenaṃ assadamako uttari kāraṇaṃ kāreti anukkame maṇḍale khurakāse dhāve davatte rājaguṇe rājavaṃse uttame jave uttame haye uttame sākhalye.}}\\
\begin{addmargin}[1em]{2em}
\setstretch{.5}
{\PaliGlossB{When it has done this, the horse trainer next makes it get used to walking in procession, circling, prancing, galloping, charging, the protocols and traditions of court, and in the very best speed, fleetness, and friendliness.}}\\
\end{addmargin}
\end{absolutelynopagebreak}

\begin{absolutelynopagebreak}
\setstretch{.7}
{\PaliGlossA{Tassa uttame jave uttame haye uttame sākhalye kāraṇaṃ kāriyamānassa hontiyeva visūkāyitāni visevitāni vipphanditāni kānici kānici, yathā taṃ akāritapubbaṃ kāraṇaṃ kāriyamānassa.}}\\
\begin{addmargin}[1em]{2em}
\setstretch{.5}
{\PaliGlossB{Because it has not done this before, it still resorts to some tricks, dodges, and evasions.}}\\
\end{addmargin}
\end{absolutelynopagebreak}

\begin{absolutelynopagebreak}
\setstretch{.7}
{\PaliGlossA{So abhiṇhakāraṇā anupubbakāraṇā tasmiṃ ṭhāne parinibbāyati.}}\\
\begin{addmargin}[1em]{2em}
\setstretch{.5}
{\PaliGlossB{But with regular and gradual practice it quells that bad habit.}}\\
\end{addmargin}
\end{absolutelynopagebreak}

\begin{absolutelynopagebreak}
\setstretch{.7}
{\PaliGlossA{Yato kho, bhaddāli, bhadro assājānīyo abhiṇhakāraṇā anupubbakāraṇā tasmiṃ ṭhāne parinibbuto hoti, tamenaṃ assadamako uttari vaṇṇiyañca pāṇiyañca anuppavecchati.}}\\
\begin{addmargin}[1em]{2em}
\setstretch{.5}
{\PaliGlossB{When it has done this, the horse trainer next rewards it with a grooming and a rub down.}}\\
\end{addmargin}
\end{absolutelynopagebreak}

\begin{absolutelynopagebreak}
\setstretch{.7}
{\PaliGlossA{Imehi kho, bhaddāli, dasahaṅgehi samannāgato bhadro assājānīyo rājāraho hoti rājabhoggo rañño aṅganteva saṅkhyaṃ gacchati.}}\\
\begin{addmargin}[1em]{2em}
\setstretch{.5}
{\PaliGlossB{A fine royal thoroughbred with these ten factors is worthy of a king, fit to serve a king, and reckoned as a factor of kingship.}}\\
\end{addmargin}
\end{absolutelynopagebreak}

\vskip 0.05in
\begin{absolutelynopagebreak}
\setstretch{.7}
{\PaliGlossA{Evameva kho, bhaddāli, dasahi dhammehi samannāgato bhikkhu āhuneyyo hoti pāhuneyyo dakkhiṇeyyo añjalikaraṇīyo anuttaraṃ puññakkhettaṃ lokassa.}}\\
\begin{addmargin}[1em]{2em}
\setstretch{.5}
{\PaliGlossB{In the same way, a mendicant with ten qualities is worthy of offerings dedicated to the gods, worthy of hospitality, worthy of a religious donation, worthy of veneration with joined palms, and is the supreme field of merit for the world.}}\\
\end{addmargin}
\end{absolutelynopagebreak}

\begin{absolutelynopagebreak}
\setstretch{.7}
{\PaliGlossA{Katamehi dasahi?}}\\
\begin{addmargin}[1em]{2em}
\setstretch{.5}
{\PaliGlossB{What ten?}}\\
\end{addmargin}
\end{absolutelynopagebreak}

\begin{absolutelynopagebreak}
\setstretch{.7}
{\PaliGlossA{Idha, bhaddāli, bhikkhu asekhāya sammādiṭṭhiyā samannāgato hoti, asekhena sammāsaṅkappena samannāgato hoti, asekhāya sammāvācāya samannāgato hoti, asekhena sammākammantena samannāgato hoti, asekhena sammāājīvena samannāgato hoti, asekhena sammāvāyāmena samannāgato hoti, asekhāya sammāsatiyā samannāgato hoti, asekhena sammāsamādhinā samannāgato hoti, asekhena sammāñāṇena samannāgato hoti, asekhāya sammāvimuttiyā samannāgato hoti—}}\\
\begin{addmargin}[1em]{2em}
\setstretch{.5}
{\PaliGlossB{It’s when a mendicant has an adept’s right view, right thought, right speech, right action, right livelihood, right effort, right mindfulness, right immersion, right knowledge, and right freedom.}}\\
\end{addmargin}
\end{absolutelynopagebreak}

\begin{absolutelynopagebreak}
\setstretch{.7}
{\PaliGlossA{imehi kho, bhaddāli, dasahi dhammehi samannāgato bhikkhu āhuneyyo hoti pāhuneyyo dakkhiṇeyyo añjalikaraṇīyo anuttaraṃ puññakkhettaṃ lokassā”ti.}}\\
\begin{addmargin}[1em]{2em}
\setstretch{.5}
{\PaliGlossB{A mendicant with these ten factors is worthy of offerings dedicated to the gods, worthy of hospitality, worthy of a religious donation, worthy of veneration with joined palms, and is the supreme field of merit for the world.”}}\\
\end{addmargin}
\end{absolutelynopagebreak}

\begin{absolutelynopagebreak}
\setstretch{.7}
{\PaliGlossA{Idamavoca bhagavā.}}\\
\begin{addmargin}[1em]{2em}
\setstretch{.5}
{\PaliGlossB{That is what the Buddha said.}}\\
\end{addmargin}
\end{absolutelynopagebreak}

\begin{absolutelynopagebreak}
\setstretch{.7}
{\PaliGlossA{Attamano āyasmā bhaddāli bhagavato bhāsitaṃ abhinandīti.}}\\
\begin{addmargin}[1em]{2em}
\setstretch{.5}
{\PaliGlossB{Satisfied, Venerable Bhaddāli was happy with what the Buddha said.}}\\
\end{addmargin}
\end{absolutelynopagebreak}

\begin{absolutelynopagebreak}
\setstretch{.7}
{\PaliGlossA{Bhaddālisuttaṃ niṭṭhitaṃ pañcamaṃ.}}\\
\begin{addmargin}[1em]{2em}
\setstretch{.5}
{\PaliGlossB{    -}}\\
\end{addmargin}
\end{absolutelynopagebreak}
