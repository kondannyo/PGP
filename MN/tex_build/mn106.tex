
\vskip 0.05in
\begin{absolutelynopagebreak}
\setstretch{.7}
{\PaliGlossA{Majjhima Nikāya 106}}\\
\begin{addmargin}[1em]{2em}
\setstretch{.5}
{\PaliGlossB{Middle Discourses 106}}\\
\end{addmargin}
\end{absolutelynopagebreak}

\begin{absolutelynopagebreak}
\setstretch{.7}
{\PaliGlossA{Āneñjasappāyasutta}}\\
\begin{addmargin}[1em]{2em}
\setstretch{.5}
{\PaliGlossB{Conducive to the Imperturbable}}\\
\end{addmargin}
\end{absolutelynopagebreak}

\vskip 0.05in
\begin{absolutelynopagebreak}
\setstretch{.7}
{\PaliGlossA{1. Evaṃ me sutaṃ—}}\\
\begin{addmargin}[1em]{2em}
\setstretch{.5}
{\PaliGlossB{So I have heard.}}\\
\end{addmargin}
\end{absolutelynopagebreak}

\begin{absolutelynopagebreak}
\setstretch{.7}
{\PaliGlossA{ekaṃ samayaṃ bhagavā kurūsu viharati kammāsadhammaṃ nāma kurūnaṃ nigamo.}}\\
\begin{addmargin}[1em]{2em}
\setstretch{.5}
{\PaliGlossB{At one time the Buddha was staying in the land of the Kurus, near the Kuru town named Kammāsadamma.}}\\
\end{addmargin}
\end{absolutelynopagebreak}

\begin{absolutelynopagebreak}
\setstretch{.7}
{\PaliGlossA{Tatra kho bhagavā bhikkhū āmantesi:}}\\
\begin{addmargin}[1em]{2em}
\setstretch{.5}
{\PaliGlossB{There the Buddha addressed the mendicants,}}\\
\end{addmargin}
\end{absolutelynopagebreak}

\begin{absolutelynopagebreak}
\setstretch{.7}
{\PaliGlossA{“bhikkhavo”ti.}}\\
\begin{addmargin}[1em]{2em}
\setstretch{.5}
{\PaliGlossB{“Mendicants!”}}\\
\end{addmargin}
\end{absolutelynopagebreak}

\begin{absolutelynopagebreak}
\setstretch{.7}
{\PaliGlossA{“Bhadante”ti te bhikkhū bhagavato paccassosuṃ.}}\\
\begin{addmargin}[1em]{2em}
\setstretch{.5}
{\PaliGlossB{“Venerable sir,” they replied.}}\\
\end{addmargin}
\end{absolutelynopagebreak}

\begin{absolutelynopagebreak}
\setstretch{.7}
{\PaliGlossA{Bhagavā etadavoca:}}\\
\begin{addmargin}[1em]{2em}
\setstretch{.5}
{\PaliGlossB{The Buddha said this:}}\\
\end{addmargin}
\end{absolutelynopagebreak}

\vskip 0.05in
\begin{absolutelynopagebreak}
\setstretch{.7}
{\PaliGlossA{2. “Aniccā, bhikkhave, kāmā tucchā musā mosadhammā.}}\\
\begin{addmargin}[1em]{2em}
\setstretch{.5}
{\PaliGlossB{“Mendicants, sensual pleasures are impermanent, hollow, false, and deceptive,}}\\
\end{addmargin}
\end{absolutelynopagebreak}

\begin{absolutelynopagebreak}
\setstretch{.7}
{\PaliGlossA{Māyākatametaṃ, bhikkhave, bālalāpanaṃ.}}\\
\begin{addmargin}[1em]{2em}
\setstretch{.5}
{\PaliGlossB{made by illusion, cooed over by fools.}}\\
\end{addmargin}
\end{absolutelynopagebreak}

\begin{absolutelynopagebreak}
\setstretch{.7}
{\PaliGlossA{Ye ca diṭṭhadhammikā kāmā, ye ca samparāyikā kāmā;}}\\
\begin{addmargin}[1em]{2em}
\setstretch{.5}
{\PaliGlossB{Sensual pleasures in this life and in lives to come,}}\\
\end{addmargin}
\end{absolutelynopagebreak}

\begin{absolutelynopagebreak}
\setstretch{.7}
{\PaliGlossA{yā ca diṭṭhadhammikā kāmasaññā, yā ca samparāyikā kāmasaññā—}}\\
\begin{addmargin}[1em]{2em}
\setstretch{.5}
{\PaliGlossB{sensual perceptions in this life and in lives to come;}}\\
\end{addmargin}
\end{absolutelynopagebreak}

\begin{absolutelynopagebreak}
\setstretch{.7}
{\PaliGlossA{ubhayametaṃ māradheyyaṃ, mārassesa visayo, mārassesa nivāpo, mārassesa gocaro.}}\\
\begin{addmargin}[1em]{2em}
\setstretch{.5}
{\PaliGlossB{both of these are Māra’s sovereignty, Māra’s domain, and Māra’s territory.}}\\
\end{addmargin}
\end{absolutelynopagebreak}

\begin{absolutelynopagebreak}
\setstretch{.7}
{\PaliGlossA{Etthete pāpakā akusalā mānasā abhijjhāpi byāpādāpi sārambhāpi saṃvattanti.}}\\
\begin{addmargin}[1em]{2em}
\setstretch{.5}
{\PaliGlossB{They conduce to bad, unskillful qualities such as desire, ill will, and aggression.}}\\
\end{addmargin}
\end{absolutelynopagebreak}

\begin{absolutelynopagebreak}
\setstretch{.7}
{\PaliGlossA{Teva ariyasāvakassa idhamanusikkhato antarāyāya sambhavanti.}}\\
\begin{addmargin}[1em]{2em}
\setstretch{.5}
{\PaliGlossB{And they create an obstacle for a noble disciple training here.}}\\
\end{addmargin}
\end{absolutelynopagebreak}

\vskip 0.05in
\begin{absolutelynopagebreak}
\setstretch{.7}
{\PaliGlossA{3. Tatra, bhikkhave, ariyasāvako iti paṭisañcikkhati:}}\\
\begin{addmargin}[1em]{2em}
\setstretch{.5}
{\PaliGlossB{A noble disciple reflects on this:}}\\
\end{addmargin}
\end{absolutelynopagebreak}

\begin{absolutelynopagebreak}
\setstretch{.7}
{\PaliGlossA{‘ye ca diṭṭhadhammikā kāmā, ye ca samparāyikā kāmā;}}\\
\begin{addmargin}[1em]{2em}
\setstretch{.5}
{\PaliGlossB{‘Sensual pleasures in this life and in lives to come,}}\\
\end{addmargin}
\end{absolutelynopagebreak}

\begin{absolutelynopagebreak}
\setstretch{.7}
{\PaliGlossA{yā ca diṭṭhadhammikā kāmasaññā, yā ca samparāyikā kāmasaññā—}}\\
\begin{addmargin}[1em]{2em}
\setstretch{.5}
{\PaliGlossB{sensual perceptions in this life and in lives to come;}}\\
\end{addmargin}
\end{absolutelynopagebreak}

\begin{absolutelynopagebreak}
\setstretch{.7}
{\PaliGlossA{ubhayametaṃ māradheyyaṃ, mārassesa visayo, mārassesa nivāpo, mārassesa gocaro.}}\\
\begin{addmargin}[1em]{2em}
\setstretch{.5}
{\PaliGlossB{both of these are Māra’s sovereignty, Māra’s domain, and Māra’s territory.}}\\
\end{addmargin}
\end{absolutelynopagebreak}

\begin{absolutelynopagebreak}
\setstretch{.7}
{\PaliGlossA{Etthete pāpakā akusalā mānasā abhijjhāpi byāpādāpi sārambhāpi saṃvattanti,}}\\
\begin{addmargin}[1em]{2em}
\setstretch{.5}
{\PaliGlossB{They conduce to bad, unskillful qualities such as desire, ill will, and aggression.}}\\
\end{addmargin}
\end{absolutelynopagebreak}

\begin{absolutelynopagebreak}
\setstretch{.7}
{\PaliGlossA{teva ariyasāvakassa idhamanusikkhato antarāyāya sambhavanti.}}\\
\begin{addmargin}[1em]{2em}
\setstretch{.5}
{\PaliGlossB{And they create an obstacle for a noble disciple training here.}}\\
\end{addmargin}
\end{absolutelynopagebreak}

\begin{absolutelynopagebreak}
\setstretch{.7}
{\PaliGlossA{Yannūnāhaṃ vipulena mahaggatena cetasā vihareyyaṃ abhibhuyya lokaṃ adhiṭṭhāya manasā.}}\\
\begin{addmargin}[1em]{2em}
\setstretch{.5}
{\PaliGlossB{Why don’t I meditate with an abundant, expansive heart, having mastered the world and stabilized the mind?}}\\
\end{addmargin}
\end{absolutelynopagebreak}

\begin{absolutelynopagebreak}
\setstretch{.7}
{\PaliGlossA{Vipulena hi me mahaggatena cetasā viharato abhibhuyya lokaṃ adhiṭṭhāya manasā ye pāpakā akusalā mānasā abhijjhāpi byāpādāpi sārambhāpi te na bhavissanti.}}\\
\begin{addmargin}[1em]{2em}
\setstretch{.5}
{\PaliGlossB{Then I will have no more bad, unskillful qualities such as desire, ill will, and aggression.}}\\
\end{addmargin}
\end{absolutelynopagebreak}

\begin{absolutelynopagebreak}
\setstretch{.7}
{\PaliGlossA{Tesaṃ pahānā aparittañca me cittaṃ bhavissati appamāṇaṃ subhāvitan’ti.}}\\
\begin{addmargin}[1em]{2em}
\setstretch{.5}
{\PaliGlossB{And by giving them up my mind, no longer limited, will become limitless and well developed.’}}\\
\end{addmargin}
\end{absolutelynopagebreak}

\begin{absolutelynopagebreak}
\setstretch{.7}
{\PaliGlossA{Tassa evaṃpaṭipannassa tabbahulavihārino āyatane cittaṃ pasīdati.}}\\
\begin{addmargin}[1em]{2em}
\setstretch{.5}
{\PaliGlossB{Practicing in this way and meditating on it often their mind becomes confident in this dimension.}}\\
\end{addmargin}
\end{absolutelynopagebreak}

\begin{absolutelynopagebreak}
\setstretch{.7}
{\PaliGlossA{Sampasāde sati etarahi vā āneñjaṃ samāpajjati paññāya vā adhimuccati kāyassa bhedā paraṃ maraṇā.}}\\
\begin{addmargin}[1em]{2em}
\setstretch{.5}
{\PaliGlossB{Being confident, they either attain the imperturbable now, or are freed by wisdom.}}\\
\end{addmargin}
\end{absolutelynopagebreak}

\begin{absolutelynopagebreak}
\setstretch{.7}
{\PaliGlossA{Ṭhānametaṃ vijjati yaṃ taṃsaṃvattanikaṃ viññāṇaṃ assa āneñjūpagaṃ.}}\\
\begin{addmargin}[1em]{2em}
\setstretch{.5}
{\PaliGlossB{When their body breaks up, after death, it’s possible that the consciousness headed that way will be reborn in the imperturbable.}}\\
\end{addmargin}
\end{absolutelynopagebreak}

\begin{absolutelynopagebreak}
\setstretch{.7}
{\PaliGlossA{Ayaṃ, bhikkhave, paṭhamā āneñjasappāyā paṭipadā akkhāyati.}}\\
\begin{addmargin}[1em]{2em}
\setstretch{.5}
{\PaliGlossB{This is said to be the first way of practice suitable for attaining the imperturbable.}}\\
\end{addmargin}
\end{absolutelynopagebreak}

\vskip 0.05in
\begin{absolutelynopagebreak}
\setstretch{.7}
{\PaliGlossA{4. Puna caparaṃ, bhikkhave, ariyasāvako iti paṭisañcikkhati:}}\\
\begin{addmargin}[1em]{2em}
\setstretch{.5}
{\PaliGlossB{Furthermore, a noble disciple reflects:}}\\
\end{addmargin}
\end{absolutelynopagebreak}

\begin{absolutelynopagebreak}
\setstretch{.7}
{\PaliGlossA{‘ye ca diṭṭhadhammikā kāmā, ye ca samparāyikā kāmā;}}\\
\begin{addmargin}[1em]{2em}
\setstretch{.5}
{\PaliGlossB{‘Sensual pleasures in this life and in lives to come,}}\\
\end{addmargin}
\end{absolutelynopagebreak}

\begin{absolutelynopagebreak}
\setstretch{.7}
{\PaliGlossA{yā ca diṭṭhadhammikā kāmasaññā, yā ca samparāyikā kāmasaññā;}}\\
\begin{addmargin}[1em]{2em}
\setstretch{.5}
{\PaliGlossB{sensual perceptions in this life and in lives to come;}}\\
\end{addmargin}
\end{absolutelynopagebreak}

\begin{absolutelynopagebreak}
\setstretch{.7}
{\PaliGlossA{yaṃ kiñci rūpaṃ sabbaṃ rūpaṃ cattāri ca mahābhūtāni, catunnañca mahābhūtānaṃ upādāyarūpan’ti.}}\\
\begin{addmargin}[1em]{2em}
\setstretch{.5}
{\PaliGlossB{whatever is form, all form is the four primary elements, or form derived from the four primary elements.’}}\\
\end{addmargin}
\end{absolutelynopagebreak}

\begin{absolutelynopagebreak}
\setstretch{.7}
{\PaliGlossA{Tassa evaṃpaṭipannassa tabbahulavihārino āyatane cittaṃ pasīdati.}}\\
\begin{addmargin}[1em]{2em}
\setstretch{.5}
{\PaliGlossB{Practicing in this way and meditating on it often their mind becomes confident in this dimension.}}\\
\end{addmargin}
\end{absolutelynopagebreak}

\begin{absolutelynopagebreak}
\setstretch{.7}
{\PaliGlossA{Sampasāde sati etarahi vā āneñjaṃ samāpajjati paññāya vā adhimuccati kāyassa bhedā paraṃ maraṇā.}}\\
\begin{addmargin}[1em]{2em}
\setstretch{.5}
{\PaliGlossB{Being confident, they either attain the imperturbable now, or are freed by wisdom.}}\\
\end{addmargin}
\end{absolutelynopagebreak}

\begin{absolutelynopagebreak}
\setstretch{.7}
{\PaliGlossA{Ṭhānametaṃ vijjati yaṃ taṃsaṃvattanikaṃ viññāṇaṃ assa āneñjūpagaṃ.}}\\
\begin{addmargin}[1em]{2em}
\setstretch{.5}
{\PaliGlossB{When their body breaks up, after death, it’s possible that the consciousness headed that way will be reborn in the imperturbable.}}\\
\end{addmargin}
\end{absolutelynopagebreak}

\begin{absolutelynopagebreak}
\setstretch{.7}
{\PaliGlossA{Ayaṃ, bhikkhave, dutiyā āneñjasappāyā paṭipadā akkhāyati.}}\\
\begin{addmargin}[1em]{2em}
\setstretch{.5}
{\PaliGlossB{This is said to be the second way of practice suitable for attaining the imperturbable.}}\\
\end{addmargin}
\end{absolutelynopagebreak}

\vskip 0.05in
\begin{absolutelynopagebreak}
\setstretch{.7}
{\PaliGlossA{5. Puna caparaṃ, bhikkhave, ariyasāvako iti paṭisañcikkhati:}}\\
\begin{addmargin}[1em]{2em}
\setstretch{.5}
{\PaliGlossB{Furthermore, a noble disciple reflects:}}\\
\end{addmargin}
\end{absolutelynopagebreak}

\begin{absolutelynopagebreak}
\setstretch{.7}
{\PaliGlossA{‘ye ca diṭṭhadhammikā kāmā, ye ca samparāyikā kāmā;}}\\
\begin{addmargin}[1em]{2em}
\setstretch{.5}
{\PaliGlossB{‘Sensual pleasures in this life and in lives to come,}}\\
\end{addmargin}
\end{absolutelynopagebreak}

\begin{absolutelynopagebreak}
\setstretch{.7}
{\PaliGlossA{yā ca diṭṭhadhammikā kāmasaññā, yā ca samparāyikā kāmasaññā;}}\\
\begin{addmargin}[1em]{2em}
\setstretch{.5}
{\PaliGlossB{sensual perceptions in this life and in lives to come,}}\\
\end{addmargin}
\end{absolutelynopagebreak}

\begin{absolutelynopagebreak}
\setstretch{.7}
{\PaliGlossA{ye ca diṭṭhadhammikā rūpā, ye ca samparāyikā rūpā;}}\\
\begin{addmargin}[1em]{2em}
\setstretch{.5}
{\PaliGlossB{visions in this life and in lives to come,}}\\
\end{addmargin}
\end{absolutelynopagebreak}

\begin{absolutelynopagebreak}
\setstretch{.7}
{\PaliGlossA{yā ca diṭṭhadhammikā rūpasaññā, yā ca samparāyikā rūpasaññā—}}\\
\begin{addmargin}[1em]{2em}
\setstretch{.5}
{\PaliGlossB{perceptions of visions in this life and in lives to come;}}\\
\end{addmargin}
\end{absolutelynopagebreak}

\begin{absolutelynopagebreak}
\setstretch{.7}
{\PaliGlossA{ubhayametaṃ aniccaṃ.}}\\
\begin{addmargin}[1em]{2em}
\setstretch{.5}
{\PaliGlossB{all of these are impermanent.}}\\
\end{addmargin}
\end{absolutelynopagebreak}

\begin{absolutelynopagebreak}
\setstretch{.7}
{\PaliGlossA{Yadaniccaṃ taṃ nālaṃ abhinandituṃ, nālaṃ abhivadituṃ, nālaṃ ajjhositun’ti.}}\\
\begin{addmargin}[1em]{2em}
\setstretch{.5}
{\PaliGlossB{And what’s impermanent is not worth approving, welcoming, or clinging to.’}}\\
\end{addmargin}
\end{absolutelynopagebreak}

\begin{absolutelynopagebreak}
\setstretch{.7}
{\PaliGlossA{Tassa evaṃpaṭipannassa tabbahulavihārino āyatane cittaṃ pasīdati.}}\\
\begin{addmargin}[1em]{2em}
\setstretch{.5}
{\PaliGlossB{Practicing in this way and meditating on it often their mind becomes confident in this dimension.}}\\
\end{addmargin}
\end{absolutelynopagebreak}

\begin{absolutelynopagebreak}
\setstretch{.7}
{\PaliGlossA{Sampasāde sati etarahi vā āneñjaṃ samāpajjati paññāya vā adhimuccati kāyassa bhedā paraṃ maraṇā.}}\\
\begin{addmargin}[1em]{2em}
\setstretch{.5}
{\PaliGlossB{Being confident, they either attain the imperturbable now, or are freed by wisdom.}}\\
\end{addmargin}
\end{absolutelynopagebreak}

\begin{absolutelynopagebreak}
\setstretch{.7}
{\PaliGlossA{Ṭhānametaṃ vijjati yaṃ taṃsaṃvattanikaṃ viññāṇaṃ assa āneñjūpagaṃ.}}\\
\begin{addmargin}[1em]{2em}
\setstretch{.5}
{\PaliGlossB{When their body breaks up, after death, it’s possible that the consciousness headed that way will be reborn in the imperturbable.}}\\
\end{addmargin}
\end{absolutelynopagebreak}

\begin{absolutelynopagebreak}
\setstretch{.7}
{\PaliGlossA{Ayaṃ, bhikkhave, tatiyā āneñjasappāyā paṭipadā akkhāyati.}}\\
\begin{addmargin}[1em]{2em}
\setstretch{.5}
{\PaliGlossB{This is said to be the third way of practice suitable for attaining the imperturbable.}}\\
\end{addmargin}
\end{absolutelynopagebreak}

\vskip 0.05in
\begin{absolutelynopagebreak}
\setstretch{.7}
{\PaliGlossA{6. Puna caparaṃ, bhikkhave, ariyasāvako iti paṭisañcikkhati:}}\\
\begin{addmargin}[1em]{2em}
\setstretch{.5}
{\PaliGlossB{Furthermore, a noble disciple reflects:}}\\
\end{addmargin}
\end{absolutelynopagebreak}

\begin{absolutelynopagebreak}
\setstretch{.7}
{\PaliGlossA{‘ye ca diṭṭhadhammikā kāmā, ye ca samparāyikā kāmā;}}\\
\begin{addmargin}[1em]{2em}
\setstretch{.5}
{\PaliGlossB{‘Sensual pleasures in this life and in lives to come,}}\\
\end{addmargin}
\end{absolutelynopagebreak}

\begin{absolutelynopagebreak}
\setstretch{.7}
{\PaliGlossA{yā ca diṭṭhadhammikā kāmasaññā, yā ca samparāyikā kāmasaññā;}}\\
\begin{addmargin}[1em]{2em}
\setstretch{.5}
{\PaliGlossB{sensual perceptions in this life and in lives to come,}}\\
\end{addmargin}
\end{absolutelynopagebreak}

\begin{absolutelynopagebreak}
\setstretch{.7}
{\PaliGlossA{ye ca diṭṭhadhammikā rūpā, ye ca samparāyikā rūpā;}}\\
\begin{addmargin}[1em]{2em}
\setstretch{.5}
{\PaliGlossB{visions in this life and in lives to come,}}\\
\end{addmargin}
\end{absolutelynopagebreak}

\begin{absolutelynopagebreak}
\setstretch{.7}
{\PaliGlossA{yā ca diṭṭhadhammikā rūpasaññā, yā ca samparāyikā rūpasaññā;}}\\
\begin{addmargin}[1em]{2em}
\setstretch{.5}
{\PaliGlossB{perceptions of visions in this life and in lives to come,}}\\
\end{addmargin}
\end{absolutelynopagebreak}

\begin{absolutelynopagebreak}
\setstretch{.7}
{\PaliGlossA{yā ca āneñjasaññā—}}\\
\begin{addmargin}[1em]{2em}
\setstretch{.5}
{\PaliGlossB{and perceptions of the imperturbable;}}\\
\end{addmargin}
\end{absolutelynopagebreak}

\begin{absolutelynopagebreak}
\setstretch{.7}
{\PaliGlossA{sabbā saññā.}}\\
\begin{addmargin}[1em]{2em}
\setstretch{.5}
{\PaliGlossB{all are perceptions.}}\\
\end{addmargin}
\end{absolutelynopagebreak}

\begin{absolutelynopagebreak}
\setstretch{.7}
{\PaliGlossA{Yatthetā aparisesā nirujjhanti etaṃ santaṃ etaṃ paṇītaṃ—}}\\
\begin{addmargin}[1em]{2em}
\setstretch{.5}
{\PaliGlossB{Where they cease without anything left over, that is peaceful, that is sublime,}}\\
\end{addmargin}
\end{absolutelynopagebreak}

\begin{absolutelynopagebreak}
\setstretch{.7}
{\PaliGlossA{yadidaṃ ākiñcaññāyatanan’ti.}}\\
\begin{addmargin}[1em]{2em}
\setstretch{.5}
{\PaliGlossB{namely the dimension of nothingness.’}}\\
\end{addmargin}
\end{absolutelynopagebreak}

\begin{absolutelynopagebreak}
\setstretch{.7}
{\PaliGlossA{Tassa evaṃpaṭipannassa tabbahulavihārino āyatane cittaṃ pasīdati.}}\\
\begin{addmargin}[1em]{2em}
\setstretch{.5}
{\PaliGlossB{Practicing in this way and meditating on it often their mind becomes confident in this dimension.}}\\
\end{addmargin}
\end{absolutelynopagebreak}

\begin{absolutelynopagebreak}
\setstretch{.7}
{\PaliGlossA{Sampasāde sati etarahi vā ākiñcaññāyatanaṃ samāpajjati paññāya vā adhimuccati kāyassa bhedā paraṃ maraṇā.}}\\
\begin{addmargin}[1em]{2em}
\setstretch{.5}
{\PaliGlossB{Being confident, they either attain the dimension of nothingness now, or are freed by wisdom.}}\\
\end{addmargin}
\end{absolutelynopagebreak}

\begin{absolutelynopagebreak}
\setstretch{.7}
{\PaliGlossA{Ṭhānametaṃ vijjati yaṃ taṃsaṃvattanikaṃ viññāṇaṃ assa ākiñcaññāyatanūpagaṃ.}}\\
\begin{addmargin}[1em]{2em}
\setstretch{.5}
{\PaliGlossB{When their body breaks up, after death, it’s possible that the consciousness headed that way will be reborn in the dimension of nothingness.}}\\
\end{addmargin}
\end{absolutelynopagebreak}

\begin{absolutelynopagebreak}
\setstretch{.7}
{\PaliGlossA{Ayaṃ, bhikkhave, paṭhamā ākiñcaññāyatanasappāyā paṭipadā akkhāyati.}}\\
\begin{addmargin}[1em]{2em}
\setstretch{.5}
{\PaliGlossB{This is said to be the first way of practice suitable for attaining the dimension of nothingness.}}\\
\end{addmargin}
\end{absolutelynopagebreak}

\vskip 0.05in
\begin{absolutelynopagebreak}
\setstretch{.7}
{\PaliGlossA{7. Puna caparaṃ, bhikkhave, ariyasāvako araññagato vā rukkhamūlagato vā suññāgāragato vā iti paṭisañcikkhati:}}\\
\begin{addmargin}[1em]{2em}
\setstretch{.5}
{\PaliGlossB{Furthermore, a noble disciple has gone to a wilderness, or to the root of a tree, or to an empty hut, and reflects like this:}}\\
\end{addmargin}
\end{absolutelynopagebreak}

\begin{absolutelynopagebreak}
\setstretch{.7}
{\PaliGlossA{‘suññamidaṃ attena vā attaniyena vā’ti.}}\\
\begin{addmargin}[1em]{2em}
\setstretch{.5}
{\PaliGlossB{‘This is empty of a self or what belongs to a self.’}}\\
\end{addmargin}
\end{absolutelynopagebreak}

\begin{absolutelynopagebreak}
\setstretch{.7}
{\PaliGlossA{Tassa evaṃpaṭipannassa tabbahulavihārino āyatane cittaṃ pasīdati.}}\\
\begin{addmargin}[1em]{2em}
\setstretch{.5}
{\PaliGlossB{Practicing in this way and meditating on it often their mind becomes confident in this dimension.}}\\
\end{addmargin}
\end{absolutelynopagebreak}

\begin{absolutelynopagebreak}
\setstretch{.7}
{\PaliGlossA{Sampasāde sati etarahi vā ākiñcaññāyatanaṃ samāpajjati paññāya vā adhimuccati kāyassa bhedā paraṃ maraṇā.}}\\
\begin{addmargin}[1em]{2em}
\setstretch{.5}
{\PaliGlossB{Being confident, they either attain the dimension of nothingness now, or are freed by wisdom.}}\\
\end{addmargin}
\end{absolutelynopagebreak}

\begin{absolutelynopagebreak}
\setstretch{.7}
{\PaliGlossA{Ṭhānametaṃ vijjati yaṃ taṃsaṃvattanikaṃ viññāṇaṃ assa ākiñcaññāyatanūpagaṃ.}}\\
\begin{addmargin}[1em]{2em}
\setstretch{.5}
{\PaliGlossB{When their body breaks up, after death, it’s possible that the consciousness headed that way will be reborn in the dimension of nothingness.}}\\
\end{addmargin}
\end{absolutelynopagebreak}

\begin{absolutelynopagebreak}
\setstretch{.7}
{\PaliGlossA{Ayaṃ, bhikkhave, dutiyā ākiñcaññāyatanasappāyā paṭipadā akkhāyati.}}\\
\begin{addmargin}[1em]{2em}
\setstretch{.5}
{\PaliGlossB{This is said to be the second way of practice suitable for attaining the dimension of nothingness.}}\\
\end{addmargin}
\end{absolutelynopagebreak}

\vskip 0.05in
\begin{absolutelynopagebreak}
\setstretch{.7}
{\PaliGlossA{8. Puna caparaṃ, bhikkhave, ariyasāvako iti paṭisañcikkhati:}}\\
\begin{addmargin}[1em]{2em}
\setstretch{.5}
{\PaliGlossB{Furthermore, a noble disciple reflects:}}\\
\end{addmargin}
\end{absolutelynopagebreak}

\begin{absolutelynopagebreak}
\setstretch{.7}
{\PaliGlossA{‘nāhaṃ kvacani kassaci kiñcanatasmiṃ, na ca mama kvacani kismiñci kiñcanaṃ natthī’ti.}}\\
\begin{addmargin}[1em]{2em}
\setstretch{.5}
{\PaliGlossB{‘I don’t belong to anyone anywhere! And nothing belongs to me anywhere!’}}\\
\end{addmargin}
\end{absolutelynopagebreak}

\begin{absolutelynopagebreak}
\setstretch{.7}
{\PaliGlossA{Tassa evaṃpaṭipannassa tabbahulavihārino āyatane cittaṃ pasīdati.}}\\
\begin{addmargin}[1em]{2em}
\setstretch{.5}
{\PaliGlossB{Practicing in this way and meditating on it often their mind becomes confident in this dimension.}}\\
\end{addmargin}
\end{absolutelynopagebreak}

\begin{absolutelynopagebreak}
\setstretch{.7}
{\PaliGlossA{Sampasāde sati etarahi vā ākiñcaññāyatanaṃ samāpajjati paññāya vā adhimuccati kāyassa bhedā paraṃ maraṇā.}}\\
\begin{addmargin}[1em]{2em}
\setstretch{.5}
{\PaliGlossB{Being confident, they either attain the dimension of nothingness now, or are freed by wisdom.}}\\
\end{addmargin}
\end{absolutelynopagebreak}

\begin{absolutelynopagebreak}
\setstretch{.7}
{\PaliGlossA{Ṭhānametaṃ vijjati yaṃ taṃsaṃvattanikaṃ viññāṇaṃ assa ākiñcaññāyatanūpagaṃ.}}\\
\begin{addmargin}[1em]{2em}
\setstretch{.5}
{\PaliGlossB{When their body breaks up, after death, it’s possible that the consciousness headed that way will be reborn in the dimension of nothingness.}}\\
\end{addmargin}
\end{absolutelynopagebreak}

\begin{absolutelynopagebreak}
\setstretch{.7}
{\PaliGlossA{Ayaṃ, bhikkhave, tatiyā ākiñcaññāyatanasappāyā paṭipadā akkhāyati.}}\\
\begin{addmargin}[1em]{2em}
\setstretch{.5}
{\PaliGlossB{This is said to be the third way of practice suitable for attaining the dimension of nothingness.}}\\
\end{addmargin}
\end{absolutelynopagebreak}

\vskip 0.05in
\begin{absolutelynopagebreak}
\setstretch{.7}
{\PaliGlossA{9. Puna caparaṃ, bhikkhave, ariyasāvako iti paṭisañcikkhati:}}\\
\begin{addmargin}[1em]{2em}
\setstretch{.5}
{\PaliGlossB{Furthermore, a noble disciple reflects:}}\\
\end{addmargin}
\end{absolutelynopagebreak}

\begin{absolutelynopagebreak}
\setstretch{.7}
{\PaliGlossA{‘ye ca diṭṭhadhammikā kāmā, ye ca samparāyikā kāmā;}}\\
\begin{addmargin}[1em]{2em}
\setstretch{.5}
{\PaliGlossB{‘Sensual pleasures in this life and in lives to come,}}\\
\end{addmargin}
\end{absolutelynopagebreak}

\begin{absolutelynopagebreak}
\setstretch{.7}
{\PaliGlossA{yā ca diṭṭhadhammikā kāmasaññā, yā ca samparāyikā kāmasaññā;}}\\
\begin{addmargin}[1em]{2em}
\setstretch{.5}
{\PaliGlossB{sensual perceptions in this life and in lives to come,}}\\
\end{addmargin}
\end{absolutelynopagebreak}

\begin{absolutelynopagebreak}
\setstretch{.7}
{\PaliGlossA{ye ca diṭṭhadhammikā rūpā, ye ca samparāyikā rūpā;}}\\
\begin{addmargin}[1em]{2em}
\setstretch{.5}
{\PaliGlossB{visions in this life and in lives to come,}}\\
\end{addmargin}
\end{absolutelynopagebreak}

\begin{absolutelynopagebreak}
\setstretch{.7}
{\PaliGlossA{yā ca diṭṭhadhammikā rūpasaññā, yā ca samparāyikā rūpasaññā;}}\\
\begin{addmargin}[1em]{2em}
\setstretch{.5}
{\PaliGlossB{perceptions of visions in this life and in lives to come,}}\\
\end{addmargin}
\end{absolutelynopagebreak}

\begin{absolutelynopagebreak}
\setstretch{.7}
{\PaliGlossA{yā ca āneñjasaññā, yā ca ākiñcaññāyatanasaññā—}}\\
\begin{addmargin}[1em]{2em}
\setstretch{.5}
{\PaliGlossB{perceptions of the imperturbable, and perceptions of the dimension of nothingness;}}\\
\end{addmargin}
\end{absolutelynopagebreak}

\begin{absolutelynopagebreak}
\setstretch{.7}
{\PaliGlossA{sabbā saññā.}}\\
\begin{addmargin}[1em]{2em}
\setstretch{.5}
{\PaliGlossB{all are perceptions.}}\\
\end{addmargin}
\end{absolutelynopagebreak}

\begin{absolutelynopagebreak}
\setstretch{.7}
{\PaliGlossA{Yatthetā aparisesā nirujjhanti etaṃ santaṃ etaṃ paṇītaṃ—}}\\
\begin{addmargin}[1em]{2em}
\setstretch{.5}
{\PaliGlossB{Where they cease without anything left over, that is peaceful, that is sublime,}}\\
\end{addmargin}
\end{absolutelynopagebreak}

\begin{absolutelynopagebreak}
\setstretch{.7}
{\PaliGlossA{yadidaṃ nevasaññānāsaññāyatanan’ti.}}\\
\begin{addmargin}[1em]{2em}
\setstretch{.5}
{\PaliGlossB{namely the dimension of neither perception nor non-perception.’}}\\
\end{addmargin}
\end{absolutelynopagebreak}

\begin{absolutelynopagebreak}
\setstretch{.7}
{\PaliGlossA{Tassa evaṃpaṭipannassa tabbahulavihārino āyatane cittaṃ pasīdati.}}\\
\begin{addmargin}[1em]{2em}
\setstretch{.5}
{\PaliGlossB{Practicing in this way and meditating on it often their mind becomes confident in this dimension.}}\\
\end{addmargin}
\end{absolutelynopagebreak}

\begin{absolutelynopagebreak}
\setstretch{.7}
{\PaliGlossA{Sampasāde sati etarahi vā nevasaññānāsaññāyatanaṃ samāpajjati paññāya vā adhimuccati kāyassa bhedā paraṃ maraṇā.}}\\
\begin{addmargin}[1em]{2em}
\setstretch{.5}
{\PaliGlossB{Being confident, they either attain the dimension of neither perception nor non-perception now, or are freed by wisdom.}}\\
\end{addmargin}
\end{absolutelynopagebreak}

\begin{absolutelynopagebreak}
\setstretch{.7}
{\PaliGlossA{Ṭhānametaṃ vijjati yaṃ taṃsaṃvattanikaṃ viññāṇaṃ assa nevasaññānāsaññāyatanūpagaṃ.}}\\
\begin{addmargin}[1em]{2em}
\setstretch{.5}
{\PaliGlossB{When their body breaks up, after death, it’s possible that the consciousness headed that way will be reborn in the dimension of neither perception nor non-perception.}}\\
\end{addmargin}
\end{absolutelynopagebreak}

\begin{absolutelynopagebreak}
\setstretch{.7}
{\PaliGlossA{Ayaṃ, bhikkhave, nevasaññānāsaññāyatanasappāyā paṭipadā akkhāyatī”ti.}}\\
\begin{addmargin}[1em]{2em}
\setstretch{.5}
{\PaliGlossB{This is said to be the way of practice suitable for attaining the dimension of neither perception nor non-perception.”}}\\
\end{addmargin}
\end{absolutelynopagebreak}

\vskip 0.05in
\begin{absolutelynopagebreak}
\setstretch{.7}
{\PaliGlossA{10. Evaṃ vutte, āyasmā ānando bhagavantaṃ etadavoca:}}\\
\begin{addmargin}[1em]{2em}
\setstretch{.5}
{\PaliGlossB{When he said this, Venerable Ānanda said to the Buddha:}}\\
\end{addmargin}
\end{absolutelynopagebreak}

\begin{absolutelynopagebreak}
\setstretch{.7}
{\PaliGlossA{“idha, bhante, bhikkhu evaṃ paṭipanno hoti:}}\\
\begin{addmargin}[1em]{2em}
\setstretch{.5}
{\PaliGlossB{“Sir, take a mendicant who practices like this:}}\\
\end{addmargin}
\end{absolutelynopagebreak}

\begin{absolutelynopagebreak}
\setstretch{.7}
{\PaliGlossA{‘no cassa, no ca me siyā; na bhavissati, na me bhavissati; yadatthi yaṃ, bhūtaṃ—taṃ pajahāmī’ti.}}\\
\begin{addmargin}[1em]{2em}
\setstretch{.5}
{\PaliGlossB{‘It might not be, and it might not be mine. It will not be, and it will not be mine. I am giving up what exists, what has come to be.’}}\\
\end{addmargin}
\end{absolutelynopagebreak}

\begin{absolutelynopagebreak}
\setstretch{.7}
{\PaliGlossA{Evaṃ upekkhaṃ paṭilabhati.}}\\
\begin{addmargin}[1em]{2em}
\setstretch{.5}
{\PaliGlossB{In this way they gain equanimity.}}\\
\end{addmargin}
\end{absolutelynopagebreak}

\begin{absolutelynopagebreak}
\setstretch{.7}
{\PaliGlossA{Parinibbāyeyya nu kho so, bhante, bhikkhu na vā parinibbāyeyyā”ti?}}\\
\begin{addmargin}[1em]{2em}
\setstretch{.5}
{\PaliGlossB{Would that mendicant become extinguished or not?”}}\\
\end{addmargin}
\end{absolutelynopagebreak}

\begin{absolutelynopagebreak}
\setstretch{.7}
{\PaliGlossA{“Apetthekacco, ānanda, bhikkhu parinibbāyeyya, apetthekacco bhikkhu na parinibbāyeyyā”ti.}}\\
\begin{addmargin}[1em]{2em}
\setstretch{.5}
{\PaliGlossB{“One such mendicant might become extinguished, Ānanda, while another might not.”}}\\
\end{addmargin}
\end{absolutelynopagebreak}

\begin{absolutelynopagebreak}
\setstretch{.7}
{\PaliGlossA{“Ko nu kho, bhante, hetu ko paccayo yenapetthekacco bhikkhu parinibbāyeyya, apetthekacco bhikkhu na parinibbāyeyyā”ti?}}\\
\begin{addmargin}[1em]{2em}
\setstretch{.5}
{\PaliGlossB{“What is the cause, sir, what is the reason for this?”}}\\
\end{addmargin}
\end{absolutelynopagebreak}

\begin{absolutelynopagebreak}
\setstretch{.7}
{\PaliGlossA{“Idhānanda, bhikkhu evaṃ paṭipanno hoti:}}\\
\begin{addmargin}[1em]{2em}
\setstretch{.5}
{\PaliGlossB{“Ānanda, take a mendicant who practices like this:}}\\
\end{addmargin}
\end{absolutelynopagebreak}

\begin{absolutelynopagebreak}
\setstretch{.7}
{\PaliGlossA{‘no cassa, no ca me siyā; na bhavissati, na me bhavissati; yadatthi, yaṃ bhūtaṃ—taṃ pajahāmī’ti.}}\\
\begin{addmargin}[1em]{2em}
\setstretch{.5}
{\PaliGlossB{‘It might not be, and it might not be mine. It will not be, and it will not be mine. I am giving up what exists, what has come to be.’}}\\
\end{addmargin}
\end{absolutelynopagebreak}

\begin{absolutelynopagebreak}
\setstretch{.7}
{\PaliGlossA{Evaṃ upekkhaṃ paṭilabhati.}}\\
\begin{addmargin}[1em]{2em}
\setstretch{.5}
{\PaliGlossB{In this way they gain equanimity.}}\\
\end{addmargin}
\end{absolutelynopagebreak}

\begin{absolutelynopagebreak}
\setstretch{.7}
{\PaliGlossA{So taṃ upekkhaṃ abhinandati, abhivadati, ajjhosāya tiṭṭhati.}}\\
\begin{addmargin}[1em]{2em}
\setstretch{.5}
{\PaliGlossB{They approve, welcome, and keep clinging to that equanimity.}}\\
\end{addmargin}
\end{absolutelynopagebreak}

\begin{absolutelynopagebreak}
\setstretch{.7}
{\PaliGlossA{Tassa taṃ upekkhaṃ abhinandato abhivadato ajjhosāya tiṭṭhato tannissitaṃ hoti viññāṇaṃ tadupādānaṃ.}}\\
\begin{addmargin}[1em]{2em}
\setstretch{.5}
{\PaliGlossB{Their consciousness relies on that and grasps it.}}\\
\end{addmargin}
\end{absolutelynopagebreak}

\begin{absolutelynopagebreak}
\setstretch{.7}
{\PaliGlossA{Saupādāno, ānanda, bhikkhu na parinibbāyatī”ti.}}\\
\begin{addmargin}[1em]{2em}
\setstretch{.5}
{\PaliGlossB{A mendicant with grasping does not become extinguished.”}}\\
\end{addmargin}
\end{absolutelynopagebreak}

\vskip 0.05in
\begin{absolutelynopagebreak}
\setstretch{.7}
{\PaliGlossA{11. “Kahaṃ pana so, bhante, bhikkhu upādiyamāno upādiyatī”ti?}}\\
\begin{addmargin}[1em]{2em}
\setstretch{.5}
{\PaliGlossB{“But sir, what is that mendicant grasping?”}}\\
\end{addmargin}
\end{absolutelynopagebreak}

\begin{absolutelynopagebreak}
\setstretch{.7}
{\PaliGlossA{“Nevasaññānāsaññāyatanaṃ, ānandā”ti.}}\\
\begin{addmargin}[1em]{2em}
\setstretch{.5}
{\PaliGlossB{“The dimension of neither perception nor non-perception.”}}\\
\end{addmargin}
\end{absolutelynopagebreak}

\begin{absolutelynopagebreak}
\setstretch{.7}
{\PaliGlossA{“Upādānaseṭṭhaṃ kira so, bhante, bhikkhu upādiyamāno upādiyatī”ti?}}\\
\begin{addmargin}[1em]{2em}
\setstretch{.5}
{\PaliGlossB{“Sir, it seems that mendicant is grasping the best thing to grasp!”}}\\
\end{addmargin}
\end{absolutelynopagebreak}

\begin{absolutelynopagebreak}
\setstretch{.7}
{\PaliGlossA{“Upādānaseṭṭhañhi so, ānanda, bhikkhu upādiyamāno upādiyati.}}\\
\begin{addmargin}[1em]{2em}
\setstretch{.5}
{\PaliGlossB{“Indeed, Ānanda.}}\\
\end{addmargin}
\end{absolutelynopagebreak}

\begin{absolutelynopagebreak}
\setstretch{.7}
{\PaliGlossA{Upādānaseṭṭhañhetaṃ, ānanda, yadidaṃ—}}\\
\begin{addmargin}[1em]{2em}
\setstretch{.5}
{\PaliGlossB{For the best thing to grasp is}}\\
\end{addmargin}
\end{absolutelynopagebreak}

\begin{absolutelynopagebreak}
\setstretch{.7}
{\PaliGlossA{nevasaññānāsaññāyatanaṃ.}}\\
\begin{addmargin}[1em]{2em}
\setstretch{.5}
{\PaliGlossB{the dimension of neither perception nor non-perception.}}\\
\end{addmargin}
\end{absolutelynopagebreak}

\vskip 0.05in
\begin{absolutelynopagebreak}
\setstretch{.7}
{\PaliGlossA{12. Idhānanda, bhikkhu evaṃ paṭipanno hoti:}}\\
\begin{addmargin}[1em]{2em}
\setstretch{.5}
{\PaliGlossB{Take a mendicant who practices like this:}}\\
\end{addmargin}
\end{absolutelynopagebreak}

\begin{absolutelynopagebreak}
\setstretch{.7}
{\PaliGlossA{‘no cassa, no ca me siyā; na bhavissati, na me bhavissati; yadatthi, yaṃ bhūtaṃ—taṃ pajahāmī’ti.}}\\
\begin{addmargin}[1em]{2em}
\setstretch{.5}
{\PaliGlossB{‘It might not be, and it might not be mine. It will not be, and it will not be mine. I am giving up what exists, what has come to be.’}}\\
\end{addmargin}
\end{absolutelynopagebreak}

\begin{absolutelynopagebreak}
\setstretch{.7}
{\PaliGlossA{Evaṃ upekkhaṃ paṭilabhati.}}\\
\begin{addmargin}[1em]{2em}
\setstretch{.5}
{\PaliGlossB{In this way they gain equanimity.}}\\
\end{addmargin}
\end{absolutelynopagebreak}

\begin{absolutelynopagebreak}
\setstretch{.7}
{\PaliGlossA{So taṃ upekkhaṃ nābhinandati, nābhivadati, na ajjhosāya tiṭṭhati.}}\\
\begin{addmargin}[1em]{2em}
\setstretch{.5}
{\PaliGlossB{They don’t approve, welcome, or keep clinging to that equanimity.}}\\
\end{addmargin}
\end{absolutelynopagebreak}

\begin{absolutelynopagebreak}
\setstretch{.7}
{\PaliGlossA{Tassa taṃ upekkhaṃ anabhinandato anabhivadato anajjhosāya tiṭṭhato na tannissitaṃ hoti viññāṇaṃ na tadupādānaṃ.}}\\
\begin{addmargin}[1em]{2em}
\setstretch{.5}
{\PaliGlossB{So their consciousness doesn’t rely on that and grasp it.}}\\
\end{addmargin}
\end{absolutelynopagebreak}

\begin{absolutelynopagebreak}
\setstretch{.7}
{\PaliGlossA{Anupādāno, ānanda, bhikkhu parinibbāyatī”ti.}}\\
\begin{addmargin}[1em]{2em}
\setstretch{.5}
{\PaliGlossB{A mendicant free of grasping becomes extinguished.”}}\\
\end{addmargin}
\end{absolutelynopagebreak}

\vskip 0.05in
\begin{absolutelynopagebreak}
\setstretch{.7}
{\PaliGlossA{13. “Acchariyaṃ, bhante, abbhutaṃ, bhante.}}\\
\begin{addmargin}[1em]{2em}
\setstretch{.5}
{\PaliGlossB{“It’s incredible, sir, it’s amazing!}}\\
\end{addmargin}
\end{absolutelynopagebreak}

\begin{absolutelynopagebreak}
\setstretch{.7}
{\PaliGlossA{Nissāya nissāya kira no, bhante, bhagavatā oghassa nittharaṇā akkhātā.}}\\
\begin{addmargin}[1em]{2em}
\setstretch{.5}
{\PaliGlossB{The Buddha has explained to us how to cross over the flood by relying on one support or the other.}}\\
\end{addmargin}
\end{absolutelynopagebreak}

\begin{absolutelynopagebreak}
\setstretch{.7}
{\PaliGlossA{Katamo pana, bhante, ariyo vimokkho”ti?}}\\
\begin{addmargin}[1em]{2em}
\setstretch{.5}
{\PaliGlossB{But sir, what is noble liberation?”}}\\
\end{addmargin}
\end{absolutelynopagebreak}

\begin{absolutelynopagebreak}
\setstretch{.7}
{\PaliGlossA{“Idhānanda, bhikkhu ariyasāvako iti paṭisañcikkhati:}}\\
\begin{addmargin}[1em]{2em}
\setstretch{.5}
{\PaliGlossB{“Ananda, it’s when a mendicant reflects like this:}}\\
\end{addmargin}
\end{absolutelynopagebreak}

\begin{absolutelynopagebreak}
\setstretch{.7}
{\PaliGlossA{‘ye ca diṭṭhadhammikā kāmā, ye ca samparāyikā kāmā;}}\\
\begin{addmargin}[1em]{2em}
\setstretch{.5}
{\PaliGlossB{‘Sensual pleasures in this life and in lives to come,}}\\
\end{addmargin}
\end{absolutelynopagebreak}

\begin{absolutelynopagebreak}
\setstretch{.7}
{\PaliGlossA{yā ca diṭṭhadhammikā kāmasaññā, yā ca samparāyikā kāmasaññā;}}\\
\begin{addmargin}[1em]{2em}
\setstretch{.5}
{\PaliGlossB{sensual perceptions in this life and in lives to come,}}\\
\end{addmargin}
\end{absolutelynopagebreak}

\begin{absolutelynopagebreak}
\setstretch{.7}
{\PaliGlossA{ye ca diṭṭhadhammikā rūpā, ye ca samparāyikā rūpā;}}\\
\begin{addmargin}[1em]{2em}
\setstretch{.5}
{\PaliGlossB{visions in this life and in lives to come,}}\\
\end{addmargin}
\end{absolutelynopagebreak}

\begin{absolutelynopagebreak}
\setstretch{.7}
{\PaliGlossA{yā ca diṭṭhadhammikā rūpasaññā, yā ca samparāyikā rūpasaññā;}}\\
\begin{addmargin}[1em]{2em}
\setstretch{.5}
{\PaliGlossB{perceptions of visions in this life and in lives to come,}}\\
\end{addmargin}
\end{absolutelynopagebreak}

\begin{absolutelynopagebreak}
\setstretch{.7}
{\PaliGlossA{yā ca āneñjasaññā, yā ca ākiñcaññāyatanasaññā, yā ca nevasaññānāsaññāyatanasaññā—}}\\
\begin{addmargin}[1em]{2em}
\setstretch{.5}
{\PaliGlossB{perceptions of the imperturbable, perceptions of the dimension of nothingness, perceptions of the dimension of neither perception nor non-perception;}}\\
\end{addmargin}
\end{absolutelynopagebreak}

\begin{absolutelynopagebreak}
\setstretch{.7}
{\PaliGlossA{esa sakkāyo yāvatā sakkāyo.}}\\
\begin{addmargin}[1em]{2em}
\setstretch{.5}
{\PaliGlossB{that is identity as far as identity extends.}}\\
\end{addmargin}
\end{absolutelynopagebreak}

\begin{absolutelynopagebreak}
\setstretch{.7}
{\PaliGlossA{Etaṃ amataṃ yadidaṃ anupādā cittassa vimokkho.}}\\
\begin{addmargin}[1em]{2em}
\setstretch{.5}
{\PaliGlossB{This is the deathless, namely the liberation of the mind through not grasping.}}\\
\end{addmargin}
\end{absolutelynopagebreak}

\vskip 0.05in
\begin{absolutelynopagebreak}
\setstretch{.7}
{\PaliGlossA{14. Iti, kho, ānanda, desitā mayā āneñjasappāyā paṭipadā, desitā ākiñcaññāyatanasappāyā paṭipadā, desitā nevasaññānāsaññāyatanasappāyā paṭipadā, desitā nissāya nissāya oghassa nittharaṇā, desito ariyo vimokkho.}}\\
\begin{addmargin}[1em]{2em}
\setstretch{.5}
{\PaliGlossB{So, Ānanda, I have taught the ways of practice suitable for attaining the imperturbable, the dimension of nothingness, and the dimension of neither perception nor non-perception. I have taught how to cross the flood by relying on one support or the other, and I have taught noble liberation.}}\\
\end{addmargin}
\end{absolutelynopagebreak}

\vskip 0.05in
\begin{absolutelynopagebreak}
\setstretch{.7}
{\PaliGlossA{15. Yaṃ kho, ānanda, satthārā karaṇīyaṃ sāvakānaṃ hitesinā anukampakena anukampaṃ upādāya, kataṃ vo taṃ mayā.}}\\
\begin{addmargin}[1em]{2em}
\setstretch{.5}
{\PaliGlossB{Out of compassion, I’ve done what a teacher should do who wants what’s best for their disciples.}}\\
\end{addmargin}
\end{absolutelynopagebreak}

\begin{absolutelynopagebreak}
\setstretch{.7}
{\PaliGlossA{Etāni, ānanda, rukkhamūlāni, etāni suññāgārāni. Jhāyathānanda, mā pamādattha, mā pacchā vippaṭisārino ahuvattha. Ayaṃ vo amhākaṃ anusāsanī’”ti.}}\\
\begin{addmargin}[1em]{2em}
\setstretch{.5}
{\PaliGlossB{Here are these roots of trees, and here are these empty huts. Practice absorption, Ānanda! Don’t be negligent! Don’t regret it later! This is my instruction to you.”}}\\
\end{addmargin}
\end{absolutelynopagebreak}

\begin{absolutelynopagebreak}
\setstretch{.7}
{\PaliGlossA{Idamavoca bhagavā.}}\\
\begin{addmargin}[1em]{2em}
\setstretch{.5}
{\PaliGlossB{That is what the Buddha said.}}\\
\end{addmargin}
\end{absolutelynopagebreak}

\begin{absolutelynopagebreak}
\setstretch{.7}
{\PaliGlossA{Attamano āyasmā ānando bhagavato bhāsitaṃ abhinandīti.}}\\
\begin{addmargin}[1em]{2em}
\setstretch{.5}
{\PaliGlossB{Satisfied, Venerable Ānanda was happy with what the Buddha said.}}\\
\end{addmargin}
\end{absolutelynopagebreak}

\begin{absolutelynopagebreak}
\setstretch{.7}
{\PaliGlossA{Āneñjasappāyasuttaṃ niṭṭhitaṃ chaṭṭhaṃ.}}\\
\begin{addmargin}[1em]{2em}
\setstretch{.5}
{\PaliGlossB{    -}}\\
\end{addmargin}
\end{absolutelynopagebreak}
