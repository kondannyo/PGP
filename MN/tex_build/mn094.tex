
\vskip 0.05in
\begin{absolutelynopagebreak}
\setstretch{.7}
{\PaliGlossA{Majjhima Nikāya 94}}\\
\begin{addmargin}[1em]{2em}
\setstretch{.5}
{\PaliGlossB{Middle Discourses 94}}\\
\end{addmargin}
\end{absolutelynopagebreak}

\begin{absolutelynopagebreak}
\setstretch{.7}
{\PaliGlossA{Ghoṭamukhasutta}}\\
\begin{addmargin}[1em]{2em}
\setstretch{.5}
{\PaliGlossB{With Ghoṭamukha}}\\
\end{addmargin}
\end{absolutelynopagebreak}

\vskip 0.05in
\begin{absolutelynopagebreak}
\setstretch{.7}
{\PaliGlossA{1. Evaṃ me sutaṃ—}}\\
\begin{addmargin}[1em]{2em}
\setstretch{.5}
{\PaliGlossB{So I have heard.}}\\
\end{addmargin}
\end{absolutelynopagebreak}

\begin{absolutelynopagebreak}
\setstretch{.7}
{\PaliGlossA{ekaṃ samayaṃ āyasmā udeno bārāṇasiyaṃ viharati khemiyambavane.}}\\
\begin{addmargin}[1em]{2em}
\setstretch{.5}
{\PaliGlossB{At one time Venerable Udena was staying near Benares in the Khemiya Mango Grove.}}\\
\end{addmargin}
\end{absolutelynopagebreak}

\vskip 0.05in
\begin{absolutelynopagebreak}
\setstretch{.7}
{\PaliGlossA{2. Tena kho pana samayena ghoṭamukho brāhmaṇo bārāṇasiṃ anuppatto hoti kenacideva karaṇīyena.}}\\
\begin{addmargin}[1em]{2em}
\setstretch{.5}
{\PaliGlossB{Now at that time the brahmin Ghoṭamukha had arrived at Benares on some business.}}\\
\end{addmargin}
\end{absolutelynopagebreak}

\begin{absolutelynopagebreak}
\setstretch{.7}
{\PaliGlossA{Atha kho ghoṭamukho brāhmaṇo jaṅghāvihāraṃ anucaṅkamamāno anuvicaramāno yena khemiyambavanaṃ tenupasaṅkami.}}\\
\begin{addmargin}[1em]{2em}
\setstretch{.5}
{\PaliGlossB{Then as he was going for a walk he went to the Khemiya Mango Grove.}}\\
\end{addmargin}
\end{absolutelynopagebreak}

\begin{absolutelynopagebreak}
\setstretch{.7}
{\PaliGlossA{Tena kho pana samayena āyasmā udeno abbhokāse caṅkamati.}}\\
\begin{addmargin}[1em]{2em}
\setstretch{.5}
{\PaliGlossB{At that time Venerable Udena was walking meditation in the open air.}}\\
\end{addmargin}
\end{absolutelynopagebreak}

\begin{absolutelynopagebreak}
\setstretch{.7}
{\PaliGlossA{Atha kho ghoṭamukho brāhmaṇo yenāyasmā udeno tenupasaṅkami; upasaṅkamitvā āyasmatā udenena saddhiṃ sammodi.}}\\
\begin{addmargin}[1em]{2em}
\setstretch{.5}
{\PaliGlossB{Ghoṭamukha approached and exchanged greetings with him.}}\\
\end{addmargin}
\end{absolutelynopagebreak}

\begin{absolutelynopagebreak}
\setstretch{.7}
{\PaliGlossA{Sammodanīyaṃ kathaṃ sāraṇīyaṃ vītisāretvā āyasmantaṃ udenaṃ caṅkamantaṃ anucaṅkamamāno evamāha:}}\\
\begin{addmargin}[1em]{2em}
\setstretch{.5}
{\PaliGlossB{Walking alongside Udena, he said,}}\\
\end{addmargin}
\end{absolutelynopagebreak}

\begin{absolutelynopagebreak}
\setstretch{.7}
{\PaliGlossA{“ambho samaṇa, ‘natthi dhammiko paribbajo’—}}\\
\begin{addmargin}[1em]{2em}
\setstretch{.5}
{\PaliGlossB{“Mister ascetic, there is no such thing as a principled renunciate life;}}\\
\end{addmargin}
\end{absolutelynopagebreak}

\begin{absolutelynopagebreak}
\setstretch{.7}
{\PaliGlossA{evaṃ me ettha hoti.}}\\
\begin{addmargin}[1em]{2em}
\setstretch{.5}
{\PaliGlossB{that’s what I think.}}\\
\end{addmargin}
\end{absolutelynopagebreak}

\begin{absolutelynopagebreak}
\setstretch{.7}
{\PaliGlossA{Tañca kho bhavantarūpānaṃ vā adassanā, yo vā panettha dhammo”ti.}}\\
\begin{addmargin}[1em]{2em}
\setstretch{.5}
{\PaliGlossB{And that’s without seeing gentlemen such as yourself, or a relevant teaching.”}}\\
\end{addmargin}
\end{absolutelynopagebreak}

\vskip 0.05in
\begin{absolutelynopagebreak}
\setstretch{.7}
{\PaliGlossA{3. Evaṃ vutte, āyasmā udeno caṅkamā orohitvā vihāraṃ pavisitvā paññatte āsane nisīdi.}}\\
\begin{addmargin}[1em]{2em}
\setstretch{.5}
{\PaliGlossB{When he said this, Udena stepped down from the walking path, entered his dwelling, and sat down on the seat spread out.}}\\
\end{addmargin}
\end{absolutelynopagebreak}

\begin{absolutelynopagebreak}
\setstretch{.7}
{\PaliGlossA{Ghoṭamukhopi kho brāhmaṇo caṅkamā orohitvā vihāraṃ pavisitvā ekamantaṃ aṭṭhāsi.}}\\
\begin{addmargin}[1em]{2em}
\setstretch{.5}
{\PaliGlossB{Ghoṭamukha also stepped down from the walking path and entered the dwelling, where he stood to one side.}}\\
\end{addmargin}
\end{absolutelynopagebreak}

\begin{absolutelynopagebreak}
\setstretch{.7}
{\PaliGlossA{Ekamantaṃ ṭhitaṃ kho ghoṭamukhaṃ brāhmaṇaṃ āyasmā udeno etadavoca:}}\\
\begin{addmargin}[1em]{2em}
\setstretch{.5}
{\PaliGlossB{Udena said to him,}}\\
\end{addmargin}
\end{absolutelynopagebreak}

\begin{absolutelynopagebreak}
\setstretch{.7}
{\PaliGlossA{“saṃvijjanti kho, brāhmaṇa, āsanāni.}}\\
\begin{addmargin}[1em]{2em}
\setstretch{.5}
{\PaliGlossB{“There are seats, brahmin.}}\\
\end{addmargin}
\end{absolutelynopagebreak}

\begin{absolutelynopagebreak}
\setstretch{.7}
{\PaliGlossA{Sace ākaṅkhasi, nisīdā”ti.}}\\
\begin{addmargin}[1em]{2em}
\setstretch{.5}
{\PaliGlossB{Please sit if you wish.”}}\\
\end{addmargin}
\end{absolutelynopagebreak}

\begin{absolutelynopagebreak}
\setstretch{.7}
{\PaliGlossA{“Etadeva kho pana mayaṃ bhoto udenassa āgamayamānā nisīdāma.}}\\
\begin{addmargin}[1em]{2em}
\setstretch{.5}
{\PaliGlossB{“I was just waiting for you to sit down.}}\\
\end{addmargin}
\end{absolutelynopagebreak}

\begin{absolutelynopagebreak}
\setstretch{.7}
{\PaliGlossA{Kathañhi nāma mādiso pubbe animantito āsane nisīditabbaṃ maññeyyā”ti?}}\\
\begin{addmargin}[1em]{2em}
\setstretch{.5}
{\PaliGlossB{For how could one such as I presume to sit first without being invited?”}}\\
\end{addmargin}
\end{absolutelynopagebreak}

\vskip 0.05in
\begin{absolutelynopagebreak}
\setstretch{.7}
{\PaliGlossA{4. Atha kho ghoṭamukho brāhmaṇo aññataraṃ nīcaṃ āsanaṃ gahetvā ekamantaṃ nisīdi.}}\\
\begin{addmargin}[1em]{2em}
\setstretch{.5}
{\PaliGlossB{Then he took a low seat and sat to one side,}}\\
\end{addmargin}
\end{absolutelynopagebreak}

\begin{absolutelynopagebreak}
\setstretch{.7}
{\PaliGlossA{Ekamantaṃ nisinno kho ghoṭamukho brāhmaṇo āyasmantaṃ udenaṃ etadavoca:}}\\
\begin{addmargin}[1em]{2em}
\setstretch{.5}
{\PaliGlossB{where he said,}}\\
\end{addmargin}
\end{absolutelynopagebreak}

\begin{absolutelynopagebreak}
\setstretch{.7}
{\PaliGlossA{“ambho samaṇa, ‘natthi dhammiko paribbajo’—}}\\
\begin{addmargin}[1em]{2em}
\setstretch{.5}
{\PaliGlossB{“Mister ascetic, there is no such thing as a principled renunciate life;}}\\
\end{addmargin}
\end{absolutelynopagebreak}

\begin{absolutelynopagebreak}
\setstretch{.7}
{\PaliGlossA{evaṃ me ettha hoti.}}\\
\begin{addmargin}[1em]{2em}
\setstretch{.5}
{\PaliGlossB{that’s what I think.}}\\
\end{addmargin}
\end{absolutelynopagebreak}

\begin{absolutelynopagebreak}
\setstretch{.7}
{\PaliGlossA{Tañca kho bhavantarūpānaṃ vā adassanā, yo vā panettha dhammo”ti.}}\\
\begin{addmargin}[1em]{2em}
\setstretch{.5}
{\PaliGlossB{And that’s without seeing gentlemen such as yourself, or a relevant teaching.”}}\\
\end{addmargin}
\end{absolutelynopagebreak}

\begin{absolutelynopagebreak}
\setstretch{.7}
{\PaliGlossA{“Sace kho pana me tvaṃ, brāhmaṇa, anuññeyyaṃ anujāneyyāsi, paṭikkositabbañca paṭikkoseyyāsi; yassa ca pana me bhāsitassa atthaṃ na jāneyyāsi, mamaṃyeva tattha uttari paṭipuccheyyāsi:}}\\
\begin{addmargin}[1em]{2em}
\setstretch{.5}
{\PaliGlossB{“Brahmin, we can discuss this. But only if you allow what should be allowed, and reject what should be rejected. And if you ask me the meaning of anything you don’t understand, saying:}}\\
\end{addmargin}
\end{absolutelynopagebreak}

\begin{absolutelynopagebreak}
\setstretch{.7}
{\PaliGlossA{‘idaṃ, bho udena, kathaṃ, imassa kvattho’ti? Evaṃ katvā siyā no ettha kathāsallāpo”ti.}}\\
\begin{addmargin}[1em]{2em}
\setstretch{.5}
{\PaliGlossB{‘Sir, why is this? What does that mean?’”}}\\
\end{addmargin}
\end{absolutelynopagebreak}

\begin{absolutelynopagebreak}
\setstretch{.7}
{\PaliGlossA{“Anuññeyyaṃ khvāhaṃ bhoto udenassa anujānissāmi, paṭikkositabbañca paṭikkosissāmi; yassa ca panāhaṃ bhoto udenassa bhāsitassa atthaṃ na jānissāmi, bhavantaṃyeva tattha udenaṃ uttari paṭipucchissāmi: ‘idaṃ, bho udena, kathaṃ, imassa kvattho’ti? Evaṃ katvā hotu no ettha kathāsallāpo”ti.}}\\
\begin{addmargin}[1em]{2em}
\setstretch{.5}
{\PaliGlossB{“Let us discuss this. I will do as you say.”}}\\
\end{addmargin}
\end{absolutelynopagebreak}

\vskip 0.05in
\begin{absolutelynopagebreak}
\setstretch{.7}
{\PaliGlossA{5. “Cattārome, brāhmaṇa, puggalā santo saṃvijjamānā lokasmiṃ.}}\\
\begin{addmargin}[1em]{2em}
\setstretch{.5}
{\PaliGlossB{“Brahmin, these four people are found in the world.}}\\
\end{addmargin}
\end{absolutelynopagebreak}

\begin{absolutelynopagebreak}
\setstretch{.7}
{\PaliGlossA{Katame cattāro?}}\\
\begin{addmargin}[1em]{2em}
\setstretch{.5}
{\PaliGlossB{What four?}}\\
\end{addmargin}
\end{absolutelynopagebreak}

\begin{absolutelynopagebreak}
\setstretch{.7}
{\PaliGlossA{Idha, brāhmaṇa, ekacco puggalo attantapo hoti attaparitāpanānuyogamanuyutto.}}\\
\begin{addmargin}[1em]{2em}
\setstretch{.5}
{\PaliGlossB{One person mortifies themselves, committed to the practice of mortifying themselves.}}\\
\end{addmargin}
\end{absolutelynopagebreak}

\begin{absolutelynopagebreak}
\setstretch{.7}
{\PaliGlossA{Idha pana, brāhmaṇa, ekacco puggalo parantapo hoti paraparitāpanānuyogamanuyutto.}}\\
\begin{addmargin}[1em]{2em}
\setstretch{.5}
{\PaliGlossB{One person mortifies others, committed to the practice of mortifying others.}}\\
\end{addmargin}
\end{absolutelynopagebreak}

\begin{absolutelynopagebreak}
\setstretch{.7}
{\PaliGlossA{Idha pana, brāhmaṇa, ekacco puggalo attantapo ca hoti attaparitāpanānuyogamanuyutto parantapo ca paraparitāpanānuyogamanuyutto.}}\\
\begin{addmargin}[1em]{2em}
\setstretch{.5}
{\PaliGlossB{One person mortifies themselves and others, committed to the practice of mortifying themselves and others.}}\\
\end{addmargin}
\end{absolutelynopagebreak}

\begin{absolutelynopagebreak}
\setstretch{.7}
{\PaliGlossA{Idha pana, brāhmaṇa, ekacco puggalo nevattantapo hoti nāttaparitāpanānuyogamanuyutto, na parantapo na paraparitāpanānuyogamanuyutto.}}\\
\begin{addmargin}[1em]{2em}
\setstretch{.5}
{\PaliGlossB{One person doesn’t mortify either themselves or others, committed to the practice of not mortifying themselves or others.}}\\
\end{addmargin}
\end{absolutelynopagebreak}

\begin{absolutelynopagebreak}
\setstretch{.7}
{\PaliGlossA{So anattantapo aparantapo diṭṭheva dhamme nicchāto nibbuto sītībhūto sukhappaṭisaṃvedī brahmabhūtena attanā viharati.}}\\
\begin{addmargin}[1em]{2em}
\setstretch{.5}
{\PaliGlossB{They live without wishes in the present life, extinguished, cooled, experiencing bliss, having become holy in themselves.}}\\
\end{addmargin}
\end{absolutelynopagebreak}

\begin{absolutelynopagebreak}
\setstretch{.7}
{\PaliGlossA{Imesaṃ, brāhmaṇa, catunnaṃ puggalānaṃ katamo te puggalo cittaṃ ārādhetī”ti?}}\\
\begin{addmargin}[1em]{2em}
\setstretch{.5}
{\PaliGlossB{Which one of these four people do you like the sound of?”}}\\
\end{addmargin}
\end{absolutelynopagebreak}

\begin{absolutelynopagebreak}
\setstretch{.7}
{\PaliGlossA{“Yvāyaṃ, bho udena, puggalo attantapo attaparitāpanānuyogamanuyutto ayaṃ me puggalo cittaṃ nārādheti;}}\\
\begin{addmargin}[1em]{2em}
\setstretch{.5}
{\PaliGlossB{“Sir, I don’t like the sound of the first three people.}}\\
\end{addmargin}
\end{absolutelynopagebreak}

\begin{absolutelynopagebreak}
\setstretch{.7}
{\PaliGlossA{yopāyaṃ, bho udena, puggalo parantapo paraparitāpanānuyogamanuyutto ayampi me puggalo cittaṃ nārādheti;}}\\
\begin{addmargin}[1em]{2em}
\setstretch{.5}
{\PaliGlossB{    -}}\\
\end{addmargin}
\end{absolutelynopagebreak}

\begin{absolutelynopagebreak}
\setstretch{.7}
{\PaliGlossA{yopāyaṃ, bho udena, puggalo attantapo ca attaparitāpanānuyogamanuyutto parantapo ca paraparitāpanānuyogamanuyutto ayampi me puggalo cittaṃ nārādheti;}}\\
\begin{addmargin}[1em]{2em}
\setstretch{.5}
{\PaliGlossB{    -}}\\
\end{addmargin}
\end{absolutelynopagebreak}

\begin{absolutelynopagebreak}
\setstretch{.7}
{\PaliGlossA{yo ca kho ayaṃ, bho udena, puggalo nevattantapo nāttaparitāpanānuyogamanuyutto na parantapo na paraparitāpanānuyogamanuyutto so anattantapo aparantapo diṭṭheva dhamme nicchāto nibbuto sītībhūto sukhappaṭisaṃvedī brahmabhūtena attanā viharati.}}\\
\begin{addmargin}[1em]{2em}
\setstretch{.5}
{\PaliGlossB{    -}}\\
\end{addmargin}
\end{absolutelynopagebreak}

\begin{absolutelynopagebreak}
\setstretch{.7}
{\PaliGlossA{Ayameva me puggalo cittaṃ ārādhetī”ti.}}\\
\begin{addmargin}[1em]{2em}
\setstretch{.5}
{\PaliGlossB{I only like the sound of the last person, who doesn’t mortify either themselves or others.”}}\\
\end{addmargin}
\end{absolutelynopagebreak}

\vskip 0.05in
\begin{absolutelynopagebreak}
\setstretch{.7}
{\PaliGlossA{6. “Kasmā pana te, brāhmaṇa, ime tayo puggalā cittaṃ nārādhentī”ti?}}\\
\begin{addmargin}[1em]{2em}
\setstretch{.5}
{\PaliGlossB{“But why don’t you like the sound of those three people?”}}\\
\end{addmargin}
\end{absolutelynopagebreak}

\begin{absolutelynopagebreak}
\setstretch{.7}
{\PaliGlossA{“Yvāyaṃ, bho udena, puggalo attantapo attaparitāpanānuyogamanuyutto so attānaṃ sukhakāmaṃ dukkhapaṭikkūlaṃ ātāpeti paritāpeti;}}\\
\begin{addmargin}[1em]{2em}
\setstretch{.5}
{\PaliGlossB{“Sir, the person who mortifies themselves does so even though they want to be happy and recoil from pain.}}\\
\end{addmargin}
\end{absolutelynopagebreak}

\begin{absolutelynopagebreak}
\setstretch{.7}
{\PaliGlossA{iminā me ayaṃ puggalo cittaṃ nārādheti.}}\\
\begin{addmargin}[1em]{2em}
\setstretch{.5}
{\PaliGlossB{That’s why I don’t like the sound of that person.}}\\
\end{addmargin}
\end{absolutelynopagebreak}

\begin{absolutelynopagebreak}
\setstretch{.7}
{\PaliGlossA{Yopāyaṃ, bho udena, puggalo parantapo paraparitāpanānuyogamanuyutto so paraṃ sukhakāmaṃ dukkhapaṭikkūlaṃ ātāpeti paritāpeti;}}\\
\begin{addmargin}[1em]{2em}
\setstretch{.5}
{\PaliGlossB{The person who mortifies others does so even though others want to be happy and recoil from pain.}}\\
\end{addmargin}
\end{absolutelynopagebreak}

\begin{absolutelynopagebreak}
\setstretch{.7}
{\PaliGlossA{iminā me ayaṃ puggalo cittaṃ nārādheti.}}\\
\begin{addmargin}[1em]{2em}
\setstretch{.5}
{\PaliGlossB{That’s why I don’t like the sound of that person.}}\\
\end{addmargin}
\end{absolutelynopagebreak}

\begin{absolutelynopagebreak}
\setstretch{.7}
{\PaliGlossA{Yopāyaṃ, bho udena, puggalo attantapo ca attaparitāpanānuyogamanuyutto parantapo ca paraparitāpanānuyogamanuyutto so attānañca parañca sukhakāmaṃ dukkhapaṭikkūlaṃ ātāpeti paritāpeti;}}\\
\begin{addmargin}[1em]{2em}
\setstretch{.5}
{\PaliGlossB{The person who mortifies themselves and others does so even though both themselves and others want to be happy and recoil from pain.}}\\
\end{addmargin}
\end{absolutelynopagebreak}

\begin{absolutelynopagebreak}
\setstretch{.7}
{\PaliGlossA{iminā me ayaṃ puggalo cittaṃ nārādheti.}}\\
\begin{addmargin}[1em]{2em}
\setstretch{.5}
{\PaliGlossB{That’s why I don’t like the sound of that person.}}\\
\end{addmargin}
\end{absolutelynopagebreak}

\begin{absolutelynopagebreak}
\setstretch{.7}
{\PaliGlossA{Yo ca kho ayaṃ, bho udena, puggalo nevattantapo nāttaparitāpanānuyogamanuyutto na parantapo na paraparitāpanānuyogamanuyutto so anattantapo aparantapo diṭṭheva dhamme nicchāto nibbuto sītībhūto sukhappaṭisaṃvedī brahmabhūtena attanā viharati, so attānañca parañca sukhakāmaṃ dukkhapaṭikkūlaṃ neva ātāpeti na paritāpeti;}}\\
\begin{addmargin}[1em]{2em}
\setstretch{.5}
{\PaliGlossB{The person who doesn’t mortify either themselves or others—living without wishes, extinguished, cooled, experiencing bliss, having become holy in themselves—does not torment themselves or others, both of whom want to be happy and recoil from pain.}}\\
\end{addmargin}
\end{absolutelynopagebreak}

\begin{absolutelynopagebreak}
\setstretch{.7}
{\PaliGlossA{iminā me ayaṃ puggalo cittaṃ ārādhetī”ti.}}\\
\begin{addmargin}[1em]{2em}
\setstretch{.5}
{\PaliGlossB{That’s why I like the sound of that person.”}}\\
\end{addmargin}
\end{absolutelynopagebreak}

\vskip 0.05in
\begin{absolutelynopagebreak}
\setstretch{.7}
{\PaliGlossA{7. “Dvemā, brāhmaṇa, parisā.}}\\
\begin{addmargin}[1em]{2em}
\setstretch{.5}
{\PaliGlossB{“There are, brahmin, these two groups of people.}}\\
\end{addmargin}
\end{absolutelynopagebreak}

\begin{absolutelynopagebreak}
\setstretch{.7}
{\PaliGlossA{Katamā dve?}}\\
\begin{addmargin}[1em]{2em}
\setstretch{.5}
{\PaliGlossB{What two?}}\\
\end{addmargin}
\end{absolutelynopagebreak}

\begin{absolutelynopagebreak}
\setstretch{.7}
{\PaliGlossA{Idha, brāhmaṇa, ekaccā parisā sārattarattā maṇikuṇḍalesu puttabhariyaṃ pariyesati, dāsidāsaṃ pariyesati, khettavatthuṃ pariyesati, jātarūparajataṃ pariyesati.}}\\
\begin{addmargin}[1em]{2em}
\setstretch{.5}
{\PaliGlossB{There’s one group of people who, being infatuated with jewels and earrings, seeks partners and children, male and female bondservants, fields and lands, and gold and money.}}\\
\end{addmargin}
\end{absolutelynopagebreak}

\begin{absolutelynopagebreak}
\setstretch{.7}
{\PaliGlossA{Idha pana, brāhmaṇa, ekaccā parisā asārattarattā maṇikuṇḍalesu puttabhariyaṃ pahāya, dāsidāsaṃ pahāya, khettavatthuṃ pahāya, jātarūparajataṃ pahāya, agārasmā anagāriyaṃ pabbajitā.}}\\
\begin{addmargin}[1em]{2em}
\setstretch{.5}
{\PaliGlossB{And there’s another group of people who, not being infatuated with jewels and earrings, has given up partner and children, male and female bondservants, fields and lands, and gold and money, and goes forth from the lay life to homelessness.}}\\
\end{addmargin}
\end{absolutelynopagebreak}

\begin{absolutelynopagebreak}
\setstretch{.7}
{\PaliGlossA{Svāyaṃ, brāhmaṇa, puggalo nevattantapo nāttaparitāpanānuyogamanuyutto na parantapo na paraparitāpanānuyogamanuyutto.}}\\
\begin{addmargin}[1em]{2em}
\setstretch{.5}
{\PaliGlossB{Now, brahmin, that person who doesn’t mortify either themselves or others—}}\\
\end{addmargin}
\end{absolutelynopagebreak}

\begin{absolutelynopagebreak}
\setstretch{.7}
{\PaliGlossA{So anattantapo aparantapo diṭṭheva dhamme nicchāto nibbuto sītībhūto sukhappaṭisaṃvedī brahmabhūtena attanā viharati.}}\\
\begin{addmargin}[1em]{2em}
\setstretch{.5}
{\PaliGlossB{    -}}\\
\end{addmargin}
\end{absolutelynopagebreak}

\begin{absolutelynopagebreak}
\setstretch{.7}
{\PaliGlossA{Idha katamaṃ tvaṃ, brāhmaṇa, puggalaṃ katamāya parisāya bahulaṃ samanupassasi—}}\\
\begin{addmargin}[1em]{2em}
\setstretch{.5}
{\PaliGlossB{in which of these two groups of people do you usually find such a person?”}}\\
\end{addmargin}
\end{absolutelynopagebreak}

\begin{absolutelynopagebreak}
\setstretch{.7}
{\PaliGlossA{yā cāyaṃ parisā sārattarattā maṇikuṇḍalesu puttabhariyaṃ pariyesati dāsidāsaṃ pariyesati khettavatthuṃ pariyesati jātarūparajataṃ pariyesati, yā cāyaṃ parisā asārattarattā maṇikuṇḍalesu puttabhariyaṃ pahāya dāsidāsaṃ pahāya khettavatthuṃ pahāya jātarūparajataṃ pahāya agārasmā anagāriyaṃ pabbajitā”ti?}}\\
\begin{addmargin}[1em]{2em}
\setstretch{.5}
{\PaliGlossB{    -}}\\
\end{addmargin}
\end{absolutelynopagebreak}

\begin{absolutelynopagebreak}
\setstretch{.7}
{\PaliGlossA{“Yvāyaṃ, bho udena, puggalo nevattantapo nāttaparitāpanānuyogamanuyutto na parantapo na paraparitāpanānuyogamanuyutto so anattantapo aparantapo diṭṭheva dhamme nicchāto nibbuto sītībhūto sukhappaṭisaṃvedī brahmabhūtena attanā viharati;}}\\
\begin{addmargin}[1em]{2em}
\setstretch{.5}
{\PaliGlossB{“I usually find such a person in}}\\
\end{addmargin}
\end{absolutelynopagebreak}

\begin{absolutelynopagebreak}
\setstretch{.7}
{\PaliGlossA{imāhaṃ puggalaṃ yāyaṃ parisā asārattarattā maṇikuṇḍalesu puttabhariyaṃ pahāya dāsidāsaṃ pahāya khettavatthuṃ pahāya jātarūparajataṃ pahāya agārasmā anagāriyaṃ pabbajitā imissaṃ parisāyaṃ bahulaṃ samanupassāmī”ti.}}\\
\begin{addmargin}[1em]{2em}
\setstretch{.5}
{\PaliGlossB{the group that has gone forth from the lay life to homelessness.”}}\\
\end{addmargin}
\end{absolutelynopagebreak}

\vskip 0.05in
\begin{absolutelynopagebreak}
\setstretch{.7}
{\PaliGlossA{8. “Idāneva kho pana te, brāhmaṇa, bhāsitaṃ: ‘mayaṃ evaṃ ājānāma—}}\\
\begin{addmargin}[1em]{2em}
\setstretch{.5}
{\PaliGlossB{“Just now I understood you to say:}}\\
\end{addmargin}
\end{absolutelynopagebreak}

\begin{absolutelynopagebreak}
\setstretch{.7}
{\PaliGlossA{ambho samaṇa, natthi dhammiko paribbajo,}}\\
\begin{addmargin}[1em]{2em}
\setstretch{.5}
{\PaliGlossB{‘Mister ascetic, there is no such thing as a principled renunciate life;}}\\
\end{addmargin}
\end{absolutelynopagebreak}

\begin{absolutelynopagebreak}
\setstretch{.7}
{\PaliGlossA{evaṃ me ettha hoti.}}\\
\begin{addmargin}[1em]{2em}
\setstretch{.5}
{\PaliGlossB{that’s what I think.}}\\
\end{addmargin}
\end{absolutelynopagebreak}

\begin{absolutelynopagebreak}
\setstretch{.7}
{\PaliGlossA{Tañca kho bhavantarūpānaṃ vā adassanā, yo vā panettha dhammo’”ti.}}\\
\begin{addmargin}[1em]{2em}
\setstretch{.5}
{\PaliGlossB{And that’s without seeing gentlemen such as yourself, nor a relevant teaching.’”}}\\
\end{addmargin}
\end{absolutelynopagebreak}

\begin{absolutelynopagebreak}
\setstretch{.7}
{\PaliGlossA{“Addhā mesā, bho udena, sānuggahā vācā bhāsitā.}}\\
\begin{addmargin}[1em]{2em}
\setstretch{.5}
{\PaliGlossB{“Well, I obviously had my reasons for saying that, master Udena.}}\\
\end{addmargin}
\end{absolutelynopagebreak}

\begin{absolutelynopagebreak}
\setstretch{.7}
{\PaliGlossA{‘Atthi dhammiko paribbajo’—}}\\
\begin{addmargin}[1em]{2em}
\setstretch{.5}
{\PaliGlossB{But there is such a thing as a principled renunciate life;}}\\
\end{addmargin}
\end{absolutelynopagebreak}

\begin{absolutelynopagebreak}
\setstretch{.7}
{\PaliGlossA{evaṃ me ettha hoti.}}\\
\begin{addmargin}[1em]{2em}
\setstretch{.5}
{\PaliGlossB{that’s what I think.}}\\
\end{addmargin}
\end{absolutelynopagebreak}

\begin{absolutelynopagebreak}
\setstretch{.7}
{\PaliGlossA{Evañca pana maṃ bhavaṃ udeno dhāretu.}}\\
\begin{addmargin}[1em]{2em}
\setstretch{.5}
{\PaliGlossB{Please remember me as saying this.}}\\
\end{addmargin}
\end{absolutelynopagebreak}

\begin{absolutelynopagebreak}
\setstretch{.7}
{\PaliGlossA{Ye ca me bhotā udenena cattāro puggalā saṅkhittena vuttā vitthārena avibhattā, sādhu me bhavaṃ, udeno ime cattāro puggale vitthārena vibhajatu anukampaṃ upādāyā”ti.}}\\
\begin{addmargin}[1em]{2em}
\setstretch{.5}
{\PaliGlossB{Now, these four kinds of people that you’ve spoken of in a brief summary: please explain them to me in detail, out of compassion.”}}\\
\end{addmargin}
\end{absolutelynopagebreak}

\vskip 0.05in
\begin{absolutelynopagebreak}
\setstretch{.7}
{\PaliGlossA{9. “Tena hi, brāhmaṇa, suṇāhi, sādhukaṃ manasi karohi, bhāsissāmī”ti.}}\\
\begin{addmargin}[1em]{2em}
\setstretch{.5}
{\PaliGlossB{“Well then, brahmin, listen and pay close attention, I will speak.”}}\\
\end{addmargin}
\end{absolutelynopagebreak}

\begin{absolutelynopagebreak}
\setstretch{.7}
{\PaliGlossA{“Evaṃ, bho”ti kho ghoṭamukho brāhmaṇo āyasmato udenassa paccassosi.}}\\
\begin{addmargin}[1em]{2em}
\setstretch{.5}
{\PaliGlossB{“Yes, sir,” replied Ghoṭamukha.}}\\
\end{addmargin}
\end{absolutelynopagebreak}

\begin{absolutelynopagebreak}
\setstretch{.7}
{\PaliGlossA{Āyasmā udeno etadavoca:}}\\
\begin{addmargin}[1em]{2em}
\setstretch{.5}
{\PaliGlossB{Udena said this:}}\\
\end{addmargin}
\end{absolutelynopagebreak}

\vskip 0.05in
\begin{absolutelynopagebreak}
\setstretch{.7}
{\PaliGlossA{10. “Katamo ca, brāhmaṇa, puggalo attantapo attaparitāpanānuyogamanuyutto?}}\\
\begin{addmargin}[1em]{2em}
\setstretch{.5}
{\PaliGlossB{“What person mortifies themselves, committed to the practice of mortifying themselves?}}\\
\end{addmargin}
\end{absolutelynopagebreak}

\begin{absolutelynopagebreak}
\setstretch{.7}
{\PaliGlossA{Idha, brāhmaṇa, ekacco puggalo acelako hoti muttācāro hatthāpalekhano naehibhaddantiko natiṭṭhabhaddantiko, nābhihaṭaṃ na uddissakataṃ na nimantanaṃ sādiyati.}}\\
\begin{addmargin}[1em]{2em}
\setstretch{.5}
{\PaliGlossB{It’s when someone goes naked, ignoring conventions. They lick their hands, and don’t come or wait when asked. They don’t consent to food brought to them, or food prepared on purpose for them, or an invitation for a meal.}}\\
\end{addmargin}
\end{absolutelynopagebreak}

\begin{absolutelynopagebreak}
\setstretch{.7}
{\PaliGlossA{So na kumbhimukhā paṭiggaṇhāti, na kaḷopimukhā paṭiggaṇhāti, na eḷakamantaraṃ, na daṇḍamantaraṃ, na musalamantaraṃ, na dvinnaṃ bhuñjamānānaṃ, na gabbhiniyā, na pāyamānāya, na purisantaragatāya, na saṅkittīsu, na yattha sā upaṭṭhito hoti, na yattha makkhikā saṇḍasaṇḍacārinī, na macchaṃ na maṃsaṃ, na suraṃ na merayaṃ na thusodakaṃ pivati.}}\\
\begin{addmargin}[1em]{2em}
\setstretch{.5}
{\PaliGlossB{They don’t receive anything from a pot or bowl; or from someone who keeps sheep, or who has a weapon or a shovel in their home; or where a couple is eating; or where there is a woman who is pregnant, breastfeeding, or who has a man in her home; or where there’s a dog waiting or flies buzzing. They accept no fish or meat or liquor or wine, and drink no beer.}}\\
\end{addmargin}
\end{absolutelynopagebreak}

\begin{absolutelynopagebreak}
\setstretch{.7}
{\PaliGlossA{So ekāgāriko vā hoti ekālopiko, dvāgāriko vā hoti dvālopiko … pe … sattāgāriko vā hoti sattālopiko;}}\\
\begin{addmargin}[1em]{2em}
\setstretch{.5}
{\PaliGlossB{They go to just one house for alms, taking just one mouthful, or two houses and two mouthfuls, up to seven houses and seven mouthfuls.}}\\
\end{addmargin}
\end{absolutelynopagebreak}

\begin{absolutelynopagebreak}
\setstretch{.7}
{\PaliGlossA{ekissāpi dattiyā yāpeti, dvīhipi dattīhi yāpeti … pe … sattahipi dattīhi yāpeti;}}\\
\begin{addmargin}[1em]{2em}
\setstretch{.5}
{\PaliGlossB{They feed on one saucer a day, two saucers a day, up to seven saucers a day.}}\\
\end{addmargin}
\end{absolutelynopagebreak}

\begin{absolutelynopagebreak}
\setstretch{.7}
{\PaliGlossA{ekāhikampi āhāraṃ āhāreti, dvīhikampi āhāraṃ āhāreti … pe … sattāhikampi āhāraṃ āhāreti—iti evarūpaṃ addhamāsikaṃ pariyāyabhattabhojanānuyogamanuyutto viharati.}}\\
\begin{addmargin}[1em]{2em}
\setstretch{.5}
{\PaliGlossB{They eat once a day, once every second day, up to once a week, and so on, even up to once a fortnight. They live committed to the practice of eating food at set intervals.}}\\
\end{addmargin}
\end{absolutelynopagebreak}

\begin{absolutelynopagebreak}
\setstretch{.7}
{\PaliGlossA{So sākabhakkho vā hoti, sāmākabhakkho vā hoti, nīvārabhakkho vā hoti, daddulabhakkho vā hoti, haṭabhakkho vā hoti, kaṇabhakkho vā hoti, ācāmabhakkho vā hoti, piññākabhakkho vā hoti, tiṇabhakkho vā hoti, gomayabhakkho vā hoti, vanamūlaphalāhāro yāpeti pavattaphalabhojī.}}\\
\begin{addmargin}[1em]{2em}
\setstretch{.5}
{\PaliGlossB{They eat herbs, millet, wild rice, poor rice, water lettuce, rice bran, scum from boiling rice, sesame flour, grass, or cow dung. They survive on forest roots and fruits, or eating fallen fruit.}}\\
\end{addmargin}
\end{absolutelynopagebreak}

\begin{absolutelynopagebreak}
\setstretch{.7}
{\PaliGlossA{So sāṇānipi dhāreti, masāṇānipi dhāreti, chavadussānipi dhāreti, paṃsukūlānipi dhāreti, tirīṭānipi dhāreti, ajinampi dhāreti, ajinakkhipampi dhāreti, kusacīrampi dhāreti, vākacīrampi dhāreti, phalakacīrampi dhāreti, kesakambalampi dhāreti, vāḷakambalampi dhāreti, ulūkapakkhampi dhāreti;}}\\
\begin{addmargin}[1em]{2em}
\setstretch{.5}
{\PaliGlossB{They wear robes of sunn hemp, mixed hemp, corpse-wrapping cloth, rags, lodh tree bark, antelope hide (whole or in strips), kusa grass, bark, wood-chips, human hair, horse-tail hair, or owls’ wings.}}\\
\end{addmargin}
\end{absolutelynopagebreak}

\begin{absolutelynopagebreak}
\setstretch{.7}
{\PaliGlossA{kesamassulocakopi hoti kesamassulocanānuyogamanuyutto,}}\\
\begin{addmargin}[1em]{2em}
\setstretch{.5}
{\PaliGlossB{They tear out their hair and beard, committed to this practice.}}\\
\end{addmargin}
\end{absolutelynopagebreak}

\begin{absolutelynopagebreak}
\setstretch{.7}
{\PaliGlossA{ubbhaṭṭhakopi hoti āsanapaṭikkhitto,}}\\
\begin{addmargin}[1em]{2em}
\setstretch{.5}
{\PaliGlossB{They constantly stand, refusing seats.}}\\
\end{addmargin}
\end{absolutelynopagebreak}

\begin{absolutelynopagebreak}
\setstretch{.7}
{\PaliGlossA{ukkuṭikopi hoti ukkuṭikappadhānamanuyutto,}}\\
\begin{addmargin}[1em]{2em}
\setstretch{.5}
{\PaliGlossB{They squat, committed to persisting in the squatting position.}}\\
\end{addmargin}
\end{absolutelynopagebreak}

\begin{absolutelynopagebreak}
\setstretch{.7}
{\PaliGlossA{kaṇṭakāpassayikopi hoti kaṇṭakāpassaye seyyaṃ kappeti;}}\\
\begin{addmargin}[1em]{2em}
\setstretch{.5}
{\PaliGlossB{They lie on a mat of thorns, making a mat of thorns their bed.}}\\
\end{addmargin}
\end{absolutelynopagebreak}

\begin{absolutelynopagebreak}
\setstretch{.7}
{\PaliGlossA{sāyatatiyakampi udakorohanānuyogamanuyutto viharati—}}\\
\begin{addmargin}[1em]{2em}
\setstretch{.5}
{\PaliGlossB{They’re committed to the practice of immersion in water three times a day, including the evening.}}\\
\end{addmargin}
\end{absolutelynopagebreak}

\begin{absolutelynopagebreak}
\setstretch{.7}
{\PaliGlossA{iti evarūpaṃ anekavihitaṃ kāyassa ātāpanaparitāpanānuyogamanuyutto viharati.}}\\
\begin{addmargin}[1em]{2em}
\setstretch{.5}
{\PaliGlossB{And so they live committed to practicing these various ways of mortifying and tormenting the body.}}\\
\end{addmargin}
\end{absolutelynopagebreak}

\begin{absolutelynopagebreak}
\setstretch{.7}
{\PaliGlossA{Ayaṃ vuccati, brāhmaṇa, puggalo attantapo attaparitāpanānuyogamanuyutto.}}\\
\begin{addmargin}[1em]{2em}
\setstretch{.5}
{\PaliGlossB{This is called a person who mortifies themselves, being committed to the practice of mortifying themselves.}}\\
\end{addmargin}
\end{absolutelynopagebreak}

\vskip 0.05in
\begin{absolutelynopagebreak}
\setstretch{.7}
{\PaliGlossA{11. Katamo ca, brāhmaṇa, puggalo parantapo paraparitāpanānuyogamanuyutto?}}\\
\begin{addmargin}[1em]{2em}
\setstretch{.5}
{\PaliGlossB{And what person mortifies others, committed to the practice of mortifying others?}}\\
\end{addmargin}
\end{absolutelynopagebreak}

\begin{absolutelynopagebreak}
\setstretch{.7}
{\PaliGlossA{Idha, brāhmaṇa, ekacco puggalo orabbhiko hoti sūkariko sākuṇiko māgaviko luddo macchaghātako coro coraghātako goghātako bandhanāgāriko—ye vā panaññepi keci kurūrakammantā.}}\\
\begin{addmargin}[1em]{2em}
\setstretch{.5}
{\PaliGlossB{It’s when a person is a slaughterer of sheep, pigs, poultry, or deer, a hunter or fisher, a bandit, an executioner, a butcher of cattle, a jailer, or has some other cruel livelihood.}}\\
\end{addmargin}
\end{absolutelynopagebreak}

\begin{absolutelynopagebreak}
\setstretch{.7}
{\PaliGlossA{Ayaṃ vuccati, brāhmaṇa, puggalo parantapo paraparitāpanānuyogamanuyutto.}}\\
\begin{addmargin}[1em]{2em}
\setstretch{.5}
{\PaliGlossB{This is called a person who mortifies others, being committed to the practice of mortifying others.}}\\
\end{addmargin}
\end{absolutelynopagebreak}

\vskip 0.05in
\begin{absolutelynopagebreak}
\setstretch{.7}
{\PaliGlossA{12. Katamo ca, brāhmaṇa, puggalo attantapo ca attaparitāpanānuyogamanuyutto, parantapo ca paraparitāpanānuyogamanuyutto?}}\\
\begin{addmargin}[1em]{2em}
\setstretch{.5}
{\PaliGlossB{And what person mortifies themselves and others, being committed to the practice of mortifying themselves and others?}}\\
\end{addmargin}
\end{absolutelynopagebreak}

\begin{absolutelynopagebreak}
\setstretch{.7}
{\PaliGlossA{Idha, brāhmaṇa, ekacco puggalo rājā vā hoti khattiyo muddhāvasitto, brāhmaṇo vā mahāsālo.}}\\
\begin{addmargin}[1em]{2em}
\setstretch{.5}
{\PaliGlossB{It’s when a person is an anointed king or a well-to-do brahmin.}}\\
\end{addmargin}
\end{absolutelynopagebreak}

\begin{absolutelynopagebreak}
\setstretch{.7}
{\PaliGlossA{So puratthimena nagarassa navaṃ santhāgāraṃ kārāpetvā kesamassuṃ ohāretvā kharājinaṃ nivāsetvā sappitelena kāyaṃ abbhañjitvā magavisāṇena piṭṭhiṃ kaṇḍuvamāno navaṃ santhāgāraṃ pavisati saddhiṃ mahesiyā brāhmaṇena ca purohitena.}}\\
\begin{addmargin}[1em]{2em}
\setstretch{.5}
{\PaliGlossB{He has a new temple built to the east of the city. He shaves off his hair and beard, dresses in a rough antelope hide, and smears his body with ghee and oil. Scratching his back with antlers, he enters the temple with his chief queen and the brahmin high priest.}}\\
\end{addmargin}
\end{absolutelynopagebreak}

\begin{absolutelynopagebreak}
\setstretch{.7}
{\PaliGlossA{So tattha anantarahitāya bhūmiyā haritupalittāya seyyaṃ kappeti.}}\\
\begin{addmargin}[1em]{2em}
\setstretch{.5}
{\PaliGlossB{There he lies on the bare ground strewn with grass.}}\\
\end{addmargin}
\end{absolutelynopagebreak}

\begin{absolutelynopagebreak}
\setstretch{.7}
{\PaliGlossA{Ekissāya gāviyā sarūpavacchāya yaṃ ekasmiṃ thane khīraṃ hoti tena rājā yāpeti, yaṃ dutiyasmiṃ thane khīraṃ hoti tena mahesī yāpeti, yaṃ tatiyasmiṃ thane khīraṃ hoti tena brāhmaṇo purohito yāpeti, yaṃ catutthasmiṃ thane khīraṃ hoti tena aggiṃ juhati, avasesena vacchako yāpeti.}}\\
\begin{addmargin}[1em]{2em}
\setstretch{.5}
{\PaliGlossB{The king feeds on the milk from one teat of a cow that has a calf of the same color. The chief queen feeds on the milk from the second teat. The brahmin high priest feeds on the milk from the third teat. The milk from the fourth teat is offered to the flames. The calf feeds on the remainder.}}\\
\end{addmargin}
\end{absolutelynopagebreak}

\begin{absolutelynopagebreak}
\setstretch{.7}
{\PaliGlossA{So evamāha:}}\\
\begin{addmargin}[1em]{2em}
\setstretch{.5}
{\PaliGlossB{He says:}}\\
\end{addmargin}
\end{absolutelynopagebreak}

\begin{absolutelynopagebreak}
\setstretch{.7}
{\PaliGlossA{‘ettakā usabhā haññantu yaññatthāya, ettakā vacchatarā haññantu yaññatthāya, ettakā vacchatariyo haññantu yaññatthāya, ettakā ajā haññantu yaññatthāya, ettakā urabbhā haññantu yaññatthāya, ettakā assā haññantu yaññatthāya, ettakā rukkhā chijjantu yūpatthāya, ettakā dabbhā lūyantu barihisatthāyā’ti.}}\\
\begin{addmargin}[1em]{2em}
\setstretch{.5}
{\PaliGlossB{‘Slaughter this many bulls, bullocks, heifers, goats, rams, and horses for the sacrifice! Fell this many trees and reap this much grass for the sacrificial equipment!’}}\\
\end{addmargin}
\end{absolutelynopagebreak}

\begin{absolutelynopagebreak}
\setstretch{.7}
{\PaliGlossA{Yepissa te honti ‘dāsā’ti vā ‘pessā’ti vā ‘kammakarā’ti vā tepi daṇḍatajjitā bhayatajjitā assumukhā rudamānā parikammāni karonti.}}\\
\begin{addmargin}[1em]{2em}
\setstretch{.5}
{\PaliGlossB{His bondservants, employees, and workers do their jobs under threat of punishment and danger, weeping with tearful faces.}}\\
\end{addmargin}
\end{absolutelynopagebreak}

\begin{absolutelynopagebreak}
\setstretch{.7}
{\PaliGlossA{Ayaṃ vuccati, brāhmaṇa, puggalo attantapo ca attaparitāpanānuyogamanuyutto, parantapo ca paraparitāpanānuyogamanuyutto.}}\\
\begin{addmargin}[1em]{2em}
\setstretch{.5}
{\PaliGlossB{This is called a person who mortifies themselves and others, being committed to the practice of mortifying themselves and others.}}\\
\end{addmargin}
\end{absolutelynopagebreak}

\vskip 0.05in
\begin{absolutelynopagebreak}
\setstretch{.7}
{\PaliGlossA{13. Katamo ca, brāhmaṇa, puggalo nevattantapo nāttaparitāpanānuyogamanuyutto, na parantapo na paraparitāpanānuyogamanuyutto;}}\\
\begin{addmargin}[1em]{2em}
\setstretch{.5}
{\PaliGlossB{And what person doesn’t mortify either themselves or others, committed to the practice of not mortifying themselves or others,}}\\
\end{addmargin}
\end{absolutelynopagebreak}

\begin{absolutelynopagebreak}
\setstretch{.7}
{\PaliGlossA{so anattantapo aparantapo diṭṭheva dhamme nicchāto nibbuto sītībhūto sukhappaṭisaṃvedī brahmabhūtena attanā viharati?}}\\
\begin{addmargin}[1em]{2em}
\setstretch{.5}
{\PaliGlossB{living without wishes in the present life, extinguished, cooled, experiencing bliss, having become holy in themselves?}}\\
\end{addmargin}
\end{absolutelynopagebreak}

\vskip 0.05in
\begin{absolutelynopagebreak}
\setstretch{.7}
{\PaliGlossA{14. Idha, brāhmaṇa, tathāgato loke uppajjati arahaṃ sammāsambuddho vijjācaraṇasampanno sugato lokavidū anuttaro purisadammasārathi satthā devamanussānaṃ buddho bhagavā.}}\\
\begin{addmargin}[1em]{2em}
\setstretch{.5}
{\PaliGlossB{It’s when a Realized One arises in the world, perfected, a fully awakened Buddha, accomplished in knowledge and conduct, holy, knower of the world, supreme guide for those who wish to train, teacher of gods and humans, awakened, blessed.}}\\
\end{addmargin}
\end{absolutelynopagebreak}

\begin{absolutelynopagebreak}
\setstretch{.7}
{\PaliGlossA{So imaṃ lokaṃ sadevakaṃ samārakaṃ sabrahmakaṃ sassamaṇabrāhmaṇiṃ pajaṃ sadevamanussaṃ sayaṃ abhiññā sacchikatvā pavedeti.}}\\
\begin{addmargin}[1em]{2em}
\setstretch{.5}
{\PaliGlossB{He has realized with his own insight this world—with its gods, Māras and Brahmās, this population with its ascetics and brahmins, gods and humans—and he makes it known to others.}}\\
\end{addmargin}
\end{absolutelynopagebreak}

\begin{absolutelynopagebreak}
\setstretch{.7}
{\PaliGlossA{So dhammaṃ deseti ādikalyāṇaṃ majjhekalyāṇaṃ pariyosānakalyāṇaṃ sātthaṃ sabyañjanaṃ, kevalaparipuṇṇaṃ parisuddhaṃ brahmacariyaṃ pakāseti.}}\\
\begin{addmargin}[1em]{2em}
\setstretch{.5}
{\PaliGlossB{He teaches Dhamma that’s good in the beginning, good in the middle, and good in the end, meaningful and well-phrased. And he reveals a spiritual practice that’s entirely full and pure.}}\\
\end{addmargin}
\end{absolutelynopagebreak}

\vskip 0.05in
\begin{absolutelynopagebreak}
\setstretch{.7}
{\PaliGlossA{15. Taṃ dhammaṃ suṇāti gahapati vā gahapatiputto vā aññatarasmiṃ vā kule paccājāto.}}\\
\begin{addmargin}[1em]{2em}
\setstretch{.5}
{\PaliGlossB{A householder hears that teaching, or a householder’s child, or someone reborn in some clan.}}\\
\end{addmargin}
\end{absolutelynopagebreak}

\begin{absolutelynopagebreak}
\setstretch{.7}
{\PaliGlossA{So taṃ dhammaṃ sutvā tathāgate saddhaṃ paṭilabhati.}}\\
\begin{addmargin}[1em]{2em}
\setstretch{.5}
{\PaliGlossB{They gain faith in the Realized One,}}\\
\end{addmargin}
\end{absolutelynopagebreak}

\begin{absolutelynopagebreak}
\setstretch{.7}
{\PaliGlossA{So tena saddhāpaṭilābhena samannāgato iti paṭisañcikkhati:}}\\
\begin{addmargin}[1em]{2em}
\setstretch{.5}
{\PaliGlossB{and reflect:}}\\
\end{addmargin}
\end{absolutelynopagebreak}

\begin{absolutelynopagebreak}
\setstretch{.7}
{\PaliGlossA{‘sambādho gharāvāso rajopatho abbhokāso pabbajjā.}}\\
\begin{addmargin}[1em]{2em}
\setstretch{.5}
{\PaliGlossB{‘Living in a house is cramped and dirty, but the life of one gone forth is wide open.}}\\
\end{addmargin}
\end{absolutelynopagebreak}

\begin{absolutelynopagebreak}
\setstretch{.7}
{\PaliGlossA{Nayidaṃ sukaraṃ agāraṃ ajjhāvasatā ekantaparipuṇṇaṃ ekantaparisuddhaṃ saṅkhalikhitaṃ brahmacariyaṃ carituṃ.}}\\
\begin{addmargin}[1em]{2em}
\setstretch{.5}
{\PaliGlossB{It’s not easy for someone living at home to lead the spiritual life utterly full and pure, like a polished shell.}}\\
\end{addmargin}
\end{absolutelynopagebreak}

\begin{absolutelynopagebreak}
\setstretch{.7}
{\PaliGlossA{Yannūnāhaṃ kesamassuṃ ohāretvā kāsāyāni vatthāni acchādetvā agārasmā anagāriyaṃ pabbajeyyan’ti.}}\\
\begin{addmargin}[1em]{2em}
\setstretch{.5}
{\PaliGlossB{Why don’t I shave off my hair and beard, dress in ocher robes, and go forth from the lay life to homelessness?’}}\\
\end{addmargin}
\end{absolutelynopagebreak}

\begin{absolutelynopagebreak}
\setstretch{.7}
{\PaliGlossA{So aparena samayena appaṃ vā bhogakkhandhaṃ pahāya mahantaṃ vā bhogakkhandhaṃ pahāya, appaṃ vā ñātiparivaṭṭaṃ pahāya mahantaṃ vā ñātiparivaṭṭaṃ pahāya, kesamassuṃ ohāretvā kāsāyāni vatthāni acchādetvā agārasmā anagāriyaṃ pabbajati.}}\\
\begin{addmargin}[1em]{2em}
\setstretch{.5}
{\PaliGlossB{After some time they give up a large or small fortune, and a large or small family circle. They shave off hair and beard, dress in ocher robes, and go forth from the lay life to homelessness.}}\\
\end{addmargin}
\end{absolutelynopagebreak}

\begin{absolutelynopagebreak}
\setstretch{.7}
{\PaliGlossA{So evaṃ pabbajito samāno bhikkhūnaṃ sikkhāsājīvasamāpanno pāṇātipātaṃ pahāya pāṇātipātā paṭivirato hoti, nihitadaṇḍo nihitasattho lajjī dayāpanno sabbapāṇabhūtahitānukampī viharati.}}\\
\begin{addmargin}[1em]{2em}
\setstretch{.5}
{\PaliGlossB{Once they’ve gone forth, they take up the training and livelihood of the mendicants. They give up killing living creatures, renouncing the rod and the sword. They’re scrupulous and kind, living full of compassion for all living beings.}}\\
\end{addmargin}
\end{absolutelynopagebreak}

\vskip 0.05in
\begin{absolutelynopagebreak}
\setstretch{.7}
{\PaliGlossA{16. Adinnādānaṃ pahāya adinnādānā paṭivirato hoti dinnādāyī dinnapāṭikaṅkhī. Athenena sucibhūtena attanā viharati.}}\\
\begin{addmargin}[1em]{2em}
\setstretch{.5}
{\PaliGlossB{They give up stealing. They take only what’s given, and expect only what’s given. They keep themselves clean by not thieving.}}\\
\end{addmargin}
\end{absolutelynopagebreak}

\begin{absolutelynopagebreak}
\setstretch{.7}
{\PaliGlossA{Abrahmacariyaṃ pahāya brahmacārī hoti ārācārī virato methunā gāmadhammā.}}\\
\begin{addmargin}[1em]{2em}
\setstretch{.5}
{\PaliGlossB{They give up unchastity. They are celibate, set apart, avoiding the common practice of sex.}}\\
\end{addmargin}
\end{absolutelynopagebreak}

\begin{absolutelynopagebreak}
\setstretch{.7}
{\PaliGlossA{Musāvādaṃ pahāya musāvādā paṭivirato hoti saccavādī saccasandho theto paccayiko avisaṃvādako lokassa.}}\\
\begin{addmargin}[1em]{2em}
\setstretch{.5}
{\PaliGlossB{They give up lying. They speak the truth and stick to the truth. They’re honest and trustworthy, and don’t trick the world with their words.}}\\
\end{addmargin}
\end{absolutelynopagebreak}

\begin{absolutelynopagebreak}
\setstretch{.7}
{\PaliGlossA{Pisuṇaṃ vācaṃ pahāya pisuṇāya vācāya paṭivirato hoti; ito sutvā na amutra akkhātā imesaṃ bhedāya, amutra vā sutvā na imesaṃ akkhātā amūsaṃ bhedāya. Iti bhinnānaṃ vā sandhātā sahitānaṃ vā anuppadātā, samaggārāmo samaggarato samagganandī samaggakaraṇiṃ vācaṃ bhāsitā hoti.}}\\
\begin{addmargin}[1em]{2em}
\setstretch{.5}
{\PaliGlossB{They give up divisive speech. They don’t repeat in one place what they heard in another so as to divide people against each other. Instead, they reconcile those who are divided, supporting unity, delighting in harmony, loving harmony, speaking words that promote harmony.}}\\
\end{addmargin}
\end{absolutelynopagebreak}

\begin{absolutelynopagebreak}
\setstretch{.7}
{\PaliGlossA{Pharusaṃ vācaṃ pahāya pharusāya vācāya paṭivirato hoti. Yā sā vācā nelā kaṇṇasukhā pemanīyā hadayaṅgamā porī bahujanakantā bahujanamanāpā tathārūpiṃ vācaṃ bhāsitā hoti.}}\\
\begin{addmargin}[1em]{2em}
\setstretch{.5}
{\PaliGlossB{They give up harsh speech. They speak in a way that’s mellow, pleasing to the ear, lovely, going to the heart, polite, likable and agreeable to the people.}}\\
\end{addmargin}
\end{absolutelynopagebreak}

\begin{absolutelynopagebreak}
\setstretch{.7}
{\PaliGlossA{Samphappalāpaṃ pahāya samphappalāpā paṭivirato hoti, kālavādī bhūtavādī atthavādī dhammavādī vinayavādī, nidhānavatiṃ vācaṃ bhāsitā kālena sāpadesaṃ pariyantavatiṃ atthasaṃhitaṃ.}}\\
\begin{addmargin}[1em]{2em}
\setstretch{.5}
{\PaliGlossB{They give up talking nonsense. Their words are timely, true, and meaningful, in line with the teaching and training. They say things at the right time which are valuable, reasonable, succinct, and beneficial.}}\\
\end{addmargin}
\end{absolutelynopagebreak}

\begin{absolutelynopagebreak}
\setstretch{.7}
{\PaliGlossA{So bījagāmabhūtagāmasamārambhā paṭivirato hoti.}}\\
\begin{addmargin}[1em]{2em}
\setstretch{.5}
{\PaliGlossB{They avoid injuring plants and seeds.}}\\
\end{addmargin}
\end{absolutelynopagebreak}

\begin{absolutelynopagebreak}
\setstretch{.7}
{\PaliGlossA{Ekabhattiko hoti rattūparato virato vikālabhojanā.}}\\
\begin{addmargin}[1em]{2em}
\setstretch{.5}
{\PaliGlossB{They eat in one part of the day, abstaining from eating at night and food at the wrong time.}}\\
\end{addmargin}
\end{absolutelynopagebreak}

\begin{absolutelynopagebreak}
\setstretch{.7}
{\PaliGlossA{Naccagītavāditavisūkadassanā paṭivirato hoti.}}\\
\begin{addmargin}[1em]{2em}
\setstretch{.5}
{\PaliGlossB{They avoid dancing, singing, music, and seeing shows.}}\\
\end{addmargin}
\end{absolutelynopagebreak}

\begin{absolutelynopagebreak}
\setstretch{.7}
{\PaliGlossA{Mālāgandhavilepanadhāraṇamaṇḍanavibhūsanaṭṭhānā paṭivirato hoti.}}\\
\begin{addmargin}[1em]{2em}
\setstretch{.5}
{\PaliGlossB{They avoid beautifying and adorning themselves with garlands, perfumes, and makeup.}}\\
\end{addmargin}
\end{absolutelynopagebreak}

\begin{absolutelynopagebreak}
\setstretch{.7}
{\PaliGlossA{Uccāsayanamahāsayanā paṭivirato hoti.}}\\
\begin{addmargin}[1em]{2em}
\setstretch{.5}
{\PaliGlossB{They avoid high and luxurious beds.}}\\
\end{addmargin}
\end{absolutelynopagebreak}

\begin{absolutelynopagebreak}
\setstretch{.7}
{\PaliGlossA{Jātarūparajatapaṭiggahaṇā paṭivirato hoti.}}\\
\begin{addmargin}[1em]{2em}
\setstretch{.5}
{\PaliGlossB{They avoid receiving gold and money,}}\\
\end{addmargin}
\end{absolutelynopagebreak}

\begin{absolutelynopagebreak}
\setstretch{.7}
{\PaliGlossA{Āmakadhaññapaṭiggahaṇā paṭivirato hoti.}}\\
\begin{addmargin}[1em]{2em}
\setstretch{.5}
{\PaliGlossB{raw grains,}}\\
\end{addmargin}
\end{absolutelynopagebreak}

\begin{absolutelynopagebreak}
\setstretch{.7}
{\PaliGlossA{Āmakamaṃsapaṭiggahaṇā paṭivirato hoti.}}\\
\begin{addmargin}[1em]{2em}
\setstretch{.5}
{\PaliGlossB{raw meat,}}\\
\end{addmargin}
\end{absolutelynopagebreak}

\begin{absolutelynopagebreak}
\setstretch{.7}
{\PaliGlossA{Itthikumārikapaṭiggahaṇā paṭivirato hoti.}}\\
\begin{addmargin}[1em]{2em}
\setstretch{.5}
{\PaliGlossB{women and girls,}}\\
\end{addmargin}
\end{absolutelynopagebreak}

\begin{absolutelynopagebreak}
\setstretch{.7}
{\PaliGlossA{Dāsidāsapaṭiggahaṇā paṭivirato hoti.}}\\
\begin{addmargin}[1em]{2em}
\setstretch{.5}
{\PaliGlossB{male and female bondservants,}}\\
\end{addmargin}
\end{absolutelynopagebreak}

\begin{absolutelynopagebreak}
\setstretch{.7}
{\PaliGlossA{Ajeḷakapaṭiggahaṇā paṭivirato hoti.}}\\
\begin{addmargin}[1em]{2em}
\setstretch{.5}
{\PaliGlossB{goats and sheep,}}\\
\end{addmargin}
\end{absolutelynopagebreak}

\begin{absolutelynopagebreak}
\setstretch{.7}
{\PaliGlossA{Kukkuṭasūkarapaṭiggahaṇā paṭivirato hoti.}}\\
\begin{addmargin}[1em]{2em}
\setstretch{.5}
{\PaliGlossB{chickens and pigs,}}\\
\end{addmargin}
\end{absolutelynopagebreak}

\begin{absolutelynopagebreak}
\setstretch{.7}
{\PaliGlossA{Hatthigavassavaḷavapaṭiggahaṇā paṭivirato hoti.}}\\
\begin{addmargin}[1em]{2em}
\setstretch{.5}
{\PaliGlossB{elephants, cows, horses, and mares,}}\\
\end{addmargin}
\end{absolutelynopagebreak}

\begin{absolutelynopagebreak}
\setstretch{.7}
{\PaliGlossA{Khettavatthupaṭiggahaṇā paṭivirato hoti.}}\\
\begin{addmargin}[1em]{2em}
\setstretch{.5}
{\PaliGlossB{and fields and land.}}\\
\end{addmargin}
\end{absolutelynopagebreak}

\begin{absolutelynopagebreak}
\setstretch{.7}
{\PaliGlossA{Dūteyyapahiṇagamanānuyogā paṭivirato hoti.}}\\
\begin{addmargin}[1em]{2em}
\setstretch{.5}
{\PaliGlossB{They avoid running errands and messages;}}\\
\end{addmargin}
\end{absolutelynopagebreak}

\begin{absolutelynopagebreak}
\setstretch{.7}
{\PaliGlossA{Kayavikkayā paṭivirato hoti.}}\\
\begin{addmargin}[1em]{2em}
\setstretch{.5}
{\PaliGlossB{buying and selling;}}\\
\end{addmargin}
\end{absolutelynopagebreak}

\begin{absolutelynopagebreak}
\setstretch{.7}
{\PaliGlossA{Tulākūṭakaṃsakūṭamānakūṭā paṭivirato hoti.}}\\
\begin{addmargin}[1em]{2em}
\setstretch{.5}
{\PaliGlossB{falsifying weights, metals, or measures;}}\\
\end{addmargin}
\end{absolutelynopagebreak}

\begin{absolutelynopagebreak}
\setstretch{.7}
{\PaliGlossA{Ukkoṭanavañcananikatisāciyogā paṭivirato hoti.}}\\
\begin{addmargin}[1em]{2em}
\setstretch{.5}
{\PaliGlossB{bribery, fraud, cheating, and duplicity;}}\\
\end{addmargin}
\end{absolutelynopagebreak}

\begin{absolutelynopagebreak}
\setstretch{.7}
{\PaliGlossA{Chedanavadhabandhanaviparāmosaālopasahasākārā paṭivirato hoti.}}\\
\begin{addmargin}[1em]{2em}
\setstretch{.5}
{\PaliGlossB{mutilation, murder, abduction, banditry, plunder, and violence.}}\\
\end{addmargin}
\end{absolutelynopagebreak}

\vskip 0.05in
\begin{absolutelynopagebreak}
\setstretch{.7}
{\PaliGlossA{17. So santuṭṭho hoti kāyaparihārikena cīvarena, kucchiparihārikena piṇḍapātena. So yena yeneva pakkamati samādāyeva pakkamati.}}\\
\begin{addmargin}[1em]{2em}
\setstretch{.5}
{\PaliGlossB{They’re content with robes to look after the body and alms-food to look after the belly. Wherever they go, they set out taking only these things.}}\\
\end{addmargin}
\end{absolutelynopagebreak}

\begin{absolutelynopagebreak}
\setstretch{.7}
{\PaliGlossA{Seyyathāpi nāma pakkhī sakuṇo yena yeneva ḍeti sapattabhārova ḍeti;}}\\
\begin{addmargin}[1em]{2em}
\setstretch{.5}
{\PaliGlossB{They’re like a bird: wherever it flies, wings are its only burden.}}\\
\end{addmargin}
\end{absolutelynopagebreak}

\begin{absolutelynopagebreak}
\setstretch{.7}
{\PaliGlossA{evameva bhikkhu santuṭṭho hoti kāyaparihārikena cīvarena, kucchiparihārikena piṇḍapātena. So yena yeneva pakkamati samādāyeva pakkamati.}}\\
\begin{addmargin}[1em]{2em}
\setstretch{.5}
{\PaliGlossB{In the same way, a mendicant is content with robes to look after the body and alms-food to look after the belly. Wherever they go, they set out taking only these things.}}\\
\end{addmargin}
\end{absolutelynopagebreak}

\begin{absolutelynopagebreak}
\setstretch{.7}
{\PaliGlossA{So iminā ariyena sīlakkhandhena samannāgato ajjhattaṃ anavajjasukhaṃ paṭisaṃvedeti.}}\\
\begin{addmargin}[1em]{2em}
\setstretch{.5}
{\PaliGlossB{When they have this entire spectrum of noble ethics, they experience a blameless happiness inside themselves.}}\\
\end{addmargin}
\end{absolutelynopagebreak}

\vskip 0.05in
\begin{absolutelynopagebreak}
\setstretch{.7}
{\PaliGlossA{18. So cakkhunā rūpaṃ disvā na nimittaggāhī hoti nānubyañjanaggāhī.}}\\
\begin{addmargin}[1em]{2em}
\setstretch{.5}
{\PaliGlossB{When they see a sight with their eyes, they don’t get caught up in the features and details.}}\\
\end{addmargin}
\end{absolutelynopagebreak}

\begin{absolutelynopagebreak}
\setstretch{.7}
{\PaliGlossA{Yatvādhikaraṇamenaṃ cakkhundriyaṃ asaṃvutaṃ viharantaṃ abhijjhādomanassā pāpakā akusalā dhammā anvāssaveyyuṃ tassa saṃvarāya paṭipajjati, rakkhati cakkhundriyaṃ, cakkhundriye saṃvaraṃ āpajjati.}}\\
\begin{addmargin}[1em]{2em}
\setstretch{.5}
{\PaliGlossB{If the faculty of sight were left unrestrained, bad unskillful qualities of desire and aversion would become overwhelming. For this reason, they practice restraint, protecting the faculty of sight, and achieving its restraint.}}\\
\end{addmargin}
\end{absolutelynopagebreak}

\begin{absolutelynopagebreak}
\setstretch{.7}
{\PaliGlossA{Sotena saddaṃ sutvā … pe …}}\\
\begin{addmargin}[1em]{2em}
\setstretch{.5}
{\PaliGlossB{When they hear a sound with their ears …}}\\
\end{addmargin}
\end{absolutelynopagebreak}

\begin{absolutelynopagebreak}
\setstretch{.7}
{\PaliGlossA{ghānena gandhaṃ ghāyitvā …}}\\
\begin{addmargin}[1em]{2em}
\setstretch{.5}
{\PaliGlossB{When they smell an odor with their nose …}}\\
\end{addmargin}
\end{absolutelynopagebreak}

\begin{absolutelynopagebreak}
\setstretch{.7}
{\PaliGlossA{jivhāya rasaṃ sāyitvā …}}\\
\begin{addmargin}[1em]{2em}
\setstretch{.5}
{\PaliGlossB{When they taste a flavor with their tongue …}}\\
\end{addmargin}
\end{absolutelynopagebreak}

\begin{absolutelynopagebreak}
\setstretch{.7}
{\PaliGlossA{kāyena phoṭṭhabbaṃ phusitvā …}}\\
\begin{addmargin}[1em]{2em}
\setstretch{.5}
{\PaliGlossB{When they feel a touch with their body …}}\\
\end{addmargin}
\end{absolutelynopagebreak}

\begin{absolutelynopagebreak}
\setstretch{.7}
{\PaliGlossA{manasā dhammaṃ viññāya na nimittaggāhī hoti nānubyañjanaggāhī.}}\\
\begin{addmargin}[1em]{2em}
\setstretch{.5}
{\PaliGlossB{When they know a thought with their mind, they don’t get caught up in the features and details.}}\\
\end{addmargin}
\end{absolutelynopagebreak}

\begin{absolutelynopagebreak}
\setstretch{.7}
{\PaliGlossA{Yatvādhikaraṇamenaṃ manindriyaṃ asaṃvutaṃ viharantaṃ abhijjhādomanassā pāpakā akusalā dhammā anvāssaveyyuṃ tassa saṃvarāya paṭipajjati, rakkhati manindriyaṃ, manindriye saṃvaraṃ āpajjati.}}\\
\begin{addmargin}[1em]{2em}
\setstretch{.5}
{\PaliGlossB{If the faculty of mind were left unrestrained, bad unskillful qualities of desire and aversion would become overwhelming. For this reason, they practice restraint, protecting the faculty of mind, and achieving its restraint.}}\\
\end{addmargin}
\end{absolutelynopagebreak}

\begin{absolutelynopagebreak}
\setstretch{.7}
{\PaliGlossA{So iminā ariyena indriyasaṃvarena samannāgato ajjhattaṃ abyāsekasukhaṃ paṭisaṃvedeti.}}\\
\begin{addmargin}[1em]{2em}
\setstretch{.5}
{\PaliGlossB{When they have this noble sense restraint, they experience an unsullied bliss inside themselves.}}\\
\end{addmargin}
\end{absolutelynopagebreak}

\vskip 0.05in
\begin{absolutelynopagebreak}
\setstretch{.7}
{\PaliGlossA{19. So abhikkante paṭikkante sampajānakārī hoti, ālokite vilokite sampajānakārī hoti, samiñjite pasārite sampajānakārī hoti, saṅghāṭipattacīvaradhāraṇe sampajānakārī hoti, asite pīte khāyite sāyite sampajānakārī hoti, uccārapassāvakamme sampajānakārī hoti, gate ṭhite nisinne sutte jāgarite bhāsite tuṇhībhāve sampajānakārī hoti.}}\\
\begin{addmargin}[1em]{2em}
\setstretch{.5}
{\PaliGlossB{They act with situational awareness when going out and coming back; when looking ahead and aside; when bending and extending the limbs; when bearing the outer robe, bowl and robes; when eating, drinking, chewing, and tasting; when urinating and defecating; when walking, standing, sitting, sleeping, waking, speaking, and keeping silent.}}\\
\end{addmargin}
\end{absolutelynopagebreak}

\vskip 0.05in
\begin{absolutelynopagebreak}
\setstretch{.7}
{\PaliGlossA{20. So iminā ca ariyena sīlakkhandhena samannāgato, imāya ca ariyāya santuṭṭhiyā samannāgato, iminā ca ariyena indriyasaṃvarena samannāgato, iminā ca ariyena satisampajaññena samannāgato}}\\
\begin{addmargin}[1em]{2em}
\setstretch{.5}
{\PaliGlossB{When they have this noble spectrum of ethics, this noble sense restraint, and this noble mindfulness and situational awareness,}}\\
\end{addmargin}
\end{absolutelynopagebreak}

\begin{absolutelynopagebreak}
\setstretch{.7}
{\PaliGlossA{vivittaṃ senāsanaṃ bhajati araññaṃ rukkhamūlaṃ pabbataṃ kandaraṃ giriguhaṃ susānaṃ vanapatthaṃ abbhokāsaṃ palālapuñjaṃ.}}\\
\begin{addmargin}[1em]{2em}
\setstretch{.5}
{\PaliGlossB{they frequent a secluded lodging—a wilderness, the root of a tree, a hill, a ravine, a mountain cave, a charnel ground, a forest, the open air, a heap of straw.}}\\
\end{addmargin}
\end{absolutelynopagebreak}

\vskip 0.05in
\begin{absolutelynopagebreak}
\setstretch{.7}
{\PaliGlossA{21. So pacchābhattaṃ piṇḍapātapaṭikkanto nisīdati pallaṅkaṃ ābhujitvā, ujuṃ kāyaṃ paṇidhāya, parimukhaṃ satiṃ upaṭṭhapetvā.}}\\
\begin{addmargin}[1em]{2em}
\setstretch{.5}
{\PaliGlossB{After the meal, they return from alms-round, sit down cross-legged with their body straight, and establish mindfulness right there.}}\\
\end{addmargin}
\end{absolutelynopagebreak}

\begin{absolutelynopagebreak}
\setstretch{.7}
{\PaliGlossA{So abhijjhaṃ loke pahāya vigatābhijjhena cetasā viharati, abhijjhāya cittaṃ parisodheti;}}\\
\begin{addmargin}[1em]{2em}
\setstretch{.5}
{\PaliGlossB{Giving up desire for the world, they meditate with a heart rid of desire, cleansing the mind of desire.}}\\
\end{addmargin}
\end{absolutelynopagebreak}

\begin{absolutelynopagebreak}
\setstretch{.7}
{\PaliGlossA{byāpādapadosaṃ pahāya abyāpannacitto viharati sabbapāṇabhūtahitānukampī, byāpādapadosā cittaṃ parisodheti;}}\\
\begin{addmargin}[1em]{2em}
\setstretch{.5}
{\PaliGlossB{Giving up ill will and malevolence, they meditate with a mind rid of ill will, full of compassion for all living beings, cleansing the mind of ill will.}}\\
\end{addmargin}
\end{absolutelynopagebreak}

\begin{absolutelynopagebreak}
\setstretch{.7}
{\PaliGlossA{thinamiddhaṃ pahāya vigatathinamiddho viharati ālokasaññī sato sampajāno, thinamiddhā cittaṃ parisodheti;}}\\
\begin{addmargin}[1em]{2em}
\setstretch{.5}
{\PaliGlossB{Giving up dullness and drowsiness, they meditate with a mind rid of dullness and drowsiness, perceiving light, mindful and aware, cleansing the mind of dullness and drowsiness.}}\\
\end{addmargin}
\end{absolutelynopagebreak}

\begin{absolutelynopagebreak}
\setstretch{.7}
{\PaliGlossA{uddhaccakukkuccaṃ pahāya anuddhato viharati ajjhattaṃ vūpasantacitto, uddhaccakukkuccā cittaṃ parisodheti;}}\\
\begin{addmargin}[1em]{2em}
\setstretch{.5}
{\PaliGlossB{Giving up restlessness and remorse, they meditate without restlessness, their mind peaceful inside, cleansing the mind of restlessness and remorse.}}\\
\end{addmargin}
\end{absolutelynopagebreak}

\begin{absolutelynopagebreak}
\setstretch{.7}
{\PaliGlossA{vicikicchaṃ pahāya tiṇṇavicikiccho viharati akathaṃkathī kusalesu dhammesu, vicikicchāya cittaṃ parisodheti.}}\\
\begin{addmargin}[1em]{2em}
\setstretch{.5}
{\PaliGlossB{Giving up doubt, they meditate having gone beyond doubt, not undecided about skillful qualities, cleansing the mind of doubt.}}\\
\end{addmargin}
\end{absolutelynopagebreak}

\vskip 0.05in
\begin{absolutelynopagebreak}
\setstretch{.7}
{\PaliGlossA{22. So ime pañca nīvaraṇe pahāya cetaso upakkilese paññāya dubbalīkaraṇe}}\\
\begin{addmargin}[1em]{2em}
\setstretch{.5}
{\PaliGlossB{They give up these five hindrances, corruptions of the heart that weaken wisdom.}}\\
\end{addmargin}
\end{absolutelynopagebreak}

\begin{absolutelynopagebreak}
\setstretch{.7}
{\PaliGlossA{vivicceva kāmehi vivicca akusalehi dhammehi savitakkaṃ savicāraṃ vivekajaṃ pītisukhaṃ paṭhamaṃ jhānaṃ upasampajja viharati.}}\\
\begin{addmargin}[1em]{2em}
\setstretch{.5}
{\PaliGlossB{Then, quite secluded from sensual pleasures, secluded from unskillful qualities, they enter and remain in the first absorption, which has the rapture and bliss born of seclusion, while placing the mind and keeping it connected.}}\\
\end{addmargin}
\end{absolutelynopagebreak}

\vskip 0.05in
\begin{absolutelynopagebreak}
\setstretch{.7}
{\PaliGlossA{23. Vitakkavicārānaṃ vūpasamā ajjhattaṃ sampasādanaṃ cetaso ekodibhāvaṃ avitakkaṃ avicāraṃ samādhijaṃ pītisukhaṃ dutiyaṃ jhānaṃ upasampajja viharati.}}\\
\begin{addmargin}[1em]{2em}
\setstretch{.5}
{\PaliGlossB{As the placing of the mind and keeping it connected are stilled, they enter and remain in the second absorption, which has the rapture and bliss born of immersion, with internal clarity and confidence, and unified mind, without placing the mind and keeping it connected.}}\\
\end{addmargin}
\end{absolutelynopagebreak}

\vskip 0.05in
\begin{absolutelynopagebreak}
\setstretch{.7}
{\PaliGlossA{24. Pītiyā ca virāgā upekkhako ca viharati sato ca sampajāno, sukhañca kāyena paṭisaṃvedeti, yaṃ taṃ ariyā ācikkhanti: ‘upekkhako satimā sukhavihārī’ti tatiyaṃ jhānaṃ upasampajja viharati.}}\\
\begin{addmargin}[1em]{2em}
\setstretch{.5}
{\PaliGlossB{And with the fading away of rapture, they enter and remain in the third absorption, where they meditate with equanimity, mindful and aware, personally experiencing the bliss of which the noble ones declare, ‘Equanimous and mindful, one meditates in bliss.’}}\\
\end{addmargin}
\end{absolutelynopagebreak}

\vskip 0.05in
\begin{absolutelynopagebreak}
\setstretch{.7}
{\PaliGlossA{25. Sukhassa ca pahānā dukkhassa ca pahānā, pubbeva somanassadomanassānaṃ atthaṅgamā, adukkhamasukhaṃ upekkhāsatipārisuddhiṃ catutthaṃ jhānaṃ upasampajja viharati.}}\\
\begin{addmargin}[1em]{2em}
\setstretch{.5}
{\PaliGlossB{Giving up pleasure and pain, and ending former happiness and sadness, they enter and remain in the fourth absorption, without pleasure or pain, with pure equanimity and mindfulness.}}\\
\end{addmargin}
\end{absolutelynopagebreak}

\vskip 0.05in
\begin{absolutelynopagebreak}
\setstretch{.7}
{\PaliGlossA{26. So evaṃ samāhite citte parisuddhe pariyodāte anaṅgaṇe vigatūpakkilese mudubhūte kammaniye ṭhite āneñjappatte pubbenivāsānussatiñāṇāya cittaṃ abhininnāmeti.}}\\
\begin{addmargin}[1em]{2em}
\setstretch{.5}
{\PaliGlossB{When their mind has become immersed in samādhi like this—purified, bright, flawless, rid of corruptions, pliable, workable, steady, and imperturbable—they extend it toward recollection of past lives.}}\\
\end{addmargin}
\end{absolutelynopagebreak}

\begin{absolutelynopagebreak}
\setstretch{.7}
{\PaliGlossA{So anekavihitaṃ pubbenivāsaṃ anussarati, seyyathidaṃ—ekampi jātiṃ dvepi jātiyo tissopi jātiyo catassopi jātiyo pañcapi jātiyo dasapi jātiyo vīsampi jātiyo tiṃsampi jātiyo cattālīsampi jātiyo paññāsampi jātiyo jātisatampi jātisahassampi jātisatasahassampi, anekepi saṃvaṭṭakappe anekepi vivaṭṭakappe anekepi saṃvaṭṭavivaṭṭakappe: ‘amutrāsiṃ evaṃnāmo evaṅgotto evaṃvaṇṇo evamāhāro evaṃsukhadukkhappaṭisaṃvedī evamāyupariyanto; so tato cuto amutra udapādiṃ; tatrāpāsiṃ evaṃnāmo evaṅgotto evaṃvaṇṇo evamāhāro evaṃsukhadukkhappaṭisaṃvedī evamāyupariyanto; so tato cuto idhūpapanno’ti. Iti sākāraṃ sauddesaṃ anekavihitaṃ pubbenivāsaṃ anussarati.}}\\
\begin{addmargin}[1em]{2em}
\setstretch{.5}
{\PaliGlossB{They recollect many kinds of past lives. That is: one, two, three, four, five, ten, twenty, thirty, forty, fifty, a hundred, a thousand, a hundred thousand rebirths; many eons of the world contracting, many eons of the world expanding, many eons of the world contracting and expanding. They remember: ‘There, I was named this, my clan was that, I looked like this, and that was my food. This was how I felt pleasure and pain, and that was how my life ended. When I passed away from that place I was reborn somewhere else. There, too, I was named this, my clan was that, I looked like this, and that was my food. This was how I felt pleasure and pain, and that was how my life ended. Passing away from there, I was reborn here.’ And so they recollect their many kinds of past lives, with features and details.}}\\
\end{addmargin}
\end{absolutelynopagebreak}

\vskip 0.05in
\begin{absolutelynopagebreak}
\setstretch{.7}
{\PaliGlossA{27. So evaṃ samāhite citte parisuddhe pariyodāte anaṅgaṇe vigatūpakkilese mudubhūte kammaniye ṭhite āneñjappatte sattānaṃ cutūpapātañāṇāya cittaṃ abhininnāmeti.}}\\
\begin{addmargin}[1em]{2em}
\setstretch{.5}
{\PaliGlossB{When their mind has become immersed in samādhi like this—purified, bright, flawless, rid of corruptions, pliable, workable, steady, and imperturbable—they extend it toward knowledge of the death and rebirth of sentient beings.}}\\
\end{addmargin}
\end{absolutelynopagebreak}

\begin{absolutelynopagebreak}
\setstretch{.7}
{\PaliGlossA{So dibbena cakkhunā visuddhena atikkantamānusakena satte passati cavamāne upapajjamāne hīne paṇīte suvaṇṇe dubbaṇṇe sugate duggate yathākammūpage satte pajānāti: ‘ime vata bhonto sattā kāyaduccaritena samannāgatā … pe … ariyānaṃ upavādakā micchādiṭṭhikā micchādiṭṭhikammasamādānā, te kāyassa bhedā paraṃ maraṇā apāyaṃ duggatiṃ vinipātaṃ nirayaṃ upapannā. Ime vā pana bhonto sattā kāyasucaritena samannāgatā … pe … ariyānaṃ anupavādakā sammādiṭṭhikā sammādiṭṭhikammasamādānā, te kāyassa bhedā paraṃ maraṇā sugatiṃ saggaṃ lokaṃ upapannā’ti. Iti dibbena cakkhunā visuddhena atikkantamānusakena satte passati cavamāne upapajjamāne hīne paṇīte suvaṇṇe dubbaṇṇe sugate duggate yathākammūpage satte pajānāti.}}\\
\begin{addmargin}[1em]{2em}
\setstretch{.5}
{\PaliGlossB{With clairvoyance that is purified and superhuman, they see sentient beings passing away and being reborn—inferior and superior, beautiful and ugly, in a good place or a bad place. They understand how sentient beings are reborn according to their deeds: ‘These dear beings did bad things by way of body, speech, and mind. They spoke ill of the noble ones; they had wrong view; and they chose to act out of that wrong view. When their body breaks up, after death, they’re reborn in a place of loss, a bad place, the underworld, hell. These dear beings, however, did good things by way of body, speech, and mind. They never spoke ill of the noble ones; they had right view; and they chose to act out of that right view. When their body breaks up, after death, they’re reborn in a good place, a heavenly realm.’ And so, with clairvoyance that is purified and superhuman, they see sentient beings passing away and being reborn—inferior and superior, beautiful and ugly, in a good place or a bad place. They understand how sentient beings are reborn according to their deeds.}}\\
\end{addmargin}
\end{absolutelynopagebreak}

\vskip 0.05in
\begin{absolutelynopagebreak}
\setstretch{.7}
{\PaliGlossA{28. So evaṃ samāhite citte parisuddhe pariyodāte anaṅgaṇe vigatūpakkilese mudubhūte kammaniye ṭhite āneñjappatte āsavānaṃ khayañāṇāya cittaṃ abhininnāmeti.}}\\
\begin{addmargin}[1em]{2em}
\setstretch{.5}
{\PaliGlossB{When their mind has become immersed in samādhi like this—purified, bright, flawless, rid of corruptions, pliable, workable, steady, and imperturbable—they extend it toward knowledge of the ending of defilements.}}\\
\end{addmargin}
\end{absolutelynopagebreak}

\begin{absolutelynopagebreak}
\setstretch{.7}
{\PaliGlossA{So ‘idaṃ dukkhan’ti yathābhūtaṃ pajānāti, ‘ayaṃ dukkhasamudayo’ti yathābhūtaṃ pajānāti, ‘ayaṃ dukkhanirodho’ti yathābhūtaṃ pajānāti, ‘ayaṃ dukkhanirodhagāminī paṭipadā’ti yathābhūtaṃ pajānāti;}}\\
\begin{addmargin}[1em]{2em}
\setstretch{.5}
{\PaliGlossB{They truly understand: ‘This is suffering’ … ‘This is the origin of suffering’ … ‘This is the cessation of suffering’ … ‘This is the practice that leads to the cessation of suffering’.}}\\
\end{addmargin}
\end{absolutelynopagebreak}

\begin{absolutelynopagebreak}
\setstretch{.7}
{\PaliGlossA{‘ime āsavā’ti yathābhūtaṃ pajānāti, ‘ayaṃ āsavasamudayo’ti yathābhūtaṃ pajānāti, ‘ayaṃ āsavanirodho’ti yathābhūtaṃ pajānāti, ‘ayaṃ āsavanirodhagāminī paṭipadā’ti yathābhūtaṃ pajānāti.}}\\
\begin{addmargin}[1em]{2em}
\setstretch{.5}
{\PaliGlossB{They truly understand: ‘These are defilements’ … ‘This is the origin of defilements’ … ‘This is the cessation of defilements’ … ‘This is the practice that leads to the cessation of defilements’.}}\\
\end{addmargin}
\end{absolutelynopagebreak}

\vskip 0.05in
\begin{absolutelynopagebreak}
\setstretch{.7}
{\PaliGlossA{29. Tassa evaṃ jānato evaṃ passato kāmāsavāpi cittaṃ vimuccati, bhavāsavāpi cittaṃ vimuccati, avijjāsavāpi cittaṃ vimuccati.}}\\
\begin{addmargin}[1em]{2em}
\setstretch{.5}
{\PaliGlossB{Knowing and seeing like this, their mind is freed from the defilements of sensuality, desire to be reborn, and ignorance.}}\\
\end{addmargin}
\end{absolutelynopagebreak}

\begin{absolutelynopagebreak}
\setstretch{.7}
{\PaliGlossA{Vimuttasmiṃ vimuttamiti ñāṇaṃ hoti.}}\\
\begin{addmargin}[1em]{2em}
\setstretch{.5}
{\PaliGlossB{When they’re freed, they know they’re freed.}}\\
\end{addmargin}
\end{absolutelynopagebreak}

\begin{absolutelynopagebreak}
\setstretch{.7}
{\PaliGlossA{‘Khīṇā jāti, vusitaṃ brahmacariyaṃ, kataṃ karaṇīyaṃ, nāparaṃ itthattāyā’ti pajānāti.}}\\
\begin{addmargin}[1em]{2em}
\setstretch{.5}
{\PaliGlossB{They understand: ‘Rebirth is ended, the spiritual journey has been completed, what had to be done has been done, there is no return to any state of existence.’}}\\
\end{addmargin}
\end{absolutelynopagebreak}

\vskip 0.05in
\begin{absolutelynopagebreak}
\setstretch{.7}
{\PaliGlossA{30. Ayaṃ vuccati, brāhmaṇa, puggalo nevattantapo nāttaparitāpanānuyogamanuyutto, na parantapo na paraparitāpanānuyogamanuyutto.}}\\
\begin{addmargin}[1em]{2em}
\setstretch{.5}
{\PaliGlossB{This is called a person who neither mortifies themselves or others, being committed to the practice of not mortifying themselves or others.}}\\
\end{addmargin}
\end{absolutelynopagebreak}

\begin{absolutelynopagebreak}
\setstretch{.7}
{\PaliGlossA{So anattantapo aparantapo diṭṭheva dhamme nicchāto nibbuto sītībhūto sukhappaṭisaṃvedī brahmabhūtena attanā viharatī”ti.}}\\
\begin{addmargin}[1em]{2em}
\setstretch{.5}
{\PaliGlossB{They live without wishes in the present life, extinguished, cooled, experiencing bliss, having become holy in themselves.”}}\\
\end{addmargin}
\end{absolutelynopagebreak}

\vskip 0.05in
\begin{absolutelynopagebreak}
\setstretch{.7}
{\PaliGlossA{31. Evaṃ vutte, ghoṭamukho brāhmaṇo āyasmantaṃ udenaṃ etadavoca:}}\\
\begin{addmargin}[1em]{2em}
\setstretch{.5}
{\PaliGlossB{When he had spoken, Ghoṭamukha said to him,}}\\
\end{addmargin}
\end{absolutelynopagebreak}

\begin{absolutelynopagebreak}
\setstretch{.7}
{\PaliGlossA{“abhikkantaṃ, bho udena, abhikkantaṃ, bho udena.}}\\
\begin{addmargin}[1em]{2em}
\setstretch{.5}
{\PaliGlossB{“Excellent, Master Udena! Excellent!}}\\
\end{addmargin}
\end{absolutelynopagebreak}

\begin{absolutelynopagebreak}
\setstretch{.7}
{\PaliGlossA{Seyyathāpi, bho udena, nikkujjitaṃ vā ukkujjeyya, paṭicchannaṃ vā vivareyya, mūḷhassa vā maggaṃ ācikkheyya, andhakāre vā telapajjotaṃ dhāreyya ‘cakkhumanto rūpāni dakkhantī’ti; evamevaṃ bhotā udenena anekapariyāyena dhammo pakāsito.}}\\
\begin{addmargin}[1em]{2em}
\setstretch{.5}
{\PaliGlossB{As if he were righting the overturned, or revealing the hidden, or pointing out the path to the lost, or lighting a lamp in the dark so people with good eyes can see what’s there, Master Udena has made the teaching clear in many ways.}}\\
\end{addmargin}
\end{absolutelynopagebreak}

\begin{absolutelynopagebreak}
\setstretch{.7}
{\PaliGlossA{Esāhaṃ bhavantaṃ udenaṃ saraṇaṃ gacchāmi dhammañca bhikkhusaṅghañca.}}\\
\begin{addmargin}[1em]{2em}
\setstretch{.5}
{\PaliGlossB{I go for refuge to Master Udena, to the teaching, and to the mendicant Saṅgha.}}\\
\end{addmargin}
\end{absolutelynopagebreak}

\begin{absolutelynopagebreak}
\setstretch{.7}
{\PaliGlossA{Upāsakaṃ maṃ bhavaṃ udeno dhāretu ajjatagge pāṇupetaṃ saraṇaṃ gatan”ti.}}\\
\begin{addmargin}[1em]{2em}
\setstretch{.5}
{\PaliGlossB{From this day forth, may Master Udena remember me as a lay follower who has gone for refuge for life.”}}\\
\end{addmargin}
\end{absolutelynopagebreak}

\vskip 0.05in
\begin{absolutelynopagebreak}
\setstretch{.7}
{\PaliGlossA{32. “Mā kho maṃ tvaṃ, brāhmaṇa, saraṇaṃ agamāsi.}}\\
\begin{addmargin}[1em]{2em}
\setstretch{.5}
{\PaliGlossB{“Brahmin, don’t go for refuge to me.}}\\
\end{addmargin}
\end{absolutelynopagebreak}

\begin{absolutelynopagebreak}
\setstretch{.7}
{\PaliGlossA{Tameva bhagavantaṃ saraṇaṃ gacchāhi yamahaṃ saraṇaṃ gato”ti.}}\\
\begin{addmargin}[1em]{2em}
\setstretch{.5}
{\PaliGlossB{You should go for refuge to that same Blessed One to whom I have gone for refuge.”}}\\
\end{addmargin}
\end{absolutelynopagebreak}

\begin{absolutelynopagebreak}
\setstretch{.7}
{\PaliGlossA{“Kahaṃ pana, bho udena, etarahi so bhavaṃ gotamo viharati arahaṃ sammāsambuddho”ti?}}\\
\begin{addmargin}[1em]{2em}
\setstretch{.5}
{\PaliGlossB{“But Master Udena, where is the Blessed One at present, the perfected one, the fully awakened Buddha?”}}\\
\end{addmargin}
\end{absolutelynopagebreak}

\begin{absolutelynopagebreak}
\setstretch{.7}
{\PaliGlossA{“Parinibbuto kho, brāhmaṇa, etarahi so bhagavā arahaṃ sammāsambuddho”ti.}}\\
\begin{addmargin}[1em]{2em}
\setstretch{.5}
{\PaliGlossB{“Brahmin, the Buddha has already become fully extinguished.”}}\\
\end{addmargin}
\end{absolutelynopagebreak}

\begin{absolutelynopagebreak}
\setstretch{.7}
{\PaliGlossA{“Sacepi mayaṃ, bho udena, suṇeyyāma taṃ bhavantaṃ gotamaṃ dasasu yojanesu, dasapi mayaṃ yojanāni gaccheyyāma taṃ bhavantaṃ gotamaṃ dassanāya arahantaṃ sammāsambuddhaṃ.}}\\
\begin{addmargin}[1em]{2em}
\setstretch{.5}
{\PaliGlossB{“Master Udena, if I heard that the Buddha was within ten leagues, or twenty, or even up to a hundred leagues away, I’d go a hundred leagues to see him.}}\\
\end{addmargin}
\end{absolutelynopagebreak}

\begin{absolutelynopagebreak}
\setstretch{.7}
{\PaliGlossA{Sacepi mayaṃ, bho udena, suṇeyyāma taṃ bhavantaṃ gotamaṃ vīsatiyā yojanesu …}}\\
\begin{addmargin}[1em]{2em}
\setstretch{.5}
{\PaliGlossB{    -}}\\
\end{addmargin}
\end{absolutelynopagebreak}

\begin{absolutelynopagebreak}
\setstretch{.7}
{\PaliGlossA{tiṃsāya yojanesu …}}\\
\begin{addmargin}[1em]{2em}
\setstretch{.5}
{\PaliGlossB{    -}}\\
\end{addmargin}
\end{absolutelynopagebreak}

\begin{absolutelynopagebreak}
\setstretch{.7}
{\PaliGlossA{cattārīsāya yojanesu …}}\\
\begin{addmargin}[1em]{2em}
\setstretch{.5}
{\PaliGlossB{    -}}\\
\end{addmargin}
\end{absolutelynopagebreak}

\begin{absolutelynopagebreak}
\setstretch{.7}
{\PaliGlossA{paññāsāya yojanesu, paññāsampi mayaṃ yojanāni gaccheyyāma taṃ bhavantaṃ gotamaṃ dassanāya arahantaṃ sammāsambuddhaṃ.}}\\
\begin{addmargin}[1em]{2em}
\setstretch{.5}
{\PaliGlossB{    -}}\\
\end{addmargin}
\end{absolutelynopagebreak}

\begin{absolutelynopagebreak}
\setstretch{.7}
{\PaliGlossA{Yojanasate cepi mayaṃ, bho udena, suṇeyyāma taṃ bhavantaṃ gotamaṃ, yojanasatampi mayaṃ gaccheyyāma taṃ bhavantaṃ gotamaṃ dassanāya arahantaṃ sammāsambuddhaṃ.}}\\
\begin{addmargin}[1em]{2em}
\setstretch{.5}
{\PaliGlossB{    -}}\\
\end{addmargin}
\end{absolutelynopagebreak}

\begin{absolutelynopagebreak}
\setstretch{.7}
{\PaliGlossA{Yato ca kho, bho udena, parinibbuto so bhavaṃ gotamo, parinibbutampi mayaṃ taṃ bhavantaṃ gotamaṃ saraṇaṃ gacchāma dhammañca bhikkhusaṅghañca.}}\\
\begin{addmargin}[1em]{2em}
\setstretch{.5}
{\PaliGlossB{But since the Buddha has become fully extinguished, I go for refuge to that fully extinguished Buddha, to the teaching, and to the Saṅgha.}}\\
\end{addmargin}
\end{absolutelynopagebreak}

\begin{absolutelynopagebreak}
\setstretch{.7}
{\PaliGlossA{Upāsakaṃ maṃ bhavaṃ udeno dhāretu ajjatagge pāṇupetaṃ saraṇaṃ gataṃ.}}\\
\begin{addmargin}[1em]{2em}
\setstretch{.5}
{\PaliGlossB{From this day forth, may Master Udena remember me as a lay follower who has gone for refuge for life.}}\\
\end{addmargin}
\end{absolutelynopagebreak}

\vskip 0.05in
\begin{absolutelynopagebreak}
\setstretch{.7}
{\PaliGlossA{33. Atthi ca me, bho udena, aṅgarājā devasikaṃ niccabhikkhaṃ dadāti, tato ahaṃ bhoto udenassa ekaṃ niccabhikkhaṃ dadāmī”ti.}}\\
\begin{addmargin}[1em]{2em}
\setstretch{.5}
{\PaliGlossB{Master Udena, the king of Aṅga gives me a regular daily allowance. I will give you one portion of that.”}}\\
\end{addmargin}
\end{absolutelynopagebreak}

\begin{absolutelynopagebreak}
\setstretch{.7}
{\PaliGlossA{“Kiṃ pana te, brāhmaṇa, aṅgarājā devasikaṃ niccabhikkhaṃ dadātī”ti?}}\\
\begin{addmargin}[1em]{2em}
\setstretch{.5}
{\PaliGlossB{“But brahmin, what does the king of Aṅga give you as a regular daily allowance?”}}\\
\end{addmargin}
\end{absolutelynopagebreak}

\begin{absolutelynopagebreak}
\setstretch{.7}
{\PaliGlossA{“Pañca, bho udena, kahāpaṇasatānī”ti.}}\\
\begin{addmargin}[1em]{2em}
\setstretch{.5}
{\PaliGlossB{“Five hundred dollars.”}}\\
\end{addmargin}
\end{absolutelynopagebreak}

\begin{absolutelynopagebreak}
\setstretch{.7}
{\PaliGlossA{“Na kho no, brāhmaṇa, kappati jātarūparajataṃ paṭiggahetun”ti.}}\\
\begin{addmargin}[1em]{2em}
\setstretch{.5}
{\PaliGlossB{“It’s not proper for us to receive gold and money.”}}\\
\end{addmargin}
\end{absolutelynopagebreak}

\begin{absolutelynopagebreak}
\setstretch{.7}
{\PaliGlossA{“Sace taṃ bhoto udenassa na kappati vihāraṃ bhoto udenassa kārāpessāmī”ti.}}\\
\begin{addmargin}[1em]{2em}
\setstretch{.5}
{\PaliGlossB{“If that’s not proper, I will have a dwelling built for Master Udena.”}}\\
\end{addmargin}
\end{absolutelynopagebreak}

\begin{absolutelynopagebreak}
\setstretch{.7}
{\PaliGlossA{“Sace kho me tvaṃ, brāhmaṇa, vihāraṃ, kārāpetukāmo, pāṭaliputte saṅghassa upaṭṭhānasālaṃ kārāpehī”ti.}}\\
\begin{addmargin}[1em]{2em}
\setstretch{.5}
{\PaliGlossB{“If you want to build me a dwelling, then build an assembly hall for the Saṅgha at Pāṭaliputta.”}}\\
\end{addmargin}
\end{absolutelynopagebreak}

\begin{absolutelynopagebreak}
\setstretch{.7}
{\PaliGlossA{“Imināpāhaṃ bhoto udenassa bhiyyoso mattāya attamano abhiraddho yaṃ maṃ bhavaṃ udeno saṅghe dāne samādapeti.}}\\
\begin{addmargin}[1em]{2em}
\setstretch{.5}
{\PaliGlossB{“Now I’m even more delighted and satisfied with Master Udena, since he encourages me to give to the Saṅgha.}}\\
\end{addmargin}
\end{absolutelynopagebreak}

\begin{absolutelynopagebreak}
\setstretch{.7}
{\PaliGlossA{Esāhaṃ, bho udena, etissā ca niccabhikkhāya aparāya ca niccabhikkhāya pāṭaliputte saṅghassa upaṭṭhānasālaṃ kārāpessāmī”ti.}}\\
\begin{addmargin}[1em]{2em}
\setstretch{.5}
{\PaliGlossB{So with this allowance and another one I will have an assembly hall built for the Saṅgha at Pāṭaliputta.”}}\\
\end{addmargin}
\end{absolutelynopagebreak}

\begin{absolutelynopagebreak}
\setstretch{.7}
{\PaliGlossA{Atha kho ghoṭamukho brāhmaṇo etissā ca niccabhikkhāya aparāya ca niccabhikkhāya pāṭaliputte saṅghassa upaṭṭhānasālaṃ kārāpesi.}}\\
\begin{addmargin}[1em]{2em}
\setstretch{.5}
{\PaliGlossB{And so he had that hall built.}}\\
\end{addmargin}
\end{absolutelynopagebreak}

\begin{absolutelynopagebreak}
\setstretch{.7}
{\PaliGlossA{Sā etarahi “ghoṭamukhī”ti vuccatīti.}}\\
\begin{addmargin}[1em]{2em}
\setstretch{.5}
{\PaliGlossB{And these days it’s called the “Ghoṭamukhī”.}}\\
\end{addmargin}
\end{absolutelynopagebreak}

\begin{absolutelynopagebreak}
\setstretch{.7}
{\PaliGlossA{Ghoṭamukhasuttaṃ niṭṭhitaṃ catutthaṃ.}}\\
\begin{addmargin}[1em]{2em}
\setstretch{.5}
{\PaliGlossB{    -}}\\
\end{addmargin}
\end{absolutelynopagebreak}
