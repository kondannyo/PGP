
\vskip 0.05in
\begin{absolutelynopagebreak}
\setstretch{.7}
{\PaliGlossA{Majjhima Nikāya 91}}\\
\begin{addmargin}[1em]{2em}
\setstretch{.5}
{\PaliGlossB{Middle Discourses 91}}\\
\end{addmargin}
\end{absolutelynopagebreak}

\begin{absolutelynopagebreak}
\setstretch{.7}
{\PaliGlossA{Brahmāyusutta}}\\
\begin{addmargin}[1em]{2em}
\setstretch{.5}
{\PaliGlossB{With Brahmāyu}}\\
\end{addmargin}
\end{absolutelynopagebreak}

\vskip 0.05in
\begin{absolutelynopagebreak}
\setstretch{.7}
{\PaliGlossA{1. Evaṃ me sutaṃ—}}\\
\begin{addmargin}[1em]{2em}
\setstretch{.5}
{\PaliGlossB{So I have heard.}}\\
\end{addmargin}
\end{absolutelynopagebreak}

\begin{absolutelynopagebreak}
\setstretch{.7}
{\PaliGlossA{ekaṃ samayaṃ bhagavā videhesu cārikaṃ carati mahatā bhikkhusaṃghena saddhiṃ pañcamattehi bhikkhusatehi.}}\\
\begin{addmargin}[1em]{2em}
\setstretch{.5}
{\PaliGlossB{At one time the Buddha was wandering in the land of the Videhans together with a large Saṅgha of five hundred mendicants.}}\\
\end{addmargin}
\end{absolutelynopagebreak}

\vskip 0.05in
\begin{absolutelynopagebreak}
\setstretch{.7}
{\PaliGlossA{2. Tena kho pana samayena brahmāyu brāhmaṇo mithilāyaṃ paṭivasati jiṇṇo vuḍḍho mahallako addhagato vayoanuppatto, vīsavassasatiko jātiyā, tiṇṇaṃ vedānaṃ pāragū sanighaṇḍukeṭubhānaṃ sākkharappabhedānaṃ itihāsapañcamānaṃ, padako, veyyākaraṇo, lokāyatamahāpurisalakkhaṇesu anavayo.}}\\
\begin{addmargin}[1em]{2em}
\setstretch{.5}
{\PaliGlossB{Now at that time the brahmin Brahmāyu was residing in Mithilā. He was old, elderly, and senior, advanced in years, having reached the final stage of life; he was a hundred and twenty years old. He had mastered the three Vedas, together with their vocabularies, ritual, phonology and etymology, and the testament as fifth. He knew philology and grammar, and was well versed in cosmology and the marks of a great man.}}\\
\end{addmargin}
\end{absolutelynopagebreak}

\vskip 0.05in
\begin{absolutelynopagebreak}
\setstretch{.7}
{\PaliGlossA{3. Assosi kho brahmāyu brāhmaṇo:}}\\
\begin{addmargin}[1em]{2em}
\setstretch{.5}
{\PaliGlossB{He heard:}}\\
\end{addmargin}
\end{absolutelynopagebreak}

\begin{absolutelynopagebreak}
\setstretch{.7}
{\PaliGlossA{“samaṇo khalu bho, gotamo sakyaputto sakyakulā pabbajito videhesu cārikaṃ carati mahatā bhikkhusaṃghena saddhiṃ pañcamattehi bhikkhusatehi.}}\\
\begin{addmargin}[1em]{2em}
\setstretch{.5}
{\PaliGlossB{“It seems the ascetic Gotama—a Sakyan, gone forth from a Sakyan family—is wandering in the land of the Videhans, together with a large Saṅgha of around five hundred mendicants.}}\\
\end{addmargin}
\end{absolutelynopagebreak}

\begin{absolutelynopagebreak}
\setstretch{.7}
{\PaliGlossA{Taṃ kho pana bhavantaṃ gotamaṃ evaṃ kalyāṇo kittisaddo abbhuggato:}}\\
\begin{addmargin}[1em]{2em}
\setstretch{.5}
{\PaliGlossB{He has this good reputation:}}\\
\end{addmargin}
\end{absolutelynopagebreak}

\begin{absolutelynopagebreak}
\setstretch{.7}
{\PaliGlossA{‘itipi so bhagavā arahaṃ sammāsambuddho vijjācaraṇasampanno sugato lokavidū anuttaro purisadammasārathi satthā devamanussānaṃ buddho bhagavā’ti.}}\\
\begin{addmargin}[1em]{2em}
\setstretch{.5}
{\PaliGlossB{‘That Blessed One is perfected, a fully awakened Buddha, accomplished in knowledge and conduct, holy, knower of the world, supreme guide for those who wish to train, teacher of gods and humans, awakened, blessed.’}}\\
\end{addmargin}
\end{absolutelynopagebreak}

\begin{absolutelynopagebreak}
\setstretch{.7}
{\PaliGlossA{So imaṃ lokaṃ sadevakaṃ samārakaṃ sabrahmakaṃ sassamaṇabrāhmaṇiṃ pajaṃ sadevamanussaṃ sayaṃ abhiññā sacchikatvā pavedeti.}}\\
\begin{addmargin}[1em]{2em}
\setstretch{.5}
{\PaliGlossB{He has realized with his own insight this world—with its gods, Māras and Brahmās, this population with its ascetics and brahmins, gods and humans—and he makes it known to others.}}\\
\end{addmargin}
\end{absolutelynopagebreak}

\begin{absolutelynopagebreak}
\setstretch{.7}
{\PaliGlossA{So dhammaṃ deseti ādikalyāṇaṃ majjhekalyāṇaṃ pariyosānakalyāṇaṃ sātthaṃ sabyañjanaṃ, kevalaparipuṇṇaṃ parisuddhaṃ brahmacariyaṃ pakāseti.}}\\
\begin{addmargin}[1em]{2em}
\setstretch{.5}
{\PaliGlossB{He explains a teaching that’s good in the beginning, good in the middle, and good in the end, meaningful and well-phrased. And he reveals a spiritual practice that’s entirely full and pure.}}\\
\end{addmargin}
\end{absolutelynopagebreak}

\begin{absolutelynopagebreak}
\setstretch{.7}
{\PaliGlossA{Sādhu kho pana tathārūpānaṃ arahataṃ dassanaṃ hotī”ti.}}\\
\begin{addmargin}[1em]{2em}
\setstretch{.5}
{\PaliGlossB{It’s good to see such perfected ones.”}}\\
\end{addmargin}
\end{absolutelynopagebreak}

\vskip 0.05in
\begin{absolutelynopagebreak}
\setstretch{.7}
{\PaliGlossA{4. Tena kho pana samayena brahmāyussa brāhmaṇassa uttaro nāma māṇavo antevāsī hoti tiṇṇaṃ vedānaṃ pāragū sanighaṇḍukeṭubhānaṃ sākkharappabhedānaṃ itihāsapañcamānaṃ, padako, veyyākaraṇo, lokāyatamahāpurisalakkhaṇesu anavayo.}}\\
\begin{addmargin}[1em]{2em}
\setstretch{.5}
{\PaliGlossB{Brahmāyu told Uttara of the Buddha’s presence in the land of the Videhans, and added:}}\\
\end{addmargin}
\end{absolutelynopagebreak}

\begin{absolutelynopagebreak}
\setstretch{.7}
{\PaliGlossA{Atha kho brahmāyu brāhmaṇo uttaraṃ māṇavaṃ āmantesi:}}\\
\begin{addmargin}[1em]{2em}
\setstretch{.5}
{\PaliGlossB{    -}}\\
\end{addmargin}
\end{absolutelynopagebreak}

\begin{absolutelynopagebreak}
\setstretch{.7}
{\PaliGlossA{“ayaṃ, tāta uttara, samaṇo gotamo sakyaputto sakyakulā pabbajito videhesu cārikaṃ carati mahatā bhikkhusaṃghena saddhiṃ pañcamattehi bhikkhusatehi.}}\\
\begin{addmargin}[1em]{2em}
\setstretch{.5}
{\PaliGlossB{    -}}\\
\end{addmargin}
\end{absolutelynopagebreak}

\begin{absolutelynopagebreak}
\setstretch{.7}
{\PaliGlossA{Taṃ kho pana bhavantaṃ gotamaṃ evaṃ kalyāṇo kittisaddo abbhuggato:}}\\
\begin{addmargin}[1em]{2em}
\setstretch{.5}
{\PaliGlossB{    -}}\\
\end{addmargin}
\end{absolutelynopagebreak}

\begin{absolutelynopagebreak}
\setstretch{.7}
{\PaliGlossA{‘itipi so bhagavā arahaṃ sammāsambuddho … pe …}}\\
\begin{addmargin}[1em]{2em}
\setstretch{.5}
{\PaliGlossB{    -}}\\
\end{addmargin}
\end{absolutelynopagebreak}

\begin{absolutelynopagebreak}
\setstretch{.7}
{\PaliGlossA{sādhu kho pana tathārūpānaṃ arahataṃ dassanaṃ hotī’ti.}}\\
\begin{addmargin}[1em]{2em}
\setstretch{.5}
{\PaliGlossB{“Please, dear Uttara, go to the ascetic Gotama and find out whether or not he lives up to his reputation.}}\\
\end{addmargin}
\end{absolutelynopagebreak}

\begin{absolutelynopagebreak}
\setstretch{.7}
{\PaliGlossA{Ehi tvaṃ, tāta uttara, yena samaṇo gotamo tenupasaṅkama; upasaṅkamitvā samaṇaṃ gotamaṃ jānāhi, yadi vā taṃ bhavantaṃ gotamaṃ tathā santaṃyeva saddo abbhuggato, yadi vā no tathā;}}\\
\begin{addmargin}[1em]{2em}
\setstretch{.5}
{\PaliGlossB{    -}}\\
\end{addmargin}
\end{absolutelynopagebreak}

\begin{absolutelynopagebreak}
\setstretch{.7}
{\PaliGlossA{yadi vā so bhavaṃ gotamo tādiso, yadi vā na tādiso.}}\\
\begin{addmargin}[1em]{2em}
\setstretch{.5}
{\PaliGlossB{# NOTE: Following BB in accepting PTS reading tayā. Curiously enough, neither MS nor PTS acknowledges the variant.}}\\
\end{addmargin}
\end{absolutelynopagebreak}

\begin{absolutelynopagebreak}
\setstretch{.7}
{\PaliGlossA{Tathā mayaṃ taṃ bhavantaṃ gotamaṃ vedissāmā”ti.}}\\
\begin{addmargin}[1em]{2em}
\setstretch{.5}
{\PaliGlossB{    -}}\\
\end{addmargin}
\end{absolutelynopagebreak}

\begin{absolutelynopagebreak}
\setstretch{.7}
{\PaliGlossA{    -}}\\
\begin{addmargin}[1em]{2em}
\setstretch{.5}
{\PaliGlossB{“But sir, how shall I find out whether or not the ascetic Gotama lives up to his reputation?”}}\\
\end{addmargin}
\end{absolutelynopagebreak}

\begin{absolutelynopagebreak}
\setstretch{.7}
{\PaliGlossA{5. “Yathā kathaṃ panāhaṃ, bho, taṃ bhavantaṃ gotamaṃ jānissāmi yadi vā taṃ bhavantaṃ gotamaṃ tathā santaṃyeva saddo abbhuggato, yadi vā no tathā;}}\\
\begin{addmargin}[1em]{2em}
\setstretch{.5}
{\PaliGlossB{    -}}\\
\end{addmargin}
\end{absolutelynopagebreak}

\begin{absolutelynopagebreak}
\setstretch{.7}
{\PaliGlossA{yadi vā so bhavaṃ gotamo tādiso, yadi vā na tādiso”ti.}}\\
\begin{addmargin}[1em]{2em}
\setstretch{.5}
{\PaliGlossB{“Dear Uttara, the thirty-two marks of a great man have been handed down in our hymns. A great man who possesses these has only two possible destinies, no other.}}\\
\end{addmargin}
\end{absolutelynopagebreak}

\begin{absolutelynopagebreak}
\setstretch{.7}
{\PaliGlossA{“Āgatāni kho, tāta uttara, amhākaṃ mantesu dvattiṃsamahāpurisalakkhaṇāni, yehi samannāgatassa mahāpurisassa dveyeva gatiyo bhavanti anaññā.}}\\
\begin{addmargin}[1em]{2em}
\setstretch{.5}
{\PaliGlossB{If he stays at home he becomes a king, a wheel-turning monarch, a just and principled king. His dominion extends to all four sides, he achieves stability in the country, and he possesses the seven treasures.}}\\
\end{addmargin}
\end{absolutelynopagebreak}

\begin{absolutelynopagebreak}
\setstretch{.7}
{\PaliGlossA{Sace agāraṃ ajjhāvasati, rājā hoti cakkavattī dhammiko dhammarājā cāturanto vijitāvī janapadatthāvariyappatto sattaratanasamannāgato.}}\\
\begin{addmargin}[1em]{2em}
\setstretch{.5}
{\PaliGlossB{He has the following seven treasures:}}\\
\end{addmargin}
\end{absolutelynopagebreak}

\begin{absolutelynopagebreak}
\setstretch{.7}
{\PaliGlossA{Tassimāni satta ratanāni bhavanti, seyyathidaṃ—}}\\
\begin{addmargin}[1em]{2em}
\setstretch{.5}
{\PaliGlossB{the wheel, the elephant, the horse, the jewel, the woman, the treasurer, and the counselor as the seventh treasure.}}\\
\end{addmargin}
\end{absolutelynopagebreak}

\begin{absolutelynopagebreak}
\setstretch{.7}
{\PaliGlossA{cakkaratanaṃ, hatthiratanaṃ, assaratanaṃ, maṇiratanaṃ, itthiratanaṃ, gahapatiratanaṃ, pariṇāyakaratanameva sattamaṃ.}}\\
\begin{addmargin}[1em]{2em}
\setstretch{.5}
{\PaliGlossB{He has over a thousand sons who are valiant and heroic, crushing the armies of his enemies.}}\\
\end{addmargin}
\end{absolutelynopagebreak}

\begin{absolutelynopagebreak}
\setstretch{.7}
{\PaliGlossA{Parosahassaṃ kho panassa puttā bhavanti sūrā vīraṅgarūpā parasenappamaddanā.}}\\
\begin{addmargin}[1em]{2em}
\setstretch{.5}
{\PaliGlossB{After conquering this land girt by sea, he reigns by principle, without rod or sword.}}\\
\end{addmargin}
\end{absolutelynopagebreak}

\begin{absolutelynopagebreak}
\setstretch{.7}
{\PaliGlossA{So imaṃ pathaviṃ sāgarapariyantaṃ adaṇḍena asatthena dhammena abhivijiya ajjhāvasati.}}\\
\begin{addmargin}[1em]{2em}
\setstretch{.5}
{\PaliGlossB{But if he goes forth from the lay life to homelessness, he becomes a perfected one, a fully awakened Buddha, who draws back the veil from the world.}}\\
\end{addmargin}
\end{absolutelynopagebreak}

\begin{absolutelynopagebreak}
\setstretch{.7}
{\PaliGlossA{Sace kho pana agārasmā anagāriyaṃ pabbajati, arahaṃ hoti sammāsambuddho loke vivaṭṭacchado.}}\\
\begin{addmargin}[1em]{2em}
\setstretch{.5}
{\PaliGlossB{But, dear Uttara, I am the one who gives the hymns,}}\\
\end{addmargin}
\end{absolutelynopagebreak}

\begin{absolutelynopagebreak}
\setstretch{.7}
{\PaliGlossA{Ahaṃ kho pana, tāta uttara, mantānaṃ dātā;}}\\
\begin{addmargin}[1em]{2em}
\setstretch{.5}
{\PaliGlossB{and you are the one who receives them.”}}\\
\end{addmargin}
\end{absolutelynopagebreak}

\begin{absolutelynopagebreak}
\setstretch{.7}
{\PaliGlossA{tvaṃ mantānaṃ paṭiggahetā”ti.}}\\
\begin{addmargin}[1em]{2em}
\setstretch{.5}
{\PaliGlossB{    -}}\\
\end{addmargin}
\end{absolutelynopagebreak}

\begin{absolutelynopagebreak}
\setstretch{.7}
{\PaliGlossA{    -}}\\
\begin{addmargin}[1em]{2em}
\setstretch{.5}
{\PaliGlossB{“Yes, sir,” replied Uttara. He got up from his seat, bowed, and respectfully circled Brahmāyu before setting out for the land of the Videhans where the Buddha was wandering.}}\\
\end{addmargin}
\end{absolutelynopagebreak}

\begin{absolutelynopagebreak}
\setstretch{.7}
{\PaliGlossA{6. “Evaṃ, bho”ti kho uttaro māṇavo brahmāyussa brāhmaṇassa paṭissutvā uṭṭhāyāsanā brahmāyuṃ brāhmaṇaṃ abhivādetvā padakkhiṇaṃ katvā videhesu yena bhagavā tena cārikaṃ pakkāmi.}}\\
\begin{addmargin}[1em]{2em}
\setstretch{.5}
{\PaliGlossB{Traveling stage by stage, he came to the Buddha and exchanged greetings with him.}}\\
\end{addmargin}
\end{absolutelynopagebreak}

\begin{absolutelynopagebreak}
\setstretch{.7}
{\PaliGlossA{Anupubbena cārikaṃ caramāno yena bhagavā tenupasaṅkami; upasaṅkamitvā bhagavatā saddhiṃ sammodi.}}\\
\begin{addmargin}[1em]{2em}
\setstretch{.5}
{\PaliGlossB{When the greetings and polite conversation were over, he sat down to one side,}}\\
\end{addmargin}
\end{absolutelynopagebreak}

\begin{absolutelynopagebreak}
\setstretch{.7}
{\PaliGlossA{Sammodanīyaṃ kathaṃ sāraṇīyaṃ vītisāretvā ekamantaṃ nisīdi.}}\\
\begin{addmargin}[1em]{2em}
\setstretch{.5}
{\PaliGlossB{and scrutinized his body for the thirty-two marks of a great man.}}\\
\end{addmargin}
\end{absolutelynopagebreak}

\begin{absolutelynopagebreak}
\setstretch{.7}
{\PaliGlossA{Ekamantaṃ nisinno kho uttaro māṇavo bhagavato kāye dvattiṃsamahāpurisalakkhaṇāni samannesi.}}\\
\begin{addmargin}[1em]{2em}
\setstretch{.5}
{\PaliGlossB{He saw all of them except for two,}}\\
\end{addmargin}
\end{absolutelynopagebreak}

\begin{absolutelynopagebreak}
\setstretch{.7}
{\PaliGlossA{Addasā kho uttaro māṇavo bhagavato kāye dvattiṃsamahāpurisalakkhaṇāni, yebhuyyena thapetvā dve.}}\\
\begin{addmargin}[1em]{2em}
\setstretch{.5}
{\PaliGlossB{which he had doubts about:}}\\
\end{addmargin}
\end{absolutelynopagebreak}

\begin{absolutelynopagebreak}
\setstretch{.7}
{\PaliGlossA{Dvīsu mahāpurisalakkhaṇesu kaṅkhati vicikicchati nādhimuccati na sampasīdati—}}\\
\begin{addmargin}[1em]{2em}
\setstretch{.5}
{\PaliGlossB{# NOTE: https://discourse.suttacentral.net/t/the-politics-of-the-buddha-s-genitals/4876}}\\
\end{addmargin}
\end{absolutelynopagebreak}

\begin{absolutelynopagebreak}
\setstretch{.7}
{\PaliGlossA{kosohite ca vatthaguyhe pahūtajivhatāya ca.}}\\
\begin{addmargin}[1em]{2em}
\setstretch{.5}
{\PaliGlossB{Then it occurred to the Buddha,}}\\
\end{addmargin}
\end{absolutelynopagebreak}

\begin{absolutelynopagebreak}
\setstretch{.7}
{\PaliGlossA{Atha kho bhagavato etadahosi:}}\\
\begin{addmargin}[1em]{2em}
\setstretch{.5}
{\PaliGlossB{“This brahmin student Uttara sees all the marks except for two,}}\\
\end{addmargin}
\end{absolutelynopagebreak}

\begin{absolutelynopagebreak}
\setstretch{.7}
{\PaliGlossA{“passati kho me ayaṃ uttaro māṇavo dvattiṃsamahāpurisalakkhaṇāni, yebhuyyena thapetvā dve.}}\\
\begin{addmargin}[1em]{2em}
\setstretch{.5}
{\PaliGlossB{which he has doubts about:}}\\
\end{addmargin}
\end{absolutelynopagebreak}

\begin{absolutelynopagebreak}
\setstretch{.7}
{\PaliGlossA{Dvīsu mahāpurisalakkhaṇesu kaṅkhati vicikicchati nādhimuccati na sampasīdati—}}\\
\begin{addmargin}[1em]{2em}
\setstretch{.5}
{\PaliGlossB{whether the private parts are retracted, and the largeness of the tongue.”}}\\
\end{addmargin}
\end{absolutelynopagebreak}

\begin{absolutelynopagebreak}
\setstretch{.7}
{\PaliGlossA{kosohite ca vatthaguyhe pahūtajivhatāya cā”ti.}}\\
\begin{addmargin}[1em]{2em}
\setstretch{.5}
{\PaliGlossB{    -}}\\
\end{addmargin}
\end{absolutelynopagebreak}

\begin{absolutelynopagebreak}
\setstretch{.7}
{\PaliGlossA{    -}}\\
\begin{addmargin}[1em]{2em}
\setstretch{.5}
{\PaliGlossB{So the Buddha used his psychic power to will that Uttara would see his retracted private parts.}}\\
\end{addmargin}
\end{absolutelynopagebreak}

\begin{absolutelynopagebreak}
\setstretch{.7}
{\PaliGlossA{7. Atha kho bhagavā tathārūpaṃ iddhābhisaṅkhāraṃ abhisaṅkhāsi yathā addasa uttaro māṇavo bhagavato kosohitaṃ vatthaguyhaṃ.}}\\
\begin{addmargin}[1em]{2em}
\setstretch{.5}
{\PaliGlossB{And he stuck out his tongue and stroked back and forth on his ear holes and nostrils, and covered his entire forehead with his tongue.}}\\
\end{addmargin}
\end{absolutelynopagebreak}

\begin{absolutelynopagebreak}
\setstretch{.7}
{\PaliGlossA{Atha kho bhagavā jivhaṃ ninnāmetvā ubhopi kaṇṇasotāni anumasi paṭimasi; ubhopi nāsikasotāni anumasi paṭimasi; kevalampi nalāṭamaṇḍalaṃ jivhāya chādesi.}}\\
\begin{addmargin}[1em]{2em}
\setstretch{.5}
{\PaliGlossB{    -}}\\
\end{addmargin}
\end{absolutelynopagebreak}

\begin{absolutelynopagebreak}
\setstretch{.7}
{\PaliGlossA{    -}}\\
\begin{addmargin}[1em]{2em}
\setstretch{.5}
{\PaliGlossB{Then Uttara thought,}}\\
\end{addmargin}
\end{absolutelynopagebreak}

\begin{absolutelynopagebreak}
\setstretch{.7}
{\PaliGlossA{8. Atha kho uttarassa māṇavassa etadahosi:}}\\
\begin{addmargin}[1em]{2em}
\setstretch{.5}
{\PaliGlossB{“The ascetic Gotama possesses the thirty-two marks.}}\\
\end{addmargin}
\end{absolutelynopagebreak}

\begin{absolutelynopagebreak}
\setstretch{.7}
{\PaliGlossA{“samannāgato kho samaṇo gotamo dvattiṃsamahāpurisalakkhaṇehi.}}\\
\begin{addmargin}[1em]{2em}
\setstretch{.5}
{\PaliGlossB{Why don’t I follow him and observe his deportment?”}}\\
\end{addmargin}
\end{absolutelynopagebreak}

\begin{absolutelynopagebreak}
\setstretch{.7}
{\PaliGlossA{Yannūnāhaṃ samaṇaṃ gotamaṃ anubandheyyaṃ, iriyāpathamassa passeyyan”ti.}}\\
\begin{addmargin}[1em]{2em}
\setstretch{.5}
{\PaliGlossB{# NOTE: It seems like a relaxed pace, given that his teacher was 120!}}\\
\end{addmargin}
\end{absolutelynopagebreak}

\begin{absolutelynopagebreak}
\setstretch{.7}
{\PaliGlossA{Atha kho uttaro māṇavo sattamāsāni bhagavantaṃ anubandhi chāyāva anapāyinī.}}\\
\begin{addmargin}[1em]{2em}
\setstretch{.5}
{\PaliGlossB{When seven months had passed he set out wandering towards Mithilā.}}\\
\end{addmargin}
\end{absolutelynopagebreak}

\begin{absolutelynopagebreak}
\setstretch{.7}
{\PaliGlossA{Atha kho uttaro māṇavo sattannaṃ māsānaṃ accayena videhesu yena mithilā tena cārikaṃ pakkāmi.}}\\
\begin{addmargin}[1em]{2em}
\setstretch{.5}
{\PaliGlossB{There he approached the brahmin Brahmāyu, bowed, and sat down to one side. Brahmāyu said to him,}}\\
\end{addmargin}
\end{absolutelynopagebreak}

\begin{absolutelynopagebreak}
\setstretch{.7}
{\PaliGlossA{Anupubbena cārikaṃ caramāno yena mithilā yena brahmāyu brāhmaṇo tenupasaṅkami; upasaṅkamitvā brahmāyuṃ brāhmaṇaṃ abhivādetvā ekamantaṃ nisīdi. Ekamantaṃ nisinnaṃ kho uttaraṃ māṇavaṃ brahmāyu brāhmaṇo etadavoca:}}\\
\begin{addmargin}[1em]{2em}
\setstretch{.5}
{\PaliGlossB{“Well, dear Uttara, does Master Gotama live up to his reputation or not?”}}\\
\end{addmargin}
\end{absolutelynopagebreak}

\begin{absolutelynopagebreak}
\setstretch{.7}
{\PaliGlossA{“kacci, tāta uttara, taṃ bhavantaṃ gotamaṃ tathā santaṃyeva saddo abbhuggato, no aññathā?}}\\
\begin{addmargin}[1em]{2em}
\setstretch{.5}
{\PaliGlossB{    -}}\\
\end{addmargin}
\end{absolutelynopagebreak}

\begin{absolutelynopagebreak}
\setstretch{.7}
{\PaliGlossA{Kacci pana so bhavaṃ gotamo tādiso, no aññādiso”ti?}}\\
\begin{addmargin}[1em]{2em}
\setstretch{.5}
{\PaliGlossB{“He does, sir.}}\\
\end{addmargin}
\end{absolutelynopagebreak}

\begin{absolutelynopagebreak}
\setstretch{.7}
{\PaliGlossA{“Tathā santaṃyeva, bho, taṃ bhavantaṃ gotamaṃ saddo abbhuggato, no aññathā;}}\\
\begin{addmargin}[1em]{2em}
\setstretch{.5}
{\PaliGlossB{    -}}\\
\end{addmargin}
\end{absolutelynopagebreak}

\begin{absolutelynopagebreak}
\setstretch{.7}
{\PaliGlossA{tādisova so bhavaṃ gotamo, no aññādiso.}}\\
\begin{addmargin}[1em]{2em}
\setstretch{.5}
{\PaliGlossB{Master Gotama possesses the thirty-two marks.}}\\
\end{addmargin}
\end{absolutelynopagebreak}

\begin{absolutelynopagebreak}
\setstretch{.7}
{\PaliGlossA{Samannāgato ca so bhavaṃ gotamo dvattiṃsamahāpurisalakkhaṇehi.}}\\
\begin{addmargin}[1em]{2em}
\setstretch{.5}
{\PaliGlossB{    -}}\\
\end{addmargin}
\end{absolutelynopagebreak}

\begin{absolutelynopagebreak}
\setstretch{.7}
{\PaliGlossA{    -}}\\
\begin{addmargin}[1em]{2em}
\setstretch{.5}
{\PaliGlossB{He has well-planted feet.}}\\
\end{addmargin}
\end{absolutelynopagebreak}

\begin{absolutelynopagebreak}
\setstretch{.7}
{\PaliGlossA{9. Suppatiṭṭhitapādo kho pana bho bhavaṃ gotamo;}}\\
\begin{addmargin}[1em]{2em}
\setstretch{.5}
{\PaliGlossB{    -}}\\
\end{addmargin}
\end{absolutelynopagebreak}

\begin{absolutelynopagebreak}
\setstretch{.7}
{\PaliGlossA{idampi tassa bhoto gotamassa mahāpurisassa mahāpurisalakkhaṇaṃ bhavati. (1)}}\\
\begin{addmargin}[1em]{2em}
\setstretch{.5}
{\PaliGlossB{On the soles of his feet there are thousand-spoked wheels, with rims and hubs, complete in every detail.}}\\
\end{addmargin}
\end{absolutelynopagebreak}

\begin{absolutelynopagebreak}
\setstretch{.7}
{\PaliGlossA{Heṭṭhā kho pana tassa bhoto gotamassa pādatalesu cakkāni jātāni sahassārāni sanemikāni sanābhikāni sabbākāraparipūrāni … (2)}}\\
\begin{addmargin}[1em]{2em}
\setstretch{.5}
{\PaliGlossB{He has projecting heels.}}\\
\end{addmargin}
\end{absolutelynopagebreak}

\begin{absolutelynopagebreak}
\setstretch{.7}
{\PaliGlossA{Āyatapaṇhi kho pana so bhavaṃ gotamo … (3)}}\\
\begin{addmargin}[1em]{2em}
\setstretch{.5}
{\PaliGlossB{He has long fingers.}}\\
\end{addmargin}
\end{absolutelynopagebreak}

\begin{absolutelynopagebreak}
\setstretch{.7}
{\PaliGlossA{Dīghaṅguli kho pana so bhavaṃ gotamo … (4)}}\\
\begin{addmargin}[1em]{2em}
\setstretch{.5}
{\PaliGlossB{His hands and feet are tender.}}\\
\end{addmargin}
\end{absolutelynopagebreak}

\begin{absolutelynopagebreak}
\setstretch{.7}
{\PaliGlossA{Mudutalunahatthapādo kho pana so bhavaṃ gotamo … (5)}}\\
\begin{addmargin}[1em]{2em}
\setstretch{.5}
{\PaliGlossB{His hands and feet cling gracefully.}}\\
\end{addmargin}
\end{absolutelynopagebreak}

\begin{absolutelynopagebreak}
\setstretch{.7}
{\PaliGlossA{Jālahatthapādo kho pana so bhavaṃ gotamo … (6)}}\\
\begin{addmargin}[1em]{2em}
\setstretch{.5}
{\PaliGlossB{His feet are arched.}}\\
\end{addmargin}
\end{absolutelynopagebreak}

\begin{absolutelynopagebreak}
\setstretch{.7}
{\PaliGlossA{Ussaṅkhapādo kho pana so bhavaṃ gotamo … (7)}}\\
\begin{addmargin}[1em]{2em}
\setstretch{.5}
{\PaliGlossB{His calves are like those of an antelope.}}\\
\end{addmargin}
\end{absolutelynopagebreak}

\begin{absolutelynopagebreak}
\setstretch{.7}
{\PaliGlossA{Eṇijaṅgho kho pana so bhavaṃ gotamo … (8)}}\\
\begin{addmargin}[1em]{2em}
\setstretch{.5}
{\PaliGlossB{When standing upright and not bending over, the palms of both hands touch the knees.}}\\
\end{addmargin}
\end{absolutelynopagebreak}

\begin{absolutelynopagebreak}
\setstretch{.7}
{\PaliGlossA{Ṭhitako kho pana so bhavaṃ gotamo anonamanto ubhohi pāṇitalehi jaṇṇukāni parimasati parimajjati … (9)}}\\
\begin{addmargin}[1em]{2em}
\setstretch{.5}
{\PaliGlossB{His private parts are retracted.}}\\
\end{addmargin}
\end{absolutelynopagebreak}

\begin{absolutelynopagebreak}
\setstretch{.7}
{\PaliGlossA{Kosohitavatthaguyho kho pana so bhavaṃ gotamo … (10)}}\\
\begin{addmargin}[1em]{2em}
\setstretch{.5}
{\PaliGlossB{He is gold colored; his skin has a golden sheen.}}\\
\end{addmargin}
\end{absolutelynopagebreak}

\begin{absolutelynopagebreak}
\setstretch{.7}
{\PaliGlossA{Suvaṇṇavaṇṇo kho pana so bhavaṃ gotamo kañcanasannibhattaco … (11)}}\\
\begin{addmargin}[1em]{2em}
\setstretch{.5}
{\PaliGlossB{He has delicate skin, so delicate that dust and dirt don’t stick to his body.}}\\
\end{addmargin}
\end{absolutelynopagebreak}

\begin{absolutelynopagebreak}
\setstretch{.7}
{\PaliGlossA{Sukhumacchavi kho pana so bhavaṃ gotamo. Sukhumattā chaviyā rajojallaṃ kāye na upalimpati … (12)}}\\
\begin{addmargin}[1em]{2em}
\setstretch{.5}
{\PaliGlossB{His hairs grow one per pore.}}\\
\end{addmargin}
\end{absolutelynopagebreak}

\begin{absolutelynopagebreak}
\setstretch{.7}
{\PaliGlossA{Ekekalomo kho pana so bhavaṃ gotamo; ekekāni lomāni lomakūpesu jātāni … (13)}}\\
\begin{addmargin}[1em]{2em}
\setstretch{.5}
{\PaliGlossB{His hairs stand up; they’re blue-black and curl clockwise.}}\\
\end{addmargin}
\end{absolutelynopagebreak}

\begin{absolutelynopagebreak}
\setstretch{.7}
{\PaliGlossA{Uddhaggalomo kho pana so bhavaṃ gotamo; uddhaggāni lomāni jātāni nīlāni añjanavaṇṇāni kuṇḍalāvaṭṭāni dakkhiṇāvaṭṭakajātāni … (14)}}\\
\begin{addmargin}[1em]{2em}
\setstretch{.5}
{\PaliGlossB{His body is as straight as Brahmā’s.}}\\
\end{addmargin}
\end{absolutelynopagebreak}

\begin{absolutelynopagebreak}
\setstretch{.7}
{\PaliGlossA{Brahmujugatto kho pana so bhavaṃ gotamo … (15)}}\\
\begin{addmargin}[1em]{2em}
\setstretch{.5}
{\PaliGlossB{He has bulging muscles in seven places.}}\\
\end{addmargin}
\end{absolutelynopagebreak}

\begin{absolutelynopagebreak}
\setstretch{.7}
{\PaliGlossA{Sattussado kho pana so bhavaṃ gotamo … (16)}}\\
\begin{addmargin}[1em]{2em}
\setstretch{.5}
{\PaliGlossB{His chest is like that of a lion.}}\\
\end{addmargin}
\end{absolutelynopagebreak}

\begin{absolutelynopagebreak}
\setstretch{.7}
{\PaliGlossA{Sīhapubbaddhakāyo kho pana so bhavaṃ gotamo … (17)}}\\
\begin{addmargin}[1em]{2em}
\setstretch{.5}
{\PaliGlossB{The gap between the shoulder-blades is filled in.}}\\
\end{addmargin}
\end{absolutelynopagebreak}

\begin{absolutelynopagebreak}
\setstretch{.7}
{\PaliGlossA{Citantaraṃso kho pana so bhavaṃ gotamo … (18)}}\\
\begin{addmargin}[1em]{2em}
\setstretch{.5}
{\PaliGlossB{He has the proportional circumference of a banyan tree: the span of his arms equals the height of his body.}}\\
\end{addmargin}
\end{absolutelynopagebreak}

\begin{absolutelynopagebreak}
\setstretch{.7}
{\PaliGlossA{Nigrodhaparimaṇḍalo kho pana so bhavaṃ gotamo; yāvatakvassa kāyo tāvatakvassa byāmo, yāvatakvassa byāmo tāvatakvassa kāyo … (19)}}\\
\begin{addmargin}[1em]{2em}
\setstretch{.5}
{\PaliGlossB{His torso is cylindrical.}}\\
\end{addmargin}
\end{absolutelynopagebreak}

\begin{absolutelynopagebreak}
\setstretch{.7}
{\PaliGlossA{Samavaṭṭakkhandho kho pana so bhavaṃ gotamo … (20)}}\\
\begin{addmargin}[1em]{2em}
\setstretch{.5}
{\PaliGlossB{He has an excellent sense of taste.}}\\
\end{addmargin}
\end{absolutelynopagebreak}

\begin{absolutelynopagebreak}
\setstretch{.7}
{\PaliGlossA{Rasaggasaggī kho pana so bhavaṃ gotamo … (21)}}\\
\begin{addmargin}[1em]{2em}
\setstretch{.5}
{\PaliGlossB{His jaw is like that of a lion.}}\\
\end{addmargin}
\end{absolutelynopagebreak}

\begin{absolutelynopagebreak}
\setstretch{.7}
{\PaliGlossA{Sīhahanu kho pana so bhavaṃ gotamo … (22)}}\\
\begin{addmargin}[1em]{2em}
\setstretch{.5}
{\PaliGlossB{He has forty teeth.}}\\
\end{addmargin}
\end{absolutelynopagebreak}

\begin{absolutelynopagebreak}
\setstretch{.7}
{\PaliGlossA{Cattālīsadanto kho pana so bhavaṃ gotamo … (23)}}\\
\begin{addmargin}[1em]{2em}
\setstretch{.5}
{\PaliGlossB{His teeth are even.}}\\
\end{addmargin}
\end{absolutelynopagebreak}

\begin{absolutelynopagebreak}
\setstretch{.7}
{\PaliGlossA{Samadanto kho pana so bhavaṃ gotamo … (24)}}\\
\begin{addmargin}[1em]{2em}
\setstretch{.5}
{\PaliGlossB{His teeth have no gaps.}}\\
\end{addmargin}
\end{absolutelynopagebreak}

\begin{absolutelynopagebreak}
\setstretch{.7}
{\PaliGlossA{Aviraḷadanto kho pana so bhavaṃ gotamo … (25)}}\\
\begin{addmargin}[1em]{2em}
\setstretch{.5}
{\PaliGlossB{His teeth are perfectly white.}}\\
\end{addmargin}
\end{absolutelynopagebreak}

\begin{absolutelynopagebreak}
\setstretch{.7}
{\PaliGlossA{Susukkadāṭho kho pana so bhavaṃ gotamo … (26)}}\\
\begin{addmargin}[1em]{2em}
\setstretch{.5}
{\PaliGlossB{He has a large tongue.}}\\
\end{addmargin}
\end{absolutelynopagebreak}

\begin{absolutelynopagebreak}
\setstretch{.7}
{\PaliGlossA{Pahūtajivho kho pana so bhavaṃ gotamo … (27)}}\\
\begin{addmargin}[1em]{2em}
\setstretch{.5}
{\PaliGlossB{He has the voice of Brahmā, like a cuckoo’s call.}}\\
\end{addmargin}
\end{absolutelynopagebreak}

\begin{absolutelynopagebreak}
\setstretch{.7}
{\PaliGlossA{Brahmassaro kho pana so bhavaṃ gotamo karavikabhāṇī … (28)}}\\
\begin{addmargin}[1em]{2em}
\setstretch{.5}
{\PaliGlossB{His eyes are deep blue.}}\\
\end{addmargin}
\end{absolutelynopagebreak}

\begin{absolutelynopagebreak}
\setstretch{.7}
{\PaliGlossA{Abhinīlanetto kho pana so bhavaṃ gotamo … (29)}}\\
\begin{addmargin}[1em]{2em}
\setstretch{.5}
{\PaliGlossB{He has eyelashes like a cow’s.}}\\
\end{addmargin}
\end{absolutelynopagebreak}

\begin{absolutelynopagebreak}
\setstretch{.7}
{\PaliGlossA{Gopakhumo kho pana so bhavaṃ gotamo … (30)}}\\
\begin{addmargin}[1em]{2em}
\setstretch{.5}
{\PaliGlossB{Between his eyebrows there grows a tuft, soft and white like cotton-wool.}}\\
\end{addmargin}
\end{absolutelynopagebreak}

\begin{absolutelynopagebreak}
\setstretch{.7}
{\PaliGlossA{Uṇṇā kho panassa bhoto gotamassa bhamukantare jātā odātā mudutūlasannibhā … (31)}}\\
\begin{addmargin}[1em]{2em}
\setstretch{.5}
{\PaliGlossB{His head is shaped like a turban.}}\\
\end{addmargin}
\end{absolutelynopagebreak}

\begin{absolutelynopagebreak}
\setstretch{.7}
{\PaliGlossA{Uṇhīsasīso kho pana so bhavaṃ gotamo; idampi tassa bhoto gotamassa mahāpurisassa mahāpurisalakkhaṇaṃ bhavati. (32)}}\\
\begin{addmargin}[1em]{2em}
\setstretch{.5}
{\PaliGlossB{These are the thirty-two marks of a great man possessed by Master Gotama.}}\\
\end{addmargin}
\end{absolutelynopagebreak}

\begin{absolutelynopagebreak}
\setstretch{.7}
{\PaliGlossA{Imehi kho, bho, so bhavaṃ gotamo dvattiṃsamahāpurisalakkhaṇehi samannāgato.}}\\
\begin{addmargin}[1em]{2em}
\setstretch{.5}
{\PaliGlossB{    -}}\\
\end{addmargin}
\end{absolutelynopagebreak}

\begin{absolutelynopagebreak}
\setstretch{.7}
{\PaliGlossA{    -}}\\
\begin{addmargin}[1em]{2em}
\setstretch{.5}
{\PaliGlossB{When he’s walking he takes the first step with the right foot.}}\\
\end{addmargin}
\end{absolutelynopagebreak}

\begin{absolutelynopagebreak}
\setstretch{.7}
{\PaliGlossA{10. Gacchanto kho pana so bhavaṃ gotamo dakkhiṇeneva pādena paṭhamaṃ pakkamati.}}\\
\begin{addmargin}[1em]{2em}
\setstretch{.5}
{\PaliGlossB{He doesn’t lift his foot too far or place it too near.}}\\
\end{addmargin}
\end{absolutelynopagebreak}

\begin{absolutelynopagebreak}
\setstretch{.7}
{\PaliGlossA{So nātidūre pādaṃ uddharati, nāccāsanne pādaṃ nikkhipati;}}\\
\begin{addmargin}[1em]{2em}
\setstretch{.5}
{\PaliGlossB{He doesn’t walk too slow or too fast.}}\\
\end{addmargin}
\end{absolutelynopagebreak}

\begin{absolutelynopagebreak}
\setstretch{.7}
{\PaliGlossA{so nātisīghaṃ gacchati, nātisaṇikaṃ gacchati;}}\\
\begin{addmargin}[1em]{2em}
\setstretch{.5}
{\PaliGlossB{# NOTE: This is apparently the only occurrence of adduva. Comm says “knees”, and the various dicts and translations follow this. But who walks knocking their knees? However, the Chinese text at T 76 clearly mentions knee: 行時踝膝不相切摩. The other parallel, at MA 161 does not seem to mention the knees: 行時兩踝終不相掁}}\\
\end{addmargin}
\end{absolutelynopagebreak}

\begin{absolutelynopagebreak}
\setstretch{.7}
{\PaliGlossA{na ca adduvena adduvaṃ saṅghaṭṭento gacchati, na ca gopphakena gopphakaṃ saṅghaṭṭento gacchati.}}\\
\begin{addmargin}[1em]{2em}
\setstretch{.5}
{\PaliGlossB{When he’s walking he keeps his thighs neither too straight nor too bent, neither too tight nor too loose.}}\\
\end{addmargin}
\end{absolutelynopagebreak}

\begin{absolutelynopagebreak}
\setstretch{.7}
{\PaliGlossA{So gacchanto na satthiṃ unnāmeti, na satthiṃ onāmeti; na satthiṃ sannāmeti, na satthiṃ vināmeti.}}\\
\begin{addmargin}[1em]{2em}
\setstretch{.5}
{\PaliGlossB{When he walks, only the lower half of his body moves, and he walks effortlessly.}}\\
\end{addmargin}
\end{absolutelynopagebreak}

\begin{absolutelynopagebreak}
\setstretch{.7}
{\PaliGlossA{Gacchato kho pana tassa bhoto gotamassa adharakāyova iñjati, na ca kāyabalena gacchati.}}\\
\begin{addmargin}[1em]{2em}
\setstretch{.5}
{\PaliGlossB{When he turns to look he does so with the whole body.}}\\
\end{addmargin}
\end{absolutelynopagebreak}

\begin{absolutelynopagebreak}
\setstretch{.7}
{\PaliGlossA{Apalokento kho pana so bhavaṃ gotamo sabbakāyeneva apaloketi;}}\\
\begin{addmargin}[1em]{2em}
\setstretch{.5}
{\PaliGlossB{He doesn’t look directly up or down.}}\\
\end{addmargin}
\end{absolutelynopagebreak}

\begin{absolutelynopagebreak}
\setstretch{.7}
{\PaliGlossA{so na uddhaṃ ulloketi, na adho oloketi;}}\\
\begin{addmargin}[1em]{2em}
\setstretch{.5}
{\PaliGlossB{He doesn’t look all around while walking, but focuses a plough’s length in front. Beyond that he has unhindered knowledge and vision.}}\\
\end{addmargin}
\end{absolutelynopagebreak}

\begin{absolutelynopagebreak}
\setstretch{.7}
{\PaliGlossA{na ca vipekkhamāno gacchati, yugamattañca pekkhati; tato cassa uttari anāvaṭaṃ ñāṇadassanaṃ bhavati.}}\\
\begin{addmargin}[1em]{2em}
\setstretch{.5}
{\PaliGlossB{# NOTE: “indoors” is incorrect.}}\\
\end{addmargin}
\end{absolutelynopagebreak}

\begin{absolutelynopagebreak}
\setstretch{.7}
{\PaliGlossA{    -}}\\
\begin{addmargin}[1em]{2em}
\setstretch{.5}
{\PaliGlossB{When entering an inhabited area he keeps his body neither too straight nor too bent, neither too tight nor too loose.}}\\
\end{addmargin}
\end{absolutelynopagebreak}

\begin{absolutelynopagebreak}
\setstretch{.7}
{\PaliGlossA{11. So antaragharaṃ pavisanto na kāyaṃ unnāmeti, na kāyaṃ onāmeti; na kāyaṃ sannāmeti, na kāyaṃ vināmeti.}}\\
\begin{addmargin}[1em]{2em}
\setstretch{.5}
{\PaliGlossB{    -}}\\
\end{addmargin}
\end{absolutelynopagebreak}

\begin{absolutelynopagebreak}
\setstretch{.7}
{\PaliGlossA{    -}}\\
\begin{addmargin}[1em]{2em}
\setstretch{.5}
{\PaliGlossB{He turns around neither too far nor too close to the seat. He doesn’t lean on his hand when sitting down. And he doesn’t just plonk his body down on the seat.}}\\
\end{addmargin}
\end{absolutelynopagebreak}

\begin{absolutelynopagebreak}
\setstretch{.7}
{\PaliGlossA{12. So nātidūre nāccāsanne āsanassa parivattati, na ca pāṇinā ālambitvā āsane nisīdati, na ca āsanasmiṃ kāyaṃ pakkhipati.}}\\
\begin{addmargin}[1em]{2em}
\setstretch{.5}
{\PaliGlossB{# NOTE: Note, again antaraghara does not mean indoors. Such contexts as these might be read in that way, as typically we think of sitting down indoors. However in the suttas we find cases of, for example Ratthapala, who sits down in an inhabited aread, but outside by a wall. In some places, eg northern Thailand, it is normal for monks to sit in a little open sala in or near the village where they receive offerings.}}\\
\end{addmargin}
\end{absolutelynopagebreak}

\begin{absolutelynopagebreak}
\setstretch{.7}
{\PaliGlossA{So antaraghare nisinno samāno na hatthakukkuccaṃ āpajjati, na pādakukkuccaṃ āpajjati;}}\\
\begin{addmargin}[1em]{2em}
\setstretch{.5}
{\PaliGlossB{He doesn’t sit with his knees or ankles crossed.}}\\
\end{addmargin}
\end{absolutelynopagebreak}

\begin{absolutelynopagebreak}
\setstretch{.7}
{\PaliGlossA{na adduvena adduvaṃ āropetvā nisīdati; na ca gopphakena gopphakaṃ āropetvā nisīdati;}}\\
\begin{addmargin}[1em]{2em}
\setstretch{.5}
{\PaliGlossB{He doesn’t sit with his hand holding his chin.}}\\
\end{addmargin}
\end{absolutelynopagebreak}

\begin{absolutelynopagebreak}
\setstretch{.7}
{\PaliGlossA{na ca pāṇinā hanukaṃ upadahitvā nisīdati.}}\\
\begin{addmargin}[1em]{2em}
\setstretch{.5}
{\PaliGlossB{When sitting in inhabited areas he doesn’t cower or shake or tremble or get anxious, and so he is not nervous at all.}}\\
\end{addmargin}
\end{absolutelynopagebreak}

\begin{absolutelynopagebreak}
\setstretch{.7}
{\PaliGlossA{So antaraghare nisinno samāno na chambhati na kampati na vedhati na paritassati. So achambhī akampī avedhī aparitassī vigatalomahaṃso.}}\\
\begin{addmargin}[1em]{2em}
\setstretch{.5}
{\PaliGlossB{When sitting in inhabited areas he still practices seclusion.}}\\
\end{addmargin}
\end{absolutelynopagebreak}

\begin{absolutelynopagebreak}
\setstretch{.7}
{\PaliGlossA{Vivekavatto ca so bhavaṃ gotamo antaraghare nisinno hoti.}}\\
\begin{addmargin}[1em]{2em}
\setstretch{.5}
{\PaliGlossB{    -}}\\
\end{addmargin}
\end{absolutelynopagebreak}

\begin{absolutelynopagebreak}
\setstretch{.7}
{\PaliGlossA{    -}}\\
\begin{addmargin}[1em]{2em}
\setstretch{.5}
{\PaliGlossB{When receiving water for rinsing the bowl, he holds the bowl neither too straight nor too bent, neither too tight nor too loose.}}\\
\end{addmargin}
\end{absolutelynopagebreak}

\begin{absolutelynopagebreak}
\setstretch{.7}
{\PaliGlossA{13. So pattodakaṃ paṭiggaṇhanto na pattaṃ unnāmeti, na pattaṃ onāmeti; na pattaṃ sannāmeti, na pattaṃ vināmeti.}}\\
\begin{addmargin}[1em]{2em}
\setstretch{.5}
{\PaliGlossB{    -}}\\
\end{addmargin}
\end{absolutelynopagebreak}

\begin{absolutelynopagebreak}
\setstretch{.7}
{\PaliGlossA{    -}}\\
\begin{addmargin}[1em]{2em}
\setstretch{.5}
{\PaliGlossB{He receives neither too little nor too much water.}}\\
\end{addmargin}
\end{absolutelynopagebreak}

\begin{absolutelynopagebreak}
\setstretch{.7}
{\PaliGlossA{14. So pattodakaṃ paṭiggaṇhāti nātithokaṃ nātibahuṃ.}}\\
\begin{addmargin}[1em]{2em}
\setstretch{.5}
{\PaliGlossB{He rinses the bowl without making a sloshing noise, or spinning it around. He doesn’t put the bowl on the ground to rinse his hands; his hands and bowl are rinsed at the same time.}}\\
\end{addmargin}
\end{absolutelynopagebreak}

\begin{absolutelynopagebreak}
\setstretch{.7}
{\PaliGlossA{So na khulukhulukārakaṃ pattaṃ dhovati, na samparivattakaṃ pattaṃ dhovati, na pattaṃ bhūmiyaṃ nikkhipitvā hatthe dhovati; hatthesu dhotesu patto dhoto hoti, patte dhote hatthā dhotā honti.}}\\
\begin{addmargin}[1em]{2em}
\setstretch{.5}
{\PaliGlossB{He doesn’t throw the bowl rinsing water away too far or too near, or splash it about.}}\\
\end{addmargin}
\end{absolutelynopagebreak}

\begin{absolutelynopagebreak}
\setstretch{.7}
{\PaliGlossA{So pattodakaṃ chaḍḍeti nātidūre nāccāsanne, na ca vicchaḍḍayamāno.}}\\
\begin{addmargin}[1em]{2em}
\setstretch{.5}
{\PaliGlossB{When receiving rice, he holds the bowl neither too straight nor too bent, neither too close nor too loose.}}\\
\end{addmargin}
\end{absolutelynopagebreak}

\begin{absolutelynopagebreak}
\setstretch{.7}
{\PaliGlossA{So odanaṃ paṭiggaṇhanto na pattaṃ unnāmeti, na pattaṃ onāmeti; na pattaṃ sannāmeti, na pattaṃ vināmeti.}}\\
\begin{addmargin}[1em]{2em}
\setstretch{.5}
{\PaliGlossB{He receives neither too little nor too much rice.}}\\
\end{addmargin}
\end{absolutelynopagebreak}

\begin{absolutelynopagebreak}
\setstretch{.7}
{\PaliGlossA{So odanaṃ paṭiggaṇhāti nātithokaṃ nātibahuṃ.}}\\
\begin{addmargin}[1em]{2em}
\setstretch{.5}
{\PaliGlossB{He eats sauce in a moderate proportion, and doesn’t spend too much time saucing his portions.}}\\
\end{addmargin}
\end{absolutelynopagebreak}

\begin{absolutelynopagebreak}
\setstretch{.7}
{\PaliGlossA{Byañjanaṃ kho pana bhavaṃ gotamo byañjanamattāya āhāreti, na ca byañjanena ālopaṃ atināmeti.}}\\
\begin{addmargin}[1em]{2em}
\setstretch{.5}
{\PaliGlossB{He chews over each portion two or three times before swallowing.}}\\
\end{addmargin}
\end{absolutelynopagebreak}

\begin{absolutelynopagebreak}
\setstretch{.7}
{\PaliGlossA{Dvattikkhattuṃ kho bhavaṃ gotamo mukhe ālopaṃ samparivattetvā ajjhoharati;}}\\
\begin{addmargin}[1em]{2em}
\setstretch{.5}
{\PaliGlossB{But no grain of rice enters his body unchewed, and none remain in his mouth.}}\\
\end{addmargin}
\end{absolutelynopagebreak}

\begin{absolutelynopagebreak}
\setstretch{.7}
{\PaliGlossA{na cassa kāci odanamiñjā asambhinnā kāyaṃ pavisati, na cassa kāci odanamiñjā mukhe avasiṭṭhā hoti;}}\\
\begin{addmargin}[1em]{2em}
\setstretch{.5}
{\PaliGlossB{Only then does he raise another portion to his lips.}}\\
\end{addmargin}
\end{absolutelynopagebreak}

\begin{absolutelynopagebreak}
\setstretch{.7}
{\PaliGlossA{athāparaṃ ālopaṃ upanāmeti.}}\\
\begin{addmargin}[1em]{2em}
\setstretch{.5}
{\PaliGlossB{He eats experiencing the taste, but without experiencing greed for the taste.}}\\
\end{addmargin}
\end{absolutelynopagebreak}

\begin{absolutelynopagebreak}
\setstretch{.7}
{\PaliGlossA{Rasapaṭisaṃvedī kho pana so bhavaṃ gotamo āhāraṃ āhāreti, no ca rasarāgapaṭisaṃvedī.}}\\
\begin{addmargin}[1em]{2em}
\setstretch{.5}
{\PaliGlossB{He eats food thinking of eight reasons:}}\\
\end{addmargin}
\end{absolutelynopagebreak}

\begin{absolutelynopagebreak}
\setstretch{.7}
{\PaliGlossA{Aṭṭhaṅgasamannāgataṃ kho pana so bhavaṃ gotamo āhāraṃ āhāreti—}}\\
\begin{addmargin}[1em]{2em}
\setstretch{.5}
{\PaliGlossB{‘Not for fun, indulgence, adornment, or decoration, but only to sustain this body, to avoid harm, and to support spiritual practice. In this way, I shall put an end to old discomfort and not give rise to new discomfort, and I will live blamelessly and at ease.’}}\\
\end{addmargin}
\end{absolutelynopagebreak}

\begin{absolutelynopagebreak}
\setstretch{.7}
{\PaliGlossA{neva davāya, na madāya na maṇḍanāya na vibhūsanāya, yāvadeva imassa kāyassa ṭhitiyā yāpanāya, vihiṃsūparatiyā brahmacariyānuggahāya: ‘iti purāṇañca vedanaṃ paṭihaṅkhāmi navañca vedanaṃ na uppādessāmi, yātrā ca me bhavissati anavajjatā ca phāsuvihāro cā’ti.}}\\
\begin{addmargin}[1em]{2em}
\setstretch{.5}
{\PaliGlossB{    -}}\\
\end{addmargin}
\end{absolutelynopagebreak}

\begin{absolutelynopagebreak}
\setstretch{.7}
{\PaliGlossA{    -}}\\
\begin{addmargin}[1em]{2em}
\setstretch{.5}
{\PaliGlossB{After eating, when receiving water for washing the bowl, he holds the bowl neither too straight nor too bent, neither too tight nor too loose.}}\\
\end{addmargin}
\end{absolutelynopagebreak}

\begin{absolutelynopagebreak}
\setstretch{.7}
{\PaliGlossA{15. So bhuttāvī pattodakaṃ paṭiggaṇhanto na pattaṃ unnāmeti, na pattaṃ onāmeti; na pattaṃ sannāmeti, na pattaṃ vināmeti.}}\\
\begin{addmargin}[1em]{2em}
\setstretch{.5}
{\PaliGlossB{He receives neither too little nor too much water.}}\\
\end{addmargin}
\end{absolutelynopagebreak}

\begin{absolutelynopagebreak}
\setstretch{.7}
{\PaliGlossA{So pattodakaṃ paṭiggaṇhāti nātithokaṃ nātibahuṃ.}}\\
\begin{addmargin}[1em]{2em}
\setstretch{.5}
{\PaliGlossB{He washes the bowl without making a sloshing noise, or spinning it around. He doesn’t put the bowl on the ground to wash his hands; his hands and bowl are washed at the same time.}}\\
\end{addmargin}
\end{absolutelynopagebreak}

\begin{absolutelynopagebreak}
\setstretch{.7}
{\PaliGlossA{So na khulukhulukārakaṃ pattaṃ dhovati, na samparivattakaṃ pattaṃ dhovati, na pattaṃ bhūmiyaṃ nikkhipitvā hatthe dhovati; hatthesu dhotesu patto dhoto hoti, patte dhote hatthā dhotā honti.}}\\
\begin{addmargin}[1em]{2em}
\setstretch{.5}
{\PaliGlossB{He doesn’t throw the bowl washing water away too far or too near, or splash it about.}}\\
\end{addmargin}
\end{absolutelynopagebreak}

\begin{absolutelynopagebreak}
\setstretch{.7}
{\PaliGlossA{So pattodakaṃ chaḍḍeti nātidūre nāccāsanne, na ca vicchaḍḍayamāno.}}\\
\begin{addmargin}[1em]{2em}
\setstretch{.5}
{\PaliGlossB{    -}}\\
\end{addmargin}
\end{absolutelynopagebreak}

\begin{absolutelynopagebreak}
\setstretch{.7}
{\PaliGlossA{    -}}\\
\begin{addmargin}[1em]{2em}
\setstretch{.5}
{\PaliGlossB{After eating he doesn’t put the bowl on the ground too far away or too close. He’s not careless with his bowl, nor does he spend too much time on it.}}\\
\end{addmargin}
\end{absolutelynopagebreak}

\begin{absolutelynopagebreak}
\setstretch{.7}
{\PaliGlossA{16. So bhuttāvī na pattaṃ bhūmiyaṃ nikkhipati nātidūre nāccāsanne, na ca anatthiko pattena hoti, na ca ativelānurakkhī pattasmiṃ.}}\\
\begin{addmargin}[1em]{2em}
\setstretch{.5}
{\PaliGlossB{    -}}\\
\end{addmargin}
\end{absolutelynopagebreak}

\begin{absolutelynopagebreak}
\setstretch{.7}
{\PaliGlossA{    -}}\\
\begin{addmargin}[1em]{2em}
\setstretch{.5}
{\PaliGlossB{After eating he sits for a while in silence, but doesn’t wait too long to give the verses of appreciation.}}\\
\end{addmargin}
\end{absolutelynopagebreak}

\begin{absolutelynopagebreak}
\setstretch{.7}
{\PaliGlossA{17. So bhuttāvī muhuttaṃ tuṇhī nisīdati, na ca anumodanassa kālamatināmeti.}}\\
\begin{addmargin}[1em]{2em}
\setstretch{.5}
{\PaliGlossB{After eating he expresses appreciation without criticizing the meal or expecting another one.}}\\
\end{addmargin}
\end{absolutelynopagebreak}

\begin{absolutelynopagebreak}
\setstretch{.7}
{\PaliGlossA{So bhuttāvī anumodati, na taṃ bhattaṃ garahati, na aññaṃ bhattaṃ paṭikaṅkhati;}}\\
\begin{addmargin}[1em]{2em}
\setstretch{.5}
{\PaliGlossB{Invariably, he educates, encourages, fires up, and inspires that assembly with a Dhamma talk.}}\\
\end{addmargin}
\end{absolutelynopagebreak}

\begin{absolutelynopagebreak}
\setstretch{.7}
{\PaliGlossA{aññadatthu dhammiyā kathāya taṃ parisaṃ sandasseti samādapeti samuttejeti sampahaṃseti.}}\\
\begin{addmargin}[1em]{2em}
\setstretch{.5}
{\PaliGlossB{Then he gets up from his seat and leaves.}}\\
\end{addmargin}
\end{absolutelynopagebreak}

\begin{absolutelynopagebreak}
\setstretch{.7}
{\PaliGlossA{So taṃ parisaṃ dhammiyā kathāya sandassetvā samādapetvā samuttejetvā sampahaṃsetvā uṭṭhāyāsanā pakkamati.}}\\
\begin{addmargin}[1em]{2em}
\setstretch{.5}
{\PaliGlossB{    -}}\\
\end{addmargin}
\end{absolutelynopagebreak}

\begin{absolutelynopagebreak}
\setstretch{.7}
{\PaliGlossA{    -}}\\
\begin{addmargin}[1em]{2em}
\setstretch{.5}
{\PaliGlossB{He walks neither too fast nor too slow, without wanting to get out of there.}}\\
\end{addmargin}
\end{absolutelynopagebreak}

\begin{absolutelynopagebreak}
\setstretch{.7}
{\PaliGlossA{18. So nātisīghaṃ gacchati, nātisaṇikaṃ gacchati, na ca muccitukāmo gacchati;}}\\
\begin{addmargin}[1em]{2em}
\setstretch{.5}
{\PaliGlossB{    -}}\\
\end{addmargin}
\end{absolutelynopagebreak}

\begin{absolutelynopagebreak}
\setstretch{.7}
{\PaliGlossA{    -}}\\
\begin{addmargin}[1em]{2em}
\setstretch{.5}
{\PaliGlossB{He wears his robe on his body neither too high nor too low, neither too tight nor too loose.}}\\
\end{addmargin}
\end{absolutelynopagebreak}

\begin{absolutelynopagebreak}
\setstretch{.7}
{\PaliGlossA{19. na ca tassa bhoto gotamassa kāye cīvaraṃ accukkaṭṭhaṃ hoti na ca accokkaṭṭhaṃ, na ca kāyasmiṃ allīnaṃ na ca kāyasmā apakaṭṭhaṃ;}}\\
\begin{addmargin}[1em]{2em}
\setstretch{.5}
{\PaliGlossB{The wind doesn’t blow his robe off his body.}}\\
\end{addmargin}
\end{absolutelynopagebreak}

\begin{absolutelynopagebreak}
\setstretch{.7}
{\PaliGlossA{na ca tassa bhoto gotamassa kāyamhā vāto cīvaraṃ apavahati;}}\\
\begin{addmargin}[1em]{2em}
\setstretch{.5}
{\PaliGlossB{And dust and dirt don’t stick to his body.}}\\
\end{addmargin}
\end{absolutelynopagebreak}

\begin{absolutelynopagebreak}
\setstretch{.7}
{\PaliGlossA{na ca tassa bhoto gotamassa kāye rajojallaṃ upalimpati.}}\\
\begin{addmargin}[1em]{2em}
\setstretch{.5}
{\PaliGlossB{    -}}\\
\end{addmargin}
\end{absolutelynopagebreak}

\begin{absolutelynopagebreak}
\setstretch{.7}
{\PaliGlossA{    -}}\\
\begin{addmargin}[1em]{2em}
\setstretch{.5}
{\PaliGlossB{When he has gone to the monastery he sits on a seat spread out and washes his feet.}}\\
\end{addmargin}
\end{absolutelynopagebreak}

\begin{absolutelynopagebreak}
\setstretch{.7}
{\PaliGlossA{20. So ārāmagato nisīdati paññatte āsane. Nisajja pāde pakkhāleti;}}\\
\begin{addmargin}[1em]{2em}
\setstretch{.5}
{\PaliGlossB{But he doesn’t waste time with pedicures.}}\\
\end{addmargin}
\end{absolutelynopagebreak}

\begin{absolutelynopagebreak}
\setstretch{.7}
{\PaliGlossA{na ca so bhavaṃ gotamo pādamaṇḍanānuyogamanuyutto viharati.}}\\
\begin{addmargin}[1em]{2em}
\setstretch{.5}
{\PaliGlossB{When he has washed his feet, he sits down cross-legged, with his body straight, and establishes mindfulness right there.}}\\
\end{addmargin}
\end{absolutelynopagebreak}

\begin{absolutelynopagebreak}
\setstretch{.7}
{\PaliGlossA{So pāde pakkhāletvā nisīdati pallaṅkaṃ ābhujitvā ujuṃ kāyaṃ paṇidhāya parimukhaṃ satiṃ upaṭṭhapetvā.}}\\
\begin{addmargin}[1em]{2em}
\setstretch{.5}
{\PaliGlossB{He has no intention to hurt himself, hurt others, or hurt both.}}\\
\end{addmargin}
\end{absolutelynopagebreak}

\begin{absolutelynopagebreak}
\setstretch{.7}
{\PaliGlossA{So neva attabyābādhāya ceteti, na parabyābādhāya ceteti, na ubhayabyābādhāya ceteti;}}\\
\begin{addmargin}[1em]{2em}
\setstretch{.5}
{\PaliGlossB{He only wishes for the welfare of himself, of others, of both, and of the whole world.}}\\
\end{addmargin}
\end{absolutelynopagebreak}

\begin{absolutelynopagebreak}
\setstretch{.7}
{\PaliGlossA{attahitaparahitaubhayahitasabbalokahitameva so bhavaṃ gotamo cintento nisinno hoti.}}\\
\begin{addmargin}[1em]{2em}
\setstretch{.5}
{\PaliGlossB{In the monastery when he teaches Dhamma to an assembly, he neither flatters them nor rebukes them.}}\\
\end{addmargin}
\end{absolutelynopagebreak}

\begin{absolutelynopagebreak}
\setstretch{.7}
{\PaliGlossA{So ārāmagato parisati dhammaṃ deseti, na taṃ parisaṃ ussādeti, na taṃ parisaṃ apasādeti;}}\\
\begin{addmargin}[1em]{2em}
\setstretch{.5}
{\PaliGlossB{Invariably, he educates, encourages, fires up, and inspires that assembly with a Dhamma talk.}}\\
\end{addmargin}
\end{absolutelynopagebreak}

\begin{absolutelynopagebreak}
\setstretch{.7}
{\PaliGlossA{aññadatthu dhammiyā kathāya taṃ parisaṃ sandasseti samādapeti samuttejeti sampahaṃseti.}}\\
\begin{addmargin}[1em]{2em}
\setstretch{.5}
{\PaliGlossB{    -}}\\
\end{addmargin}
\end{absolutelynopagebreak}

\begin{absolutelynopagebreak}
\setstretch{.7}
{\PaliGlossA{    -}}\\
\begin{addmargin}[1em]{2em}
\setstretch{.5}
{\PaliGlossB{His voice has eight qualities:}}\\
\end{addmargin}
\end{absolutelynopagebreak}

\begin{absolutelynopagebreak}
\setstretch{.7}
{\PaliGlossA{22. Aṭṭhaṅgasamannāgato kho panassa bhoto gotamassa mukhato ghoso niccharati—}}\\
\begin{addmargin}[1em]{2em}
\setstretch{.5}
{\PaliGlossB{it is clear, comprehensible, charming, audible, rounded, undistorted, deep, and resonant.}}\\
\end{addmargin}
\end{absolutelynopagebreak}

\begin{absolutelynopagebreak}
\setstretch{.7}
{\PaliGlossA{vissaṭṭho ca, viññeyyo ca, mañju ca, savanīyo ca, bindu ca, avisārī ca, gambhīro ca, ninnādī ca.}}\\
\begin{addmargin}[1em]{2em}
\setstretch{.5}
{\PaliGlossB{He makes sure his voice is intelligible as far as the assembly goes, but it doesn’t extend outside the assembly.}}\\
\end{addmargin}
\end{absolutelynopagebreak}

\begin{absolutelynopagebreak}
\setstretch{.7}
{\PaliGlossA{Yathāparisaṃ kho pana so bhavaṃ gotamo sarena viññāpeti, na cassa bahiddhā parisāya ghoso niccharati.}}\\
\begin{addmargin}[1em]{2em}
\setstretch{.5}
{\PaliGlossB{And when they’ve been inspired with a Dhamma talk by Master Gotama they get up from their seats and leave looking back at him alone, and not forgetting their lesson.}}\\
\end{addmargin}
\end{absolutelynopagebreak}

\begin{absolutelynopagebreak}
\setstretch{.7}
{\PaliGlossA{Te tena bhotā gotamena dhammiyā kathāya sandassitā samādapitā samuttejitā sampahaṃsitā uṭṭhāyāsanā pakkamanti avalokayamānāyeva avijahitattā.}}\\
\begin{addmargin}[1em]{2em}
\setstretch{.5}
{\PaliGlossB{I have seen Master Gotama walking and standing; entering inhabited areas, and sitting and eating there; sitting silently after eating, and expressing appreciation; going to the monastery, sitting silently there, and teaching Dhamma to an assembly there.}}\\
\end{addmargin}
\end{absolutelynopagebreak}

\begin{absolutelynopagebreak}
\setstretch{.7}
{\PaliGlossA{Addasāma kho mayaṃ, bho, taṃ bhavantaṃ gotamaṃ gacchantaṃ, addasāma ṭhitaṃ, addasāma antaragharaṃ pavisantaṃ, addasāma antaraghare nisinnaṃ tuṇhībhūtaṃ, addasāma antaraghare bhuñjantaṃ, addasāma bhuttāviṃ nisinnaṃ tuṇhībhūtaṃ, addasāma bhuttāviṃ anumodantaṃ, addasāma ārāmaṃ gacchantaṃ, addasāma ārāmagataṃ nisinnaṃ tuṇhībhūtaṃ, addasāma ārāmagataṃ parisati dhammaṃ desentaṃ.}}\\
\begin{addmargin}[1em]{2em}
\setstretch{.5}
{\PaliGlossB{Such is Master Gotama; such he is and more than that.”}}\\
\end{addmargin}
\end{absolutelynopagebreak}

\begin{absolutelynopagebreak}
\setstretch{.7}
{\PaliGlossA{Ediso ca ediso ca so bhavaṃ gotamo, tato ca bhiyyo”ti.}}\\
\begin{addmargin}[1em]{2em}
\setstretch{.5}
{\PaliGlossB{    -}}\\
\end{addmargin}
\end{absolutelynopagebreak}

\begin{absolutelynopagebreak}
\setstretch{.7}
{\PaliGlossA{    -}}\\
\begin{addmargin}[1em]{2em}
\setstretch{.5}
{\PaliGlossB{When he had spoken, the brahmin Brahmāyu got up from his seat, arranged his robe over one shoulder, knelt on his right knee, raised his joined palms toward the Buddha, and uttered this aphorism three times:}}\\
\end{addmargin}
\end{absolutelynopagebreak}

\begin{absolutelynopagebreak}
\setstretch{.7}
{\PaliGlossA{23. Evaṃ vutte, brahmāyu brāhmaṇo uṭṭhāyāsanā ekaṃsaṃ uttarāsaṅgaṃ karitvā yena bhagavā tenañjaliṃ paṇāmetvā tikkhattuṃ udānaṃ udāneti:}}\\
\begin{addmargin}[1em]{2em}
\setstretch{.5}
{\PaliGlossB{“Homage to that Blessed One, the perfected one, the fully awakened Buddha!}}\\
\end{addmargin}
\end{absolutelynopagebreak}

\begin{absolutelynopagebreak}
\setstretch{.7}
{\PaliGlossA{“Namo tassa bhagavato arahato sammāsambuddhassa.}}\\
\begin{addmargin}[1em]{2em}
\setstretch{.5}
{\PaliGlossB{Homage to that Blessed One, the perfected one, the fully awakened Buddha!}}\\
\end{addmargin}
\end{absolutelynopagebreak}

\begin{absolutelynopagebreak}
\setstretch{.7}
{\PaliGlossA{Namo tassa bhagavato arahato sammāsambuddhassa.}}\\
\begin{addmargin}[1em]{2em}
\setstretch{.5}
{\PaliGlossB{Homage to that Blessed One, the perfected one, the fully awakened Buddha!}}\\
\end{addmargin}
\end{absolutelynopagebreak}

\begin{absolutelynopagebreak}
\setstretch{.7}
{\PaliGlossA{Namo tassa bhagavato arahato sammāsambuddhassāti.}}\\
\begin{addmargin}[1em]{2em}
\setstretch{.5}
{\PaliGlossB{Hopefully, some time or other I’ll get to meet him, and we can have a discussion.”}}\\
\end{addmargin}
\end{absolutelynopagebreak}

\begin{absolutelynopagebreak}
\setstretch{.7}
{\PaliGlossA{Appeva nāma mayaṃ kadāci karahaci tena bhotā gotamena samāgaccheyyāma? Appeva nāma siyā kocideva kathāsallāpo”ti.}}\\
\begin{addmargin}[1em]{2em}
\setstretch{.5}
{\PaliGlossB{    -}}\\
\end{addmargin}
\end{absolutelynopagebreak}

\begin{absolutelynopagebreak}
\setstretch{.7}
{\PaliGlossA{    -}}\\
\begin{addmargin}[1em]{2em}
\setstretch{.5}
{\PaliGlossB{And then the Buddha, traveling stage by stage in the Videhan lands, arrived at Mithilā,}}\\
\end{addmargin}
\end{absolutelynopagebreak}

\begin{absolutelynopagebreak}
\setstretch{.7}
{\PaliGlossA{24. Atha kho bhagavā videhesu anupubbena cārikaṃ caramāno yena mithilā tadavasari.}}\\
\begin{addmargin}[1em]{2em}
\setstretch{.5}
{\PaliGlossB{where he stayed in the Makhādeva Mango Grove.}}\\
\end{addmargin}
\end{absolutelynopagebreak}

\begin{absolutelynopagebreak}
\setstretch{.7}
{\PaliGlossA{Tatra sudaṃ bhagavā mithilāyaṃ viharati maghadevaambavane.}}\\
\begin{addmargin}[1em]{2em}
\setstretch{.5}
{\PaliGlossB{The brahmins and householders of Mithilā heard:}}\\
\end{addmargin}
\end{absolutelynopagebreak}

\begin{absolutelynopagebreak}
\setstretch{.7}
{\PaliGlossA{Assosuṃ kho mithileyyakā brāhmaṇagahapatikā:}}\\
\begin{addmargin}[1em]{2em}
\setstretch{.5}
{\PaliGlossB{“It seems the ascetic Gotama—a Sakyan, gone forth from a Sakyan family—has arrived at Mithilā, where he is staying in the Makhādeva Mango Grove.}}\\
\end{addmargin}
\end{absolutelynopagebreak}

\begin{absolutelynopagebreak}
\setstretch{.7}
{\PaliGlossA{“samaṇo khalu, bho, gotamo sakyaputto sakyakulā pabbajito videhesu cārikaṃ caramāno mahatā bhikkhusaṃghena saddhiṃ pañcamattehi bhikkhusatehi mithilaṃ anuppatto, mithilāyaṃ viharati maghadevaambavane.}}\\
\begin{addmargin}[1em]{2em}
\setstretch{.5}
{\PaliGlossB{He has this good reputation:}}\\
\end{addmargin}
\end{absolutelynopagebreak}

\begin{absolutelynopagebreak}
\setstretch{.7}
{\PaliGlossA{Taṃ kho pana bhavantaṃ gotamaṃ evaṃ kalyāṇo kittisaddo abbhuggato:}}\\
\begin{addmargin}[1em]{2em}
\setstretch{.5}
{\PaliGlossB{‘That Blessed One is perfected, a fully awakened Buddha, accomplished in knowledge and conduct, holy, knower of the world, supreme guide for those who wish to train, teacher of gods and humans, awakened, blessed.’}}\\
\end{addmargin}
\end{absolutelynopagebreak}

\begin{absolutelynopagebreak}
\setstretch{.7}
{\PaliGlossA{‘itipi so bhagavā arahaṃ sammāsambuddho vijjācaraṇasampanno sugato lokavidū anuttaro purisadammasārathi satthā devamanussānaṃ buddho bhagavā’ti.}}\\
\begin{addmargin}[1em]{2em}
\setstretch{.5}
{\PaliGlossB{He has realized with his own insight this world—with its gods, Māras and Brahmās, this population with its ascetics and brahmins, gods and humans—and he makes it known to others.}}\\
\end{addmargin}
\end{absolutelynopagebreak}

\begin{absolutelynopagebreak}
\setstretch{.7}
{\PaliGlossA{So imaṃ lokaṃ sadevakaṃ samārakaṃ sabrahmakaṃ sassamaṇabrāhmaṇiṃ pajaṃ sadevamanussaṃ sayaṃ abhiññā sacchikatvā pavedeti.}}\\
\begin{addmargin}[1em]{2em}
\setstretch{.5}
{\PaliGlossB{He teaches Dhamma that’s good in the beginning, good in the middle, and good in the end, meaningful and well-phrased. And he reveals a spiritual practice that’s entirely full and pure.}}\\
\end{addmargin}
\end{absolutelynopagebreak}

\begin{absolutelynopagebreak}
\setstretch{.7}
{\PaliGlossA{So dhammaṃ deseti ādikalyāṇaṃ majjhekalyāṇaṃ pariyosānakalyāṇaṃ sātthaṃ sabyañjanaṃ, kevalaparipuṇṇaṃ parisuddhaṃ brahmacariyaṃ pakāseti.}}\\
\begin{addmargin}[1em]{2em}
\setstretch{.5}
{\PaliGlossB{It’s good to see such perfected ones.”}}\\
\end{addmargin}
\end{absolutelynopagebreak}

\begin{absolutelynopagebreak}
\setstretch{.7}
{\PaliGlossA{Sādhu kho pana tathārūpānaṃ arahataṃ dassanaṃ hotī”ti.}}\\
\begin{addmargin}[1em]{2em}
\setstretch{.5}
{\PaliGlossB{    -}}\\
\end{addmargin}
\end{absolutelynopagebreak}

\begin{absolutelynopagebreak}
\setstretch{.7}
{\PaliGlossA{    -}}\\
\begin{addmargin}[1em]{2em}
\setstretch{.5}
{\PaliGlossB{Then the brahmins and householders of Mithilā went up to the Buddha. Before sitting down to one side, some bowed, some exchanged greetings and polite conversation, some held up their joined palms toward the Buddha, some announced their name and clan, while some kept silent.}}\\
\end{addmargin}
\end{absolutelynopagebreak}

\begin{absolutelynopagebreak}
\setstretch{.7}
{\PaliGlossA{25. Atha kho mithileyyakā brāhmaṇagahapatikā yena bhagavā tenupasaṅkamiṃsu; upasaṅkamitvā appekacce bhagavantaṃ abhivādetvā ekamantaṃ nisīdiṃsu; appekacce bhagavatā saddhiṃ sammodiṃsu, sammodanīyaṃ kathaṃ sāraṇīyaṃ vītisāretvā ekamantaṃ nisīdiṃsu; appekacce yena bhagavā tenañjaliṃ paṇāmetvā ekamantaṃ nisīdiṃsu; appekacce bhagavato santike nāmagottaṃ sāvetvā ekamantaṃ nisīdiṃsu; appekacce tuṇhībhūtā ekamantaṃ nisīdiṃsu.}}\\
\begin{addmargin}[1em]{2em}
\setstretch{.5}
{\PaliGlossB{    -}}\\
\end{addmargin}
\end{absolutelynopagebreak}

\begin{absolutelynopagebreak}
\setstretch{.7}
{\PaliGlossA{    -}}\\
\begin{addmargin}[1em]{2em}
\setstretch{.5}
{\PaliGlossB{The brahmin Brahmāyu also heard that the Buddha had arrived.}}\\
\end{addmargin}
\end{absolutelynopagebreak}

\begin{absolutelynopagebreak}
\setstretch{.7}
{\PaliGlossA{26. Assosi kho brahmāyu brāhmaṇo: “samaṇo khalu, bho, gotamo sakyaputto sakyakulā pabbajito mithilaṃ anuppatto, mithilāyaṃ viharati maghadevaambavane”ti.}}\\
\begin{addmargin}[1em]{2em}
\setstretch{.5}
{\PaliGlossB{So he went to the Makhādeva Mango Grove together with several disciples.}}\\
\end{addmargin}
\end{absolutelynopagebreak}

\begin{absolutelynopagebreak}
\setstretch{.7}
{\PaliGlossA{Atha kho brahmāyu brāhmaṇo sambahulehi sāvakehi saddhiṃ yena maghadevaambavanaṃ tenupasaṅkami.}}\\
\begin{addmargin}[1em]{2em}
\setstretch{.5}
{\PaliGlossB{Not far from the grove he thought,}}\\
\end{addmargin}
\end{absolutelynopagebreak}

\begin{absolutelynopagebreak}
\setstretch{.7}
{\PaliGlossA{Atha kho brahmāyuno brāhmaṇassa avidūre ambavanassa etadahosi:}}\\
\begin{addmargin}[1em]{2em}
\setstretch{.5}
{\PaliGlossB{“It wouldn’t be appropriate for me to go to see the ascetic Gotama without first letting him know.”}}\\
\end{addmargin}
\end{absolutelynopagebreak}

\begin{absolutelynopagebreak}
\setstretch{.7}
{\PaliGlossA{“na kho metaṃ patirūpaṃ yohaṃ pubbe appaṭisaṃvidito samaṇaṃ gotamaṃ dassanāya upasaṅkameyyan”ti.}}\\
\begin{addmargin}[1em]{2em}
\setstretch{.5}
{\PaliGlossB{So he addressed one of his students:}}\\
\end{addmargin}
\end{absolutelynopagebreak}

\begin{absolutelynopagebreak}
\setstretch{.7}
{\PaliGlossA{Atha kho brahmāyu brāhmaṇo aññataraṃ māṇavakaṃ āmantesi:}}\\
\begin{addmargin}[1em]{2em}
\setstretch{.5}
{\PaliGlossB{“Here, student, go to the ascetic Gotama and in my name bow with your head to his feet. Ask him if he is healthy and well, nimble, strong, and living comfortably.}}\\
\end{addmargin}
\end{absolutelynopagebreak}

\begin{absolutelynopagebreak}
\setstretch{.7}
{\PaliGlossA{“ehi tvaṃ, māṇavaka, yena samaṇo gotamo tenupasaṅkama; upasaṅkamitvā mama vacanena samaṇaṃ gotamaṃ appābādhaṃ appātaṅkaṃ lahuṭṭhānaṃ balaṃ phāsuvihāraṃ puccha:}}\\
\begin{addmargin}[1em]{2em}
\setstretch{.5}
{\PaliGlossB{    -}}\\
\end{addmargin}
\end{absolutelynopagebreak}

\begin{absolutelynopagebreak}
\setstretch{.7}
{\PaliGlossA{‘brahmāyu, bho gotama, brāhmaṇo bhavantaṃ gotamaṃ appābādhaṃ appātaṅkaṃ lahuṭṭhānaṃ balaṃ phāsuvihāraṃ pucchatī’ti.}}\\
\begin{addmargin}[1em]{2em}
\setstretch{.5}
{\PaliGlossB{And then say:}}\\
\end{addmargin}
\end{absolutelynopagebreak}

\begin{absolutelynopagebreak}
\setstretch{.7}
{\PaliGlossA{Evañca vadehi:}}\\
\begin{addmargin}[1em]{2em}
\setstretch{.5}
{\PaliGlossB{‘Master Gotama, the brahmin Brahmāyu is old, elderly, and senior, advanced in years, having reached the final stage of life; he is a hundred and twenty years old. He has mastered the three Vedas, together with their vocabularies, ritual, phonology and etymology, and the testament as fifth. He knows philology and grammar, and is well versed in cosmology and the marks of a great man.}}\\
\end{addmargin}
\end{absolutelynopagebreak}

\begin{absolutelynopagebreak}
\setstretch{.7}
{\PaliGlossA{‘brahmāyu, bho gotama, brāhmaṇo jiṇṇo vuḍḍho mahallako addhagato vayoanuppatto, vīsavassasatiko jātiyā, tiṇṇaṃ vedānaṃ pāragū sanighaṇḍukeṭubhānaṃ sākkharappabhedānaṃ itihāsapañcamānaṃ, padako, veyyākaraṇo, lokāyatamahāpurisalakkhaṇesu anavayo.}}\\
\begin{addmargin}[1em]{2em}
\setstretch{.5}
{\PaliGlossB{Of all the brahmins and householders residing in Mithilā, Brahmāyu is said to be the foremost in}}\\
\end{addmargin}
\end{absolutelynopagebreak}

\begin{absolutelynopagebreak}
\setstretch{.7}
{\PaliGlossA{Yāvatā, bho, brāhmaṇagahapatikā mithilāyaṃ paṭivasanti, brahmāyu tesaṃ brāhmaṇo aggamakkhāyati—}}\\
\begin{addmargin}[1em]{2em}
\setstretch{.5}
{\PaliGlossB{wealth,}}\\
\end{addmargin}
\end{absolutelynopagebreak}

\begin{absolutelynopagebreak}
\setstretch{.7}
{\PaliGlossA{yadidaṃ bhogehi;}}\\
\begin{addmargin}[1em]{2em}
\setstretch{.5}
{\PaliGlossB{    -}}\\
\end{addmargin}
\end{absolutelynopagebreak}

\begin{absolutelynopagebreak}
\setstretch{.7}
{\PaliGlossA{brahmāyu tesaṃ brāhmaṇo aggamakkhāyati—}}\\
\begin{addmargin}[1em]{2em}
\setstretch{.5}
{\PaliGlossB{hymns,}}\\
\end{addmargin}
\end{absolutelynopagebreak}

\begin{absolutelynopagebreak}
\setstretch{.7}
{\PaliGlossA{yadidaṃ mantehi;}}\\
\begin{addmargin}[1em]{2em}
\setstretch{.5}
{\PaliGlossB{    -}}\\
\end{addmargin}
\end{absolutelynopagebreak}

\begin{absolutelynopagebreak}
\setstretch{.7}
{\PaliGlossA{brahmāyu tesaṃ brāhmaṇo aggamakkhāyati—}}\\
\begin{addmargin}[1em]{2em}
\setstretch{.5}
{\PaliGlossB{lifespan, and fame.}}\\
\end{addmargin}
\end{absolutelynopagebreak}

\begin{absolutelynopagebreak}
\setstretch{.7}
{\PaliGlossA{yadidaṃ āyunā ceva yasasā ca.}}\\
\begin{addmargin}[1em]{2em}
\setstretch{.5}
{\PaliGlossB{He wants to see Master Gotama.’”}}\\
\end{addmargin}
\end{absolutelynopagebreak}

\begin{absolutelynopagebreak}
\setstretch{.7}
{\PaliGlossA{So bhoto gotamassa dassanakāmo’”ti.}}\\
\begin{addmargin}[1em]{2em}
\setstretch{.5}
{\PaliGlossB{“Yes, sir,” that student replied. He did as he was asked, and the Buddha said,}}\\
\end{addmargin}
\end{absolutelynopagebreak}

\begin{absolutelynopagebreak}
\setstretch{.7}
{\PaliGlossA{“Evaṃ, bho”ti kho so māṇavako brahmāyussa brāhmaṇassa paṭissutvā yena bhagavā tenupasaṅkami; upasaṅkamitvā bhagavatā saddhiṃ sammodi.}}\\
\begin{addmargin}[1em]{2em}
\setstretch{.5}
{\PaliGlossB{    -}}\\
\end{addmargin}
\end{absolutelynopagebreak}

\begin{absolutelynopagebreak}
\setstretch{.7}
{\PaliGlossA{Sammodanīyaṃ kathaṃ sāraṇīyaṃ vītisāretvā ekamantaṃ aṭṭhāsi. Ekamantaṃ ṭhito kho so māṇavako bhagavantaṃ etadavoca:}}\\
\begin{addmargin}[1em]{2em}
\setstretch{.5}
{\PaliGlossB{    -}}\\
\end{addmargin}
\end{absolutelynopagebreak}

\begin{absolutelynopagebreak}
\setstretch{.7}
{\PaliGlossA{“brahmāyu, bho gotama, brāhmaṇo bhavantaṃ gotamaṃ appābādhaṃ appātaṅkaṃ lahuṭṭhānaṃ balaṃ phāsuvihāraṃ pucchati;}}\\
\begin{addmargin}[1em]{2em}
\setstretch{.5}
{\PaliGlossB{    -}}\\
\end{addmargin}
\end{absolutelynopagebreak}

\begin{absolutelynopagebreak}
\setstretch{.7}
{\PaliGlossA{evañca vadeti:}}\\
\begin{addmargin}[1em]{2em}
\setstretch{.5}
{\PaliGlossB{    -}}\\
\end{addmargin}
\end{absolutelynopagebreak}

\begin{absolutelynopagebreak}
\setstretch{.7}
{\PaliGlossA{‘brahmāyu, bho gotama, brāhmaṇo jiṇṇo vuḍḍho mahallako addhagato vayoanuppatto, vīsavassasatiko jātiyā, tiṇṇaṃ vedānaṃ pāragū sanighaṇḍukeṭubhānaṃ sākkharappabhedānaṃ itihāsapañcamānaṃ, padako, veyyākaraṇo, lokāyatamahāpurisalakkhaṇesu anavayo.}}\\
\begin{addmargin}[1em]{2em}
\setstretch{.5}
{\PaliGlossB{    -}}\\
\end{addmargin}
\end{absolutelynopagebreak}

\begin{absolutelynopagebreak}
\setstretch{.7}
{\PaliGlossA{Yāvatā, bho, brāhmaṇagahapatikā mithilāyaṃ paṭivasanti, brahmāyu tesaṃ brāhmaṇo aggamakkhāyati—}}\\
\begin{addmargin}[1em]{2em}
\setstretch{.5}
{\PaliGlossB{    -}}\\
\end{addmargin}
\end{absolutelynopagebreak}

\begin{absolutelynopagebreak}
\setstretch{.7}
{\PaliGlossA{yadidaṃ bhogehi;}}\\
\begin{addmargin}[1em]{2em}
\setstretch{.5}
{\PaliGlossB{    -}}\\
\end{addmargin}
\end{absolutelynopagebreak}

\begin{absolutelynopagebreak}
\setstretch{.7}
{\PaliGlossA{brahmāyu tesaṃ brāhmaṇo aggamakkhāyati—}}\\
\begin{addmargin}[1em]{2em}
\setstretch{.5}
{\PaliGlossB{    -}}\\
\end{addmargin}
\end{absolutelynopagebreak}

\begin{absolutelynopagebreak}
\setstretch{.7}
{\PaliGlossA{yadidaṃ mantehi;}}\\
\begin{addmargin}[1em]{2em}
\setstretch{.5}
{\PaliGlossB{    -}}\\
\end{addmargin}
\end{absolutelynopagebreak}

\begin{absolutelynopagebreak}
\setstretch{.7}
{\PaliGlossA{brahmāyu tesaṃ brāhmaṇo aggamakkhāyati—}}\\
\begin{addmargin}[1em]{2em}
\setstretch{.5}
{\PaliGlossB{    -}}\\
\end{addmargin}
\end{absolutelynopagebreak}

\begin{absolutelynopagebreak}
\setstretch{.7}
{\PaliGlossA{yadidaṃ āyunā ceva yasasā ca.}}\\
\begin{addmargin}[1em]{2em}
\setstretch{.5}
{\PaliGlossB{    -}}\\
\end{addmargin}
\end{absolutelynopagebreak}

\begin{absolutelynopagebreak}
\setstretch{.7}
{\PaliGlossA{So bhoto gotamassa dassanakāmo’”ti.}}\\
\begin{addmargin}[1em]{2em}
\setstretch{.5}
{\PaliGlossB{“Please, student, let Brahmāyu come when he’s ready.”}}\\
\end{addmargin}
\end{absolutelynopagebreak}

\begin{absolutelynopagebreak}
\setstretch{.7}
{\PaliGlossA{“Yassadāni, māṇava, brahmāyu brāhmaṇo kālaṃ maññatī”ti.}}\\
\begin{addmargin}[1em]{2em}
\setstretch{.5}
{\PaliGlossB{    -}}\\
\end{addmargin}
\end{absolutelynopagebreak}

\begin{absolutelynopagebreak}
\setstretch{.7}
{\PaliGlossA{    -}}\\
\begin{addmargin}[1em]{2em}
\setstretch{.5}
{\PaliGlossB{The student went back to Brahmāyu and said to him,}}\\
\end{addmargin}
\end{absolutelynopagebreak}

\begin{absolutelynopagebreak}
\setstretch{.7}
{\PaliGlossA{27. Atha kho so māṇavako yena brahmāyu brāhmaṇo tenupasaṅkami; upasaṅkamitvā brahmāyuṃ brāhmaṇaṃ etadavoca:}}\\
\begin{addmargin}[1em]{2em}
\setstretch{.5}
{\PaliGlossB{“Your request for an audience with the ascetic Gotama has been granted.}}\\
\end{addmargin}
\end{absolutelynopagebreak}

\begin{absolutelynopagebreak}
\setstretch{.7}
{\PaliGlossA{“katāvakāso khomhi bhavatā samaṇena gotamena.}}\\
\begin{addmargin}[1em]{2em}
\setstretch{.5}
{\PaliGlossB{Please go at your convenience.”}}\\
\end{addmargin}
\end{absolutelynopagebreak}

\begin{absolutelynopagebreak}
\setstretch{.7}
{\PaliGlossA{Yassadāni bhavaṃ kālaṃ maññatī”ti.}}\\
\begin{addmargin}[1em]{2em}
\setstretch{.5}
{\PaliGlossB{    -}}\\
\end{addmargin}
\end{absolutelynopagebreak}

\begin{absolutelynopagebreak}
\setstretch{.7}
{\PaliGlossA{    -}}\\
\begin{addmargin}[1em]{2em}
\setstretch{.5}
{\PaliGlossB{Then the brahmin Brahmāyu went up to the Buddha.}}\\
\end{addmargin}
\end{absolutelynopagebreak}

\begin{absolutelynopagebreak}
\setstretch{.7}
{\PaliGlossA{28. Atha kho brahmāyu brāhmaṇo yena bhagavā tenupasaṅkami.}}\\
\begin{addmargin}[1em]{2em}
\setstretch{.5}
{\PaliGlossB{The assembly saw him coming off in the distance,}}\\
\end{addmargin}
\end{absolutelynopagebreak}

\begin{absolutelynopagebreak}
\setstretch{.7}
{\PaliGlossA{Addasā kho sā parisā brahmāyuṃ brāhmaṇaṃ dūratova āgacchantaṃ.}}\\
\begin{addmargin}[1em]{2em}
\setstretch{.5}
{\PaliGlossB{and made way for him, as he was well-known and famous.}}\\
\end{addmargin}
\end{absolutelynopagebreak}

\begin{absolutelynopagebreak}
\setstretch{.7}
{\PaliGlossA{Disvāna oramiya okāsamakāsi yathā taṃ ñātassa yasassino.}}\\
\begin{addmargin}[1em]{2em}
\setstretch{.5}
{\PaliGlossB{Brahmāyu said to that retinue,}}\\
\end{addmargin}
\end{absolutelynopagebreak}

\begin{absolutelynopagebreak}
\setstretch{.7}
{\PaliGlossA{Atha kho brahmāyu brāhmaṇo taṃ parisaṃ etadavoca:}}\\
\begin{addmargin}[1em]{2em}
\setstretch{.5}
{\PaliGlossB{“Enough, gentlemen.}}\\
\end{addmargin}
\end{absolutelynopagebreak}

\begin{absolutelynopagebreak}
\setstretch{.7}
{\PaliGlossA{“alaṃ, bho.}}\\
\begin{addmargin}[1em]{2em}
\setstretch{.5}
{\PaliGlossB{Please sit on your own seats.}}\\
\end{addmargin}
\end{absolutelynopagebreak}

\begin{absolutelynopagebreak}
\setstretch{.7}
{\PaliGlossA{Nisīdatha tumhe sake āsane.}}\\
\begin{addmargin}[1em]{2em}
\setstretch{.5}
{\PaliGlossB{I shall sit here by the ascetic Gotama.”}}\\
\end{addmargin}
\end{absolutelynopagebreak}

\begin{absolutelynopagebreak}
\setstretch{.7}
{\PaliGlossA{Idhāhaṃ samaṇassa gotamassa santike nisīdissāmī”ti.}}\\
\begin{addmargin}[1em]{2em}
\setstretch{.5}
{\PaliGlossB{    -}}\\
\end{addmargin}
\end{absolutelynopagebreak}

\begin{absolutelynopagebreak}
\setstretch{.7}
{\PaliGlossA{    -}}\\
\begin{addmargin}[1em]{2em}
\setstretch{.5}
{\PaliGlossB{Then the brahmin Brahmāyu went up to the Buddha, and exchanged greetings with him.}}\\
\end{addmargin}
\end{absolutelynopagebreak}

\begin{absolutelynopagebreak}
\setstretch{.7}
{\PaliGlossA{29. Atha kho brahmāyu brāhmaṇo yena bhagavā tenupasaṅkami; upasaṅkamitvā bhagavatā saddhiṃ sammodi.}}\\
\begin{addmargin}[1em]{2em}
\setstretch{.5}
{\PaliGlossB{When the greetings and polite conversation were over, he sat down to one side,}}\\
\end{addmargin}
\end{absolutelynopagebreak}

\begin{absolutelynopagebreak}
\setstretch{.7}
{\PaliGlossA{Sammodanīyaṃ kathaṃ sāraṇīyaṃ vītisāretvā ekamantaṃ nisīdi.}}\\
\begin{addmargin}[1em]{2em}
\setstretch{.5}
{\PaliGlossB{and scrutinized the Buddha’s body for the thirty-two marks of a great man.}}\\
\end{addmargin}
\end{absolutelynopagebreak}

\begin{absolutelynopagebreak}
\setstretch{.7}
{\PaliGlossA{Ekamantaṃ nisinno kho brahmāyu brāhmaṇo bhagavato kāye dvattiṃsamahāpurisalakkhaṇāni samannesi.}}\\
\begin{addmargin}[1em]{2em}
\setstretch{.5}
{\PaliGlossB{He saw all of them except for two,}}\\
\end{addmargin}
\end{absolutelynopagebreak}

\begin{absolutelynopagebreak}
\setstretch{.7}
{\PaliGlossA{Addasā kho brahmāyu brāhmaṇo bhagavato kāye dvattiṃsamahāpurisalakkhaṇāni, yebhuyyena ṭhapetvā dve.}}\\
\begin{addmargin}[1em]{2em}
\setstretch{.5}
{\PaliGlossB{which he had doubts about:}}\\
\end{addmargin}
\end{absolutelynopagebreak}

\begin{absolutelynopagebreak}
\setstretch{.7}
{\PaliGlossA{Dvīsu mahāpurisalakkhaṇesu kaṅkhati vicikicchati nādhimuccati na sampasīdati—}}\\
\begin{addmargin}[1em]{2em}
\setstretch{.5}
{\PaliGlossB{whether the private parts are retracted, and the largeness of the tongue.}}\\
\end{addmargin}
\end{absolutelynopagebreak}

\begin{absolutelynopagebreak}
\setstretch{.7}
{\PaliGlossA{kosohite ca vatthaguyhe pahūtajivhatāya ca.}}\\
\begin{addmargin}[1em]{2em}
\setstretch{.5}
{\PaliGlossB{Then Brahmāyu addressed the Buddha in verse:}}\\
\end{addmargin}
\end{absolutelynopagebreak}

\begin{absolutelynopagebreak}
\setstretch{.7}
{\PaliGlossA{Atha kho brahmāyu brāhmaṇo bhagavantaṃ gāthāhi ajjhabhāsi:}}\\
\begin{addmargin}[1em]{2em}
\setstretch{.5}
{\PaliGlossB{“I have learned of the thirty-two}}\\
\end{addmargin}
\end{absolutelynopagebreak}

\begin{absolutelynopagebreak}
\setstretch{.7}
{\PaliGlossA{“Ye me dvattiṃsāti sutā,}}\\
\begin{addmargin}[1em]{2em}
\setstretch{.5}
{\PaliGlossB{marks of a great man.}}\\
\end{addmargin}
\end{absolutelynopagebreak}

\begin{absolutelynopagebreak}
\setstretch{.7}
{\PaliGlossA{mahāpurisalakkhaṇā;}}\\
\begin{addmargin}[1em]{2em}
\setstretch{.5}
{\PaliGlossB{There are two that I don’t see}}\\
\end{addmargin}
\end{absolutelynopagebreak}

\begin{absolutelynopagebreak}
\setstretch{.7}
{\PaliGlossA{Duve tesaṃ na passāmi,}}\\
\begin{addmargin}[1em]{2em}
\setstretch{.5}
{\PaliGlossB{on the body of the ascetic Gotama.}}\\
\end{addmargin}
\end{absolutelynopagebreak}

\begin{absolutelynopagebreak}
\setstretch{.7}
{\PaliGlossA{bhoto kāyasmiṃ gotama.}}\\
\begin{addmargin}[1em]{2em}
\setstretch{.5}
{\PaliGlossB{Are the private parts retracted,}}\\
\end{addmargin}
\end{absolutelynopagebreak}

\begin{absolutelynopagebreak}
\setstretch{.7}
{\PaliGlossA{Kacci kosohitaṃ bhoto,}}\\
\begin{addmargin}[1em]{2em}
\setstretch{.5}
{\PaliGlossB{O supreme person?}}\\
\end{addmargin}
\end{absolutelynopagebreak}

\begin{absolutelynopagebreak}
\setstretch{.7}
{\PaliGlossA{vatthaguyhaṃ naruttama;}}\\
\begin{addmargin}[1em]{2em}
\setstretch{.5}
{\PaliGlossB{Though called by a word of the feminine gender,}}\\
\end{addmargin}
\end{absolutelynopagebreak}

\begin{absolutelynopagebreak}
\setstretch{.7}
{\PaliGlossA{Nārīsamānasavhayā,}}\\
\begin{addmargin}[1em]{2em}
\setstretch{.5}
{\PaliGlossB{# NOTE: Read narassikā}}\\
\end{addmargin}
\end{absolutelynopagebreak}

\begin{absolutelynopagebreak}
\setstretch{.7}
{\PaliGlossA{kacci jivhā na dassakā.}}\\
\begin{addmargin}[1em]{2em}
\setstretch{.5}
{\PaliGlossB{Perhaps your tongue is large,}}\\
\end{addmargin}
\end{absolutelynopagebreak}

\begin{absolutelynopagebreak}
\setstretch{.7}
{\PaliGlossA{Kacci pahūtajivhosi,}}\\
\begin{addmargin}[1em]{2em}
\setstretch{.5}
{\PaliGlossB{as we have been informed.}}\\
\end{addmargin}
\end{absolutelynopagebreak}

\begin{absolutelynopagebreak}
\setstretch{.7}
{\PaliGlossA{yathā taṃ jāniyāmase;}}\\
\begin{addmargin}[1em]{2em}
\setstretch{.5}
{\PaliGlossB{# NOTE: tanuka as PTS and accepted by BB hardly makes sense here.}}\\
\end{addmargin}
\end{absolutelynopagebreak}

\begin{absolutelynopagebreak}
\setstretch{.7}
{\PaliGlossA{Ninnāmayetaṃ pahūtaṃ,}}\\
\begin{addmargin}[1em]{2em}
\setstretch{.5}
{\PaliGlossB{and so, O hermit, dispel my doubt.}}\\
\end{addmargin}
\end{absolutelynopagebreak}

\begin{absolutelynopagebreak}
\setstretch{.7}
{\PaliGlossA{kaṅkhaṃ vinaya no ise.}}\\
\begin{addmargin}[1em]{2em}
\setstretch{.5}
{\PaliGlossB{For my welfare and benefit in this life,}}\\
\end{addmargin}
\end{absolutelynopagebreak}

\begin{absolutelynopagebreak}
\setstretch{.7}
{\PaliGlossA{Diṭṭhadhammahitatthāya,}}\\
\begin{addmargin}[1em]{2em}
\setstretch{.5}
{\PaliGlossB{and happiness in the next.}}\\
\end{addmargin}
\end{absolutelynopagebreak}

\begin{absolutelynopagebreak}
\setstretch{.7}
{\PaliGlossA{samparāyasukhāya ca;}}\\
\begin{addmargin}[1em]{2em}
\setstretch{.5}
{\PaliGlossB{And I ask you to grant the opportunity}}\\
\end{addmargin}
\end{absolutelynopagebreak}

\begin{absolutelynopagebreak}
\setstretch{.7}
{\PaliGlossA{Katāvakāsā pucchāma,}}\\
\begin{addmargin}[1em]{2em}
\setstretch{.5}
{\PaliGlossB{to ask whatever I desire.”}}\\
\end{addmargin}
\end{absolutelynopagebreak}

\begin{absolutelynopagebreak}
\setstretch{.7}
{\PaliGlossA{yaṃ kiñci abhipatthitan”ti.}}\\
\begin{addmargin}[1em]{2em}
\setstretch{.5}
{\PaliGlossB{    -}}\\
\end{addmargin}
\end{absolutelynopagebreak}

\begin{absolutelynopagebreak}
\setstretch{.7}
{\PaliGlossA{    -}}\\
\begin{addmargin}[1em]{2em}
\setstretch{.5}
{\PaliGlossB{Then the Buddha thought,}}\\
\end{addmargin}
\end{absolutelynopagebreak}

\begin{absolutelynopagebreak}
\setstretch{.7}
{\PaliGlossA{30. Atha kho bhagavato etadahosi:}}\\
\begin{addmargin}[1em]{2em}
\setstretch{.5}
{\PaliGlossB{“Brahmāyu sees all the marks except for two,}}\\
\end{addmargin}
\end{absolutelynopagebreak}

\begin{absolutelynopagebreak}
\setstretch{.7}
{\PaliGlossA{“passati kho me ayaṃ brahmāyu brāhmaṇo dvattiṃsamahāpurisalakkhaṇāni, yebhuyyena ṭhapetvā dve.}}\\
\begin{addmargin}[1em]{2em}
\setstretch{.5}
{\PaliGlossB{which he has doubts about:}}\\
\end{addmargin}
\end{absolutelynopagebreak}

\begin{absolutelynopagebreak}
\setstretch{.7}
{\PaliGlossA{Dvīsu mahāpurisalakkhaṇesu kaṅkhati vicikicchati nādhimuccati na sampasīdati—}}\\
\begin{addmargin}[1em]{2em}
\setstretch{.5}
{\PaliGlossB{whether the private parts are retracted, and the largeness of the tongue.”}}\\
\end{addmargin}
\end{absolutelynopagebreak}

\begin{absolutelynopagebreak}
\setstretch{.7}
{\PaliGlossA{kosohite ca vatthaguyhe pahūtajivhatāya cā”ti.}}\\
\begin{addmargin}[1em]{2em}
\setstretch{.5}
{\PaliGlossB{So the Buddha used his psychic power to will that Brahmāyu would see his retracted private parts.}}\\
\end{addmargin}
\end{absolutelynopagebreak}

\begin{absolutelynopagebreak}
\setstretch{.7}
{\PaliGlossA{Atha kho bhagavā tathārūpaṃ iddhābhisaṅkhāraṃ abhisaṅkhāsi yathā addasa brahmāyu brāhmaṇo bhagavato kosohitaṃ vatthaguyhaṃ.}}\\
\begin{addmargin}[1em]{2em}
\setstretch{.5}
{\PaliGlossB{And he stuck out his tongue and stroked back and forth on his ear holes and nostrils, and covered his entire forehead with his tongue.}}\\
\end{addmargin}
\end{absolutelynopagebreak}

\begin{absolutelynopagebreak}
\setstretch{.7}
{\PaliGlossA{Atha kho bhagavā jivhaṃ ninnāmetvā ubhopi kaṇṇasotāni anumasi paṭimasi; ubhopi nāsikasotāni anumasi paṭimasi; kevalampi nalāṭamaṇḍalaṃ jivhāya chādesi.}}\\
\begin{addmargin}[1em]{2em}
\setstretch{.5}
{\PaliGlossB{Then the Buddha replied to Brahmāyu in verse:}}\\
\end{addmargin}
\end{absolutelynopagebreak}

\begin{absolutelynopagebreak}
\setstretch{.7}
{\PaliGlossA{Atha kho bhagavā brahmāyuṃ brāhmaṇaṃ gāthāhi paccabhāsi:}}\\
\begin{addmargin}[1em]{2em}
\setstretch{.5}
{\PaliGlossB{    -}}\\
\end{addmargin}
\end{absolutelynopagebreak}

\begin{absolutelynopagebreak}
\setstretch{.7}
{\PaliGlossA{    -}}\\
\begin{addmargin}[1em]{2em}
\setstretch{.5}
{\PaliGlossB{“The thirty-two marks of a great man}}\\
\end{addmargin}
\end{absolutelynopagebreak}

\begin{absolutelynopagebreak}
\setstretch{.7}
{\PaliGlossA{31. “Ye te dvattiṃsāti sutā,}}\\
\begin{addmargin}[1em]{2em}
\setstretch{.5}
{\PaliGlossB{that you have learned}}\\
\end{addmargin}
\end{absolutelynopagebreak}

\begin{absolutelynopagebreak}
\setstretch{.7}
{\PaliGlossA{mahāpurisalakkhaṇā;}}\\
\begin{addmargin}[1em]{2em}
\setstretch{.5}
{\PaliGlossB{are all found on my body:}}\\
\end{addmargin}
\end{absolutelynopagebreak}

\begin{absolutelynopagebreak}
\setstretch{.7}
{\PaliGlossA{Sabbe te mama kāyasmiṃ,}}\\
\begin{addmargin}[1em]{2em}
\setstretch{.5}
{\PaliGlossB{so do not doubt, brahmin.}}\\
\end{addmargin}
\end{absolutelynopagebreak}

\begin{absolutelynopagebreak}
\setstretch{.7}
{\PaliGlossA{mā te kaṅkhāhu brāhmaṇa.}}\\
\begin{addmargin}[1em]{2em}
\setstretch{.5}
{\PaliGlossB{I have known what should be known,}}\\
\end{addmargin}
\end{absolutelynopagebreak}

\begin{absolutelynopagebreak}
\setstretch{.7}
{\PaliGlossA{Abhiññeyyaṃ abhiññātaṃ,}}\\
\begin{addmargin}[1em]{2em}
\setstretch{.5}
{\PaliGlossB{and developed what should be developed,}}\\
\end{addmargin}
\end{absolutelynopagebreak}

\begin{absolutelynopagebreak}
\setstretch{.7}
{\PaliGlossA{bhāvetabbañca bhāvitaṃ;}}\\
\begin{addmargin}[1em]{2em}
\setstretch{.5}
{\PaliGlossB{and given up what should be given up:}}\\
\end{addmargin}
\end{absolutelynopagebreak}

\begin{absolutelynopagebreak}
\setstretch{.7}
{\PaliGlossA{Pahātabbaṃ pahīnaṃ me,}}\\
\begin{addmargin}[1em]{2em}
\setstretch{.5}
{\PaliGlossB{and so, brahmin, I am a Buddha.}}\\
\end{addmargin}
\end{absolutelynopagebreak}

\begin{absolutelynopagebreak}
\setstretch{.7}
{\PaliGlossA{tasmā buddhosmi brāhmaṇa.}}\\
\begin{addmargin}[1em]{2em}
\setstretch{.5}
{\PaliGlossB{For your welfare and benefit in this life,}}\\
\end{addmargin}
\end{absolutelynopagebreak}

\begin{absolutelynopagebreak}
\setstretch{.7}
{\PaliGlossA{Diṭṭhadhammahitatthāya,}}\\
\begin{addmargin}[1em]{2em}
\setstretch{.5}
{\PaliGlossB{and happiness in the next:}}\\
\end{addmargin}
\end{absolutelynopagebreak}

\begin{absolutelynopagebreak}
\setstretch{.7}
{\PaliGlossA{samparāyasukhāya ca;}}\\
\begin{addmargin}[1em]{2em}
\setstretch{.5}
{\PaliGlossB{I grant you the opportunity}}\\
\end{addmargin}
\end{absolutelynopagebreak}

\begin{absolutelynopagebreak}
\setstretch{.7}
{\PaliGlossA{Katāvakāso pucchassu,}}\\
\begin{addmargin}[1em]{2em}
\setstretch{.5}
{\PaliGlossB{to ask whatever you desire.”}}\\
\end{addmargin}
\end{absolutelynopagebreak}

\begin{absolutelynopagebreak}
\setstretch{.7}
{\PaliGlossA{yaṃ kiñci abhipatthitan”ti.}}\\
\begin{addmargin}[1em]{2em}
\setstretch{.5}
{\PaliGlossB{    -}}\\
\end{addmargin}
\end{absolutelynopagebreak}

\begin{absolutelynopagebreak}
\setstretch{.7}
{\PaliGlossA{    -}}\\
\begin{addmargin}[1em]{2em}
\setstretch{.5}
{\PaliGlossB{Then Brahmāyu thought:}}\\
\end{addmargin}
\end{absolutelynopagebreak}

\begin{absolutelynopagebreak}
\setstretch{.7}
{\PaliGlossA{32. Atha kho brahmāyussa brāhmaṇassa etadahosi:}}\\
\begin{addmargin}[1em]{2em}
\setstretch{.5}
{\PaliGlossB{“My request has been granted.}}\\
\end{addmargin}
\end{absolutelynopagebreak}

\begin{absolutelynopagebreak}
\setstretch{.7}
{\PaliGlossA{“katāvakāso khomhi samaṇena gotamena.}}\\
\begin{addmargin}[1em]{2em}
\setstretch{.5}
{\PaliGlossB{Should I ask him about}}\\
\end{addmargin}
\end{absolutelynopagebreak}

\begin{absolutelynopagebreak}
\setstretch{.7}
{\PaliGlossA{Kiṃ nu kho ahaṃ samaṇaṃ gotamaṃ puccheyyaṃ:}}\\
\begin{addmargin}[1em]{2em}
\setstretch{.5}
{\PaliGlossB{what is beneficial in this life or the next?”}}\\
\end{addmargin}
\end{absolutelynopagebreak}

\begin{absolutelynopagebreak}
\setstretch{.7}
{\PaliGlossA{‘diṭṭhadhammikaṃ vā atthaṃ samparāyikaṃ vā’”ti.}}\\
\begin{addmargin}[1em]{2em}
\setstretch{.5}
{\PaliGlossB{Then he thought,}}\\
\end{addmargin}
\end{absolutelynopagebreak}

\begin{absolutelynopagebreak}
\setstretch{.7}
{\PaliGlossA{Atha kho brahmāyussa brāhmaṇassa etadahosi:}}\\
\begin{addmargin}[1em]{2em}
\setstretch{.5}
{\PaliGlossB{“I’m well versed in the benefits that apply to this life,}}\\
\end{addmargin}
\end{absolutelynopagebreak}

\begin{absolutelynopagebreak}
\setstretch{.7}
{\PaliGlossA{“kusalo kho ahaṃ diṭṭhadhammikānaṃ atthānaṃ.}}\\
\begin{addmargin}[1em]{2em}
\setstretch{.5}
{\PaliGlossB{and others ask me about this.}}\\
\end{addmargin}
\end{absolutelynopagebreak}

\begin{absolutelynopagebreak}
\setstretch{.7}
{\PaliGlossA{Aññepi maṃ diṭṭhadhammikaṃ atthaṃ pucchanti.}}\\
\begin{addmargin}[1em]{2em}
\setstretch{.5}
{\PaliGlossB{Why don’t I ask the ascetic Gotama about the benefit that specifically applies to lives to come?”}}\\
\end{addmargin}
\end{absolutelynopagebreak}

\begin{absolutelynopagebreak}
\setstretch{.7}
{\PaliGlossA{Yannūnāhaṃ samaṇaṃ gotamaṃ samparāyikaṃyeva atthaṃ puccheyyan”ti.}}\\
\begin{addmargin}[1em]{2em}
\setstretch{.5}
{\PaliGlossB{So Brahmāyu addressed the Buddha in verse:}}\\
\end{addmargin}
\end{absolutelynopagebreak}

\begin{absolutelynopagebreak}
\setstretch{.7}
{\PaliGlossA{Atha kho brahmāyu brāhmaṇo bhagavantaṃ gāthāhi ajjhabhāsi:}}\\
\begin{addmargin}[1em]{2em}
\setstretch{.5}
{\PaliGlossB{“How do you become a brahmin?}}\\
\end{addmargin}
\end{absolutelynopagebreak}

\begin{absolutelynopagebreak}
\setstretch{.7}
{\PaliGlossA{“Kathaṃ kho brāhmaṇo hoti,}}\\
\begin{addmargin}[1em]{2em}
\setstretch{.5}
{\PaliGlossB{And how do you become a knowledge master?}}\\
\end{addmargin}
\end{absolutelynopagebreak}

\begin{absolutelynopagebreak}
\setstretch{.7}
{\PaliGlossA{kathaṃ bhavati vedagū;}}\\
\begin{addmargin}[1em]{2em}
\setstretch{.5}
{\PaliGlossB{How a master of the three knowledges?}}\\
\end{addmargin}
\end{absolutelynopagebreak}

\begin{absolutelynopagebreak}
\setstretch{.7}
{\PaliGlossA{Tevijjo bho kathaṃ hoti,}}\\
\begin{addmargin}[1em]{2em}
\setstretch{.5}
{\PaliGlossB{And how is one called a scholar?}}\\
\end{addmargin}
\end{absolutelynopagebreak}

\begin{absolutelynopagebreak}
\setstretch{.7}
{\PaliGlossA{sotthiyo kinti vuccati.}}\\
\begin{addmargin}[1em]{2em}
\setstretch{.5}
{\PaliGlossB{How do you become a perfected one?}}\\
\end{addmargin}
\end{absolutelynopagebreak}

\begin{absolutelynopagebreak}
\setstretch{.7}
{\PaliGlossA{Arahaṃ bho kathaṃ hoti,}}\\
\begin{addmargin}[1em]{2em}
\setstretch{.5}
{\PaliGlossB{And how a consummate one?}}\\
\end{addmargin}
\end{absolutelynopagebreak}

\begin{absolutelynopagebreak}
\setstretch{.7}
{\PaliGlossA{kathaṃ bhavati kevalī;}}\\
\begin{addmargin}[1em]{2em}
\setstretch{.5}
{\PaliGlossB{How do you become a sage?}}\\
\end{addmargin}
\end{absolutelynopagebreak}

\begin{absolutelynopagebreak}
\setstretch{.7}
{\PaliGlossA{Muni ca bho kathaṃ hoti,}}\\
\begin{addmargin}[1em]{2em}
\setstretch{.5}
{\PaliGlossB{And how is one declared to be awakened?”}}\\
\end{addmargin}
\end{absolutelynopagebreak}

\begin{absolutelynopagebreak}
\setstretch{.7}
{\PaliGlossA{buddho kinti pavuccatī”ti.}}\\
\begin{addmargin}[1em]{2em}
\setstretch{.5}
{\PaliGlossB{    -}}\\
\end{addmargin}
\end{absolutelynopagebreak}

\begin{absolutelynopagebreak}
\setstretch{.7}
{\PaliGlossA{    -}}\\
\begin{addmargin}[1em]{2em}
\setstretch{.5}
{\PaliGlossB{Then the Buddha replied to Brahmāyu in verse:}}\\
\end{addmargin}
\end{absolutelynopagebreak}

\begin{absolutelynopagebreak}
\setstretch{.7}
{\PaliGlossA{33. Atha kho bhagavā brahmāyuṃ brāhmaṇaṃ gāthāhi paccabhāsi:}}\\
\begin{addmargin}[1em]{2em}
\setstretch{.5}
{\PaliGlossB{“One who knows their past lives,}}\\
\end{addmargin}
\end{absolutelynopagebreak}

\begin{absolutelynopagebreak}
\setstretch{.7}
{\PaliGlossA{“Pubbenivāsaṃ yo vedi,}}\\
\begin{addmargin}[1em]{2em}
\setstretch{.5}
{\PaliGlossB{and sees heaven and places of loss,}}\\
\end{addmargin}
\end{absolutelynopagebreak}

\begin{absolutelynopagebreak}
\setstretch{.7}
{\PaliGlossA{saggāpāyañca passati;}}\\
\begin{addmargin}[1em]{2em}
\setstretch{.5}
{\PaliGlossB{and has attained the end of rebirth:}}\\
\end{addmargin}
\end{absolutelynopagebreak}

\begin{absolutelynopagebreak}
\setstretch{.7}
{\PaliGlossA{Atho jātikkhayaṃ patto,}}\\
\begin{addmargin}[1em]{2em}
\setstretch{.5}
{\PaliGlossB{that sage has perfect insight.}}\\
\end{addmargin}
\end{absolutelynopagebreak}

\begin{absolutelynopagebreak}
\setstretch{.7}
{\PaliGlossA{abhiññā vosito muni.}}\\
\begin{addmargin}[1em]{2em}
\setstretch{.5}
{\PaliGlossB{They know their mind is pure,}}\\
\end{addmargin}
\end{absolutelynopagebreak}

\begin{absolutelynopagebreak}
\setstretch{.7}
{\PaliGlossA{Cittaṃ visuddhaṃ jānāti,}}\\
\begin{addmargin}[1em]{2em}
\setstretch{.5}
{\PaliGlossB{completely freed from greed;}}\\
\end{addmargin}
\end{absolutelynopagebreak}

\begin{absolutelynopagebreak}
\setstretch{.7}
{\PaliGlossA{muttaṃ rāgehi sabbaso;}}\\
\begin{addmargin}[1em]{2em}
\setstretch{.5}
{\PaliGlossB{they’ve given up birth and death,}}\\
\end{addmargin}
\end{absolutelynopagebreak}

\begin{absolutelynopagebreak}
\setstretch{.7}
{\PaliGlossA{Pahīnajātimaraṇo,}}\\
\begin{addmargin}[1em]{2em}
\setstretch{.5}
{\PaliGlossB{and have completed the spiritual journey.}}\\
\end{addmargin}
\end{absolutelynopagebreak}

\begin{absolutelynopagebreak}
\setstretch{.7}
{\PaliGlossA{brahmacariyassa kevalī;}}\\
\begin{addmargin}[1em]{2em}
\setstretch{.5}
{\PaliGlossB{Gone beyond all things,}}\\
\end{addmargin}
\end{absolutelynopagebreak}

\begin{absolutelynopagebreak}
\setstretch{.7}
{\PaliGlossA{Pāragū sabbadhammānaṃ,}}\\
\begin{addmargin}[1em]{2em}
\setstretch{.5}
{\PaliGlossB{such a one is declared to be awakened.”}}\\
\end{addmargin}
\end{absolutelynopagebreak}

\begin{absolutelynopagebreak}
\setstretch{.7}
{\PaliGlossA{buddho tādī pavuccatī”ti.}}\\
\begin{addmargin}[1em]{2em}
\setstretch{.5}
{\PaliGlossB{    -}}\\
\end{addmargin}
\end{absolutelynopagebreak}

\begin{absolutelynopagebreak}
\setstretch{.7}
{\PaliGlossA{    -}}\\
\begin{addmargin}[1em]{2em}
\setstretch{.5}
{\PaliGlossB{When he said this, Brahmāyu got up from his seat and arranged his robe on one shoulder. He bowed with his head to the Buddha’s feet, caressing them and covering them with kisses, and pronounced his name:}}\\
\end{addmargin}
\end{absolutelynopagebreak}

\begin{absolutelynopagebreak}
\setstretch{.7}
{\PaliGlossA{34. Evaṃ vutte, brahmāyu brāhmaṇo uṭṭhāyāsanā ekaṃsaṃ uttarāsaṅgaṃ karitvā bhagavato pādesu sirasā nipatitvā bhagavato pādāni mukhena ca paricumbati, pāṇīhi ca parisambāhati, nāmañca sāveti:}}\\
\begin{addmargin}[1em]{2em}
\setstretch{.5}
{\PaliGlossB{“I am the brahmin Brahmāyu, Master Gotama!}}\\
\end{addmargin}
\end{absolutelynopagebreak}

\begin{absolutelynopagebreak}
\setstretch{.7}
{\PaliGlossA{“brahmāyu ahaṃ, bho gotama, brāhmaṇo;}}\\
\begin{addmargin}[1em]{2em}
\setstretch{.5}
{\PaliGlossB{I am the brahmin Brahmāyu!”}}\\
\end{addmargin}
\end{absolutelynopagebreak}

\begin{absolutelynopagebreak}
\setstretch{.7}
{\PaliGlossA{brahmāyu ahaṃ, bho gotama, brāhmaṇo”ti.}}\\
\begin{addmargin}[1em]{2em}
\setstretch{.5}
{\PaliGlossB{    -}}\\
\end{addmargin}
\end{absolutelynopagebreak}

\begin{absolutelynopagebreak}
\setstretch{.7}
{\PaliGlossA{    -}}\\
\begin{addmargin}[1em]{2em}
\setstretch{.5}
{\PaliGlossB{Then that assembly, their minds full of wonder and amazement, thought,}}\\
\end{addmargin}
\end{absolutelynopagebreak}

\begin{absolutelynopagebreak}
\setstretch{.7}
{\PaliGlossA{35. Atha kho sā parisā acchariyabbhutacittajātā ahosi:}}\\
\begin{addmargin}[1em]{2em}
\setstretch{.5}
{\PaliGlossB{“It’s incredible, it’s amazing,}}\\
\end{addmargin}
\end{absolutelynopagebreak}

\begin{absolutelynopagebreak}
\setstretch{.7}
{\PaliGlossA{“acchariyaṃ vata bho, abbhutaṃ vata bho.}}\\
\begin{addmargin}[1em]{2em}
\setstretch{.5}
{\PaliGlossB{that Brahmāyu, who is so well-known and famous, should show the Buddha such utmost devotion.”}}\\
\end{addmargin}
\end{absolutelynopagebreak}

\begin{absolutelynopagebreak}
\setstretch{.7}
{\PaliGlossA{Yatra hi nāmāyaṃ brahmāyu brāhmaṇo ñāto yasassī evarūpaṃ paramanipaccakāraṃ karissatī”ti.}}\\
\begin{addmargin}[1em]{2em}
\setstretch{.5}
{\PaliGlossB{Then the Buddha said to Brahmāyu,}}\\
\end{addmargin}
\end{absolutelynopagebreak}

\begin{absolutelynopagebreak}
\setstretch{.7}
{\PaliGlossA{Atha kho bhagavā brahmāyuṃ brāhmaṇaṃ etadavoca:}}\\
\begin{addmargin}[1em]{2em}
\setstretch{.5}
{\PaliGlossB{“Enough, brahmin. Get up, and sit in your own seat, since your mind has such confidence in me.”}}\\
\end{addmargin}
\end{absolutelynopagebreak}

\begin{absolutelynopagebreak}
\setstretch{.7}
{\PaliGlossA{“alaṃ, brāhmaṇa, uṭṭhaha nisīda tvaṃ sake āsane yato te mayi cittaṃ pasannan”ti.}}\\
\begin{addmargin}[1em]{2em}
\setstretch{.5}
{\PaliGlossB{So Brahmāyu got up and sat in his own seat.}}\\
\end{addmargin}
\end{absolutelynopagebreak}

\begin{absolutelynopagebreak}
\setstretch{.7}
{\PaliGlossA{Atha kho brahmāyu brāhmaṇo uṭṭhahitvā sake āsane nisīdi.}}\\
\begin{addmargin}[1em]{2em}
\setstretch{.5}
{\PaliGlossB{    -}}\\
\end{addmargin}
\end{absolutelynopagebreak}

\begin{absolutelynopagebreak}
\setstretch{.7}
{\PaliGlossA{    -}}\\
\begin{addmargin}[1em]{2em}
\setstretch{.5}
{\PaliGlossB{Then the Buddha taught him step by step, with}}\\
\end{addmargin}
\end{absolutelynopagebreak}

\begin{absolutelynopagebreak}
\setstretch{.7}
{\PaliGlossA{36. Atha kho bhagavā brahmāyussa brāhmaṇassa anupubbiṃ kathaṃ kathesi, seyyathidaṃ—}}\\
\begin{addmargin}[1em]{2em}
\setstretch{.5}
{\PaliGlossB{a talk on giving, ethical conduct, and heaven.}}\\
\end{addmargin}
\end{absolutelynopagebreak}

\begin{absolutelynopagebreak}
\setstretch{.7}
{\PaliGlossA{dānakathaṃ, sīlakathaṃ, saggakathaṃ;}}\\
\begin{addmargin}[1em]{2em}
\setstretch{.5}
{\PaliGlossB{He explained the drawbacks of sensual pleasures, so sordid and corrupt, and the benefit of renunciation.}}\\
\end{addmargin}
\end{absolutelynopagebreak}

\begin{absolutelynopagebreak}
\setstretch{.7}
{\PaliGlossA{kāmānaṃ ādīnavaṃ okāraṃ saṅkilesaṃ nekkhamme ānisaṃsaṃ pakāsesi.}}\\
\begin{addmargin}[1em]{2em}
\setstretch{.5}
{\PaliGlossB{And when the Buddha knew that Brahmāyu’s mind was ready, pliable, rid of hindrances, joyful, and confident he explained the special teaching of the Buddhas:}}\\
\end{addmargin}
\end{absolutelynopagebreak}

\begin{absolutelynopagebreak}
\setstretch{.7}
{\PaliGlossA{Yadā bhagavā aññāsi brahmāyuṃ brāhmaṇaṃ kallacittaṃ muducittaṃ vinīvaraṇacittaṃ udaggacittaṃ pasannacittaṃ, atha yā buddhānaṃ sāmukkaṃsikā dhammadesanā taṃ pakāsesi—}}\\
\begin{addmargin}[1em]{2em}
\setstretch{.5}
{\PaliGlossB{suffering, its origin, its cessation, and the path.}}\\
\end{addmargin}
\end{absolutelynopagebreak}

\begin{absolutelynopagebreak}
\setstretch{.7}
{\PaliGlossA{dukkhaṃ, samudayaṃ, nirodhaṃ, maggaṃ.}}\\
\begin{addmargin}[1em]{2em}
\setstretch{.5}
{\PaliGlossB{Just as a clean cloth rid of stains would properly absorb dye,}}\\
\end{addmargin}
\end{absolutelynopagebreak}

\begin{absolutelynopagebreak}
\setstretch{.7}
{\PaliGlossA{Seyyathāpi nāma suddhaṃ vatthaṃ apagatakāḷakaṃ sammadeva rajanaṃ paṭiggaṇheyya;}}\\
\begin{addmargin}[1em]{2em}
\setstretch{.5}
{\PaliGlossB{in that very seat the stainless, immaculate vision of the Dhamma arose in the brahmin Brahmāyu:}}\\
\end{addmargin}
\end{absolutelynopagebreak}

\begin{absolutelynopagebreak}
\setstretch{.7}
{\PaliGlossA{evameva brahmāyussa brāhmaṇassa tasmiṃyeva āsane virajaṃ vītamalaṃ dhammacakkhuṃ udapādi:}}\\
\begin{addmargin}[1em]{2em}
\setstretch{.5}
{\PaliGlossB{“Everything that has a beginning has an end.”}}\\
\end{addmargin}
\end{absolutelynopagebreak}

\begin{absolutelynopagebreak}
\setstretch{.7}
{\PaliGlossA{“yaṃ kiñci samudayadhammaṃ sabbaṃ taṃ nirodhadhamman”ti.}}\\
\begin{addmargin}[1em]{2em}
\setstretch{.5}
{\PaliGlossB{Then Brahmāyu saw, attained, understood, and fathomed the Dhamma. He went beyond doubt, got rid of indecision, and became self-assured and independent of others regarding the Teacher’s instructions. He said to the Buddha:}}\\
\end{addmargin}
\end{absolutelynopagebreak}

\begin{absolutelynopagebreak}
\setstretch{.7}
{\PaliGlossA{Atha kho brahmāyu brāhmaṇo diṭṭhadhammo pattadhammo viditadhammo pariyogāḷhadhammo tiṇṇavicikiccho vigatakathaṅkatho vesārajjappatto aparappaccayo satthusāsane bhagavantaṃ etadavoca:}}\\
\begin{addmargin}[1em]{2em}
\setstretch{.5}
{\PaliGlossB{    -}}\\
\end{addmargin}
\end{absolutelynopagebreak}

\begin{absolutelynopagebreak}
\setstretch{.7}
{\PaliGlossA{    -}}\\
\begin{addmargin}[1em]{2em}
\setstretch{.5}
{\PaliGlossB{“Excellent, Master Gotama! Excellent!}}\\
\end{addmargin}
\end{absolutelynopagebreak}

\begin{absolutelynopagebreak}
\setstretch{.7}
{\PaliGlossA{37. “abhikkantaṃ, bho gotama, abhikkantaṃ, bho gotama.}}\\
\begin{addmargin}[1em]{2em}
\setstretch{.5}
{\PaliGlossB{As if he were righting the overturned, or revealing the hidden, or pointing out the path to the lost, or lighting a lamp in the dark so people with good eyes can see what’s there, Master Gotama has made the teaching clear in many ways.}}\\
\end{addmargin}
\end{absolutelynopagebreak}

\begin{absolutelynopagebreak}
\setstretch{.7}
{\PaliGlossA{Seyyathāpi, bho gotama, nikkujjitaṃ vā ukkujjeyya, paṭicchannaṃ vā vivareyya, mūḷhassa vā maggaṃ ācikkheyya, andhakāre vā telapajjotaṃ dhāreyya ‘cakkhumanto rūpāni dakkhantī’ti; evamevaṃ bhotā gotamena anekapariyāyena dhammo pakāsito.}}\\
\begin{addmargin}[1em]{2em}
\setstretch{.5}
{\PaliGlossB{I go for refuge to Master Gotama, to the teaching, and to the mendicant Saṅgha.}}\\
\end{addmargin}
\end{absolutelynopagebreak}

\begin{absolutelynopagebreak}
\setstretch{.7}
{\PaliGlossA{Esāhaṃ bhavantaṃ gotamaṃ saraṇaṃ gacchāmi dhammañca bhikkhusaṅghañca.}}\\
\begin{addmargin}[1em]{2em}
\setstretch{.5}
{\PaliGlossB{From this day forth, may Master Gotama remember me as a lay follower who has gone for refuge for life.}}\\
\end{addmargin}
\end{absolutelynopagebreak}

\begin{absolutelynopagebreak}
\setstretch{.7}
{\PaliGlossA{Upāsakaṃ maṃ bhavaṃ gotamo dhāretu ajjatagge pāṇupetaṃ saraṇaṃ gataṃ.}}\\
\begin{addmargin}[1em]{2em}
\setstretch{.5}
{\PaliGlossB{Would you and the mendicant Saṅgha please accept a meal from me tomorrow?”}}\\
\end{addmargin}
\end{absolutelynopagebreak}

\begin{absolutelynopagebreak}
\setstretch{.7}
{\PaliGlossA{Adhivāsetu ca me bhavaṃ gotamo svātanāya bhattaṃ saddhiṃ bhikkhusaṅghenā”ti.}}\\
\begin{addmargin}[1em]{2em}
\setstretch{.5}
{\PaliGlossB{The Buddha consented in silence.}}\\
\end{addmargin}
\end{absolutelynopagebreak}

\begin{absolutelynopagebreak}
\setstretch{.7}
{\PaliGlossA{Adhivāsesi bhagavā tuṇhībhāvena.}}\\
\begin{addmargin}[1em]{2em}
\setstretch{.5}
{\PaliGlossB{Then, knowing that the Buddha had consented, Brahmāyu got up from his seat, bowed, and respectfully circled the Buddha, keeping him on his right, before leaving.}}\\
\end{addmargin}
\end{absolutelynopagebreak}

\begin{absolutelynopagebreak}
\setstretch{.7}
{\PaliGlossA{Atha kho brahmāyu brāhmaṇo bhagavato adhivāsanaṃ viditvā uṭṭhāyāsanā bhagavantaṃ abhivādetvā padakkhiṇaṃ katvā pakkāmi.}}\\
\begin{addmargin}[1em]{2em}
\setstretch{.5}
{\PaliGlossB{    -}}\\
\end{addmargin}
\end{absolutelynopagebreak}

\begin{absolutelynopagebreak}
\setstretch{.7}
{\PaliGlossA{    -}}\\
\begin{addmargin}[1em]{2em}
\setstretch{.5}
{\PaliGlossB{And when the night had passed Brahmāyu had a variety of delicious foods prepared in his own home. Then he had the Buddha informed of the time, saying,}}\\
\end{addmargin}
\end{absolutelynopagebreak}

\begin{absolutelynopagebreak}
\setstretch{.7}
{\PaliGlossA{38. Atha kho brahmāyu brāhmaṇo tassā rattiyā accayena sake nivesane paṇītaṃ khādanīyaṃ bhojanīyaṃ paṭiyādāpetvā bhagavato kālaṃ ārocāpesi:}}\\
\begin{addmargin}[1em]{2em}
\setstretch{.5}
{\PaliGlossB{“Itʼs time, Master Gotama, the meal is ready.”}}\\
\end{addmargin}
\end{absolutelynopagebreak}

\begin{absolutelynopagebreak}
\setstretch{.7}
{\PaliGlossA{“kālo, bho gotama, niṭṭhitaṃ bhattan”ti.}}\\
\begin{addmargin}[1em]{2em}
\setstretch{.5}
{\PaliGlossB{Then the Buddha robed up in the morning and, taking his bowl and robe, went to the home of the brahmin Brahmāyu, where he sat on the seat spread out, together with the Saṅgha of mendicants.}}\\
\end{addmargin}
\end{absolutelynopagebreak}

\begin{absolutelynopagebreak}
\setstretch{.7}
{\PaliGlossA{Atha kho bhagavā pubbaṇhasamayaṃ nivāsetvā pattacīvaramādāya yena brahmāyussa brāhmaṇassa nivesanaṃ tenupasaṅkami; upasaṅkamitvā paññatte āsane nisīdi saddhiṃ bhikkhusaṅghena.}}\\
\begin{addmargin}[1em]{2em}
\setstretch{.5}
{\PaliGlossB{Brahmāyu served and satisfied the mendicant Saṅgha headed by the Buddha with his own hands with a variety of delicious foods.}}\\
\end{addmargin}
\end{absolutelynopagebreak}

\begin{absolutelynopagebreak}
\setstretch{.7}
{\PaliGlossA{Atha kho brahmāyu brāhmaṇo sattāhaṃ buddhappamukhaṃ bhikkhusaṅghaṃ paṇītena khādanīyena bhojanīyena sahatthā santappesi sampavāresi.}}\\
\begin{addmargin}[1em]{2em}
\setstretch{.5}
{\PaliGlossB{    -}}\\
\end{addmargin}
\end{absolutelynopagebreak}

\begin{absolutelynopagebreak}
\setstretch{.7}
{\PaliGlossA{    -}}\\
\begin{addmargin}[1em]{2em}
\setstretch{.5}
{\PaliGlossB{A week later, the Buddha departed to wander in the Videhan lands.}}\\
\end{addmargin}
\end{absolutelynopagebreak}

\begin{absolutelynopagebreak}
\setstretch{.7}
{\PaliGlossA{39. Atha kho bhagavā tassa sattāhassa accayena videhesu cārikaṃ pakkāmi.}}\\
\begin{addmargin}[1em]{2em}
\setstretch{.5}
{\PaliGlossB{Not long after the Buddha left, Brahmāyu passed away.}}\\
\end{addmargin}
\end{absolutelynopagebreak}

\begin{absolutelynopagebreak}
\setstretch{.7}
{\PaliGlossA{Atha kho brahmāyu brāhmaṇo acirapakkantassa bhagavato kālamakāsi.}}\\
\begin{addmargin}[1em]{2em}
\setstretch{.5}
{\PaliGlossB{Then several mendicants went up to the Buddha, bowed, sat down to one side, and said to him,}}\\
\end{addmargin}
\end{absolutelynopagebreak}

\begin{absolutelynopagebreak}
\setstretch{.7}
{\PaliGlossA{Atha kho sambahulā bhikkhū yena bhagavā tenupasaṅkamiṃsu; upasaṅkamitvā bhagavantaṃ abhivādetvā ekamantaṃ nisīdiṃsu. Ekamantaṃ nisinnā kho te bhikkhū bhagavantaṃ etadavocuṃ:}}\\
\begin{addmargin}[1em]{2em}
\setstretch{.5}
{\PaliGlossB{“Sir, Brahmāyu has passed away.}}\\
\end{addmargin}
\end{absolutelynopagebreak}

\begin{absolutelynopagebreak}
\setstretch{.7}
{\PaliGlossA{“brahmāyu, bhante, brāhmaṇo kālaṅkato.}}\\
\begin{addmargin}[1em]{2em}
\setstretch{.5}
{\PaliGlossB{Where has he been reborn in his next life?”}}\\
\end{addmargin}
\end{absolutelynopagebreak}

\begin{absolutelynopagebreak}
\setstretch{.7}
{\PaliGlossA{Tassa kā gati, ko abhisamparāyo”ti?}}\\
\begin{addmargin}[1em]{2em}
\setstretch{.5}
{\PaliGlossB{“Mendicants, the brahmin Brahmāyu was astute. He practiced in line with the teachings, and did not trouble me about the teachings.}}\\
\end{addmargin}
\end{absolutelynopagebreak}

\begin{absolutelynopagebreak}
\setstretch{.7}
{\PaliGlossA{“Paṇḍito, bhikkhave, brahmāyu brāhmaṇo paccapādi dhammassānudhammaṃ, na ca maṃ dhammādhikaraṇaṃ vihesesi.}}\\
\begin{addmargin}[1em]{2em}
\setstretch{.5}
{\PaliGlossB{With the ending of the five lower fetters, he’s been reborn spontaneously and will become extinguished there, not liable to return from that world.”}}\\
\end{addmargin}
\end{absolutelynopagebreak}

\begin{absolutelynopagebreak}
\setstretch{.7}
{\PaliGlossA{Brahmāyu, bhikkhave, brāhmaṇo pañcannaṃ orambhāgiyānaṃ saṃyojanānaṃ parikkhayā opapātiko hoti, tattha parinibbāyī, anāvattidhammo tasmā lokā”ti.}}\\
\begin{addmargin}[1em]{2em}
\setstretch{.5}
{\PaliGlossB{That is what the Buddha said.}}\\
\end{addmargin}
\end{absolutelynopagebreak}

\begin{absolutelynopagebreak}
\setstretch{.7}
{\PaliGlossA{Idamavoca bhagavā.}}\\
\begin{addmargin}[1em]{2em}
\setstretch{.5}
{\PaliGlossB{Satisfied, the mendicants were happy with what the Buddha said.}}\\
\end{addmargin}
\end{absolutelynopagebreak}

\begin{absolutelynopagebreak}
\setstretch{.7}
{\PaliGlossA{Attamanā te bhikkhū bhagavato bhāsitaṃ abhinandunti.}}\\
\begin{addmargin}[1em]{2em}
\setstretch{.5}
{\PaliGlossB{    -}}\\
\end{addmargin}
\end{absolutelynopagebreak}
