\documentclass[12pt]{article}
\usepackage[xetex, margin=55pt,landscape, ansibpaper]{geometry}
% ansibpaper is equivalent to 2 page letter
\usepackage{titlesec}
\usepackage{paracol}
\usepackage{expex}
\usepackage[dvipsnames]{xcolor}
\usepackage{fontspec}
\usepackage{setspace}
\usepackage{xfrac}
%\usepackage{tgpagella}
%\usepackage[QX]{fontenc}
%\setcolumnwidth{310pt/60pt,310pt}
\columnratio{0.53, 0.53}
\sloppy
\raggedright

\widowpenalties 1 1000
\raggedbottom

%\newcommand{\vsp}{\vspace{2mm}}

%defining colors for grammatical highlighting
\newcommand{\EnglishColumn}[1]{\itshape\footnotesize{#1}}

%defining colors for grammatical highlighting
\input{PM_ColorDefintions.tex}
%\input{PM_BWDefintions.tex}
%\newcommand{\GMU}[1]{{\textsc{#1}}}

\newcommand{\NUL}[1]{\textcolor{Apricot}{#1}}
%noun
\newcommand{\NOM}[1]{\textcolor{Apricot}{#1}}
\newcommand{\ACC}[1]{\textcolor{Apricot}{#1}}
\newcommand{\INS}[1]{\textcolor{Apricot}{#1}}
\newcommand{\DAT}[1]{\textcolor{Apricot}{#1}}
\newcommand{\ABL}[1]{\textcolor{Apricot}{#1}}
\newcommand{\GEN}[1]{\textcolor{Apricot}{#1}}
\newcommand{\LOC}[1]{\textcolor{Apricot}{#1}}
\newcommand{\VOC}[1]{\textcolor{Apricot}{#1}}
%\newcommand{\ACC}[1]{{\small\textsc{#1}}} how to adjust font size

%verb
\newcommand{\ABS}[1]{\textcolor{Apricot}{#1}}
\newcommand{\OPT}[1]{\textcolor{Apricot}{#1}}
\newcommand{\FUT}[1]{\textcolor{Apricot}{#1}}
\newcommand{\IMP}[1]{\textcolor{Apricot}{#1}}
\newcommand{\IND}[1]{\textcolor{Apricot}{#1}}
\newcommand{\INF}[1]{\textcolor{Apricot}{#1}}
\newcommand{\PRES}[1]{\textcolor{Apricot}{#1}}
\newcommand{\PRESIND}[1]{\textcolor{Apricot}{#1}}
\newcommand{\AOR}[1]{\textcolor{Apricot}{#1}}
\newcommand{\PAST}[1]{\textcolor{Apricot}{#1}}
\newcommand{\PASS}[1]{\textcolor{Apricot}{#1}} % only entry is for viyyati, needs to be revisited


%Misc
\newcommand{\ADJ}[1]{\textcolor{Apricot}{#1}}
\newcommand{\ADV}[1]{\textcolor{Apricot}{#1}}
\newcommand{\PERS}[1]{\textcolor{Apricot}{#1}}
\newcommand{\PRO}[1]{\textcolor{Apricot}{#1}}
\newcommand{\PART}[1]{\textcolor{Apricot}{#1}}
\newcommand{\EMPH}[1]{\textcolor{Apricot}{#1}}
\newcommand{\NEG}[1]{\textcolor{Apricot}{#1}}
\newcommand{\NUM}[1]{\textcolor{Apricot}{#1}}
\newcommand{\ORD}[1]{\textcolor{Apricot}{#1}}
\newcommand{\INDE}[1]{\textcolor{Apricot}{#1}}
\newcommand{\TBD}[1]{\textcolor{Apricot}{#1}} %all defined as 'Apricot', that which is black is undefined. 

%Glossing Definitions
\lingset{glstyle=nlevel,%sets the compile method for ExPex to the nlevel style (alternates words with []) as opposed to the wrap style uses gla glb separted lines
	glhangindent=0pt,% sets the indentation to the entire gloss after the first set of lines
	glossbreaking=true, %Allows for Glosses to break across pages
	glwordalign=left, %sets allignment on pairs of words, center is other option
	glnabovelineskip={,-.4em}, %sets the space above the glossing {gla,glb,...}
	extraglskip= -.1em,
	glspace=.3em plus .2em minus .1em} %interword space default .5em plus.4em minus.15em

%\newcommand{\GMU}[1]{{\textsc{#1}}}

\begin{document}
\begin{paracol}{2}
\begin{column}



{\EnglishColumn

\begin{doublespace}
The Disciplinary Code of the Bhikkhu
\end{doublespace}}

\switchcolumn


\begin{flushleft}
\begingl
 BHIKKHUPĀṬIMOKKHAṀ[]
\endgl
\switchcolumn*
\end{flushleft}


{\EnglishColumn

\begin{doublespace}
Homage to the Blessed, Noble, and Perfectly Enlightened One.
(3 times)
\end{doublespace}}

\switchcolumn


\begin{flushleft}
\begingl
 Namo[] tassa[of that-\GEN{\GMU{gen-sg}}] bhagavato[blessed one-\GEN{\GMU{gen-sg}}] arahato[] sammāsambuddhassa.[] (tikkhattuṁ)[]
\endgl
\switchcolumn*
\end{flushleft}


{\EnglishColumn

\begin{doublespace}
Venerable Sir, let the Community listen to me! Today is a fifteenth (day) Observance. If it is suitable to the Community , (then) the Community should do the Observance (and) should recite the Disciplinary Code.
\end{doublespace}}

\switchcolumn


\begin{flushleft}
\begingl
 Suṇātu[listen-\IMP{\GMU{3-sg-imp}}] me[me-\DAT{\GMU{dat-sg}}] bhante[venerable sir-\VOC{\GMU{voc-sg}}] (āvuso)[] saṅgho.[community-\NOM{\GMU{nom-sg}}] Ajj’uposatho[] paṇṇaraso[15th-\ADJ{\GMU{adj}}] (cātuddaso).[] Yadi[if-\IND{\GMU{ind}}] saṅghassa[community-\DAT{\GMU{dat-sg}}] pattakallaṁ,[suitable-\NOM{\GMU{nom-sg-n}}] saṅgho[community-\NOM{\GMU{nom-sg}}] uposathaṁ[observance-\ACC{\GMU{acc-sg}}] kareyya,[do-\OPT{\GMU{3-sg-opt}}] pāṭimokkhaṁ[disciplinary code-\ACC{\GMU{acc-sg-n}}] uddiseyya.[recite-\OPT{\GMU{3-sg-opt}}]
\endgl
\switchcolumn*
\end{flushleft}


{\EnglishColumn

\begin{doublespace}
What is the preliminary for the Community? Venerables, announce the purity, (for) I shall recite the Disciplinary Code. Let us all (who are) present listen to it carefully (and) let us take it to mind.
\end{doublespace}}

\switchcolumn


\begin{flushleft}
\begingl
 Kiṁ[what-\NUL{\GMU{}}] saṅghassa[community-\DAT{\GMU{dat-sg}}] pubbakiccaṁ?[before.duty-\ACC{\GMU{acc-sg-n}}] Pārisuddhiṁ[purity-\ACC{\GMU{acc-sg-f}}] āyasmanto[Ven.-\VOC{\GMU{voc-pl}}] ārocetha.[announce-\IMP{\GMU{2-pl-imp}}] Pāṭimokkhaṁ[disciplinary code-\ACC{\GMU{acc-sg-n}}] uddisissāmi.[recite-\FUT{\GMU{1-sg-fut}}] Taṁ[that-\ACC{\GMU{acc-sg}}] sabbeva[] santā[exist-\PRES{\GMU{pres-part}}] sādhukaṁ[well-\ADV{\GMU{adv}}] suṇoma[listen-\IMP{\GMU{1-pl-imp}}] manasikaroma.[mind.attend-\IMP{\GMU{1-pl-imp}}]
\endgl
\switchcolumn*
\end{flushleft}


{\EnglishColumn

\begin{doublespace}
Whoever may have an offence, he should disclose (it). When there is no offence, (then it) is to be silent. By the silence I shall know the Venerables (with the thought): “(They are) pure.” As an answer occurs to (a bhikkhu) who is asked individually, just so in such an assembly (as this one) there is the announcement up to the third time. But if any bhikkhu, (who is) remembering (an offence) when the announcement is being made up to the third time, should not disclose the existing offence, there is (a further offence of) deliberate false speech for him.
\end{doublespace}}

\switchcolumn


\begin{flushleft}
\begingl
 Yassa[for whoever-\PRO{\GMU{pro}}] siyā[be-\OPT{\GMU{3-sg-opt}}] āpatti,[offense-\NOM{\GMU{nom-sg-f}}] so[he-\NOM{\GMU{nom-sg}}] āvikareyya.[disclose-\OPT{\GMU{3-sg-opt}}] Asantiyā[not.exist-\ADJ{\GMU{adj}}] āpattiyā[offense-\INS{\GMU{ins-sg-f}}] tuṇhī[silent-\ADV{\GMU{adv}}] bhavitabbaṁ.[to be-\FUT{\GMU{fut-pass-part}}] Tuṇhī[silent-\ADV{\GMU{adv}}] bhāvena[state of being-\INS{\GMU{ins-sg}}] kho[indeed!-\EMPH{\GMU{emph}}] pan’āyasmante[then.venerable-\ACC{\GMU{acc-pl}}] parisuddhā[pure-\ADJ{\GMU{adj}}] ti[-\NUL{\GMU{}}] vedissāmi.[know-\FUT{\GMU{1-sg-fut}}] Yathā[just as-\IND{\GMU{ind}}] kho[indeed!-\EMPH{\GMU{emph}}] pana[(and)-\PART{\GMU{part}}] paccekapuṭṭhassa[individually.ask-\ADJ{\GMU{adj}}] veyyākaraṇaṁ[answer-\NOM{\GMU{nom-sg-n}}] hoti.[he is-\PRESIND{\GMU{3-sg-presind}}] Evam’evaṁ[in same way-\ADV{\GMU{adv}}] evarūpāya[] parisāya[assembly-\DAT{\GMU{dat-sg-f}}] yāvatatiyaṁ[up to.3rd time-\ADV{\GMU{adv}}] anussāvitaṁ[announcement-\NOM{\GMU{nom-sg-n}}] hoti.[he is-\PRESIND{\GMU{3-sg-presind}}] Yo[who-\NOM{\GMU{nom-sg}}] pana[(and)-\PART{\GMU{part}}] bhikkhu[bhikkhu-\NOM{\GMU{nom-sg}}] yāvatatiyaṁ[up to.3rd time-\ADV{\GMU{adv}}] anussāviyamāne[announce-\LOC{\GMU{loc-sg}}] saramāno[remember-\PRES{\GMU{pres-part}}] santiṁ[exist-\PRES{\GMU{pres-part}}] āpattiṁ[offense-\ACC{\GMU{acc-sg-f}}] n’āvikareyya,[not.disclose-\OPT{\GMU{3-sg-opt}}] sampajānamusāvād’assa[deliberate.false.speech.for him-\NOM{\GMU{nom-sg}}] hoti.[he is-\PRESIND{\GMU{3-sg-presind}}]
\endgl
\switchcolumn*
\end{flushleft}


{\EnglishColumn

\begin{doublespace}
Now, venerables, deliberate false speech has been called an obstructive act by the Fortunate One. Therefore, by a bhikkhu who is remembering, who has committed (an offence), who is desiring purification, an existing offence is to be disclosed; because, (after) having disclosed (it), there is comfort for him.
\end{doublespace}}

\switchcolumn


\begin{flushleft}
\begingl
 Sampajānamusāvādo[deliberate.false.speech-\NOM{\GMU{nom-sg}}] kho[indeed!-\EMPH{\GMU{emph}}] pan’āyasmanto[venerable-\VOC{\GMU{voc-pl}}] antarāyiko[obstruct-\ADJ{\GMU{adj}}] dhammo[case-\NOM{\GMU{nom-sg}}] vutto[say-\PAST{\GMU{past-part}}] bhagavatā.[blessed one-\INS{\GMU{ins-sg}}] Tasmā[therefore-\ABL{\GMU{abl-sg}}] saramānena[remember-\PRES{\GMU{pres-part}}] bhikkhunā[bhikkhu-\INS{\GMU{ins-sg}}] āpannena[commit-\PAST{\GMU{past-part}}] visuddh’āpekkhena[purify.desire-\ADJ{\GMU{adj}}] santī[exist-\PRES{\GMU{pres-part}}] āpatti[offense-\NOM{\GMU{nom-sg-f}}] āvikātabbā.[disclose-\FUT{\GMU{fut-pass-part}}] Āvikatā[] hi’ssa[] phāsu[ease-\ADV{\GMU{adv}}] hoti.[he is-\PRESIND{\GMU{3-sg-presind}}]
\endgl
\switchcolumn*
\end{flushleft}


{\EnglishColumn

\begin{doublespace}
The recitation of the introduction is finished.
\end{doublespace}}

\switchcolumn


\begin{flushleft}
\begingl
 Nidān’uddeso[] niṭṭhito[]
\endgl
\switchcolumn*
\end{flushleft}


{\EnglishColumn

\begin{doublespace}
Herein these four cases involving disqualification come up for recitation.
\end{doublespace}}

\switchcolumn


\begin{flushleft}
\begingl
 Tatr’ime[] cattāro[4-\NUM{\GMU{num}}] pārājikā[defeat-\ADJ{\GMU{adj}}] dhammā[rule-\NOM{\GMU{nom-pl}}] uddesaṁ[recitation-\ACC{\GMU{acc-sg}}] āgacchanti.[come up-\PRESIND{\GMU{3-pl-presind}}]
\endgl
\switchcolumn*
\end{flushleft}


{\EnglishColumn

\begin{doublespace}
1. If any bhikkhu (who) has entered upon the training and livelihood for bhikkhus, not having rejected the training, not having disclosed (his) incapability, should engage in the act of sexual intercourse, even with just a female animal, he is disqualified, not in communion.
\end{doublespace}}

\switchcolumn


\begin{flushleft}
\begingl
 1.[] Yo[who-\NOM{\GMU{nom-sg}}] pana[(and)-\PART{\GMU{part}}] bhikkhu[bhikkhu-\NOM{\GMU{nom-sg}}] bhikkhūnaṁ[bhikkhu-\DAT{\GMU{dat-pl}}] sikkhāsājīvasamāpanno,[training-livelihood-enter-\ADJ{\GMU{adj}}] sikkhaṁ[training-\ACC{\GMU{acc-sg-f}}] appaccakkhāya[not reject-\ABS{\GMU{abs}}] dubbalyaṁ[weakness-\ACC{\GMU{acc-sg-n}}] anāvikatvā,[not.disclosed-\ABS{\GMU{abs}}] methunaṁ[coitus-\ADJ{\GMU{adj}}] dhammaṁ[act-\ACC{\GMU{acc-sg}}] paṭiseveyya[engage-\OPT{\GMU{3-sg-opt}}] antamaso[even so much as-\IND{\GMU{ind}}] tiracchānagatāya’pi:[female animal-\INS{\GMU{ins-sg-f}}] pārājiko[defeat-\ADJ{\GMU{adj}}] hoti[he is-\PRESIND{\GMU{3-sg-presind}}] asaṁvāso.[not communion-\ADJ{\GMU{adj}}]
\endgl
\switchcolumn*
\end{flushleft}

\pagebreak
{\EnglishColumn

\begin{doublespace}
2. If any bhikkhu should take (what has) not been given from a village or wilderness-area, which is reckoned as theft, (and) the taking of what has not been given (is) of the kind (that) on account of (it) kings, having caught the robber, would physically punish or imprison or banish (him, saying): “You are a robber! You are a fool! You are insane! You are a thief!,” a bhikkhu taking (what has) not been given of such a kind, is also disqualified, not in communion.
\end{doublespace}}

\switchcolumn


\begin{flushleft}
\begingl
 2.[] Yo[who-\NOM{\GMU{nom-sg}}] pana[(and)-\PART{\GMU{part}}] bhikkhu[bhikkhu-\NOM{\GMU{nom-sg}}] gāmā[village-\ABL{\GMU{abl-sg}}] vā[or-\IND{\GMU{ind}}] araññā[forest-\ABL{\GMU{abl-sg}}] vā[or-\IND{\GMU{ind}}] adinnaṁ[not.given-\ACC{\GMU{acc-sg-n}}] theyyasaṅkhātaṁ[theft.reckoned-\ACC{\GMU{acc-sg-n}}] ādiyeyya,[take-\OPT{\GMU{3-sg-opt}}] yathārūpe[like.kind-\ADJ{\GMU{adj}}] adinnādāne[not.given.take-\LOC{\GMU{loc-sg-n}}] rājāno[king-\NOM{\GMU{nom-pl}}] coraṁ[robber-\ACC{\GMU{acc-sg}}] gahetvā,[caught-\ABS{\GMU{abs}}] haneyyuṁ[beat-\OPT{\GMU{3-pl-opt}}] vā[or-\IND{\GMU{ind}}] bandheyyuṁ[imprison-\OPT{\GMU{3-pl-opt}}] vā[or-\IND{\GMU{ind}}] pabbājeyyuṁ[banish-\OPT{\GMU{3-pl-opt}}] vā,[or-\IND{\GMU{ind}}] “Coro’si[] bālo’si[fool.is-\NOM{\GMU{nom-sg-n}}] mūḷho’si[insane.is-\PAST{\GMU{past-part}}] theno’sī”[theif.are-\NOM{\GMU{nom-sg}}] ti.[-\NUL{\GMU{}}] Tathārūpaṁ[of such.kind-\ADJ{\GMU{adj}}] bhikkhu[bhikkhu-\NOM{\GMU{nom-sg}}] adinnaṁ[not.given-\ACC{\GMU{acc-sg-n}}] ādiyamāno:[take-\PRES{\GMU{pres-part}}] ayam’pi[] pārājiko[defeat-\ADJ{\GMU{adj}}] hoti[he is-\PRESIND{\GMU{3-sg-presind}}] asaṁvāso.[not communion-\ADJ{\GMU{adj}}]
\endgl
\switchcolumn*
\end{flushleft}


{\EnglishColumn

\begin{doublespace}
3. If any bhikkhu should deliberately deprive a human being of life, or should seek an assassin for him, or should praise the attractiveness of death, or should incite (him) to death (saying): “Dear man, what (use) is this bad, wretched life for you? Death is better than life for you!” should he, (having) such-thought-and- mind, (having such-) thought-and-intention, praise in manifold ways the beauty of death or incite (him) to death, he also is disqualified, not in communion.
\end{doublespace}}

\switchcolumn


\begin{flushleft}
\begingl
 3.[] Yo[who-\NOM{\GMU{nom-sg}}] pana[(and)-\PART{\GMU{part}}] bhikkhu[bhikkhu-\NOM{\GMU{nom-sg}}] sañcicca[deliberate-\ABS{\GMU{abs}}] manussaviggahaṁ[human being-\ACC{\GMU{acc-sg}}] jīvitā[life-\ABL{\GMU{abl-sg-n}}] voropeyya,[deprive-\OPT{\GMU{3-sg-opt}}] satthahārakaṁ[assassin-\ACC{\GMU{acc-sg}}] vāssa[] pariyeseyya,[seek-\OPT{\GMU{3-sg-opt}}] maraṇavaṇṇaṁ[death.beauty-\ACC{\GMU{acc-sg}}] vā[or-\IND{\GMU{ind}}] saṁvaṇṇeyya,[praise-\OPT{\GMU{3-sg-opt}}] maraṇāya[death-\DAT{\GMU{dat-sg}}] vā[or-\IND{\GMU{ind}}] samādapeyya,[incite-\OPT{\GMU{3-sg-opt}}] “Ambho[hey!-\NUL{\GMU{}}] purisa[man-\VOC{\GMU{voc-sg}}] kiṁ[what-\NUL{\GMU{}}] tuyh’iminā[you.this-\INS{\GMU{ins-sg}}] pāpakena[wretched-\ADJ{\GMU{adj}}] dujjīvitena?[difficult.life-\INS{\GMU{ins-sg-n}}] Matante[death.you-\NOM{\GMU{nom-sg-n}}] jīvitā[life-\ABL{\GMU{abl-sg-n}}] seyyo”[better-\ADV{\GMU{adv}}] ti.[-\NUL{\GMU{}}] Iti[so-\NUL{\GMU{}}] cittamano[cast down.eyes-\ADJ{\GMU{adj}}] cittasaṅkappo[mind.intention-\ADJ{\GMU{adj}}] anekapariyāyena[various ways-\ADV{\GMU{adv}}] maraṇavaṇṇaṁ[death.beauty-\ACC{\GMU{acc-sg}}] vā[or-\IND{\GMU{ind}}] saṁvaṇṇeyya,[praise-\OPT{\GMU{3-sg-opt}}] maraṇāya[death-\DAT{\GMU{dat-sg}}] vā[or-\IND{\GMU{ind}}] samādapeyya:[incite-\OPT{\GMU{3-sg-opt}}] ayam’pi[] pārājiko[defeat-\ADJ{\GMU{adj}}] hoti[he is-\PRESIND{\GMU{3-sg-presind}}] asaṁvāso.[not communion-\ADJ{\GMU{adj}}]
\endgl
\switchcolumn*
\end{flushleft}


{\EnglishColumn

\begin{doublespace}
4. If any bhikkhu, (though) not directly knowing (it), should claim a superhuman state pertaining to himself, (a state of) knowing and seeing (that is) suitable for the noble (ones), (saying): “Thus I know! Thus I see!,” (and) then, on another occasion, (whether) being interrogated or not being interrogated, having committed (the offence), desiring purification, should say so: “(Although) not knowing (it,) I spoke thus (saying): `I know,’ not seeing (it, I spoke, saying:) `I see.’ I bluffed vainly (and) falsely,” except (when said) in overestimation, he also is disqualified, not in communion.
\end{doublespace}}

\switchcolumn


\begin{flushleft}
\begingl
 4.[] Yo[who-\NOM{\GMU{nom-sg}}] pana[(and)-\PART{\GMU{part}}] bhikkhu[bhikkhu-\NOM{\GMU{nom-sg}}] anabhijānaṁ[not.fully.know-\NOM{\GMU{nom-sg}}] uttarimanussadhammaṁ[beyond.human.state-\ACC{\GMU{acc-sg}}] attūpanāyikaṁ[self.concerning-\ADJ{\GMU{adj}}] alamariyañāṇadassanaṁ[worthy.noble.know.see-\ADJ{\GMU{adj}}] samudācareyya:[boast-\OPT{\GMU{3-sg-opt}}] “Iti[so-\NUL{\GMU{}}] jānāmi,[know-\PRESIND{\GMU{1-sg-presind}}] iti[so-\NUL{\GMU{}}] passāmī”[see-\PRESIND{\GMU{1-sg-presind}}] ti.[-\NUL{\GMU{}}] Tato[then-\ABL{\GMU{abl}}] aparena[another-\ADJ{\GMU{adj}}] samayena[time-\INS{\GMU{ins-sg}}] samanuggāhiyamāno[interogate-\NOM{\GMU{nom-sg}}] vā[or-\IND{\GMU{ind}}] asamanuggāhiyamāno[not interrogate-\NOM{\GMU{nom-sg}}] vā[or-\IND{\GMU{ind}}] āpanno[commit-\PAST{\GMU{past-part}}] visuddh’āpekkho[purify.desire-\ADJ{\GMU{adj}}] evaṁ[thus-\ADV{\GMU{adv}}] vadeyya,[say-\OPT{\GMU{3-sg-opt}}] “Ajānam[not.know-\NOM{\GMU{nom-sg}}] evaṁ[thus-\ADV{\GMU{adv}}] āvuso[friend-\VOC{\GMU{voc-sg}}] avacaṁ,[say-\AOR{\GMU{1-sg-aor}}] ‘jānāmi,’[know-\PRESIND{\GMU{1-sg-presind}}] apassaṁ,[not.see-\PRES{\GMU{pres-part}}] ‘passāmi.’[see-\PRESIND{\GMU{1-sg-presind}}] Tucchaṁ[empty-\ADV{\GMU{adv}}] musā[FALSE-\ADV{\GMU{adv}}] vilapin”[boast-\AOR{\GMU{1-sg-aor}}] ti.[-\NUL{\GMU{}}] Aññatra[unless-\ABL{\GMU{abl}}] adhimānā:[overestimate-\ABL{\GMU{abl-sg}}] ayam’pi[] pārājiko[defeat-\ADJ{\GMU{adj}}] hoti[he is-\PRESIND{\GMU{3-sg-presind}}] asaṁvāso.[not communion-\ADJ{\GMU{adj}}]
\endgl
\switchcolumn*
\end{flushleft}


{\EnglishColumn

\begin{doublespace}
Venerables, the four cases involving disqualification have been recited, a bhikkhu who has committed any one of them, does not obtain the communion with bhikkhus. As (he was) before, so (he is) after (committing it): he is one who is disqualified, not in communion.
\end{doublespace}}

\switchcolumn


\begin{flushleft}
\begingl
 Uddiṭṭhā[recite-\PAST{\GMU{past-part}}] kho[indeed!-\EMPH{\GMU{emph}}] āyasmanto[Ven.-\VOC{\GMU{voc-pl}}] cattāro[4-\NUM{\GMU{num}}] pārājikā[defeat-\ADJ{\GMU{adj}}] dhammā,[rule-\NOM{\GMU{nom-pl}}] yesaṁ[them-\GEN{\GMU{gen-pl}}] bhikkhu[bhikkhu-\NOM{\GMU{nom-sg}}] aññataraṁ[any one, another-\ADJ{\GMU{adj}}] vā[or-\IND{\GMU{ind}}] aññataraṁ[any one, another-\ADJ{\GMU{adj}}] vā[or-\IND{\GMU{ind}}] āpajjitvā[commit-\ABS{\GMU{abs}}] na[not-\PART{\GMU{part}}] labhati[gain-\PRESIND{\GMU{3-sg-presind}}] bhikkhūhi[bhikkhu-\INS{\GMU{ins-pl}}] saddhiṁ[together-\INS{\GMU{ins}}] saṁvāsaṁ,[communion-\ACC{\GMU{acc-sg}}] yathā[just as-\IND{\GMU{ind}}] pure,[before-\IND{\GMU{ind}}] tathā[so-\ADV{\GMU{adv}}] pacchā,[after-\IND{\GMU{ind}}] pārājiko[defeat-\ADJ{\GMU{adj}}] hoti[he is-\PRESIND{\GMU{3-sg-presind}}] asaṁvāso.[not communion-\ADJ{\GMU{adj}}]
\endgl
\switchcolumn*
\end{flushleft}


{\EnglishColumn

\begin{doublespace}
Concerning that I ask the Venerables: (Are you) pure in this?\\
A second time again I ask: (Are you) pure in this?\\
A third time again I ask: (Are you) pure in this?\\
The venerables are pure in this, therefore there is silence, so do I bear this (in mind).
\end{doublespace}}

\switchcolumn


\begin{flushleft}
\begingl
 Tatth’āyasmante[] pucchāmi:[ask-\PRESIND{\GMU{1-sg-presind}}] Kacci’ttha[] parisuddhā?[pure-\ADJ{\GMU{adj}}]+ Dutiyam’pi[second time-\ACC{\GMU{acc-sg-nt}}] pucchāmi:[ask-\PRESIND{\GMU{1-sg-presind}}] Kacci’ttha[] parisuddhā?[pure-\ADJ{\GMU{adj}}]+ Tatiyam’pi[] pucchāmi:[ask-\PRESIND{\GMU{1-sg-presind}}] Kacci’ttha[] parisuddhā?[pure-\ADJ{\GMU{adj}}]+ Parisuddh’etth’āyasmanto,[] tasmā[therefore-\ABL{\GMU{abl-sg}}] tuṇhī,[silent-\ADV{\GMU{adv}}] evam’etaṁ[thus.this-\ACC{\GMU{acc-sg-n}}] dhārayāmi.[keep in mind-\PRESIND{\GMU{1-sg-presind}}]
\endgl
\switchcolumn*
\end{flushleft}


{\EnglishColumn

\begin{doublespace}
The recitation of the (cases involving) disqualification is finished
\end{doublespace}}

\switchcolumn


\begin{flushleft}
\begingl
 Pārājik’uddeso[] niṭṭhito[]
\endgl
\switchcolumn*
\end{flushleft}


{\EnglishColumn

\begin{doublespace}
Venerables, these thirteen cases (concerning) the community in the beginning and in the rest (of the procedure) come up for recitation.
\end{doublespace}}

\switchcolumn


\begin{flushleft}
\begingl
 Ime[this-\NOM{\GMU{nom-pl}}] kho[indeed!-\EMPH{\GMU{emph}}] pan’āyasmanto[venerable-\VOC{\GMU{voc-pl}}] terasa[13-\ADJ{\GMU{adj}}] saṅghādisesā[] dhammā[rule-\NOM{\GMU{nom-pl}}] uddesaṁ[recitation-\ACC{\GMU{acc-sg}}] āgacchanti.[come up-\PRESIND{\GMU{3-pl-presind}}]
\endgl
\switchcolumn*
\end{flushleft}


{\EnglishColumn

\begin{doublespace}
1. The intentional emission of semen, except in a dream: (this is a case concerning) the community in the beginning and in the rest (of the procedure).
\end{doublespace}}

\switchcolumn


\begin{flushleft}
\begingl
 1.[] Sañcetanikā[deliberate-\ADJ{\GMU{adj}}] sukkavisaṭṭhi[semen.emission-\NOM{\GMU{nom-sg-f}}] aññatra[unless-\ABL{\GMU{abl}}] supinantā,[dream.in-\ABL{\GMU{abl-sg-n}}] saṅghādiseso.[-\NUL{\GMU{}}]
\endgl
\switchcolumn*
\end{flushleft}


{\EnglishColumn

\begin{doublespace}
2. If any bhikkhu, under the influence of an altered mind, should engage in (intimate) physical contact together with a woman (such as): the holding of a hand, or holding a braid (of hair), or caressing any limb: (this is a case concerning) the community in the beginning and in the rest (of the procedure).
\end{doublespace}}

\switchcolumn


\begin{flushleft}
\begingl
 2.[] Yo[who-\NOM{\GMU{nom-sg}}] pana[(and)-\PART{\GMU{part}}] bhikkhu[bhikkhu-\NOM{\GMU{nom-sg}}] otiṇṇo[beset-\ADJ{\GMU{adj}}] vipariṇatena[alter-\ADJ{\GMU{adj}}] cittena[mind-\INS{\GMU{ins-sg}}] mātugāmena[woman-\INS{\GMU{ins-sg}}] saddhiṁ[together-\INS{\GMU{ins}}] kāyasaṁsaggaṁ[body.contact-\ACC{\GMU{acc-sg}}] samāpajjeyya,[enter-\OPT{\GMU{3-sg-opt}}] hatthagāhaṁ[hand.hold-\ACC{\GMU{acc-sg}}] vā[or-\IND{\GMU{ind}}] veṇigāhaṁ[braid.hold-\ACC{\GMU{acc-sg}}] vā[or-\IND{\GMU{ind}}] aññatarassa[any one, another-\ADJ{\GMU{adj}}] vā[or-\IND{\GMU{ind}}] aññatarassa[any one, another-\ADJ{\GMU{adj}}] vā[or-\IND{\GMU{ind}}] aṅgassa[limb-\GEN{\GMU{gen-sg-n}}] parāmasanaṁ,[over close.touch-\NUL{\GMU{}}] saṅghādiseso.[-\NUL{\GMU{}}]
\endgl
\switchcolumn*
\end{flushleft}


{\EnglishColumn

\begin{doublespace}
3. If any bhikkhu, under the influence of an altered mind, should speak suggestively with depraved words to a woman, like a young man to a young woman, (with words) concerned with sexual intercourse: (this is a case concerning) the community in the beginning and in the rest (of the procedure).
\end{doublespace}}

\switchcolumn


\begin{flushleft}
\begingl
 3.[] Yo[who-\NOM{\GMU{nom-sg}}] pana[(and)-\PART{\GMU{part}}] bhikkhu[bhikkhu-\NOM{\GMU{nom-sg}}] otiṇṇo[beset-\ADJ{\GMU{adj}}] vipariṇatena[alter-\ADJ{\GMU{adj}}] cittena[mind-\INS{\GMU{ins-sg}}] mātugāmaṁ[woman-\ACC{\GMU{acc-sg}}] duṭṭhullāhi[obscene-\ADJ{\GMU{adj}}] vācāhi[word-\INS{\GMU{ins-pl-f}}] obhāseyya,[suggest-\OPT{\GMU{3-sg-opt}}] yathā[just as-\IND{\GMU{ind}}] taṁ[that-\ACC{\GMU{acc-sg}}] yuvā[y. man-\NOM{\GMU{nom-sg}}] yuvatiṁ[y. woman-\ACC{\GMU{acc-sg-f}}] methunūpasaṁhitāhi,[coitus.concern with-\ADJ{\GMU{adj}}] saṅghādiseso.[-\NUL{\GMU{}}]
\endgl
\switchcolumn*
\end{flushleft}


{\EnglishColumn

\begin{doublespace}
4. If any bhikkhu, under the influence of an altered mind, (and) in the presence of a woman, should speak praise about the ministering to himself with sex: “Sister, this is the best of ministerings: she who would minister to a virtuous, good natured celibate like me with this act!,” (which is something) connected with sexual intercourse: (this is a case concerning) the community in the beginning and in the rest (of the procedure).
\end{doublespace}}

\switchcolumn


\begin{flushleft}
\begingl
 4.[] Yo[who-\NOM{\GMU{nom-sg}}] pana[(and)-\PART{\GMU{part}}] bhikkhu[bhikkhu-\NOM{\GMU{nom-sg}}] otiṇṇo[beset-\ADJ{\GMU{adj}}] vipariṇatena[alter-\ADJ{\GMU{adj}}] cittena[mind-\INS{\GMU{ins-sg}}] mātugāmassa[woman-\GEN{\GMU{gen-sg}}] santike[with near-\LOC{\GMU{loc-sg-n}}] attakāmapāricariyāya[himself.sex.minister-\LOC{\GMU{loc-sg-f}}] vaṇṇaṁ[praise-\ACC{\GMU{acc-sg}}] bhāseyya,[speak-\OPT{\GMU{3-sg-opt}}] “Etadaggaṁ[highest-\NOM{\GMU{nom-sg-n}}] bhagini[sister-\VOC{\GMU{voc-sg-f}}] pāricariyānaṁ,[minister-\GEN{\GMU{gen-pl-f}}] yā[who-\NOM{\GMU{nom-sg-f}}] m’ādisaṁ[like me-\ACC{\GMU{acc-sg}}] sīlavantaṁ[virtue-\ADJ{\GMU{adj}}] kalyāṇadhammaṁ[good.nature-\ADJ{\GMU{adj}}] brahmacāriṁ[holy life-\ACC{\GMU{acc-sg-n}}] etena[this-\INS{\GMU{ins-sg}}] dhammena[act-\INS{\GMU{ins-sg}}] paricareyyā”[minister-\OPT{\GMU{3-sg-opt}}] ti,[-\NUL{\GMU{}}] methunūpasaṁhitena,[coitus.concern with-\ADJ{\GMU{adj}}] saṅghādiseso.[-\NUL{\GMU{}}]
\endgl
\switchcolumn*
\end{flushleft}


{\EnglishColumn

\begin{doublespace}
5. If any bhikkhu should engage in mediating a man's intention to a woman, or a woman's intention to a man, for being a wife or for being a mistress, even for being one on (just) that occasion: (this is a case concerning) the community in the beginning and in the rest (of the procedure).
\end{doublespace}}

\switchcolumn


\begin{flushleft}
\begingl
 5.[] Yo[who-\NOM{\GMU{nom-sg}}] pana[(and)-\PART{\GMU{part}}] bhikkhu[bhikkhu-\NOM{\GMU{nom-sg}}] sañcarittaṁ[mediate-\ACC{\GMU{acc-sg-n}}] samāpajjeyya,[enter-\OPT{\GMU{3-sg-opt}}] itthiyā[woman-\DAT{\GMU{dat-sg-f}}] vā[or-\IND{\GMU{ind}}] purisamatiṁ,[man's intent-\ACC{\GMU{acc-sg}}] purisassa[man-\DAT{\GMU{dat-sg}}] vā[or-\IND{\GMU{ind}}] itthīmatiṁ,[woman intent-\ACC{\GMU{acc-sg-f}}] jāyattane[wife-\LOC{\GMU{loc-sg-n}}] vā[or-\IND{\GMU{ind}}] jārattane[mistress-\LOC{\GMU{loc-sg-f}}] vā[or-\IND{\GMU{ind}}] antamaso[even so much as-\IND{\GMU{ind}}] taṁkhaṇikāya’pi,[that.moment-\LOC{\GMU{loc-sg-f}}] saṅghādiseso.[-\NUL{\GMU{}}]
\endgl
\switchcolumn*
\end{flushleft}


{\EnglishColumn

\begin{doublespace}
6. entailing harm (to creatures and which is) having a surrounding space. If a bhikkhu, having requested it himself, should have a hut built on a site entailing harm (to creatures), (and) not having a surrounding space, or if he should not bring bhikkhus to (it) for appointing the site, or if he should let (it) exceed the measure: (this is a case concerning) the community in the beginning and in the rest (of the procedure).
\end{doublespace}}

\switchcolumn


\begin{flushleft}
\begingl
 6.[] Saññācikāya[own request-\INS{\GMU{ins-sg-f}}] pana[(and)-\PART{\GMU{part}}] bhikkhunā[bhikkhu-\INS{\GMU{ins-sg}}] kuṭiṁ[hut-\ACC{\GMU{acc-sg-f}}] kārayamānena[build-\PRES{\GMU{pres-part}}] assāmikaṁ[without owner-\ADJ{\GMU{adj}}] att’uddesaṁ[self.designate-\ADJ{\GMU{adj}}] pamāṇikā[measure-\ADJ{\GMU{adj}}] kāretabbā.[make-\FUT{\GMU{fut-pass-part}}] Tatr’idaṁ[here.this-\NUL{\GMU{}}] pamāṇaṁ:[measure-\NOM{\GMU{nom-sg}}] dīghaso[length-\ADV{\GMU{adv}}] dvādasa[12-\ADJ{\GMU{adj}}] vidatthiyo[span-\ACC{\GMU{acc-pl-f}}] sugatavidatthiyā,[well.gone.span-\INS{\GMU{ins-sg-f}}] tiriyaṁ[width-\IND{\GMU{ind}}] satt’antarā.[] Bhikkhū[bhikkhu-\NOM{\GMU{nom-pl}}] abhinetabbā[led to-\FUT{\GMU{fut-pass-part}}] vatthudesanāya.[site.designate-\DAT{\GMU{dat-sg-f}}] Tehi[those-\INS{\GMU{ins-pl}}] bhikkhūhi[bhikkhu-\INS{\GMU{ins-pl}}] vatthuṁ[site-\NOM{\GMU{nom-sg-n}}] desetabbaṁ[appoint-\FUT{\GMU{fut-pass-part}}] anārambhaṁ[not.harm-\ADJ{\GMU{adj}}] saparikkamanaṁ.[with.around.space-\ADJ{\GMU{adj}}] Sārambhe[with.harm-\ADJ{\GMU{adj}}] ce[if-\NUL{\GMU{}}] bhikkhu[bhikkhu-\NOM{\GMU{nom-sg}}] vatthusmiṁ[site-\LOC{\GMU{loc-sg-n}}] aparikkamane[not.with.around.space-\ADJ{\GMU{adj}}] saññācikāya[own request-\INS{\GMU{ins-sg-f}}] kuṭiṁ[hut-\ACC{\GMU{acc-sg-f}}] kāreyya,[make-\OPT{\GMU{3-sg-opt}}] bhikkhū[bhikkhu-\NOM{\GMU{nom-pl}}] vā[or-\IND{\GMU{ind}}] anabhineyya[not.bring-\OPT{\GMU{3-sg-opt}}] vatthudesanāya,[site.designate-\DAT{\GMU{dat-sg-f}}] pamāṇaṁ[measure-\NOM{\GMU{nom-sg}}] vā[or-\IND{\GMU{ind}}] atikkāmeyya,[beyond.go-\OPT{\GMU{3-sg-opt}}] saṅghādiseso.[-\NUL{\GMU{}}]
\endgl
\switchcolumn*
\end{flushleft}


{\EnglishColumn

\begin{doublespace}
7. By a bhikkhu who is having a large dwelling built, which has an owner, (and) is designated for himself, bhikkhus are to be brought to (it) for appointing the site. By those bhikkhus a site not entailing harm (to any creatures) (and) having a surrounding space is to be appointed. If a bhikkhu should have a hut built on a site entailing harm (to creatures), (and) not having a surrounding space, or if he should not bring bhikkhus to (it) for appointing the site, (this is a case concerning) the community in the beginning and in the rest (of the procedure).
\end{doublespace}}

\switchcolumn


\begin{flushleft}
\begingl
 7.[] Mahallakaṁ[large-\ADJ{\GMU{adj}}] pana[(and)-\PART{\GMU{part}}] bhikkhunā[bhikkhu-\INS{\GMU{ins-sg}}] vihāraṁ[dwell-\ACC{\GMU{acc-sg}}] kārayamānena,[build-\PRES{\GMU{pres-part}}] sassāmikaṁ[with.owner-\ADJ{\GMU{adj}}] att’uddesaṁ[self.designate-\ADJ{\GMU{adj}}] bhikkhū[bhikkhu-\NOM{\GMU{nom-pl}}] abhinetabbā[led to-\FUT{\GMU{fut-pass-part}}] vatthudesanāya.[site.designate-\DAT{\GMU{dat-sg-f}}] Tehi[those-\INS{\GMU{ins-pl}}] bhikkhūhi[bhikkhu-\INS{\GMU{ins-pl}}] vatthuṁ[site-\NOM{\GMU{nom-sg-n}}] desetabbaṁ[appoint-\FUT{\GMU{fut-pass-part}}] anārambhaṁ[not.harm-\ADJ{\GMU{adj}}] saparikkamanaṁ.[with.around.space-\ADJ{\GMU{adj}}] Sārambhe[with.harm-\ADJ{\GMU{adj}}] ce[if-\NUL{\GMU{}}] bhikkhu[bhikkhu-\NOM{\GMU{nom-sg}}] vatthusmiṁ[site-\LOC{\GMU{loc-sg-n}}] aparikkamane[not.with.around.space-\ADJ{\GMU{adj}}] mahallakaṁ[large-\ADJ{\GMU{adj}}] vihāraṁ[dwell-\ACC{\GMU{acc-sg}}] kāreyya,[make-\OPT{\GMU{3-sg-opt}}] bhikkhū[bhikkhu-\NOM{\GMU{nom-pl}}] vā[or-\IND{\GMU{ind}}] anabhineyya[not.bring-\OPT{\GMU{3-sg-opt}}] vatthudesanāya,[site.designate-\DAT{\GMU{dat-sg-f}}] saṅghādiseso.[-\NUL{\GMU{}}]
\endgl
\switchcolumn*
\end{flushleft}


{\EnglishColumn

\begin{doublespace}
8. If any bhikkhu, corrupted by malice (and) upset, should accuse a bhikkhu with a groundless case involving disqualification (thinking): “If only I could make him fall away from this holy life!,”\\
(and) then, on another occasion, (whether) being interrogated or not being interrogated, if that legal issue is really groundless, and if the bhikkhu stands firm in malice: (this is a case concerning) the community in the beginning and in the rest (of the procedure).
\end{doublespace}}

\switchcolumn


\begin{flushleft}
\begingl
 8.[] Yo[who-\NOM{\GMU{nom-sg}}] pana[(and)-\PART{\GMU{part}}] bhikkhu[bhikkhu-\NOM{\GMU{nom-sg}}] bhikkhuṁ[bhikkhu-\ACC{\GMU{acc-sg}}] duṭṭho[corrupted-\PAST{\GMU{past-part}}] doso[anger-\NOM{\GMU{nom-sg}}] appatīto[displeased-\ADJ{\GMU{adj}}] amūlakena[without cause-\ADJ{\GMU{adj}}] pārājikena[defeat-\ADJ{\GMU{adj}}] dhammena[act-\INS{\GMU{ins-sg}}] anuddhaṁseyya,[accuse-\OPT{\GMU{3-sg-opt}}] “App’eva[if.only-\EMPH{\GMU{emph-part}}] nāma[indeed!-\EMPH{\GMU{emph}}] naṁ[him-\ACC{\GMU{3-sg-acc}}] imamhā[from this-\ABL{\GMU{3-sg-abl}}] brahmacariyā[holy life-\ABL{\GMU{abl-sg-n}}] cāveyyan”[fall-\OPT{\GMU{1-sg-opt}}] ti.[-\NUL{\GMU{}}]+ Tato[then-\ABL{\GMU{abl}}] aparena[another-\ADJ{\GMU{adj}}] samayena[time-\INS{\GMU{ins-sg}}] samanuggāhiyamāno[interogate-\NOM{\GMU{nom-sg}}] vā[or-\IND{\GMU{ind}}] asamanuggāhiyamāno[not interrogate-\NOM{\GMU{nom-sg}}] vā,[or-\IND{\GMU{ind}}] amūlakañc’eva[without root.emph-\ADJ{\GMU{adj}}] taṁ[that-\ACC{\GMU{acc-sg}}] adhikaraṇaṁ[legal issue-\NOM{\GMU{nom-sg-n}}] hoti,[he is-\PRESIND{\GMU{3-sg-presind}}] bhikkhu[bhikkhu-\NOM{\GMU{nom-sg}}] ca[-\NUL{\GMU{}}] dosaṁ[malice-\ACC{\GMU{acc-sg}}] patiṭṭhāti,[stand firm-\PRESIND{\GMU{3-sg-presind}}] saṅghādiseso.[-\NUL{\GMU{}}]
\endgl
\switchcolumn*
\end{flushleft}


{\EnglishColumn

\begin{doublespace}
9. If any bhikkhu, corrupted by malice (and) upset, should accuse a bhikkhu with a case involving disqualification, having taken (it) up (with) some point, which is a mere pretext, of a legal issue belonging to another class (thinking): “If only I could make him fall away from this holy life!,”\\
(and) then, on another occasion, (whether) being interrogated or not being interrogated, if that legal issue is really belonging to another class, (and) some point, which a mere pretext, has been taken up, and if the bhikkhu stands firm in malice: (this is a case concerning) the community in the beginning and in the rest (of the procedure).
\end{doublespace}}

\switchcolumn


\begin{flushleft}
\begingl
 9.[] Yo[who-\NOM{\GMU{nom-sg}}] pana[(and)-\PART{\GMU{part}}] bhikkhu[bhikkhu-\NOM{\GMU{nom-sg}}] bhikkhuṁ[bhikkhu-\ACC{\GMU{acc-sg}}] duṭṭho[corrupted-\PAST{\GMU{past-part}}] doso[anger-\NOM{\GMU{nom-sg}}] appatīto[displeased-\ADJ{\GMU{adj}}] aññabhāgiyassa[other class-\ADJ{\GMU{adj}}] adhikaraṇassa[legal issue-\GEN{\GMU{gen-sg-n}}] kiñci[some-\PRO{\GMU{pro}}] desaṁ[point-\ACC{\GMU{acc-sg}}] lesamattaṁ[ploy.mere-\ADJ{\GMU{adj}}] upādāya[take up-\ABS{\GMU{abs}}] pārājikena[defeat-\ADJ{\GMU{adj}}] dhammena[act-\INS{\GMU{ins-sg}}] anuddhaṁseyya,[accuse-\OPT{\GMU{3-sg-opt}}] “App’eva[if.only-\EMPH{\GMU{emph-part}}] nāma[indeed!-\EMPH{\GMU{emph}}] naṁ[him-\ACC{\GMU{3-sg-acc}}] imamhā[from this-\ABL{\GMU{3-sg-abl}}] brahmacariyā[holy life-\ABL{\GMU{abl-sg-n}}] cāveyyan”[fall-\OPT{\GMU{1-sg-opt}}] ti.[-\NUL{\GMU{}}]+ Tato[then-\ABL{\GMU{abl}}] aparena[another-\ADJ{\GMU{adj}}] samayena[time-\INS{\GMU{ins-sg}}] samanuggāhiyamāno[interogate-\NOM{\GMU{nom-sg}}] vā[or-\IND{\GMU{ind}}] asamanuggāhiyamāno[not interrogate-\NOM{\GMU{nom-sg}}] vā,[or-\IND{\GMU{ind}}] aññabhāgiyañc’eva[other class.emph-\ADJ{\GMU{adj}}] taṁ[that-\ACC{\GMU{acc-sg}}] adhikaraṇaṁ[legal issue-\NOM{\GMU{nom-sg-n}}] hoti,[he is-\PRESIND{\GMU{3-sg-presind}}] koci[someone-\PRO{\GMU{pro}}] deso[point-\NOM{\GMU{nom-sg}}] lesamatto[ploy.mere-\NOM{\GMU{nom-sg}}] upādinno,[take up-\PAST{\GMU{past-part}}] bhikkhu[bhikkhu-\NOM{\GMU{nom-sg}}] ca[-\NUL{\GMU{}}] dosaṁ[malice-\ACC{\GMU{acc-sg}}] patiṭṭhāti,[stand firm-\PRESIND{\GMU{3-sg-presind}}] saṅghādiseso.[-\NUL{\GMU{}}]
\endgl
\switchcolumn*
\end{flushleft}


{\EnglishColumn

\begin{doublespace}
10. If any bhikkhu should endeavor for the schism of a united community, or having undertaken, should persist in upholding a legal issue conducive to schism, (then) that bhikkhu should be spoken to thus by the bhikkhus:\\
“Let the venerable one not endeavor for the schism of the united community, or having undertaken, persist in upholding a legal issue conducive to schism. Let the venerable one convene with the community, for a united community, which is on friendly terms, which is not disputing, which has a single recitation, dwells in comfort,”\\
and (if) that bhikkhu being spoken to thus by the bhikkhus should persist in the same way (as before), (then) that bhikkhu is to be argued with up to three times by the bhikkhus for the relinquishing of that (course), (and if that bhikkhu,) being argued with up to three times, should relinquish that (course), then this is good, (but) if he should not relinquish (it): (this is a case concerning) the community in the beginning and in the rest (of the procedure).
\end{doublespace}}

\switchcolumn


\begin{flushleft}
\begingl
 10.[] Yo[who-\NOM{\GMU{nom-sg}}] pana[(and)-\PART{\GMU{part}}] bhikkhu[bhikkhu-\NOM{\GMU{nom-sg}}] samaggassa[united-\ADJ{\GMU{adj}}] saṅghassa[community-\DAT{\GMU{dat-sg}}] bhedāya[schism-\DAT{\GMU{dat-sg}}] parakkameyya,[endeavor-\OPT{\GMU{3-sg-opt}}] bhedanasaṁvattanikaṁ[schism.conduce-\ADJ{\GMU{adj}}] vā[or-\IND{\GMU{ind}}] adhikaraṇaṁ[legal issue-\NOM{\GMU{nom-sg-n}}] samādāya[undertake-\ABS{\GMU{abs}}] paggayha[uphold-\ABS{\GMU{abs}}] tiṭṭheyya,[persist-\OPT{\GMU{3-sg-opt}}] so[he-\NOM{\GMU{nom-sg}}] bhikkhu[bhikkhu-\NOM{\GMU{nom-sg}}] bhikkhūhi[bhikkhu-\INS{\GMU{ins-pl}}] evam[thus-\ADV{\GMU{adv}}] assa[to be-\OPT{\GMU{3-sg-opt}}] vacanīyo,[address-\FUT{\GMU{fut-pass-part}}]+ “Mā[do not-\PART{\GMU{part}}] āyasmā[Ven.-\NOM{\GMU{nom-sg}}] samaggassa[united-\ADJ{\GMU{adj}}] saṅghassa[community-\DAT{\GMU{dat-sg}}] bhedāya[schism-\DAT{\GMU{dat-sg}}] parakkami.[endeavor-\AOR{\GMU{3-sg-aor}}] Bhedanasaṁvattanikaṁ[schism.conduce-\ADJ{\GMU{adj}}] vā[or-\IND{\GMU{ind}}] adhikaraṇaṁ[legal issue-\NOM{\GMU{nom-sg-n}}] samādāya[undertake-\ABS{\GMU{abs}}] paggayha[uphold-\ABS{\GMU{abs}}] aṭṭhāsi.[persist-\AOR{\GMU{3-sg-aor}}] Samet’āyasmā[agree.venerable-\IMP{\GMU{3-sg-imp}}] saṅghena,[community-\INS{\GMU{ins-sg}}] samaggo[united-\ADJ{\GMU{adj}}] hi[for-\IND{\GMU{ind}}] saṅgho[community-\NOM{\GMU{nom-sg}}] sammodamāno[agreement-\PRES{\GMU{pres-part}}] avivadamāno[not.dispute-\ADJ{\GMU{adj}}] ek’uddeso[one recital-\ADJ{\GMU{adj}}] phāsu[ease-\ADV{\GMU{adv}}] viharatī”[dwell-\PRESIND{\GMU{3-sg-presind}}] ti.[-\NUL{\GMU{}}]+ Evañca[thus-\ADV{\GMU{adv}}] so[he-\NOM{\GMU{nom-sg}}] bhikkhu[bhikkhu-\NOM{\GMU{nom-sg}}] bhikkhūhi[bhikkhu-\INS{\GMU{ins-pl}}] vuccamāno[address-\PRES{\GMU{pres-pass-part}}] tath’eva[in same way-\NUL{\GMU{}}] paggaṇheyya,[uphold-\OPT{\GMU{3-sg-opt}}] so[he-\NOM{\GMU{nom-sg}}] bhikkhu[bhikkhu-\NOM{\GMU{nom-sg}}] bhikkhūhi[bhikkhu-\INS{\GMU{ins-pl}}] yāvatatiyaṁ[up to.3rd time-\ADV{\GMU{adv}}] samanubhāsitabbo[admonish-\FUT{\GMU{fut-pass-part}}] tassa[of that-\GEN{\GMU{gen-sg}}] paṭinissaggāya.[relinquish-\DAT{\GMU{dat-sg}}] Yāvatatiyañ’ce[up to.3rd time-\ADV{\GMU{adv}}] samanubhāsiyamāno[admonish-\PRES{\GMU{pres-part}}] taṁ[that-\ACC{\GMU{acc-sg}}] paṭinissajjeyya,[relinquish-\OPT{\GMU{3-sg-opt}}] icc’etaṁ[thus.this-\ACC{\GMU{acc-sg}}] kusalaṁ.[good-\NOM{\GMU{nom-sg-n}}] No[not-\NEG{\GMU{neg-part}}] ce[if-\NUL{\GMU{}}] paṭinissajjeyya,[relinquish-\OPT{\GMU{3-sg-opt}}] saṅghādiseso.[-\NUL{\GMU{}}]
\endgl
\switchcolumn*
\end{flushleft}


{\EnglishColumn

\begin{doublespace}
11. Now, there are bhikkhus who are followers of that same bhikkhu, (and) who are speaking for (his) faction: one, or two, or three, (and) they should say so: “Venerables, don't say anything to this bhikkhu! This bhikkhu is one who speaks in accordance with the Teaching and this bhikkhu is one who speaks in accordance the Discipline; this (bhikkhu), having received (our) consent and favour defines (the Teaching & Discipline). Knowing us, he speaks, (and) this suits us too.”\\
(Then) those bhikkhus should be spoken to thus by the bhikkhus: “Venerables, don't say so! This bhikkhu does not speak in accordance with the Teaching, and this bhikkhu does not speak in accordance with the Discipline! Don't let the venerables too favour the schism of the community. Let there be convening with the community for the venerables, for a united community, which is on friendly terms, which is not disputing, which has a single recitation, dwells in comfort,”\\
and (if) those bhikkhus being spoken to thus by the bhikkhus should persist in the same way (as before), (then) those bhikkhus are to be argued with up to three times by the bhikkhus for the relinquishing of that (course), (and if those bhikkhus) being argued with up to three times, should relinquish that (course), then this is good, (but) if they should not relinquish (it): (this is a case concerning) the community in the beginning and in the rest (of the procedure).
\end{doublespace}}

\switchcolumn


\begin{flushleft}
\begingl
 11.[] Tass’eva[that.same-\GEN{\GMU{gen-sg}}] kho[indeed!-\EMPH{\GMU{emph}}] pana[(and)-\PART{\GMU{part}}] bhikkhussa[bhikkhu-\GEN{\GMU{gen-sg}}] bhikkhū[bhikkhu-\NOM{\GMU{nom-pl}}] honti[there are-\PRESIND{\GMU{3-pl-presind}}] anuvattakā[followers-\ADJ{\GMU{adj}}] vaggavādakā,[faction.speak-\ADJ{\GMU{adj}}] eko[one-\NUM{\GMU{num}}] vā[or-\IND{\GMU{ind}}] dve[2-\NUM{\GMU{num}}] vā[or-\IND{\GMU{ind}}] tayo[3-\NUM{\GMU{num}}] vā,[or-\IND{\GMU{ind}}] te[you-\DAT{\GMU{dat-sg-n}}] evaṁ[thus-\ADV{\GMU{adv}}] vadeyyuṁ,[say-\OPT{\GMU{3-pl-opt}}] “Mā[do not-\PART{\GMU{part}}] āyasmanto[Ven.-\VOC{\GMU{voc-pl}}] etaṁ[this-\ACC{\GMU{acc-sg}}] bhikkhuṁ[bhikkhu-\ACC{\GMU{acc-sg}}] kiñci[some-\PRO{\GMU{pro}}] avacuttha.[say-\AOR{\GMU{2-pl-aor}}] Dhammavādī[doctrine.speak-\ADJ{\GMU{adj}}] c’eso[and.this-\NOM{\GMU{nom-sg}}] bhikkhu,[bhikkhu-\NOM{\GMU{nom-sg}}] vinayavādī[discipline.speak-\ADJ{\GMU{adj}}] c’eso[and.this-\NOM{\GMU{nom-sg}}] bhikkhu,[bhikkhu-\NOM{\GMU{nom-sg}}] amhākañc’eso[us.and.this-\GEN{\GMU{gen-sg}}] bhikkhu[bhikkhu-\NOM{\GMU{nom-sg}}] chandañca[consent-\ACC{\GMU{acc-sg}}] ruciñca[approval.and-\ACC{\GMU{acc-sg}}] ādāya[take-\ABS{\GMU{abs}}] voharati.[express-\PRESIND{\GMU{3-sg-presind}}] Jānāti[know-\PRESIND{\GMU{3-sg-presind}}] no[not-\NEG{\GMU{neg-part}}] bhāsati,[speak-\PRESIND{\GMU{3-sg-presind}}] amhākam’p’etaṁ[us.to.this-\DAT{\GMU{dat-sg}}] khamatī”[agree-\PRESIND{\GMU{3-sg-presind}}] ti.[-\NUL{\GMU{}}]+ Te[you-\DAT{\GMU{dat-sg-n}}] bhikkhū[bhikkhu-\NOM{\GMU{nom-pl}}] bhikkhūhi[bhikkhu-\INS{\GMU{ins-pl}}] evamassu[thus-\NUL{\GMU{}}] vacanīyā,[address-\FUT{\GMU{fut-pass-part}}] “Mā[do not-\PART{\GMU{part}}] āyasmanto[Ven.-\VOC{\GMU{voc-pl}}] evaṁ[thus-\ADV{\GMU{adv}}] avacuttha.[say-\AOR{\GMU{2-pl-aor}}] Na[not-\PART{\GMU{part}}] c’eso[and.this-\NOM{\GMU{nom-sg}}] bhikkhu[bhikkhu-\NOM{\GMU{nom-sg}}] dhammavādī,[doctrine.speak-\ADJ{\GMU{adj}}] na[not-\PART{\GMU{part}}] c’eso[and.this-\NOM{\GMU{nom-sg}}] bhikkhu[bhikkhu-\NOM{\GMU{nom-sg}}] vinayavādī.[discipline.speak-\ADJ{\GMU{adj}}] Mā[do not-\PART{\GMU{part}}] āyasmantānam’pi[Ven.-\DAT{\GMU{dat-pl}}] saṅghabhedo[community.schism-\NOM{\GMU{nom-sg}}] rucittha.[favor-\AOR{\GMU{2-pl-aor}}] Samet’āyasmantānaṁ[agree.venerable-\DAT{\GMU{dat-pl}}] saṅghena,[community-\INS{\GMU{ins-sg}}] samaggo[united-\ADJ{\GMU{adj}}] hi[for-\IND{\GMU{ind}}] saṅgho[community-\NOM{\GMU{nom-sg}}] sammodamāno[agreement-\PRES{\GMU{pres-part}}] avivadamāno[not.dispute-\ADJ{\GMU{adj}}] ek’uddeso[one recital-\ADJ{\GMU{adj}}] phāsu[ease-\ADV{\GMU{adv}}] viharatī”[dwell-\PRESIND{\GMU{3-sg-presind}}] ti.[-\NUL{\GMU{}}]+ Evañca[thus-\ADV{\GMU{adv}}] te[you-\DAT{\GMU{dat-sg-n}}] bhikkhū[bhikkhu-\NOM{\GMU{nom-pl}}] bhikkhūhi[bhikkhu-\INS{\GMU{ins-pl}}] vuccamānā[address-\PRES{\GMU{pres-pass-part}}] tath’eva[in same way-\NUL{\GMU{}}] paggaṇheyyuṁ,[uphold-\OPT{\GMU{3-pl-opt}}] te[you-\DAT{\GMU{dat-sg-n}}] bhikkhū[bhikkhu-\NOM{\GMU{nom-pl}}] bhikkhūhi[bhikkhu-\INS{\GMU{ins-pl}}] yāvatatiyaṁ[up to.3rd time-\ADV{\GMU{adv}}] samanubhāsitabbā[admonish-\FUT{\GMU{fut-pass-part}}] tassa[of that-\GEN{\GMU{gen-sg}}] paṭinissaggāya.[relinquish-\DAT{\GMU{dat-sg}}] Yāvatatiyañce[up to.3rd time-\ADV{\GMU{adv}}] samanubhāsiyamānā[admonish-\PRES{\GMU{pres-part}}] taṁ[that-\ACC{\GMU{acc-sg}}] paṭinissajjeyyuṁ,[relinquish-\OPT{\GMU{3-pl-opt}}] icc’etaṁ[thus.this-\ACC{\GMU{acc-sg}}] kusalaṁ.[good-\NOM{\GMU{nom-sg-n}}] No[not-\NEG{\GMU{neg-part}}] ce[if-\NUL{\GMU{}}] paṭinissajjeyyuṁ,[relinquish-\OPT{\GMU{3-pl-opt}}] saṅghādiseso.[-\NUL{\GMU{}}]
\endgl
\switchcolumn*
\end{flushleft}


{\EnglishColumn

\begin{doublespace}
12. Now, a bhikkhu is of a nature difficult to be spoken to, (and when) being righteously spoken to by the bhikkhus about the training precepts included in the recitation, he makes himself (one) who can not be spoken to (saying): “Venerables, don't say anything good or bad to me, and I too shall not say anything good or bad to the venerables! Venerables, refrain from speaking to me!”\\
(Then) that bhikkhu should be spoken to thus by the bhikkhus: “Let the venerable one one not make himself (one) who cannot be spoken to. Let the venerable one make himself (one) who can be spoken to. Let the venerable one speak to the bhikkhus with righteousness and the monks too will speak to the venerable one with righteousness. For the Blessed One's assembly has grown thus, that is, by the speaking of one to another, by the rehabilitating of one another,”\\
and (if) that bhikkhu being spoken to thus by the bhikkhus should persist in the same way (as before), (then) that bhikkhu is to be argued with up to three times by the bhikkhus for the relinquishing of that (course), (and if that bhikkhu,) being argued with up to three times, should relinquish that (course), then this is good, (but) if he should not relinquish (it): (this is a case concerning) the community in the beginning and in the rest (of the procedure).
\end{doublespace}}

\switchcolumn


\begin{flushleft}
\begingl
 12.[] Bhikkhu[bhikkhu-\NOM{\GMU{nom-sg}}] pan’eva[now.if-\PART{\GMU{part}}] dubbacajātiko[diff.speak.nature-\ADJ{\GMU{adj}}] hoti,[he is-\PRESIND{\GMU{3-sg-presind}}] uddesapariyāpannesu[recitation.included-\PAST{\GMU{past-part}}] sikkhāpadesu[train.rule-\LOC{\GMU{loc-pl-n}}] bhikkhūhi[bhikkhu-\INS{\GMU{ins-pl}}] sahadhammikaṁ[with.dhamma-\ADJ{\GMU{adj}}] vuccamāno[address-\PRES{\GMU{pres-pass-part}}] attānaṁ[himself-\ACC{\GMU{acc-sg}}] avacanīyaṁ[not say-\FUT{\GMU{fut-past-part}}] karoti,[make-\PRESIND{\GMU{3-sg-presind}}] “Mā[do not-\PART{\GMU{part}}] maṁ[measure-\ACC{\GMU{acc-sg-n}}] āyasmanto[Ven.-\VOC{\GMU{voc-pl}}] kiñci[some-\PRO{\GMU{pro}}] avacuttha[say-\AOR{\GMU{2-pl-aor}}] kalyāṇaṁ[good-\ADJ{\GMU{adj}}] vā[or-\IND{\GMU{ind}}] pāpakaṁ[bad-\ADJ{\GMU{adj}}] vā.[or-\IND{\GMU{ind}}] Aham’p’āyasmante[] na[not-\PART{\GMU{part}}] kiñci[some-\PRO{\GMU{pro}}] vakkhāmi[admonish-\FUT{\GMU{1-sg-fut}}] kalyāṇaṁ[good-\ADJ{\GMU{adj}}] vā[or-\IND{\GMU{ind}}] pāpakaṁ[bad-\ADJ{\GMU{adj}}] vā.[or-\IND{\GMU{ind}}] Viramath’āyasmanto[refrain.friend-\IMP{\GMU{2-pl-imp}}] mama[measure-\DAT{\GMU{dat-sg}}] vacanāyā”[speak-\DAT{\GMU{dat-sg-n}}] ti.[-\NUL{\GMU{}}]+ So[he-\NOM{\GMU{nom-sg}}] bhikkhu[bhikkhu-\NOM{\GMU{nom-sg}}] bhikkhūhi[bhikkhu-\INS{\GMU{ins-pl}}] evam’assa[thus-\TBD{\GMU{tbd}}] vacanīyo,[address-\FUT{\GMU{fut-pass-part}}] “Mā[do not-\PART{\GMU{part}}] āyasmā[Ven.-\NOM{\GMU{nom-sg}}] attānaṁ[himself-\ACC{\GMU{acc-sg}}] avacanīyaṁ[not say-\FUT{\GMU{fut-past-part}}] akāsi.[make-\AOR{\GMU{2-sg-aor}}] Vacanīyam’eva[spoken to.just-\ADJ{\GMU{adj}}] āyasmā[Ven.-\NOM{\GMU{nom-sg}}] attānaṁ[himself-\ACC{\GMU{acc-sg}}] karotu.[make-\IMP{\GMU{3-sg-imp}}] Āyasmā’pi[] bhikkhū[bhikkhu-\NOM{\GMU{nom-pl}}] vadetu[say-\IMP{\GMU{3-sg-imp}}] sahadhammena,[with.dhamma-\INS{\GMU{ins-sg}}] bhikkhū’pi[bhikkhu-\NOM{\GMU{nom-pl}}] āyasmantaṁ[Ven.-\ACC{\GMU{acc-sg}}] vakkhanti[admonish-\FUT{\GMU{3-pl-fut}}] sahadhammena.[with.dhamma-\INS{\GMU{ins-sg}}] Evaṁ[thus-\ADV{\GMU{adv}}] saṁvaddhā[grown-\ADJ{\GMU{adj}}] hi[for-\IND{\GMU{ind}}] tassa[of that-\GEN{\GMU{gen-sg}}] bhagavato[blessed one-\GEN{\GMU{gen-sg}}] parisā,[assembly-\NOM{\GMU{nom-sg-f}}] yad’idaṁ[that is-\IND{\GMU{ind}}] aññamaññavacanena[] aññamaññavuṭṭhāpanenā”[one.another.rehab-\INS{\GMU{ins-sg-n}}] ti.[-\NUL{\GMU{}}]+ Evañca[thus-\ADV{\GMU{adv}}] so[he-\NOM{\GMU{nom-sg}}] bhikkhu[bhikkhu-\NOM{\GMU{nom-sg}}] bhikkhūhi[bhikkhu-\INS{\GMU{ins-pl}}] vuccamāno[address-\PRES{\GMU{pres-pass-part}}] tath’eva[in same way-\NUL{\GMU{}}] paggaṇheyya,[uphold-\OPT{\GMU{3-sg-opt}}] so[he-\NOM{\GMU{nom-sg}}] bhikkhu[bhikkhu-\NOM{\GMU{nom-sg}}] bhikkhūhi[bhikkhu-\INS{\GMU{ins-pl}}] yāvatatiyaṁ[up to.3rd time-\ADV{\GMU{adv}}] samanubhāsitabbo[admonish-\FUT{\GMU{fut-pass-part}}] tassa[of that-\GEN{\GMU{gen-sg}}] paṭinissaggāya.[relinquish-\DAT{\GMU{dat-sg}}] Yāvatatiyañce[up to.3rd time-\ADV{\GMU{adv}}] samanubhāsiyamāno[admonish-\PRES{\GMU{pres-part}}] taṁ[that-\ACC{\GMU{acc-sg}}] paṭinissajjeyya,[relinquish-\OPT{\GMU{3-sg-opt}}] icc’etaṁ[thus.this-\ACC{\GMU{acc-sg}}] kusalaṁ.[good-\NOM{\GMU{nom-sg-n}}] No[not-\NEG{\GMU{neg-part}}] ce[if-\NUL{\GMU{}}] paṭinissajjeyya,[relinquish-\OPT{\GMU{3-sg-opt}}] saṅghādiseso.[-\NUL{\GMU{}}]
\endgl
\switchcolumn*
\end{flushleft}


{\EnglishColumn

\begin{doublespace}
13. Now, a bhikkhu lives dependent upon a certain village or town who is a spoiler of families, who is of bad behaviour. His bad behaviour is seen and is heard about, and the families spoilt by him are seen and heard about. That bhikkhu is to be spoken to thus by the bhikkhus: “The venerable one is a spoiler of families, one who is of bad behaviour. The bad behaviour of the venerable one is seen and is heard about, and the families spoilt by the venerable one are seen and are heard about. Let the venerable one depart from this dwelling-place! Enough of you dwelling here!”\\
and (if) that bhikkhu being spoken to thus by the bhikkhus should say thus to those bhikkhus: “The bhikkhus are driven by desire; the bhikkhus are driven by anger; the bhikkhus are driven by delusion; the bhikkhus are driven by fear. They banish someone because of this kind of offence, (but) another one they do not banish.”\\
(Then) that bhikkhu is to be spoken to thus by the bhikkhus: “Let the venerable one not speak thus! The bhikkhus are not driven by desire; and the bhikkhus are not driven by anger; and the bhikkhus are not driven by delusion; and the bhikkhus are not driven by fear. The venerable one is a spoiler of families, one who is of bad behaviour. The bad  behaviour  of  the venerable  one  is  seen and  is  heard  about,  and  the families   spoilt  by the venerable  one are seen and are heard  about.  Let  the venerable  one  depart  from  this  dwelling-place! Enough of you dwelling here!”\\
and (if) that bhikkhu being spoken to thus by the bhikkhus should persist in the same way (as before), (then) that bhikkhu is to be argued with up to three times by the bhikkhus for the relinquishing of that (course), (and if that bhikkhu,) being argued with up to three times, should relinquish that (course), then this is good, (but) if he should not relinquish (it): (this is a case concerning) the community in the beginning and in the rest (of the procedure).
\end{doublespace}}

\switchcolumn


\begin{flushleft}
\begingl
 13.[] Bhikkhu[bhikkhu-\NOM{\GMU{nom-sg}}] pan’eva[now.if-\PART{\GMU{part}}] aññataraṁ[any one, another-\ADJ{\GMU{adj}}] gāmaṁ[village-\ACC{\GMU{acc-sg}}] vā[or-\IND{\GMU{ind}}] nigamaṁ[town-\ACC{\GMU{acc-sg}}] vā[or-\IND{\GMU{ind}}] upanissāya[depend on-\IND{\GMU{ind}}] viharati[dwell-\PRESIND{\GMU{3-sg-presind}}] kuladūsako[fam.spoil-\ADJ{\GMU{adj}}] pāpasamācāro.[bad.behave-\ADJ{\GMU{adj}}] Tassa[of that-\GEN{\GMU{gen-sg}}] kho[indeed!-\EMPH{\GMU{emph}}] pāpakā[bad-\ADJ{\GMU{adj}}] samācārā[behave-\NOM{\GMU{nom-pl}}] dissanti[see-\PRESIND{\GMU{3-pl-presind}}] c’eva[and.if-\NUL{\GMU{}}] suyyanti[hear-\PRESIND{\GMU{3-pl-presind}}] ca,[-\NUL{\GMU{}}] kulāni[family-\NOM{\GMU{nom}}] ca[-\NUL{\GMU{}}] tena[him-\INS{\GMU{3-sg-ins}}] duṭṭhāni[spoil-\ADJ{\GMU{adj}}] dissanti[see-\PRESIND{\GMU{3-pl-presind}}] c’eva[and.if-\NUL{\GMU{}}] suyyanti[hear-\PRESIND{\GMU{3-pl-presind}}] ca.[-\NUL{\GMU{}}] So[he-\NOM{\GMU{nom-sg}}] bhikkhu[bhikkhu-\NOM{\GMU{nom-sg}}] bhikkhūhi[bhikkhu-\INS{\GMU{ins-pl}}] evam’assa[thus-\TBD{\GMU{tbd}}] vacanīyo,[address-\FUT{\GMU{fut-pass-part}}] “Āyasmā[] kho[indeed!-\EMPH{\GMU{emph}}] kuladūsako[fam.spoil-\ADJ{\GMU{adj}}] pāpasamācāro.[bad.behave-\ADJ{\GMU{adj}}] Āyasmato[] kho[indeed!-\EMPH{\GMU{emph}}] pāpakā[bad-\ADJ{\GMU{adj}}] samācārā[behave-\NOM{\GMU{nom-pl}}] dissanti[see-\PRESIND{\GMU{3-pl-presind}}] c’eva[and.if-\NUL{\GMU{}}] suyyanti[hear-\PRESIND{\GMU{3-pl-presind}}] ca,[-\NUL{\GMU{}}] kulāni[family-\NOM{\GMU{nom}}] c’āyasmatā[Ven.-\INS{\GMU{ins-sg-n}}] duṭṭhāni[spoil-\ADJ{\GMU{adj}}] dissanti[see-\PRESIND{\GMU{3-pl-presind}}] c’eva[and.if-\NUL{\GMU{}}] suyyanti[hear-\PRESIND{\GMU{3-pl-presind}}] ca.[-\NUL{\GMU{}}] Pakkamat’āyasmā[depart.ven-\NOM{\GMU{nom-sg}}] imamhā[from this-\ABL{\GMU{3-sg-abl}}] āvāsā,[dwell-\ABL{\GMU{abl-sg}}] alante[enough.you-\DAT{\GMU{dat-sg}}] idha[here-\ADV{\GMU{adv}}] vāsenā”[dwell-\IND{\GMU{ind-sg-n}}] ti.[-\NUL{\GMU{}}]+ Evañca[thus-\ADV{\GMU{adv}}] so[he-\NOM{\GMU{nom-sg}}] bhikkhu[bhikkhu-\NOM{\GMU{nom-sg}}] bhikkhūhi[bhikkhu-\INS{\GMU{ins-pl}}] vuccamāno[address-\PRES{\GMU{pres-pass-part}}] te[you-\DAT{\GMU{dat-sg-n}}] bhikkhū[bhikkhu-\NOM{\GMU{nom-pl}}] evaṁ[thus-\ADV{\GMU{adv}}] vadeyya,[say-\OPT{\GMU{3-sg-opt}}] “Chandagāmino[desire.go-\ADJ{\GMU{adj}}] ca[-\NUL{\GMU{}}] bhikkhū,[bhikkhu-\NOM{\GMU{nom-pl}}] dosagāmino[hate.go-\ADJ{\GMU{adj}}] ca[-\NUL{\GMU{}}] bhikkhū,[bhikkhu-\NOM{\GMU{nom-pl}}] mohagāmino[delude.go-\ADJ{\GMU{adj}}] ca[-\NUL{\GMU{}}] bhikkhū,[bhikkhu-\NOM{\GMU{nom-pl}}] bhayagāmino[fear.go-\ADJ{\GMU{adj}}] ca[-\NUL{\GMU{}}] bhikkhū,[bhikkhu-\NOM{\GMU{nom-pl}}] tādisikāya[such.seen-\INS{\GMU{ins-sg-f}}] āpattiyā[offense-\INS{\GMU{ins-sg-f}}] ekaccaṁ[same one-\ACC{\GMU{acc-sg-n}}] pabbājenti,[banish-\PRESIND{\GMU{3-pl-presind}}] ekaccaṁ[same one-\ACC{\GMU{acc-sg-n}}] na[not-\PART{\GMU{part}}] pabbājentī”[banish-\PRESIND{\GMU{3-pl-presind}}] ti.[-\NUL{\GMU{}}]+ So[he-\NOM{\GMU{nom-sg}}] bhikkhu[bhikkhu-\NOM{\GMU{nom-sg}}] bhikkhūhi[bhikkhu-\INS{\GMU{ins-pl}}] evam’assa[thus-\TBD{\GMU{tbd}}] vacanīyo,[address-\FUT{\GMU{fut-pass-part}}] “Mā[do not-\PART{\GMU{part}}] āyasmā[Ven.-\NOM{\GMU{nom-sg}}] evaṁ[thus-\ADV{\GMU{adv}}] avaca.[say-\NUL{\GMU{}}] Na[not-\PART{\GMU{part}}] ca[-\NUL{\GMU{}}] bhikkhū[bhikkhu-\NOM{\GMU{nom-pl}}] chandagāmino,[desire.go-\ADJ{\GMU{adj}}] na[not-\PART{\GMU{part}}] ca[-\NUL{\GMU{}}] bhikkhū[bhikkhu-\NOM{\GMU{nom-pl}}] dosagāmino,[hate.go-\ADJ{\GMU{adj}}] na[not-\PART{\GMU{part}}] ca[-\NUL{\GMU{}}] bhikkhū[bhikkhu-\NOM{\GMU{nom-pl}}] mohagāmino,[delude.go-\ADJ{\GMU{adj}}] na[not-\PART{\GMU{part}}] ca[-\NUL{\GMU{}}] bhikkhū[bhikkhu-\NOM{\GMU{nom-pl}}] bhayagāmino.[fear.go-\ADJ{\GMU{adj}}] Āyasmā[] kho[indeed!-\EMPH{\GMU{emph}}] kuladūsako[fam.spoil-\ADJ{\GMU{adj}}] pāpasamācāro.[bad.behave-\ADJ{\GMU{adj}}] Āyasmato[] kho[indeed!-\EMPH{\GMU{emph}}] pāpakā[bad-\ADJ{\GMU{adj}}] samācārā[behave-\NOM{\GMU{nom-pl}}] dissanti[see-\PRESIND{\GMU{3-pl-presind}}] c’eva[and.if-\NUL{\GMU{}}] suyyanti[hear-\PRESIND{\GMU{3-pl-presind}}] ca,[-\NUL{\GMU{}}] kulāni[family-\NOM{\GMU{nom}}] c’āyasmatā[Ven.-\INS{\GMU{ins-sg-n}}] duṭṭhāni[spoil-\ADJ{\GMU{adj}}] dissanti[see-\PRESIND{\GMU{3-pl-presind}}] c’eva[and.if-\NUL{\GMU{}}] suyyanti[hear-\PRESIND{\GMU{3-pl-presind}}] ca.[-\NUL{\GMU{}}] Pakkamat’āyasmā[depart.ven-\NOM{\GMU{nom-sg}}] imamhā[from this-\ABL{\GMU{3-sg-abl}}] āvāsā,[dwell-\ABL{\GMU{abl-sg}}] alan’te[enough.you-\DAT{\GMU{dat-sg}}] idha[here-\ADV{\GMU{adv}}] vāsenā”[dwell-\IND{\GMU{ind-sg-n}}] ti.[-\NUL{\GMU{}}]+ Evañca[thus-\ADV{\GMU{adv}}] so[he-\NOM{\GMU{nom-sg}}] bhikkhu[bhikkhu-\NOM{\GMU{nom-sg}}] bhikkhūhi[bhikkhu-\INS{\GMU{ins-pl}}] vuccamāno[address-\PRES{\GMU{pres-pass-part}}] tath’eva[in same way-\NUL{\GMU{}}] paggaṇheyya,[uphold-\OPT{\GMU{3-sg-opt}}] so[he-\NOM{\GMU{nom-sg}}] bhikkhu[bhikkhu-\NOM{\GMU{nom-sg}}] bhikkhūhi[bhikkhu-\INS{\GMU{ins-pl}}] yāvatatiyaṁ[up to.3rd time-\ADV{\GMU{adv}}] samanubhāsitabbo[admonish-\FUT{\GMU{fut-pass-part}}] tassa[of that-\GEN{\GMU{gen-sg}}] paṭinissaggāya.[relinquish-\DAT{\GMU{dat-sg}}] Yāvatatiyañce[up to.3rd time-\ADV{\GMU{adv}}] samanubhāsiyamāno[admonish-\PRES{\GMU{pres-part}}] taṁ[that-\ACC{\GMU{acc-sg}}] paṭinissajjeyya,[relinquish-\OPT{\GMU{3-sg-opt}}] icc’etaṁ[thus.this-\ACC{\GMU{acc-sg}}] kusalaṁ.[good-\NOM{\GMU{nom-sg-n}}] No[not-\NEG{\GMU{neg-part}}] ce[if-\NUL{\GMU{}}] paṭinissajjeyya,[relinquish-\OPT{\GMU{3-sg-opt}}] saṅghādiseso.[-\NUL{\GMU{}}]
\endgl
\switchcolumn*
\end{flushleft}


{\EnglishColumn

\begin{doublespace}
Venerables, the thirteen cases (concerning) the community in the beginning and in the rest (of the procedure) have been recited, nine (cases) are of the offence-at-once (-class), four (cases) are of the up-to-the-third (time admonition-class). A bhikkhu who has committed any one of (these offenses), has to stay on probation with no choice (in the matter) for as many days as he knowingly conceals (it). Moreover, by a bhikkhu who has stayed on the probation, a six-night state of deference to (other) bhikkhus has to be entered upon. (When) the bhikkhu (is one by whom) the deference has been performed: wherever there may be a community of bhikkhus, which is a group of twenty (or more bhikkhus), there that bhikkhu should be reinstated. If a community of bhikkhus, which is a group of twenty deficient by even one (bhikkhu), should reinstate that bhikkhu (then) that bhikkhu is not reinstated, and those monks are blameworthy. This is the proper procedure here
\end{doublespace}}

\switchcolumn


\begin{flushleft}
\begingl
 Uddiṭṭhā[recite-\PAST{\GMU{past-part}}] kho[indeed!-\EMPH{\GMU{emph}}] āyasmanto[Ven.-\VOC{\GMU{voc-pl}}] terasa[13-\ADJ{\GMU{adj}}] saṅghādisesā[] dhammā,[rule-\NOM{\GMU{nom-pl}}] nava[9-\NUM{\GMU{num}}] paṭham’āpattikā[once.offense-\TBD{\GMU{tbd}}] cattāro[4-\NUM{\GMU{num}}] yāvatatiyakā.[up to.3rd time-\TBD{\GMU{tbd}}] Yesaṁ[them-\GEN{\GMU{gen-pl}}] bhikkhu[bhikkhu-\NOM{\GMU{nom-sg}}] aññataraṁ[any one, another-\ADJ{\GMU{adj}}] vā[or-\IND{\GMU{ind}}] aññataraṁ[any one, another-\ADJ{\GMU{adj}}] vā[or-\IND{\GMU{ind}}] āpajjitvā[commit-\ABS{\GMU{abs}}] yāvatihaṁ[] jānaṁ[know-\NOM{\GMU{nom-sg}}] paṭicchādeti,[concel-\TBD{\GMU{tbd}}] tāvatihaṁ[] tena[him-\INS{\GMU{3-sg-ins}}] bhikkhunā[bhikkhu-\INS{\GMU{ins-sg}}] akāmā[] parivatthabbaṁ.[] Parivutthaparivāsena[] bhikkhunā[bhikkhu-\INS{\GMU{ins-sg}}] uttariṁ[more-\ADV{\GMU{adv}}] chārattaṁ,[6.night-\TBD{\GMU{tbd}}] bhikkhumānattāya[] paṭipajjitabbaṁ.[] Ciṇṇamānatto[perform.penance-\TBD{\GMU{tbd}}] bhikkhu,[bhikkhu-\NOM{\GMU{nom-sg}}] yattha[wherever-\TBD{\GMU{tbd}}] siyā[be-\OPT{\GMU{3-sg-opt}}] vīsatigaṇo[20.group-\TBD{\GMU{tbd}}] bhikkhusaṅgho,[] tattha[about that-\ADV{\GMU{adv}}] so[he-\NOM{\GMU{nom-sg}}] bhikkhu[bhikkhu-\NOM{\GMU{nom-sg}}] abbhetabbo.[rehabilitate-\TBD{\GMU{tbd}}] Ekena’pi[] ce[if-\NUL{\GMU{}}] ūno[] vīsatigaṇo[20.group-\TBD{\GMU{tbd}}] bhikkhusaṅgho[] taṁ[that-\ACC{\GMU{acc-sg}}] bhikkhuṁ[bhikkhu-\ACC{\GMU{acc-sg}}] abbheyya,[rehabilitate-\TBD{\GMU{tbd}}] so[he-\NOM{\GMU{nom-sg}}] ca[-\NUL{\GMU{}}] bhikkhu[bhikkhu-\NOM{\GMU{nom-sg}}] anabbhito,[restore-\TBD{\GMU{tbd}}] te[you-\DAT{\GMU{dat-sg-n}}] ca[-\NUL{\GMU{}}] bhikkhū[bhikkhu-\NOM{\GMU{nom-pl}}] gārayhā.[blame-\FUT{\GMU{fut-pass-part}}] Ayaṁ[this-\NOM{\GMU{nom-sg}}] tattha[about that-\ADV{\GMU{adv}}] sāmīci.[proper procedure-\NOM{\GMU{nom-sg-f}}]
\endgl
\switchcolumn*
\end{flushleft}


{\EnglishColumn

\begin{doublespace}
Concerning that I ask the venerables: (Are you) pure in this?\\
A second time again I ask: (Are you) pure in this?\\
A third time again I ask: (Are you) pure in this?\\
The venerables are pure in this, therefore there is silence, so do I bear this (in mind).
\end{doublespace}}

\switchcolumn


\begin{flushleft}
\begingl
 Tatth’āyasmante[] pucchāmi:[ask-\PRESIND{\GMU{1-sg-presind}}] Kacci’ttha[] parisuddhā?[pure-\ADJ{\GMU{adj}}]+ Dutiyam’pi[second time-\ACC{\GMU{acc-sg-nt}}] pucchāmi:[ask-\PRESIND{\GMU{1-sg-presind}}] Kacci’ttha[] parisuddhā?[pure-\ADJ{\GMU{adj}}]+ Tatiyam’pi[] pucchāmi:[ask-\PRESIND{\GMU{1-sg-presind}}] Kacci’ttha[] parisuddhā?[pure-\ADJ{\GMU{adj}}]+ Parisuddh’etth’āyasmanto,[] tasmā[therefore-\ABL{\GMU{abl-sg}}] tuṇhī,[silent-\ADV{\GMU{adv}}] evam’etaṁ[thus.this-\ACC{\GMU{acc-sg-n}}] dhārayāmi.[keep in mind-\PRESIND{\GMU{1-sg-presind}}]
\endgl
\switchcolumn*
\end{flushleft}


{\EnglishColumn

\begin{doublespace}
The recitation concerning the community in the beginning and the rest (of the procedure) is finished.
\end{doublespace}}

\switchcolumn


\begin{flushleft}
\begingl
 Saṅghādises’uddeso[] niṭṭhito[]
\endgl
\switchcolumn*
\end{flushleft}


{\EnglishColumn

\begin{doublespace}
Venerables, these two uncertain cases come up for recitation.
\end{doublespace}}

\switchcolumn


\begin{flushleft}
\begingl
 Ime[this-\NOM{\GMU{nom-pl}}] kho[indeed!-\EMPH{\GMU{emph}}] pan’āyasmanto[venerable-\VOC{\GMU{voc-pl}}] dve[2-\NUM{\GMU{num}}] aniyatā[] dhammā[rule-\NOM{\GMU{nom-pl}}] uddesaṁ[recitation-\ACC{\GMU{acc-sg}}] āgacchanti.[come up-\PRESIND{\GMU{3-pl-presind}}]
\endgl
\switchcolumn*
\end{flushleft}


{\EnglishColumn

\begin{doublespace}
1. If any bhikkhu should sit down together with a woman, one (man) with one (woman), privately, on a concealed seat (that is) sufficiently fit for doing (it), (and then if) a female lay-follower whose words can be trusted having seen that, should speak according to one of three cases: according to disqualification, according to what concerns the community in the beginning and in the rest (of the procedure), or according to expiation, (then) the bhikkhu who is admitting the sitting down should be made to do (what is) according to one of three cases: according to disqualification, or according to what concerns the community in the beginning and in the rest (of the procedure), or according to expiation, or according to whatever that female lay-follower whose words can be trusted should say, according to that the bhikkhu is to be made to do. This is an uncertain case.
\end{doublespace}}

\switchcolumn


\begin{flushleft}
\begingl
 1.[] Yo[who-\NOM{\GMU{nom-sg}}] pana[(and)-\PART{\GMU{part}}] bhikkhu[bhikkhu-\NOM{\GMU{nom-sg}}] mātugāmena[woman-\INS{\GMU{ins-sg}}] saddhiṁ[together-\INS{\GMU{ins}}] eko[one-\NUM{\GMU{num}}] ekāya[one-\INS{\GMU{ins-sg-f}}] raho[private-\ADV{\GMU{adv}}] paṭicchanne[seclude-\PAST{\GMU{past-part}}] āsane[seat-\LOC{\GMU{loc-sg-n}}] alaṁkammaniye[fit for doing-\ADJ{\GMU{adj}}] nisajjaṁ[seat-\ACC{\GMU{acc-sg-f}}] kappeyya.[use-\OPT{\GMU{3-sg-opt}}] Tam’enaṁ[] saddheyyavacasā[credible.speach-\ADJ{\GMU{adj}}] upāsikā[f.lay devotee-\NOM{\GMU{nom-sg-f}}] disvā[see-\ABS{\GMU{abs}}] tiṇṇaṁ[3-\GEN{\GMU{gen-m}}] dhammānaṁ[-\NUL{\GMU{}}] aññatarena[a certain-\ADJ{\GMU{adj}}] vadeyya,[say-\OPT{\GMU{3-sg-opt}}] pārājikena[defeat-\ADJ{\GMU{adj}}] vā[or-\IND{\GMU{ind}}] saṅghādisesena[-\INS{\GMU{ins-s}}] vā[or-\IND{\GMU{ind}}] pācittiyena[confess-\ADJ{\GMU{adj}}] vā.[or-\IND{\GMU{ind}}] Nisajjaṁ[seat-\ACC{\GMU{acc-sg-f}}] bhikkhu[bhikkhu-\NOM{\GMU{nom-sg}}] paṭijānamāno[admit-\PRES{\GMU{pres-part}}] tiṇṇaṁ[3-\GEN{\GMU{gen-m}}] dhammānaṁ[-\NUL{\GMU{}}] aññatarena[a certain-\ADJ{\GMU{adj}}] kāretabbo,[make-\FUT{\GMU{fut-pass-part}}] pārājikena[defeat-\ADJ{\GMU{adj}}] vā[or-\IND{\GMU{ind}}] saṅghādisesena[-\INS{\GMU{ins-s}}] vā[or-\IND{\GMU{ind}}] pācittiyena[confess-\ADJ{\GMU{adj}}] vā.[or-\IND{\GMU{ind}}] Yena[with whatever-\ADV{\GMU{adv}}] vā[or-\IND{\GMU{ind}}] sā[that-\NOM{\GMU{nom-f}}] saddheyyavacasā[credible.speach-\ADJ{\GMU{adj}}] upāsikā[f.lay devotee-\NOM{\GMU{nom-sg-f}}] vadeyya,[say-\OPT{\GMU{3-sg-opt}}] tena[him-\INS{\GMU{3-sg-ins}}] so[he-\NOM{\GMU{nom-sg}}] bhikkhu[bhikkhu-\NOM{\GMU{nom-sg}}] kāretabbo.[make-\FUT{\GMU{fut-pass-part}}] Ayaṁ[this-\NOM{\GMU{nom-sg}}] dhammo[case-\NOM{\GMU{nom-sg}}] aniyato.[indefinite-\NOM{\GMU{nom-sg}}]
\endgl
\pagebreak
\switchcolumn*
\end{flushleft}

{\EnglishColumn

\begin{doublespace}
2. But even if the seat is neither concealed nor sufficiently fit for doing it, but is sufficient for speaking suggestively to a woman with depraved words: if any bhikkhu should sit down on such a seat together with a woman—one (man) with one (woman), privately—(and then if) a female lay-follower whose words can be trusted having seen that, should speak according to one of two cases: according to what concerns the community in the beginning and in the rest, or according to expiation, (then) the bhikkhu admitting the sitting down is to be made to do according to one of two cases: according to what concerns the community in the beginning and in the rest (of the procedure), or according to expiation, or according to whatever that female lay-follower whose words can be trusted should say, according to that the bhikkhu is to be made to do, this too is an uncertain case.
\end{doublespace}}

\switchcolumn


\begin{flushleft}
\begingl
 2.[] Na[not-\PART{\GMU{part}}] h’eva[-\NUL{\GMU{}}] kho[indeed!-\EMPH{\GMU{emph}}] pana[(and)-\PART{\GMU{part}}] paṭicchannaṁ[seclude-\PAST{\GMU{past-part}}] āsanaṁ[seat-\NOM{\GMU{nom-sg-n}}] hoti[he is-\PRESIND{\GMU{3-sg-presind}}] nālaṁkammaniyaṁ.[fit for doing-\ADJ{\GMU{adj}}] Alañca[?-\NUL{\GMU{}}] kho[indeed!-\EMPH{\GMU{emph}}] hoti[he is-\PRESIND{\GMU{3-sg-presind}}] mātugāmaṁ[woman-\ACC{\GMU{acc-sg}}] duṭṭhullāhi[obscene-\ADJ{\GMU{adj}}] vācāhi[word-\INS{\GMU{ins-pl-f}}] obhāsituṁ.[suggest-\INF{\GMU{inf}}] Yo[who-\NOM{\GMU{nom-sg}}] pana[(and)-\PART{\GMU{part}}] bhikkhu[bhikkhu-\NOM{\GMU{nom-sg}}] tathārūpe[such kind-\ADJ{\GMU{adj}}] āsane[seat-\LOC{\GMU{loc-sg-n}}] mātugāmena[woman-\INS{\GMU{ins-sg}}] saddhiṁ[together-\INS{\GMU{ins}}] eko[one-\NUM{\GMU{num}}] ekāya[one-\INS{\GMU{ins-sg-f}}] raho[private-\ADV{\GMU{adv}}] nisajjaṁ[seat-\ACC{\GMU{acc-sg-f}}] kappeyya.[use-\OPT{\GMU{3-sg-opt}}] Tam’enaṁ[] saddheyyavacasā[credible.speach-\ADJ{\GMU{adj}}] upāsikā[f.lay devotee-\NOM{\GMU{nom-sg-f}}] disvā[see-\ABS{\GMU{abs}}] dvinnaṁ[2-\GEN{\GMU{gen-pl}}] dhammānaṁ[-\NUL{\GMU{}}] aññatarena[a certain-\ADJ{\GMU{adj}}] vadeyya,[say-\OPT{\GMU{3-sg-opt}}] saṅghādisesena[-\INS{\GMU{ins-s}}] vā[or-\IND{\GMU{ind}}] pācittiyena[confess-\ADJ{\GMU{adj}}] vā.[or-\IND{\GMU{ind}}] Nisajjaṁ[seat-\ACC{\GMU{acc-sg-f}}] bhikkhu[bhikkhu-\NOM{\GMU{nom-sg}}] paṭijānamāno[admit-\PRES{\GMU{pres-part}}] dvinnaṁ[2-\GEN{\GMU{gen-pl}}] dhammānaṁ[-\NUL{\GMU{}}] aññatarena[a certain-\ADJ{\GMU{adj}}] kāretabbo,[make-\FUT{\GMU{fut-pass-part}}] saṅghādisesena[-\INS{\GMU{ins-s}}] vā[or-\IND{\GMU{ind}}] pācittiyena[confess-\ADJ{\GMU{adj}}] vā.[or-\IND{\GMU{ind}}] Yena[with whatever-\ADV{\GMU{adv}}] vā[or-\IND{\GMU{ind}}] sā[that-\NOM{\GMU{nom-f}}] saddheyyavacasā[credible.speach-\ADJ{\GMU{adj}}] upāsikā[f.lay devotee-\NOM{\GMU{nom-sg-f}}] vadeyya,[say-\OPT{\GMU{3-sg-opt}}] tena[him-\INS{\GMU{3-sg-ins}}] so[he-\NOM{\GMU{nom-sg}}] bhikkhu[bhikkhu-\NOM{\GMU{nom-sg}}] kāretabbo.[make-\FUT{\GMU{fut-pass-part}}] Ayam’pi[] dhammo[case-\NOM{\GMU{nom-sg}}] aniyato.[indefinite-\NOM{\GMU{nom-sg}}]
\endgl
\switchcolumn*
\end{flushleft}


{\EnglishColumn

\begin{doublespace}
Venerables, the two uncertain cases have been recited.\\
Concerning that I ask the Venerables: (Are you) pure in this?\\
A second time again I ask: (Are you) pure in this?\\
A third time again I ask: (Are you) pure in this?\\
The venerables are pure in this, therefore there is silence, thus I bear this (in mind).
\end{doublespace}}

\switchcolumn


\begin{flushleft}
\begingl
 Uddiṭṭhā[recite-\PAST{\GMU{past-part}}] kho[indeed!-\EMPH{\GMU{emph}}] āyasmanto[Ven.-\VOC{\GMU{voc-pl}}] dve[2-\NUM{\GMU{num}}] aniyatā[] dhammā.[rule-\NOM{\GMU{nom-pl}}]+ Tatth’āyasmante[] pucchāmi:[ask-\PRESIND{\GMU{1-sg-presind}}] Kacci’ttha[] parisuddhā?[pure-\ADJ{\GMU{adj}}]+ Dutiyam’pi[second time-\ACC{\GMU{acc-sg-nt}}] pucchāmi:[ask-\PRESIND{\GMU{1-sg-presind}}] Kacci’ttha[] parisuddhā?[pure-\ADJ{\GMU{adj}}]+ Tatiyam’pi[] pucchāmi:[ask-\PRESIND{\GMU{1-sg-presind}}] Kacci’ttha[] parisuddhā?[pure-\ADJ{\GMU{adj}}]+ Parisuddh’etth’āyasmanto,[] tasmā[therefore-\ABL{\GMU{abl-sg}}] tuṇhī,[silent-\ADV{\GMU{adv}}] evam’etaṁ[thus.this-\ACC{\GMU{acc-sg-n}}] dhārayāmi.[keep in mind-\PRESIND{\GMU{1-sg-presind}}]
\endgl
\switchcolumn*
\end{flushleft}


{\EnglishColumn

\begin{doublespace}
The recitation of the uncertain (cases) is finished.
\end{doublespace}}

\switchcolumn


\begin{flushleft}
\begingl
 Aniyat’uddeso[] niṭṭhito[]
\endgl
\switchcolumn*
\end{flushleft}


{\EnglishColumn

\begin{doublespace}
Venerables, these thirty cases involving expiation with forfeiture come up for recitation.
\end{doublespace}}

\switchcolumn


\begin{flushleft}
\begingl
 Ime[this-\NOM{\GMU{nom-pl}}] kho[indeed!-\EMPH{\GMU{emph}}] pan’āyasmanto[venerable-\VOC{\GMU{voc-pl}}] tiṁsa[] nissaggiyā[] pācittiyā[] dhammā[rule-\NOM{\GMU{nom-pl}}] uddesaṁ[recitation-\ACC{\GMU{acc-sg}}] āgacchanti.[come up-\PRESIND{\GMU{3-pl-presind}}]
\endgl
\switchcolumn*
\end{flushleft}


{\EnglishColumn

\begin{doublespace}
1. When the robe (-cloth) has been finished by a bhikkhu, when the kaþhina (-frame-privileges) have been withdrawn, (then) extra robe (-cloth) is to be kept for ten days at the most. For one who lets it pass beyond (the ten days), (this is a case) involving expiation with forfeiture.
\end{doublespace}}

\switchcolumn


\begin{flushleft}
\begingl
 1.[] Niṭṭhitacīvarasmiṁ[finish.robe-\LOC{\GMU{loc-sg-n}}] bhikkhunā[bhikkhu-\INS{\GMU{ins-sg}}] ubbhatasmiṁ[withdraw-\LOC{\GMU{loc-sg-n}}] kaṭhine,[robeframe-\LOC{\GMU{loc-sg-n}}] dasāhaparamaṁ[10.days.at most-\ADV{\GMU{adv}}] atirekacīvaraṁ[extra cloth-\ACC{\GMU{acc-sg-n}}] dhāretabbaṁ.[keep-\FUT{\GMU{fut-pass-part}}] Taṁ[that-\ACC{\GMU{acc-sg}}] atikkāmayato,[beyond.go-\DAT{\GMU{dat-pres-part}}] nissaggiyaṁ[relinquish-\ADJ{\GMU{adj}}] pācittiyaṁ.[confess-\ADJ{\GMU{adj}}]
\endgl
\switchcolumn*
\end{flushleft}


{\EnglishColumn

\begin{doublespace}
2. When the robe (-cloth) has been finished by a bhikkhu, when the kaþhina (-frame-privileges) have been withdrawn, if even for a single night a bhikkhu should stay apart from the three robes, except with the authorization of bhikkhus, (this is a case) involving expiation with forfeiture.
\end{doublespace}}

\switchcolumn


\begin{flushleft}
\begingl
 2.[] Niṭṭhitacīvarasmiṁ[finish.robe-\LOC{\GMU{loc-sg-n}}] bhikkhunā[bhikkhu-\INS{\GMU{ins-sg}}] ubbhatasmiṁ[withdraw-\LOC{\GMU{loc-sg-n}}] kaṭhine,[robeframe-\LOC{\GMU{loc-sg-n}}] ekarattam’pi[one night-\ACC{\GMU{acc-sg-n}}] ce[if-\NUL{\GMU{}}] bhikkhu[bhikkhu-\NOM{\GMU{nom-sg}}] ticīvarena[3.robes-\INS{\GMU{ins-sg-pl}}] vippavaseyya,[dwell apart-\OPT{\GMU{3-sg-opt}}] aññatra[unless-\ABL{\GMU{abl}}] bhikkhusammatiyā,[bhikkhu.consent-\INS{\GMU{ins-sg}}] nissaggiyaṁ[relinquish-\ADJ{\GMU{adj}}] pācittiyaṁ.[confess-\ADJ{\GMU{adj}}]
\endgl
\switchcolumn*
\end{flushleft}


{\EnglishColumn

\begin{doublespace}
3. When the robe (-cloth) has been finished by a bhikkhu, when the kaþhina (-frame-privileges) have been withdrawn, if out-of-season robe (-cloth) should become available to a bhikkhu, by a bhikkhu who is wishing (so, it) can be accepted; having accepted (it, it) is to be made very quickly. If (the robe-cloth) should not be (enough for) the completion (of the robe), (then) for a month at the most that robe (-cloth) can be put aside by that bhikkhu for the completion of the deficiency (of robe-cloth), when there is an expectation (that he will get more robe-cloth); if he should put (it) aside more than that, even when there is an expectation (that he will get more robe-cloth), (this is a case) involving expiation with forfeiture.
\end{doublespace}}

\switchcolumn


\begin{flushleft}
\begingl
 3.[] Niṭṭhitacīvarasmiṁ[finish.robe-\LOC{\GMU{loc-sg-n}}] bhikkhunā[bhikkhu-\INS{\GMU{ins-sg}}] ubbhatasmiṁ[withdraw-\LOC{\GMU{loc-sg-n}}] kaṭhine,[robeframe-\LOC{\GMU{loc-sg-n}}] bhikkhuno[bhikkhu-\DAT{\GMU{dat-sg}}] pan’eva[now.if-\PART{\GMU{part}}] akālacīvaraṁ[wrong.time.cloth-\ACC{\GMU{acc-sg-n}}] uppajjeyya,[available-\OPT{\GMU{3-sg-opt}}] ākaṅkhamānena[wish for-\ADJ{\GMU{adj-pres-part}}] bhikkhunā[bhikkhu-\INS{\GMU{ins-sg}}] paṭiggahetabbaṁ.[receive-\FUT{\GMU{fut-pass-part}}] Paṭiggahetvā[accept-\ABS{\GMU{abs}}] khippam’eva[quick-\ADV{\GMU{adv}}] kāretabbaṁ.[make-\FUT{\GMU{fut-pass-part}}] No[not-\NEG{\GMU{neg-part}}] c’assa[if.it be-\OPT{\GMU{3-sg-opt}}] pāripūri,[completion-\NOM{\GMU{nom-sg-f}}] māsaparaman’tena[month.at most.that-\TBD{\GMU{tbd}}] bhikkhunā[bhikkhu-\INS{\GMU{ins-sg}}] taṁ[that-\ACC{\GMU{acc-sg}}] cīvaraṁ[robe-\ACC{\GMU{acc-sg-n}}] nikkhipitabbaṁ,ūnassa[] pāripūriyā[completion-\DAT{\GMU{dat-sg-f}}] satiyā[exist-\PRES{\GMU{pres-part}}] paccāsāya.[expect-\LOC{\GMU{loc-sg-f}}] Tato[then-\ABL{\GMU{abl}}] ce[if-\NUL{\GMU{}}] uttariṁ[more-\ADV{\GMU{adv}}] nikkhipeyya[lay aside-\OPT{\GMU{3-sg-opt}}] satiyā’pi[exist-\PRES{\GMU{pres-part}}] paccāsāya,[expect-\LOC{\GMU{loc-sg-f}}] nissaggiyaṁ[relinquish-\ADJ{\GMU{adj}}] pācittiyaṁ.[confess-\ADJ{\GMU{adj}}]
\endgl
\switchcolumn*
\end{flushleft}


{\EnglishColumn

\begin{doublespace}
4. If any bhikkhu should have a used robe (-cloth) washed, dyed, or beaten by an unrelated bhikkhunì, (this is a case) involving expiation with forfeiture.)
\end{doublespace}}

\switchcolumn


\begin{flushleft}
\begingl
 4.[] Yo[who-\NOM{\GMU{nom-sg}}] pana[(and)-\PART{\GMU{part}}] bhikkhu[bhikkhu-\NOM{\GMU{nom-sg}}] aññātikāya[unrelated-\ADJ{\GMU{adj}}] bhikkhuniyā[bhikkhuni-\INS{\GMU{ins-sg-f}}] purāṇacīvaraṁ[old.robe-\ACC{\GMU{acc-sg}}] dhovāpeyya[wash-\OPT{\GMU{3-sg-opt}}] vā[or-\IND{\GMU{ind}}] rajāpeyya[dye-\OPT{\GMU{3-sg-opt}}] vā[or-\IND{\GMU{ind}}] ākoṭāpeyya[beat-\OPT{\GMU{3-sg-opt}}] vā,[or-\IND{\GMU{ind}}] nissaggiyaṁ[relinquish-\ADJ{\GMU{adj}}] pācittiyaṁ.[confess-\ADJ{\GMU{adj}}]
\endgl
\switchcolumn*
\end{flushleft}


{\EnglishColumn

\begin{doublespace}
5. If any bhikkhu should accept a robe (-cloth) from the hand of an unrelated bhikkhunì, except in an exchange (of robes), (this is a case) involving expiation with forfeiture.
\end{doublespace}}

\switchcolumn


\begin{flushleft}
\begingl
 5.[] Yo[who-\NOM{\GMU{nom-sg}}] pana[(and)-\PART{\GMU{part}}] bhikkhu[bhikkhu-\NOM{\GMU{nom-sg}}] aññātikāya[unrelated-\ADJ{\GMU{adj}}] bhikkhuniyā[bhikkhuni-\INS{\GMU{ins-sg-f}}] hatthato[hand-\ABL{\GMU{abl-sg}}] cīvaraṁ[robe-\ACC{\GMU{acc-sg-n}}] paṭiggaṇheyya[receive-\OPT{\GMU{3-sg-opt}}] aññatra[unless-\ABL{\GMU{abl}}] pārivaṭṭakā,[exchange-\INS{\GMU{ins-sg}}] nissaggiyaṁ[relinquish-\ADJ{\GMU{adj}}] pācittiyaṁ.[confess-\ADJ{\GMU{adj}}]
\endgl
\switchcolumn*
\end{flushleft}


{\EnglishColumn

\begin{doublespace}
6. If any bhikkhu should request a robe (-cloth) to an unrelated male householder or female householder, except at the (right) occasion, (this is a case) involving expiation with forfeiture. Here the occasion is this: he is a bhikkhu whose robe has been robbed or whose robe has been lost; this is the occasion here.
\end{doublespace}}

\switchcolumn


\begin{flushleft}
\begingl
 6.[] Yo[who-\NOM{\GMU{nom-sg}}] pana[(and)-\PART{\GMU{part}}] bhikkhu[bhikkhu-\NOM{\GMU{nom-sg}}] aññātakaṁ[unrelated-\ADJ{\GMU{adj}}] gahapatiṁ[householder.m-\ACC{\GMU{acc-sg}}] vā[or-\IND{\GMU{ind}}] gahapatāniṁ[householder.f-\ACC{\GMU{acc-sg-f}}] vā[or-\IND{\GMU{ind}}] cīvaraṁ[robe-\ACC{\GMU{acc-sg-n}}] viññāpeyya[request-\OPT{\GMU{3-sg-opt}}] aññatra[unless-\ABL{\GMU{abl}}] samayā,[time-\ABL{\GMU{abl-sg}}] nissaggiyaṁ[relinquish-\ADJ{\GMU{adj}}] pācittiyaṁ.[confess-\ADJ{\GMU{adj}}] Tatth’āyaṁ[here.this-\NOM{\GMU{nom-sg}}] samayo:[time-\NOM{\GMU{nom-sg}}] Acchinnacīvaro[robbed.robe-\ADJ{\GMU{adj}}] vā[or-\IND{\GMU{ind}}] hoti[he is-\PRESIND{\GMU{3-sg-presind}}] bhikkhu[bhikkhu-\NOM{\GMU{nom-sg}}] naṭṭhacīvaro[lost.robe-\ADJ{\GMU{adj}}] vā.[or-\IND{\GMU{ind}}] Ayaṁ[this-\NOM{\GMU{nom-sg}}] tattha[about that-\ADV{\GMU{adv}}] samayo.[time-\NOM{\GMU{nom-sg}}]
\endgl
\switchcolumn*
\end{flushleft}


{\EnglishColumn

\begin{doublespace}
7. If the unrelated male householder or female householder should invite him to take (as many) robe (-cloth)s (as he likes), (then) robe (-cloths for) an upper (robe) together with an inner (robe) can be accepted at the most from that robe (-cloth) by that bhikkhu; if he should accept more from that (robe-cloth), (this is a case) involving expiation with forfeiture.
\end{doublespace}}

\switchcolumn


\begin{flushleft}
\begingl
 7.[] Tañce[him-\ACC{\GMU{acc-sg}}] aññātako[unrelated-\ADJ{\GMU{adj}}] gahapati[householder.m-\NOM{\GMU{nom-sg}}] vā[or-\IND{\GMU{ind}}] gahapatānī[householder.f-\NOM{\GMU{nom-sg-f}}] vā[or-\IND{\GMU{ind}}] bahūhi[many-\ADJ{\GMU{adj}}] cīvarehi[robe-\INS{\GMU{ins-pl-n}}] abhihaṭṭhuṁ[take-\INF{\GMU{inf}}] pavāreyya,[invite-\OPT{\GMU{3-sg-opt}}] santaruttaraparaman’tena[with.inner.outter.at most.that-\ACC{\GMU{acc-sg-n}}] bhikkhunā[bhikkhu-\INS{\GMU{ins-sg}}] tato[then-\ABL{\GMU{abl}}] cīvaraṁ[robe-\ACC{\GMU{acc-sg-n}}] sāditabbaṁ.[accept-\FUT{\GMU{fut-pass-part}}] Tato[then-\ABL{\GMU{abl}}] ce[if-\NUL{\GMU{}}] uttariṁ[more-\ADV{\GMU{adv}}] sādiyeyya,[accept-\OPT{\GMU{3-sg-opt}}] nissaggiyaṁ[relinquish-\ADJ{\GMU{adj}}] pācittiyaṁ.[confess-\ADJ{\GMU{adj}}]
\endgl
\switchcolumn*
\end{flushleft}


{\EnglishColumn

\begin{doublespace}
8. Now, if an robe-fund has been set up for a bhikkhu by an unrelated male householder or female householder (thinking): “Having traded this robe-fund for a robe, I shall clothe the bhikkhu named so and so with a robe,”\\
and then if that bhikkhu, previously uninvited, having approached (the householder), should make a suggestion about the robe (-cloth) (saying): “It would be good indeed, Sir, (if you) having traded this robe-fund for a such and such a robe, were to clothe me (with a robe),” (if the suggestion is made) out of a liking for what is fine, (this is a case) involving expiation with forfeiture.
\end{doublespace}}

\switchcolumn


\begin{flushleft}
\begingl
 8.[] Bhikkhuṁ[bhikkhu-\ACC{\GMU{acc-sg}}] pan’eva[now.if-\PART{\GMU{part}}] uddissa[for-\IND{\GMU{ind}}] aññātakassa[unrelated-\ADJ{\GMU{adj}}] gahapatissa[householder.m-\GEN{\GMU{gen-sg}}] vā[or-\IND{\GMU{ind}}] gahapatāniyā[householder.f-\GEN{\GMU{gen-sg-f}}] vā[or-\IND{\GMU{ind}}] cīvaracetāpanaṁ[robe.fund-\NOM{\GMU{nom-sg-n}}] upakkhaṭaṁ[setup-\ADJ{\GMU{adj}}] hoti,[he is-\PRESIND{\GMU{3-sg-presind}}] “Iminā[this-\INS{\GMU{ins-sg-n}}] cīvaracetāpanena[robe.fund-\ACC{\GMU{acc-sg-n}}] cīvaraṁ[robe-\ACC{\GMU{acc-sg-n}}] cetāpetvā[exchange-\ABS{\GMU{abs}}] itthannāmaṁ[such name-\ADJ{\GMU{adj}}] bhikkhuṁ[bhikkhu-\ACC{\GMU{acc-sg}}] cīvarena[robe-\INS{\GMU{ins-sg-n}}] acchādessāmī”[clothe-\FUT{\GMU{1-sg-fut}}] ti.[-\NUL{\GMU{}}]+ Tatra[then-\ADV{\GMU{adv}}] ce[if-\NUL{\GMU{}}] so[he-\NOM{\GMU{nom-sg}}] bhikkhu[bhikkhu-\NOM{\GMU{nom-sg}}] pubbe[previous-\ADV{\GMU{adv}}] appavārito[uninvite-\PAST{\GMU{past-part}}] upasaṅkamitvā[approach-\ABS{\GMU{abs}}] cīvare[robe-\LOC{\GMU{loc-sg-n}}] vikappaṁ[suggest-\ACC{\GMU{acc-sg-n}}] āpajjeyya,[engage-\OPT{\GMU{3-sg-opt}}] “Sādhu[good-\IND{\GMU{ind}}] vata[indeed!-\EMPH{\GMU{emph}}] maṁ[measure-\ACC{\GMU{acc-sg-n}}] āyasmā[Ven.-\NOM{\GMU{nom-sg}}] iminā[this-\INS{\GMU{ins-sg-n}}] cīvaracetāpanena,[robe.fund-\ACC{\GMU{acc-sg-n}}] evarūpaṁ[likethis.that-\ADJ{\GMU{adj}}] vā[or-\IND{\GMU{ind}}] evarūpaṁ[likethis.that-\ADJ{\GMU{adj}}] vā[or-\IND{\GMU{ind}}] cīvaraṁ[robe-\ACC{\GMU{acc-sg-n}}] cetāpetvā[exchange-\ABS{\GMU{abs}}] acchādehī”[clothe-\IMP{\GMU{2-sg-imp}}] ti,[-\NUL{\GMU{}}] kalyāṇakamyataṁ[fine.liking-\ACC{\GMU{acc-sg-n}}] upādāya,[take up-\ABS{\GMU{abs}}] nissaggiyaṁ[relinquish-\ADJ{\GMU{adj}}] pācittiyaṁ.[confess-\ADJ{\GMU{adj}}]
\endgl
\switchcolumn*
\end{flushleft}


{\EnglishColumn

\begin{doublespace}
9. Now, if separate robe-funds have been set up for a bhikkhu by both unrelated male householders or female householders (thinking): “Having traded these separate robe-funds for separate robes, we shall clothe the bhikkhu named so and so with robes,”\\
and then if that bhikkhu, previously uninvited, having approached (the householders), should make a suggestion about the robe (saying): “It would be good indeed, Sirs, (if you) having traded these separate robe-funds for a such and such a robe, were to clothe me (with a robe), (you) both being one (donor),” (if the suggestion is made) out of a liking for what is fine, (this is a case) involving expiation with forfeiture.
\end{doublespace}}

\switchcolumn


\begin{flushleft}
\begingl
 9.[] Bhikkhuṁ[bhikkhu-\ACC{\GMU{acc-sg}}] pan’eva[now.if-\PART{\GMU{part}}] uddissa[for-\IND{\GMU{ind}}] ubhinnaṁ[both-\ADJ{\GMU{adj}}] aññātakānaṁ[unrelated-\ADJ{\GMU{adj}}] gahapatīnaṁ[householder.m-\GEN{\GMU{gen-pl}}] vā[or-\IND{\GMU{ind}}] gahapatānīnaṁ[householder.f-\GEN{\GMU{gen-pl-f}}] vā[or-\IND{\GMU{ind}}] paccekacīvaracetāpanā[separate.robe.fun-\NOM{\GMU{nom-pl-n}}] upakkhaṭā[setup-\ADJ{\GMU{adj}}] honti,[there are-\PRESIND{\GMU{3-pl-presind}}] “Imehi[this-\INS{\GMU{ins-pl-n}}] mayaṁ[we-\NOM{\GMU{nom-pl}}] paccekacīvaracetāpanehi[separate.robe.fun-\INS{\GMU{ins-pl-n}}] paccekacīvarāni[separate.robe-\NOM{\GMU{nom-pl-n}}] cetāpetvā[exchange-\ABS{\GMU{abs}}] itthannāmaṁ[such name-\ADJ{\GMU{adj}}] bhikkhuṁ[bhikkhu-\ACC{\GMU{acc-sg}}] cīvarehi[robe-\INS{\GMU{ins-pl-n}}] acchādessāmā”[clothe-\FUT{\GMU{1-pl-fut}}] ti.[-\NUL{\GMU{}}]+ Tatra[then-\ADV{\GMU{adv}}] ce[if-\NUL{\GMU{}}] so[he-\NOM{\GMU{nom-sg}}] bhikkhu[bhikkhu-\NOM{\GMU{nom-sg}}] pubbe[previous-\ADV{\GMU{adv}}] appavārito[uninvite-\PAST{\GMU{past-part}}] upasaṅkamitvā[approach-\ABS{\GMU{abs}}] cīvare[robe-\LOC{\GMU{loc-sg-n}}] vikappaṁ[suggest-\ACC{\GMU{acc-sg-n}}] āpajjeyya,[engage-\OPT{\GMU{3-sg-opt}}] “Sādhu[good-\IND{\GMU{ind}}] vata[indeed!-\EMPH{\GMU{emph}}] maṁ[measure-\ACC{\GMU{acc-sg-n}}] āyasmanto[Ven.-\VOC{\GMU{voc-pl}}] imehi[this-\INS{\GMU{ins-pl-n}}] paccekacīvaracetāpanehi,[separate.robe.fun-\INS{\GMU{ins-pl-n}}] evarūpaṁ[likethis.that-\ADJ{\GMU{adj}}] vā[or-\IND{\GMU{ind}}] evarūpaṁ[likethis.that-\ADJ{\GMU{adj}}] vā[or-\IND{\GMU{ind}}] cīvaraṁ[robe-\ACC{\GMU{acc-sg-n}}] cetāpetvā[exchange-\ABS{\GMU{abs}}] acchādetha[clothe-\IMP{\GMU{2-pl-imp}}] ubho’va[] santā[exist-\PRES{\GMU{pres-part}}] ekenā[one-\INS{\GMU{ins}}] ”[-\NUL{\GMU{}}] ti,[-\NUL{\GMU{}}] kalyāṇakamyataṁ[fine.liking-\ACC{\GMU{acc-sg-n}}] upādāya,[take up-\ABS{\GMU{abs}}] nissaggiyaṁ[relinquish-\ADJ{\GMU{adj}}] pācittiyaṁ.[confess-\ADJ{\GMU{adj}}]
\endgl
\switchcolumn*
\end{flushleft}


{\EnglishColumn

\begin{doublespace}
10. Now, if a king or a kings’ official or a brahmin or a male householder should convey by messenger a robe-fund for a bhikkhu (saying): “Having traded this robe-fund for a robe, clothe the bhikkhu named so and so with a robe,” \\
and if that messenger, having approached that bhikkhu, should say so: “Venerable Sir, this robe-fund has been brought for the venerable one. Let the venerable one accept the robe-fund!”\\
(then) that messenger should be spoken to thus by that bhikkhu: “Sir, we do not accept a robe-fund, but we do accept a robe at the right time (when it is) allowable.”\\
If that messenger should say thus to that bhikkhu: “Is there, perhaps, someone who is the steward of the venerable one?” (then,) bhikkhus, by a bhikkhu who is in need of a robe, a steward can be appointed: a monastery attendant or a male lay-follower (saying): “Sir, this is the bhikkhus' steward.”\\
If that messenger having instructed that steward, having approached that bhikkhu, should say so: “Venerable Sir, the steward whom the venerable one has appointed has been instructed by me. Let the venerable one approach (him) at the right time (and) he will clothe you with a robe,” (then) bhikkhus, having approached the steward, (the steward) can be prompted (and) can be reminded two or three times by the bhikkhu who is in need of a robe (saying): “Sir, I am in need of a robe.”\\
(If through) prompting (and) reminding (him) two or three times, he should have (him) bring forth that robe, it is good. If he should not have (him) bring (it) forth, (then) four times, five times, six times at the most, (it) can be stood (for) by (a bhikkhu) who has become silent. (If through) standing silently for (it) four times, five times, six times at the most, he should have (him) bring forth that robe, it is good; if (through) making effort more than that, he should have (him) produce that robe, (this is a case) involving expiation with forfeiture.\\
If he should not have (him) produce (it), (then) from wherever (that) the robe-fund may have been brought, there (he) himself can go, or a messenger can be sent (saying): “Sirs, that robe-fund which you conveyed for the bhikkhu does not fulfil any need of that bhikkhu. Let the sirs endeavour for (what is their) own. Let not (what is their) own get lost.” This is the proper procedure here.
\end{doublespace}}

\switchcolumn


\begin{flushleft}
\begingl
 10.[] Bhikkhuṁ[bhikkhu-\ACC{\GMU{acc-sg}}] pan’eva[now.if-\PART{\GMU{part}}] uddissa[for-\IND{\GMU{ind}}] rājā[king-\NOM{\GMU{nom-sg}}] vā[or-\IND{\GMU{ind}}] rājabhoggo[king official-\NOM{\GMU{nom-sg}}] vā[or-\IND{\GMU{ind}}] brāhmaṇo[brahmin-\NOM{\GMU{nom-sg}}] vā[or-\IND{\GMU{ind}}] gahapatiko[householder.m-\NOM{\GMU{nom-sg}}] vā[or-\IND{\GMU{ind}}] dūtena[messenger-\INS{\GMU{ins-sg}}] cīvaracetāpanaṁ[robe.fund-\NOM{\GMU{nom-sg-n}}] pahiṇeyya,[convey-\OPT{\GMU{3-sg-opt}}] “Iminā[this-\INS{\GMU{ins-sg-n}}] cīvaracetāpanena[robe.fund-\ACC{\GMU{acc-sg-n}}] cīvaraṁ[robe-\ACC{\GMU{acc-sg-n}}] cetāpetvā[exchange-\ABS{\GMU{abs}}] itthannāmaṁ[such name-\ADJ{\GMU{adj}}] bhikkhuṁ[bhikkhu-\ACC{\GMU{acc-sg}}] cīvarena[robe-\INS{\GMU{ins-sg-n}}] acchādehī”[clothe-\IMP{\GMU{2-sg-imp}}] ti.[-\NUL{\GMU{}}]+ So[he-\NOM{\GMU{nom-sg}}] ce[if-\NUL{\GMU{}}] dūto[messenger-\NOM{\GMU{nom-sg}}] taṁ[that-\ACC{\GMU{acc-sg}}] bhikkhuṁ[bhikkhu-\ACC{\GMU{acc-sg}}] upasaṅkamitvā[approach-\ABS{\GMU{abs}}] evaṁ[thus-\ADV{\GMU{adv}}] vadeyya,[say-\OPT{\GMU{3-sg-opt}}] “Idaṁ[this-\ACC{\GMU{acc-sg-n}}] kho[indeed!-\EMPH{\GMU{emph}}] bhante[venerable sir-\VOC{\GMU{voc-sg}}] āyasmantaṁ[Ven.-\ACC{\GMU{acc-sg}}] uddissa[for-\IND{\GMU{ind}}] cīvaracetāpanaṁ[robe.fund-\NOM{\GMU{nom-sg-n}}] ābhataṁ.[bring-\PAST{\GMU{past-part}}] Paṭiggaṇhātu[receive-\IMP{\GMU{3-sg-imp}}] āyasmā[Ven.-\NOM{\GMU{nom-sg}}] cīvaracetāpanan”[robe.fund-\ACC{\GMU{acc-sg-n}}] ti.[-\NUL{\GMU{}}]+ Tena[him-\INS{\GMU{3-sg-ins}}] bhikkhunā[bhikkhu-\INS{\GMU{ins-sg}}] so[he-\NOM{\GMU{nom-sg}}] dūto[messenger-\NOM{\GMU{nom-sg}}] evam’assa[thus-\TBD{\GMU{tbd}}] vacanīyo,[address-\FUT{\GMU{fut-pass-part}}] “Na[not-\PART{\GMU{part}}] kho[indeed!-\EMPH{\GMU{emph}}] mayaṁ[we-\NOM{\GMU{nom-pl}}] āvuso[friend-\VOC{\GMU{voc-sg}}] cīvaracetāpanaṁ[robe.fund-\NOM{\GMU{nom-sg-n}}] paṭiggaṇhāma,[receive-\PRESIND{\GMU{1-pl-presind}}] cīvarañ[] ca[-\NUL{\GMU{}}] kho[indeed!-\EMPH{\GMU{emph}}] mayaṁ[we-\NOM{\GMU{nom-pl}}] paṭiggaṇhāma[receive-\PRESIND{\GMU{1-pl-presind}}] kālena[time-\ADV{\GMU{adv}}] kappiyan”[allow-\ADJ{\GMU{adj}}] ti.[-\NUL{\GMU{}}]+ So[he-\NOM{\GMU{nom-sg}}] ce[if-\NUL{\GMU{}}] dūto[messenger-\NOM{\GMU{nom-sg}}] taṁ[that-\ACC{\GMU{acc-sg}}] bhikkhuṁ[bhikkhu-\ACC{\GMU{acc-sg}}] evaṁ[thus-\ADV{\GMU{adv}}] vadeyya,[say-\OPT{\GMU{3-sg-opt}}] “Atthi[has-\PRESIND{\GMU{3-sg-presind}}] pan’āyasmato[then.venerable-\DAT{\GMU{dat-sg}}] koci[someone-\PRO{\GMU{pro}}] veyyāvaccakaro”[service.do-\NOM{\GMU{nom-sg}}] ti.[-\NUL{\GMU{}}] Cīvar’atthikena[robe.need-\ADJ{\GMU{adj}}] bhikkhave[bhikkhu-\VOC{\GMU{voc-pl}}] bhikkhunā[bhikkhu-\INS{\GMU{ins-sg}}] veyyāvaccakaro[service.do-\NOM{\GMU{nom-sg}}] niddisitabbo,[appoint-\FUT{\GMU{fut-pass-part}}] ārāmiko[attendant-\NOM{\GMU{nom-sg}}] vā[or-\IND{\GMU{ind}}] upāsako[m.lay devotee-\NOM{\GMU{nom-sg}}] vā,[or-\IND{\GMU{ind}}] “Eso[this one-\NOM{\GMU{nom-sg}}] kho[indeed!-\EMPH{\GMU{emph}}] āvuso[friend-\VOC{\GMU{voc-sg}}] bhikkhūnaṁ[bhikkhu-\DAT{\GMU{dat-pl}}] veyyāvaccakaro”[service.do-\NOM{\GMU{nom-sg}}] ti.[-\NUL{\GMU{}}]+ So[he-\NOM{\GMU{nom-sg}}] ce[if-\NUL{\GMU{}}] dūto[messenger-\NOM{\GMU{nom-sg}}] taṁ[that-\ACC{\GMU{acc-sg}}] veyyāvaccakaraṁ[service.do-\ACC{\GMU{acc-sg}}] saññāpetvā[instruct-\ABS{\GMU{abs}}] taṁ[that-\ACC{\GMU{acc-sg}}] bhikkhuṁ[bhikkhu-\ACC{\GMU{acc-sg}}] upasaṅkamitvā[approach-\ABS{\GMU{abs}}] evaṁ[thus-\ADV{\GMU{adv}}] vadeyya,[say-\OPT{\GMU{3-sg-opt}}] “Yaṁ[that-\ACC{\GMU{acc-sg}}] kho[indeed!-\EMPH{\GMU{emph}}] bhante[venerable sir-\VOC{\GMU{voc-sg}}] āyasmā[Ven.-\NOM{\GMU{nom-sg}}] veyyāvaccakaraṁ[service.do-\ACC{\GMU{acc-sg}}] niddisi,[appoint-\AOR{\GMU{3-sg-aor}}] saññatto[instruct-\PAST{\GMU{past-part}}] so[he-\NOM{\GMU{nom-sg}}] mayā.[me-\INS{\GMU{ins-sg}}] Upasaṅkamatu[approach-\IMP{\GMU{3-sg-imp}}] āyasmā[Ven.-\NOM{\GMU{nom-sg}}] kālena[time-\ADV{\GMU{adv}}] cīvarena[robe-\INS{\GMU{ins-sg-n}}] taṁ[that-\ACC{\GMU{acc-sg}}] acchādessatī”[clothe-\FUT{\GMU{3-sg-fut}}] ti.[-\NUL{\GMU{}}] Cīvar’atthikena[robe.need-\ADJ{\GMU{adj}}] bhikkhave[bhikkhu-\VOC{\GMU{voc-pl}}] bhikkhunā[bhikkhu-\INS{\GMU{ins-sg}}] veyyāvaccakaro[service.do-\NOM{\GMU{nom-sg}}] upasaṅkamitvā[approach-\ABS{\GMU{abs}}] dvittikkhattuṁ[2.or.3.times-\ADV{\GMU{adv}}] codetabbo[prompt-\FUT{\GMU{fut-pass-part}}] sāretabbo,[remind-\FUT{\GMU{fut-pass-part}}] “Attho[need-\NOM{\GMU{nom-sg}}] me[me-\DAT{\GMU{dat-sg}}] āvuso[friend-\VOC{\GMU{voc-sg}}] cīvarenā”[robe-\INS{\GMU{ins-sg}}] ti.[-\NUL{\GMU{}}]+ Dvittikkhattuṁ[2.or.3.times-\ADV{\GMU{adv}}] codayamāno[prompt-\PRES{\GMU{pres-part}}] sārayamāno[remind-\PRES{\GMU{pres-part}}] taṁ[that-\ACC{\GMU{acc-sg}}] cīvaraṁ[robe-\ACC{\GMU{acc-sg-n}}] abhinipphādeyya,[produce-\OPT{\GMU{3-sg-opt}}] icc’etaṁ[thus.this-\ACC{\GMU{acc-sg}}] kusalaṁ.[good-\NOM{\GMU{nom-sg-n}}] No[not-\NEG{\GMU{neg-part}}] ce[if-\NUL{\GMU{}}] abhinipphādeyya,[produce-\OPT{\GMU{3-sg-opt}}] catukkhattuṁ[4.times-\ADV{\GMU{adv}}] pañcakkhattuṁ[5 times-\ADV{\GMU{adv}}] chakkhattuparamaṁ[6.times.at most-\NUL{\GMU{}}] tuṇhībhūtena[silent.become-\INS{\GMU{ins-sg}}] uddissa[for-\IND{\GMU{ind}}] ṭhātabbaṁ.[stand-\FUT{\GMU{fut-pass-part}}] Catukkhattuṁ[4.times-\ADV{\GMU{adv}}] pañcakkhattuṁ[5 times-\ADV{\GMU{adv}}] chakkhattuparamaṁ[6.times.at most-\NUL{\GMU{}}] tuṇhībhūto[silent.become-\NOM{\GMU{nom-sg}}] uddissa[for-\IND{\GMU{ind}}] tiṭṭhamāno[stand-\PRES{\GMU{pres-part}}] taṁ[that-\ACC{\GMU{acc-sg}}] cīvaraṁ[robe-\ACC{\GMU{acc-sg-n}}] abhinipphādeyya,[produce-\OPT{\GMU{3-sg-opt}}] icc’etaṁ[thus.this-\ACC{\GMU{acc-sg}}] kusalaṁ.[good-\NOM{\GMU{nom-sg-n}}] No[not-\NEG{\GMU{neg-part}}] ce[if-\NUL{\GMU{}}] abhinipphādeyya,[produce-\OPT{\GMU{3-sg-opt}}] tato[then-\ABL{\GMU{abl}}] ce[if-\NUL{\GMU{}}] uttariṁ[more-\ADV{\GMU{adv}}] vāyamamāno[exert-\NUL{\GMU{}}] taṁ[that-\ACC{\GMU{acc-sg}}] cīvaraṁ[robe-\ACC{\GMU{acc-sg-n}}] abhinipphādeyya,[produce-\OPT{\GMU{3-sg-opt}}] nissaggiyaṁ[relinquish-\ADJ{\GMU{adj}}] pācittiyaṁ.[confess-\ADJ{\GMU{adj}}]+ No[not-\NEG{\GMU{neg-part}}] ce[if-\NUL{\GMU{}}] abhinipphādeyya,[produce-\OPT{\GMU{3-sg-opt}}] yatassa[from.be-\NUL{\GMU{}}] cīvaracetāpanaṁ[robe.fund-\NOM{\GMU{nom-sg-n}}] ābhataṁ,[bring-\PAST{\GMU{past-part}}] tattha[about that-\ADV{\GMU{adv}}] sāmaṁ[himself-\ADV{\GMU{adv}}] vā[or-\IND{\GMU{ind}}] gantabbaṁ,[go-\FUT{\GMU{fut-pass-part}}] dūto[messenger-\NOM{\GMU{nom-sg}}] vā[or-\IND{\GMU{ind}}] pāhetabbo,[send-\FUT{\GMU{fut-pass-part}}] “Yaṁ[that-\ACC{\GMU{acc-sg}}] kho[indeed!-\EMPH{\GMU{emph}}] tumhe[you-\VOC{\GMU{2-pl-voc}}] āyasmanto[Ven.-\VOC{\GMU{voc-pl}}] bhikkhuṁ[bhikkhu-\ACC{\GMU{acc-sg}}] uddissa[for-\IND{\GMU{ind}}] cīvaracetāpanaṁ[robe.fund-\NOM{\GMU{nom-sg-n}}] pahiṇittha.[convey-\AOR{\GMU{2-pl-aor}}] Na[not-\PART{\GMU{part}}] tantassa[that.that-\TBD{\GMU{tbd}}] bhikkhuno[bhikkhu-\DAT{\GMU{dat-sg}}] kiñci[some-\PRO{\GMU{pro}}] atthaṁ[need-\ACC{\GMU{acc-sg}}] anubhoti.[fulfil-\PRESIND{\GMU{3-sg-presind}}] Yuñjant’āyasmanto[endeavor.ven-\IMP{\GMU{3-pl-imp}}] sakaṁ.[own-\ADJ{\GMU{adj}}] Mā[do not-\PART{\GMU{part}}] vo[you-\GEN{\GMU{gen-pl}}] sakaṁ[own-\ADJ{\GMU{adj}}] vinassī”[lose-\IMP{\GMU{3-sg-imp}}] ti.[-\NUL{\GMU{}}] Ayaṁ[this-\NOM{\GMU{nom-sg}}] tattha[about that-\ADV{\GMU{adv}}] sāmīci.[proper procedure-\NOM{\GMU{nom-sg-f}}]
\endgl
\switchcolumn*
\end{flushleft}


{\EnglishColumn

\begin{doublespace}
The section (starting with the rule) on robes is first.
\end{doublespace}}

\switchcolumn


\begin{flushleft}
\begingl
 Cīvaravaggo[] paṭhamo.[first-\ADJ{\GMU{adj}}]
\endgl
\switchcolumn*
\end{flushleft}


{\EnglishColumn

\begin{doublespace}
11. If any bhikkhu should have a rug mixed with silk made, (this is a case) involving expiation with forfeiture.
\end{doublespace}}

\switchcolumn


\begin{flushleft}
\begingl
 11.[] Yo[who-\NOM{\GMU{nom-sg}}] pana[(and)-\PART{\GMU{part}}] bhikkhu[bhikkhu-\NOM{\GMU{nom-sg}}] kosiyamissakaṁ[silk.mix-\ADJ{\GMU{adj}}] santhataṁ[rug spread-\ACC{\GMU{acc-sg-n}}] kārāpeyya,[make-\OPT{\GMU{3-sg-opt}}] nissaggiyaṁ[relinquish-\ADJ{\GMU{adj}}] pācittiyaṁ.[confess-\ADJ{\GMU{adj}}]
\endgl
\switchcolumn*
\end{flushleft}


{\EnglishColumn

\begin{doublespace}
12. If any bhikkhu should have a rug made of pure black sheep's wool; (this is a case) involving expiation with forfeiture.
\end{doublespace}}

\switchcolumn


\begin{flushleft}
\begingl
 12.[] Yo[who-\NOM{\GMU{nom-sg}}] pana[(and)-\PART{\GMU{part}}] bhikkhu[bhikkhu-\NOM{\GMU{nom-sg}}] suddhakāḷakānaṁ[pure black-\ADJ{\GMU{adj}}] eḷakalomānaṁ[sheep.wool-\GEN{\GMU{gen-pl-n}}] santhataṁkārāpeyya,[] nissaggiyaṁ[relinquish-\ADJ{\GMU{adj}}] pācittiyaṁ.[confess-\ADJ{\GMU{adj}}]
\endgl
\switchcolumn*
\end{flushleft}


{\EnglishColumn

\begin{doublespace}
13. By a bhikkhu who is having a new rug made, two parts of pure black sheep-wool are to be taken, (and) a third (part) of white, a fourth (part) of ruddy brown. If a bhikkhu should have a rug made, without having taken two parts of pure black sheep's hair, (and) a third (part) of white, a fourth (part) of ruddy brown, (this is a case) involving expiation with forfeiture.
\end{doublespace}}

\switchcolumn


\begin{flushleft}
\begingl
 13.[] Navam’pana[new.-\ADJ{\GMU{adj}}] bhikkhunā[bhikkhu-\INS{\GMU{ins-sg}}] santhataṁ[rug spread-\ACC{\GMU{acc-sg-n}}] kārayamānena,[build-\PRES{\GMU{pres-part}}] dve[2-\NUM{\GMU{num}}] bhāgā[part-\NOM{\GMU{nom-pl}}] suddhakāḷakānaṁ[pure black-\ADJ{\GMU{adj}}] eḷakalomānaṁ[sheep.wool-\GEN{\GMU{gen-pl-n}}] ādātabbā,[take-\FUT{\GMU{fut-pass-part}}] tatiyaṁ[third time-\ORD{\GMU{ord}}] odātānaṁ[white-\ADJ{\GMU{adj}}] catutthaṁ[a fourth-\ORD{\GMU{ord}}] gocariyānaṁ.[brown-\ADJ{\GMU{adj}}] Anādā[not.take-\ABS{\GMU{abs}}] ce[if-\NUL{\GMU{}}] bhikkhu[bhikkhu-\NOM{\GMU{nom-sg}}] dve[2-\NUM{\GMU{num}}] bhāge[part-\ACC{\GMU{acc-pl}}] suddhakāḷakānaṁ[pure black-\ADJ{\GMU{adj}}] eḷakalomānaṁ,[sheep.wool-\GEN{\GMU{gen-pl-n}}] tatiyaṁ[third time-\ORD{\GMU{ord}}] odātānaṁ[white-\ADJ{\GMU{adj}}] catutthaṁ[a fourth-\ORD{\GMU{ord}}] gocariyānaṁ[brown-\ADJ{\GMU{adj}}] navaṁ[new-\ADJ{\GMU{adj}}] santhataṁ[rug spread-\ACC{\GMU{acc-sg-n}}] kārāpeyya,[make-\OPT{\GMU{3-sg-opt}}] nissaggiyaṁ[relinquish-\ADJ{\GMU{adj}}] pācittiyaṁ.[confess-\ADJ{\GMU{adj}}]
\endgl
\switchcolumn*
\end{flushleft}


{\EnglishColumn

\begin{doublespace}
14. By a bhikkhu who has had a new rug made, it is to be kept for six years (at least). If within less than six years, having given up or not having given up that rug, he should have another new rug made, except with the authorisation of bhikkhus, (this is a case) involving expiation with forfeiture.
\end{doublespace}}

\switchcolumn


\begin{flushleft}
\begingl
 14.[] Navam’pana[new.-\ADJ{\GMU{adj}}] bhikkhunā[bhikkhu-\INS{\GMU{ins-sg}}] santhataṁ[rug spread-\ACC{\GMU{acc-sg-n}}] kārāpetvā[make-\ABS{\GMU{abs}}] chabbassāni[6.years-\NOM{\GMU{nom-pl-n}}] dhāretabbaṁ.[keep-\FUT{\GMU{fut-pass-part}}] Orena[less-\INS{\GMU{ins-sg-n}}] ce[if-\NUL{\GMU{}}] channaṁ[6-\ADJ{\GMU{adj}}] vassānaṁ[year-\GEN{\GMU{gen-pl-n}}] taṁ[that-\ACC{\GMU{acc-sg}}] santhataṁ[rug spread-\ACC{\GMU{acc-sg-n}}] vissajjetvā[give up-\ABS{\GMU{abs}}] vā[or-\IND{\GMU{ind}}] avissajjetvā[-\NUL{\GMU{}}] vā[or-\IND{\GMU{ind}}] aññaṁ[another-\ADJ{\GMU{adj}}] navaṁ[new-\ADJ{\GMU{adj}}] santhataṁ[rug spread-\ACC{\GMU{acc-sg-n}}] kārāpeyya,[make-\OPT{\GMU{3-sg-opt}}] aññatra[unless-\ABL{\GMU{abl}}] bhikkhusammatiyā,[bhikkhu.consent-\INS{\GMU{ins-sg}}] nissaggiyaṁ[relinquish-\ADJ{\GMU{adj}}] pācittiyaṁ.[confess-\ADJ{\GMU{adj}}]
\endgl
\pagebreak
\switchcolumn*
\end{flushleft}


{\EnglishColumn

\begin{doublespace}
15. By a bhikkhu who is having a sitting-rug made, a sugata-span from the border of an old rug is to be taken for making (it) stained. If a bhikkhu, without having taken a sugata-span from the border of an old rug, should have a new sitting cloth made, (this is a case) involving expiation with forfeiture.
\end{doublespace}}

\switchcolumn


\begin{flushleft}
\begingl
 15.[] Nisīdanasanthataṁ[sit.rug-\ACC{\GMU{acc-sg-n}}] pana[(and)-\PART{\GMU{part}}] bhikkhunā[bhikkhu-\INS{\GMU{ins-sg}}] kārayamānena[build-\PRES{\GMU{pres-part}}] purāṇasanthatassa[old.rug-\GEN{\GMU{gen-sg-n}}] sāmantā[all around-\ADV{\GMU{adv}}] sugatavidatthi[] ādātabbā[take-\FUT{\GMU{fut-pass-part}}] dubbaṇṇakaraṇāya.[stain.make-\DAT{\GMU{dat-sg-n}}] Anādā[not.take-\ABS{\GMU{abs}}] ce[if-\NUL{\GMU{}}] bhikkhu[bhikkhu-\NOM{\GMU{nom-sg}}] purāṇasanthatassa[old.rug-\GEN{\GMU{gen-sg-n}}] sāmantā[all around-\ADV{\GMU{adv}}] sugatavidatthiṁ[well.gone.span-\ACC{\GMU{acc-sg-f}}] navaṁ[new-\ADJ{\GMU{adj}}] nisīdanasanthataṁ[sit.rug-\ACC{\GMU{acc-sg-n}}] kārāpeyya,[make-\OPT{\GMU{3-sg-opt}}] nissaggiyaṁ[relinquish-\ADJ{\GMU{adj}}] pācittiyaṁ.[confess-\ADJ{\GMU{adj}}]
\endgl
\switchcolumn*
\end{flushleft}


{\EnglishColumn

\begin{doublespace}
16. Now, if sheep-wool should become available to a bhikkhu who is travelling on a main road, by a bhikkhu who is wishing (so, it) can be accepted, having accepted (it, it) can be carried with his own hand for three yojanas at the most when there is no one present who can carry it; if he should carry it more than that, even when there is no one present who can carry it, (this is a case) involving expiation with forfeiture.
\end{doublespace}}

\switchcolumn


\begin{flushleft}
\begingl
 16.[] Bhikkhuno[bhikkhu-\DAT{\GMU{dat-sg}}] pan’eva[now.if-\PART{\GMU{part}}] addhānamaggapaṭipannassa[main.road.go.along-\ADJ{\GMU{adj}}] eḷakalomāni[sheep.wool-\ACC{\GMU{acc-pl-n}}] uppajjeyyuṁ.[available-\OPT{\GMU{3-pl-opt}}] Ākaṅkhamānena[] bhikkhunā[bhikkhu-\INS{\GMU{ins-sg}}] paṭiggahetabbāni.[accept-\FUT{\GMU{fut-pass-part}}] Paṭiggahetvā[accept-\ABS{\GMU{abs}}] tiyojanaparamaṁ[3.yojana.at most-\ADV{\GMU{adv}}] sahatthā[with.hand-\INS{\GMU{ins-sg}}] hāretabbāni,[carry-\FUT{\GMU{fut-pass-part}}] asante[not.present-\ADJ{\GMU{adj}}] hārake.[carry-\LOC{\GMU{loc-sg}}] Tato[then-\ABL{\GMU{abl}}] ce[if-\NUL{\GMU{}}] uttariṁ[more-\ADV{\GMU{adv}}] hareyya[carry-\OPT{\GMU{3-sg-opt}}] asante’pi[not.present-\ADJ{\GMU{adj}}] hārake,[carry-\LOC{\GMU{loc-sg}}] nissaggiyaṁ[relinquish-\ADJ{\GMU{adj}}] pācittiyaṁ.[confess-\ADJ{\GMU{adj}}]
\endgl
\switchcolumn*
\end{flushleft}


{\EnglishColumn

\begin{doublespace}
17. If any bhikkhu should have sheep-wool washed, dyed, or carded by an unrelated bhikkhunì, (this is a case) involving expiation with forfeiture.
\end{doublespace}}

\switchcolumn


\begin{flushleft}
\begingl
 17.[] Yo[who-\NOM{\GMU{nom-sg}}] pana[(and)-\PART{\GMU{part}}] bhikkhu[bhikkhu-\NOM{\GMU{nom-sg}}] aññātikāya[unrelated-\ADJ{\GMU{adj}}] bhikkhuniyā[bhikkhuni-\INS{\GMU{ins-sg-f}}] eḷakalomāni[sheep.wool-\ACC{\GMU{acc-pl-n}}] dhovāpeyya[wash-\OPT{\GMU{3-sg-opt}}] vā[or-\IND{\GMU{ind}}] rajāpeyya[dye-\OPT{\GMU{3-sg-opt}}] vā[or-\IND{\GMU{ind}}] vijaṭāpeyya[card-\OPT{\GMU{3-sg-opt}}] vā,[or-\IND{\GMU{ind}}] nissaggiyaṁ[relinquish-\ADJ{\GMU{adj}}] pācittiyaṁ.[confess-\ADJ{\GMU{adj}}]
\endgl
\switchcolumn*
\end{flushleft}


{\EnglishColumn

\begin{doublespace}
18. If any bhikkhu should take gold and silver, or should have (it) taken, or should consent to (it) being deposited (for him), (this is a case) involving expiation with forfeiture.
\end{doublespace}}

\switchcolumn


\begin{flushleft}
\begingl
 18.[] Yo[who-\NOM{\GMU{nom-sg}}] pana[(and)-\PART{\GMU{part}}] bhikkhu[bhikkhu-\NOM{\GMU{nom-sg}}] jātarūparajataṁ[gold.silver-\ACC{\GMU{acc-sg-n}}] uggaṇheyya[take-\OPT{\GMU{3-sg-opt}}] vā[or-\IND{\GMU{ind}}] uggaṇhāpeyya[other take-\OPT{\GMU{3-sg-opt}}] vā[or-\IND{\GMU{ind}}] upanikkhittaṁ[place near-\PRES{\GMU{pres-part}}] vā[or-\IND{\GMU{ind}}] sādiyeyya,[accept-\OPT{\GMU{3-sg-opt}}] nissaggiyaṁ[relinquish-\ADJ{\GMU{adj}}] pācittiyaṁ.[confess-\ADJ{\GMU{adj}}]
\endgl
\switchcolumn*
\end{flushleft}


{\EnglishColumn

\begin{doublespace}
19. If any bhikkhu should engage in the various kinds of trading in money, (this is a case) involving expiation with forfeiture.
\end{doublespace}}

\switchcolumn


\begin{flushleft}
\begingl
 19.[] Yo[who-\NOM{\GMU{nom-sg}}] pana[(and)-\PART{\GMU{part}}] bhikkhu[bhikkhu-\NOM{\GMU{nom-sg}}] nānappakārakaṁ[various.kind-\ADJ{\GMU{adj}}] rūpiyasaṁvohāraṁ[money.trade-\ACC{\GMU{acc-sg}}] samāpajjeyya,[enter-\OPT{\GMU{3-sg-opt}}] nissaggiyaṁ[relinquish-\ADJ{\GMU{adj}}] pācittiyaṁ.[confess-\ADJ{\GMU{adj}}]
\endgl
\switchcolumn*
\end{flushleft}


{\EnglishColumn

\begin{doublespace}
20. If any bhikkhu should engage in the various kinds of bartering, (this is a case) involving expiation with forfeiture.
\end{doublespace}}

\switchcolumn


\begin{flushleft}
\begingl
 20.[] Yo[who-\NOM{\GMU{nom-sg}}] pana[(and)-\PART{\GMU{part}}] bhikkhu[bhikkhu-\NOM{\GMU{nom-sg}}] nānappakārakaṁ[various.kind-\ADJ{\GMU{adj}}] kayavikkayaṁ[trade-\ACC{\GMU{acc-sg}}] samāpajjeyya,[enter-\OPT{\GMU{3-sg-opt}}] nissaggiyaṁ[relinquish-\ADJ{\GMU{adj}}] pācittiyaṁ.[confess-\ADJ{\GMU{adj}}]
\endgl
\switchcolumn*
\end{flushleft}


{\EnglishColumn

\begin{doublespace}
The section on sheepwool is second.
\end{doublespace}}

\switchcolumn


\begin{flushleft}
\begingl
 Kosiyavaggo[silk.section-\NOM{\GMU{nom-sg}}] dutiyo[second-\ORD{\GMU{ord}}]
\endgl
\switchcolumn*
\end{flushleft}


{\EnglishColumn

\begin{doublespace}
21. An extra bowl can be kept for ten days at the most. For one who lets it pass beyond (the ten days); (this is a case) involving expiation with forfeiture.
\end{doublespace}}

\switchcolumn


\begin{flushleft}
\begingl
 21.[] Dasāhaparamaṁ[10.days.at most-\ADV{\GMU{adv}}] atirekapatto[extra bowl-\NOM{\GMU{nom-sg}}] dhāretabbo.[keep-\FUT{\GMU{fut-pass-part}}] Taṁ[that-\ACC{\GMU{acc-sg}}] atikkāmayato,[beyond.go-\DAT{\GMU{dat-pres-part}}] nissaggiyaṁ[relinquish-\ADJ{\GMU{adj}}] pācittiyaṁ.[confess-\ADJ{\GMU{adj}}]
\endgl
\switchcolumn*
\end{flushleft}


{\EnglishColumn

\begin{doublespace}
22. If any bhikkhu should exchange a bowl with less than five mends for another new bowl, (this is a case) involving expiation with forfeiture. That bowl is to be relinquished by that bhikkhu to the assembly of bhikkhus, and whichever (bowl) is the last bowl of that assembly of bhikkhus, that (bowl) is to be bestowed on that bhikkhu (thus): “Bhikkhu, this bowl is for you, it is to be kept until breaking.” This is the proper procedure here.
\end{doublespace}}

\switchcolumn


\begin{flushleft}
\begingl
 22.[] Yo[who-\NOM{\GMU{nom-sg}}] pana[(and)-\PART{\GMU{part}}] bhikkhu[bhikkhu-\NOM{\GMU{nom-sg}}] ūnapañcabandhanena[less.5.mends-\ADJ{\GMU{adj}}] pattena[bowl-\INS{\GMU{ins-sg-n}}] aññaṁ[another-\ADJ{\GMU{adj}}] navaṁ[new-\ADJ{\GMU{adj}}] pattaṁ[bowl-\ACC{\GMU{acc-sg}}] cetāpeyya,[exchange-\OPT{\GMU{3-sg-opt}}] nissaggiyaṁ[relinquish-\ADJ{\GMU{adj}}] pācittiyaṁ.[confess-\ADJ{\GMU{adj}}] Tena[him-\INS{\GMU{3-sg-ins}}] bhikkhunā[bhikkhu-\INS{\GMU{ins-sg}}] so[he-\NOM{\GMU{nom-sg}}] patto[bowl-\NOM{\GMU{nom-sg}}] bhikkhuparisāya[bhikkhu.assembly-\DAT{\GMU{dat-sg}}] nissajjitabbo.[relinquish-\ADJ{\GMU{adj}}] Yo[who-\NOM{\GMU{nom-sg}}] ca[-\NUL{\GMU{}}] tassā[that-\ADJ{\GMU{adj}}] bhikkhuparisāya[bhikkhu.assembly-\DAT{\GMU{dat-sg}}] pattapariyanto,[bowl.last-\ADJ{\GMU{adj}}] so[he-\NOM{\GMU{nom-sg}}] ca[-\NUL{\GMU{}}] tassa[of that-\GEN{\GMU{gen-sg}}] bhikkhuno[bhikkhu-\DAT{\GMU{dat-sg}}] padātabbo,[give to-\FUT{\GMU{fut-pass-part}}] “Ayan’te[] bhikkhu[bhikkhu-\NOM{\GMU{nom-sg}}] patto,[bowl-\NOM{\GMU{nom-sg}}] yāva[until-\IND{\GMU{ind}}] bhedanāya[break-\DAT{\GMU{dat-sg-n}}] dhāretabbo”[keep-\FUT{\GMU{fut-pass-part}}] ti.[-\NUL{\GMU{}}] Ayaṁ[this-\NOM{\GMU{nom-sg}}] tattha[about that-\ADV{\GMU{adv}}] sāmīci.[proper procedure-\NOM{\GMU{nom-sg-f}}]
\endgl
\switchcolumn*
\end{flushleft}


{\EnglishColumn

\begin{doublespace}
23. Now, (there are) those medicines which are permissable for sick bhikkhus, namely: ghee, butter, oil, (and) honey and molasses—having been accepted, they can be partaken of (while) being kept in store for seven days at the most. For one who lets it pass beyond (the seven days), (this is a case) involving expiation with forfeiture.
\end{doublespace}}

\switchcolumn


\begin{flushleft}
\begingl
 23.[] Yāni[which-\NOM{\GMU{nom-pl-n}}] kho[indeed!-\EMPH{\GMU{emph}}] pana[(and)-\PART{\GMU{part}}] tāni[those-\NOM{\GMU{nom-pl}}] gilānānaṁ[sick-\ADJ{\GMU{adj}}] bhikkhūnaṁ[bhikkhu-\DAT{\GMU{dat-pl}}] paṭisāyanīyāni[allow-\FUT{\GMU{fut-pass-part}}] bhesajjāni,[medicine-\NOM{\GMU{nom-pl-n}}] seyyathīdaṁ:[as follows-\NOM{\GMU{nom-sg}}] sappi[ghee-\NOM{\GMU{nom-sg-n}}] navanītaṁ[butter-\NOM{\GMU{nom-sg-n}}] telaṁ[oil-\NOM{\GMU{nom-sg}}] madhu[honey-\NOM{\GMU{nom-sg-n}}] phāṇitaṁ;[molasses-\NOM{\GMU{nom-sg}}] tāni[those-\NOM{\GMU{nom-pl}}] paṭiggahetvā[accept-\ABS{\GMU{abs}}] sattāhaparamaṁ[7.days.atmost-\ADV{\GMU{adv}}] sannidhikārakaṁ[store keep-\ABS{\GMU{abs}}] paribhuñjitabbāni.[use-\FUT{\GMU{fut-pass-part}}] Taṁ[that-\ACC{\GMU{acc-sg}}] atikkāmayato,[beyond.go-\DAT{\GMU{dat-pres-part}}] nissaggiyaṁ[relinquish-\ADJ{\GMU{adj}}] pācittiyaṁ.[confess-\ADJ{\GMU{adj}}]
\endgl
\switchcolumn*
\end{flushleft}


{\EnglishColumn

\begin{doublespace}
24. (Thinking:) “One month is what remains of the hot season,” (then) the robe-cloth for the rain's bathing-cloth can be sought by a bhikkhu. (Thinking:) “A half month is what remains of the hot season,” (after) having made (it, it) can be worn. If earlier than (what is reckoned as) “One month is what remains of the hot season,” he should seek robe-cloth for the rain's bathing-cloth, (and) (if) earlier than (what is reckoned as) “A half month is what remains of the hot season,” he should wear (it), (this is a case) involving expiation with forfeiture.
\end{doublespace}}

\switchcolumn


\begin{flushleft}
\begingl
 24.[] “Māso[month-\NOM{\GMU{nom-sg}}] seso[reamin-\NOM{\GMU{nom-sg-n}}] gimhānan”[hot.season-\GEN{\GMU{gen-pl}}] ti[-\NUL{\GMU{}}] bhikkhunā[bhikkhu-\INS{\GMU{ins-sg}}] vassikasāṭikacīvaraṁ[rain.cloth-\ACC{\GMU{acc-sg}}] pariyesitabbaṁ.[seek-\FUT{\GMU{fut-pass-part}}] “Aḍḍhamāso[half month-\NOM{\GMU{nom-sg-n}}] seso[reamin-\NOM{\GMU{nom-sg-n}}] gimhānan”[hot.season-\GEN{\GMU{gen-pl}}] ti[-\NUL{\GMU{}}] katvā[make take-\ABS{\GMU{abs}}] nivāsetabbaṁ.[wear-\FUT{\GMU{fut-pass-part}}] “Orena[less-\INS{\GMU{ins-sg-n}}] ce[if-\NUL{\GMU{}}] māso[month-\NOM{\GMU{nom-sg}}] seso[reamin-\NOM{\GMU{nom-sg-n}}] gimhānan”[hot.season-\GEN{\GMU{gen-pl}}] ti[-\NUL{\GMU{}}] vassikasāṭikacīvaraṁ[rain.cloth-\ACC{\GMU{acc-sg}}] pariyeseyya,[seek-\OPT{\GMU{3-sg-opt}}] “Oren’aḍḍhamāso[less $\sfrac{1}{2}$ month-\NOM{\GMU{nom-sg}}] seso[reamin-\NOM{\GMU{nom-sg-n}}] gimhānan”[hot.season-\GEN{\GMU{gen-pl}}] ti[-\NUL{\GMU{}}] katvā[make take-\ABS{\GMU{abs}}] nivāseyya,[wear-\OPT{\GMU{3-sg-opt}}] nissaggiyaṁ[relinquish-\ADJ{\GMU{adj}}] pācittiyaṁ.[confess-\ADJ{\GMU{adj}}]
\endgl
\switchcolumn*
\end{flushleft}


{\EnglishColumn

\begin{doublespace}
25. If any bhikkhu, having himself given a robe to a bhikkhu, should, being resentful (and) displeased,snatch (it) away or should have it snatched away (from the bhikkhu), (this is a case) involving expiation with forfeiture.
\end{doublespace}}

\switchcolumn


\begin{flushleft}
\begingl
 25.[] Yo[who-\NOM{\GMU{nom-sg}}] pana[(and)-\PART{\GMU{part}}] bhikkhu[bhikkhu-\NOM{\GMU{nom-sg}}] bhikkhussa[bhikkhu-\GEN{\GMU{gen-sg}}] sāmaṁ[himself-\ADV{\GMU{adv}}] cīvaraṁ[robe-\ACC{\GMU{acc-sg-n}}] datvā[give-\ABS{\GMU{abs}}] kupito[disturb-\PAST{\GMU{past-part}}] anattamano[displeased-\ADJ{\GMU{adj}}] acchindeyya[snatch-\OPT{\GMU{3-sg-opt}}] vā[or-\IND{\GMU{ind}}] acchindāpeyya[snatch-\OPT{\GMU{3-sg-opt}}] vā,[or-\IND{\GMU{ind}}] nissaggiyaṁ[relinquish-\ADJ{\GMU{adj}}] pācittiyaṁ.[confess-\ADJ{\GMU{adj}}]
\endgl
\switchcolumn*
\end{flushleft}


{\EnglishColumn

\begin{doublespace}
26. If any bhikkhu, having himself requested the thread (to be used), should have a robe-cloth woven by cloth-weavers, (this is a case) involving expiation with forfeiture.
\end{doublespace}}

\switchcolumn


\begin{flushleft}
\begingl
 26.[] Yo[who-\NOM{\GMU{nom-sg}}] pana[(and)-\PART{\GMU{part}}] bhikkhu[bhikkhu-\NOM{\GMU{nom-sg}}] sāmaṁ[himself-\ADV{\GMU{adv}}] suttaṁ[thread-\ACC{\GMU{acc-sg-n}}] viññāpetvā[request-\ABS{\GMU{abs}}] tantavāyehi[thread.weaver-\INS{\GMU{ins-pl}}] cīvaraṁ[robe-\ACC{\GMU{acc-sg-n}}] vāyāpeyya,[weave-\OPT{\GMU{3-sg-opt}}] nissaggiyaṁ[relinquish-\ADJ{\GMU{adj}}] pācittiyaṁ.[confess-\ADJ{\GMU{adj}}]
\endgl
\switchcolumn*
\end{flushleft}


{\EnglishColumn

\begin{doublespace}
27. Now, if an unrelated male householder or female householder should have a robe-cloth woven for a bhikkhu by cloth-weavers, and then if that bhikkhu, uninvited beforehand, having approached the cloth-weavers, should make a suggestion about the robe-cloth (saying): “Friends, this robe-cloth which is being woven for me: make (it) long, wide, thick, well woven, well diffused, well scraped, and well plucked! Certainly we will also (then) present a little something to the sirs,” and if that bhikkhu, having said so, should present a little something, even just a little alms-food, (this is a case) involving expiation with forfeiture.
\end{doublespace}}

\switchcolumn


\begin{flushleft}
\begingl
 27.[] Bhikkhuṁ[bhikkhu-\ACC{\GMU{acc-sg}}] pan’eva[now.if-\PART{\GMU{part}}] uddissa[for-\IND{\GMU{ind}}] aññātako[unrelated-\ADJ{\GMU{adj}}] gahapati[householder.m-\NOM{\GMU{nom-sg}}] vā[or-\IND{\GMU{ind}}] gahapatānī[householder.f-\NOM{\GMU{nom-sg-f}}] vā[or-\IND{\GMU{ind}}] tantavāyehi[thread.weaver-\INS{\GMU{ins-pl}}] cīvaraṁ[robe-\ACC{\GMU{acc-sg-n}}] vāyāpeyya.[weave-\OPT{\GMU{3-sg-opt}}] Tatra[then-\ADV{\GMU{adv}}] ce[if-\NUL{\GMU{}}] so[he-\NOM{\GMU{nom-sg}}] bhikkhu[bhikkhu-\NOM{\GMU{nom-sg}}] pubbe[previous-\ADV{\GMU{adv}}] appavārito[uninvite-\PAST{\GMU{past-part}}] tantavāye[cloth.weaver-\ACC{\GMU{acc-pl}}] upasaṅkamitvā[approach-\ABS{\GMU{abs}}] cīvare[robe-\LOC{\GMU{loc-sg-n}}] vikappaṁ[suggest-\ACC{\GMU{acc-sg-n}}] āpajjeyya,[engage-\OPT{\GMU{3-sg-opt}}] “Idaṁ[this-\ACC{\GMU{acc-sg-n}}] kho[indeed!-\EMPH{\GMU{emph}}] āvuso[friend-\VOC{\GMU{voc-sg}}] cīvaraṁ[robe-\ACC{\GMU{acc-sg-n}}] maṁ[measure-\ACC{\GMU{acc-sg-n}}] uddissa[for-\IND{\GMU{ind}}] vīyati.[weave-\PASS?{\GMU{pass?}}] Āyatañca[] karotha[make-\IMP{\GMU{2-pl-imp}}] vitthatañca[wide-\ADJ{\GMU{adj}}] appitañca[thick-\ADJ{\GMU{adj}}] suvītañca[well.weave-\PAST{\GMU{past-part}}] supavāyitañca[well.diffuse-\ADJ{\GMU{adj}}] suvilekhitañca[well.scrape-\PAST{\GMU{past-part}}] suvitacchitañca[well.brush-\PAST{\GMU{past-part}}] karotha;[make-\IMP{\GMU{2-pl-imp}}] app’eva[if.only-\EMPH{\GMU{emph-part}}] nāma[indeed!-\EMPH{\GMU{emph}}] mayam’pi[] āyasmantānaṁ[Ven.-\DAT{\GMU{dat-pl}}] kiñcimattaṁ[some.more-\ACC{\GMU{acc-sg}}] anupadajjeyyāmā”[present-\OPT{\GMU{3-sg-opt}}] ti.[-\NUL{\GMU{}}] Evañca[thus-\ADV{\GMU{adv}}] so[he-\NOM{\GMU{nom-sg}}] bhikkhu[bhikkhu-\NOM{\GMU{nom-sg}}] vatvā[say-\ABS{\GMU{abs}}] kiñcimattaṁ[some.more-\ACC{\GMU{acc-sg}}] anupadajjeyya,[present-\OPT{\GMU{1-pl-opt}}] antamaso[even so much as-\IND{\GMU{ind}}] piṇḍapātamattam’pi,[alms food.mere-\ACC{\GMU{acc-sg}}] nissaggiyaṁ[relinquish-\ADJ{\GMU{adj}}] pācittiyaṁ.[confess-\ADJ{\GMU{adj}}]
\endgl
\switchcolumn*
\end{flushleft}


{\EnglishColumn

\begin{doublespace}
28. For the ten days coming up to the three-month Kattiká full moon: if extraordinary robe (-cloth) should become available to a bhikkhu, (then) after considering (it as) extraordinary (robe-cloth, it) can be accepted by a bhikkhu, having been accepted, (it) is to be put aside until the occasion of the robe-season; if he should put (it) aside for more than that, (this is a case) involving expiation with forfeiture.
\end{doublespace}}

\switchcolumn


\begin{flushleft}
\begingl
 28.[] Dasāhānāgataṁ[10.days.not.come-\ADJ{\GMU{adj}}] kattikatemāsipuṇṇamaṁ,[kattika.3.month.full.moon-\ACC{\GMU{acc-sg-f}}] bhikkhuno[bhikkhu-\DAT{\GMU{dat-sg}}] pan’eva[now.if-\PART{\GMU{part}}] accekacīvaraṁ[special.robe-\ACC{\GMU{acc-sg-n}}] uppajjeyya.[available-\OPT{\GMU{3-sg-opt}}] Accekaṁ[special-\ADJ{\GMU{adj}}] maññamānena[consider-\PRES{\GMU{pres-part}}] bhikkhunā[bhikkhu-\INS{\GMU{ins-sg}}] paṭiggahetabbaṁ.[receive-\FUT{\GMU{fut-pass-part}}] Paṭiggahetvā[accept-\ABS{\GMU{abs}}] yāva[until-\IND{\GMU{ind}}] cīvarakālasamayaṁ[robe.make.time-\ACC{\GMU{acc-sg}}] nikkhipitabbaṁ.[lay aside-\FUT{\GMU{fut-pass-part}}] Tato[then-\ABL{\GMU{abl}}] ce[if-\NUL{\GMU{}}] uttariṁ[more-\ADV{\GMU{adv}}] nikkhipeyya,[lay aside-\OPT{\GMU{3-sg-opt}}] nissaggiyaṁ[relinquish-\ADJ{\GMU{adj}}] pācittiyaṁ.[confess-\ADJ{\GMU{adj}}]
\endgl
\switchcolumn*
\end{flushleft}


{\EnglishColumn

\begin{doublespace}
29. Now, the Kattika-full-moon has been observed. (There are) those wilderness lodgings which are considered risky, which are dangerous. A bhikkhu dwelling in such kind of lodgings, who is wishing (to do so), may put aside one of the three robes inside an inhabited area. And if there may be any reason for that bhikkhu for dwelling apart from that robe, the bhikkhu can dwell apart from that robe for six days at the most; if he should dwell apart for more than that, except with the authorisation of bhikkhus, (this is a case) involving expiation with forfeiture.
\end{doublespace}}

\switchcolumn


\begin{flushleft}
\begingl
 29.[] Upavassaṁ[observe-\PAST{\GMU{past-part}}] kho[indeed!-\EMPH{\GMU{emph}}] pana[(and)-\PART{\GMU{part}}] kattikapuṇṇamaṁ.[kattika.full.moon-\ACC{\GMU{acc-sg}}] Yāni[which-\NOM{\GMU{nom-pl-n}}] kho[indeed!-\EMPH{\GMU{emph}}] pana[(and)-\PART{\GMU{part}}] tāni[those-\NOM{\GMU{nom-pl}}] āraññakāni[wilderness-\ADJ{\GMU{adj}}] senāsanāni[lodging-\NOM{\GMU{nom-pl-n}}] sāsaṅkasammatāni[risky.recond-\ADJ{\GMU{adj}}] sappaṭibhayāni,[frighten-\ADJ{\GMU{adj}}] tathārūpesu[such kind-\ADJ{\GMU{adj}}] bhikkhu[bhikkhu-\NOM{\GMU{nom-sg}}] senāsanesu[lodging-\LOC{\GMU{loc-pl-n}}] viharanto,[dwell-\ADJ{\GMU{adj}}] ākaṅkhamāno[wish for-\ADJ{\GMU{adj-pres-part}}] tiṇṇaṁ[3-\GEN{\GMU{gen-m}}] cīvarānaṁ[robe-\GEN{\GMU{gen-pl-n}}] aññataraṁ[any one, another-\ADJ{\GMU{adj}}] cīvaraṁ[robe-\ACC{\GMU{acc-sg-n}}] antaraghare[inside house-\LOC{\GMU{loc-sg-n}}] nikkhipeyya.[lay aside-\OPT{\GMU{3-sg-opt}}] Siyā[be-\OPT{\GMU{3-sg-opt}}] ca[-\NUL{\GMU{}}] tassa[of that-\GEN{\GMU{gen-sg}}] bhikkhuno[bhikkhu-\DAT{\GMU{dat-sg}}] kocid’eva[any.just-\NOM{\GMU{nom}}] paccayo[reason-\NOM{\GMU{nom-sg}}] tena[him-\INS{\GMU{3-sg-ins}}] cīvarena[robe-\INS{\GMU{ins-sg-n}}] vippavāsāya,[dwell apart-\DAT{\GMU{dat-sg}}] chārattaparaman[6.night.at most-\ADV{\GMU{adv}}] tena[him-\INS{\GMU{3-sg-ins}}] bhikkhunā[bhikkhu-\INS{\GMU{ins-sg}}] tena[him-\INS{\GMU{3-sg-ins}}] cīvarena[robe-\INS{\GMU{ins-sg-n}}] vippavasitabbaṁ.[be apart-\FUT{\GMU{fut-pass-part}}] Tato[then-\ABL{\GMU{abl}}] ce[if-\NUL{\GMU{}}] uttariṁ[more-\ADV{\GMU{adv}}] vippavaseyya,[dwell apart-\OPT{\GMU{3-sg-opt}}] aññatra[unless-\ABL{\GMU{abl}}] bhikkhusammatiyā,[bhikkhu.consent-\INS{\GMU{ins-sg}}] nissaggiyaṁ[relinquish-\ADJ{\GMU{adj}}] pācittiyaṁ.[confess-\ADJ{\GMU{adj}}]
\endgl
\switchcolumn*
\end{flushleft}


{\EnglishColumn

\begin{doublespace}
30. If any bhikkhu should knowingly allocate for himself a gain belonging to (and) allocated to the  community, (this is a case) involving expiation with forfeiture.
\end{doublespace}}

\switchcolumn


\begin{flushleft}
\begingl
 30.[] Yo[who-\NOM{\GMU{nom-sg}}] pana[(and)-\PART{\GMU{part}}] bhikkhu[bhikkhu-\NOM{\GMU{nom-sg}}] jānaṁ[know-\NOM{\GMU{nom-sg}}] saṅghikaṁ[community.owned-\ADJ{\GMU{adj}}] lābhaṁ[gain-\ACC{\GMU{acc-sg}}] pariṇataṁ[allocate-\PAST{\GMU{past-part}}] attano[self-\DAT{\GMU{dat-sg}}] pariṇāmeyya,[allocate-\OPT{\GMU{3-sg-opt}}] nissaggiyaṁ[relinquish-\ADJ{\GMU{adj}}] pācittiyaṁ.[confess-\ADJ{\GMU{adj}}]
\endgl
\switchcolumn*
\end{flushleft}


{\EnglishColumn

\begin{doublespace}
The section on bowls is third.
\end{doublespace}}

\switchcolumn


\begin{flushleft}
\begingl
 Pattavaggo[bowl.section-\NUL{\GMU{}}] tatiyo.[third-\ORD{\GMU{ord}}]
\endgl
\switchcolumn*
\end{flushleft}


{\EnglishColumn

\begin{doublespace}
Venerables, the thirty cases involving expiation with forfeiture have been recited.
Concerning this I ask the Venerables: (Are you) pure in this?\\
A second time again I ask: (Are you) pure in this?\\
A third time again I ask: (Are you) pure in this?\\
The Venerables are pure in this, therefore there is silence, thus I keep this (in mind).
\end{doublespace}}

\switchcolumn


\begin{flushleft}
\begingl
 Uddiṭṭhā[recite-\PAST{\GMU{past-part}}] kho[indeed!-\EMPH{\GMU{emph}}] āyasmanto[Ven.-\VOC{\GMU{voc-pl}}] tiṁsa[] nissaggiyā[] pācittiyā[] dhammā.[rule-\NOM{\GMU{nom-pl}}]+ Tatth’āyasmante[] pucchāmi:[ask-\PRESIND{\GMU{1-sg-presind}}] Kacci’ttha[] parisuddhā?[pure-\ADJ{\GMU{adj}}]+ Dutiyam’pi[second time-\ACC{\GMU{acc-sg-nt}}] pucchāmi:[ask-\PRESIND{\GMU{1-sg-presind}}] Kacci’ttha[] parisuddhā?[pure-\ADJ{\GMU{adj}}]+ Tatiyam’pi[] pucchāmi:[ask-\PRESIND{\GMU{1-sg-presind}}] Kacci’ttha[] parisuddhā?[pure-\ADJ{\GMU{adj}}]+ Parisuddh’etth’āyasmanto,[] tasmā[therefore-\ABL{\GMU{abl-sg}}] tuṇhī,[silent-\ADV{\GMU{adv}}] evam’etaṁ[thus.this-\ACC{\GMU{acc-sg-n}}] dhārayāmi.[keep in mind-\PRESIND{\GMU{1-sg-presind}}]
\endgl
\switchcolumn*
\end{flushleft}


{\EnglishColumn

\begin{doublespace}
The cases involving expiation with forfeiture are finished.
\end{doublespace}}

\switchcolumn


\begin{flushleft}
\begingl
 Nissaggiyā[] pācittiyā[] dhammā[rule-\NOM{\GMU{nom-pl}}] niṭṭhitā[]
\endgl
\switchcolumn*
\end{flushleft}


{\EnglishColumn

\begin{doublespace}
Venerables, these ninety-two cases involving expiation come up for recitation.
\end{doublespace}}

\switchcolumn


\begin{flushleft}
\begingl
 Ime[this-\NOM{\GMU{nom-pl}}] kho[indeed!-\EMPH{\GMU{emph}}] pan’āyasmanto[venerable-\VOC{\GMU{voc-pl}}] dvenavuti[] pācittiyā[] dhammā[rule-\NOM{\GMU{nom-pl}}] uddesaṁ[recitation-\ACC{\GMU{acc-sg}}] āgacchanti.[come up-\PRESIND{\GMU{3-pl-presind}}]
\endgl
\switchcolumn*
\end{flushleft}


{\EnglishColumn

\begin{doublespace}
1. In deliberate false speech, (there is a case) involving expiation.
\end{doublespace}}

\switchcolumn


\begin{flushleft}
\begingl
 1.[] Sampajānamusāvāde[deliberate.false.speech-\LOC{\GMU{loc-sg}}] pācittiyaṁ.[confess-\ADJ{\GMU{adj}}]
\endgl
\switchcolumn*
\end{flushleft}


{\EnglishColumn

\begin{doublespace}
2. In abusive speech, (there is a case) involving expiation.
\end{doublespace}}

\switchcolumn


\begin{flushleft}
\begingl
 2.[] Omasavāde[abusive speech-\LOC{\GMU{loc-sg}}] pācittiyaṁ.[confess-\ADJ{\GMU{adj}}]
\endgl
\switchcolumn*
\end{flushleft}


{\EnglishColumn

\begin{doublespace}
3. In the backbiting of a bhikkhu, (there is a case) involving expiation.
\end{doublespace}}

\switchcolumn


\begin{flushleft}
\begingl
 3.[] Bhikkhupesuññe[bhikkhu.slander-\LOC{\GMU{loc-sg}}] pācittiyaṁ.[confess-\ADJ{\GMU{adj}}]
\endgl
\switchcolumn*
\end{flushleft}


{\EnglishColumn

\begin{doublespace}
4. If any bhikkhu should have one who has not been fully admitted (into the community) recite the Dhamma (line) by line, (this is a case) involving expiation.
\end{doublespace}}

\switchcolumn


\begin{flushleft}
\begingl
 4.[] Yo[who-\NOM{\GMU{nom-sg}}] pana[(and)-\PART{\GMU{part}}] bhikkhu[bhikkhu-\NOM{\GMU{nom-sg}}] anupasampannaṁ[not.admitted-\ACC{\GMU{acc-sg-n}}] padaso[line-\ADV{\GMU{adv}}] dhammaṁ[act-\ACC{\GMU{acc-sg}}] vāceyya,[recite-\OPT{\GMU{3-sg-opt}}] pācittiyaṁ.[confess-\ADJ{\GMU{adj}}]
\endgl
\switchcolumn*
\end{flushleft}


{\EnglishColumn

\begin{doublespace}
5. If any bhikkhu should make use of a sleeping place for more than two nights or three nights together with one who has not been fully admitted (into the bhikkhu-community), (this is a case) involving expiation.
\end{doublespace}}

\switchcolumn


\begin{flushleft}
\begingl
 5.[] Yo[who-\NOM{\GMU{nom-sg}}] pana[(and)-\PART{\GMU{part}}] bhikkhu[bhikkhu-\NOM{\GMU{nom-sg}}] anupasampannena[not.admitted-\INS{\GMU{ins-sg}}] uttaridvirattatirattaṁ[more.2.3.nights-\ACC{\GMU{acc-sg}}] sahaseyyaṁ[with.bedding-\ACC{\GMU{acc-sg-f}}] kappeyya,[use-\OPT{\GMU{3-sg-opt}}] pācittiyaṁ.[confess-\ADJ{\GMU{adj}}]
\endgl
\switchcolumn*
\end{flushleft}


{\EnglishColumn

\begin{doublespace}
6. If any bhikkhu should make use of a sleeping place together with a woman, (this is a case) involving expiation.
\end{doublespace}}

\switchcolumn


\begin{flushleft}
\begingl
 6.[] Yo[who-\NOM{\GMU{nom-sg}}] pana[(and)-\PART{\GMU{part}}] bhikkhu[bhikkhu-\NOM{\GMU{nom-sg}}] mātugāmena[woman-\INS{\GMU{ins-sg}}] sahaseyyaṁ[with.bedding-\ACC{\GMU{acc-sg-f}}] kappeyya,[use-\OPT{\GMU{3-sg-opt}}] pācittiyaṁ.[confess-\ADJ{\GMU{adj}}]
\endgl
\switchcolumn*
\end{flushleft}


{\EnglishColumn

\begin{doublespace}
7. If any bhikkhu should teach the Dhamma to a woman by (means of) more than five or six sentences, except (when being together) with a discerning male human being, (this is a case) involving expiation.
\end{doublespace}}

\switchcolumn


\begin{flushleft}
\begingl
 7.[] Yo[who-\NOM{\GMU{nom-sg}}] pana[(and)-\PART{\GMU{part}}] bhikkhu[bhikkhu-\NOM{\GMU{nom-sg}}] mātugāmassa[woman-\GEN{\GMU{gen-sg}}] uttarichappañcavācāhi[more.5.6.sentence-\INS{\GMU{ins-pl-f}}] dhammaṁ[act-\ACC{\GMU{acc-sg}}] deseyya,[teach-\OPT{\GMU{3-sg-opt}}] aññatra[unless-\ABL{\GMU{abl}}] viññunā[know-\ADJ{\GMU{adj}}] purisaviggahena,[male being-\INS{\GMU{ins-sg}}] pācittiyaṁ.[confess-\ADJ{\GMU{adj}}]
\endgl
\switchcolumn*
\end{flushleft}


{\EnglishColumn

\begin{doublespace}
8. If any bhikkhu should declare a superhuman state to one who has not been fully admitted (into the bhikkhu-community), (even) when it is a fact, (this is a case) involving expiation.
\end{doublespace}}

\switchcolumn


\begin{flushleft}
\begingl
 8.[] Yo[who-\NOM{\GMU{nom-sg}}] pana[(and)-\PART{\GMU{part}}] bhikkhu[bhikkhu-\NOM{\GMU{nom-sg}}] anupasampannassa[not.admitted-\DAT{\GMU{dat-sg}}] uttarimanussadhammaṁ[beyond.human.state-\ACC{\GMU{acc-sg}}] āroceyya,[announce-\OPT{\GMU{3-sg-opt}}] bhūtasmiṁ[become-\ABS{\GMU{abs}}] pācittiyaṁ.[confess-\ADJ{\GMU{adj}}]
\endgl
\switchcolumn*
\end{flushleft}


{\EnglishColumn

\begin{doublespace}
9. If any bhikkhu should declare the depraved offence of (another) bhikkhu to one who has not been fully admitted (into the bhikkhu-community), except with the authorisation of bhikkhus, (this is a case) involving expiation.
\end{doublespace}}

\switchcolumn


\begin{flushleft}
\begingl
 9.[] Yo[who-\NOM{\GMU{nom-sg}}] pana[(and)-\PART{\GMU{part}}] bhikkhu[bhikkhu-\NOM{\GMU{nom-sg}}] bhikkhussa[bhikkhu-\GEN{\GMU{gen-sg}}] duṭṭhullaṁ[obscene-\ADJ{\GMU{adj}}] āpattiṁ[offense-\ACC{\GMU{acc-sg-f}}] anupasampannassa[not.admitted-\DAT{\GMU{dat-sg}}] āroceyya[announce-\OPT{\GMU{3-sg-opt}}] aññatra[unless-\ABL{\GMU{abl}}] bhikkhusammatiyā,[bhikkhu.consent-\INS{\GMU{ins-sg}}] pācittiyaṁ.[confess-\ADJ{\GMU{adj}}]
\endgl
\switchcolumn*
\end{flushleft}


{\EnglishColumn

\begin{doublespace}
10. If any bhikkhu should dig the earth or should have it dug, (this is a case) involving expiation.
\end{doublespace}}

\switchcolumn


\begin{flushleft}
\begingl
 10.[] Yo[who-\NOM{\GMU{nom-sg}}] pana[(and)-\PART{\GMU{part}}] bhikkhu[bhikkhu-\NOM{\GMU{nom-sg}}] paṭhaviṁ[earth-\ACC{\GMU{acc-sg-f}}] khaṇeyya[dig-\OPT{\GMU{3-sg-opt}}] vā[or-\IND{\GMU{ind}}] khaṇāpeyya[dig-\OPT{\GMU{3-sg-opt}}] vā,[or-\IND{\GMU{ind}}] pācittiyaṁ.[confess-\ADJ{\GMU{adj}}]
\endgl
\switchcolumn*
\end{flushleft}


{\EnglishColumn

\begin{doublespace}
The section (starting with the rule) on false speech is first.
\end{doublespace}}

\switchcolumn


\begin{flushleft}
\begingl
 Musāvādavaggo[false.speech.section-\NUL{\GMU{}}] Paṭhamo.[first-\ADJ{\GMU{adj}}]
\endgl
\switchcolumn*
\end{flushleft}


{\EnglishColumn

\begin{doublespace}
11. In the destroying of vegetation, (there is a case) involving expiation.
\end{doublespace}}

\switchcolumn


\begin{flushleft}
\begingl
 11.[] Bhūtagāmapātabyatāya[vegetation.destroy-\LOC{\GMU{loc-sg-f}}] pācittiyaṁ.[confess-\ADJ{\GMU{adj}}]
\endgl
\switchcolumn*
\end{flushleft}


{\EnglishColumn

\begin{doublespace}
12. In evading, in vexing, (there is a case) involving expiation.
\end{doublespace}}

\switchcolumn


\begin{flushleft}
\begingl
 12.[] Aññavādake[other speak-\LOC{\GMU{loc-sg-n}}] vihesake[vex-\LOC{\GMU{loc-sg}}] pācittiyaṁ.[confess-\ADJ{\GMU{adj}}]
\endgl
\switchcolumn*
\end{flushleft}


{\EnglishColumn

\begin{doublespace}
13. In making (another bhikkhu) find fault, in criticising, (there is a case) involving expiation.
\end{doublespace}}

\switchcolumn


\begin{flushleft}
\begingl
 13.[] Ujjhāpanake[find fault-\LOC{\GMU{loc-sg}}] khiyyanake[criticize-\LOC{\GMU{loc-sg}}] pācittiyaṁ.[confess-\ADJ{\GMU{adj}}]
\endgl
\switchcolumn*
\end{flushleft}


{\EnglishColumn

\begin{doublespace}
14. If any bhikkhu, having (himself) put out or after having (someone else) put out in the open air, a bed or seat or mattress or stool belonging to the community, (and) then, when departing, should not take (it) away or should not have (it) taken away or should go without asking (someone to put it back), (this is a case) involving expiation.
\end{doublespace}}

\switchcolumn


\begin{flushleft}
\begingl
 14.[] Yo[who-\NOM{\GMU{nom-sg}}] pana[(and)-\PART{\GMU{part}}] bhikkhu[bhikkhu-\NOM{\GMU{nom-sg}}] saṅghikaṁ[community.owned-\ADJ{\GMU{adj}}] mañcaṁ[bed-\ACC{\GMU{acc-sg}}] vā[or-\IND{\GMU{ind}}] pīṭhaṁ[chair-\ACC{\GMU{acc-sg-n}}] vā[or-\IND{\GMU{ind}}] bhisiṁ[cushion-\ACC{\GMU{acc-sg-f}}] vā[or-\IND{\GMU{ind}}] kocchaṁ[stool-\ACC{\GMU{acc-sg-n}}] vā[or-\IND{\GMU{ind}}] ajjhokāse[in.air-\LOC{\GMU{loc-sg}}] santharitvā[layout-\ABS{\GMU{abs}}] vā[or-\IND{\GMU{ind}}] santharāpetvā[make layout-\ABS{\GMU{abs}}] vā,[or-\IND{\GMU{ind}}] taṁ[that-\ACC{\GMU{acc-sg}}] pakkamanto[depart-\PRES{\GMU{pres-part}}] n’eva[nor-\NUL{\GMU{}}] uddhareyya[take away-\OPT{\GMU{3-sg-opt}}] na[not-\PART{\GMU{part}}] uddharāpeyya,[make take away-\OPT{\GMU{3-sg-opt}}] anāpucchaṁ[not.ask-\PRES{\GMU{pres-part}}] vā[or-\IND{\GMU{ind}}] gaccheyya,[go-\OPT{\GMU{3-sg-opt}}] pācittiyaṁ.[confess-\ADJ{\GMU{adj}}]
\endgl
\pagebreak
\switchcolumn*
\end{flushleft}


{\EnglishColumn

\begin{doublespace}
15. If any bhikkhu, having (himself) put out or having (someone else) put out, bedding in a dwelling belonging to the community, (and) then, when departing, should not take (it) away or should not have (it) taken away, or should go without asking (someone to put it back), (this is a case) involving expiation.
\end{doublespace}}

\switchcolumn


\begin{flushleft}
\begingl
 15.[] Yo[who-\NOM{\GMU{nom-sg}}] pana[(and)-\PART{\GMU{part}}] bhikkhu[bhikkhu-\NOM{\GMU{nom-sg}}] saṅghike[community-\ADJ{\GMU{adj}}] vihāre[dwell-\LOC{\GMU{loc-sg}}] seyyaṁ[bedding-\ACC{\GMU{acc-sg-f}}] santharitvā[layout-\ABS{\GMU{abs}}] vā[or-\IND{\GMU{ind}}] santharāpetvā[make layout-\ABS{\GMU{abs}}] vā,[or-\IND{\GMU{ind}}] taṁ[that-\ACC{\GMU{acc-sg}}] pakkamanto[depart-\PRES{\GMU{pres-part}}] n’eva[nor-\NUL{\GMU{}}] uddhareyya[take away-\OPT{\GMU{3-sg-opt}}] na[not-\PART{\GMU{part}}] uddharāpeyya,[make take away-\OPT{\GMU{3-sg-opt}}] anāpucchaṁ[not.ask-\PRES{\GMU{pres-part}}] vā[or-\IND{\GMU{ind}}] gaccheyya,[go-\OPT{\GMU{3-sg-opt}}] pācittiyaṁ.[confess-\ADJ{\GMU{adj}}]
\endgl
\switchcolumn*
\end{flushleft}


{\EnglishColumn

\begin{doublespace}
16. If any bhikkhu, having encroached upon a bhikkhu who has arrived before, should knowingly use a sleeping place in a dwelling belonging to the community (saying): “He for whom it is (too) cramped, will leave,” having done (it) for just this reason, (and) not another, (this is a case) involving expiation.
\end{doublespace}}

\switchcolumn


\begin{flushleft}
\begingl
 16.[] Yo[who-\NOM{\GMU{nom-sg}}] pana[(and)-\PART{\GMU{part}}] bhikkhu[bhikkhu-\NOM{\GMU{nom-sg}}] saṅghike[community-\ADJ{\GMU{adj}}] vihāre[dwell-\LOC{\GMU{loc-sg}}] jānaṁ[know-\NOM{\GMU{nom-sg}}] pubbūpagataṁ[before.arrive-\ADJ{\GMU{adj}}] bhikkhuṁ[bhikkhu-\ACC{\GMU{acc-sg}}] anūpakhajja[encroach-\ABS{\GMU{abs}}] seyyaṁ[bedding-\ACC{\GMU{acc-sg-f}}] kappeyya,[use-\OPT{\GMU{3-sg-opt}}] “Yassa[for whoever-\PRO{\GMU{pro}}] sambādho[cramped-\NOM{\GMU{nom-sg}}] bhavissati,[to be-\FUT{\GMU{3-sg-fut}}] so[he-\NOM{\GMU{nom-sg}}] pakkamissatī”[depart-\FUT{\GMU{3-sg-fut}}] ti.[-\NUL{\GMU{}}] Etad’eva[this.just-\ACC{\GMU{acc-sg-n}}] paccayaṁ[reason-\ACC{\GMU{acc-sg}}] karitvā[done-\ABS{\GMU{abs}}] anaññaṁ,[not.another-\ADJ{\GMU{adj}}] pācittiyaṁ.[confess-\ADJ{\GMU{adj}}]
\endgl
\switchcolumn*
\end{flushleft}


{\EnglishColumn

\begin{doublespace}
17. If any bhikkhu, being resentful and displeased, should drive out a bhikkhu or have (him) driven out from a dwelling belonging to the community, (this is a case) involving expiation.
\end{doublespace}}

\switchcolumn


\begin{flushleft}
\begingl
 17.[] Yo[who-\NOM{\GMU{nom-sg}}] pana[(and)-\PART{\GMU{part}}] bhikkhu[bhikkhu-\NOM{\GMU{nom-sg}}] bhikkhuṁ[bhikkhu-\ACC{\GMU{acc-sg}}] kupito[disturb-\PAST{\GMU{past-part}}] anattamano[displeased-\ADJ{\GMU{adj}}] saṅghikā[community-\ADJ{\GMU{adj}}] vihārā[dwell-\ABL{\GMU{abl-sg}}] nikkaḍḍheyya[drive out-\OPT{\GMU{3-sg-opt}}] vā[or-\IND{\GMU{ind}}] nikkaḍḍhāpeyya[drive out-\OPT{\GMU{3-sg-opt}}] vā,[or-\IND{\GMU{ind}}] pācittiyaṁ.[confess-\ADJ{\GMU{adj}}]
\endgl
\switchcolumn*
\end{flushleft}


{\EnglishColumn

\begin{doublespace}
18. If any bhikkhu should (brusquely) sit down or lie down on a bed or seat with detachable legs in a hut with an upper-floor in a dwelling belonging to the community, (this is a case) involving expiation.
\end{doublespace}}

\switchcolumn


\begin{flushleft}
\begingl
 18.[] Yo[who-\NOM{\GMU{nom-sg}}] pana[(and)-\PART{\GMU{part}}] bhikkhu[bhikkhu-\NOM{\GMU{nom-sg}}] saṅghike[community-\ADJ{\GMU{adj}}] vihāre[dwell-\LOC{\GMU{loc-sg}}] uparivehāsakuṭiyā[up.air.hunt-\LOC{\GMU{loc-sg}}] āhaccapādakaṁ[remove foot-\ADJ{\GMU{adj}}] mañcaṁ[bed-\ACC{\GMU{acc-sg}}] vā[or-\IND{\GMU{ind}}] pīṭhaṁ[chair-\ACC{\GMU{acc-sg-n}}] vā[or-\IND{\GMU{ind}}] abhinisīdeyya[sit down-\OPT{\GMU{3-sg-opt}}] vā[or-\IND{\GMU{ind}}] abhinipajjeyya[lie down-\OPT{\GMU{3-sg-opt}}] vā,[or-\IND{\GMU{ind}}] pācittiyaṁ.[confess-\ADJ{\GMU{adj}}]
\endgl
\switchcolumn*
\end{flushleft}


{\EnglishColumn

\begin{doublespace}
19. By a bhikkhu who is having a large dwelling built, a surrounding-layer of two or three coverings can be ordered, by (a bhikku) standing on (a place which has) few crops, upto the frame of the door for (the purpose of) fixing the bolt, (and) for surrounding the window. If he should order more than that, even (when) standing on (a place which has) few crops, (this is a case) involving expiation.
\end{doublespace}}

\switchcolumn


\begin{flushleft}
\begingl
 19.[] Mahallakaṁ[large-\ADJ{\GMU{adj}}] pana[(and)-\PART{\GMU{part}}] bhikkhunā[bhikkhu-\INS{\GMU{ins-sg}}] vihāraṁ[dwell-\ACC{\GMU{acc-sg}}] kārayamānena,[build-\PRES{\GMU{pres-part}}] yāva[until-\IND{\GMU{ind}}] dvārakosā[door.frame-\ABL{\GMU{abl-sg}}] aggalaṭṭhapanāya,[bolt.fix-\DAT{\GMU{dat-sg-n}}] ālokasandhiparikammāya,[light.open.prepare-\DAT{\GMU{dat-sg-n}}] dvitticchadanassa[2.or.3.times-\GEN{\GMU{gen-sg-n}}] pariyāyaṁ,[layer-\ACC{\GMU{acc-sg}}] appaharite[few crops-\LOC{\GMU{loc-sg-n}}] ṭhitena[stand-\ADJ{\GMU{adj}}] adhiṭṭhātabbaṁ.[apply-\FUT{\GMU{fut-pass-part}}] Tato[then-\ABL{\GMU{abl}}] ce[if-\NUL{\GMU{}}] uttariṁ[more-\ADV{\GMU{adv}}] appaharite’pi[few crops-\LOC{\GMU{loc-sg-n}}] ṭhito[stand-\ADJ{\GMU{adj}}] adhiṭṭhaheyya,[apply-\OPT{\GMU{3-sg-opt}}] pācittiyaṁ.[confess-\ADJ{\GMU{adj}}]
\endgl
\switchcolumn*
\end{flushleft}


{\EnglishColumn

\begin{doublespace}
20. If any bhikkhu should knowingly pour out, or should have (someone else) pour out, water containing living beings on grass or clay, (this is a case) involving expiation.
\end{doublespace}}

\switchcolumn


\begin{flushleft}
\begingl
 20.[] Yo[who-\NOM{\GMU{nom-sg}}] pana[(and)-\PART{\GMU{part}}] bhikkhu[bhikkhu-\NOM{\GMU{nom-sg}}] jānaṁ[know-\NOM{\GMU{nom-sg}}] sappāṇakaṁ[with life-\ADJ{\GMU{adj}}] udakaṁ[water-\ACC{\GMU{acc-sg-n}}] tiṇaṁ[grass-\ACC{\GMU{acc-sg-n}}] vā[or-\IND{\GMU{ind}}] mattikaṁ[clay-\ACC{\GMU{acc-sg-f}}] vā[or-\IND{\GMU{ind}}] siñceyya[pour-\OPT{\GMU{3-sg-opt}}] vā[or-\IND{\GMU{ind}}] siñcāpeyya[pour-\OPT{\GMU{3-sg-opt}}] vā,[or-\IND{\GMU{ind}}] pācittiyaṁ.[confess-\ADJ{\GMU{adj}}]
\endgl
\switchcolumn*
\end{flushleft}


{\EnglishColumn

\begin{doublespace}
The section (starting with the rule) on vegetation is second.
\end{doublespace}}

\switchcolumn


\begin{flushleft}
\begingl
 Bhūtagāmavaggo[veg.destroy.section-\NUL{\GMU{}}] Dutiyo.[second-\ORD{\GMU{ord}}]
\endgl
\switchcolumn*
\end{flushleft}


{\EnglishColumn

\begin{doublespace}
21. If any bhikkhu who has not been authorised should exhort the bhikkhunìs, (this is a case) involving expiation.
\end{doublespace}}

\switchcolumn


\begin{flushleft}
\begingl
 21.[] Yo[who-\NOM{\GMU{nom-sg}}] pana[(and)-\PART{\GMU{part}}] bhikkhu[bhikkhu-\NOM{\GMU{nom-sg}}] asammato[not.consent-\ADJ{\GMU{adj}}] bhikkhuniyo[bhikkhuni-\ACC{\GMU{acc-pl-f}}] ovadeyya,[exort-\OPT{\GMU{3-sg-opt}}] pācittiyaṁ.[confess-\ADJ{\GMU{adj}}]
\endgl
\switchcolumn*
\end{flushleft}


{\EnglishColumn

\begin{doublespace}
22. Even if a bhikkhu who has been authorised should exhort the bhikkhunìs after the sun has set, (this is a case) involving expiation.
\end{doublespace}}

\switchcolumn


\begin{flushleft}
\begingl
 22.[] Sammato’pi[authorized-\ADJ{\GMU{adj}}] ce[if-\NUL{\GMU{}}] bhikkhu[bhikkhu-\NOM{\GMU{nom-sg}}] atthaṅgate[set-\ADJ{\GMU{adj}}] suriye[sun-\LOC{\GMU{loc-sg}}] bhikkhuniyo[bhikkhuni-\ACC{\GMU{acc-pl-f}}] ovadeyya,[exort-\OPT{\GMU{3-sg-opt}}] pācittiyaṁ.[confess-\ADJ{\GMU{adj}}]
\endgl
\switchcolumn*
\end{flushleft}


{\EnglishColumn

\begin{doublespace}
23. If any bhikkhu, having approached the bhikkhunì-quarters, should exhort the bhikkhunìs, except at the (right) occasion, (this is a case) involving expiation.
\end{doublespace}}

\switchcolumn


\begin{flushleft}
\begingl
 23.[] Yo[who-\NOM{\GMU{nom-sg}}] pana[(and)-\PART{\GMU{part}}] bhikkhu[bhikkhu-\NOM{\GMU{nom-sg}}] bhikkhunūpassayaṁ[bhikkhuni.quarters-\ACC{\GMU{acc-sg}}] upasaṅkamitvā[approach-\ABS{\GMU{abs}}] bhikkhuniyo[bhikkhuni-\ACC{\GMU{acc-pl-f}}] ovadeyya[exort-\OPT{\GMU{3-sg-opt}}] aññatra[unless-\ABL{\GMU{abl}}] samayā,[time-\ABL{\GMU{abl-sg}}] pācittiyaṁ.[confess-\ADJ{\GMU{adj}}] Tatthāyaṁ[here.this-\NOM{\GMU{nom-sg}}] samayo:[time-\NOM{\GMU{nom-sg}}] gilānā[sick-\ADJ{\GMU{adj}}] hoti[he is-\PRESIND{\GMU{3-sg-presind}}] bhikkhunī.[bhikkhuni-\NOM{\GMU{nom-sg-f}}] Ayaṁ[this-\NOM{\GMU{nom-sg}}] tattha[about that-\ADV{\GMU{adv}}] samayo.[time-\NOM{\GMU{nom-sg}}]
\endgl
\switchcolumn*
\end{flushleft}


{\EnglishColumn

\begin{doublespace}
24. If any bhikkhu should say so: “The bhikkhus exhort bhikkhunìs for the sake of reward,” (this is a case) involving expiation.
\end{doublespace}}

\switchcolumn


\begin{flushleft}
\begingl
 24.[] Yo[who-\NOM{\GMU{nom-sg}}] pana[(and)-\PART{\GMU{part}}] bhikkhu[bhikkhu-\NOM{\GMU{nom-sg}}] evaṁ[thus-\ADV{\GMU{adv}}] vadeyya,[say-\OPT{\GMU{3-sg-opt}}] “āmisahetu[gain sake-\DAT{\GMU{dat-sg}}] bhikkhū[bhikkhu-\NOM{\GMU{nom-pl}}] bhikkhuniyo[bhikkhuni-\ACC{\GMU{acc-pl-f}}] ovadantī”[look down-\PRESIND{\GMU{3-pl-presind}}] ti,[-\NUL{\GMU{}}] pācittiyaṁ[confess-\ADJ{\GMU{adj}}]
\endgl
\switchcolumn*
\end{flushleft}


{\EnglishColumn

\begin{doublespace}
25. If any bhikkhu should give a robe (-cloth) to an unrelated bhikkhunì, except in an exchange, (this is a case) involving expiation.
\end{doublespace}}

\switchcolumn


\begin{flushleft}
\begingl
 25.[] Yo[who-\NOM{\GMU{nom-sg}}] pana[(and)-\PART{\GMU{part}}] bhikkhu[bhikkhu-\NOM{\GMU{nom-sg}}] aññātikāya[unrelated-\ADJ{\GMU{adj}}] bhikkhuniyā[bhikkhuni-\INS{\GMU{ins-sg-f}}] cīvaraṁ[robe-\ACC{\GMU{acc-sg-n}}] dadeyya,[give-\OPT{\GMU{3-sg-opt}}] aññatra[unless-\ABL{\GMU{abl}}] pārivaṭṭakā,[exchange-\INS{\GMU{ins-sg}}] pācittiyaṁ.[confess-\ADJ{\GMU{adj}}]
\endgl
\switchcolumn*
\end{flushleft}


{\EnglishColumn

\begin{doublespace}
26. If any bhikkhu should sew a robe or should have a robe sewn for an unrelated bhikkhunì, (this is a case) involving expiation.
\end{doublespace}}

\switchcolumn


\begin{flushleft}
\begingl
 26.[] Yo[who-\NOM{\GMU{nom-sg}}] pana[(and)-\PART{\GMU{part}}] bhikkhu[bhikkhu-\NOM{\GMU{nom-sg}}] aññātikāya[unrelated-\ADJ{\GMU{adj}}] bhikkhuniyā[bhikkhuni-\INS{\GMU{ins-sg-f}}] cīvaraṁ[robe-\ACC{\GMU{acc-sg-n}}] sibbeyya[sew-\OPT{\GMU{3-sg-opt}}] vā[or-\IND{\GMU{ind}}] sibbāpeyya[sew-\OPT{\GMU{3-sg-opt}}] vā,[or-\IND{\GMU{ind}}] pācittiyaṁ.[confess-\ADJ{\GMU{adj}}]
\endgl
\switchcolumn*
\end{flushleft}


{\EnglishColumn

\begin{doublespace}
27. If any bhikkhu, having made an arrangement, should travel together with a bhikkhunì on the same main road, even (if) just the distance between villages, except at the (right) occasion, (this is a case) involving expiation.
\end{doublespace}}

\switchcolumn


\begin{flushleft}
\begingl
 27.[] Yo[who-\NOM{\GMU{nom-sg}}] pana[(and)-\PART{\GMU{part}}] bhikkhu[bhikkhu-\NOM{\GMU{nom-sg}}] bhikkhuniyā[bhikkhuni-\INS{\GMU{ins-sg-f}}] saddhiṁ[together-\INS{\GMU{ins}}] saṁvidhāya[arrange-\ABS{\GMU{abs}}] ekaddhānamaggaṁ[same road-\ACC{\GMU{acc-sg}}] paṭipajjeyya,[travel-\OPT{\GMU{3-sg-opt}}] antamaso[even so much as-\IND{\GMU{ind}}] gām’antaram’pi[village.between-\ACC{\GMU{acc-sg-n}}] aññatra[unless-\ABL{\GMU{abl}}] samayā,[time-\ABL{\GMU{abl-sg}}] pācittiyaṁ.[confess-\ADJ{\GMU{adj}}] Tatthāyaṁ[here.this-\NOM{\GMU{nom-sg}}] samayo:[time-\NOM{\GMU{nom-sg}}] satthagamanīyo[company.go-\ADJ{\GMU{adj}}] hoti[he is-\PRESIND{\GMU{3-sg-presind}}] maggo[road-\NOM{\GMU{nom-sg}}] sāsaṅkasammato[risky.recond-\ADJ{\GMU{adj}}] sappaṭibhayo.[frighten-\ADJ{\GMU{adj}}] Ayaṁ[this-\NOM{\GMU{nom-sg}}] tattha[about that-\ADV{\GMU{adv}}] samayo.[time-\NOM{\GMU{nom-sg}}]
\endgl
\switchcolumn*
\end{flushleft}


{\EnglishColumn

\begin{doublespace}
28. If any bhikkhu, having made an arrangement, should embark (on a voyage) together with a bhikkhunì on the same boat, which is going up (-stream) or which is going down (-stream), except with (a boat which is)crossing over (a river), (this is a case) involving expiation.
\end{doublespace}}

\switchcolumn


\begin{flushleft}
\begingl
 28.[] Yo[who-\NOM{\GMU{nom-sg}}] pana[(and)-\PART{\GMU{part}}] bhikkhu[bhikkhu-\NOM{\GMU{nom-sg}}] bhikkhuniyā[bhikkhuni-\INS{\GMU{ins-sg-f}}] saddhiṁ[together-\INS{\GMU{ins}}] saṁvidhāya[arrange-\ABS{\GMU{abs}}] ekaṁ[same-\ADJ{\GMU{adj}}] nāvaṁ[boat-\ACC{\GMU{acc-sg-n}}] abhirūheyya,[voyage-\OPT{\GMU{3-sg-opt}}] uddhagāminiṁ[up.go-\ADJ{\GMU{adj}}] vā[or-\IND{\GMU{ind}}] adhogāminiṁ[down.go-\ADJ{\GMU{adj}}] vā,[or-\IND{\GMU{ind}}] aññatra[unless-\ABL{\GMU{abl}}] tiriy’antaraṇāya,[over.cross-\INS{\GMU{ins-sg-f}}] pācittiyaṁ.[confess-\ADJ{\GMU{adj}}]
\endgl
\switchcolumn*
\end{flushleft}


{\EnglishColumn

\begin{doublespace}
29. If any bhikkhu should knowingly eat alms-food which a bhikkhunì has caused to be prepared, except through previous arrangement of householders, (this is a case) involving expiation.
\end{doublespace}}

\switchcolumn


\begin{flushleft}
\begingl
 29.[] Yo[who-\NOM{\GMU{nom-sg}}] pana[(and)-\PART{\GMU{part}}] bhikkhu[bhikkhu-\NOM{\GMU{nom-sg}}] jānaṁ[know-\NOM{\GMU{nom-sg}}] bhikkhunīparipācitaṁ[bhikkhuni.prompt-\ADJ{\GMU{adj}}] piṇḍapātaṁ[alms food-\ACC{\GMU{acc-sg}}] bhuñjeyya,[eat-\OPT{\GMU{3-sg-opt}}] aññatra[unless-\ABL{\GMU{abl}}] pubbe[previous-\ADV{\GMU{adv}}] gihisamārambhā,[h.h.arrange-\ABL{\GMU{abl-sg}}] pācittiyaṁ.[confess-\ADJ{\GMU{adj}}]
\endgl
\switchcolumn*
\end{flushleft}


{\EnglishColumn

\begin{doublespace}
30. If any bhikkhu should sit down together with a bhikkhunì, privately, one (man) with one (woman), (this is a case) involving expiation.
\end{doublespace}}

\switchcolumn


\begin{flushleft}
\begingl
 30.[] Yo[who-\NOM{\GMU{nom-sg}}] pana[(and)-\PART{\GMU{part}}] bhikkhu[bhikkhu-\NOM{\GMU{nom-sg}}] bhikkhuniyā[bhikkhuni-\INS{\GMU{ins-sg-f}}] saddhiṁ[together-\INS{\GMU{ins}}] eko[one-\NUM{\GMU{num}}] ekāya[one-\INS{\GMU{ins-sg-f}}] raho[private-\ADV{\GMU{adv}}] nisajjaṁ[seat-\ACC{\GMU{acc-sg-f}}] kappeyya,[use-\OPT{\GMU{3-sg-opt}}] pācittiyaṁ.[confess-\ADJ{\GMU{adj}}]
\endgl
\switchcolumn*
\end{flushleft}


{\EnglishColumn

\begin{doublespace}
The section (starting with the rule) on exhortation is third.
\end{doublespace}}

\switchcolumn


\begin{flushleft}
\begingl
 Ovādavaggo[] Tatiyo.[third-\ORD{\GMU{ord}}]
\endgl
\switchcolumn*
\end{flushleft}


{\EnglishColumn

\begin{doublespace}
31. By a bhikkhu who is not ill one alms-meal in a resthouse can be eaten; if he should eat more than that, (this is a case) involving expiation.
\end{doublespace}}

\switchcolumn


\begin{flushleft}
\begingl
 31.[] Agilānena[not.sick-\ADJ{\GMU{adj}}] bhikkhunā[bhikkhu-\INS{\GMU{ins-sg}}] eko[one-\NUM{\GMU{num}}] āvasatha’piṇḍo[] bhuñjitabbo.[eat-\FUT{\GMU{fut-pass-part}}] Tato[then-\ABL{\GMU{abl}}] ce[if-\NUL{\GMU{}}] uttariṁ[more-\ADV{\GMU{adv}}] bhuñjeyya,[eat-\OPT{\GMU{3-sg-opt}}] pācittiyaṁ.[confess-\ADJ{\GMU{adj}}]
\endgl
\switchcolumn*
\end{flushleft}


{\EnglishColumn

\begin{doublespace}
32. In eating (a meal) in a group, except at the (right) occasion, (there is a case) involving expiation.Here the occasion is this: the occasion of illness; the occasion of a giving of robe (-cloth)s; the occasion of a robe-making; the occasion of going on a (long) journey; the occasion of voyaging on a boat; the occasion of a great (gathering); the occasion of a meal (made) by an ascetic; this is the occasion here.
\end{doublespace}}

\switchcolumn


\begin{flushleft}
\begingl
 32.[] Gaṇabhojane[group.meal-\LOC{\GMU{loc-sg-n}}] aññatra[unless-\ABL{\GMU{abl}}] samayā,[time-\ABL{\GMU{abl-sg}}] pācittiyaṁ.[confess-\ADJ{\GMU{adj}}] Tatthāyaṁ[here.this-\NOM{\GMU{nom-sg}}] samayo:[time-\NOM{\GMU{nom-sg}}] gilānasamayo,[sick.time-\NOM{\GMU{nom-sg}}] cīvaradānasamayo,[robe.give.time-\NOM{\GMU{nom-sg}}] cīvarakārasamayo,[robe.make.time-\NOM{\GMU{nom-sg}}] addhānagamanasamayo,[journey.go.time-\NOM{\GMU{nom-sg}}] nāvābhirūhanasamayo,[boat.board.time-\NOM{\GMU{nom-sg-n}}] mahāsamayo,[great.time-\NOM{\GMU{nom-sg}}] samaṇabhattasamayo.[ascetic.meal.time-\NOM{\GMU{nom-sg}}] Ayaṁ[this-\NOM{\GMU{nom-sg}}] tattha[about that-\ADV{\GMU{adv}}] samayo.[time-\NOM{\GMU{nom-sg}}]
\endgl
\switchcolumn*
\end{flushleft}


{\EnglishColumn

\begin{doublespace}
33. In (taking) a meal before another (invitation-meal), except at the (right) occasion, (there is a case) involving expiation.
\end{doublespace}}

\switchcolumn


\begin{flushleft}
\begingl
 33.[] Paramparabhojane[after.other.meal-\LOC{\GMU{loc-sg}}] aññatra[unless-\ABL{\GMU{abl}}] samayā,[time-\ABL{\GMU{abl-sg}}] pācittiyaṁ.[confess-\ADJ{\GMU{adj}}] Tatthāyaṁ[here.this-\NOM{\GMU{nom-sg}}] samayo:[time-\NOM{\GMU{nom-sg}}] gilānasamayo,[sick.time-\NOM{\GMU{nom-sg}}] cīvaradānasamayo,[robe.give.time-\NOM{\GMU{nom-sg}}] cīvarakārasamayo.[robe.make.time-\NOM{\GMU{nom-sg}}] Ayaṁ[this-\NOM{\GMU{nom-sg}}] tattha[about that-\ADV{\GMU{adv}}] samayo.[time-\NOM{\GMU{nom-sg}}]
\endgl
\switchcolumn*
\end{flushleft}


{\EnglishColumn

\begin{doublespace}
34. Now, should a family invite a bhikkhu who has approached to take as many cakes and parched cakes (as he likes), by a bhikkhu who is wishing (so) two or three bowls full (of cakes) can be accepted; if he should accept more than that, (this is a case) involving expiation.
\end{doublespace}}

\switchcolumn


\begin{flushleft}
\begingl
 34.[] Bhikkhuṁ[bhikkhu-\ACC{\GMU{acc-sg}}] pan’eva[now.if-\PART{\GMU{part}}] kulaṁ[family-\NOM{\GMU{nom-sg-n}}] upagataṁ[approach-\PAST{\GMU{past-part}}] pūvehi[cake-\INS{\GMU{ins-pl}}] vā[or-\IND{\GMU{ind}}] manthehi[parch cake-\INS{\GMU{ins-sg}}] vā[or-\IND{\GMU{ind}}] abhihaṭṭhumpavāreyya,[take.invite-\OPT{\GMU{3-sg-opt}}] ākaṅkhamānena[wish for-\ADJ{\GMU{adj-pres-part}}] bhikkhunā[bhikkhu-\INS{\GMU{ins-sg}}] dvittipattapūrā[2.or.3.bowl.full-\ADJ{\GMU{adj}}] paṭiggahetabbā.[accept-\FUT{\GMU{fut-pass-part}}] Tato[then-\ABL{\GMU{abl}}] ce[if-\NUL{\GMU{}}] uttariṁ[more-\ADV{\GMU{adv}}] paṭiggaṇheyya,[receive-\OPT{\GMU{3-sg-opt}}] pācittiyaṁ.[confess-\ADJ{\GMU{adj}}] Dvittipattapūre[2.or.3.bowl.full-\ACC{\GMU{acc-pl}}] paṭiggahetvā[accept-\ABS{\GMU{abs}}] tato[then-\ABL{\GMU{abl}}] nīharitvā[take away-\ABS{\GMU{abs}}] bhikkhūhi[bhikkhu-\INS{\GMU{ins-pl}}] saddhiṁ[together-\INS{\GMU{ins}}] saṁvibhajitabbaṁ.[share-\FUT{\GMU{fut-pass-part}}] Ayaṁ[this-\NOM{\GMU{nom-sg}}] tattha[about that-\ADV{\GMU{adv}}] sāmīci.[proper procedure-\NOM{\GMU{nom-sg-f}}]
\endgl
\switchcolumn*
\end{flushleft}


{\EnglishColumn

\begin{doublespace}
35. If any bhikkhu who has eaten (a meal), who has been invited (to take more and refused), should chew uncooked food or eat cooked food which is not left over, (this is a case) involving expiation.
\end{doublespace}}

\switchcolumn


\begin{flushleft}
\begingl
 35.[] Yo[who-\NOM{\GMU{nom-sg}}] pana[(and)-\PART{\GMU{part}}] bhikkhu[bhikkhu-\NOM{\GMU{nom-sg}}] bhuttāvī[eat-\ADJ{\GMU{adj}}] pavārito[invite-\ADJ{\GMU{adj}}] anatirittaṁ[not.left over-\ADJ{\GMU{adj}}] khādanīyaṁ[uncooked food-\ACC{\GMU{acc-sg-n}}] vā[or-\IND{\GMU{ind}}] bhojanīyaṁ[cooked food-\ACC{\GMU{acc-sg}}] vā[or-\IND{\GMU{ind}}] khādeyya[chew-\OPT{\GMU{3-sg-opt}}] vā[or-\IND{\GMU{ind}}] bhuñjeyya[eat-\OPT{\GMU{3-sg-opt}}] vā,[or-\IND{\GMU{ind}}] pācittiyaṁ.[confess-\ADJ{\GMU{adj}}]
\endgl
\switchcolumn*
\end{flushleft}


{\EnglishColumn

\begin{doublespace}
36. If any bhikkhu, knowingly (and) desiring to cause offence, should invite a bhikkhu, who has eaten (a meal and) who has been invited (to take more), to take uncooked food or cooked food which is not left over (saying): “Here, bhikkhu, chew and eat!,” when (the bhikkhu) has eaten, (this is a case) involving expiation.
\end{doublespace}}

\switchcolumn


\begin{flushleft}
\begingl
 36.[] Yo[who-\NOM{\GMU{nom-sg}}] pana[(and)-\PART{\GMU{part}}] bhikkhu[bhikkhu-\NOM{\GMU{nom-sg}}] bhikkhuṁ[bhikkhu-\ACC{\GMU{acc-sg}}] bhuttāviṁ[eat-\ADJ{\GMU{adj}}] pavāritaṁ[invite-\ADJ{\GMU{adj}}] anatirittena[not.left over-\ADJ{\GMU{adj}}] khādanīyena[uncooked food-\INS{\GMU{ins-sg}}] vā[or-\IND{\GMU{ind}}] bhojanīyena[cooked food-\INS{\GMU{ins-sg}}] vā[or-\IND{\GMU{ind}}] abhihaṭṭhumpavāreyya,[take.invite-\OPT{\GMU{3-sg-opt}}] “Handa[come!-\EMPH{\GMU{emph}}] bhikkhu[bhikkhu-\NOM{\GMU{nom-sg}}] khāda[chew-\IMP{\GMU{2-sg-imp}}] vā[or-\IND{\GMU{ind}}] bhuñja[eat-\IMP{\GMU{2-sg-imp}}] vā”[or-\IND{\GMU{ind}}] ti,[-\NUL{\GMU{}}] jānaṁ[know-\NOM{\GMU{nom-sg}}] āsādan’āpekkho,[revenge.desire-\ADJ{\GMU{adj}}] bhuttasmiṁ[eat-\PAST{\GMU{past-part}}] pācittiyaṁ.[confess-\ADJ{\GMU{adj}}]
\endgl
\switchcolumn*
\end{flushleft}


{\EnglishColumn

\begin{doublespace}
37. If any bhikkhu should chew uncooked food or eat cooked food at the wrong time, (this is a case) involving expiation.
\end{doublespace}}

\switchcolumn


\begin{flushleft}
\begingl
 37.[] Yo[who-\NOM{\GMU{nom-sg}}] pana[(and)-\PART{\GMU{part}}] bhikkhu[bhikkhu-\NOM{\GMU{nom-sg}}] vikāle[wrong time-\LOC{\GMU{loc-sg}}] khādanīyaṁ[uncooked food-\ACC{\GMU{acc-sg-n}}] vā[or-\IND{\GMU{ind}}] bhojanīyaṁ[cooked food-\ACC{\GMU{acc-sg}}] vā[or-\IND{\GMU{ind}}] khādeyya[chew-\OPT{\GMU{3-sg-opt}}] vā[or-\IND{\GMU{ind}}] bhuñjeyya[eat-\OPT{\GMU{3-sg-opt}}] vā,[or-\IND{\GMU{ind}}] pācittiyaṁ.[confess-\ADJ{\GMU{adj}}]
\endgl
\switchcolumn*
\end{flushleft}


{\EnglishColumn

\begin{doublespace}
38. If any bhikkhu should chew uncooked food or eat cooked food (while) keeping (it) in store, (this is a case) involving expiation.
\end{doublespace}}

\switchcolumn


\begin{flushleft}
\begingl
 38.[] Yo[who-\NOM{\GMU{nom-sg}}] pana[(and)-\PART{\GMU{part}}] bhikkhu[bhikkhu-\NOM{\GMU{nom-sg}}] sannidhikārakaṁ[store keep-\ABS{\GMU{abs}}] khādanīyaṁ[uncooked food-\ACC{\GMU{acc-sg-n}}] vā[or-\IND{\GMU{ind}}] bhojanīyaṁ[cooked food-\ACC{\GMU{acc-sg}}] vā[or-\IND{\GMU{ind}}] khādeyya[chew-\OPT{\GMU{3-sg-opt}}] vā[or-\IND{\GMU{ind}}] bhuñjeyya[eat-\OPT{\GMU{3-sg-opt}}] vā,[or-\IND{\GMU{ind}}] pācittiyaṁ.[confess-\ADJ{\GMU{adj}}]
\endgl
\switchcolumn*
\end{flushleft}


{\EnglishColumn

\begin{doublespace}
39. Those foods which are superior, namely: ghee, butter, oil, honey and molasses, fish, meat, milk, curd; whichever bhikkhu, who is not ill, having requested such superior foods for his own benefit, should eat (them), (this is a case) involving expiation.
\end{doublespace}}

\switchcolumn


\begin{flushleft}
\begingl
 39.[] Yāni[which-\NOM{\GMU{nom-pl-n}}] kho[indeed!-\EMPH{\GMU{emph}}] pana[(and)-\PART{\GMU{part}}] tāni[those-\NOM{\GMU{nom-pl}}] paṇītabhojanāni,[superior.food-\NOM{\GMU{nom-pl-n}}] seyyathīdaṁ:[as follows-\NOM{\GMU{nom-sg}}] sappi[ghee-\NOM{\GMU{nom-sg-n}}] navanītaṁ[butter-\NOM{\GMU{nom-sg-n}}] telaṁ[oil-\NOM{\GMU{nom-sg}}] madhu[honey-\NOM{\GMU{nom-sg-n}}] phāṇitaṁ,[molasses-\NOM{\GMU{nom-sg}}] maccho[fish-\NOM{\GMU{nom-sg}}] maṁsaṁ[meat-\NOM{\GMU{nom-sg-n}}] khīraṁ[milk-\NOM{\GMU{nom-sg-n}}] dadhi.[curd-\NOM{\GMU{nom-sg-f}}] Yo[who-\NOM{\GMU{nom-sg}}] pana[(and)-\PART{\GMU{part}}] bhikkhu[bhikkhu-\NOM{\GMU{nom-sg}}] evarūpāni[such kind-\ADJ{\GMU{adj}}] paṇītabhojanāni[superior.food-\NOM{\GMU{nom-pl-n}}] agilāno[not.sick-\ADJ{\GMU{adj}}] attano[self-\DAT{\GMU{dat-sg}}] atthāya[need-\DAT{\GMU{dat-sg}}] viññāpetvā[request-\ABS{\GMU{abs}}] bhuñjeyya,[eat-\OPT{\GMU{3-sg-opt}}] pācittiyaṁ.[confess-\ADJ{\GMU{adj}}]
\endgl
\pagebreak
\switchcolumn*
\end{flushleft}


{\EnglishColumn

\begin{doublespace}
40. If any bhikkhu should take into the mouth (any) nutriment that has not been given (to bhikkhus); except water and tooth-wood, (this is a case) involving expiation.
\end{doublespace}}

\switchcolumn


\begin{flushleft}
\begingl
 40.[] Yo[who-\NOM{\GMU{nom-sg}}] pana[(and)-\PART{\GMU{part}}] bhikkhu[bhikkhu-\NOM{\GMU{nom-sg}}] adinnaṁ[not.given-\ACC{\GMU{acc-sg-n}}] mukhadvāraṁ[mouth.door-\ACC{\GMU{acc-sg}}] āhāraṁ[nutriment-\ACC{\GMU{acc-sg}}] āhareyya,[ingest-\OPT{\GMU{3-sg-opt}}] aññatra[unless-\ABL{\GMU{abl}}] udakadantapoṇā,[water.tooth.wood-\ABL{\GMU{abl-sg-n}}] pācittiyaṁ.[confess-\ADJ{\GMU{adj}}]
\endgl
\switchcolumn*
\end{flushleft}


{\EnglishColumn

\begin{doublespace}
The section (starting with the rule) on eating is fourth
\end{doublespace}}

\switchcolumn


\begin{flushleft}
\begingl
 Bhojanavaggo[food.section-\NUL{\GMU{}}] Catuttho.[fourth-\NUL{\GMU{}}]
\endgl
\switchcolumn*
\end{flushleft}


{\EnglishColumn

\begin{doublespace}
41. If any bhikkhu should give with his own hand uncooked food or cooked food to a naked ascetic or to a male wanderer or to a female wanderer, (this is a case) involving expiation.
\end{doublespace}}

\switchcolumn


\begin{flushleft}
\begingl
 41.[] Yo[who-\NOM{\GMU{nom-sg}}] pana[(and)-\PART{\GMU{part}}] bhikkhu[bhikkhu-\NOM{\GMU{nom-sg}}] acelakassa[no.cloth-\DAT{\GMU{dat-sg}}] vā[or-\IND{\GMU{ind}}] paribbājakassa[around.wander-\DAT{\GMU{dat-sg}}] vā[or-\IND{\GMU{ind}}] paribbājikāya[around.wander-\DAT{\GMU{dat-sg-f}}] vā[or-\IND{\GMU{ind}}] sahatthā[with.hand-\INS{\GMU{ins-sg}}] khādanīyaṁ[uncooked food-\ACC{\GMU{acc-sg-n}}] vā[or-\IND{\GMU{ind}}] bhojanīyaṁ[cooked food-\ACC{\GMU{acc-sg}}] vā[or-\IND{\GMU{ind}}] dadeyya,[give-\OPT{\GMU{3-sg-opt}}] pācittiyaṁ.[confess-\ADJ{\GMU{adj}}]
\endgl
\switchcolumn*
\end{flushleft}


{\EnglishColumn

\begin{doublespace}
42. If any bhikkhu should say so to a bhikkhu, “Come friend! We shall enter a village or town for alms,” (then after) having had (food) given or not having had (food) given to him, should he dismiss (the bhikkhu saying), “Go friend! There is no ease for me talking or sitting down together with you; there is ease for me talking or sitting down by myself;” having made just this the reason, (and) not another, (this is a case) involving expiation.
\end{doublespace}}

\switchcolumn


\begin{flushleft}
\begingl
 42.[] Yo[who-\NOM{\GMU{nom-sg}}] pana[(and)-\PART{\GMU{part}}] bhikkhu[bhikkhu-\NOM{\GMU{nom-sg}}] bhikkhuṁ[bhikkhu-\ACC{\GMU{acc-sg}}] evaṁ[thus-\ADV{\GMU{adv}}] vadeyya:[say-\OPT{\GMU{3-sg-opt}}] “Eh’āvuso[come friend-\VOC{\GMU{voc-sg}}] gāmaṁ[village-\ACC{\GMU{acc-sg}}] vā[or-\IND{\GMU{ind}}] nigamaṁ[town-\ACC{\GMU{acc-sg}}] vā[or-\IND{\GMU{ind}}] piṇḍāya[alms-\DAT{\GMU{dat-sg}}] pavisissāmā”[enter-\FUT{\GMU{1-pl-fut}}] ti.[-\NUL{\GMU{}}] Tassa[of that-\GEN{\GMU{gen-sg}}] dāpetvā[give-\ABS{\GMU{abs}}] vā[or-\IND{\GMU{ind}}] adāpetvā[not.give-\ABS{\GMU{abs}}] vā[or-\IND{\GMU{ind}}] uyyojeyya,[dismiss-\OPT{\GMU{opt-sg}}] “Gacch’āvuso.[go.friend-\IMP{\GMU{imp}}] Na[not-\PART{\GMU{part}}] me[me-\DAT{\GMU{dat-sg}}] tayā[you-\INS{\GMU{ins-sg}}] saddhiṁ[together-\INS{\GMU{ins}}] kathā[speak-\NOM{\GMU{nom-sg-f}}] vā[or-\IND{\GMU{ind}}] nisajjā[sit-\NOM{\GMU{nom-sg-f}}] vā[or-\IND{\GMU{ind}}] phāsu[ease-\ADV{\GMU{adv}}] hoti.[he is-\PRESIND{\GMU{3-sg-presind}}] Ekakassa[alone-\DAT{\GMU{dat-sg}}] me[me-\DAT{\GMU{dat-sg}}] kathā[speak-\NOM{\GMU{nom-sg-f}}] vā[or-\IND{\GMU{ind}}] nisajjā[sit-\NOM{\GMU{nom-sg-f}}] vā[or-\IND{\GMU{ind}}] phāsu[ease-\ADV{\GMU{adv}}] hotī”[he is-\PRESIND{\GMU{3-sg-presind}}] ti.[-\NUL{\GMU{}}] Etad’eva[this.just-\ACC{\GMU{acc-sg-n}}] paccayaṁ[reason-\ACC{\GMU{acc-sg}}] karitvā[done-\ABS{\GMU{abs}}] anaññaṁ,[not.another-\ADJ{\GMU{adj}}] pācittiyaṁ.[confess-\ADJ{\GMU{adj}}]
\endgl
\switchcolumn*
\end{flushleft}


{\EnglishColumn

\begin{doublespace}
43. If any bhikkhu, having intruded upon an family having a meal, should sit down, (this is a case) involving expiation.
\end{doublespace}}

\switchcolumn


\begin{flushleft}
\begingl
 43.[] Yo[who-\NOM{\GMU{nom-sg}}] pana[(and)-\PART{\GMU{part}}] bhikkhu[bhikkhu-\NOM{\GMU{nom-sg}}] sabhojane[with food-\ADJ{\GMU{adj}}] kule[family-\LOC{\GMU{loc-sg}}] anūpakhajja[encroach-\ABS{\GMU{abs}}] nisajjaṁ[seat-\ACC{\GMU{acc-sg-f}}] kappeyya,[use-\OPT{\GMU{3-sg-opt}}] pācittiyaṁ.[confess-\ADJ{\GMU{adj}}]
\endgl
\switchcolumn*
\end{flushleft}


{\EnglishColumn

\begin{doublespace}
44 .If any bhikkhu should sit down together with a woman, privately, on a concealed seat, (this is a case) involving expiation.
\end{doublespace}}

\switchcolumn


\begin{flushleft}
\begingl
 44.[] Yo[who-\NOM{\GMU{nom-sg}}] pana[(and)-\PART{\GMU{part}}] bhikkhu[bhikkhu-\NOM{\GMU{nom-sg}}] mātugāmena[woman-\INS{\GMU{ins-sg}}] saddhiṁ[together-\INS{\GMU{ins}}] raho[private-\ADV{\GMU{adv}}] paṭicchanne[seclude-\PAST{\GMU{past-part}}] āsane[seat-\LOC{\GMU{loc-sg-n}}] nisajjaṁ[seat-\ACC{\GMU{acc-sg-f}}] kappeyya,[use-\OPT{\GMU{3-sg-opt}}] pācittiyaṁ.[confess-\ADJ{\GMU{adj}}]
\endgl
\switchcolumn*
\end{flushleft}


{\EnglishColumn

\begin{doublespace}
45. If any bhikkhu sit down together with a woman, one (man) with one (woman), privately, (this is a case) involving expiation.
\end{doublespace}}

\switchcolumn


\begin{flushleft}
\begingl
 45.[] Yo[who-\NOM{\GMU{nom-sg}}] pana[(and)-\PART{\GMU{part}}] bhikkhu[bhikkhu-\NOM{\GMU{nom-sg}}] mātugāmena[woman-\INS{\GMU{ins-sg}}] saddhiṁ[together-\INS{\GMU{ins}}] eko[one-\NUM{\GMU{num}}] ekāya[one-\INS{\GMU{ins-sg-f}}] raho[private-\ADV{\GMU{adv}}] nisajjaṁ[seat-\ACC{\GMU{acc-sg-f}}] kappeyya,[use-\OPT{\GMU{3-sg-opt}}] pācittiyaṁ.[confess-\ADJ{\GMU{adj}}]
\endgl
\switchcolumn*
\end{flushleft}


{\EnglishColumn

\begin{doublespace}
46. If any bhikkhu who has been invited for a meal, not having asked (permission to) a bhikkhu who is present (in the monastery), should go visiting families before the meal or after the meal, except at the (right) occasion, (this is a case) involving expiation. Here the occasion is this: the occasion of a giving of robe (-cloth)s; the occasion of a making of robes; this is the occasion here.
\end{doublespace}}

\switchcolumn


\begin{flushleft}
\begingl
 46.[] Yo[who-\NOM{\GMU{nom-sg}}] pana[(and)-\PART{\GMU{part}}] bhikkhu[bhikkhu-\NOM{\GMU{nom-sg}}] nimantito[invite-\PAST{\GMU{past-part}}] sabhatto[wtih meal-\ADJ{\GMU{adj}}] samāno[exist-\PRES{\GMU{pres-part}}] santaṁ[exist-\PRES{\GMU{pres-part}}] bhikkhuṁ[bhikkhu-\ACC{\GMU{acc-sg}}] anāpucchā[not.ask-\ABS{\GMU{abs}}] purebhattaṁ[before.meal-\NUL{\GMU{}}] vā[or-\IND{\GMU{ind}}] pacchābhattaṁ[after.meal-\ADV{\GMU{adv}}] vā[or-\IND{\GMU{ind}}] kulesu[family-\LOC{\GMU{loc-pl-n}}] cārittaṁ[visit-\ACC{\GMU{acc-sg-n}}] āpajjeyya[engage-\OPT{\GMU{3-sg-opt}}] aññatra[unless-\ABL{\GMU{abl}}] samayā,[time-\ABL{\GMU{abl-sg}}] pācittiyaṁ.[confess-\ADJ{\GMU{adj}}] Tatthāyaṁ[here.this-\NOM{\GMU{nom-sg}}] samayo:[time-\NOM{\GMU{nom-sg}}] cīvaradānasamayo,[robe.give.time-\NOM{\GMU{nom-sg}}] cīvarakārasamayo.[robe.make.time-\NOM{\GMU{nom-sg}}] Ayaṁ[this-\NOM{\GMU{nom-sg}}] tattha[about that-\ADV{\GMU{adv}}] samayo.[time-\NOM{\GMU{nom-sg}}]
\endgl
\switchcolumn*
\end{flushleft}


{\EnglishColumn

\begin{doublespace}
47. By a bhikkhu who is not ill a four-month invitation for requisites can be accepted; except with a repeated invitation, except with a permanent invitation; if he should accept more than that, (this is a case) involving expiation.
\end{doublespace}}

\switchcolumn


\begin{flushleft}
\begingl
 47.[] Agilānena[not.sick-\ADJ{\GMU{adj}}] bhikkhunā[bhikkhu-\INS{\GMU{ins-sg}}] cātumāsapaccayapavāraṇā[4.month.requisite.invite-\NOM{\GMU{nom-sg-f}}] sāditabbā,[accept-\FUT{\GMU{fut-pass-part}}] aññatra[unless-\ABL{\GMU{abl}}] punapavāraṇāya,[again.invite-\INS{\GMU{ins-sg-f}}] aññatra[unless-\ABL{\GMU{abl}}] niccapavāraṇāya.[perm.invite-\INS{\GMU{ins-sg-f}}] Tato[then-\ABL{\GMU{abl}}] ce[if-\NUL{\GMU{}}] uttariṁ[more-\ADV{\GMU{adv}}] sādiyeyya,[accept-\OPT{\GMU{3-sg-opt}}] pācittiyaṁ.[confess-\ADJ{\GMU{adj}}]
\endgl
\switchcolumn*
\end{flushleft}


{\EnglishColumn

\begin{doublespace}
48. If any bhikkhu should should go to visit an army in action; except with an appropriate reason, (this is a case) involving expiation.
\end{doublespace}}

\switchcolumn


\begin{flushleft}
\begingl
 48.[] Yo[who-\NOM{\GMU{nom-sg}}] pana[(and)-\PART{\GMU{part}}] bhikkhu[bhikkhu-\NOM{\GMU{nom-sg}}] uyyuttaṁ[deploy-\PAST{\GMU{past-part}}] senaṁ[army-\ACC{\GMU{acc-sg-f}}] dassanāya[see-\DAT{\GMU{dat-sg-f}}] gaccheyya,[go-\OPT{\GMU{3-sg-opt}}] aññatra[unless-\ABL{\GMU{abl}}] tathārūpapaccayā,[of such.kind.reason-\INS{\GMU{ins-sg}}] pācittiyaṁ.[confess-\ADJ{\GMU{adj}}]
\endgl
\switchcolumn*
\end{flushleft}


{\EnglishColumn

\begin{doublespace}
49. And if there might be any reason for that bhikkhu for going to the army, two nights or three nights can be stayed within the army by that bhikkhu; if he should stay more than that, (this is a case) involving expiation.
\end{doublespace}}

\switchcolumn


\begin{flushleft}
\begingl
 49.[] Siyā[be-\OPT{\GMU{3-sg-opt}}] ca[-\NUL{\GMU{}}] tassa[of that-\GEN{\GMU{gen-sg}}] bhikkhuno[bhikkhu-\DAT{\GMU{dat-sg}}] kocid’eva[any.just-\NOM{\GMU{nom}}] paccayo[reason-\NOM{\GMU{nom-sg}}] senaṁ[army-\ACC{\GMU{acc-sg-f}}] gamanāya,[go-\DAT{\GMU{dat-sg-n}}] dvirattatirattaṁ[2.night.3.night-\ACC{\GMU{acc-sg-n}}] tena[him-\INS{\GMU{3-sg-ins}}] bhikkhunā[bhikkhu-\INS{\GMU{ins-sg}}] senāya[army-\INS{\GMU{ins-sg-f}}] vasitabbaṁ.[stay-\FUT{\GMU{fut-pass-part}}] Tato[then-\ABL{\GMU{abl}}] ce[if-\NUL{\GMU{}}] uttariṁ[more-\ADV{\GMU{adv}}] vaseyya,[stay-\OPT{\GMU{3-sg-opt}}] pācittiyaṁ.[confess-\ADJ{\GMU{adj}}]
\endgl
\switchcolumn*
\end{flushleft}


{\EnglishColumn

\begin{doublespace}
50. If a bhikkhu staying two nights or three nights within an army should go to a battle-field, or a review, or a massing of the army, or an inspection of units, (this is a case) involving expiation.
\end{doublespace}}

\switchcolumn


\begin{flushleft}
\begingl
 50.[] Dvirattatirattañce[2.night.3.night.if-\ACC{\GMU{acc-sg-n}}] bhikkhu[bhikkhu-\NOM{\GMU{nom-sg}}] senāya[army-\INS{\GMU{ins-sg-f}}] vasamāno,[stay-\PRES{\GMU{pres-part}}] uyyodhikaṁ[battlefield-\NUL{\GMU{}}] vā[or-\IND{\GMU{ind}}] balaggaṁ[review-\ACC{\GMU{acc-sg}}] vā[or-\IND{\GMU{ind}}] senābyūhaṁ[army.mass-\ACC{\GMU{acc-sg}}] vā[or-\IND{\GMU{ind}}] anīkadassanaṁ[front see-\ACC{\GMU{acc-sg-n}}] vā[or-\IND{\GMU{ind}}] gaccheyya,[go-\OPT{\GMU{3-sg-opt}}] pācittiyaṁ.[confess-\ADJ{\GMU{adj}}]
\endgl
\switchcolumn*
\end{flushleft}


{\EnglishColumn

\begin{doublespace}
The section (starting with the rule) on naked ascetics is fifth
\end{doublespace}}

\switchcolumn


\begin{flushleft}
\begingl
 Acelakavaggo[] Arisuddh’etth’āyasmantoañcamo.[]
\endgl
\switchcolumn*
\end{flushleft}


{\EnglishColumn

\begin{doublespace}
51. In drinking alcoholic drink made of grain (-products) or fruit (and/or flower products), (there is a case) involving expiation.
\end{doublespace}}

\switchcolumn


\begin{flushleft}
\begingl
 51.[] Surāmerayapāne[alchohol.drink-\LOC{\GMU{loc-sg-n}}] pācittiyaṁ.[confess-\ADJ{\GMU{adj}}]
\endgl
\switchcolumn*
\end{flushleft}


{\EnglishColumn

\begin{doublespace}
52. In tickling with the fingers, (there is a case) involving expiation.
\end{doublespace}}

\switchcolumn


\begin{flushleft}
\begingl
 52.[] Aṅgulipatodake[finger.poke-\LOC{\GMU{loc-sg-n}}] pācittiyaṁ.[confess-\ADJ{\GMU{adj}}]
\endgl
\switchcolumn*
\end{flushleft}


{\EnglishColumn

\begin{doublespace}
53. In the act of playing in water, (there is a case) involving expiation.
\end{doublespace}}

\switchcolumn


\begin{flushleft}
\begingl
 53.[] Udake[water-\LOC{\GMU{loc-sg-n}}] hassadhamme[fun act-\LOC{\GMU{loc-sg}}] pācittiyaṁ.[confess-\ADJ{\GMU{adj}}]
\endgl
\switchcolumn*
\end{flushleft}


{\EnglishColumn

\begin{doublespace}
54. In disrespect, (there is a case) involving expiation.
\end{doublespace}}

\switchcolumn


\begin{flushleft}
\begingl
 54.[] Anādariye[disrespect-\ADJ{\GMU{adj}}] pācittiyaṁ.[confess-\ADJ{\GMU{adj}}]
\endgl
\switchcolumn*
\end{flushleft}


{\EnglishColumn

\begin{doublespace}
55. If any bhikkhu should scare (another) bhikkhu, (this is a case) involving expiation.
\end{doublespace}}

\switchcolumn


\begin{flushleft}
\begingl
 55.[] Yo[who-\NOM{\GMU{nom-sg}}] pana[(and)-\PART{\GMU{part}}] bhikkhu[bhikkhu-\NOM{\GMU{nom-sg}}] bhikkhuṁ[bhikkhu-\ACC{\GMU{acc-sg}}] bhiṁsāpeyya,[scare-\OPT{\GMU{3-sg-opt}}] pācittiyaṁ.[confess-\ADJ{\GMU{adj}}]
\endgl
\switchcolumn*
\end{flushleft}


{\EnglishColumn

\begin{doublespace}
56. If any bhikkhu who is not ill, desiring to warm (himself), should light a fire or should have (it) lit, except with an appropriate reason, (this is a case) involving expiation.
\end{doublespace}}

\switchcolumn


\begin{flushleft}
\begingl
 56.[] Yo[who-\NOM{\GMU{nom-sg}}] pana[(and)-\PART{\GMU{part}}] bhikkhu[bhikkhu-\NOM{\GMU{nom-sg}}] agilāno[not.sick-\ADJ{\GMU{adj}}] visīvan’āpekkho,[warm.desire-\ADJ{\GMU{adj}}] jotiṁ[fire-\ACC{\GMU{acc-sg}}] samādaheyya[kindle-\OPT{\GMU{3-sg-opt}}] vā[or-\IND{\GMU{ind}}] samādahāpeyya[kindle-\OPT{\GMU{3-sg-opt}}] vā,[or-\IND{\GMU{ind}}] aññatra[unless-\ABL{\GMU{abl}}] tathārūpapaccayā,[of such.kind.reason-\INS{\GMU{ins-sg}}] pācittiyaṁ.[confess-\ADJ{\GMU{adj}}]
\endgl
\switchcolumn*
\end{flushleft}


{\EnglishColumn

\begin{doublespace}
57. If any bhikkhu should should bathe within less than half a month, except at the (right) occasion, (this is a case) involving expiation.
\end{doublespace}}

\switchcolumn


\begin{flushleft}
\begingl
 57.[] Yo[who-\NOM{\GMU{nom-sg}}] pana[(and)-\PART{\GMU{part}}] bhikkhu[bhikkhu-\NOM{\GMU{nom-sg}}] oren’aḍḍhamāsaṁ[less $\sfrac{1}{2}$ month-\ACC{\GMU{acc-sg}}] nhāyeyya,[bathe-\OPT{\GMU{3-sg-opt}}] aññatra[unless-\ABL{\GMU{abl}}] samayā,[time-\ABL{\GMU{abl-sg}}] pācittiyaṁ.[confess-\ADJ{\GMU{adj}}] tatthāyaṁ[here.this-\NOM{\GMU{nom-sg}}] samayo:[time-\NOM{\GMU{nom-sg}}] “Diyaḍḍho[1 ½-\NUM{\GMU{num}}] māso[month-\NOM{\GMU{nom-sg}}] seso[reamin-\NOM{\GMU{nom-sg-n}}] gimhānan”[hot.season-\GEN{\GMU{gen-pl}}] ti,[-\NUL{\GMU{}}] vassānassa[rain season-\GEN{\GMU{gen-sg}}] paṭhamo[first-\ADJ{\GMU{adj}}] māso,[month-\NOM{\GMU{nom-sg}}] icc’ete[these are-\ACC{\GMU{acc-pl}}] aḍḍhateyyamāsā;[2 $\sfrac{1}{2}$ month-\NOM{\GMU{nom-pl}}] uṇhasamayo,[dry.time-\NOM{\GMU{nom-sg}}] pariḷāhasamayo,[humid.time-\NOM{\GMU{nom-sg}}] gilānasamayo,[sick.time-\NOM{\GMU{nom-sg}}] kammasamayo,[work.time-\NOM{\GMU{nom-sg}}] addhānagamanasamayo,[journey.go.time-\NOM{\GMU{nom-sg}}] vātavuṭṭhisamayo.[wind.rain.time-\NOM{\GMU{nom-sg}}] Ayaṁ[this-\NOM{\GMU{nom-sg}}] tattha[about that-\ADV{\GMU{adv}}] samayo.[time-\NOM{\GMU{nom-sg}}]
\endgl
\switchcolumn*
\end{flushleft}


{\EnglishColumn

\begin{doublespace}
58. By a monk with the gain of a new robe a certain stain (from) amongst the three stains is to be applied: dark-blue or muddy (-grey) or dark-brown. If a bhikkhu, not having applied a certain stain (from) amongst the three stains, should use a new robe, (this is a case) involving expiation.
\end{doublespace}}

\switchcolumn


\begin{flushleft}
\begingl
 58.[] Navam’pana[new.-\ADJ{\GMU{adj}}] bhikkhunā[bhikkhu-\INS{\GMU{ins-sg}}] cīvaralābhena[robe.gain-\ADJ{\GMU{adj}}] tiṇṇaṁ[3-\GEN{\GMU{gen-m}}] dubbaṇṇakaraṇānaṁ[stain.make-\ACC{\GMU{acc-sg}}] aññataraṁ[any one, another-\ADJ{\GMU{adj}}] dubbaṇṇakaraṇaṁ[stain.make-\GEN{\GMU{gen-pl-n}}] ādātabbaṁ,[take-\FUT{\GMU{fut-pass-part}}] nīlaṁ[dark blue-\ACC{\GMU{acc-sg}}] vā[or-\IND{\GMU{ind}}] kaddamaṁ[mud-\ACC{\GMU{acc-sg}}] vā[or-\IND{\GMU{ind}}] kāḷasāmaṁ[black.brown-\ACC{\GMU{acc-sg}}] vā.[or-\IND{\GMU{ind}}] Anādā[not.take-\ABS{\GMU{abs}}] ce[if-\NUL{\GMU{}}] bhikkhu[bhikkhu-\NOM{\GMU{nom-sg}}] tiṇṇaṁ[3-\GEN{\GMU{gen-m}}] dubbaṇṇakaraṇānaṁ[stain.make-\ACC{\GMU{acc-sg}}] aññataraṁ[any one, another-\ADJ{\GMU{adj}}] dubbaṇṇakaraṇaṁ[stain.make-\GEN{\GMU{gen-pl-n}}] navaṁ[new-\ADJ{\GMU{adj}}] cīvaraṁ[robe-\ACC{\GMU{acc-sg-n}}] paribhuñjeyya,[use-\OPT{\GMU{3-sg-opt}}] pācittiyaṁ.[confess-\ADJ{\GMU{adj}}]
\endgl
\switchcolumn*
\end{flushleft}


{\EnglishColumn

\begin{doublespace}
59. If any bhikkhu, having himself assigned a robe to a bhikkhu or a bhikkhunì or a male novice or a female novice, should use (it) without withdrawing (the assignment), (this is a case) involving expiation.
\end{doublespace}}

\switchcolumn


\begin{flushleft}
\begingl
 59.[] Yo[who-\NOM{\GMU{nom-sg}}] pana[(and)-\PART{\GMU{part}}] bhikkhu[bhikkhu-\NOM{\GMU{nom-sg}}] bhikkhussa[bhikkhu-\GEN{\GMU{gen-sg}}] vā[or-\IND{\GMU{ind}}] bhikkhuniyā[bhikkhuni-\INS{\GMU{ins-sg-f}}] vā[or-\IND{\GMU{ind}}] sikkhamānāya[trainee-\DAT{\GMU{dat-sg-f}}] vā[or-\IND{\GMU{ind}}] sāmaṇerassa[novice-\DAT{\GMU{dat-sg}}] vā[or-\IND{\GMU{ind}}] sāmaṇeriyā[novice-\DAT{\GMU{dat-sg-f}}] vā[or-\IND{\GMU{ind}}] sāmaṁ[himself-\ADV{\GMU{adv}}] cīvaraṁ[robe-\ACC{\GMU{acc-sg-n}}] vikappetvā[assign-\ABS{\GMU{abs}}] apaccuddhārakaṁ[not.withdraw-\ABS{\GMU{abs}}] paribhuñjeyya,[use-\OPT{\GMU{3-sg-opt}}] pācittiyaṁ.[confess-\ADJ{\GMU{adj}}]
\endgl
\switchcolumn*
\end{flushleft}


{\EnglishColumn

\begin{doublespace}
60. If any bhikkhu should hide a bhikkhu's bowl or robe or sitting-cloth or needle case or body-belt, or have (it) hidden, even if just desiring amusement, (this is a case) involving expiation.
\end{doublespace}}

\switchcolumn


\begin{flushleft}
\begingl
 60.[] Yo[who-\NOM{\GMU{nom-sg}}] pana[(and)-\PART{\GMU{part}}] bhikkhu[bhikkhu-\NOM{\GMU{nom-sg}}] bhikkhussa[bhikkhu-\GEN{\GMU{gen-sg}}] pattaṁ[bowl-\ACC{\GMU{acc-sg}}] vā[or-\IND{\GMU{ind}}] cīvaraṁ[robe-\ACC{\GMU{acc-sg-n}}] vā[or-\IND{\GMU{ind}}] nisīdanaṁ[sit cloth-\ACC{\GMU{acc-sg-n}}] vā[or-\IND{\GMU{ind}}] sūcigharaṁ[needle case-\ACC{\GMU{acc-sg-n}}] vā[or-\IND{\GMU{ind}}] kāyabandhanaṁ[body.belt-\ACC{\GMU{acc-sg-n}}] vā[or-\IND{\GMU{ind}}] apanidheyya[hide-\OPT{\GMU{3-sg-opt}}] vā[or-\IND{\GMU{ind}}] apanidhāpeyya[hide-\OPT{\GMU{3-sg-opt}}] vā,[or-\IND{\GMU{ind}}] antamaso[even so much as-\IND{\GMU{ind}}] hass’āpekkho’pi,[fun desire-\ADJ{\GMU{adj}}] pācittiyaṁ.[confess-\ADJ{\GMU{adj}}]
\endgl
\switchcolumn*
\end{flushleft}


{\EnglishColumn

\begin{doublespace}
The section (starting with the rule) on alcoholic drink is sixth.
\end{doublespace}}

\switchcolumn


\begin{flushleft}
\begingl
 Surāpānavaggo[] Chaṭṭho.[]
\endgl
\switchcolumn*
\end{flushleft}


{\EnglishColumn

\begin{doublespace}
61. If any bhikkhu should intentionally deprive a living being of life, (this is a case) involving expiation.
\end{doublespace}}

\switchcolumn


\begin{flushleft}
\begingl
 61.[] Yo[who-\NOM{\GMU{nom-sg}}] pana[(and)-\PART{\GMU{part}}] bhikkhu[bhikkhu-\NOM{\GMU{nom-sg}}] sañcicca[deliberate-\ABS{\GMU{abs}}] pāṇaṁ[being-\ACC{\GMU{acc-sg}}] jīvitā[life-\ABL{\GMU{abl-sg-n}}] voropeyya,[deprive-\OPT{\GMU{3-sg-opt}}] pācittiyaṁ.[confess-\ADJ{\GMU{adj}}]
\endgl
\switchcolumn*
\end{flushleft}


{\EnglishColumn

\begin{doublespace}
62. If any bhikkhu should knowingly use water containing living beings, (this is a case) involving expiation.
\end{doublespace}}

\switchcolumn


\begin{flushleft}
\begingl
 62.[] Yo[who-\NOM{\GMU{nom-sg}}] pana[(and)-\PART{\GMU{part}}] bhikkhu[bhikkhu-\NOM{\GMU{nom-sg}}] jānaṁ[know-\NOM{\GMU{nom-sg}}] sappāṇakaṁ[with life-\ADJ{\GMU{adj}}] udakaṁ[water-\ACC{\GMU{acc-sg-n}}] paribhuñjeyya,[use-\OPT{\GMU{3-sg-opt}}] pācittiyaṁ.[confess-\ADJ{\GMU{adj}}]
\endgl
\switchcolumn*
\end{flushleft}


{\EnglishColumn

\begin{doublespace}
63. If any bhikkhu should knowingly agitate for further (legal) action a legal issue which has been disposed of according to the law, (this is a case) involving expiation.
\end{doublespace}}

\switchcolumn


\begin{flushleft}
\begingl
 63.[] Yo[who-\NOM{\GMU{nom-sg}}] pana[(and)-\PART{\GMU{part}}] bhikkhu[bhikkhu-\NOM{\GMU{nom-sg}}] jānaṁ[know-\NOM{\GMU{nom-sg}}] yathādhammaṁ[accord.law-\ADV{\GMU{adv}}] nīhatādhikaraṇaṁ[settle.issue-\ACC{\GMU{acc-sg-n}}] punakammāya[further.action-\DAT{\GMU{dat-sg}}] ukkoṭeyya,[agitate-\OPT{\GMU{3-sg-opt}}] pācittiyaṁ.[confess-\ADJ{\GMU{adj}}]
\endgl
\switchcolumn*
\end{flushleft}


{\EnglishColumn

\begin{doublespace}
64. If any bhikkhu should knowingly have a person who is less than twenty years (old) fully admitted (into the bhikkhu-community), then that person is one who has not been fully admitted and those bhikkhus are blameworthy. Because of that, this (is a case) involving expiation.
\end{doublespace}}

\switchcolumn


\begin{flushleft}
\begingl
 64.[] Yo[who-\NOM{\GMU{nom-sg}}] pana[(and)-\PART{\GMU{part}}] bhikkhu[bhikkhu-\NOM{\GMU{nom-sg}}] bhikkhussa[bhikkhu-\GEN{\GMU{gen-sg}}] jānaṁ[know-\NOM{\GMU{nom-sg}}] duṭṭhullaṁ[obscene-\ADJ{\GMU{adj}}] āpattiṁ[offense-\ACC{\GMU{acc-sg-f}}] paṭicchādeyya,[conceal-\OPT{\GMU{3-sg-opt}}] pācittiyaṁ.[confess-\ADJ{\GMU{adj}}]
\endgl
\switchcolumn*
\end{flushleft}


{\EnglishColumn

\begin{doublespace}
65. If any bhikkhu should knowingly have a person who is less than twenty years (old) fully admitted (into the bhikkhu-community), then that person is one who has not been fully admitted and those bhikkhus are blameworthy. Because of that, this (is a case) involving expiation.
\end{doublespace}}

\switchcolumn


\begin{flushleft}
\begingl
 65.[] Yo[who-\NOM{\GMU{nom-sg}}] pana[(and)-\PART{\GMU{part}}] bhikkhu[bhikkhu-\NOM{\GMU{nom-sg}}] jānaṁ[know-\NOM{\GMU{nom-sg}}] ūnavīsativassaṁ[less.20.year-\ADJ{\GMU{adj}}] puggalaṁ[person-\ACC{\GMU{acc-sg}}] upasampādeyya,[admitt-\OPT{\GMU{3-sg-opt}}] so[he-\NOM{\GMU{nom-sg}}] ca[-\NUL{\GMU{}}] puggalo[person-\NOM{\GMU{nom-sg}}] anupasampanno,[not.admitted-\ADJ{\GMU{adj}}] te[you-\DAT{\GMU{dat-sg-n}}] ca[-\NUL{\GMU{}}] bhikkhū[bhikkhu-\NOM{\GMU{nom-pl}}] gārayhā.[blame-\FUT{\GMU{fut-pass-part}}] Idaṁ[this-\ACC{\GMU{acc-sg-n}}] tasmiṁ[on account of-\LOC{\GMU{loc-sg}}] pācittiyaṁ.[confess-\ADJ{\GMU{adj}}]
\endgl
\switchcolumn*
\end{flushleft}


{\EnglishColumn

\begin{doublespace}
66. If any bhikkhu, having made an arrangement, should knowingly travel together on the same main road with a company of thieves, even (if) just the distance between villages, (this is a case) involving expiation.
\end{doublespace}}

\switchcolumn


\begin{flushleft}
\begingl
 66.[] Yo[who-\NOM{\GMU{nom-sg}}] pana[(and)-\PART{\GMU{part}}] bhikkhu[bhikkhu-\NOM{\GMU{nom-sg}}] jānaṁ[know-\NOM{\GMU{nom-sg}}] theyyasatthena[theif.caravan-\INS{\GMU{ins-sg}}] saddhiṁ[together-\INS{\GMU{ins}}] saṁvidhāya[arrange-\ABS{\GMU{abs}}] ekaddhānamaggaṁ[same road-\ACC{\GMU{acc-sg}}] paṭipajjeyya,[travel-\OPT{\GMU{3-sg-opt}}] antamaso[even so much as-\IND{\GMU{ind}}] gām’antaram’pi,[village.between-\ACC{\GMU{acc-sg-n}}] pācittiyaṁ.[confess-\ADJ{\GMU{adj}}]
\endgl
\switchcolumn*
\end{flushleft}


{\EnglishColumn

\begin{doublespace}
67. If any bhikkhu, having made an arrangement, should travel together with a woman on the same main road, even (if) just the distance between villages, (this is a case) involving expiation.
\end{doublespace}}

\switchcolumn


\begin{flushleft}
\begingl
 67.[] Yo[who-\NOM{\GMU{nom-sg}}] pana[(and)-\PART{\GMU{part}}] bhikkhu[bhikkhu-\NOM{\GMU{nom-sg}}] mātugāmena[woman-\INS{\GMU{ins-sg}}] saddhiṁ[together-\INS{\GMU{ins}}] saṁvidhāya[arrange-\ABS{\GMU{abs}}] ekaddhānamaggaṁ[same road-\ACC{\GMU{acc-sg}}] paṭipajjeyya,[travel-\OPT{\GMU{3-sg-opt}}] antamaso[even so much as-\IND{\GMU{ind}}] gām’antaram’pi,[village.between-\ACC{\GMU{acc-sg-n}}] pācittiyaṁ.[confess-\ADJ{\GMU{adj}}]
\endgl
\switchcolumn*
\end{flushleft}


{\EnglishColumn

\begin{doublespace}
68. If any bhikkhu should say so, “As I understand the Teaching taught by the Fortunate One, these obstructive acts which are spoken of by the Fortunate One: they are not enough to be an obstruction for the one who is being engaged in (them),” (then) that bhikkhu is to be spoken to thus by the bhikkhus: “Venerable, don't say so! Don't misrepresent the Fortunate One; for the misrepresentation of the Fortunate One is not good; for the Fortunate One would not say so; friend, (that) obstructive acts are (really) obstructive is spoken of in manifold ways by the Fortunate One and they are enough to be an obstruction for the one who is being engaged in (them),” and (if) that bhikkhu being spoken to thus by the bhikkhus should persist in the same way (as before), (then) that bhikkhu is to be argued with up to three times by the bhikkhus for the relinquishing of that (view), (and if that bhikkhu,) being argued with up to three times, should relinquish that (view), then this is good, (but) if he should not relinquish (it): (this is a case) involving expiation.
\end{doublespace}}

\switchcolumn


\begin{flushleft}
\begingl
 68.[] Yo[who-\NOM{\GMU{nom-sg}}] pana[(and)-\PART{\GMU{part}}] bhikkhu[bhikkhu-\NOM{\GMU{nom-sg}}] evaṁ[thus-\ADV{\GMU{adv}}] vadeyya,[say-\OPT{\GMU{3-sg-opt}}] “Tathāhaṁ[as.I-\PERS{\GMU{pers}}] bhagavatā[blessed one-\INS{\GMU{ins-sg}}] dhammaṁ[act-\ACC{\GMU{acc-sg}}] desitaṁ[teach-\PAST{\GMU{past-part}}] ājānāmi,[understand-\PRESIND{\GMU{3-sg-presind}}] yathā[just as-\IND{\GMU{ind}}] ye’me[which.these-\NUL{\GMU{}}] antarāyikā[obstruct-\ADJ{\GMU{adj}}] dhammā[rule-\NOM{\GMU{nom-pl}}] vuttā[say-\PAST{\GMU{past-part}}] bhagavatā,[blessed one-\INS{\GMU{ins-sg}}] te[you-\DAT{\GMU{dat-sg-n}}] paṭisevato[engage-\PRES{\GMU{pres-part}}] nālaṁ[not.enough-\IND{\GMU{ind}}] antarāyāyā”[obstruct-\DAT{\GMU{dat-sg}}] ti.[-\NUL{\GMU{}}] So[he-\NOM{\GMU{nom-sg}}] bhikkhu[bhikkhu-\NOM{\GMU{nom-sg}}] bhikkhūhi[bhikkhu-\INS{\GMU{ins-pl}}] evam’assa[thus-\TBD{\GMU{tbd}}] vacanīyo,[address-\FUT{\GMU{fut-pass-part}}] “Mā[do not-\PART{\GMU{part}}] āyasmā[Ven.-\NOM{\GMU{nom-sg}}] evaṁ[thus-\ADV{\GMU{adv}}] avaca.[say-\NUL{\GMU{}}] Mā[do not-\PART{\GMU{part}}] bhagavantaṁ[blessed one-\ACC{\GMU{acc-sg}}] abbhācikkhi.[misrepresent-\AOR{\GMU{2-sg-aor}}] Na[not-\PART{\GMU{part}}] hi[for-\IND{\GMU{ind}}] sādhu[good-\IND{\GMU{ind}}] bhagavato[blessed one-\GEN{\GMU{gen-sg}}] abbhakkhānaṁ.[misrepresent-\NUL{\GMU{}}] Na[not-\PART{\GMU{part}}] hi[for-\IND{\GMU{ind}}] bhagavā[blessed one-\NOM{\GMU{nom-sg}}] evaṁ[thus-\ADV{\GMU{adv}}] vadeyya.[say-\OPT{\GMU{3-sg-opt}}] Anekapariyāyena[various ways-\ADV{\GMU{adv}}] āvuso[friend-\VOC{\GMU{voc-sg}}] antarāyikā[obstruct-\ADJ{\GMU{adj}}] dhammā[rule-\NOM{\GMU{nom-pl}}] vuttā[say-\PAST{\GMU{past-part}}] bhagavatā,[blessed one-\INS{\GMU{ins-sg}}] alañca[?-\NUL{\GMU{}}] pana[(and)-\PART{\GMU{part}}] te[you-\DAT{\GMU{dat-sg-n}}] paṭisevato[engage-\PRES{\GMU{pres-part}}] antarāyāyā”[obstruct-\DAT{\GMU{dat-sg}}] ti.[-\NUL{\GMU{}}] Evañca[thus-\ADV{\GMU{adv}}] so[he-\NOM{\GMU{nom-sg}}] bhikkhu[bhikkhu-\NOM{\GMU{nom-sg}}] bhikkhūhi[bhikkhu-\INS{\GMU{ins-pl}}] vuccamāno[address-\PRES{\GMU{pres-pass-part}}] tath’eva[in same way-\NUL{\GMU{}}] paggaṇheyya,[uphold-\OPT{\GMU{3-sg-opt}}] so[he-\NOM{\GMU{nom-sg}}] bhikkhu[bhikkhu-\NOM{\GMU{nom-sg}}] bhikkhūhi[bhikkhu-\INS{\GMU{ins-pl}}] yāvatatiyaṁ[up to.3rd time-\ADV{\GMU{adv}}] samanubhāsitabbo[admonish-\FUT{\GMU{fut-pass-part}}] tassa[of that-\GEN{\GMU{gen-sg}}] paṭinissaggāya.[relinquish-\DAT{\GMU{dat-sg}}] Yāvatatiyañce[up to.3rd time-\ADV{\GMU{adv}}] samanubhāsiyamāno[admonish-\PRES{\GMU{pres-part}}] taṁ[that-\ACC{\GMU{acc-sg}}] paṭinissajjeyya,[relinquish-\OPT{\GMU{3-sg-opt}}] icc’etaṁ[thus.this-\ACC{\GMU{acc-sg}}] kusalaṁ.[good-\NOM{\GMU{nom-sg-n}}] No[not-\NEG{\GMU{neg-part}}] ce[if-\NUL{\GMU{}}] paṭinissajjeyya,[relinquish-\OPT{\GMU{3-sg-opt}}] pācittiyaṁ.[confess-\ADJ{\GMU{adj}}]
\endgl
\switchcolumn*
\end{flushleft}


{\EnglishColumn

\begin{doublespace}
69. If any bhikkhu knowingly should eat together with, or should live together with, or should use a sleeping place together with a bhikkhu who is speaking thus, who has not performed the normal procedure, who has not relinquished that view, (this is a case) involving expiation.
\end{doublespace}}

\switchcolumn


\begin{flushleft}
\begingl
 69.[] Yo[who-\NOM{\GMU{nom-sg}}] pana[(and)-\PART{\GMU{part}}] bhikkhu[bhikkhu-\NOM{\GMU{nom-sg}}] jānaṁ[know-\NOM{\GMU{nom-sg}}] tathāvādinā[thus.speak-\ADJ{\GMU{adj}}] bhikkhunā[bhikkhu-\INS{\GMU{ins-sg}}] akaṭānudhammena[not.make.normal.procedure-\ADJ{\GMU{adj}}] taṁ[that-\ACC{\GMU{acc-sg}}] diṭṭhiṁ[view-\ACC{\GMU{acc-sg}}] appaṭinissaṭṭhena,[not.relinquish-\ADJ{\GMU{adj}}] saddhiṁ[together-\INS{\GMU{ins}}] sambhuñjeyya[with.eat-\OPT{\GMU{3-sg-opt}}] vā[or-\IND{\GMU{ind}}] saṁvaseyya[live-\OPT{\GMU{3-sg-opt}}] vā[or-\IND{\GMU{ind}}] saha[with face.remove-\IND{\GMU{ind}}] vā[or-\IND{\GMU{ind}}] seyyaṁ[bedding-\ACC{\GMU{acc-sg-f}}] kappeyya,[use-\OPT{\GMU{3-sg-opt}}] pācittiyaṁ.[confess-\ADJ{\GMU{adj}}]
\endgl
\switchcolumn*
\end{flushleft}


{\EnglishColumn

\begin{doublespace}
70. If a novice should say so too, “As I understand the Teaching taught by the Fortunate One, these obstructive acts which are spoken of by the Fortunate One: they are not enough to be an obstruction for the one who is being engaged in (them),” (then) that novice is to be spoken to thus by the bhikkhus, “Friend novice, don't say so! Don't misrepresent the Fortunate One; for the misrepresentation of the Fortunate One is not good; for the Fortunate One would not say so; friend novice, (that) obstructive acts are (really) obstructive is spoken of in manifold ways by the Fortunate One and they are enough to be an obstruction for the one who is engaging (in them),” and if that novice being spoken to thus by the bhikkhus should persist in the same way (as before), (then) that novice is to be spoken to thus by the bhikkhus, “From today on, friend novice, the Fortunate One is not to be referred to as the teacher by you, and also the two or three nights sleeping together (in one room) with bhikkhus that other novices get, that too is not for you. Go away, disappear!” If any bhikkhu knowingly should treat kindly such an expelled novice, or should make (him) attend (to himself), or should eat together with (him), or should use a sleeping place together with (him), (this is a case) involving expiation.
\end{doublespace}}

\switchcolumn


\begin{flushleft}
\begingl
 70.[] Samaṇuddeso’pi[novice-\NOM{\GMU{nom-sg}}] ce[if-\NUL{\GMU{}}] evaṁ[thus-\ADV{\GMU{adv}}] vadeyya,[say-\OPT{\GMU{3-sg-opt}}] “Tathāhaṁ[as.I-\PERS{\GMU{pers}}] bhagavatā[blessed one-\INS{\GMU{ins-sg}}] dhammaṁ[act-\ACC{\GMU{acc-sg}}] desitaṁ[teach-\PAST{\GMU{past-part}}] ājānāmi,[understand-\PRESIND{\GMU{3-sg-presind}}] yathā[just as-\IND{\GMU{ind}}] ye’me[which.these-\NUL{\GMU{}}] antarāyikā[obstruct-\ADJ{\GMU{adj}}] dhammā[rule-\NOM{\GMU{nom-pl}}] vuttā[say-\PAST{\GMU{past-part}}] bhagavatā,[blessed one-\INS{\GMU{ins-sg}}] te[you-\DAT{\GMU{dat-sg-n}}] paṭisevato[engage-\PRES{\GMU{pres-part}}] nālaṁ[not.enough-\IND{\GMU{ind}}] antarāyāyā”[obstruct-\DAT{\GMU{dat-sg}}] ti.[-\NUL{\GMU{}}] So[he-\NOM{\GMU{nom-sg}}] samaṇuddeso[novice-\NOM{\GMU{nom-sg}}] bhikkhūhi[bhikkhu-\INS{\GMU{ins-pl}}] evam’assa[thus-\TBD{\GMU{tbd}}] vacanīyo,[address-\FUT{\GMU{fut-pass-part}}] “Mā[do not-\PART{\GMU{part}}] āvuso[friend-\VOC{\GMU{voc-sg}}] samaṇuddesa[novice-\VOC{\GMU{voc-sg}}] evaṁ[thus-\ADV{\GMU{adv}}] avaca.[say-\NUL{\GMU{}}] Mā[do not-\PART{\GMU{part}}] bhagavantaṁ[blessed one-\ACC{\GMU{acc-sg}}] abbhācikkhi.[misrepresent-\AOR{\GMU{2-sg-aor}}] Na[not-\PART{\GMU{part}}] hi[for-\IND{\GMU{ind}}] sādhu[good-\IND{\GMU{ind}}] bhagavato[blessed one-\GEN{\GMU{gen-sg}}] abbhakkhānaṁ.[misrepresent-\NUL{\GMU{}}] na[not-\PART{\GMU{part}}] hi[for-\IND{\GMU{ind}}] bhagavā[blessed one-\NOM{\GMU{nom-sg}}] evaṁ[thus-\ADV{\GMU{adv}}] vadeyya.[say-\OPT{\GMU{3-sg-opt}}] anekapariyāyena[various ways-\ADV{\GMU{adv}}] āvuso[friend-\VOC{\GMU{voc-sg}}] samaṇuddesa[novice-\VOC{\GMU{voc-sg}}] antarāyikā[obstruct-\ADJ{\GMU{adj}}] dhammā[rule-\NOM{\GMU{nom-pl}}] vuttā[say-\PAST{\GMU{past-part}}] bhagavatā,[blessed one-\INS{\GMU{ins-sg}}] alañca[?-\NUL{\GMU{}}] pana[(and)-\PART{\GMU{part}}] te[you-\DAT{\GMU{dat-sg-n}}] paṭisevato[engage-\PRES{\GMU{pres-part}}] antarāyāyā”[obstruct-\DAT{\GMU{dat-sg}}] ti.[-\NUL{\GMU{}}] Evañca[thus-\ADV{\GMU{adv}}] so[he-\NOM{\GMU{nom-sg}}] samaṇuddeso[novice-\NOM{\GMU{nom-sg}}] bhikkhūhi[bhikkhu-\INS{\GMU{ins-pl}}] vuccamāno[address-\PRES{\GMU{pres-pass-part}}] tath’eva[in same way-\NUL{\GMU{}}] paggaṇheyya,[uphold-\OPT{\GMU{3-sg-opt}}] so[he-\NOM{\GMU{nom-sg}}] samaṇuddeso[novice-\NOM{\GMU{nom-sg}}] bhikkhūhi[bhikkhu-\INS{\GMU{ins-pl}}] evam’assa[thus-\TBD{\GMU{tbd}}] vacanīyo,[address-\FUT{\GMU{fut-pass-part}}] “Ajjatagge[today.from-\ADV{\GMU{adv}}] te[you-\DAT{\GMU{dat-sg-n}}] āvuso[friend-\VOC{\GMU{voc-sg}}] samaṇuddesa[novice-\VOC{\GMU{voc-sg}}] na[not-\PART{\GMU{part}}] c’eva[and.if-\NUL{\GMU{}}] so[he-\NOM{\GMU{nom-sg}}] bhagavā[blessed one-\NOM{\GMU{nom-sg}}] satthā[teacher-\NOM{\GMU{nom-sg}}] apadisitabbo,[refer-\FUT{\GMU{fut-pass-part}}] yam’pi[] c’aññe[and.other-\ADJ{\GMU{adj}}] samaṇuddesā[novice-\NOM{\GMU{nom-pl}}] labhanti[gain-\PRESIND{\GMU{3-pl-presind}}] bhikkhūhi[bhikkhu-\INS{\GMU{ins-pl}}] saddhiṁ[together-\INS{\GMU{ins}}] dvirattatirattaṁ[2.night.3.night-\ACC{\GMU{acc-sg-n}}] sahaseyyaṁ,[with.bedding-\ACC{\GMU{acc-sg-f}}] sā’pi[that-\NOM{\GMU{nom-f}}] te[you-\DAT{\GMU{dat-sg-n}}] n’atthi.[not.is-\PRESIND{\GMU{3-sg-presind}}] Cara’pi[go-\IMP{\GMU{2-sg-imp}}] re[] vinassā”[lose-\IMP{\GMU{2-sg-imp}}] ti.[-\NUL{\GMU{}}] Yo[who-\NOM{\GMU{nom-sg}}] pana[(and)-\PART{\GMU{part}}] bhikkhu[bhikkhu-\NOM{\GMU{nom-sg}}] jānaṁ[know-\NOM{\GMU{nom-sg}}] tathānāsitaṁ[thus.expel-\PAST{\GMU{past-part}}] samaṇuddesaṁ[novice-\ACC{\GMU{acc-sg}}] upalāpeyya[console-\OPT{\GMU{3-sg-opt}}] vā[or-\IND{\GMU{ind}}] upaṭṭhāpeyya[attend-\OPT{\GMU{3-sg-opt}}] vā[or-\IND{\GMU{ind}}] sambhuñjeyya[with.eat-\OPT{\GMU{3-sg-opt}}] vā[or-\IND{\GMU{ind}}] saha[with face.remove-\IND{\GMU{ind}}] vā[or-\IND{\GMU{ind}}] seyyaṁ[bedding-\ACC{\GMU{acc-sg-f}}] kappeyya,[use-\OPT{\GMU{3-sg-opt}}] pācittiyaṁ.[confess-\ADJ{\GMU{adj}}]
\endgl
\switchcolumn*
\end{flushleft}


{\EnglishColumn

\begin{doublespace}
The section (starting with the rule) on living beings is seventh
\end{doublespace}}

\switchcolumn


\begin{flushleft}
\begingl
 Sappāṇavaggo[] Sattamo.[]
\endgl
\switchcolumn*
\end{flushleft}


{\EnglishColumn

\begin{doublespace}
71. If any bhikkhu when being righteously spoken to by bhikkhus should say so, “Friends, I shall not train in this training precept for as long as I can not question another bhikkhu (about it) who is a learned memoriser of the discipline,” (this is a case) involving expiation.
\end{doublespace}}

\switchcolumn


\begin{flushleft}
\begingl
 71.[] Yo[who-\NOM{\GMU{nom-sg}}] pana[(and)-\PART{\GMU{part}}] bhikkhu[bhikkhu-\NOM{\GMU{nom-sg}}] bhikkhūhi[bhikkhu-\INS{\GMU{ins-pl}}] sahadhammikaṁ[with.dhamma-\ADJ{\GMU{adj}}] vuccamāno[address-\PRES{\GMU{pres-pass-part}}] evaṁ[thus-\ADV{\GMU{adv}}] vadeyya,[say-\OPT{\GMU{3-sg-opt}}] “Na[not-\PART{\GMU{part}}] tāvāhaṁ[I-\PERS PRO{\GMU{1-sg-pers pro}}] āvuso[friend-\VOC{\GMU{voc-sg}}] etasmiṁ[this-\LOC{\GMU{loc-sg}}] sikkhāpade[train.rule-\LOC{\GMU{loc-sg-n}}] sikkhissāmi,[train-\FUT{\GMU{1-pl-fut}}] yāva[until-\IND{\GMU{ind}}] n’aññaṁ[not.another-\ADJ{\GMU{adj}}] bhikkhuṁ[bhikkhu-\ACC{\GMU{acc-sg}}] byattaṁ[wise-\ADJ{\GMU{adj}}] vinayadharaṁ[discipline.bearer-\ACC{\GMU{acc-sg}}] paripucchāmī”[about.question-\PRESIND{\GMU{1-sg-presind}}] ti,[-\NUL{\GMU{}}] pācittiyaṁ.[confess-\ADJ{\GMU{adj}}] Sikkhamānena[train-\ADJ{\GMU{adj}}] bhikkhave[bhikkhu-\VOC{\GMU{voc-pl}}] bhikkhunā[bhikkhu-\INS{\GMU{ins-sg}}] aññātabbaṁ[know-\FUT{\GMU{fut-pass-part}}] paripucchitabbaṁ[about.question-\FUT{\GMU{fut-pass-part}}] paripañhitabbaṁ.[consider-\FUT{\GMU{fut-pass-part}}] Ayaṁ[this-\NOM{\GMU{nom-sg}}] tattha[about that-\ADV{\GMU{adv}}] sāmīci.[proper procedure-\NOM{\GMU{nom-sg-f}}]
\endgl
\switchcolumn*
\end{flushleft}


{\EnglishColumn

\begin{doublespace}
72. If any bhikkhu, when the Disciplinary Code is being recited, should say so, “But why these small and minute training precepts that are recited? They just lead to worry, annoyance, (and) discomfort.” In the disparaging of training precepts, (there is a case) involving expiation.
\end{doublespace}}

\switchcolumn


\begin{flushleft}
\begingl
 72.[] Yo[who-\NOM{\GMU{nom-sg}}] pana[(and)-\PART{\GMU{part}}] bhikkhu[bhikkhu-\NOM{\GMU{nom-sg}}] pāṭimokkhe[disciplinary code-\LOC{\GMU{loc-sg-n}}] uddissamāne[recite-\PRES{\GMU{pres-part}}] evaṁ[thus-\ADV{\GMU{adv}}] vadeyya,[say-\OPT{\GMU{3-sg-opt}}] “Kimpan’imehi[] khuddānukhuddakehi[small.very.small-\ADJ{\GMU{adj}}] sikkhāpadehi[train.rule-\INS{\GMU{ins-pl-n}}] uddiṭṭhehi,[tear off-\ADJ{\GMU{adj}}] yāvad’eva[until.just-\ADV{\GMU{adv}}] kukkuccāya[worry-\DAT{\GMU{dat-sg-n}}] vihesāya[annoy-\DAT{\GMU{dat-sg-f}}] vilekhāya[discomfort-\DAT{\GMU{dat-sg}}] saṁvattantī”[conduce-\PRESIND{\GMU{3-pl-presind}}] ti.[-\NUL{\GMU{}}] Sikkhāpadavivaṇṇanake,[train.rule.disparage-\LOC{\GMU{loc-sg}}] pācittiyaṁ.[confess-\ADJ{\GMU{adj}}]
\endgl
\switchcolumn*
\end{flushleft}


{\EnglishColumn

\begin{doublespace}
73. If any bhikkhu when the Disciplinary Code is being recited half-monthly should say so, “Only now I know! This too, indeed, is a case which has been handed down in the Sutta, which has been included in the Sutta, which comes up for recitation half-monthly!” (and) if other bhikkhus should know (about) that bhikkhu (thus), “This bhikkhu has sat (in) two or three times previously when the Disciplinary Code was being recited. What to say about more (times than that)!” (then) there is no release for that bhikkhu through not-knowing, and whatever the offence is that he has committed there, he is to be made to do according to that case and moreover his deluding is to be exposed, “Because of that (there are) losses for you, because of that (it) has been ill-gained by you, that you, when the Disciplinary Code is being recited, do not take (it) to mind (after) having focussed carefully (on it).” Because of that deluding, this (is a case) involving expiation.
\end{doublespace}}

\switchcolumn


\begin{flushleft}
\begingl
 73.[] Yo[who-\NOM{\GMU{nom-sg}}] pana[(and)-\PART{\GMU{part}}] bhikkhu[bhikkhu-\NOM{\GMU{nom-sg}}] anvaḍḍhamāsaṁ[after $\sfrac{1}{2}$ month-\ACC{\GMU{acc-sg}}] pāṭimokkhe[disciplinary code-\LOC{\GMU{loc-sg-n}}] uddissamāne[recite-\PRES{\GMU{pres-part}}] evaṁ[thus-\ADV{\GMU{adv}}] vadeyya,[say-\OPT{\GMU{3-sg-opt}}] “Idān’eva[now only-\ADV{\GMU{adv}}] kho[indeed!-\EMPH{\GMU{emph}}] ahaṁ[I-\PERS PRO{\GMU{1-sg-pers pro}}] ājānāmi,[understand-\PRESIND{\GMU{3-sg-presind}}] ‘Ayam’pi[] kira[really!-\PART{\GMU{part}}] dhammo[case-\NOM{\GMU{nom-sg}}] sutt’āgato[sutta.become-\ADJ{\GMU{adj}}] suttapariyāpanno[sutta.include-\ADJ{\GMU{adj}}] anvaḍḍhamāsaṁ[after $\sfrac{1}{2}$ month-\ACC{\GMU{acc-sg}}] uddesaṁ[recitation-\ACC{\GMU{acc-sg}}] āgacchatī’”[come up-\PRESIND{\GMU{presind-sg}}] ti.[-\NUL{\GMU{}}] Tañce[him-\ACC{\GMU{acc-sg}}] bhikkhuṁ[bhikkhu-\ACC{\GMU{acc-sg}}] aññe[other class-\ADJ{\GMU{adj}}] bhikkhū[bhikkhu-\NOM{\GMU{nom-pl}}] jāneyyuṁ,[know-\OPT{\GMU{1-sg-opt}}] “Nisinnapubbaṁ[sit.before-\ACC{\GMU{acc-sg-n}}] iminā[this-\INS{\GMU{ins-sg-n}}] bhikkhunā[bhikkhu-\INS{\GMU{ins-sg}}] dvittikkhattuṁ[2.or.3.times-\ADV{\GMU{adv}}] pāṭimokkhe[disciplinary code-\LOC{\GMU{loc-sg-n}}] uddissamāne,[recite-\PRES{\GMU{pres-part}}] ko[who-\NOM{\GMU{nom-sg}}] pana[(and)-\PART{\GMU{part}}] vādo[speech-\NOM{\GMU{nom-sg}}] bhiyyo”[more-\ADV{\GMU{adv}}] ti,[-\NUL{\GMU{}}] na[not-\PART{\GMU{part}}] ca[-\NUL{\GMU{}}] tassa[of that-\GEN{\GMU{gen-sg}}] bhikkhuno[bhikkhu-\DAT{\GMU{dat-sg}}] aññāṇakena[not.know-\INS{\GMU{ins-sg}}] mutti[release-\NOM{\GMU{nom-sg-f}}] atthi.[has-\PRESIND{\GMU{3-sg-presind}}] Yañca[and whatever-\NUL{\GMU{}}] tattha[about that-\ADV{\GMU{adv}}] āpattiṁ[offense-\ACC{\GMU{acc-sg-f}}] āpanno,[commit-\PAST{\GMU{past-part}}] tañca[that-\ACC{\GMU{acc-sg}}] yathādhammo[accord.law-\NOM{\GMU{nom-sg}}] kāretabbo,[make-\FUT{\GMU{fut-pass-part}}] uttariñc’assa[moreover.\&.his-\DAT{\GMU{dat-sg}}] moho[delusion-\NOM{\GMU{nom-sg}}] āropetabbo,[expose-\FUT{\GMU{fut-pass-part}}] “Tassa[of that-\GEN{\GMU{gen-sg}}] te[you-\DAT{\GMU{dat-sg-n}}] āvuso[friend-\VOC{\GMU{voc-sg}}] alābhā,[non.gain-\NOM{\GMU{nom-sg-pl}}] tassa[of that-\GEN{\GMU{gen-sg}}] te[you-\DAT{\GMU{dat-sg-n}}] dulladdhaṁ,[ill.gain-\NOM{\GMU{nom-sg-n}}] yaṁ[that-\ACC{\GMU{acc-sg}}] tvaṁ[you-\NOM{\GMU{nom-sg}}] pāṭimokkhe[disciplinary code-\LOC{\GMU{loc-sg-n}}] uddissamāne[recite-\PRES{\GMU{pres-part}}] na[not-\PART{\GMU{part}}] sādhukaṁ[well-\ADV{\GMU{adv}}] aṭṭhikatvā[purpose.made-\ABS{\GMU{abs}}] manasikarosī”[mind.attend-\NOM{\GMU{2-sg-nom}}] ti.[-\NUL{\GMU{}}] Idaṁ[this-\ACC{\GMU{acc-sg-n}}] tasmiṁ[on account of-\LOC{\GMU{loc-sg}}] mohanake,[delusion-\LOC{\GMU{loc-sg}}] pācittiyaṁ.[confess-\ADJ{\GMU{adj}}]
\endgl
\switchcolumn*
\end{flushleft}


{\EnglishColumn

\begin{doublespace}
74. If any bhikkhu who is resentful (and) displeased should give a blow to a bhikkhu, (this is a case) involving expiation.
\end{doublespace}}

\switchcolumn


\begin{flushleft}
\begingl
 74.[] Yo[who-\NOM{\GMU{nom-sg}}] pana[(and)-\PART{\GMU{part}}] bhikkhu[bhikkhu-\NOM{\GMU{nom-sg}}] bhikkhussa[bhikkhu-\GEN{\GMU{gen-sg}}] kupito[disturb-\PAST{\GMU{past-part}}] anattamano[displeased-\ADJ{\GMU{adj}}] pahāraṁ[blow-\ACC{\GMU{acc-sg}}] dadeyya,[give-\OPT{\GMU{3-sg-opt}}] pācittiyaṁ.[confess-\ADJ{\GMU{adj}}]
\endgl
\switchcolumn*
\end{flushleft}


{\EnglishColumn

\begin{doublespace}
75. If any bhikkhu should brandish the palm of the hand (threateningly) like (one holds) a dagger to a bhikkhu, (this is a case) involving expiation.
\end{doublespace}}

\switchcolumn


\begin{flushleft}
\begingl
 75.[] Yo[who-\NOM{\GMU{nom-sg}}] pana[(and)-\PART{\GMU{part}}] bhikkhu[bhikkhu-\NOM{\GMU{nom-sg}}] bhikkhussa[bhikkhu-\GEN{\GMU{gen-sg}}] kupito[disturb-\PAST{\GMU{past-part}}] anattamano[displeased-\ADJ{\GMU{adj}}] talasattikaṁ[palm.spear-\ACC{\GMU{acc-sg-n}}] uggireyya,[raise-\OPT{\GMU{3-sg-opt}}] pācittiyaṁ.[confess-\ADJ{\GMU{adj}}]
\endgl
\switchcolumn*
\end{flushleft}


{\EnglishColumn

\begin{doublespace}
76. If any bhikkhu should should accuse a bhikkhu with a groundless (case concerning) the community in the beginning and in the rest (of the procedure), (this is a case) involving expiation.
\end{doublespace}}

\switchcolumn


\begin{flushleft}
\begingl
 76.[] Yo[who-\NOM{\GMU{nom-sg}}] pana[(and)-\PART{\GMU{part}}] bhikkhu[bhikkhu-\NOM{\GMU{nom-sg}}] bhikkhuṁ[bhikkhu-\ACC{\GMU{acc-sg}}] amūlakena[without cause-\ADJ{\GMU{adj}}] saṅghādisesena[-\INS{\GMU{ins-s}}] anuddhaṁseyya,[accuse-\OPT{\GMU{3-sg-opt}}] pācittiyaṁ.[confess-\ADJ{\GMU{adj}}]
\endgl
\switchcolumn*
\end{flushleft}


{\EnglishColumn

\begin{doublespace}
77. If any bhikkhu should deliberately provoke worry for a bhikkhu (thinking), “Thus there will be discomfort for him, even (if only) for a short time,” having made just this the reason, (and) not another, (this is a case) involving expiation.
\end{doublespace}}

\switchcolumn


\begin{flushleft}
\begingl
 77.[] Yo[who-\NOM{\GMU{nom-sg}}] pana[(and)-\PART{\GMU{part}}] bhikkhu[bhikkhu-\NOM{\GMU{nom-sg}}] bhikkhussa[bhikkhu-\GEN{\GMU{gen-sg}}] sañcicca[deliberate-\ABS{\GMU{abs}}] kukkuccaṁ[worry-\ACC{\GMU{acc-sg-n}}] upadaheyya,[provoke-\OPT{\GMU{3-sg-opt}}] “Iti’ssa[thus him-\DAT{\GMU{dat-sg}}] muhuttam’pi[moment-\ACC{\GMU{acc-sg}}] aphāsu[disease-\NOM{\GMU{nom-sg-n}}] bhavissatī”[to be-\FUT{\GMU{3-sg-fut}}] ti.[-\NUL{\GMU{}}] Etad’eva[this.just-\ACC{\GMU{acc-sg-n}}] paccayaṁ[reason-\ACC{\GMU{acc-sg}}] karitvā[done-\ABS{\GMU{abs}}] anaññaṁ,[not.another-\ADJ{\GMU{adj}}] pācittiyaṁ.[confess-\ADJ{\GMU{adj}}]
\endgl
\switchcolumn*
\end{flushleft}


{\EnglishColumn

\begin{doublespace}
78. If any bhikkhu should stand overhearing bhikkhus who are arguing, who are quarrelling, who are engaged in dispute (thinking), “I shall hear what these ones will say,” having made just this the reason, (and) not another, (this is a case) involving expiation.
\end{doublespace}}

\switchcolumn


\begin{flushleft}
\begingl
 78.[] Yo[who-\NOM{\GMU{nom-sg}}] pana[(and)-\PART{\GMU{part}}] bhikkhu[bhikkhu-\NOM{\GMU{nom-sg}}] bhikkhūnaṁ[bhikkhu-\DAT{\GMU{dat-pl}}] bhaṇḍanajātānaṁ[argue.become-\ADJ{\GMU{adj}}] kalahajātānaṁ[quarrel.become-\ADJ{\GMU{adj}}] vivādāpannānaṁ[dispute.engage-\ADJ{\GMU{adj}}] upassutiṁ[over.hear-\ACC{\GMU{acc-sg-f}}] tiṭṭheyya,[persist-\OPT{\GMU{3-sg-opt}}] “Yaṁ[that-\ACC{\GMU{acc-sg}}] ime[this-\NOM{\GMU{nom-pl}}] bhaṇissanti[say-\FUT{\GMU{3-pl-fut}}] taṁ[that-\ACC{\GMU{acc-sg}}] sossāmī”[hear-\FUT{\GMU{1-sg-fut}}] ti.[-\NUL{\GMU{}}] Etad’eva[this.just-\ACC{\GMU{acc-sg-n}}] paccayaṁ[reason-\ACC{\GMU{acc-sg}}] karitvā[done-\ABS{\GMU{abs}}] anaññaṁ,[not.another-\ADJ{\GMU{adj}}] pācittiyaṁ.[confess-\ADJ{\GMU{adj}}]
\endgl
\switchcolumn*
\end{flushleft}


{\EnglishColumn

\begin{doublespace}
79. If any bhikkhu, having given consent to legitimate (legal) actions, should afterwards engage in the act of criticising, (this is a case) involving expiation.
\end{doublespace}}

\switchcolumn


\begin{flushleft}
\begingl
 79.[] Yo[who-\NOM{\GMU{nom-sg}}] pana[(and)-\PART{\GMU{part}}] bhikkhu[bhikkhu-\NOM{\GMU{nom-sg}}] dhammikānaṁ[law-\ADJ{\GMU{adj}}] kammānaṁ[action-\DAT{\GMU{dat-pl-n}}] chandaṁ[consent-\ACC{\GMU{acc-sg}}] datvā,[give-\ABS{\GMU{abs}}] pacchā[after-\IND{\GMU{ind}}] khiyyanadhammaṁ[criticize.act-\ACC{\GMU{acc-sg}}] āpajjeyya,[engage-\OPT{\GMU{3-sg-opt}}] pācittiyaṁ.[confess-\ADJ{\GMU{adj}}]
\endgl
\switchcolumn*
\end{flushleft}


{\EnglishColumn

\begin{doublespace}
80. If any bhikkhu, when investigatory discussion is going on in the community, not having given (his) consent, having gotten up from (his) seat, should depart, (this is a case) involving expiation.
\end{doublespace}}

\switchcolumn


\begin{flushleft}
\begingl
 80.[] Yo[who-\NOM{\GMU{nom-sg}}] pana[(and)-\PART{\GMU{part}}] bhikkhu[bhikkhu-\NOM{\GMU{nom-sg}}] saṅghe[community-\LOC{\GMU{loc-sg}}] vinicchayakathāya[deliberate-\LOC{\GMU{loc-sg}}] vattamānāya,[procede-\ADJ{\GMU{adj}}] chandaṁ[consent-\ACC{\GMU{acc-sg}}] adatvā[not.give-\ABS{\GMU{abs}}] uṭṭhāy‘āsanā[get up.seat-\ABL{\GMU{abl-sg-n}}] pakkameyya,[depart-\OPT{\GMU{3-sg-opt}}] pācittiyaṁ.[confess-\ADJ{\GMU{adj}}]
\endgl
\switchcolumn*
\end{flushleft}


{\EnglishColumn

\begin{doublespace}
81. If any bhikkhu, having given a robe (-cloth) (together) with a united community, should afterwards engage in criticising (saying): “The bhikkhus allocate communal gain according to familiarity,” (this is a case) involving expiation.
\end{doublespace}}

\switchcolumn


\begin{flushleft}
\begingl
 81.[] Yo[who-\NOM{\GMU{nom-sg}}] pana[(and)-\PART{\GMU{part}}] bhikkhu[bhikkhu-\NOM{\GMU{nom-sg}}] samaggena[united-\ADJ{\GMU{adj}}] saṅghena[community-\INS{\GMU{ins-sg}}] cīvaraṁ[robe-\ACC{\GMU{acc-sg-n}}] datvā,[give-\ABS{\GMU{abs}}] pacchā[after-\IND{\GMU{ind}}] khiyyanadhammaṁ[criticize.act-\ACC{\GMU{acc-sg}}] āpajjeyya,[engage-\OPT{\GMU{3-sg-opt}}] “Yathāsanthutaṁ[as familiar-\ADV{\GMU{adv}}] bhikkhū[bhikkhu-\NOM{\GMU{nom-pl}}] saṅghikaṁ[community.owned-\ADJ{\GMU{adj}}] lābhaṁ[gain-\ACC{\GMU{acc-sg}}] pariṇāmentī”[allocate-\PRESIND{\GMU{3-pl-presind}}] ti,[-\NUL{\GMU{}}] pācittiyaṁ.[confess-\ADJ{\GMU{adj}}]
\endgl
\switchcolumn*
\end{flushleft}


{\EnglishColumn

\begin{doublespace}
82. If any bhikkhu should knowingly allocate (already) allocated communal gain to a (lay-) person, (this is a case) involving expiation.
\end{doublespace}}

\switchcolumn


\begin{flushleft}
\begingl
 82.[] Yo[who-\NOM{\GMU{nom-sg}}] pana[(and)-\PART{\GMU{part}}] bhikkhu[bhikkhu-\NOM{\GMU{nom-sg}}] jānaṁ[know-\NOM{\GMU{nom-sg}}] saṅghikaṁ[community.owned-\ADJ{\GMU{adj}}] lābhaṁ[gain-\ACC{\GMU{acc-sg}}] pariṇataṁ[allocate-\PAST{\GMU{past-part}}] puggalassa[person-\DAT{\GMU{dat-sg}}] pariṇāmeyya,[allocate-\OPT{\GMU{3-sg-opt}}] pācittiyaṁ.[confess-\ADJ{\GMU{adj}}]
\endgl
\switchcolumn*
\end{flushleft}


{\EnglishColumn

\begin{doublespace}
The section (starting with the rule) about (being spoken to) righteously is eighth.
\end{doublespace}}

\switchcolumn


\begin{flushleft}
\begingl
 sahadhammikavaggo[] aṭṭhamo.[]
\endgl
\switchcolumn*
\end{flushleft}


{\EnglishColumn

\begin{doublespace}
83. If any bhikkhu, without having been announced beforehand, should go beyond the boundary post of a noble consecrated king's (bed-room) when the king has not departed, (and) the (queen-) jewel has not withdrawn, (this is a case) involving expiation.
\end{doublespace}}

\switchcolumn


\begin{flushleft}
\begingl
 83.[] Yo[who-\NOM{\GMU{nom-sg}}] pana[(and)-\PART{\GMU{part}}] bhikkhu[bhikkhu-\NOM{\GMU{nom-sg}}] rañño[king-\GEN{\GMU{gen-sg}}] khattiyassa[noble-\ADJ{\GMU{adj}}] muddhābhisittassa[head.annoint-\ADJ{\GMU{adj}}] anikkhantarājake[not.depart.king-\ADJ{\GMU{adj}}] aniggataratanake[not.gone.queen-\ADJ{\GMU{adj}}] pubbe[previous-\ADV{\GMU{adv}}] appaṭisaṁvidito[not.announce-\ADJ{\GMU{adj}}] indakhīlaṁ[indra post-\ACC{\GMU{acc-sg}}] atikkāmeyya,[beyond.go-\OPT{\GMU{3-sg-opt}}] pācittiyaṁ.[confess-\ADJ{\GMU{adj}}]
\endgl
\switchcolumn*
\end{flushleft}


{\EnglishColumn

\begin{doublespace}
84. If any bhikkhu should pick up, or should make (someone else) pick up, a treasure or what is considered a treasure, except within a monastery or within a dwelling, (this is a case) involving expiation. However, by a bhikkhu having picked up, or having had picked up, a treasure or what is considered a treasure within a monastery or within a dwelling, (it) is to be put aside (thinking): “He to whom it belongs will take it.” This is the proper procedure here.
\end{doublespace}}

\switchcolumn


\begin{flushleft}
\begingl
 84.[] Yo[who-\NOM{\GMU{nom-sg}}] pana[(and)-\PART{\GMU{part}}] bhikkhu[bhikkhu-\NOM{\GMU{nom-sg}}] ratanaṁ[valuable-\ACC{\GMU{acc-sg-n}}] vā[or-\IND{\GMU{ind}}] ratanasammataṁ[valuable.consider-\ACC{\GMU{acc-sg-n}}] vā[or-\IND{\GMU{ind}}] aññatra[unless-\ABL{\GMU{abl}}] ajjhārāmā[in.monastery-\ABL{\GMU{abl-sg}}] vā[or-\IND{\GMU{ind}}] ajjhāvasathā[in.dwelling-\ABL{\GMU{abl-sg}}] vā[or-\IND{\GMU{ind}}] uggaṇheyya[take-\OPT{\GMU{3-sg-opt}}] vā[or-\IND{\GMU{ind}}] uggaṇhāpeyya[other take-\OPT{\GMU{3-sg-opt}}] vā,[or-\IND{\GMU{ind}}] pācittiyaṁ.[confess-\ADJ{\GMU{adj}}] Ratanaṁ[valuable-\ACC{\GMU{acc-sg-n}}] vā[or-\IND{\GMU{ind}}] pana[(and)-\PART{\GMU{part}}] bhikkhunā[bhikkhu-\INS{\GMU{ins-sg}}] ratanasammataṁ[valuable.consider-\ACC{\GMU{acc-sg-n}}] vā,[or-\IND{\GMU{ind}}] ajjhārāme[in.monastery-\LOC{\GMU{loc-sg}}] vā[or-\IND{\GMU{ind}}] ajjhāvasathe[in.dwelling-\LOC{\GMU{loc-sg}}] vā[or-\IND{\GMU{ind}}] uggahetvā[take-\ABS{\GMU{abs}}] vā[or-\IND{\GMU{ind}}] uggaṇhāpetvā[make take-\ABS{\GMU{abs}}] vā[or-\IND{\GMU{ind}}] nikkhipitabbaṁ,[lay aside-\FUT{\GMU{fut-pass-part}}] “Yassa[for whoever-\PRO{\GMU{pro}}] bhavissati[to be-\FUT{\GMU{3-sg-fut}}] so[he-\NOM{\GMU{nom-sg}}] harissatī”[take-\FUT{\GMU{3-sg-fut}}] ti.[-\NUL{\GMU{}}] Ayaṁ[this-\NOM{\GMU{nom-sg}}] tattha[about that-\ADV{\GMU{adv}}] sāmīci.[proper procedure-\NOM{\GMU{nom-sg-f}}]
\endgl
\switchcolumn*
\end{flushleft}


{\EnglishColumn

\begin{doublespace}
85. If any bhikkhu, not having asked (permission of) a bhikkhu who is present, should enter a village at the wrong time, except with an appropriate urgent duty, (this is a case) involving expiation.
\end{doublespace}}

\switchcolumn


\begin{flushleft}
\begingl
 85.[] Yo[who-\NOM{\GMU{nom-sg}}] pana[(and)-\PART{\GMU{part}}] bhikkhu[bhikkhu-\NOM{\GMU{nom-sg}}] santaṁ[exist-\PRES{\GMU{pres-part}}] bhikkhuṁ[bhikkhu-\ACC{\GMU{acc-sg}}] anāpucchā[not.ask-\ABS{\GMU{abs}}] vikāle[wrong time-\LOC{\GMU{loc-sg}}] gāmaṁ[village-\ACC{\GMU{acc-sg}}] paviseyya,[enter-\OPT{\GMU{3-sg-opt}}] aññatra[unless-\ABL{\GMU{abl}}] tathārūpā[such kind-\ADJ{\GMU{adj}}] accāyikā[urgent-\ADJ{\GMU{adj}}] karaṇīyā,[done-\INS{\GMU{ins-sg}}] pācittiyaṁ.[confess-\ADJ{\GMU{adj}}]
\endgl
\switchcolumn*
\end{flushleft}


{\EnglishColumn

\begin{doublespace}
86. If any bhikkhu should have a needle-case made, which is made of bone, or made of ivory, or made of horn, (this is a case) involving expiation with breaking up (the needle-case).
\end{doublespace}}

\switchcolumn


\begin{flushleft}
\begingl
 86.[] Yo[who-\NOM{\GMU{nom-sg}}] pana[(and)-\PART{\GMU{part}}] bhikkhu[bhikkhu-\NOM{\GMU{nom-sg}}] aṭṭhimayaṁ[bone.made-\ADJ{\GMU{adj}}] vā[or-\IND{\GMU{ind}}] dantamayaṁ[tooth.made-\ADJ{\GMU{adj}}] vā[or-\IND{\GMU{ind}}] visāṇamayaṁ[horn.made-\ADJ{\GMU{adj}}] vā[or-\IND{\GMU{ind}}] sūcigharaṁ[needle case-\ACC{\GMU{acc-sg-n}}] kārāpeyya,[make-\OPT{\GMU{3-sg-opt}}] bhedanakaṁ[break-\ADJ{\GMU{adj}}] pācittiyaṁ.[confess-\ADJ{\GMU{adj}}]
\endgl
\switchcolumn*
\end{flushleft}


{\EnglishColumn

\begin{doublespace}
87. By a bhikkhu who is having a new bed or seat made, (a bed or seat) which has legs of eight finger-breadths is to be made, according to the Sugata-finger-breadth, except the lowermost (edge of the) frame. For one who lets it exceed (this measure), (this is a case) involving expiation with cutting (down the legs).
\end{doublespace}}

\switchcolumn


\begin{flushleft}
\begingl
 87.[] Navam’pana[new.-\ADJ{\GMU{adj}}] bhikkhunā[bhikkhu-\INS{\GMU{ins-sg}}] mañcaṁ[bed-\ACC{\GMU{acc-sg}}] vā[or-\IND{\GMU{ind}}] pīṭhaṁ[chair-\ACC{\GMU{acc-sg-n}}] vā[or-\IND{\GMU{ind}}] kārayamānena,[build-\PRES{\GMU{pres-part}}] aṭṭh’aṅgulapādakaṁ[8.finger.leg-\ADJ{\GMU{adj}}] kāretabbaṁ[make-\FUT{\GMU{fut-pass-part}}] sugat’aṅgulena,[well.gone.finger-\INS{\GMU{ins-sg}}] aññatra[unless-\ABL{\GMU{abl}}] heṭṭhimāya[lowest-\IND{\GMU{ind}}] aṭaniyā.[frame-\ABL{\GMU{abl-sg-f}}] Taṁ[that-\ACC{\GMU{acc-sg}}] atikkāmayato,[beyond.go-\DAT{\GMU{dat-pres-part}}] chedanakaṁ[cut-\ADJ{\GMU{adj}}] pācittiyaṁ.[confess-\ADJ{\GMU{adj}}]
\endgl
\switchcolumn*
\end{flushleft}


{\EnglishColumn

\begin{doublespace}
88. If any bhikkhu should have a bed or seat covered with cotton made, (this is a case) involving expiation with tearing off (the cotton).
\end{doublespace}}

\switchcolumn


\begin{flushleft}
\begingl
 88.[] Yo[who-\NOM{\GMU{nom-sg}}] pana[(and)-\PART{\GMU{part}}] bhikkhu[bhikkhu-\NOM{\GMU{nom-sg}}] mañcaṁ[bed-\ACC{\GMU{acc-sg}}] vā[or-\IND{\GMU{ind}}] pīṭhaṁ[chair-\ACC{\GMU{acc-sg-n}}] vā[or-\IND{\GMU{ind}}] tūlonaddhaṁ[cotton.cover-\NUL{\GMU{}}] kārāpeyya,[make-\OPT{\GMU{3-sg-opt}}] uddālanakaṁ[tear off-\ADJ{\GMU{adj}}] pācittiyaṁ.[confess-\ADJ{\GMU{adj}}]
\endgl
\switchcolumn*
\end{flushleft}


{\EnglishColumn

\begin{doublespace}
89. By a bhikkhu who is having a sitting-cloth made, (a sitting-cloth) which has the (proper) measure is to be made. This measure here is: two spans of the sugata-span in length, one and a half across, (and) the border is a span. For one who lets it exceed (the measure), (this is a case) involving expiation with cutting (off the cloth).
\end{doublespace}}

\switchcolumn


\begin{flushleft}
\begingl
 89.[] Nisīdanam[] pana[(and)-\PART{\GMU{part}}] bhikkhunā[bhikkhu-\INS{\GMU{ins-sg}}] kārayamānena[build-\PRES{\GMU{pres-part}}] pamāṇikaṁ[measure-\ADJ{\GMU{adj}}] kāretabbaṁ.[make-\FUT{\GMU{fut-pass-part}}] Tatr’idaṁ[here.this-\NUL{\GMU{}}] pamāṇaṁ:[measure-\NOM{\GMU{nom-sg}}] dīghaso[length-\ADV{\GMU{adv}}] dve[2-\NUM{\GMU{num}}] vidatthiyo[span-\ACC{\GMU{acc-pl-f}}] sugatavidatthiyā,[well.gone.span-\INS{\GMU{ins-sg-f}}] tiriyaṁ[width-\IND{\GMU{ind}}] diyaḍḍhaṁ,[1 ½-\NUM{\GMU{num}}] dasā[border-\NOM{\GMU{nom-sg-f}}] vidatthi.[span-\NOM{\GMU{nom-sg-f}}] Taṁ[that-\ACC{\GMU{acc-sg}}] atikkāmayato,[beyond.go-\DAT{\GMU{dat-pres-part}}] chedanakaṁ[cut-\ADJ{\GMU{adj}}] pācittiyaṁ.[confess-\ADJ{\GMU{adj}}]
\endgl
\switchcolumn*
\end{flushleft}


{\EnglishColumn

\begin{doublespace}
90. By a bhikkhu who is having an itch-covering (-cloth) made, (an itch-covering) which has the (proper) measure is to be made. This measure here is: four spans of the Sugata-span in length, two spans across. For one who lets it exceed (the measure), (this is a case) involving expiation with cutting off the cloth).
\end{doublespace}}

\switchcolumn


\begin{flushleft}
\begingl
 90.[] Kaṇḍupaṭicchādiṁ[itch cover-\ACC{\GMU{acc-sg-f}}] pana[(and)-\PART{\GMU{part}}] bhikkhunā[bhikkhu-\INS{\GMU{ins-sg}}] kārayamānena[build-\PRES{\GMU{pres-part}}] pamāṇikā[measure-\ADJ{\GMU{adj}}] kāretabbā.[make-\FUT{\GMU{fut-pass-part}}] Tatr’idaṁ[here.this-\NUL{\GMU{}}] pamāṇaṁ:[measure-\NOM{\GMU{nom-sg}}] dīghaso[length-\ADV{\GMU{adv}}] catasso[4-\ADJ{\GMU{adj}}] vidatthiyo[span-\ACC{\GMU{acc-pl-f}}] sugatavidatthiyā,[well.gone.span-\INS{\GMU{ins-sg-f}}] tiriyaṁ[width-\IND{\GMU{ind}}] dve[2-\NUM{\GMU{num}}] vidatthiyo.[span-\ACC{\GMU{acc-pl-f}}] Taṁ[that-\ACC{\GMU{acc-sg}}] atikkāmayato,[beyond.go-\DAT{\GMU{dat-pres-part}}] chedanakaṁ[cut-\ADJ{\GMU{adj}}] pācittiyaṁ.[confess-\ADJ{\GMU{adj}}]
\endgl
\switchcolumn*
\end{flushleft}


{\EnglishColumn

\begin{doublespace}
91. By a bhikkhu who is having a rain's bathing-cloth made, (a bathing-cloth) which has the (proper) measure is to be made. This measure here is: six spans of the sugata-span in length, two and a half across. For one who lets it exceed (the measure), (this is a case) involving expiation with cutting (off the cloth).
\end{doublespace}}

\switchcolumn


\begin{flushleft}
\begingl
 91.[] Vassikasāṭikaṁ[rain.cloth-\ACC{\GMU{acc-sg}}] pana[(and)-\PART{\GMU{part}}] bhikkhunā[bhikkhu-\INS{\GMU{ins-sg}}] kārayamānena[build-\PRES{\GMU{pres-part}}] pamāṇikā[measure-\ADJ{\GMU{adj}}] kāretabbā.[make-\FUT{\GMU{fut-pass-part}}] Tatr’idaṁ[here.this-\NUL{\GMU{}}] pamāṇaṁ:[measure-\NOM{\GMU{nom-sg}}] dīghaso[length-\ADV{\GMU{adv}}] cha[6-\NUM{\GMU{num}}] vidatthiyo[span-\ACC{\GMU{acc-pl-f}}] sugatavidatthiyā[well.gone.span-\INS{\GMU{ins-sg-f}}] tiriyaṁ[width-\IND{\GMU{ind}}] aḍḍhateyyā.[2 $\sfrac{1}{2}$-\NUM{\GMU{num}}] Taṁ[that-\ACC{\GMU{acc-sg}}] atikkāmayato,[beyond.go-\DAT{\GMU{dat-pres-part}}] chedanakaṁ[cut-\ADJ{\GMU{adj}}] pācittiyaṁ.[confess-\ADJ{\GMU{adj}}]
\endgl
\switchcolumn*
\end{flushleft}


{\EnglishColumn

\begin{doublespace}
92. If any bhikkhu should have a robe made which has the sugata-robe measure or (one) which is more (than that), (this is a case) involving expiation with cutting (off the robe). This is the Sugata's sugata-robe measure here: nine spans of the sugata-span in length, six spans across. This is the Sugata's sugata-robe measure.
\end{doublespace}}

\switchcolumn


\begin{flushleft}
\begingl
 92.[] Yo[who-\NOM{\GMU{nom-sg}}] pana[(and)-\PART{\GMU{part}}] bhikkhu[bhikkhu-\NOM{\GMU{nom-sg}}] sugatacīvarappamāṇaṁ[well.gone.robe.measure-\ADJ{\GMU{adj}}] cīvaraṁ[robe-\ACC{\GMU{acc-sg-n}}] kārāpeyya[make-\OPT{\GMU{3-sg-opt}}] atirekaṁ[more-\ADJ{\GMU{adj}}] vā,[or-\IND{\GMU{ind}}] chedanakaṁ[cut-\ADJ{\GMU{adj}}] pācittiyaṁ.[confess-\ADJ{\GMU{adj}}] Tatr’idaṁ[here.this-\NUL{\GMU{}}] sugatassa[well.gone-\GEN{\GMU{gen-sg}}] sugatacīvarappamāṇaṁ:[well.gone.robe.measure-\ADJ{\GMU{adj}}] dīghaso[length-\ADV{\GMU{adv}}] nava[9-\NUM{\GMU{num}}] vidatthiyo[span-\ACC{\GMU{acc-pl-f}}] sugatavidatthiyā,[well.gone.span-\INS{\GMU{ins-sg-f}}] tiriyaṁ[width-\IND{\GMU{ind}}] cha[6-\NUM{\GMU{num}}] vidatthiyo.[span-\ACC{\GMU{acc-pl-f}}] Idaṁ[this-\ACC{\GMU{acc-sg-n}}] sugatassa[well.gone-\GEN{\GMU{gen-sg}}] sugatacīvarappamāṇaṁ.[well.gone.robe.measure-\ADJ{\GMU{adj}}]
\endgl
\switchcolumn*
\end{flushleft}


{\EnglishColumn

\begin{doublespace}
The section (starting with the rule) on kings is ninth.
\end{doublespace}}

\switchcolumn


\begin{flushleft}
\begingl
 ratanavaggo[] navamo.[]
\endgl
\switchcolumn*
\end{flushleft}


{\EnglishColumn

\begin{doublespace}
Venerables, the ninety-two cases involving expiation have been recited.\\
Concerning that I ask the Venerables: (Are you) pure in this?\\
A second time again I ask: (Are you) pure in this?\\
A third time again I ask: (Are you) pure in this?\\
The venerables are pure in this, therefore there is silence, thus I keep this (in mind).
\end{doublespace}}

\switchcolumn


\begin{flushleft}
\begingl
 Uddiṭṭhā[recite-\PAST{\GMU{past-part}}] kho[indeed!-\EMPH{\GMU{emph}}] āyasmanto[Ven.-\VOC{\GMU{voc-pl}}] dvenavuti[] pācittiyā[] dhammā.[rule-\NOM{\GMU{nom-pl}}]+ Tatth’āyasmante[] pucchāmi:[ask-\PRESIND{\GMU{1-sg-presind}}] kacci’ttha[] parisuddhā?[pure-\ADJ{\GMU{adj}}]+ Dutiyam’pi[second time-\ACC{\GMU{acc-sg-nt}}] pucchāmi:[ask-\PRESIND{\GMU{1-sg-presind}}] kacci’ttha[] parisuddhā?[pure-\ADJ{\GMU{adj}}]+ Tatiyam’pi[] pucchāmi:[ask-\PRESIND{\GMU{1-sg-presind}}] kacci’ttha[] parisuddhā?[pure-\ADJ{\GMU{adj}}]+ Parisuddh’etth’āyasmanto,[] tasmā[therefore-\ABL{\GMU{abl-sg}}] tuṇhī,[silent-\ADV{\GMU{adv}}] evam’etaṁ[thus.this-\ACC{\GMU{acc-sg-n}}] dhārayāmi.[keep in mind-\PRESIND{\GMU{1-sg-presind}}]
\endgl
\switchcolumn*
\end{flushleft}


{\EnglishColumn

\begin{doublespace}
The (cases) involving expiation are finished.
\end{doublespace}}

\switchcolumn


\begin{flushleft}
\begingl
 pācittiyā[] niṭṭhitā[]
\endgl
\switchcolumn*
\end{flushleft}


{\EnglishColumn

\begin{doublespace}
Venerables, these four cases that are to be acknowledged come up for recitation.
\end{doublespace}}

\switchcolumn


\begin{flushleft}
\begingl
 Ime[this-\NOM{\GMU{nom-pl}}] kho[indeed!-\EMPH{\GMU{emph}}] pan’āyasmanto[venerable-\VOC{\GMU{voc-pl}}] cattāro[4-\NUM{\GMU{num}}] pāṭidesanīyā[] dhammā[rule-\NOM{\GMU{nom-pl}}] uddesaṁ[recitation-\ACC{\GMU{acc-sg}}] āgacchanti.[come up-\PRESIND{\GMU{3-pl-presind}}]
\endgl
\switchcolumn*
\end{flushleft}


{\EnglishColumn

\begin{doublespace}
1. If any bhikkhu, having accepted (it) with his own hand from the hand of an unrelated bhikkhunì who has entered an inhabited area (for alms), should chew uncooked food or eat cooked food), (it) is to be acknowledged by that bhikkhu (saying): “Friend(s), I have committed a blameworthy act which is unsuitable, which is to be acknowledged; I acknowledge it.”
\end{doublespace}}

\switchcolumn


\begin{flushleft}
\begingl
 1.[] Yo[who-\NOM{\GMU{nom-sg}}] pana[(and)-\PART{\GMU{part}}] bhikkhu[bhikkhu-\NOM{\GMU{nom-sg}}] aññātikāya[unrelated-\ADJ{\GMU{adj}}] bhikkhuniyā[bhikkhuni-\INS{\GMU{ins-sg-f}}] antaragharaṁ[inhabited area-\ACC{\GMU{acc-sg-n}}] paviṭṭhāya[enter-\ADJ{\GMU{adj}}] hatthato,[hand-\ABL{\GMU{abl-sg}}] khādanīyaṁ[uncooked food-\ACC{\GMU{acc-sg-n}}] vā[or-\IND{\GMU{ind}}] bhojanīyaṁ[cooked food-\ACC{\GMU{acc-sg}}] vā[or-\IND{\GMU{ind}}] sahatthā[with.hand-\INS{\GMU{ins-sg}}] paṭiggahetvā[accept-\ABS{\GMU{abs}}] khādeyya[chew-\OPT{\GMU{3-sg-opt}}] vā[or-\IND{\GMU{ind}}] bhuñjeyya[eat-\OPT{\GMU{3-sg-opt}}] vā,[or-\IND{\GMU{ind}}] paṭidesetabbaṁ[acknowledge-\FUT{\GMU{fut-pass-part}}] tena[him-\INS{\GMU{3-sg-ins}}] bhikkhunā,[bhikkhu-\INS{\GMU{ins-sg}}] “Gārayhaṁ[blame-\FUT{\GMU{fut-pass-part}}] āvuso[friend-\VOC{\GMU{voc-sg}}] dhammaṁ[act-\ACC{\GMU{acc-sg}}] āpajjiṁ[commit-\AOR{\GMU{1-sg-aor}}] asappāyaṁ[unsuitable-\ADJ{\GMU{adj}}] pāṭidesanīyaṁ,[acknowledge-\ADJ{\GMU{adj}}] taṁ[that-\ACC{\GMU{acc-sg}}] paṭidesemī”[acknowledge-\PRESIND{\GMU{1-sg-presind}}] ti.[-\NUL{\GMU{}}]
\endgl
\switchcolumn*
\end{flushleft}


{\EnglishColumn

\begin{doublespace}
2. Now, bhikkhus who have been invited are eating among families, and if a bhikkhunì who is giving directions is standing there (saying), “Give curry here, give rice here!” (then) by those bhikkhus that bhikkhunì is to be dismissed (saying), “Go away, sister, for as long as the bhikkhus eat!,” and if not even one bhikkhu would speak against (it, so as) to dismiss that bhikkhunì (saying), “Go away, sister, for as long as the bhikkhus eat!,” (then it) is to be acknowledged by those bhikkhus, “Friend(s), we have committed a blameworthy act which is unsuitable, which is to be acknowledged; we acknowledge it.”
\end{doublespace}}

\switchcolumn


\begin{flushleft}
\begingl
 2.[] Bhikkhū[bhikkhu-\NOM{\GMU{nom-pl}}] pan’eva[now.if-\PART{\GMU{part}}] kulesu[family-\LOC{\GMU{loc-pl-n}}] nimantitā[invite-\ADJ{\GMU{adj}}] bhuñjanti.[eat-\PRESIND{\GMU{3-pl-presind}}] Tatra[then-\ADV{\GMU{adv}}] ce[if-\NUL{\GMU{}}] bhikkhunī[bhikkhuni-\NOM{\GMU{nom-sg-f}}] vosāsamānarūpā[give.direction-\ADJ{\GMU{adj}}] ṭhitā[stand-\NUL{\GMU{}}] hoti,[he is-\PRESIND{\GMU{3-sg-presind}}] “Idha[here-\ADV{\GMU{adv}}] sūpaṁ[-\NUL{\GMU{}}] detha,[give-\IMP{\GMU{2-pl-imp}}] idha[here-\ADV{\GMU{adv}}] odanaṁ[rice-\ACC{\GMU{acc-sg}}] dethā”[give-\IMP{\GMU{2-pl-imp}}] ti.[-\NUL{\GMU{}}] Tehi[those-\INS{\GMU{ins-pl}}] bhikkhūhi[bhikkhu-\INS{\GMU{ins-pl}}] sā[that-\NOM{\GMU{nom-f}}] bhikkhunī[bhikkhuni-\NOM{\GMU{nom-sg-f}}] apasādetabbā,[dismiss-\FUT{\GMU{fut-pass-part}}] “Apasakka[leave-\IMP{\GMU{2-sg-imp}}] tāva[so.long-\ADV{\GMU{adv}}] bhagini,[sister-\VOC{\GMU{voc-sg-f}}] yāva[until-\IND{\GMU{ind}}] bhikkhū[bhikkhu-\NOM{\GMU{nom-pl}}] bhuñjantī”[eat-\PRESIND{\GMU{3-pl-presind}}] ti.[-\NUL{\GMU{}}] Ekassa’pi[one-\DAT{\GMU{dat-sg}}] ce[if-\NUL{\GMU{}}] bhikkhuno[bhikkhu-\DAT{\GMU{dat-sg}}] nappaṭibhāseyya[-\NUL{\GMU{}}] taṁ[that-\ACC{\GMU{acc-sg}}] bhikkhuniṁ[bhikkhuni-\ACC{\GMU{acc-sg-f}}] apasādetuṁ,[dismiss-\INF{\GMU{inf}}] “Apasakka[leave-\IMP{\GMU{2-sg-imp}}] tāva[so.long-\ADV{\GMU{adv}}] bhagini,[sister-\VOC{\GMU{voc-sg-f}}] yāva[until-\IND{\GMU{ind}}] bhikkhū[bhikkhu-\NOM{\GMU{nom-pl}}] bhuñjantī”[eat-\PRESIND{\GMU{3-pl-presind}}] ti,[-\NUL{\GMU{}}] paṭidesetabbaṁ[acknowledge-\FUT{\GMU{fut-pass-part}}] tehi[those-\INS{\GMU{ins-pl}}] bhikkhūhi,[bhikkhu-\INS{\GMU{ins-pl}}] "Gārayhaṁ[blame-\FUT{\GMU{fut-pass-part}}] āvuso[friend-\VOC{\GMU{voc-sg}}] dhammaṁ[act-\ACC{\GMU{acc-sg}}] āpajjimhā[commit-\AOR{\GMU{1-pl-aor}}] asappāyaṁ[unsuitable-\ADJ{\GMU{adj}}] pāṭidesanīyaṁ,[acknowledge-\ADJ{\GMU{adj}}] taṁ[that-\ACC{\GMU{acc-sg}}] paṭidesemā”[acknowledge-\PRESIND{\GMU{1-pl-presind}}] ti.[-\NUL{\GMU{}}]
\endgl
\switchcolumn*
\end{flushleft}


{\EnglishColumn

\begin{doublespace}
3. Now, (there are) those families which are agreed upon as trainees: if any bhikkhu who has not been invited beforehand, who is not ill, should chew uncooked food or eat cooked food having accepted (it) with his own hand in families who are of such a kind, who are considered trainees, (then it) is to be acknowledged by that bhikkhu: “Friend(s), I have committed a blameworthy act which is unsuitable, which is to be acknowledged; I acknowledge it.”
\end{doublespace}}

\switchcolumn


\begin{flushleft}
\begingl
 3.[] Yāni[which-\NOM{\GMU{nom-pl-n}}] kho[indeed!-\EMPH{\GMU{emph}}] pana[(and)-\PART{\GMU{part}}] tāni[those-\NOM{\GMU{nom-pl}}] sekkhasammatāni[trainee.agreed-\ADJ{\GMU{adj}}] kulāni.[family-\NOM{\GMU{nom}}] Yo[who-\NOM{\GMU{nom-sg}}] pana[(and)-\PART{\GMU{part}}] bhikkhu[bhikkhu-\NOM{\GMU{nom-sg}}] tathārūpesu[such kind-\ADJ{\GMU{adj}}] sekkhasammatesu[trainee.agreed-\LOC{\GMU{loc-pl-nt}}] kulesu[family-\LOC{\GMU{loc-pl-n}}] pubbe[previous-\ADV{\GMU{adv}}] animantito[not.invite-\ADJ{\GMU{adj}}] agilāno[not.sick-\ADJ{\GMU{adj}}] khādanīyaṁ[uncooked food-\ACC{\GMU{acc-sg-n}}] vā[or-\IND{\GMU{ind}}] bhojanīyaṁ[cooked food-\ACC{\GMU{acc-sg}}] vā[or-\IND{\GMU{ind}}] sahatthā[with.hand-\INS{\GMU{ins-sg}}] paṭiggahetvā[accept-\ABS{\GMU{abs}}] khādeyya[chew-\OPT{\GMU{3-sg-opt}}] vā[or-\IND{\GMU{ind}}] bhuñjeyya[eat-\OPT{\GMU{3-sg-opt}}] vā,[or-\IND{\GMU{ind}}] paṭidesetabbaṁ[acknowledge-\FUT{\GMU{fut-pass-part}}] tena[him-\INS{\GMU{3-sg-ins}}] bhikkhunā,[bhikkhu-\INS{\GMU{ins-sg}}] “Gārayhaṁ[blame-\FUT{\GMU{fut-pass-part}}] āvuso[friend-\VOC{\GMU{voc-sg}}] dhammaṁ[act-\ACC{\GMU{acc-sg}}] āpajjiṁ[commit-\AOR{\GMU{1-sg-aor}}] asappāyaṁ[unsuitable-\ADJ{\GMU{adj}}] pāṭidesanīyaṁ,[acknowledge-\ADJ{\GMU{adj}}] taṁ[that-\ACC{\GMU{acc-sg}}] paṭidesemī”[acknowledge-\PRESIND{\GMU{1-sg-presind}}] ti.[-\NUL{\GMU{}}]
\endgl
\switchcolumn*
\end{flushleft}


{\EnglishColumn

\begin{doublespace}
4. Now, (there are) those those wilderness lodgings which are considered risky, which are dangerous: if any bhikkhu, (staying) in lodgings which are of such a kind, without having announced (the danger) beforehand, having accepted (the food) with his own hand inside the monastery, (and then) not being ill, should chew uncooked food or eat cooked food, (then it) is to be acknowledged by that bhikkhu, “Friend(s), I have committed a blameworthy act which is unsuitable, which is to be acknowledged; I acknowledge it.”
\end{doublespace}}

\switchcolumn


\begin{flushleft}
\begingl
 4.[] Yāni[which-\NOM{\GMU{nom-pl-n}}] kho[indeed!-\EMPH{\GMU{emph}}] pana[(and)-\PART{\GMU{part}}] tāni[those-\NOM{\GMU{nom-pl}}] āraññakāni[wilderness-\ADJ{\GMU{adj}}] senāsanāni[lodging-\NOM{\GMU{nom-pl-n}}] sāsaṅkasammatāni[risky.recond-\ADJ{\GMU{adj}}] sappaṭibhayāni.[frighten-\ADJ{\GMU{adj}}] Yo[who-\NOM{\GMU{nom-sg}}] pana[(and)-\PART{\GMU{part}}] bhikkhu[bhikkhu-\NOM{\GMU{nom-sg}}] tathārūpesu[such kind-\ADJ{\GMU{adj}}] senāsanesu[lodging-\LOC{\GMU{loc-pl-n}}] viharanto,[dwell-\ADJ{\GMU{adj}}] pubbe[previous-\ADV{\GMU{adv}}] appaṭisaṁviditaṁ[not.announce-\ADJ{\GMU{adj}}] khādanīyaṁ[uncooked food-\ACC{\GMU{acc-sg-n}}] vā[or-\IND{\GMU{ind}}] bhojanīyaṁ[cooked food-\ACC{\GMU{acc-sg}}] vā[or-\IND{\GMU{ind}}] ajjhārāme[in.monastery-\LOC{\GMU{loc-sg}}] sahatthā[with.hand-\INS{\GMU{ins-sg}}] paṭiggahetvā[accept-\ABS{\GMU{abs}}] agilāno[not.sick-\ADJ{\GMU{adj}}] khādeyya[chew-\OPT{\GMU{3-sg-opt}}] vā[or-\IND{\GMU{ind}}] bhuñjeyya[eat-\OPT{\GMU{3-sg-opt}}] vā,[or-\IND{\GMU{ind}}] paṭidesetabbaṁ[acknowledge-\FUT{\GMU{fut-pass-part}}] tena[him-\INS{\GMU{3-sg-ins}}] bhikkhunā,[bhikkhu-\INS{\GMU{ins-sg}}] “Gārayhaṁ[blame-\FUT{\GMU{fut-pass-part}}] āvuso[friend-\VOC{\GMU{voc-sg}}] dhammaṁ[act-\ACC{\GMU{acc-sg}}] āpajjiṁ[commit-\AOR{\GMU{1-sg-aor}}] asappāyaṁ[unsuitable-\ADJ{\GMU{adj}}] pāṭidesanīyaṁ,[acknowledge-\ADJ{\GMU{adj}}] taṁ[that-\ACC{\GMU{acc-sg}}] paṭidesemī”[acknowledge-\PRESIND{\GMU{1-sg-presind}}] ti.[-\NUL{\GMU{}}]
\endgl
\switchcolumn*
\end{flushleft}


{\EnglishColumn

\begin{doublespace}
Venerables, the four cases that are to be acknowledged have been recited.\\
Concerning that I ask the Venerables: (Are you) pure in this?\\
A second time again I ask: (Are you) pure in this?\\
A third time again I ask: (Are you) pure in this?\\
The venerables are pure in this, therefore there is silence, thus I bear this (in mind).
\end{doublespace}}

\switchcolumn


\begin{flushleft}
\begingl
 Uddiṭṭhā[recite-\PAST{\GMU{past-part}}] kho[indeed!-\EMPH{\GMU{emph}}] āyasmanto[Ven.-\VOC{\GMU{voc-pl}}] cattāro[4-\NUM{\GMU{num}}] pāṭidesanīyā[] dhammā.[rule-\NOM{\GMU{nom-pl}}]+ Tatth’āyasmante[] pucchāmi:[ask-\PRESIND{\GMU{1-sg-presind}}] Kacci’ttha[] parisuddhā?[pure-\ADJ{\GMU{adj}}]+ Dutiyam’pi[second time-\ACC{\GMU{acc-sg-nt}}] pucchāmi:[ask-\PRESIND{\GMU{1-sg-presind}}] Kacci’ttha[] parisuddhā?[pure-\ADJ{\GMU{adj}}]+ Tatiyam’pi[] pucchāmi:[ask-\PRESIND{\GMU{1-sg-presind}}] Kacci’ttha[] parisuddhā?[pure-\ADJ{\GMU{adj}}]+ Parisuddh’etth’āyasmanto,[] tasmā[therefore-\ABL{\GMU{abl-sg}}] tuṇhī,[silent-\ADV{\GMU{adv}}] evam’etaṁ[thus.this-\ACC{\GMU{acc-sg-n}}] dhārayāmi.[keep in mind-\PRESIND{\GMU{1-sg-presind}}]
\endgl
\switchcolumn*
\end{flushleft}


{\EnglishColumn

\begin{doublespace}
The (cases) which are to be acknowledged have finished.
\end{doublespace}}

\switchcolumn


\begin{flushleft}
\begingl
 Pāṭidesanīyā[] niṭṭhitā[]
\endgl
\switchcolumn*
\end{flushleft}


{\EnglishColumn

\begin{doublespace}
Venerables, these cases related to the training come up for recitation.
\end{doublespace}}

\switchcolumn


\begin{flushleft}
\begingl
 Ime[this-\NOM{\GMU{nom-pl}}] kho[indeed!-\EMPH{\GMU{emph}}] pan’āyasmanto[venerable-\VOC{\GMU{voc-pl}}] sekhiyā[] dhammā[rule-\NOM{\GMU{nom-pl}}] uddesaṁ[recitation-\ACC{\GMU{acc-sg}}] āgacchanti.[come up-\PRESIND{\GMU{3-pl-presind}}]
\endgl
\switchcolumn*
\end{flushleft}


{\EnglishColumn

\begin{doublespace}
1. I shall wear (the under-robe) even all around,” thus the training is to be done.\\
2. I shall wrap (the outer-robes) even all around,” thus the training is to be done.
\end{doublespace}}

\switchcolumn


\begin{flushleft}
\begingl
 1.[] “Parimaṇḍalaṁ[around.circle-\ADJ{\GMU{adj}}] nivāsessāmī”[dress-\FUT{\GMU{1-sg-fut}}] ti[-\NUL{\GMU{}}] sikkhā[train-\NOM{\GMU{nom-sg-f}}] karaṇīyā.[done-\INS{\GMU{ins-sg}}]+ 2.[] “Parimaṇḍalaṁ[around.circle-\ADJ{\GMU{adj}}] pārupissāmī”[dress-\FUT{\GMU{1-sg-fut}}] ti[-\NUL{\GMU{}}] sikkhā[train-\NOM{\GMU{nom-sg-f}}] karaṇīyā.[done-\INS{\GMU{ins-sg}}]
\endgl
\switchcolumn*
\end{flushleft}


{\EnglishColumn

\begin{doublespace}
3. I shall go well covered inside an inhabited area,” thus the training is to be done.\\
4. I shall sit well covered inside an inhabited area,” thus the training is to be done.
\end{doublespace}}

\switchcolumn


\begin{flushleft}
\begingl
 3.[] “Supaṭicchanno[well cover-\ADJ{\GMU{adj}}] antaraghare[inside house-\LOC{\GMU{loc-sg-n}}] gamissāmī”[go-\FUT{\GMU{1-sg-fut}}] ti[-\NUL{\GMU{}}] sikkhā[train-\NOM{\GMU{nom-sg-f}}] karaṇīyā.[done-\INS{\GMU{ins-sg}}]+ 4.[] “Supaṭicchanno[well cover-\ADJ{\GMU{adj}}] antaraghare[inside house-\LOC{\GMU{loc-sg-n}}] nisīdissāmī”[sit-\FUT{\GMU{1-sg-fut}}] ti[-\NUL{\GMU{}}] sikkhā[train-\NOM{\GMU{nom-sg-f}}] karaṇīyā.[done-\INS{\GMU{ins-sg}}]
\endgl
\switchcolumn*
\end{flushleft}


{\EnglishColumn

\begin{doublespace}
5. I shall go well-restrained inside an inhabited area,” thus the training is to be done.\\
6. I shall sit well-restrained inside an inhabited area,” thus the training is to be done.
\end{doublespace}}

\switchcolumn


\begin{flushleft}
\begingl
 5.[] “Susaṁvuto[well.restrain-\PAST{\GMU{past-part}}] antaraghare[inside house-\LOC{\GMU{loc-sg-n}}] gamissāmī”[go-\FUT{\GMU{1-sg-fut}}] ti[-\NUL{\GMU{}}] sikkhā[train-\NOM{\GMU{nom-sg-f}}] karaṇīyā[done-\INS{\GMU{ins-sg}}]+ 6.[] “Susaṁvuto[well.restrain-\PAST{\GMU{past-part}}] antaraghare[inside house-\LOC{\GMU{loc-sg-n}}] nisīdissāmī”[sit-\FUT{\GMU{1-sg-fut}}] ti[-\NUL{\GMU{}}] sikkhā[train-\NOM{\GMU{nom-sg-f}}] karaṇīyā.[done-\INS{\GMU{ins-sg}}]
\endgl
\switchcolumn*
\end{flushleft}


{\EnglishColumn

\begin{doublespace}
7. I shall go with the eyes cast down inside an inhabited area,” thus the training is to be done.\\
8. I shall sit with the eyes cast down inside an inhabited area,” thus the training is to be done.
\end{doublespace}}

\switchcolumn


\begin{flushleft}
\begingl
 7.[] “Okkhittacakkhu[cast down.eyes-\ADJ{\GMU{adj}}] antaraghare[inside house-\LOC{\GMU{loc-sg-n}}] gamissāmī”[go-\FUT{\GMU{1-sg-fut}}] ti[-\NUL{\GMU{}}] sikkhā[train-\NOM{\GMU{nom-sg-f}}] karaṇīyā[done-\INS{\GMU{ins-sg}}]+ 8.[] “Okkhittacakkhu[cast down.eyes-\ADJ{\GMU{adj}}] antaraghare[inside house-\LOC{\GMU{loc-sg-n}}] nisīdissāmī”[sit-\FUT{\GMU{1-sg-fut}}] ti[-\NUL{\GMU{}}] sikkhā[train-\NOM{\GMU{nom-sg-f}}] karaṇīyā[done-\INS{\GMU{ins-sg}}]
\endgl
\switchcolumn*
\end{flushleft}


{\EnglishColumn

\begin{doublespace}
9. I shall not go with (robes) lifted up inside an inhabited area,” thus the training is to be done.\\
10. I shall not sit with (robes) lifted up inside an inhabited area,” thus the training is to be done.
\end{doublespace}}

\switchcolumn


\begin{flushleft}
\begingl
 9.[] “Na[not-\PART{\GMU{part}}] ukkhittakāya[lift up-\ADJ{\GMU{adj}}] antaraghare[inside house-\LOC{\GMU{loc-sg-n}}] gamissāmī”[go-\FUT{\GMU{1-sg-fut}}] ti[-\NUL{\GMU{}}] sikkhā[train-\NOM{\GMU{nom-sg-f}}] karaṇīyā.[done-\INS{\GMU{ins-sg}}]+ 10.[] “Na[not-\PART{\GMU{part}}] ukkhittakāya[lift up-\ADJ{\GMU{adj}}] antaraghare[inside house-\LOC{\GMU{loc-sg-n}}] nisīdissāmī”[sit-\FUT{\GMU{1-sg-fut}}] ti[-\NUL{\GMU{}}] sikkhā[train-\NOM{\GMU{nom-sg-f}}] karaṇīyā.[done-\INS{\GMU{ins-sg}}]
\endgl
\switchcolumn*
\end{flushleft}


{\EnglishColumn

\begin{doublespace}
11. I shall not go with loud laughter inside an inhabited area,” thus the training is to be done.\\
12. I shall not sit with loud laughter inside an inhabited area,” thus the training is to be done.
\end{doublespace}}

\switchcolumn


\begin{flushleft}
\begingl
 11.[] “Na[not-\PART{\GMU{part}}] ujjagghikāya[loud laugh-\INS{\GMU{ins-sg-f}}] antaraghare[inside house-\LOC{\GMU{loc-sg-n}}] gamissāmī”[go-\FUT{\GMU{1-sg-fut}}] ti[-\NUL{\GMU{}}] sikkhā[train-\NOM{\GMU{nom-sg-f}}] karaṇīyā.[done-\INS{\GMU{ins-sg}}]+ 12.[] “Na[not-\PART{\GMU{part}}] ujjagghikāya[loud laugh-\INS{\GMU{ins-sg-f}}] antaraghare[inside house-\LOC{\GMU{loc-sg-n}}] nisīdissāmī”[sit-\FUT{\GMU{1-sg-fut}}] ti[-\NUL{\GMU{}}] sikkhā[train-\NOM{\GMU{nom-sg-f}}] karaṇīyā.[done-\INS{\GMU{ins-sg}}]
\endgl
\switchcolumn*
\end{flushleft}


{\EnglishColumn

\begin{doublespace}
13. I shall go quiet(ly) inside an inhabited area,” thus the training is to be done.\\
14. I shall sit quiet(ly) inside an inhabited area,” thus the training is to be done.
\end{doublespace}}

\switchcolumn


\begin{flushleft}
\begingl
 13.[] “Appasaddo[quite-\ADJ{\GMU{adj}}] antaraghare[inside house-\LOC{\GMU{loc-sg-n}}] gamissāmī”[go-\FUT{\GMU{1-sg-fut}}] ti[-\NUL{\GMU{}}] sikkhā[train-\NOM{\GMU{nom-sg-f}}] karaṇīyā.[done-\INS{\GMU{ins-sg}}]+ 14.[] “Appasaddo[quite-\ADJ{\GMU{adj}}] antaraghare[inside house-\LOC{\GMU{loc-sg-n}}] nisīdissāmī”[sit-\FUT{\GMU{1-sg-fut}}] ti[-\NUL{\GMU{}}] sikkhā[train-\NOM{\GMU{nom-sg-f}}] karaṇīyā[done-\INS{\GMU{ins-sg}}]
\endgl
\switchcolumn*
\end{flushleft}


{\EnglishColumn

\begin{doublespace}
15. I shall not go swaying the body inside an inhabited area,” thus the training is to be done.\\
16. I shall not sit swaying the body inside an inhabited area,” thus the training is to be done.
\end{doublespace}}

\switchcolumn


\begin{flushleft}
\begingl
 15.[] “Na[not-\PART{\GMU{part}}] kāyappacālakaṁ[body.sway-\ADV{\GMU{adv}}] antaraghare[inside house-\LOC{\GMU{loc-sg-n}}] gamissāmī”[go-\FUT{\GMU{1-sg-fut}}] ti[-\NUL{\GMU{}}] sikkhā[train-\NOM{\GMU{nom-sg-f}}] karaṇīyā.[done-\INS{\GMU{ins-sg}}]+ 16.[] “Na[not-\PART{\GMU{part}}] kāyappacālakaṁ[body.sway-\ADV{\GMU{adv}}] antaraghare[inside house-\LOC{\GMU{loc-sg-n}}] nisīdissāmī”[sit-\FUT{\GMU{1-sg-fut}}] ti[-\NUL{\GMU{}}] sikkhā[train-\NOM{\GMU{nom-sg-f}}] karaṇīyā.[done-\INS{\GMU{ins-sg}}]
\endgl
\switchcolumn*
\end{flushleft}


{\EnglishColumn

\begin{doublespace}
17. I shall not go swaying the arms inside an inhabited area,” thus the training is to be done.\\
18. I shall not sit swaying the arms inside an inhabited area,” thus the training is to be done.
\end{doublespace}}

\switchcolumn


\begin{flushleft}
\begingl
 17.[] “Na[not-\PART{\GMU{part}}] bāhuppacālakaṁ[arm.sway-\ACC{\GMU{acc-sg-n}}] antaraghare[inside house-\LOC{\GMU{loc-sg-n}}] gamissāmī”[go-\FUT{\GMU{1-sg-fut}}] ti[-\NUL{\GMU{}}] sikkhā[train-\NOM{\GMU{nom-sg-f}}] karaṇīyā.[done-\INS{\GMU{ins-sg}}]+ 18.[] “Na[not-\PART{\GMU{part}}] bāhuppacālakaṁ[arm.sway-\ACC{\GMU{acc-sg-n}}] antaraghare[inside house-\LOC{\GMU{loc-sg-n}}] nisīdissāmī”[sit-\FUT{\GMU{1-sg-fut}}] ti[-\NUL{\GMU{}}] sikkhā[train-\NOM{\GMU{nom-sg-f}}] karaṇīyā.[done-\INS{\GMU{ins-sg}}]
\endgl
\switchcolumn*
\end{flushleft}


{\EnglishColumn

\begin{doublespace}
19. I shall not go swaying the head inside an inhabited area,” thus the training is to be done.\\
20. I shall not sit swaying the head inside an inhabited area,” thus the training is to be done.
\end{doublespace}}

\switchcolumn


\begin{flushleft}
\begingl
 19.[] “Na[not-\PART{\GMU{part}}] sīsappacālakaṁ[-\NUL{\GMU{}}] antaraghare[inside house-\LOC{\GMU{loc-sg-n}}] gamissāmī”[go-\FUT{\GMU{1-sg-fut}}] ti[-\NUL{\GMU{}}] sikkhā[train-\NOM{\GMU{nom-sg-f}}] karaṇīyā.[done-\INS{\GMU{ins-sg}}]+ 20.[] “Na[not-\PART{\GMU{part}}] sīsappacālakaṁ[-\NUL{\GMU{}}] antaraghare[inside house-\LOC{\GMU{loc-sg-n}}] nisīdissāmī”[sit-\FUT{\GMU{1-sg-fut}}] ti[-\NUL{\GMU{}}] sikkhā[train-\NOM{\GMU{nom-sg-f}}] karaṇīyā.[done-\INS{\GMU{ins-sg}}]
\endgl
\switchcolumn*
\end{flushleft}


{\EnglishColumn

\begin{doublespace}
21. I shall not go having made (the arms) a prop inside an inhabited area,” thus the training is to be done.\\
22. I shall not sit having made (the arms) a prop inside an inhabited area,” thus the training is to be done.
\end{doublespace}}

\switchcolumn


\begin{flushleft}
\begingl
 21.[] “Na[not-\PART{\GMU{part}}] khambhakato[-\NUL{\GMU{}}] antaraghare[inside house-\LOC{\GMU{loc-sg-n}}] gamissāmī”[go-\FUT{\GMU{1-sg-fut}}] ti[-\NUL{\GMU{}}] sikkhā[train-\NOM{\GMU{nom-sg-f}}] karaṇīyā.[done-\INS{\GMU{ins-sg}}]+ 22.[] “Na[not-\PART{\GMU{part}}] khambhakato[-\NUL{\GMU{}}] antaraghare[inside house-\LOC{\GMU{loc-sg-n}}] nisīdissāmī”[sit-\FUT{\GMU{1-sg-fut}}] ti[-\NUL{\GMU{}}] sikkhā[train-\NOM{\GMU{nom-sg-f}}] karaṇīyā.[done-\INS{\GMU{ins-sg}}]
\endgl
\switchcolumn*
\end{flushleft}


{\EnglishColumn

\begin{doublespace}
23. I shall not go with (the head) covered inside an inhabited area,” thus the training is to be done.\\
24. I shall not sit with (the head) covered inside an inhabited area,” thus the training is to be done.
\end{doublespace}}

\switchcolumn


\begin{flushleft}
\begingl
 23.[] “Na[not-\PART{\GMU{part}}] oguṇṭhito[covered-\PAST{\GMU{past-part}}] antaraghare[inside house-\LOC{\GMU{loc-sg-n}}] gamissāmī”[go-\FUT{\GMU{1-sg-fut}}] ti[-\NUL{\GMU{}}] sikkhā[train-\NOM{\GMU{nom-sg-f}}] karaṇīyā.[done-\INS{\GMU{ins-sg}}]+ 24.[] “Na[not-\PART{\GMU{part}}] oguṇṭhito[covered-\PAST{\GMU{past-part}}] antaraghare[inside house-\LOC{\GMU{loc-sg-n}}] nisīdissāmī”[sit-\FUT{\GMU{1-sg-fut}}] ti[-\NUL{\GMU{}}] sikkhā[train-\NOM{\GMU{nom-sg-f}}] karaṇīyā.[done-\INS{\GMU{ins-sg}}]
\endgl
\switchcolumn*
\end{flushleft}


{\EnglishColumn

\begin{doublespace}
25. I shall not go in a crouching (posture) inside an inhabited area,” thus the training is to be done.
\end{doublespace}}

\switchcolumn


\begin{flushleft}
\begingl
 25.[] “Na[not-\PART{\GMU{part}}] ukkuṭikāya[crouch posture-\INS{\GMU{ins-sg-f}}] antaraghare[inside house-\LOC{\GMU{loc-sg-n}}] gamissāmī”[go-\FUT{\GMU{1-sg-fut}}] ti[-\NUL{\GMU{}}] sikkhā[train-\NOM{\GMU{nom-sg-f}}] karaṇīyā.[done-\INS{\GMU{ins-sg}}]
\endgl
\switchcolumn*
\end{flushleft}


{\EnglishColumn

\begin{doublespace}
26. I shall not sit with the (knees) clasped-around inside an inhabited area,” thus the training is to be done.
\end{doublespace}}

\switchcolumn


\begin{flushleft}
\begingl
 26.[] “Na[not-\PART{\GMU{part}}] pallatthikāya[-\NUL{\GMU{}}] antaraghare[inside house-\LOC{\GMU{loc-sg-n}}] nisīdissāmī”[sit-\FUT{\GMU{1-sg-fut}}] ti[-\NUL{\GMU{}}] sikkhā[train-\NOM{\GMU{nom-sg-f}}] karaṇīyā.[done-\INS{\GMU{ins-sg}}]
\endgl
\switchcolumn*
\end{flushleft}


{\EnglishColumn

\begin{doublespace}
(Here ends) the Twenty-Six on Proper Behavior
\end{doublespace}}

\switchcolumn


\begin{flushleft}
\begingl
 Chabbīsati[] sāruppā.[]
\endgl
\switchcolumn*
\end{flushleft}


{\EnglishColumn

\begin{doublespace}
27. I shall accept alms-food appreciatively,” thus the training is to be done.\\
28. I shall accept alms-food paying attention to the bowl,” thus the training is to be done.\\
29. I shall accept alms-food which has curry in the proper proportion,” thus the training is to be done.\\
30. I shall accept alms-food which is level with the rim,” thus the training is to be done.\\
31. I shall eat alms-food appreciatively,” thus the training is to be done.\\
32. I shall eat alms-food paying attention to the bowl,” thus the training is to be done.\\
33. I shall eat alms-food systematically,” thus the training is to be done.\\
34. I shall eat alms-food which has curry in the proper proportion,” thus the training is to be done.\\
35. I shall not eat alms-food, having pressed (it) down into a shall heap,” thus the training is to be done.\\
36. I shall not cover curry or condiment with rice out of liking for more,” thus the training is to be done.\\
37. I shall not eat curry or rice, (when) not ill , having requested (it) for his own benefit, thus the training is to be done.\\
38. I shall not look at another's bowl finding fault,” thus the training is to be done.\\
39. I shall not make an over-large morsel (of food),” thus the training is to be done.\\
40. I shall eat a round piece (of food),” thus the training is to be done.\\
41. I shall not open the mouth when the morsel (of food) has not been brought to (it),” thus the training is to be done.\\
42. I shall not put the whole hand onto the mouth while eating ,” thus the training is to be done.\\
43. I shall not speak with a mouth which has a morsel (of food in it),” , thus the training is to be done.\\
44. I shall not eat tossing up bits (of food),” thus the training is to be done.\\
45. I shall not eat biting off a morsel (of food),” thus the training is to be done.\\
46. I shall not eat puffing up (the cheeks),” thus the training is to be done.\\
47. I shall not eat shaking (food) off the hand,” thus the training is to be done.\\
48. I shall not eat scattering rice-grains,” thus the training is to be done.\\
49. I shall not eat sticking out the tongue,” thus the training is to be done.\\
50. I shall not eat making chomping (sounds),” thus the training is to be done.\\
51. I shall not eat making slurping (sounds),” thus the training is to be done.\\
52. I shall not eat licking the hand,” thus the training is to be done.\\
53. I shall not eat licking the bowl,” thus the training is to be done.\\
54. I shall not eat licking the lip(s),” thus the training is to be done.\\
55. I shall not accept a drinking-water cup with a hand which is (soiled) with food,” thus the training is to be done.\\
56. I shall not throw away bowl-washing water which has rice-grains (in it) in an inhabited area,” thus the training is to be done.
\end{doublespace}}

\switchcolumn


\begin{flushleft}
\begingl
 27.[] “Sakkaccaṁ[-\NUL{\GMU{}}] piṇḍapātaṁ[alms food-\ACC{\GMU{acc-sg}}] paṭiggahessāmī”[accept-\FUT{\GMU{1-sg-fut}}] ti[-\NUL{\GMU{}}] sikkhā[train-\NOM{\GMU{nom-sg-f}}] karaṇīyā.[done-\INS{\GMU{ins-sg}}]+ 28.[] “Pattasaññī[bowl.perceive-\ADJ{\GMU{adj}}] piṇḍapātaṁ[alms food-\ACC{\GMU{acc-sg}}] paṭiggahessāmī”[accept-\FUT{\GMU{1-sg-fut}}] ti[-\NUL{\GMU{}}] sikkhā[train-\NOM{\GMU{nom-sg-f}}] karaṇīyā.[done-\INS{\GMU{ins-sg}}]+ 29.[] “Samasūpakaṁ[-\NUL{\GMU{}}] piṇḍapātaṁ[alms food-\ACC{\GMU{acc-sg}}] paṭiggahessāmī”[accept-\FUT{\GMU{1-sg-fut}}] ti[-\NUL{\GMU{}}] sikkhā[train-\NOM{\GMU{nom-sg-f}}] karaṇīyā.[done-\INS{\GMU{ins-sg}}]+ 30.[] “Samatittikaṁ[-\NUL{\GMU{}}] piṇḍapātaṁ[alms food-\ACC{\GMU{acc-sg}}] paṭiggahessāmī”[accept-\FUT{\GMU{1-sg-fut}}] ti[-\NUL{\GMU{}}] sikkhā[train-\NOM{\GMU{nom-sg-f}}] karaṇīyā.[done-\INS{\GMU{ins-sg}}]+ 31.[] “Sakkaccaṁ[-\NUL{\GMU{}}] piṇḍapātaṁ[alms food-\ACC{\GMU{acc-sg}}] bhuñjissāmī”[eat-\FUT{\GMU{1-sg-fut}}] ti[-\NUL{\GMU{}}] sikkhā[train-\NOM{\GMU{nom-sg-f}}] karaṇīyā.[done-\INS{\GMU{ins-sg}}]+ 32.[] “Pattasaññī[bowl.perceive-\ADJ{\GMU{adj}}] piṇḍapātaṁ[alms food-\ACC{\GMU{acc-sg}}] bhuñjissāmī”[eat-\FUT{\GMU{1-sg-fut}}] ti[-\NUL{\GMU{}}] sikkhā[train-\NOM{\GMU{nom-sg-f}}] karaṇīyā.[done-\INS{\GMU{ins-sg}}]+ 33.[] “Sapadānaṁ[-\NUL{\GMU{}}] piṇḍapātaṁ[alms food-\ACC{\GMU{acc-sg}}] bhuñjissāmī”[eat-\FUT{\GMU{1-sg-fut}}] ti[-\NUL{\GMU{}}] sikkhā[train-\NOM{\GMU{nom-sg-f}}] karaṇīyā[done-\INS{\GMU{ins-sg}}]+ 34.[] “Samasūpakaṁ[-\NUL{\GMU{}}] piṇḍapātaṁ[alms food-\ACC{\GMU{acc-sg}}] bhuñjissāmī”[eat-\FUT{\GMU{1-sg-fut}}] ti[-\NUL{\GMU{}}] sikkhā[train-\NOM{\GMU{nom-sg-f}}] karaṇīyā.[done-\INS{\GMU{ins-sg}}]+ 35.[] “Na[not-\PART{\GMU{part}}] thūpato[-\NUL{\GMU{}}] omadditvā[work down-\ABS{\GMU{abs}}] piṇḍapātaṁ[alms food-\ACC{\GMU{acc-sg}}] bhuñjissāmī”[eat-\FUT{\GMU{1-sg-fut}}] ti[-\NUL{\GMU{}}] sikkhā[train-\NOM{\GMU{nom-sg-f}}] karaṇīyā.[done-\INS{\GMU{ins-sg}}]+ 36.[] “Na[not-\PART{\GMU{part}}] sūpaṁ[-\NUL{\GMU{}}] vā[or-\IND{\GMU{ind}}] byañjanaṁ[curry-\ACC{\GMU{acc-sg-n}}] vā[or-\IND{\GMU{ind}}] odanena[rice-\INS{\GMU{ins-sg}}] paṭicchādessāmi[-\NUL{\GMU{}}] bhiyyokamyataṁ[] upādāyā”[take up-\NUL{\GMU{}}] ti[-\NUL{\GMU{}}] sikkhā[train-\NOM{\GMU{nom-sg-f}}] karaṇīyā.[done-\INS{\GMU{ins-sg}}]+ 37.[] “Na[not-\PART{\GMU{part}}] sūpaṁ[-\NUL{\GMU{}}] vā[or-\IND{\GMU{ind}}] odanaṁ[rice-\ACC{\GMU{acc-sg}}] vā[or-\IND{\GMU{ind}}] agilāno[not.sick-\ADJ{\GMU{adj}}] attano[self-\DAT{\GMU{dat-sg}}] atthāya[need-\DAT{\GMU{dat-sg}}] viññāpetvā[request-\ABS{\GMU{abs}}] bhuñjissāmī”[eat-\FUT{\GMU{1-sg-fut}}] ti[-\NUL{\GMU{}}] sikkhā[train-\NOM{\GMU{nom-sg-f}}] karaṇīyā.[done-\INS{\GMU{ins-sg}}]+ 38.[] “Na[not-\PART{\GMU{part}}] ujjhānasaññī[fault perceive-\ADJ{\GMU{adj}}] paresaṁ[-\NUL{\GMU{}}] pattaṁ[bowl-\ACC{\GMU{acc-sg}}] olokessāmī”[look down-\PRESIND{\GMU{1-sg-presind}}] ti[-\NUL{\GMU{}}] sikkhā[train-\NOM{\GMU{nom-sg-f}}] karaṇīyā.[done-\INS{\GMU{ins-sg}}]+ 39.[] “Nātimahantaṁ[-\NUL{\GMU{}}] kavaḷaṁ[] karissāmī”[make-\FUT{\GMU{1-sg-fut}}] ti[-\NUL{\GMU{}}] sikkhā[train-\NOM{\GMU{nom-sg-f}}] karaṇīyā.[done-\INS{\GMU{ins-sg}}]+ 40.[] “Parimaṇḍalaṁ[around.circle-\ADJ{\GMU{adj}}] ālopaṁ[pc. food-\ACC{\GMU{acc-sg}}] karissāmī”[make-\FUT{\GMU{1-sg-fut}}] ti[-\NUL{\GMU{}}] sikkhā[train-\NOM{\GMU{nom-sg-f}}] karaṇīyā.[done-\INS{\GMU{ins-sg}}]+ 41.[] “Na[not-\PART{\GMU{part}}] anāhaṭe[not.take to-\ADJ{\GMU{adj}}] kavaḷe[] mukhadvāraṁ[mouth.door-\ACC{\GMU{acc-sg}}] vivarissāmī”[-\NUL{\GMU{}}] ti[-\NUL{\GMU{}}] sikkhā[train-\NOM{\GMU{nom-sg-f}}] karaṇīyā.[done-\INS{\GMU{ins-sg}}]+ 42.[] “Na[not-\PART{\GMU{part}}] bhuñjamāno[eat-\PRES{\GMU{pres-part}}] sabbaṁ[-\NUL{\GMU{}}] hatthaṁ[-\NUL{\GMU{}}] mukhe[-\NUL{\GMU{}}] pakkhipissāmī”[put onto-\FUT{\GMU{1-sg-fut}}] ti[-\NUL{\GMU{}}] sikkhā[train-\NOM{\GMU{nom-sg-f}}] karaṇīyā.[done-\INS{\GMU{ins-sg}}]+ 43.[] “Na[not-\PART{\GMU{part}}] sakavaḷena[] mukhena[-\NUL{\GMU{}}] byāharissāmī”[speak-\FUT{\GMU{1-sg-fut}}] ti[-\NUL{\GMU{}}] sikkhā[train-\NOM{\GMU{nom-sg-f}}] karaṇīyā.[done-\INS{\GMU{ins-sg}}]+ 44.[] “Na[not-\PART{\GMU{part}}] piṇḍukkhepakaṁ[alms.toss-\ADV{\GMU{adv}}] bhuñjissāmī”[eat-\FUT{\GMU{1-sg-fut}}] ti[-\NUL{\GMU{}}] sikkhā[train-\NOM{\GMU{nom-sg-f}}] karaṇīyā.[done-\INS{\GMU{ins-sg}}]+ 45.[] “Na[not-\PART{\GMU{part}}] kavaḷāvacchedakaṁ[] bhuñjissāmī”[eat-\FUT{\GMU{1-sg-fut}}] ti[-\NUL{\GMU{}}] sikkhā[train-\NOM{\GMU{nom-sg-f}}] karaṇīyā.[done-\INS{\GMU{ins-sg}}]+ 46.[] “Na[not-\PART{\GMU{part}}] avagaṇḍakārakaṁ[swell make-\ADV{\GMU{adv}}] bhuñjissāmī”[eat-\FUT{\GMU{1-sg-fut}}] ti[-\NUL{\GMU{}}] sikkhā[train-\NOM{\GMU{nom-sg-f}}] karaṇīyā.[done-\INS{\GMU{ins-sg}}]+ 47.[] “Na[not-\PART{\GMU{part}}] hatthaniddhūnakaṁ[] bhuñjissāmī”[eat-\FUT{\GMU{1-sg-fut}}] ti[-\NUL{\GMU{}}] sikkhā[train-\NOM{\GMU{nom-sg-f}}] karaṇīyā.[done-\INS{\GMU{ins-sg}}]+ 48.[] “Na[not-\PART{\GMU{part}}] sitthāvakārakaṁ[-\NUL{\GMU{}}] bhuñjissāmī”[eat-\FUT{\GMU{1-sg-fut}}] ti[-\NUL{\GMU{}}] sikkhā[train-\NOM{\GMU{nom-sg-f}}] karaṇīyā.[done-\INS{\GMU{ins-sg}}]+ 49.[] “Na[not-\PART{\GMU{part}}] jivhānicchārakaṁ[-\NUL{\GMU{}}] bhuñjissāmī”[eat-\FUT{\GMU{1-sg-fut}}] ti[-\NUL{\GMU{}}] sikkhā[train-\NOM{\GMU{nom-sg-f}}] karaṇīyā.[done-\INS{\GMU{ins-sg}}]+ 50.[] “Na[not-\PART{\GMU{part}}] capucapukārakaṁ[lip smack.make-\ADV{\GMU{adv}}] bhuñjissāmī”[eat-\FUT{\GMU{1-sg-fut}}] ti[-\NUL{\GMU{}}] sikkhā[train-\NOM{\GMU{nom-sg-f}}] karaṇīyā.[done-\INS{\GMU{ins-sg}}]+ 51.[] “Na[not-\PART{\GMU{part}}] surusurukārakaṁ[-\NUL{\GMU{}}] bhuñjissāmī”[eat-\FUT{\GMU{1-sg-fut}}] ti[-\NUL{\GMU{}}] sikkhā[train-\NOM{\GMU{nom-sg-f}}] karaṇīyā.[done-\INS{\GMU{ins-sg}}]+ 52.[] “Na[not-\PART{\GMU{part}}] hatthanillehakaṁ[-\NUL{\GMU{}}] bhuñjissāmī”[eat-\FUT{\GMU{1-sg-fut}}] ti[-\NUL{\GMU{}}] sikkhā[train-\NOM{\GMU{nom-sg-f}}] karaṇīyā.[done-\INS{\GMU{ins-sg}}]+ 53.[] “Na[not-\PART{\GMU{part}}] pattanillehakaṁ[-\NUL{\GMU{}}] bhuñjissāmī”[eat-\FUT{\GMU{1-sg-fut}}] ti[-\NUL{\GMU{}}] sikkhā[train-\NOM{\GMU{nom-sg-f}}] karaṇīyā.[done-\INS{\GMU{ins-sg}}]+ 54.[] “Na[not-\PART{\GMU{part}}] oṭṭhanillehakaṁ[lip lick-\NUL{\GMU{}}] bhuñjissāmī”[eat-\FUT{\GMU{1-sg-fut}}] ti[-\NUL{\GMU{}}] sikkhā[train-\NOM{\GMU{nom-sg-f}}] karaṇīyā.[done-\INS{\GMU{ins-sg}}]+ 55.[] “Na[not-\PART{\GMU{part}}] sāmisena[-\NUL{\GMU{}}] hatthena[-\NUL{\GMU{}}] pānīyathālakaṁ[-\NUL{\GMU{}}] paṭiggahessāmī”[accept-\FUT{\GMU{1-sg-fut}}] ti[-\NUL{\GMU{}}] sikkhā[train-\NOM{\GMU{nom-sg-f}}] karaṇīyā.[done-\INS{\GMU{ins-sg}}]+ 56.[] “Na[not-\PART{\GMU{part}}] sasitthakaṁ[-\NUL{\GMU{}}] pattadhovanaṁ[-\NUL{\GMU{}}] antaraghare[inside house-\LOC{\GMU{loc-sg-n}}] chaḍḍessāmī”[-\NUL{\GMU{}}] ti[-\NUL{\GMU{}}] sikkhā[train-\NOM{\GMU{nom-sg-f}}] karaṇīyā.[done-\INS{\GMU{ins-sg}}]+
\endgl
\switchcolumn*
\end{flushleft}


{\EnglishColumn

\begin{doublespace}
(Here ends) the Group of Thirty regarding Food.
\end{doublespace}}

\switchcolumn


\begin{flushleft}
\begingl
 Samatiṁsa[] bhojanapaṭisaṁyuttā[]
\endgl
\switchcolumn*
\end{flushleft}


{\EnglishColumn

\begin{doublespace}
57. I shall not teach Dhamma to one who has a sunshade in (his) hand, (and) who is not ill,” thus the training is to be done.\\
58. I shall not teach Dhamma to one who has a stick in (his) hand, (and) who is not ill,” thus the training is to be done.\\
59. I shall not teach Dhamma to one who has a knife in (his) hand (and) who is not ill,” thus the training is to be done.\\
60. I shall not teach Dhamma to one who has a weapon in (his) hand, (and) who is not ill,” thus the training is to be done.\\
61. I shall not teach Dhamma to one who is wearing shoes, (and) who is not ill,” thus the training is to be done.\\
62. I shall not teach Dhamma to one who is wearing sandals, (and) who is not ill,” thus the training is to be done.\\
63. I shall not teach Dhamma to one who is in a vehicle, (and) who is not ill,” thus the training is to be done.\\
64. I shall not teach Dhamma to one who is on a couch, (and) who is not ill,” thus the training is to be done.\\
65. I shall not teach Dhamma to one sitting with (the knees) clasped-around, (and) who is not ill,” thus the training is to be done.\\
66. I shall not teach Dhamma to one whose head is wrapped (with a turban), (and) who is not ill,” thus the
training is to be done.\\
67. I shall not teach Dhamma to one whose head is covered, (and) who is not ill,” thus the training is to be done.\\
68. Having sat down on the ground, I shall not teach Dhamma, to one who is sitting on a seat, (and) who is not ill,” thus the training is to be done.\\
69. Having sat down on a low seat, I shall not teach Dhamma to one who is sitting on a high seat (and) who is not ill,” thus the training is to be done.\\
70. I shall not teach Dhamma (while) standing, to one who is sitting, (and) who is not ill,” thus the training is to be done.\\
71. I shall not teach Dhamma (while) walking behind, to one who is going in front, (and) who is not ill,” thus the training is to be done.\\
72. I shall not teach Dhamma (while) walking off the path to one walking on the path, (and) who is not ill,” thus the training is to be done.\\
\end{doublespace}}

\switchcolumn


\begin{flushleft}
\begingl
 57.[] “Na[not-\PART{\GMU{part}}] chattapāṇissa[-\NUL{\GMU{}}] agilānassa[not.sick-\ADJ{\GMU{adj}}] dhammaṁ[act-\ACC{\GMU{acc-sg}}] desissāmī”[-\NUL{\GMU{}}] ti[-\NUL{\GMU{}}] sikkhā[train-\NOM{\GMU{nom-sg-f}}] karaṇīyā.[done-\INS{\GMU{ins-sg}}]+ 58.[] “Na[not-\PART{\GMU{part}}] daṇḍapāṇissa[-\NUL{\GMU{}}] agilānassa[not.sick-\ADJ{\GMU{adj}}] dhammaṁ[act-\ACC{\GMU{acc-sg}}] desissāmī”[-\NUL{\GMU{}}] ti[-\NUL{\GMU{}}] sikkhā[train-\NOM{\GMU{nom-sg-f}}] karaṇīyā.[done-\INS{\GMU{ins-sg}}]+ 59.[] “Na[not-\PART{\GMU{part}}] satthapāṇissa[-\NUL{\GMU{}}] agilānassa[not.sick-\ADJ{\GMU{adj}}] dhammaṁ[act-\ACC{\GMU{acc-sg}}] desissāmī”[-\NUL{\GMU{}}] ti[-\NUL{\GMU{}}] sikkhā[train-\NOM{\GMU{nom-sg-f}}] karaṇīyā.[done-\INS{\GMU{ins-sg}}]+ 60.[] “Na[not-\PART{\GMU{part}}] āvudhapāṇissa[weapon in hand-\DAT{\GMU{dat-sg}}] agilānassa[not.sick-\ADJ{\GMU{adj}}] dhammaṁ[act-\ACC{\GMU{acc-sg}}] desissāmī”[-\NUL{\GMU{}}] ti[-\NUL{\GMU{}}] sikkhā[train-\NOM{\GMU{nom-sg-f}}] karaṇīyā.[done-\INS{\GMU{ins-sg}}]+ 61.[] “Na[not-\PART{\GMU{part}}] pādukārūḷhassa[-\NUL{\GMU{}}] agilānassa[not.sick-\ADJ{\GMU{adj}}] dhammaṁ[act-\ACC{\GMU{acc-sg}}] desissāmī”[-\NUL{\GMU{}}] ti[-\NUL{\GMU{}}] sikkhā[train-\NOM{\GMU{nom-sg-f}}] karaṇīyā.[done-\INS{\GMU{ins-sg}}]+ 62.[] “Na[not-\PART{\GMU{part}}] upāhanārūḷhassa[-\NUL{\GMU{}}] agilānassa[not.sick-\ADJ{\GMU{adj}}] dhammaṁ[act-\ACC{\GMU{acc-sg}}] desissāmī”[-\NUL{\GMU{}}] ti[-\NUL{\GMU{}}] sikkhā[train-\NOM{\GMU{nom-sg-f}}] karaṇīyā.[done-\INS{\GMU{ins-sg}}]+ 63.[] “Na[not-\PART{\GMU{part}}] yānagatassa[-\NUL{\GMU{}}] agilānassa[not.sick-\ADJ{\GMU{adj}}] dhammaṁ[act-\ACC{\GMU{acc-sg}}] desissāmī”[-\NUL{\GMU{}}] ti[-\NUL{\GMU{}}] sikkhā[train-\NOM{\GMU{nom-sg-f}}] karaṇīyā.[done-\INS{\GMU{ins-sg}}]+ 64.[] “Na[not-\PART{\GMU{part}}] sayanagatassa[-\NUL{\GMU{}}] agilānassa[not.sick-\ADJ{\GMU{adj}}] dhammaṁ[act-\ACC{\GMU{acc-sg}}] desissāmī”[-\NUL{\GMU{}}] ti[-\NUL{\GMU{}}] sikkhā[train-\NOM{\GMU{nom-sg-f}}] karaṇīyā.[done-\INS{\GMU{ins-sg}}]+ 65.[] “Na[not-\PART{\GMU{part}}] pallatthikāya[-\NUL{\GMU{}}] nisinnassa[-\NUL{\GMU{}}] agilānassa[not.sick-\ADJ{\GMU{adj}}] dhammaṁ[act-\ACC{\GMU{acc-sg}}] desissāmī”[-\NUL{\GMU{}}] ti[-\NUL{\GMU{}}] sikkhā[train-\NOM{\GMU{nom-sg-f}}] karaṇīyā.[done-\INS{\GMU{ins-sg}}]+ 66.[] “Na[not-\PART{\GMU{part}}] veṭṭhitasīsassa[] agilānassa[not.sick-\ADJ{\GMU{adj}}] dhammaṁ[act-\ACC{\GMU{acc-sg}}] desissāmī”[-\NUL{\GMU{}}] ti[-\NUL{\GMU{}}] sikkhā[train-\NOM{\GMU{nom-sg-f}}] karaṇīyā.[done-\INS{\GMU{ins-sg}}]+ 67.[] “Na[not-\PART{\GMU{part}}] oguṇṭhitasīsassa[covered.head-\ADJ{\GMU{adj}}] agilānassa[not.sick-\ADJ{\GMU{adj}}] dhammaṁ[act-\ACC{\GMU{acc-sg}}] desissāmī”[-\NUL{\GMU{}}] ti[-\NUL{\GMU{}}] sikkhā[train-\NOM{\GMU{nom-sg-f}}] karaṇīyā.[done-\INS{\GMU{ins-sg}}]+ 68.[] “Na[not-\PART{\GMU{part}}] chamāyaṁ[] nisīditvā[-\NUL{\GMU{}}] āsane[seat-\LOC{\GMU{loc-sg-n}}] nisinnassa[-\NUL{\GMU{}}] agilānassa[not.sick-\ADJ{\GMU{adj}}] dhammaṁ[act-\ACC{\GMU{acc-sg}}] desissāmī”[-\NUL{\GMU{}}] ti[-\NUL{\GMU{}}] sikkhā[train-\NOM{\GMU{nom-sg-f}}] karaṇīyā.[done-\INS{\GMU{ins-sg}}]+ 69.[] “Na[not-\PART{\GMU{part}}] nīce[-\NUL{\GMU{}}] āsane[seat-\LOC{\GMU{loc-sg-n}}] nisīditvā[-\NUL{\GMU{}}] ucce[high-\ADJ{\GMU{adj}}] āsane[seat-\LOC{\GMU{loc-sg-n}}] nisinnassa[-\NUL{\GMU{}}] agilānassa[not.sick-\ADJ{\GMU{adj}}] dhammaṁ[act-\ACC{\GMU{acc-sg}}] desissāmī”[-\NUL{\GMU{}}] ti[-\NUL{\GMU{}}] sikkhā[train-\NOM{\GMU{nom-sg-f}}] karaṇīyā.[done-\INS{\GMU{ins-sg}}]+ 70.[] “Na[not-\PART{\GMU{part}}] ṭhito[stand-\ADJ{\GMU{adj}}] nisinnassa[-\NUL{\GMU{}}] agilānassa[not.sick-\ADJ{\GMU{adj}}] dhammaṁ[act-\ACC{\GMU{acc-sg}}] desissāmī”[-\NUL{\GMU{}}] ti[-\NUL{\GMU{}}] sikkhā[train-\NOM{\GMU{nom-sg-f}}] karaṇīyā.[done-\INS{\GMU{ins-sg}}]+ 71.[] “Na[not-\PART{\GMU{part}}] pacchato[-\NUL{\GMU{}}] gacchanto[go-\PRES{\GMU{pres-part}}] purato[-\NUL{\GMU{}}] gacchantassa[go-\DAT{\GMU{dat-sg}}] agilānassa[not.sick-\ADJ{\GMU{adj}}] dhammaṁ[act-\ACC{\GMU{acc-sg}}] desissāmī”[-\NUL{\GMU{}}] ti[-\NUL{\GMU{}}] sikkhā[train-\NOM{\GMU{nom-sg-f}}] karaṇīyā.[done-\INS{\GMU{ins-sg}}]+ 72.[] “Na[not-\PART{\GMU{part}}] uppathena[-\NUL{\GMU{}}] gacchanto[go-\PRES{\GMU{pres-part}}] pathena[-\NUL{\GMU{}}] gacchantassa[go-\DAT{\GMU{dat-sg}}] agilānassa[not.sick-\ADJ{\GMU{adj}}] dhammaṁ[act-\ACC{\GMU{acc-sg}}] desissāmī”[-\NUL{\GMU{}}] ti[-\NUL{\GMU{}}] sikkhā[train-\NOM{\GMU{nom-sg-f}}] karaṇīyā.[done-\INS{\GMU{ins-sg}}]
\endgl
\switchcolumn*
\end{flushleft}


{\EnglishColumn

\begin{doublespace}
(Here ends) the Group of Sixteen regarding Teaching Dhamma.
\end{doublespace}}

\switchcolumn


\begin{flushleft}
\begingl
 Soḷasa[] dhammadesanāpaṭisaṁyuttā[]
\endgl
\switchcolumn*
\end{flushleft}


{\EnglishColumn

\begin{doublespace}
73. I shall not excrete or urinate (while) standing (and while) not ill,” thus the training is to be done.\\
74. I shall not excrete or urinate or spit on crops, (while) not ill,” thus the training is to be done.\\
75. I shall not excrete or urinate or spit in water, (while) not ill,” thus the training is to be done.\\
\end{doublespace}}

\switchcolumn


\begin{flushleft}
\begingl
 73.[] “Na[not-\PART{\GMU{part}}] ṭhito[stand-\ADJ{\GMU{adj}}] agilāno[not.sick-\ADJ{\GMU{adj}}] uccāraṁ[feces-\ACC{\GMU{acc-sg}}] vā[or-\IND{\GMU{ind}}] passāvaṁ[urine-\ACC{\GMU{acc-sg}}] vā[or-\IND{\GMU{ind}}] karissāmī”[make-\FUT{\GMU{1-sg-fut}}] ti[-\NUL{\GMU{}}] sikkhā[train-\NOM{\GMU{nom-sg-f}}] karaṇīyā.[done-\INS{\GMU{ins-sg}}]+ 74.[] “Na[not-\PART{\GMU{part}}] harite[-\NUL{\GMU{}}] agilāno[not.sick-\ADJ{\GMU{adj}}] uccāraṁ[feces-\ACC{\GMU{acc-sg}}] vā[or-\IND{\GMU{ind}}] passāvaṁ[urine-\ACC{\GMU{acc-sg}}] vā[or-\IND{\GMU{ind}}] kheḷaṁ[spittle-\ACC{\GMU{acc-sg-n}}] vā[or-\IND{\GMU{ind}}] karissāmī”[make-\FUT{\GMU{1-sg-fut}}] ti[-\NUL{\GMU{}}] sikkhā[train-\NOM{\GMU{nom-sg-f}}] karaṇīyā.[done-\INS{\GMU{ins-sg}}]+ 75.[] “Na[not-\PART{\GMU{part}}] udake[water-\LOC{\GMU{loc-sg-n}}] agilāno[not.sick-\ADJ{\GMU{adj}}] uccāraṁ[feces-\ACC{\GMU{acc-sg}}] vā[or-\IND{\GMU{ind}}] passāvaṁ[urine-\ACC{\GMU{acc-sg}}] vā[or-\IND{\GMU{ind}}] kheḷaṁ[spittle-\ACC{\GMU{acc-sg-n}}] vā[or-\IND{\GMU{ind}}] karissāmī”[make-\FUT{\GMU{1-sg-fut}}] ti[-\NUL{\GMU{}}] sikkhā[train-\NOM{\GMU{nom-sg-f}}] karaṇīyā.[done-\INS{\GMU{ins-sg}}]+
\endgl
\switchcolumn*
\end{flushleft}


{\EnglishColumn

\begin{doublespace}
(Here ends) the Three Miscellaneous
\end{doublespace}}

\switchcolumn


\begin{flushleft}
\begingl
 Tayo[3-\NUM{\GMU{num}}] pakiṇṇakā[]
\endgl
\switchcolumn*
\end{flushleft}


{\EnglishColumn

\begin{doublespace}
Venerables, , the cases related to the training have been recited.\\
Concerning that I ask the Venerables: (Are you) pure in this?\\
A second time again I ask: (Are you) pure in this?\\
A third time again I ask: (Are you) pure in this?\\
The Venerables are pure in this, therefore there is silence, thus I keep this (in mind).
\end{doublespace}}

\switchcolumn


\begin{flushleft}
\begingl
 Uddiṭṭhā[recite-\PAST{\GMU{past-part}}] kho[indeed!-\EMPH{\GMU{emph}}] āyasmanto[Ven.-\VOC{\GMU{voc-pl}}] sekhiyā[] dhammā.[rule-\NOM{\GMU{nom-pl}}]+ Tatth’āyasmante[] pucchāmi:[ask-\PRESIND{\GMU{1-sg-presind}}] Kacci’ttha[] parisuddhā?[pure-\ADJ{\GMU{adj}}]+ Dutiyam’pi[second time-\ACC{\GMU{acc-sg-nt}}] pucchāmi:[ask-\PRESIND{\GMU{1-sg-presind}}] Kacci’ttha[] parisuddhā?[pure-\ADJ{\GMU{adj}}]+ Tatiyam’pi[] pucchāmi:[ask-\PRESIND{\GMU{1-sg-presind}}] Kacci’ttha[] parisuddhā?[pure-\ADJ{\GMU{adj}}]+ Parisuddh’etth’āyasmanto,[] tasmā[therefore-\ABL{\GMU{abl-sg}}] tuṇhī,[silent-\ADV{\GMU{adv}}] evam’etaṁ[thus.this-\ACC{\GMU{acc-sg-n}}] dhārayāmi.[keep in mind-\PRESIND{\GMU{1-sg-presind}}]+
\endgl
\switchcolumn*
\end{flushleft}


{\EnglishColumn

\begin{doublespace}
The cases related to the training have finished.

\end{doublespace}}

\switchcolumn


\begin{flushleft}
\begingl
 Sekhiyā[] niṭṭhitā[]
\endgl
\switchcolumn*
\end{flushleft}


{\EnglishColumn

\begin{doublespace}
Venerables, these seven cases that are settlements of legal issues come up for recitation.
\end{doublespace}}

\switchcolumn


\begin{flushleft}
\begingl
 Ime[this-\NOM{\GMU{nom-pl}}] kho[indeed!-\EMPH{\GMU{emph}}] pan’āyasmanto[venerable-\VOC{\GMU{voc-pl}}] sattādhikaraṇasamathā[] dhammā[rule-\NOM{\GMU{nom-pl}}] uddesaṁ[recitation-\ACC{\GMU{acc-sg}}] āgacchanti.[come up-\PRESIND{\GMU{3-pl-presind}}]
\endgl
\switchcolumn*
\end{flushleft}


{\EnglishColumn

\begin{doublespace}
For the calming, for the stilling of whichever legal issues have arisen:\\
the removal through the presence (of the bhikkhu) is to be given,\\
the removal (of the accusation) through remembrance is to be given,\\
the removal (of the accusation) through not (being) insane is to be given,\\
he is to be made to do (the offence-procedure) through admitting (the offence),\\
the (decision of the) majority,\\
(the decision making it) worse for him,\\
(the decision) covering (the offences as if) with grass.
\end{doublespace}}

\switchcolumn


\begin{flushleft}
\begingl
 Uppannuppannānaṁ[whichever arisen-\PRES{\GMU{pres-part}}] adhikaraṇānaṁ[] samathāya[calm-\DAT{\GMU{dat-sg}}] vūpasamāya:[settle-\DAT{\GMU{dat-sg}}]+ Sammukhāvinayo[with face.remove-\NOM{\GMU{nom-sg}}] dātabbo,[give-\FUT{\GMU{fut-pass-part}}]+ Sativinayo[memory.remove-\NOM{\GMU{nom-sg}}] dātabbo,[give-\FUT{\GMU{fut-pass-part}}]+ Amūḷhavinayo[not insane removal-\NOM{\GMU{nom-sg}}] dātabbo,[give-\FUT{\GMU{fut-pass-part}}]+ Paṭiññātakaraṇaṁ,[]+ Yebhuyyasikā,[which.more-\NOM{\GMU{nom-sg-f}}]+ Tassapāpiyasikā,[]+ Tiṇavatthārako’ti.[]
\endgl
\switchcolumn*
\end{flushleft}


{\EnglishColumn

\begin{doublespace}
Recited, Venerables, have been the seven cases that are settlements of legal issues.\\
Concerning that I ask the Venerables: (Are you) pure in this?\\
A second time again I ask: (Are you) pure in this?\\
A third time again I ask: (Are you) pure in this?\\
The venerables are pure in this, therefore there is silence, thus I keep this (in mind).
\end{doublespace}}

\switchcolumn


\begin{flushleft}
\begingl
 Uddiṭṭhā[recite-\PAST{\GMU{past-part}}] kho[indeed!-\EMPH{\GMU{emph}}] āyasmanto[Ven.-\VOC{\GMU{voc-pl}}] sattādhikaraṇasamathā[] dhammā.[rule-\NOM{\GMU{nom-pl}}]+ Tatth’āyasmante[] pucchāmi:[ask-\PRESIND{\GMU{1-sg-presind}}] Kacci’ttha[] parisuddhā?[pure-\ADJ{\GMU{adj}}]+ Dutiyam’pi[second time-\ACC{\GMU{acc-sg-nt}}] pucchāmi:[ask-\PRESIND{\GMU{1-sg-presind}}] Kacci’ttha[] parisuddhā?[pure-\ADJ{\GMU{adj}}]+ Tatiyam’pi[] pucchāmi:[ask-\PRESIND{\GMU{1-sg-presind}}] Kacci’ttha[] parisuddhā?[pure-\ADJ{\GMU{adj}}]+ Parisuddh’etth’āyasmanto,[] tasmā[therefore-\ABL{\GMU{abl-sg}}] tuṇhī,[silent-\ADV{\GMU{adv}}] evam’etaṁ[thus.this-\ACC{\GMU{acc-sg-n}}] dhārayāmi.[keep in mind-\PRESIND{\GMU{1-sg-presind}}]
\endgl
\switchcolumn*
\end{flushleft}


{\EnglishColumn

\begin{doublespace}
The cases that are settlements of legal issues have finished.
\end{doublespace}}

\switchcolumn


\begin{flushleft}
\begingl
 Sattādhikaraṇasamathā[] niṭṭhitā[]
\endgl
\pagebreak
\switchcolumn*
\end{flushleft}


{\EnglishColumn

\begin{doublespace}
Venerables, the introduction has been recited.
Venerables, the four cases involving disqualification have been recited.
Venerables, the thirteen cases (involving) the community in the beginning and in the rest have been recited.
Venerables, the two indefinite cases have been recited.
Venerables, the thirty cases involving expiation with forfeiture have been recited.
Venerables, the ninety-two cases involving expiation have been recited.
Venerables, the four cases that are to be acknowledged have been recited.
Venerables, the cases related to the training have been recited.
Venerables, the seven cases that are settlements of legal issues have been recited.
\end{doublespace}}

\switchcolumn


\begin{flushleft}
\begingl
 Uddiṭṭhaṁ[] kho[indeed!-\EMPH{\GMU{emph}}] āyasmanto[Ven.-\VOC{\GMU{voc-pl}}] nidānaṁ.[] Uddiṭṭhā[recite-\PAST{\GMU{past-part}}] cattāro[4-\NUM{\GMU{num}}] pārājikā[defeat-\ADJ{\GMU{adj}}] dhammā.[rule-\NOM{\GMU{nom-pl}}] Uddiṭṭhā[recite-\PAST{\GMU{past-part}}] terasa[13-\ADJ{\GMU{adj}}] saṅghādisesā[] dhammā.[rule-\NOM{\GMU{nom-pl}}] Uddiṭṭhā[recite-\PAST{\GMU{past-part}}] dve[2-\NUM{\GMU{num}}] aniyatā[] dhammā.[rule-\NOM{\GMU{nom-pl}}] Uddiṭṭhā[recite-\PAST{\GMU{past-part}}] tiṁsa[] nissaggiyā[] pācittiyā[] dhammā.[rule-\NOM{\GMU{nom-pl}}] Uddiṭṭhā[recite-\PAST{\GMU{past-part}}] dvenavuti[] pācittiyā[] dhammā.[rule-\NOM{\GMU{nom-pl}}] Uddiṭṭhā[recite-\PAST{\GMU{past-part}}] cattāro[4-\NUM{\GMU{num}}] pāṭidesanīyā[] dhammā.[rule-\NOM{\GMU{nom-pl}}] Uddiṭṭhā[recite-\PAST{\GMU{past-part}}] sekhiyā[] dhammā.[rule-\NOM{\GMU{nom-pl}}] Uddiṭṭhā[recite-\PAST{\GMU{past-part}}] sattādhikaraṇasamathā[] dhammā.[rule-\NOM{\GMU{nom-pl}}]
\endgl
\switchcolumn*
\end{flushleft}


{\EnglishColumn

\begin{doublespace}
This much (of the training-rules) of the Fortunate One has been handed down in the Sutta, has been included in the Sutta, (and) comes up for recitation half-monthly. Herein is to be trained by all who are united, who are on friendly terms, who are not disputing.
\end{doublespace}}

\switchcolumn


\begin{flushleft}
\begingl
 Ettakan’tassa[] bhagavato[blessed one-\GEN{\GMU{gen-sg}}] sutt’āgataṁ[] suttapariyāpannaṁ[] anvaḍḍhamāsaṁ[after $\sfrac{1}{2}$ month-\ACC{\GMU{acc-sg}}] uddesaṁ[recitation-\ACC{\GMU{acc-sg}}] āgacchati.[] Tattha[about that-\ADV{\GMU{adv}}] sabbeh’eva[] samaggehi[] sammodamānehi[] avivadamānehi[] sikkhitabban’ti.[]
\endgl
\switchcolumn*
\end{flushleft}


{\EnglishColumn

\begin{doublespace}
The Disciplinary Code of the Bhikkhu has been finished.
\end{doublespace}}

\switchcolumn


\begin{flushleft}
\begingl
 Bhikkhupāṭimokkhaṁ[] niṭṭhitaṁ[]
\endgl
\switchcolumn*
\end{flushleft}

PAC 24 “āmisahetu Not Caps because will not be recognized

\end{column}
\end{paracol}

\end{document}

