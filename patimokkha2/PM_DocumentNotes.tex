\begin{leftcolumn*}
\begin{center}
\vspace*{4cm}
{\footnotesize
All Pali text © copyspoke by the Buddha and Noble Sangha\\
Never Grasped, No Rights Reserved.\\
English translation from Draft Version 1.7 "A Translation and Analysis of the Pātimokkha"\\}
{\scriptsize
Copyright (c) 2008, Bhikkhu Ñāṇatusita, Forest Hermitage, Kandy, August 2008\\
 - used for my personal study (Koṇḍañño B.) and not intented for public distribution.\\
 - grammar referenced from this source as well as Digital Pali Reader.
 - special thanks to Ven. Khantiko Bhikkhu for his assistance on the Pubbakiccaṁ section\\}
\end{center}
\end{leftcolumn*}
\begin{rightcolumn}
\begin{center}
\vspace*{4cm}
{\footnotesize
draft version 0.4\\
Compiled by Koṇḍañño Bhikkhu\\
\vspace*{2ex}
Deep Gratitude to Ven. Jāgaro Bhikkhu.\\
\vspace*{1ex}
This compilation would be vastly inferior without his automated scripting\\
of various components to produce the Latex code\\
\vspace*{1cm}
known issues:
\begin{itemize}
\item the current dictionary/lookup only allows a single entry per word. For the most part this has little impact, but gen/dat cases and some words are not properly represented at this time.\\
\item eventually, the english side will be gramatically marked up in order to :\\
	\begin{itemize}
	\item highlight the grammatical function of declension and conjugation
	\item highlight the component words of pali compounds
	\item show derivation of pali words
	\end{itemize}
\end{itemize}
}

\vspace*{1cm}
Abbreviations\\[.3cm]
{\scriptsize
\begin{tabular}{p{0.1\columnwidth-2\tabcolsep}
				p{0.23\columnwidth-2\tabcolsep}
				p{0.1\columnwidth-2\tabcolsep}
				p{0.23\columnwidth-2\tabcolsep}
				p{0.1\columnwidth-2\tabcolsep}
				p{0.23\columnwidth-2\tabcolsep}}
NOM	&	nominative	&	ABS		&absolutive	&	ADJ	&	adjective\\
ACC	&	accusative	&	OPT		&optative	&	ADV	&	adverb\\
INS	&	instrumental&	FUT		&future		&	PERS&	personal\\
DAT	&	dative		&	IMP		&imperative	&	PRO	&	pronoun\\
ABL	&	ablative	&	IND		&indicative	&	PART&	participle\\
GEN	&	genitive	&	INF		&infinitive	&	EMPH&	emphatic\\
LOC	&	locative	&	PRES	&present	&	NEG	&	negative\\
VOC	&	vocative	&	PRESIND	&pres ind	&	NUM	&	numeral\\
	&				&	AOR		&aorist		&	ORD	&	ordinal\\
	&				&	PAST	&past		&	INDE&	indeclinable\\
	&				&	PASS	&passive	&		&	\\

\end{tabular}
}
\end{center}
\end{rightcolumn}
\flushpage
