@ pali-npac-intro.tex
Ime kho pan’āyasmanto tiṁsa nissaggiyā pācittiyā dhammā uddesaṁ āgacchanti.

@ pali-npac-01.tex
Niṭṭhitacīvarasmiṁ bhikkhunā ubbhatasmiṁ kaṭhine, dasāhaparamaṁ atirekacīvaraṁ dhāretabbaṁ. Taṁ atikkāmayato, nissaggiyaṁ pācittiyaṁ.

@ pali-npac-02.tex
Niṭṭhitacīvarasmiṁ bhikkhunā ubbhatasmiṁ kaṭhine, ekarattam’pi ce bhikkhu ticīvarena vippavaseyya, aññatra bhikkhusammatiyā, nissaggiyaṁ pācittiyaṁ.

@ pali-npac-03.tex
Niṭṭhitacīvarasmiṁ bhikkhunā ubbhatasmiṁ kaṭhine, bhikkhuno pan’eva akālacīvaraṁ uppajjeyya, ākaṅkhamānena bhikkhunā paṭiggahetabbaṁ. Paṭiggahetvā khippam’eva kāretabbaṁ. 

@ pali-npac-03a.tex
No c’assa pāripūri, māsaparaman’tena bhikkhunā taṁ cīvaraṁ nikkhipitabbaṁ, ūnassa pāripūriyā satiyā paccāsāya. Tato ce uttariṁ nikkhipeyya satiyā’pi paccāsāya, nissaggiyaṁ pācittiyaṁ.

@ pali-npac-04.tex
Yo pana bhikkhu aññātikāya bhikkhuniyā purāṇacīvaraṁ dhovāpeyya vā rajāpeyya vā ākoṭāpeyya vā, nissaggiyaṁ pācittiyaṁ.

@ pali-npac-05.tex
Yo pana bhikkhu aññātikāya bhikkhuniyā hatthato cīvaraṁ paṭiggaṇheyya aññatra pārivaṭṭakā, nissaggiyaṁ pācittiyaṁ.

@ pali-npac-06.tex
Yo pana bhikkhu aññātakaṁ gahapatiṁ vā gahapatāniṁ vā cīvaraṁ viññāpeyya aññatra samayā, nissaggiyaṁ pācittiyaṁ.

@ pali-npac-06a.tex
Tatth’āyaṁ samayo: Acchinnacīvaro vā hoti bhikkhu naṭṭhacīvaro vā. Ayaṁ tattha samayo.

@ pali-npac-07.tex
Tañce aññātako gahapati vā gahapatānī vā bahūhi cīvarehi abhihaṭṭhuṁ pavāreyya, santaruttaraparaman’tena bhikkhunā tato cīvaraṁ sāditabbaṁ. Tato ce uttariṁ sādiyeyya, nissaggiyaṁ pācittiyaṁ.

@ pali-npac-08.tex
Bhikkhuṁ pan’eva uddissa aññātakassa gahapatissa vā gahapatāniyā vā cīvaracetāpanaṁ upakkhaṭaṁ hoti, “Iminā cīvaracetāpanena cīvaraṁ cetāpetvā itthannāmaṁ bhikkhuṁ cīvarena acchādessāmī” ti.

@ pali-npac-08a.tex
Tatra ce so bhikkhu pubbe appavārito upasaṅkamitvā cīvare vikappaṁ āpajjeyya, “Sādhu vata maṁ āyasmā iminā cīvaracetāpanena, evarūpaṁ vā evarūpaṁ vā cīvaraṁ cetāpetvā acchādehī” ti, kalyāṇakamyataṁ upādāya, nissaggiyaṁ pācittiyaṁ.

@ pali-npac-09.tex
Bhikkhuṁ pan’eva uddissa ubhinnaṁ aññātakānaṁ gahapatīnaṁ vā gahapatānīnaṁ vā paccekacīvaracetāpanā upakkhaṭā honti, “Imehi mayaṁ paccekacīvaracetāpanehi paccekacīvarāni cetāpetvā itthannāmaṁ bhikkhuṁ cīvarehi acchādessāmā” ti.

@ pali-npac-09a.tex
Tatra ce so bhikkhu pubbe appavārito upasaṅkamitvā cīvare vikappaṁ āpajjeyya, “Sādhu vata maṁ āyasmanto imehi paccekacīvaracetāpanehi, evarūpaṁ vā evarūpaṁ vā cīvaraṁ cetāpetvā acchādetha ubho’va santā ekenā ” ti, kalyāṇakamyataṁ upādāya, nissaggiyaṁ pācittiyaṁ.

@ pali-npac-10.tex
Bhikkhuṁ pan’eva uddissa rājā vā rājabhoggo vā brāhmaṇo vā gahapatiko vā dūtena cīvaracetāpanaṁ pahiṇeyya, “Iminā cīvaracetāpanena cīvaraṁ cetāpetvā itthannāmaṁ bhikkhuṁ cīvarena acchādehī” ti.

@ pali-npac-10a.tex
So ce dūto taṁ bhikkhuṁ upasaṅkamitvā evaṁ vadeyya, “Idaṁ kho bhante āyasmantaṁ uddissa cīvaracetāpanaṁ ābhataṁ. Paṭiggaṇhātu āyasmā cīvaracetāpanan” ti.

@ pali-npac-10b.tex
Tena bhikkhunā so dūto evam’assa vacanīyo, “Na kho mayaṁ āvuso cīvaracetāpanaṁ paṭiggaṇhāma, cīvarañ ca kho mayaṁ paṭiggaṇhāma kālena kappiyan” ti.

@ pali-npac-10c.tex
So ce dūto taṁ bhikkhuṁ evaṁ vadeyya, “Atthi pan’āyasmato koci veyyāvaccakaro” ti. Cīvar’atthikena bhikkhave bhikkhunā veyyāvaccakaro niddisitabbo, ārāmiko vā upāsako vā, “Eso kho āvuso bhikkhūnaṁ veyyāvaccakaro” ti.

@ pali-npac-10d.tex
So ce dūto taṁ veyyāvaccakaraṁ saññāpetvā taṁ bhikkhuṁ upasaṅkamitvā evaṁ vadeyya, “Yaṁ kho bhante āyasmā veyyāvaccakaraṁ niddisi, saññatto so mayā. Upasaṅkamatu āyasmā kālena cīvarena taṁ acchādessatī” ti. 

@ pali-npac-10e.tex
Cīvar’atthikena bhikkhave bhikkhunā veyyāvaccakaro upasaṅkamitvā dvittikkhattuṁ codetabbo sāretabbo, “Attho me āvuso cīvarenā” ti.

@ pali-npac-10f.tex
Dvittikkhattuṁ codayamāno sārayamāno taṁ cīvaraṁ abhinipphādeyya, icc’etaṁ kusalaṁ. No ce abhinipphādeyya, catukkhattuṁ pañcakkhattuṁ chakkhattuparamaṁ tuṇhībhūtena uddissa ṭhātabbaṁ.

@ pali-npac-10g.tex
Catukkhattuṁ pañcakkhattuṁ chakkhattuparamaṁ tuṇhībhūto uddissa tiṭṭhamāno taṁ cīvaraṁ abhinipphādeyya, icc’etaṁ kusalaṁ. No ce abhinipphādeyya, tato ce uttariṁ vāyamamāno taṁ cīvaraṁ abhinipphādeyya, nissaggiyaṁ pācittiyaṁ.

@ pali-npac-10h.tex
No ce abhinipphādeyya, yatassa cīvaracetāpanaṁ ābhataṁ, tattha sāmaṁ vā gantabbaṁ, dūto vā pāhetabbo,

@ pali-npac-10i.tex
“Yaṁ kho tumhe āyasmanto bhikkhuṁ uddissa cīvaracetāpanaṁ pahiṇittha. Na tantassa bhikkhuno kiñci atthaṁ anubhoti. Yuñjant’āyasmanto sakaṁ. Mā vo sakaṁ vinassī” ti. Ayaṁ tattha sāmīci.

@ pali-center-npac-civaravaggo.tex
Cīvaravaggo paṭhamo.

@ pali-npac-11.tex
Yo pana bhikkhu kosiyamissakaṁ santhataṁ kārāpeyya, nissaggiyaṁ pācittiyaṁ.

@ pali-npac-12.tex
Yo pana bhikkhu suddhakāḷakānaṁ eḷakalomānaṁ santhataṁ kārāpeyya, nissaggiyaṁ pācittiyaṁ.

@ pali-npac-13.tex
Navam’pana bhikkhunā santhataṁ kārayamānena, dve bhāgā suddhakāḷakānaṁ eḷakalomānaṁ ādātabbā, tatiyaṁ odātānaṁ catutthaṁ gocariyānaṁ.

@ pali-npac-13a.tex
Anādā ce bhikkhu dve bhāge suddhakāḷakānaṁ eḷakalomānaṁ, tatiyaṁ odātānaṁ catutthaṁ gocariyānaṁ navaṁ santhataṁ kārāpeyya, nissaggiyaṁ pācittiyaṁ.

@ pali-npac-14.tex
Navam’pana bhikkhunā santhataṁ kārāpetvā chabbassāni dhāretabbaṁ. Orena ce channaṁ vassānaṁ taṁ santhataṁ vissajjetvā vā avissajjetvā vā aññaṁ navaṁ santhataṁ kārāpeyya, aññatra bhikkhusammatiyā, nissaggiyaṁ pācittiyaṁ.

@ pali-npac-15.tex
Nisīdanasanthataṁ pana bhikkhunā kārayamānena purāṇasanthatassa sāmantā sugatavidatthi ādātabbā dubbaṇṇakaraṇāya. Anādā ce bhikkhu purāṇasanthatassa sāmantā sugatavidatthiṁ navaṁ nisīdanasanthataṁ kārāpeyya, nissaggiyaṁ pācittiyaṁ.

@ pali-npac-16.tex
Bhikkhuno pan’eva addhānamaggapaṭipannassa eḷakalomāni uppajjeyyuṁ. Ākaṅkhamānena bhikkhunā paṭiggahetabbāni. Paṭiggahetvā tiyojanaparamaṁ sahatthā hāretabbāni, asante hārake. Tato ce uttariṁ hareyya asante’pi hārake, nissaggiyaṁ pācittiyaṁ.

@ pali-npac-17.tex
Yo pana bhikkhu aññātikāya bhikkhuniyā eḷakalomāni dhovāpeyya vā rajāpeyya vā vijaṭāpeyya vā, nissaggiyaṁ pācittiyaṁ.

@ pali-npac-18.tex
Yo pana bhikkhu jātarūparajataṁ uggaṇheyya vā uggaṇhāpeyya vā upanikkhittaṁ vā sādiyeyya, nissaggiyaṁ pācittiyaṁ.

@ pali-npac-19.tex
Yo pana bhikkhu nānappakārakaṁ rūpiyasaṁvohāraṁ samāpajjeyya, nissaggiyaṁ pācittiyaṁ.

@ pali-npac-20.tex
Yo pana bhikkhu nānappakārakaṁ kayavikkayaṁ samāpajjeyya, nissaggiyaṁ pācittiyaṁ.

@ pali-center-npac-kosiyavaggo.tex
Kosiyavaggo dutiyo

@ pali-npac-21.tex
Dasāhaparamaṁ atirekapatto dhāretabbo. Taṁ atikkāmayato, nissaggiyaṁ pācittiyaṁ.

@ pali-npac-22.tex
Yo pana bhikkhu ūnapañcabandhanena pattena aññaṁ navaṁ pattaṁ cetāpeyya, nissaggiyaṁ pācittiyaṁ.

@ pali-npac-22a.tex
Tena bhikkhunā so patto bhikkhuparisāya nissajjitabbo. Yo ca tassā bhikkhuparisāya pattapariyanto, so ca tassa bhikkhuno padātabbo, “Ayan’te bhikkhu patto, yāva bhedanāya dhāretabbo” ti. Ayaṁ tattha sāmīci.

@ pali-npac-23.tex
Yāni kho pana tāni gilānānaṁ bhikkhūnaṁ paṭisāyanīyāni bhesajjāni, seyyathīdaṁ: sappi navanītaṁ telaṁ madhu phāṇitaṁ; tāni paṭiggahetvā sattāhaparamaṁ sannidhikārakaṁ paribhuñjitabbāni. Taṁ atikkāmayato, nissaggiyaṁ pācittiyaṁ.

@ pali-npac-24.tex
“Māso seso gimhānan” ti bhikkhunā vassikasāṭikacīvaraṁ pariyesitabbaṁ. “Aḍḍhamāso seso gimhānan” ti katvā nivāsetabbaṁ. “Orena ce māso seso gimhānan” ti vassikasāṭikacīvaraṁ pariyeseyya, “Oren’aḍḍhamāso seso gimhānan” ti katvā nivāseyya, nissaggiyaṁ pācittiyaṁ.

@ pali-npac-25.tex
Yo pana bhikkhu bhikkhussa sāmaṁ cīvaraṁ datvā kupito anattamano acchindeyya vā acchindāpeyya vā, nissaggiyaṁ pācittiyaṁ.

@ pali-npac-26.tex
Yo pana bhikkhu sāmaṁ suttaṁ viññāpetvā tantavāyehi cīvaraṁ vāyāpeyya, nissaggiyaṁ pācittiyaṁ.

@ pali-npac-27.tex
Bhikkhuṁ pan’eva uddissa aññātako gahapati vā gahapatānī vā tantavāyehi cīvaraṁ vāyāpeyya.

@ pali-npac-27a.tex
Tatra ce so bhikkhu pubbe appavārito tantavāye upasaṅkamitvā cīvare vikappaṁ āpajjeyya, “Idaṁ kho āvuso cīvaraṁ maṁ uddissa vīyati. Āyatañca karotha vitthatañca appitañca suvītañca supavāyitañca suvilekhitañca suvitacchitañca karotha; app’eva nāma mayam’pi āyasmantānaṁ kiñcimattaṁ anupadajjeyyāmā” ti.

@ pali-npac-27b.tex
Evañca so bhikkhu vatvā kiñcimattaṁ anupadajjeyya, antamaso piṇḍapātamattam’pi, nissaggiyaṁ pācittiyaṁ.

@ pali-npac-28.tex
Dasāhānāgataṁ kattikatemāsipuṇṇamaṁ, bhikkhuno pan’eva accekacīvaraṁ uppajjeyya. Accekaṁ maññamānena bhikkhunā paṭiggahetabbaṁ. Paṭiggahetvā yāva cīvarakālasamayaṁ nikkhipitabbaṁ. Tato ce uttariṁ nikkhipeyya, nissaggiyaṁ pācittiyaṁ.

@ pali-npac-29.tex
Upavassaṁ kho pana kattikapuṇṇamaṁ. Yāni kho pana tāni āraññakāni senāsanāni sāsaṅkasammatāni sappaṭibhayāni, tathārūpesu bhikkhu senāsanesu viharanto, ākaṅkhamāno tiṇṇaṁ cīvarānaṁ aññataraṁ cīvaraṁ antaraghare nikkhipeyya.

@ pali-npac-29a.tex
Siyā ca tassa bhikkhuno kocid’eva paccayo tena cīvarena vippavāsāya, chārattaparaman tena bhikkhunā tena cīvarena vippavasitabbaṁ. Tato ce uttariṁ vippavaseyya, aññatra bhikkhusammatiyā, nissaggiyaṁ pācittiyaṁ.

@ pali-npac-30.tex
Yo pana bhikkhu jānaṁ saṅghikaṁ lābhaṁ pariṇataṁ attano pariṇāmeyya, nissaggiyaṁ pācittiyaṁ.

@ pali-center-npac-pattavaggo.tex
Pattavaggo tatiyo.

@ pali-npac-uddittha.tex
Uddiṭṭhā kho āyasmanto tiṁsa nissaggiyā pācittiyā dhammā. +

@ pali-npac-parisuddha.tex
Tatth’āyasmante pucchāmi: Kacci’ttha parisuddhā? +
Dutiyam’pi pucchāmi: Kacci’ttha parisuddhā? +
Tatiyam’pi pucchāmi: Kacci’ttha parisuddhā? +
Parisuddh’etth’āyasmanto, tasmā tuṇhī, evam’etaṁ dhārayāmi.

@ pali-center-npac-nitthito.tex
Nissaggiyā pācittiyā dhammā niṭṭhitā
