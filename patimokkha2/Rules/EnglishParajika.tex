@ eng-par-intro.tex
Herein these four cases involving disqualification come up for recitation.

@ eng-par-01.tex
If any bhikkhu (who) has entered upon the training and livelihood for bhikkhus, not having rejected the training, not having disclosed (his) incapability, should engage in the act of sexual intercourse, even with just a female animal, he is disqualified, not in communion.

@ eng-par-02.tex
If any bhikkhu should take (what has) not been given from a village or wilderness-area, which is reckoned as theft, (and) the taking of what has not been given (is) of the kind (that) on account of (it) kings, having caught the robber, would physically punish or imprison or banish (him, saying):

@ eng-par-02a.tex
“You are a robber! You are a fool! You are insane! You are a thief!,” a bhikkhu taking (what has) not been given of such a kind, is also disqualified, not in communion.

@ eng-par-03.tex
If any bhikkhu should deliberately deprive a human being of life, or should seek an assassin for him, or should praise the attractiveness of death, or should incite (him) to death (saying):

@ eng-par-03a.tex
“Dear man, what (use) is this bad, wretched life for you? Death is better than life for you!” should he, (having) such-thought-and- mind, (having such-) thought-and-intention, praise in manifold ways the beauty of death or incite (him) to death, he also is disqualified, not in communion.

@ eng-par-04.tex
If any bhikkhu, (though) not directly knowing (it), should claim a superhuman state pertaining to himself, (a state of) knowing and seeing (that is) suitable for the noble (ones), (saying):

@ eng-par-04a.tex
“Thus I know! Thus I see!,” (and) then, on another occasion, (whether) being interrogated or not being interrogated, having committed (the offence), desiring purification, should say so:

@ eng-par-04b.tex
“(Although) not knowing (it,) I spoke thus (saying): `I know,’ not seeing (it, I spoke, saying:) `I see.’ I bluffed vainly (and) falsely,” except (when said) in overestimation, he also is disqualified, not in communion.

@ eng-par-uddittha.tex
Venerables, the four cases involving disqualification have been recited, a bhikkhu who has committed any one of them, does not obtain the communion with bhikkhus. As (he was) before, so (he is) after (committing it): he is one who is disqualified, not in communion.

@ eng-par-parisuddha.tex
Concerning that I ask the Venerables: (Are you) pure in this? +
A second time again I ask: (Are you) pure in this? +
A third time again I ask: (Are you) pure in this? +
The venerables are pure in this, therefore there is silence, so do I bear this (in mind).

@ eng-center-par-nitthito.tex
The recitation of the (cases involving) disqualification is finished
