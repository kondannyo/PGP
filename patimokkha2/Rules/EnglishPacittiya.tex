@ eng-pac-intro.tex
Venerables, these ninety-two cases involving expiation come up for recitation.

@ eng-pac-01.tex
In deliberate false speech, (there is a case) involving expiation.

@ eng-pac-02.tex
In abusive speech, (there is a case) involving expiation.

@ eng-pac-03.tex
In the backbiting of a bhikkhu, (there is a case) involving expiation.

@ eng-pac-04.tex
If any bhikkhu should have one who has not been fully admitted (into the community) recite the Dhamma (line) by line, (this is a case) involving expiation.

@ eng-pac-05.tex
If any bhikkhu should make use of a sleeping place for more than two nights or three nights together with one who has not been fully admitted (into the bhikkhu-community), (this is a case) involving expiation.

@ eng-pac-06.tex
If any bhikkhu should make use of a sleeping place together with a woman, (this is a case) involving expiation.

@ eng-pac-07.tex
If any bhikkhu should teach the Dhamma to a woman by (means of) more than five or six sentences, except (when being together) with a discerning male human being, (this is a case) involving expiation.

@ eng-pac-08.tex
If any bhikkhu should declare a superhuman state to one who has not been fully admitted (into the bhikkhu-community), (even) when it is a fact, (this is a case) involving expiation.

@ eng-pac-09.tex
If any bhikkhu should declare the depraved offence of (another) bhikkhu to one who has not been fully admitted (into the bhikkhu-community), except with the authorisation of bhikkhus, (this is a case) involving expiation.

@ eng-pac-10.tex
If any bhikkhu should dig the earth or should have it dug, (this is a case) involving expiation.

@ eng-center-pac-musavadavaggo.tex
The section (starting with the rule) on false speech is first.

@ eng-pac-11.tex
In the destroying of vegetation, (there is a case) involving expiation.

@ eng-pac-12.tex
In evading, in vexing, (there is a case) involving expiation.

@ eng-pac-13.tex
In making (another bhikkhu) find fault, in criticising, (there is a case) involving expiation.

@ eng-pac-14.tex
If any bhikkhu, having (himself) put out or after having (someone else) put out in the open air, a bed or seat or mattress or stool belonging to the community, (and) then, when departing, should not take (it) away or should not have (it) taken away or should go without asking (someone to put it back), (this is a case) involving expiation.

@ eng-pac-15.tex
If any bhikkhu, having (himself) put out or having (someone else) put out, bedding in a dwelling belonging to the community, (and) then, when departing, should not take (it) away or should not have (it) taken away, or should go without asking (someone to put it back), (this is a case) involving expiation.

@ eng-pac-16.tex
If any bhikkhu, having encroached upon a bhikkhu who has arrived before, should knowingly use a sleeping place in a dwelling belonging to the community (saying): “He for whom it is (too) cramped, will leave,” having done (it) for just this reason, (and) not another, (this is a case) involving expiation.

@ eng-pac-17.tex
If any bhikkhu, being resentful and displeased, should drive out a bhikkhu or have (him) driven out from a dwelling belonging to the community, (this is a case) involving expiation.

@ eng-pac-18.tex
If any bhikkhu should (brusquely) sit down or lie down on a bed or seat with detachable legs in a hut with an upper-floor in a dwelling belonging to the community, (this is a case) involving expiation.

@ eng-pac-19.tex
By a bhikkhu who is having a large dwelling built, a surrounding-layer of two or three coverings can be ordered, by (a bhikku) standing on (a place which has) few crops, upto the frame of the door for (the purpose of) fixing the bolt, (and) for surrounding the window. If he should order more than that, even (when) standing on (a place which has) few crops, (this is a case) involving expiation.

@ eng-pac-20.tex
If any bhikkhu should knowingly pour out, or should have (someone else) pour out, water containing living beings on grass or clay, (this is a case) involving expiation.

@ eng-center-pac-bhutagamavaggo.tex
The section (starting with the rule) on vegetation is second.

@ eng-pac-21.tex
If any bhikkhu who has not been authorised should exhort the bhikkhunīs, (this is a case) involving expiation.

@ eng-pac-22.tex
Even if a bhikkhu who has been authorised should exhort the bhikkhunīs after the sun has set, (this is a case) involving expiation.

@ eng-pac-23.tex
If any bhikkhu, having approached the bhikkhunī-quarters, should exhort the bhikkhunīs, except at the (right) occasion, (this is a case) involving expiation.

@ eng-pac-24.tex
If any bhikkhu should say so: “The bhikkhus exhort bhikkhunīs for the sake of reward,” (this is a case) involving expiation.

@ eng-pac-25.tex
If any bhikkhu should give a robe (-cloth) to an unrelated bhikkhunī, except in an exchange, (this is a case) involving expiation.

@ eng-pac-26.tex
If any bhikkhu should sew a robe or should have a robe sewn for an unrelated bhikkhunī, (this is a case) involving expiation.

@ eng-pac-27.tex
If any bhikkhu, having made an arrangement, should travel together with a bhikkhunī on the same main road, even (if) just the distance between villages, except at the (right) occasion, (this is a case) involving expiation.

@ eng-pac-27a.tex
Here the occasion is this: the road, which is considered risky (and) which is dangerous, has to be gone 
with a company (of other travellers), this is the occasion here.

@ eng-pac-28.tex
If any bhikkhu, having made an arrangement, should embark (on a voyage) together with a bhikkhunī on the same boat, which is going up (-stream) or which is going down (-stream), except with (a boat which is)crossing over (a river), (this is a case) involving expiation.

@ eng-pac-29.tex
If any bhikkhu should knowingly eat alms-food which a bhikkhunī has caused to be prepared, except through previous arrangement of householders, (this is a case) involving expiation.

@ eng-pac-30.tex
If any bhikkhu should sit down together with a bhikkhunī, privately, one (man) with one (woman), (this is a case) involving expiation.

@ eng-center-pac-ovadavaggo.tex
The section (starting with the rule) on exhortation is third.

@ eng-pac-31.tex
By a bhikkhu who is not ill one alms-meal in a resthouse can be eaten; if he should eat more than that, (this is a case) involving expiation.

@ eng-pac-32.tex
In eating (a meal) in a group, except at the (right) occasion, (there is a case) involving expiation.

@ eng-pac-32a.tex
Here the occasion is this: the occasion of illness; the occasion of a giving of robe (-cloth)s; the occasion of a robe-making; the occasion of going on a (long) journey; the occasion of voyaging on a boat; the occasion of a great (gathering); the occasion of a meal (made) by an ascetic; this is the occasion here.

@ eng-pac-33.tex
In (taking) a meal before another (invitation-meal), except at the (right) occasion, (there is a case) involving expiation.

@ eng-pac-34.tex
Now, should a family invite a bhikkhu who has approached to take as many cakes and parched cakes (as he likes), by a bhikkhu who is wishing (so) two or three bowls full (of cakes) can be accepted; if he should accept more than that, (this is a case) involving expiation.

@ eng-pac-34a.tex
Having accepted two or three bowls full, having taken (them) away from there, (it) is to be shared 
together with (other) bhikkhus. This is the proper procedure here.

@ eng-pac-35.tex
If any bhikkhu who has eaten (a meal), who has been invited (to take more and refused), should chew uncooked food or eat cooked food which is not left over, (this is a case) involving expiation.

@ eng-pac-36.tex
If any bhikkhu, knowingly (and) desiring to cause offence, should invite a bhikkhu, who has eaten (a meal and) who has been invited (to take more), to take uncooked food or cooked food which is not left over (saying): “Here, bhikkhu, chew and eat!,” when (the bhikkhu) has eaten, (this is a case) involving expiation.

@ eng-pac-37.tex
If any bhikkhu should chew uncooked food or eat cooked food at the wrong time, (this is a case) involving expiation.

@ eng-pac-38.tex
If any bhikkhu should chew uncooked food or eat cooked food (while) keeping (it) in store, (this is a case) involving expiation.

@ eng-pac-39.tex
Those foods which are superior, namely: ghee, butter, oil, honey and molasses, fish, meat, milk, curd; whichever bhikkhu, who is not ill, having requested such superior foods for his own benefit, should eat (them), (this is a case) involving expiation.

@ eng-pac-40.tex
If any bhikkhu should take into the mouth (any) nutriment that has not been given (to bhikkhus); except water and tooth-wood, (this is a case) involving expiation.

@ eng-center-pac-bhojanavaggo.tex
The section (starting with the rule) on eating is fourth

@ eng-pac-41.tex
If any bhikkhu should give with his own hand uncooked food or cooked food to a naked ascetic or to a male wanderer or to a female wanderer, (this is a case) involving expiation.

@ eng-pac-42.tex
If any bhikkhu should say so to a bhikkhu, “Come friend! We shall enter a village or town for alms,” (then after) having had (food) given or not having had (food) given to him, should he dismiss (the bhikkhu saying),

@ eng-pac-42a.tex
“Go friend! There is no ease for me talking or sitting down together with you; there is ease for me talking or sitting down by myself;” having made just this the reason, (and) not another, (this is a case) involving expiation.

@ eng-pac-43.tex
If any bhikkhu, having intruded upon an family having a meal, should sit down, (this is a case) involving expiation.

@ eng-pac-44.tex
.If any bhikkhu should sit down together with a woman, privately, on a concealed seat, (this is a case) involving expiation.

@ eng-pac-45.tex
If any bhikkhu sit down together with a woman, one (man) with one (woman), privately, (this is a case) involving expiation.

@ eng-pac-46.tex
If any bhikkhu who has been invited for a meal, not having asked (permission to) a bhikkhu who is present (in the monastery), should go visiting families before the meal or after the meal, except at the (right) occasion, (this is a case) involving expiation.

@ eng-pac-46a.tex
Here the occasion is this: the occasion of a giving of robe (-cloth)s; the occasion of a making of robes; this is the occasion here.

@ eng-pac-47.tex
By a bhikkhu who is not ill a four-month invitation for requisites can be accepted; except with a repeated invitation, except with a permanent invitation; if he should accept more than that, (this is a case) involving expiation.

@ eng-pac-48.tex
If any bhikkhu should should go to visit an army in action; except with an appropriate reason, (this is a case) involving expiation.

@ eng-pac-49.tex
And if there might be any reason for that bhikkhu for going to the army, two nights or three nights can be stayed within the army by that bhikkhu; if he should stay more than that, (this is a case) involving expiation.

@ eng-pac-50.tex
If a bhikkhu staying two nights or three nights within an army should go to a battle-field, or a review, or a massing of the army, or an inspection of units, (this is a case) involving expiation.

@ eng-center-pac-acelakavaggo.tex
The section (starting with the rule) on naked ascetics is fifth

@ eng-pac-51.tex
In drinking alcoholic drink made of grain (-products) or fruit (and/or flower products), (there is a case) involving expiation.

@ eng-pac-52.tex
In tickling with the fingers, (there is a case) involving expiation.

@ eng-pac-53.tex
In the act of playing in water, (there is a case) involving expiation.

@ eng-pac-54.tex
In disrespect, (there is a case) involving expiation.

@ eng-pac-55.tex
If any bhikkhu should scare (another) bhikkhu, (this is a case) involving expiation.

@ eng-pac-56.tex
If any bhikkhu who is not ill, desiring to warm (himself), should light a fire or should have (it) lit, except with an appropriate reason, (this is a case) involving expiation.

@ eng-pac-57.tex
If any bhikkhu should should bathe within less than half a month, except at the (right) occasion, (this is a case) involving expiation.

@ eng-pac-57a.tex
Here the occasion is this (thinking): “one and a half month is what remains of the hot season,” and “this is the first month of the rainy season”—these two and a half months (are) the occasion of dry heat, (and) the occasion of humid heat—(also:) the occasion of being sick; the occasion of work; the occasion of going on a journey; the occasion of (dusty) wind and rain; this is the occasion here.

@ eng-pac-58.tex
By a monk with the gain of a new robe a certain stain (from) amongst the three stains is to be applied: dark-blue or muddy (-grey) or dark-brown. If a bhikkhu, not having applied a certain stain (from) amongst the three stains, should use a new robe, (this is a case) involving expiation.

@ eng-pac-59.tex
If any bhikkhu, having himself assigned a robe to a bhikkhu or a bhikkhunī or a male novice or a female novice, should use (it) without withdrawing (the assignment), (this is a case) involving expiation.

@ eng-pac-60.tex
If any bhikkhu should hide a bhikkhu's bowl or robe or sitting-cloth or needle case or body-belt, or have (it) hidden, even if just desiring amusement, (this is a case) involving expiation.

@ eng-center-pac-surapanavaggo.tex
The section (starting with the rule) on alcoholic drink is sixth.

@ eng-pac-61.tex
If any bhikkhu should intentionally deprive a living being of life, (this is a case) involving expiation.

@ eng-pac-62.tex
If any bhikkhu should knowingly use water containing living beings, (this is a case) involving expiation.

@ eng-pac-63.tex
If any bhikkhu should knowingly agitate for further (legal) action a legal issue which has been disposed of according to the law, (this is a case) involving expiation.

@ eng-pac-64.tex
If any bhikkhu should knowingly have a person who is less than twenty years (old) fully admitted (into the bhikkhu-community), then that person is one who has not been fully admitted and those bhikkhus are blameworthy. Because of that, this (is a case) involving expiation.

@ eng-pac-65.tex
If any bhikkhu should knowingly have a person who is less than twenty years (old) fully admitted (into the bhikkhu-community), then that person is one who has not been fully admitted and those bhikkhus are blameworthy. Because of that, this (is a case) involving expiation.

@ eng-pac-66.tex
If any bhikkhu, having made an arrangement, should knowingly travel together on the same main road with a company of thieves, even (if) just the distance between villages, (this is a case) involving expiation.

@ eng-pac-67.tex
If any bhikkhu, having made an arrangement, should travel together with a woman on the same main road, even (if) just the distance between villages, (this is a case) involving expiation.

@ eng-pac-68.tex
If any bhikkhu should say so, “As I understand the Teaching taught by the Fortunate One, these obstructive acts which are spoken of by the Fortunate One: they are not enough to be an obstruction for the one who is being engaged in (them),” 

@ eng-pac-68a.tex
(then) that bhikkhu is to be spoken to thus by the bhikkhus: “Venerable, don't say so! Don't misrepresent the Fortunate One; for the misrepresentation of the Fortunate One is not good; for the Fortunate One would not say so; friend, (that) obstructive acts are (really) obstructive is spoken of in manifold ways by the Fortunate One and they are enough to be an obstruction for the one who is being engaged in (them),” 

@ eng-pac-68b.tex
and (if) that bhikkhu being spoken to thus by the bhikkhus should persist in the same way (as before), (then) that bhikkhu is to be argued with up to three times by the bhikkhus for the relinquishing of that (view), (and if that bhikkhu,) being argued with up to three times, should relinquish that (view), then this is good, (but) if he should not relinquish (it): (this is a case) involving expiation.

@ eng-pac-69.tex
If any bhikkhu knowingly should eat together with, or should live together with, or should use a sleeping place together with a bhikkhu who is speaking thus, who has not performed the normal procedure, who has not relinquished that view, (this is a case) involving expiation.

@ eng-pac-70.tex
If a novice should say so too, “As I understand the Teaching taught by the Fortunate One, these obstructive acts which are spoken of by the Fortunate One: they are not enough to be an obstruction for the one who is being engaged in (them),”

@ eng-pac-70a.tex
(then) that novice is to be spoken to thus by the bhikkhus, “Friend novice, don't say so! Don't misrepresent the Fortunate One; for the misrepresentation of the Fortunate One is not good; for the Fortunate One would not say so; friend novice, (that) obstructive acts are (really) obstructive is spoken of in manifold ways by the Fortunate One and they are enough to be an obstruction for the one who is engaging (in them),”

@ eng-pac-70b.tex
and if that novice being spoken to thus by the bhikkhus should persist in the same way (as before), (then) that novice is to be spoken to thus by the bhikkhus, “From today on, friend novice, the Fortunate One is not to be referred to as the teacher by you, and also the two or three nights sleeping together (in one room) with bhikkhus that other novices get, that too is not for you. Go away, disappear!”

@ eng-pac-70c.tex
If any bhikkhu knowingly should treat kindly such an expelled novice, or should make (him) attend (to himself), or should eat together with (him), or should use a sleeping place together with (him), (this is a case) involving expiation.

@ eng-center-pac-sappanavaggo.tex
The section (starting with the rule) on living beings is seventh

@ eng-pac-71.tex
If any bhikkhu when being righteously spoken to by bhikkhus should say so, “Friends, I shall not train in this training precept for as long as I can not question another bhikkhu (about it) who is a learned memoriser of the discipline,” (this is a case) involving expiation.

@ eng-pac-71a.tex
Bhikkhus, (the training precept) is to be understood, is to be questioned about, is to be investigated by a bhikkhu who is training (in it). This is the proper procedure here.

@ eng-pac-72.tex
If any bhikkhu, when the Disciplinary Code is being recited, should say so, “But why these small and minute training precepts that are recited? They just lead to worry, annoyance, (and) discomfort.” In the disparaging of training precepts, (there is a case) involving expiation.

@ eng-pac-73.tex
If any bhikkhu when the Disciplinary Code is being recited half-monthly should say so, “Only now I know! This too, indeed, is a case which has been handed down in the Sutta, which has been included in the Sutta, which comes up for recitation half-monthly!”

@ eng-pac-73a.tex
(and) if other bhikkhus should know (about) that bhikkhu (thus), “This bhikkhu has sat (in) two or three times previously when the Disciplinary Code was being recited. What to say about more (times than that)!”

@ eng-pac-73b.tex
(then) there is no release for that bhikkhu through not-knowing, and whatever the offence is that he has committed there, he is to be made to do according to that case and moreover his deluding is to be exposed,

@ eng-pac-73c.tex
“Because of that (there are) losses for you, because of that (it) has been ill-gained by you, that you, when the Disciplinary Code is being recited, do not take (it) to mind (after) having focussed carefully (on it).” Because of that deluding, this (is a case) involving expiation.

@ eng-pac-74.tex
If any bhikkhu who is resentful (and) displeased should give a blow to a bhikkhu, (this is a case) involving expiation.

@ eng-pac-75.tex
If any bhikkhu should brandish the palm of the hand (threateningly) like (one holds) a dagger to a bhikkhu, (this is a case) involving expiation.

@ eng-pac-76.tex
If any bhikkhu should should accuse a bhikkhu with a groundless (case concerning) the community in the beginning and in the rest (of the procedure), (this is a case) involving expiation.

@ eng-pac-77.tex
If any bhikkhu should deliberately provoke worry for a bhikkhu (thinking), “Thus there will be discomfort for him, even (if only) for a short time,” having made just this the reason, (and) not another, (this is a case) involving expiation.

@ eng-pac-78.tex
If any bhikkhu should stand overhearing bhikkhus who are arguing, who are quarrelling, who are engaged in dispute (thinking), “I shall hear what these ones will say,” having made just this the reason, (and) not another, (this is a case) involving expiation.

@ eng-pac-79.tex
If any bhikkhu, having given consent to legitimate (legal) actions, should afterwards engage in the act of criticising, (this is a case) involving expiation.

@ eng-pac-80.tex
If any bhikkhu, when investigatory discussion is going on in the community, not having given (his) consent, having gotten up from (his) seat, should depart, (this is a case) involving expiation.

@ eng-pac-81.tex
If any bhikkhu, having given a robe (-cloth) (together) with a united community, should afterwards engage in criticising (saying): “The bhikkhus allocate communal gain according to familiarity,” (this is a case) involving expiation.

@ eng-pac-82.tex
If any bhikkhu should knowingly allocate (already) allocated communal gain to a (lay-) person, (this is a case) involving expiation.

@ eng-center-pac-sahadhammikavaggo.tex
The section (starting with the rule) about (being spoken to) righteously is eighth.

@ eng-pac-83.tex
If any bhikkhu, without having been announced beforehand, should go beyond the boundary post of a noble consecrated king's (bed-room) when the king has not departed, (and) the (queen-) jewel has not withdrawn, (this is a case) involving expiation.

@ eng-pac-84.tex
If any bhikkhu should pick up, or should make (someone else) pick up, a treasure or what is considered a treasure, except within a monastery or within a dwelling, (this is a case) involving expiation.

@ eng-pac-84a.tex
However, by a bhikkhu having picked up, or having had picked up, a treasure or what is considered a treasure within a monastery or within a dwelling, (it) is to be put aside (thinking): “He to whom it belongs will take it.” This is the proper procedure here.

@ eng-pac-85.tex
If any bhikkhu, not having asked (permission of) a bhikkhu who is present, should enter a village at the wrong time, except with an appropriate urgent duty, (this is a case) involving expiation.

@ eng-pac-86.tex
If any bhikkhu should have a needle-case made, which is made of bone, or made of ivory, or made of horn, (this is a case) involving expiation with breaking up (the needle-case).

@ eng-pac-87.tex
By a bhikkhu who is having a new bed or seat made, (a bed or seat) which has legs of eight finger-breadths is to be made, according to the Sugata-finger-breadth, except the lowermost (edge of the) frame. For one who lets it exceed (this measure), (this is a case) involving expiation with cutting (down the legs).

@ eng-pac-88.tex
If any bhikkhu should have a bed or seat covered with cotton made, (this is a case) involving expiation with tearing off (the cotton).

@ eng-pac-89.tex
By a bhikkhu who is having a sitting-cloth made, (a sitting-cloth) which has the (proper) measure is to be made. This measure here is: two spans of the sugata-span in length, one and a half across, (and) the border is a span. For one who lets it exceed (the measure), (this is a case) involving expiation with cutting (off the cloth).

@ eng-pac-90.tex
By a bhikkhu who is having an itch-covering (-cloth) made, (an itch-covering) which has the (proper) measure is to be made. This measure here is: four spans of the Sugata-span in length, two spans across. For one who lets it exceed (the measure), (this is a case) involving expiation with cutting off the cloth).

@ eng-pac-91.tex
By a bhikkhu who is having a rain's bathing-cloth made, (a bathing-cloth) which has the (proper) measure is to be made. This measure here is: six spans of the sugata-span in length, two and a half across. For one who lets it exceed (the measure), (this is a case) involving expiation with cutting (off the cloth).

@ eng-pac-92.tex
If any bhikkhu should have a robe made which has the sugata-robe measure or (one) which is more (than that), (this is a case) involving expiation with cutting (off the robe).

@ eng-pac-92a.tex
This is the Sugata's sugata-robe measure here: nine spans of the sugata-span in length, six spans across. This is the Sugata's sugata-robe measure.

@ eng-center-pac-ratanavaggo.tex
The section (starting with the rule) on kings is ninth.

@ eng-pac-uddittha.tex
Venerables, the ninety-two cases involving expiation have been recited.

@ eng-pac-parisuddha.tex
Concerning that I ask the Venerables: (Are you) pure in this? +
A second time again I ask: (Are you) pure in this? +
A third time again I ask: (Are you) pure in this? +
The venerables are pure in this, therefore there is silence, thus I keep this (in mind).

@ eng-center-pac-nitthito.tex
The (cases) involving expiation are finished.
