@ pali-pac-intro.tex
Ime kho pan’āyasmanto dvenavuti pācittiyā dhammā uddesaṁ āgacchanti.

@ pali-pac-01.tex
Sampajānamusāvāde pācittiyaṁ.

@ pali-pac-02.tex
Omasavāde pācittiyaṁ.

@ pali-pac-03.tex
Bhikkhupesuññe pācittiyaṁ.

@ pali-pac-04.tex
Yo pana bhikkhu anupasampannaṁ padaso dhammaṁ vāceyya, pācittiyaṁ.

@ pali-pac-05.tex
Yo pana bhikkhu anupasampannena uttaridvirattatirattaṁ sahaseyyaṁ kappeyya, pācittiyaṁ.

@ pali-pac-06.tex
Yo pana bhikkhu mātugāmena sahaseyyaṁ kappeyya, pācittiyaṁ.

@ pali-pac-07.tex
Yo pana bhikkhu mātugāmassa uttarichappañcavācāhi dhammaṁ deseyya, aññatra viññunā purisaviggahena, pācittiyaṁ.

@ pali-pac-08.tex
Yo pana bhikkhu anupasampannassa uttarimanussadhammaṁ āroceyya, bhūtasmiṁ pācittiyaṁ.

@ pali-pac-09.tex
Yo pana bhikkhu bhikkhussa duṭṭhullaṁ āpattiṁ anupasampannassa āroceyya aññatra bhikkhusammatiyā, pācittiyaṁ.

@ pali-pac-10.tex
Yo pana bhikkhu paṭhaviṁ khaṇeyya vā khaṇāpeyya vā, pācittiyaṁ.

@ pali-center-pac-musavadavaggo.tex
Musāvādavaggo Paṭhamo.

@ pali-pac-11.tex
Bhūtagāmapātabyatāya pācittiyaṁ.

@ pali-pac-12.tex
Aññavādake vihesake pācittiyaṁ.

@ pali-pac-13.tex
Ujjhāpanake khiyyanake pācittiyaṁ.

@ pali-pac-14.tex
Yo pana bhikkhu saṅghikaṁ mañcaṁ vā pīṭhaṁ vā bhisiṁ vā kocchaṁ vā ajjhokāse santharitvā vā santharāpetvā vā, taṁ pakkamanto n’eva uddhareyya na uddharāpeyya, anāpucchaṁ vā gaccheyya, pācittiyaṁ.

@ pali-pac-15.tex
Yo pana bhikkhu saṅghike vihāre seyyaṁ santharitvā vā santharāpetvā vā, taṁ pakkamanto n’eva uddhareyya na uddharāpeyya, anāpucchaṁ vā gaccheyya, pācittiyaṁ.

@ pali-pac-16.tex
Yo pana bhikkhu saṅghike vihāre jānaṁ pubbūpagataṁ bhikkhuṁ anūpakhajja seyyaṁ kappeyya, “Yassa sambādho bhavissati, so pakkamissatī” ti. Etad’eva paccayaṁ karitvā anaññaṁ, pācittiyaṁ.

@ pali-pac-17.tex
Yo pana bhikkhu bhikkhuṁ kupito anattamano saṅghikā vihārā nikkaḍḍheyya vā nikkaḍḍhāpeyya vā, pācittiyaṁ.

@ pali-pac-18.tex
Yo pana bhikkhu saṅghike vihāre uparivehāsakuṭiyā āhaccapādakaṁ mañcaṁ vā pīṭhaṁ vā abhinisīdeyya vā abhinipajjeyya vā, pācittiyaṁ.

@ pali-pac-19.tex
Mahallakaṁ pana bhikkhunā vihāraṁ kārayamānena, yāva dvārakosā aggalaṭṭhapanāya, ālokasandhiparikammāya, dvitticchadanassa pariyāyaṁ, appaharite ṭhitena adhiṭṭhātabbaṁ. Tato ce uttariṁ appaharite’pi ṭhito adhiṭṭhaheyya, pācittiyaṁ.

@ pali-pac-20.tex
Yo pana bhikkhu jānaṁ sappāṇakaṁ udakaṁ tiṇaṁ vā mattikaṁ vā siñceyya vā siñcāpeyya vā, pācittiyaṁ.

@ pali-center-pac-bhutagamavaggo.tex
Bhūtagāmavaggo Dutiyo.

@ pali-pac-21.tex
Yo pana bhikkhu asammato bhikkhuniyo ovadeyya, pācittiyaṁ.

@ pali-pac-22.tex
Sammato’pi ce bhikkhu atthaṅgate suriye bhikkhuniyo ovadeyya, pācittiyaṁ.

@ pali-pac-23.tex
Yo pana bhikkhu bhikkhunūpassayaṁ upasaṅkamitvā bhikkhuniyo ovadeyya aññatra samayā, pācittiyaṁ. Tatthāyaṁ samayo: gilānā hoti bhikkhunī. Ayaṁ tattha samayo.

@ pali-pac-24.tex
Yo pana bhikkhu evaṁ vadeyya, “āmisahetu bhikkhū bhikkhuniyo ovadantī” ti, pācittiyaṁ

@ pali-pac-25.tex
Yo pana bhikkhu aññātikāya bhikkhuniyā cīvaraṁ dadeyya, aññatra pārivaṭṭakā, pācittiyaṁ.

@ pali-pac-26.tex
Yo pana bhikkhu aññātikāya bhikkhuniyā cīvaraṁ sibbeyya vā sibbāpeyya vā, pācittiyaṁ.

@ pali-pac-27.tex
Yo pana bhikkhu bhikkhuniyā saddhiṁ saṁvidhāya ekaddhānamaggaṁ paṭipajjeyya, antamaso gām’antaram’pi aññatra samayā, pācittiyaṁ. 

@ pali-pac-27a.tex
Tatthāyaṁ samayo: satthagamanīyo hoti maggo sāsaṅkasammato sappaṭibhayo. Ayaṁ tattha samayo.

@ pali-pac-28.tex
Yo pana bhikkhu bhikkhuniyā saddhiṁ saṁvidhāya ekaṁ nāvaṁ abhirūheyya, uddhagāminiṁ vā adhogāminiṁ vā, aññatra tiriy’antaraṇāya, pācittiyaṁ.

@ pali-pac-29.tex
Yo pana bhikkhu jānaṁ bhikkhunīparipācitaṁ piṇḍapātaṁ bhuñjeyya, aññatra pubbe gihisamārambhā, pācittiyaṁ.

@ pali-pac-30.tex
Yo pana bhikkhu bhikkhuniyā saddhiṁ eko ekāya raho nisajjaṁ kappeyya, pācittiyaṁ.

@ pali-center-pac-ovadavaggo.tex
Ovādavaggo Tatiyo.

@ pali-pac-31.tex
Agilānena bhikkhunā eko āvasathapiṇḍo bhuñjitabbo. Tato ce uttariṁ bhuñjeyya, pācittiyaṁ.

@ pali-pac-32.tex
Gaṇabhojane aññatra samayā, pācittiyaṁ.

@ pali-pac-32a.tex
Tatthāyaṁ samayo: gilānasamayo, cīvaradānasamayo, cīvarakārasamayo, addhānagamanasamayo, nāvābhirūhanasamayo, mahāsamayo, samaṇabhattasamayo. Ayaṁ tattha samayo.

@ pali-pac-33.tex
Paramparabhojane aññatra samayā, pācittiyaṁ. Tatthāyaṁ samayo: gilānasamayo, cīvaradānasamayo, cīvarakārasamayo. Ayaṁ tattha samayo.

@ pali-pac-34.tex
Bhikkhuṁ pan’eva kulaṁ upagataṁ pūvehi vā manthehi vā abhihaṭṭhumpavāreyya, ākaṅkhamānena bhikkhunā dvittipattapūrā paṭiggahetabbā. Tato ce uttariṁ paṭiggaṇheyya, pācittiyaṁ.

@ pali-pac-34a.tex
Dvittipattapūre paṭiggahetvā tato nīharitvā bhikkhūhi saddhiṁ saṁvibhajitabbaṁ. Ayaṁ tattha sāmīci.

@ pali-pac-35.tex
Yo pana bhikkhu bhuttāvī pavārito anatirittaṁ khādanīyaṁ vā bhojanīyaṁ vā khādeyya vā bhuñjeyya vā, pācittiyaṁ.

@ pali-pac-36.tex
Yo pana bhikkhu bhikkhuṁ bhuttāviṁ pavāritaṁ anatirittena khādanīyena vā bhojanīyena vā abhihaṭṭhumpavāreyya, “Handa bhikkhu khāda vā bhuñja vā” ti, jānaṁ āsādan’āpekkho, bhuttasmiṁ pācittiyaṁ.

@ pali-pac-37.tex
Yo pana bhikkhu vikāle khādanīyaṁ vā bhojanīyaṁ vā khādeyya vā bhuñjeyya vā, pācittiyaṁ.

@ pali-pac-38.tex
Yo pana bhikkhu sannidhikārakaṁ khādanīyaṁ vā bhojanīyaṁ vā khādeyya vā bhuñjeyya vā, pācittiyaṁ.

@ pali-pac-39.tex
Yāni kho pana tāni paṇītabhojanāni, seyyathīdaṁ: sappi navanītaṁ telaṁ madhu phāṇitaṁ, maccho maṁsaṁ khīraṁ dadhi. Yo pana bhikkhu evarūpāni paṇītabhojanāni agilāno attano atthāya viññāpetvā bhuñjeyya, pācittiyaṁ.

@ pali-pac-40.tex
Yo pana bhikkhu adinnaṁ mukhadvāraṁ āhāraṁ āhareyya,
aññatra udakadantapoṇā, pācittiyaṁ.

@ pali-center-pac-bhojanavaggo.tex
Bhojanavaggo Catuttho.

@ pali-pac-41.tex
Yo pana bhikkhu acelakassa vā paribbājakassa vā paribbājikāya vā sahatthā khādanīyaṁ vā bhojanīyaṁ vā dadeyya, pācittiyaṁ.

@ pali-pac-42.tex
Yo pana bhikkhu bhikkhuṁ evaṁ vadeyya: “Eh’āvuso gāmaṁ vā nigamaṁ vā piṇḍāya pavisissāmā” ti. Tassa dāpetvā vā adāpetvā vā uyyojeyya,

@ pali-pac-42a.tex
“Gacch’āvuso. Na me tayā saddhiṁ kathā vā nisajjā vā phāsu hoti. Ekakassa me kathā vā nisajjā vā phāsu hotī” ti. Etad’eva paccayaṁ karitvā anaññaṁ, pācittiyaṁ.

@ pali-pac-43.tex
Yo pana bhikkhu sabhojane kule anūpakhajja nisajjaṁ kappeyya, pācittiyaṁ.

@ pali-pac-44.tex
Yo pana bhikkhu mātugāmena saddhiṁ raho paṭicchanne āsane nisajjaṁ kappeyya, pācittiyaṁ.

@ pali-pac-45.tex
Yo pana bhikkhu mātugāmena saddhiṁ eko ekāya raho nisajjaṁ kappeyya, pācittiyaṁ.

@ pali-pac-46.tex
Yo pana bhikkhu nimantito sabhatto samāno santaṁ bhikkhuṁ anāpucchā purebhattaṁ vā pacchābhattaṁ vā kulesu cārittaṁ āpajjeyya aññatra samayā, pācittiyaṁ.

@ pali-pac-46a.tex
Tatthāyaṁ samayo: cīvaradānasamayo, cīvarakārasamayo. Ayaṁ tattha samayo.

@ pali-pac-47.tex
Agilānena bhikkhunā cātumāsapaccayapavāraṇā sāditabbā, aññatra punapavāraṇāya, aññatra niccapavāraṇāya. Tato ce uttariṁ sādiyeyya, pācittiyaṁ.

@ pali-pac-48.tex
Yo pana bhikkhu uyyuttaṁ senaṁ dassanāya gaccheyya, aññatra tathārūpapaccayā, pācittiyaṁ.

@ pali-pac-49.tex
Siyā ca tassa bhikkhuno kocid’eva paccayo senaṁ gamanāya, dvirattatirattaṁ tena bhikkhunā senāya vasitabbaṁ. Tato ce uttariṁ vaseyya, pācittiyaṁ.

@ pali-pac-50.tex
Dvirattatirattañce bhikkhu senāya vasamāno, uyyodhikaṁ vā balaggaṁ vā senābyūhaṁ vā anīkadassanaṁ vā gaccheyya, pācittiyaṁ.

@ pali-center-pac-acelakavaggo.tex
Acelakavaggo Pañcamo.

@ pali-pac-51.tex
Surāmerayapāne pācittiyaṁ.

@ pali-pac-52.tex
Aṅgulipatodake pācittiyaṁ.

@ pali-pac-53.tex
Udake hassadhamme pācittiyaṁ.

@ pali-pac-54.tex
Anādariye pācittiyaṁ.

@ pali-pac-55.tex
Yo pana bhikkhu bhikkhuṁ bhiṁsāpeyya, pācittiyaṁ.

@ pali-pac-56.tex
Yo pana bhikkhu agilāno visīvan’āpekkho, jotiṁ samādaheyya vā samādahāpeyya vā, aññatra tathārūpapaccayā, pācittiyaṁ.

@ pali-pac-57.tex
Yo pana bhikkhu oren’aḍḍhamāsaṁ nhāyeyya, aññatra samayā, pācittiyaṁ.

@ pali-pac-57a.tex
tatthāyaṁ samayo: “Diyaḍḍho māso seso gimhānan” ti, vassānassa paṭhamo māso, icc’ete aḍḍhateyyamāsā; uṇhasamayo, pariḷāhasamayo, gilānasamayo, kammasamayo, addhānagamanasamayo, vātavuṭṭhisamayo. Ayaṁ tattha samayo.

@ pali-pac-58.tex
Navam’pana bhikkhunā cīvaralābhena tiṇṇaṁ dubbaṇṇakaraṇānaṁ aññataraṁ dubbaṇṇakaraṇaṁ ādātabbaṁ, nīlaṁ vā kaddamaṁ vā kāḷasāmaṁ vā. Anādā ce bhikkhu tiṇṇaṁ dubbaṇṇakaraṇānaṁ aññataraṁ dubbaṇṇakaraṇaṁ navaṁ cīvaraṁ paribhuñjeyya, pācittiyaṁ.

@ pali-pac-59.tex
Yo pana bhikkhu bhikkhussa vā bhikkhuniyā vā sikkhamānāya vā sāmaṇerassa vā sāmaṇeriyā vā sāmaṁ cīvaraṁ vikappetvā apaccuddhārakaṁ paribhuñjeyya, pācittiyaṁ.

@ pali-pac-60.tex
Yo pana bhikkhu bhikkhussa pattaṁ vā cīvaraṁ vā nisīdanaṁ vā sūcigharaṁ vā kāyabandhanaṁ vā apanidheyya vā apanidhāpeyya vā, antamaso hass’āpekkho’pi, pācittiyaṁ.

@ pali-center-pac-surapanavaggo.tex
Surāpānavaggo Chaṭṭho.

@ pali-pac-61.tex
Yo pana bhikkhu sañcicca pāṇaṁ jīvitā voropeyya, pācittiyaṁ.

@ pali-pac-62.tex
Yo pana bhikkhu jānaṁ sappāṇakaṁ udakaṁ paribhuñjeyya,
pācittiyaṁ.

@ pali-pac-63.tex
Yo pana bhikkhu jānaṁ yathādhammaṁ nīhatādhikaraṇaṁ punakammāya ukkoṭeyya, pācittiyaṁ.

@ pali-pac-64.tex
Yo pana bhikkhu bhikkhussa jānaṁ duṭṭhullaṁ āpattiṁ paṭicchādeyya, pācittiyaṁ.

@ pali-pac-65.tex
Yo pana bhikkhu jānaṁ ūnavīsativassaṁ puggalaṁ upasampādeyya, so ca puggalo anupasampanno, te ca bhikkhū gārayhā. Idaṁ tasmiṁ pācittiyaṁ.

@ pali-pac-66.tex
Yo pana bhikkhu jānaṁ theyyasatthena saddhiṁ saṁvidhāya ekaddhānamaggaṁ paṭipajjeyya, antamaso gām’antaram’pi, pācittiyaṁ.

@ pali-pac-67.tex
Yo pana bhikkhu mātugāmena saddhiṁ saṁvidhāya ekaddhānamaggaṁ paṭipajjeyya, antamaso gām’antaram’pi, pācittiyaṁ.

@ pali-pac-68.tex
Yo pana bhikkhu evaṁ vadeyya, “Tathāhaṁ bhagavatā dhammaṁ desitaṁ ājānāmi, yathā ye’me antarāyikā dhammā vuttā bhagavatā, te paṭisevato nālaṁ antarāyāyā” ti. 

@ pali-pac-68a.tex
So bhikkhu bhikkhūhi evam’assa vacanīyo, “Mā āyasmā evaṁ avaca. Mā bhagavantaṁ abbhācikkhi. Na hi sādhu bhagavato abbhakkhānaṁ. Na hi bhagavā evaṁ vadeyya. Anekapariyāyena āvuso antarāyikā dhammā vuttā bhagavatā, alañca pana te paṭisevato antarāyāyā” ti. 

@ pali-pac-68b.tex
Evañca so bhikkhu bhikkhūhi vuccamāno tath’eva paggaṇheyya, so bhikkhu bhikkhūhi yāvatatiyaṁ samanubhāsitabbo tassa paṭinissaggāya. Yāvatatiyañce samanubhāsiyamāno taṁ paṭinissajjeyya, icc’etaṁ kusalaṁ. No ce paṭinissajjeyya, pācittiyaṁ.

@ pali-pac-69.tex
Yo pana bhikkhu jānaṁ tathāvādinā bhikkhunā akaṭānudhammena taṁ diṭṭhiṁ appaṭinissaṭṭhena, saddhiṁ sambhuñjeyya vā saṁvaseyya vā saha vā seyyaṁ kappeyya, pācittiyaṁ.

@ pali-pac-70.tex
Samaṇuddeso’pi ce evaṁ vadeyya, “Tathāhaṁ bhagavatā dhammaṁ desitaṁ ājānāmi, yathā ye’me antarāyikā dhammā vuttā bhagavatā, te paṭisevato nālaṁ antarāyāyā” ti.

@ pali-pac-70a.tex
 So samaṇuddeso bhikkhūhi evam’assa vacanīyo, “Mā āvuso samaṇuddesa evaṁ avaca. Mā bhagavantaṁ abbhācikkhi. Na hi sādhu bhagavato abbhakkhānaṁ. na hi bhagavā evaṁ vadeyya. anekapariyāyena āvuso samaṇuddesa antarāyikā dhammā vuttā bhagavatā, alañca pana te paṭisevato antarāyāyā” ti. 

@ pali-pac-70b.tex
Evañca so samaṇuddeso bhikkhūhi vuccamāno tath’eva paggaṇheyya, so samaṇuddeso bhikkhūhi evam’assa vacanīyo, “Ajjatagge te āvuso samaṇuddesa na c’eva so bhagavā satthā apadisitabbo, yam’pi c’aññe samaṇuddesā labhanti bhikkhūhi saddhiṁ dvirattatirattaṁ sahaseyyaṁ, sā’pi te n’atthi. Cara pire vinassā” ti.

@ pali-pac-70c.tex
Yo pana bhikkhu jānaṁ tathānāsitaṁ samaṇuddesaṁ upalāpeyya vā upaṭṭhāpeyya vā sambhuñjeyya vā saha vā seyyaṁ kappeyya, pācittiyaṁ.

@ pali-center-pac-sappanavaggo.tex
Sappāṇavaggo Sattamo.

@ pali-pac-71.tex
Yo pana bhikkhu bhikkhūhi sahadhammikaṁ vuccamāno evaṁ vadeyya, “Na tāvāhaṁ āvuso etasmiṁ sikkhāpade sikkhissāmi, yāva n’aññaṁ bhikkhuṁ byattaṁ vinayadharaṁ paripucchāmī” ti, pācittiyaṁ.

@ pali-pac-71a.tex
Sikkhamānena bhikkhave bhikkhunā aññātabbaṁ paripucchitabbaṁ paripañhitabbaṁ. Ayaṁ tattha sāmīci.

@ pali-pac-72.tex
Yo pana bhikkhu pāṭimokkhe uddissamāne evaṁ vadeyya, “Kimpan’imehi khuddānukhuddakehi sikkhāpadehi uddiṭṭhehi, yāvad’eva kukkuccāya vihesāya vilekhāya saṁvattantī” ti. Sikkhāpadavivaṇṇanake, pācittiyaṁ.

@ pali-pac-73.tex
Yo pana bhikkhu anvaḍḍhamāsaṁ pāṭimokkhe uddissamāne evaṁ vadeyya, “Idān’eva kho ahaṁ ājānāmi, ‘Ayam’pi kira dhammo sutt’āgato suttapariyāpanno anvaḍḍhamāsaṁ uddesaṁ āgacchatī’” ti.

@ pali-pac-73a.tex
Tañce bhikkhuṁ aññe bhikkhū jāneyyuṁ, “Nisinnapubbaṁ iminā bhikkhunā dvittikkhattuṁ pāṭimokkhe uddissamāne, ko pana vādo bhiyyo” ti,

@ pali-pac-73b.tex
na ca tassa bhikkhuno aññāṇakena mutti atthi. Yañca tattha āpattiṁ āpanno, tañca yathādhammo kāretabbo, uttariñc’assa moho āropetabbo,

@ pali-pac-73c.tex
“Tassa te āvuso alābhā, tassa te dulladdhaṁ, yaṁ tvaṁ pāṭimokkhe uddissamāne na sādhukaṁ aṭṭhikatvā manasikarosī” ti. Idaṁ tasmiṁ mohanake, pācittiyaṁ.

@ pali-pac-74.tex
Yo pana bhikkhu bhikkhussa kupito anattamano pahāraṁ dadeyya, pācittiyaṁ.

@ pali-pac-75.tex
Yo pana bhikkhu bhikkhussa kupito anattamano talasattikaṁ uggireyya, pācittiyaṁ.

@ pali-pac-76.tex
Yo pana bhikkhu bhikkhuṁ amūlakena saṅghādisesena anuddhaṁseyya, pācittiyaṁ.

@ pali-pac-77.tex
Yo pana bhikkhu bhikkhussa sañcicca kukkuccaṁ upadaheyya, “Iti’ssa muhuttam’pi aphāsu bhavissatī” ti. Etad’eva paccayaṁ karitvā anaññaṁ, pācittiyaṁ.

@ pali-pac-78.tex
Yo pana bhikkhu bhikkhūnaṁ bhaṇḍanajātānaṁ kalahajātānaṁ vivādāpannānaṁ upassutiṁ tiṭṭheyya, “Yaṁ ime bhaṇissanti taṁ sossāmī” ti. Etad’eva paccayaṁ karitvā anaññaṁ, pācittiyaṁ.

@ pali-pac-79.tex
Yo pana bhikkhu dhammikānaṁ kammānaṁ chandaṁ datvā, pacchā khiyyanadhammaṁ āpajjeyya, pācittiyaṁ.

@ pali-pac-80.tex
Yo pana bhikkhu saṅghe vinicchayakathāya vattamānāya, chandaṁ adatvā uṭṭhāy‘āsanā pakkameyya, pācittiyaṁ.

@ pali-pac-81.tex
Yo pana bhikkhu samaggena saṅghena cīvaraṁ datvā, pacchā khiyyanadhammaṁ āpajjeyya, “Yathāsanthutaṁ bhikkhū saṅghikaṁ lābhaṁ pariṇāmentī” ti, pācittiyaṁ.

@ pali-pac-82.tex
Yo pana bhikkhu jānaṁ saṅghikaṁ lābhaṁ pariṇataṁ puggalassa pariṇāmeyya, pācittiyaṁ.

@ pali-center-pac-sahadhammikavaggo.tex
Sahadhammikavaggo Aṭṭhamo.

@ pali-pac-83.tex
Yo pana bhikkhu rañño khattiyassa muddhābhisittassa anikkhantarājake aniggataratanake pubbe appaṭisaṁvidito indakhīlaṁ atikkāmeyya, pācittiyaṁ.

@ pali-pac-84.tex
Yo pana bhikkhu ratanaṁ vā ratanasammataṁ vā aññatra ajjhārāmā vā ajjhāvasathā vā uggaṇheyya vā uggaṇhāpeyya vā, pācittiyaṁ.

@ pali-pac-84a.tex
Ratanaṁ vā pana bhikkhunā ratanasammataṁ vā, ajjhārāme vā ajjhāvasathe vā uggahetvā vā uggaṇhāpetvā vā nikkhipitabbaṁ, “Yassa bhavissati so harissatī” ti. Ayaṁ tattha sāmīci.

@ pali-pac-85.tex
Yo pana bhikkhu santaṁ bhikkhuṁ anāpucchā vikāle gāmaṁ paviseyya, aññatra tathārūpā accāyikā karaṇīyā, pācittiyaṁ.

@ pali-pac-86.tex
Yo pana bhikkhu aṭṭhimayaṁ vā dantamayaṁ vā visāṇamayaṁ vā sūcigharaṁ kārāpeyya, bhedanakaṁ pācittiyaṁ.

@ pali-pac-87.tex
Navam’pana bhikkhunā mañcaṁ vā pīṭhaṁ vā kārayamānena, aṭṭh’aṅgulapādakaṁ kāretabbaṁ sugat’aṅgulena, aññatra heṭṭhimāya aṭaniyā. Taṁ atikkāmayato, chedanakaṁ pācittiyaṁ.

@ pali-pac-88.tex
Yo pana bhikkhu mañcaṁ vā pīṭhaṁ vā tūlonaddhaṁ kārāpeyya, uddālanakaṁ pācittiyaṁ.

@ pali-pac-89.tex
Nisīdanaṁ pana bhikkhunā kārayamānena pamāṇikaṁ kāretabbaṁ. Tatr’idaṁ pamāṇaṁ: dīghaso dve vidatthiyo sugatavidatthiyā, tiriyaṁ diyaḍḍhaṁ, dasā vidatthi. Taṁ atikkāmayato, chedanakaṁ pācittiyaṁ.

@ pali-pac-90.tex
Kaṇḍupaṭicchādiṁ pana bhikkhunā kārayamānena pamāṇikā kāretabbā. Tatr’idaṁ pamāṇaṁ: dīghaso catasso vidatthiyo sugatavidatthiyā, tiriyaṁ dve vidatthiyo. Taṁ atikkāmayato, chedanakaṁ pācittiyaṁ.

@ pali-pac-91.tex
Vassikasāṭikaṁ pana bhikkhunā kārayamānena pamāṇikā kāretabbā. Tatr’idaṁ pamāṇaṁ: dīghaso cha vidatthiyo sugatavidatthiyā tiriyaṁ aḍḍhateyyā. Taṁ atikkāmayato, chedanakaṁ pācittiyaṁ.

@ pali-pac-92.tex
Yo pana bhikkhu sugatacīvarappamāṇaṁ cīvaraṁ kārāpeyya atirekaṁ vā, chedanakaṁ pācittiyaṁ.

@ pali-pac-92a.tex
Tatr’idaṁ sugatassa sugatacīvarappamāṇaṁ: dīghaso nava vidatthiyo sugatavidatthiyā, tiriyaṁ cha vidatthiyo. Idaṁ sugatassa sugatacīvarappamāṇaṁ.

@ pali-center-pac-ratanavaggo.tex
Ratanavaggo Navamo.

@ pali-pac-uddittha.tex
Uddiṭṭhā kho āyasmanto dvenavuti pācittiyā dhammā.

@ pali-pac-parisuddha.tex
Tatth’āyasmante pucchāmi: kacci’ttha parisuddhā? +
Dutiyam’pi pucchāmi: kacci’ttha parisuddhā? +
Tatiyam’pi pucchāmi: kacci’ttha parisuddhā? +
Parisuddh’etth’āyasmanto, tasmā tuṇhī, evam’etaṁ dhārayāmi.

@ pali-center-pac-nitthito.tex
Pācittiyā Niṭṭhitā
