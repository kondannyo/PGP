@ eng-npac-0.tex
Venerables, these thirty cases involving expiation with forfeiture come up for recitation.

@ eng-npac-1.tex
1. When the robe (-cloth) has been finished by a bhikkhu, when the kaþhina (-frame-privileges) have been withdrawn, (then) extra robe (-cloth) is to be kept for ten days at the most. For one who lets it pass beyond (the ten days), (this is a case) involving expiation with forfeiture.

@ eng-npac-2.tex
2. When the robe (-cloth) has been finished by a bhikkhu, when the kaþhina (-frame-privileges) have been withdrawn, if even for a single night a bhikkhu should stay apart from the three robes, except with the authorization of bhikkhus, (this is a case) involving expiation with forfeiture.

@ eng-npac-3.tex
3. When the robe (-cloth) has been finished by a bhikkhu, when the kaþhina (-frame-privileges) have been withdrawn, if out-of-season robe (-cloth) should become available to a bhikkhu, by a bhikkhu who is wishing (so, it) can be accepted; having accepted (it, it) is to be made very quickly. If (the robe-cloth) should not be (enough for) the completion (of the robe), (then) for a month at the most that robe (-cloth) can be put aside by that bhikkhu for the completion of the deficiency (of robe-cloth), when there is an expectation (that he will get more robe-cloth); if he should put (it) aside more than that, even when there is an expectation (that he will get more robe-cloth), (this is a case) involving expiation with forfeiture.

@ eng-npac-4.tex
4. If any bhikkhu should have a used robe (-cloth) washed, dyed, or beaten by an unrelated bhikkhunì, (this is a case) involving expiation with forfeiture.)

@ eng-npac-5.tex
5. If any bhikkhu should accept a robe (-cloth) from the hand of an unrelated bhikkhunì, except in an exchange (of robes), (this is a case) involving expiation with forfeiture.

@ eng-npac-6.tex
6. If any bhikkhu should request a robe (-cloth) to an unrelated male householder or female householder, except at the (right) occasion, (this is a case) involving expiation with forfeiture. Here the occasion is this: he is a bhikkhu whose robe has been robbed or whose robe has been lost; this is the occasion here.

@ eng-npac-7.tex
7. If the unrelated male householder or female householder should invite him to take (as many) robe (-cloth)s (as he likes), (then) robe (-cloths for) an upper (robe) together with an inner (robe) can be accepted at the most from that robe (-cloth) by that bhikkhu; if he should accept more from that (robe-cloth), (this is a case) involving expiation with forfeiture.

@ eng-npac-8.tex
8. Now, if an robe-fund has been set up for a bhikkhu by an unrelated male householder or female householder (thinking): “Having traded this robe-fund for a robe, I shall clothe the bhikkhu named so and so with a robe,”\\
and then if that bhikkhu, previously uninvited, having approached (the householder), should make a suggestion about the robe (-cloth) (saying): “It would be good indeed, Sir, (if you) having traded this robe-fund for a such and such a robe, were to clothe me (with a robe),” (if the suggestion is made) out of a liking for what is fine, (this is a case) involving expiation with forfeiture.

@ eng-npac-9.tex
9. Now, if separate robe-funds have been set up for a bhikkhu by both unrelated male householders or female householders (thinking): “Having traded these separate robe-funds for separate robes, we shall clothe the bhikkhu named so and so with robes,”\\
and then if that bhikkhu, previously uninvited, having approached (the householders), should make a suggestion about the robe (saying): “It would be good indeed, Sirs, (if you) having traded these separate robe-funds for a such and such a robe, were to clothe me (with a robe), (you) both being one (donor),” (if the suggestion is made) out of a liking for what is fine, (this is a case) involving expiation with forfeiture.

@ eng-npac-10.tex
10. Now, if a king or a kings’ official or a brahmin or a male householder should convey by messenger a robe-fund for a bhikkhu (saying): “Having traded this robe-fund for a robe, clothe the bhikkhu named so and so with a robe,” \\
and if that messenger, having approached that bhikkhu, should say so: “Venerable Sir, this robe-fund has been brought for the venerable one. Let the venerable one accept the robe-fund!”\\
(then) that messenger should be spoken to thus by that bhikkhu: “Sir, we do not accept a robe-fund, but we do accept a robe at the right time (when it is) allowable.”\\
If that messenger should say thus to that bhikkhu: “Is there, perhaps, someone who is the steward of the venerable one?” (then,) bhikkhus, by a bhikkhu who is in need of a robe, a steward can be appointed: a monastery attendant or a male lay-follower (saying): “Sir, this is the bhikkhus' steward.”\\
If that messenger having instructed that steward, having approached that bhikkhu, should say so: “Venerable Sir, the steward whom the venerable one has appointed has been instructed by me. Let the venerable one approach (him) at the right time (and) he will clothe you with a robe,” (then) bhikkhus, having approached the steward, (the steward) can be prompted (and) can be reminded two or three times by the bhikkhu who is in need of a robe (saying): “Sir, I am in need of a robe.”\\
(If through) prompting (and) reminding (him) two or three times, he should have (him) bring forth that robe, it is good. If he should not have (him) bring (it) forth, (then) four times, five times, six times at the most, (it) can be stood (for) by (a bhikkhu) who has become silent. (If through) standing silently for (it) four times, five times, six times at the most, he should have (him) bring forth that robe, it is good; if (through) making effort more than that, he should have (him) produce that robe, (this is a case) involving expiation with forfeiture.\\
If he should not have (him) produce (it), (then) from wherever (that) the robe-fund may have been brought, there (he) himself can go, or a messenger can be sent (saying): “Sirs, that robe-fund which you conveyed for the bhikkhu does not fulfil any need of that bhikkhu. Let the sirs endeavour for (what is their) own. Let not (what is their) own get lost.” This is the proper procedure here.

@ eng-npac-civaravaggo.tex
The section (starting with the rule) on robes is first.

@ eng-npac-11.tex
11. If any bhikkhu should have a rug mixed with silk made, (this is a case) involving expiation with forfeiture.

@ eng-npac-12.tex
12. If any bhikkhu should have a rug made of pure black sheep's wool; (this is a case) involving expiation with forfeiture.

@ eng-npac-13.tex
13. By a bhikkhu who is having a new rug made, two parts of pure black sheep-wool are to be taken, (and) a third (part) of white, a fourth (part) of ruddy brown. If a bhikkhu should have a rug made, without having taken two parts of pure black sheep's hair, (and) a third (part) of white, a fourth (part) of ruddy brown, (this is a case) involving expiation with forfeiture.

@ eng-npac-14.tex
14. By a bhikkhu who has had a new rug made, it is to be kept for six years (at least). If within less than six years, having given up or not having given up that rug, he should have another new rug made, except with the authorisation of bhikkhus, (this is a case) involving expiation with forfeiture.

@ eng-npac-15.tex
15. By a bhikkhu who is having a sitting-rug made, a sugata-span from the border of an old rug is to be taken for making (it) stained. If a bhikkhu, without having taken a sugata-span from the border of an old rug, should have a new sitting cloth made, (this is a case) involving expiation with forfeiture.

@ eng-npac-16.tex
16. Now, if sheep-wool should become available to a bhikkhu who is travelling on a main road, by a bhikkhu who is wishing (so, it) can be accepted, having accepted (it, it) can be carried with his own hand for three yojanas at the most when there is no one present who can carry it; if he should carry it more than that, even when there is no one present who can carry it, (this is a case) involving expiation with forfeiture.

@ eng-npac-17.tex
17. If any bhikkhu should have sheep-wool washed, dyed, or carded by an unrelated bhikkhunì, (this is a case) involving expiation with forfeiture.

@ eng-npac-18.tex
18. If any bhikkhu should take gold and silver, or should have (it) taken, or should consent to (it) being deposited (for him), (this is a case) involving expiation with forfeiture.

@ eng-npac-19.tex
19. If any bhikkhu should engage in the various kinds of trading in money, (this is a case) involving expiation with forfeiture.

@ eng-npac-20.tex
20. If any bhikkhu should engage in the various kinds of bartering, (this is a case) involving expiation with forfeiture.

@ eng-npac-kosiyavaggo.tex
The section on sheepwool is second.

@ eng-npac-21.tex
21. An extra bowl can be kept for ten days at the most. For one who lets it pass beyond (the ten days); (this is a case) involving expiation with forfeiture.

@ eng-npac-22.tex
22. If any bhikkhu should exchange a bowl with less than five mends for another new bowl, (this is a case) involving expiation with forfeiture. That bowl is to be relinquished by that bhikkhu to the assembly of bhikkhus, and whichever (bowl) is the last bowl of that assembly of bhikkhus, that (bowl) is to be bestowed on that bhikkhu (thus): “Bhikkhu, this bowl is for you, it is to be kept until breaking.” This is the proper procedure here.

@ eng-npac-23.tex
23. Now, (there are) those medicines which are permissable for sick bhikkhus, namely: ghee, butter, oil, (and) honey and molasses—having been accepted, they can be partaken of (while) being kept in store for seven days at the most. For one who lets it pass beyond (the seven days), (this is a case) involving expiation with forfeiture.

@ eng-npac-24.tex
24. (Thinking:) “One month is what remains of the hot season,” (then) the robe-cloth for the rain's bathing-cloth can be sought by a bhikkhu. (Thinking:) “A half month is what remains of the hot season,” (after) having made (it, it) can be worn. If earlier than (what is reckoned as) “One month is what remains of the hot season,” he should seek robe-cloth for the rain's bathing-cloth, (and) (if) earlier than (what is reckoned as) “A half month is what remains of the hot season,” he should wear (it), (this is a case) involving expiation with forfeiture.

@ eng-npac-25.tex
25. If any bhikkhu, having himself given a robe to a bhikkhu, should, being resentful (and) displeased,snatch (it) away or should have it snatched away (from the bhikkhu), (this is a case) involving expiation with forfeiture.

@ eng-npac-26.tex
26. If any bhikkhu, having himself requested the thread (to be used), should have a robe-cloth woven by cloth-weavers, (this is a case) involving expiation with forfeiture.

@ eng-npac-27.tex
27. Now, if an unrelated male householder or female householder should have a robe-cloth woven for a bhikkhu by cloth-weavers, and then if that bhikkhu, uninvited beforehand, having approached the cloth-weavers, should make a suggestion about the robe-cloth (saying): “Friends, this robe-cloth which is being woven for me: make (it) long, wide, thick, well woven, well diffused, well scraped, and well plucked! Certainly we will also (then) present a little something to the sirs,” and if that bhikkhu, having said so, should present a little something, even just a little alms-food, (this is a case) involving expiation with forfeiture.

@ eng-npac-28.tex
28. For the ten days coming up to the three-month Kattiká full moon: if extraordinary robe (-cloth) should become available to a bhikkhu, (then) after considering (it as) extraordinary (robe-cloth, it) can be accepted by a bhikkhu, having been accepted, (it) is to be put aside until the occasion of the robe-season; if he should put (it) aside for more than that, (this is a case) involving expiation with forfeiture.

@ eng-npac-29.tex
29. Now, the Kattika-full-moon has been observed. (There are) those wilderness lodgings which are considered risky, which are dangerous. A bhikkhu dwelling in such kind of lodgings, who is wishing (to do so), may put aside one of the three robes inside an inhabited area. And if there may be any reason for that bhikkhu for dwelling apart from that robe, the bhikkhu can dwell apart from that robe for six days at the most; if he should dwell apart for more than that, except with the authorisation of bhikkhus, (this is a case) involving expiation with forfeiture.

@ eng-npac-30.tex
30. If any bhikkhu should knowingly allocate for himself a gain belonging to (and) allocated to the  community, (this is a case) involving expiation with forfeiture.

@ eng-npac-pattavaggo.tex
The section on bowls is third.

@ eng-npac-uddittha.tex
Venerables, the thirty cases involving expiation with forfeiture have been recited.

@ eng-npac-parisuddha.tex
Concerning this I ask the Venerables: (Are you) pure in this?\\
A second time again I ask: (Are you) pure in this?\\
A third time again I ask: (Are you) pure in this?\\
The Venerables are pure in this, therefore there is silence, thus I keep this (in mind).

@ eng-npac-nitthito.tex
The cases involving expiation with forfeiture are finished.
