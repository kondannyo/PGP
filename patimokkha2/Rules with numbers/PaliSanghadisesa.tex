@ pali-sd-0.tex
Ime kho pan’āyasmanto terasa saṅghādisesā dhammā uddesaṁ āgacchanti.

@ pali-sd-1.tex
1. Sañcetanikā sukkavisaṭṭhi aññatra supinantā, saṅghādiseso.

@ pali-sd-2.tex
2. Yo pana bhikkhu otiṇṇo vipariṇatena cittena mātugāmena saddhiṁ kāyasaṁsaggaṁ samāpajjeyya, hatthagāhaṁ vā veṇigāhaṁ vā aññatarassa vā aññatarassa vā aṅgassa parāmasanaṁ, saṅghādiseso.

@ pali-sd-3.tex
3. Yo pana bhikkhu otiṇṇo vipariṇatena cittena mātugāmaṁ duṭṭhullāhi vācāhi obhāseyya, yathā taṁ yuvā yuvatiṁ methunūpasaṁhitāhi, saṅghādiseso.

@ pali-sd-4.tex
4. Yo pana bhikkhu otiṇṇo vipariṇatena cittena mātugāmassa santike attakāmapāricariyāya vaṇṇaṁ bhāseyya, 

@ pali-sd-4a.tex
“Etadaggaṁ bhagini pāricariyānaṁ, yā m’ādisaṁ sīlavantaṁ kalyāṇadhammaṁ brahmacāriṁ etena dhammena paricareyyā” ti, methunūpasaṁhitena, saṅghādiseso.

@ pali-sd-5.tex
5. Yo pana bhikkhu sañcarittaṁ samāpajjeyya, itthiyā vā purisamatiṁ, purisassa vā itthīmatiṁ, jāyattane vā jārattane vā antamaso taṁkhaṇikāya’pi, saṅghādiseso.

@ pali-sd-6.tex
6. Saññācikāya pana bhikkhunā kuṭiṁ kārayamānena assāmikaṁ att’uddesaṁ pamāṇikā kāretabbā. 

@ pali-sd-6a.tex
Tatr’idaṁ pamāṇaṁ: dīghaso dvādasa vidatthiyo sugatavidatthiyā, tiriyaṁ satt’antarā. Bhikkhū abhinetabbā vatthudesanāya. Tehi bhikkhūhi vatthuṁ desetabbaṁ anārambhaṁ saparikkamanaṁ. 

@ pali-sd-6b.tex
Sārambhe ce bhikkhu vatthusmiṁ aparikkamane saññācikāya kuṭiṁ kāreyya, bhikkhū vā anabhineyya vatthudesanāya, pamāṇaṁ vā atikkāmeyya, saṅghādiseso.

@ pali-sd-7.tex
7. Mahallakaṁ pana bhikkhunā vihāraṁ kārayamānena, sassāmikaṁ att’uddesaṁ bhikkhū abhinetabbā vatthudesanāya. Tehi bhikkhūhi vatthuṁ desetabbaṁ anārambhaṁ saparikkamanaṁ.

@ pali-sd-7a.tex
Sārambhe ce bhikkhu vatthusmiṁ aparikkamane mahallakaṁ vihāraṁ kāreyya, bhikkhū vā anabhineyya vatthudesanāya, saṅghādiseso.

@ pali-sd-8.tex
8. Yo pana bhikkhu bhikkhuṁ duṭṭho doso appatīto amūlakena pārājikena dhammena anuddhaṁseyya, “App’eva nāma naṁ imamhā brahmacariyā cāveyyan” ti.

@ pali-sd-8a.tex
Tato aparena samayena samanuggāhiyamāno vā asamanuggāhiyamāno vā, amūlakañc’eva taṁ adhikaraṇaṁ hoti, bhikkhu ca dosaṁ patiṭṭhāti, saṅghādiseso.

@ pali-sd-9.tex
9. Yo pana bhikkhu bhikkhuṁ duṭṭho doso appatīto aññabhāgiyassa adhikaraṇassa kiñci desaṁ lesamattaṁ upādāya pārājikena dhammena anuddhaṁseyya, “App’eva nāma naṁ imamhā brahmacariyā cāveyyan” ti.

@ pali-sd-9a.tex
Tato aparena samayena samanuggāhiyamāno vā asamanuggāhiyamāno vā, aññabhāgiyañc’eva taṁ adhikaraṇaṁ hoti, koci deso lesamatto upādinno, bhikkhu ca dosaṁ patiṭṭhāti, saṅghādiseso.

@ pali-sd-10.tex
10. Yo pana bhikkhu samaggassa saṅghassa bhedāya parakkameyya, bhedanasaṁvattanikaṁ vā adhikaraṇaṁ samādāya paggayha tiṭṭheyya, so bhikkhu bhikkhūhi evam assa vacanīyo,

@ pali-sd-10a.tex
“Mā āyasmā samaggassa saṅghassa bhedāya parakkami. Bhedanasaṁvattanikaṁ vā adhikaraṇaṁ samādāya paggayha aṭṭhāsi. Samet’āyasmā saṅghena, samaggo hi saṅgho sammodamāno avivadamāno ek’uddeso phāsu viharatī” ti.

@ pali-sd-10b.tex
Evañca so bhikkhu bhikkhūhi vuccamāno tath’eva paggaṇheyya, so bhikkhu bhikkhūhi yāvatatiyaṁ samanubhāsitabbo tassa paṭinissaggāya. Yāvatatiyañ’ce samanubhāsiyamāno taṁ paṭinissajjeyya, icc’etaṁ kusalaṁ. No ce paṭinissajjeyya, saṅghādiseso.

@ pali-sd-11.tex
11. Tass’eva kho pana bhikkhussa bhikkhū honti anuvattakā vaggavādakā, eko vā dve vā tayo vā, te evaṁ vadeyyuṁ, “Mā āyasmanto etaṁ bhikkhuṁ kiñci avacuttha. Dhammavādī c’eso bhikkhu, vinayavādī c’eso bhikkhu, amhākañc’eso bhikkhu chandañca ruciñca ādāya voharati. Jānāti no bhāsati, amhākam’p’etaṁ khamatī” ti.

@ pali-sd-11a.tex
Te bhikkhū bhikkhūhi evamassu vacanīyā, “Mā āyasmanto evaṁ avacuttha. Na c’eso bhikkhu dhammavādī, na c’eso bhikkhu vinayavādī. Mā āyasmantānam’pi saṅghabhedo rucittha. Samet’āyasmantānaṁ saṅghena, samaggo hi saṅgho sammodamāno avivadamāno ek’uddeso phāsu viharatī” ti.

@ pali-sd-11b.tex
Evañca te bhikkhū bhikkhūhi vuccamānā tath’eva paggaṇheyyuṁ, te bhikkhū bhikkhūhi yāvatatiyaṁ samanubhāsitabbā tassa paṭinissaggāya. Yāvatatiyañce samanubhāsiyamānā taṁ paṭinissajjeyyuṁ, icc’etaṁ kusalaṁ. No ce paṭinissajjeyyuṁ, saṅghādiseso.

@ pali-sd-12.tex
12. Bhikkhu pan’eva dubbacajātiko hoti, uddesapariyāpannesu sikkhāpadesu bhikkhūhi sahadhammikaṁ vuccamāno attānaṁ avacanīyaṁ karoti, “Mā maṁ āyasmanto kiñci avacuttha kalyāṇaṁ vā pāpakaṁ vā. Aham’p’āyasmante na kiñci vakkhāmi kalyāṇaṁ vā pāpakaṁ vā. Viramath’āyasmanto mama vacanāyā” ti.

@ pali-sd-12a.tex
So bhikkhu bhikkhūhi evam’assa vacanīyo, “Mā āyasmā attānaṁ avacanīyaṁ akāsi. Vacanīyam’eva āyasmā attānaṁ karotu. Āyasmā’pi bhikkhū vadetu sahadhammena, bhikkhū’pi āyasmantaṁ vakkhanti sahadhammena. Evaṁ saṁvaddhā hi tassa bhagavato parisā, yad’idaṁ aññamaññavacanena aññamaññavuṭṭhāpanenā” ti.

@ pali-sd-12b.tex
Evañca so bhikkhu bhikkhūhi vuccamāno tath’eva paggaṇheyya, so bhikkhu bhikkhūhi yāvatatiyaṁ samanubhāsitabbo tassa paṭinissaggāya. Yāvatatiyañce samanubhāsiyamāno taṁ paṭinissajjeyya, icc’etaṁ kusalaṁ. No ce paṭinissajjeyya, saṅghādiseso.

@ pali-sd-13.tex
13. Bhikkhu pan’eva aññataraṁ gāmaṁ vā nigamaṁ vā upanissāya viharati kuladūsako pāpasamācāro. Tassa kho pāpakā samācārā dissanti c’eva suyyanti ca, kulāni ca tena duṭṭhāni dissanti c’eva suyyanti ca. So bhikkhu bhikkhūhi evam’assa vacanīyo,

@ pali-sd-13a.tex
“Āyasmā kho kuladūsako pāpasamācāro. Āyasmato kho pāpakā samācārā dissanti c’eva suyyanti ca, kulāni c’āyasmatā duṭṭhāni dissanti c’eva suyyanti ca. Pakkamat’āyasmā imamhā āvāsā, alante idha vāsenā” ti.

@ pali-sd-13b.tex
Evañca so bhikkhu bhikkhūhi vuccamāno te bhikkhū evaṁ vadeyya, “Chandagāmino ca bhikkhū, dosagāmino ca bhikkhū, mohagāmino ca bhikkhū, bhayagāmino ca bhikkhū, tādisikāya āpattiyā ekaccaṁ pabbājenti, ekaccaṁ na pabbājentī” ti.

@ pali-sd-13c.tex
So bhikkhu bhikkhūhi evam’assa vacanīyo, “Mā āyasmā evaṁ avaca. Na ca bhikkhū chandagāmino, na ca bhikkhū dosagāmino, na ca bhikkhū mohagāmino, na ca bhikkhū bhayagāmino.

@ pali-sd-13d.tex
Āyasmā kho kuladūsako pāpasamācāro. Āyasmato kho pāpakā samācārā dissanti c’eva suyyanti ca, kulāni c’āyasmatā duṭṭhāni dissanti c’eva suyyanti ca. Pakkamat’āyasmā imamhā āvāsā, alan’te idha vāsenā” ti.

@ pali-sd-13e.tex
Evañca so bhikkhu bhikkhūhi vuccamāno tath’eva paggaṇheyya, so bhikkhu bhikkhūhi yāvatatiyaṁ samanubhāsitabbo tassa paṭinissaggāya. Yāvatatiyañce samanubhāsiyamāno taṁ paṭinissajjeyya, icc’etaṁ kusalaṁ. No ce paṭinissajjeyya, saṅghādiseso.

@ pali-sd-uddittha.tex
Uddiṭṭhā kho āyasmanto terasa saṅghādisesā dhammā, nava paṭham’āpattikā cattāro yāvatatiyakā.

@ pali-sd-udditthaa.tex
Yesaṁ bhikkhu aññataraṁ vā aññataraṁ vā āpajjitvā yāvatihaṁ jānaṁ paṭicchādeti, tāvatihaṁ tena bhikkhunā akāmā parivatthabbaṁ. Parivutthaparivāsena bhikkhunā uttariṁ chārattaṁ, bhikkhumānattāya paṭipajjitabbaṁ.

@ pali-sd-udditthab.tex
Ciṇṇamānatto bhikkhu, yattha siyā vīsatigaṇo bhikkhusaṅgho, tattha so bhikkhu abbhetabbo. Ekena’pi ce ūno vīsatigaṇo bhikkhusaṅgho taṁ bhikkhuṁ abbheyya, so ca bhikkhu anabbhito, te ca bhikkhū gārayhā. Ayaṁ tattha sāmīci.

@ pali-sd-parisuddha.tex
Tatth’āyasmante pucchāmi: Kacci’ttha parisuddhā? +
Dutiyam’pi pucchāmi: Kacci’ttha parisuddhā? +
Tatiyam’pi pucchāmi: Kacci’ttha parisuddhā? +
Parisuddh’etth’āyasmanto, tasmā tuṇhī, evam’etaṁ dhārayāmi.

@ pali-sd-nitthito.tex
Saṅghādises’uddeso niṭṭhito
