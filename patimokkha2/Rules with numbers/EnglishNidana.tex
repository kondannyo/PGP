@ eng-center-nid-title.tex
The Disciplinary Code of the Bhikkhu

@ eng-nid-1.tex
Homage to the Blessed, Noble, and Perfectly Enlightened One.
(3 times)

@ eng-nid-2.tex
Venerable Sir, let the Community listen to me! Today is a fifteenth (day) Observance. If it is suitable to the Community , (then) the Community should do the Observance (and) should recite the Disciplinary Code.

@ eng-nid-3.tex
What is the preliminary for the Community? Venerables, announce the purity, (for) I shall recite the Disciplinary Code. Let us all (who are) present listen to it carefully (and) let us take it to mind.

@ eng-nid-4.tex
Whoever may have an offence, he should disclose (it). When there is no offence, (then it) is to be silent.

@ eng-nid-4a.tex
By the silence I shall know the Venerables (with the thought): “(They are) pure.” As an answer occurs to (a bhikkhu) who is asked individually, just so in such an assembly (as this one) there is the announcement up to the third time. 

@ eng-nid-4b.tex
But if any bhikkhu, (who is) remembering (an offence) when the announcement is being made up to the third time, should not disclose the existing offence, there is (a further offence of) deliberate false speech for him.

@ eng-nid-5.tex
Now, venerables, deliberate false speech has been called an obstructive act by the Fortunate One. Therefore, by a bhikkhu who is remembering, who has committed (an offence), who is desiring purification, an existing offence is to be disclosed; because, (after) having disclosed (it), there is comfort for him.

@ eng-nid-6.tex
The recitation of the introduction is finished.
