@ pali-par-0.tex
Tatr’ime cattāro pārājikā dhammā uddesaṁ āgacchanti.

@ pali-par-1.tex
1. Yo pana bhikkhu bhikkhūnaṁ sikkhāsājīvasamāpanno, sikkhaṁ appaccakkhāya dubbalyaṁ anāvikatvā, methunaṁ dhammaṁ paṭiseveyya antamaso tiracchānagatāya’pi: pārājiko hoti asaṁvāso.

@ pali-par-2.tex
2. Yo pana bhikkhu gāmā vā araññā vā adinnaṁ theyyasaṅkhātaṁ ādiyeyya, yathārūpe adinnādāne rājāno coraṁ gahetvā, haneyyuṁ vā bandheyyuṁ vā pabbājeyyuṁ vā,

@ pali-par-2a.tex
“Coro’si bālo’si mūḷho’si theno’sī” ti. Tathārūpaṁ bhikkhu adinnaṁ ādiyamāno: ayam’pi pārājiko hoti asaṁvāso.

@ pali-par-3.tex
3. Yo pana bhikkhu sañcicca manussaviggahaṁ jīvitā voropeyya, satthahārakaṁ vāssa pariyeseyya, maraṇavaṇṇaṁ vā saṁvaṇṇeyya, maraṇāya vā samādapeyya,

@ pali-par-3a.tex
“Ambho purisa kiṁ tuyh’iminā pāpakena dujjīvitena? Matante jīvitā seyyo” ti. Iti cittamano cittasaṅkappo anekapariyāyena maraṇavaṇṇaṁ vā saṁvaṇṇeyya, maraṇāya vā samādapeyya: ayam’pi pārājiko hoti asaṁvāso.

@ pali-par-4.tex
4. Yo pana bhikkhu anabhijānaṁ uttarimanussadhammaṁ attūpanāyikaṁ alamariyañāṇadassanaṁ samudācareyya:

@ pali-par-4a.tex
“Iti jānāmi, iti passāmī” ti. Tato aparena samayena samanuggāhiyamāno vā asamanuggāhiyamāno vā āpanno visuddh’āpekkho evaṁ vadeyya, 

@ pali-par-4b.tex
“Ajānam evaṁ āvuso avacaṁ, ‘jānāmi,’ apassaṁ, ‘passāmi.’ Tucchaṁ musā vilapin” ti. Aññatra adhimānā: ayam’pi pārājiko hoti asaṁvāso.

@ pali-par-uddittha.tex
Uddiṭṭhā kho āyasmanto cattāro pārājikā dhammā, yesaṁ bhikkhu aññataraṁ vā aññataraṁ vā āpajjitvā na labhati bhikkhūhi saddhiṁ saṁvāsaṁ, yathā pure, tathā pacchā, pārājiko hoti asaṁvāso.

@ pali-par-parisuddha.tex
Tatth’āyasmante pucchāmi: Kacci’ttha parisuddhā? +
Dutiyam’pi pucchāmi: Kacci’ttha parisuddhā? +
Tatiyam’pi pucchāmi: Kacci’ttha parisuddhā? +
Parisuddh’etth’āyasmanto, tasmā tuṇhī, evam’etaṁ dhārayāmi.

@ pali-par-nitthito.tex
Pārājik’uddeso niṭṭhito
