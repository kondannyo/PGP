NOTES TO THE PUBBA-KICCAṀ

1. If the recitation is held at night, change “Tattha purimesu catūsu kiccesu padīpa-kiccaṁ idāni suriy’ālokassa atthitāya n’atthi. Aparāni tīṇi” to “Tattha purimāni cattāri”: “Of the first four….”
2. If sāmaṇeras help with the tasks, change “bhikkhūhi” to “sāmaṇerehi-pi bhikkhūhi-pi”: “Novices and bhikkhus….” If laypeople living in the monastery help with the tasks, change this to “ārāmikehi-pi bhikkhūhi-i”: “Monastery dwellers and bhikkhus….”
3. If there are bhikkhus outside of hatthapāsa but within the territory (sīmā) who have sent their consent and purity, then for a recitation during the day, the
entire passage within brackets should be: “Tattha purimesu chasu kiccesu padīpa-kiccaṁ idāni suriy’ālokassa atthitāya n’atthi. Aparāni pañca
bhikkhūnaṁ vattaṁ jānantehi bhikkhūhi2 katāni pariniṭṭhitāni honti.”

For a recitation at night in the same situation, the entire passage should be:
“Tattha purimāni cha bhikkhūnaṁ vattaṁ jānantehi bhikkhūhi2 katāni
pariniṭṭhitāni honti.”
4. During the hot season, change “hemantotu” to “gimhotu.” During the rainy season, change it to “vassānotu.”
5. During a normal rainy season, change “aṭṭha uposathā” to “sattā ca uposathā ekā ca pavāraṇā”: “Seven uposathas and one pavāraṇā.”
During a hot or cold season with an additional month, change it to “adhikamāsa-vasena dasa uposathā”: “Because of the additional month, ten
uposathās….”
During a rainy season with an additional month, change it to “adhikamāsa-vasena nava ca uposathā ekā ca pavāraṇā”: Because of the
additional month, nine uposathas and one pavāraṇā….”
6. This is the calculation for the first uposatha in a normal hot or cold season. The calculation for other dates — to be stated after “iminā pakkhena
eko uposatho sampatto” — is as follows:

During a normal hot or cold season:
Second: eko uposatho atikkanto, cha uposathā avasiṭṭhā.
Third: dve uposathā atikkantā, pañca uposathā avasiṭṭhā.
Fourth: tayo uposathā atikkantā, cattāro uposathā avasiṭṭhā.
Fifth: cattāro uposathā atikkantā, tayo uposathā avasiṭṭhā.
Sixth: pañca uposathā atikkantā, dve uposathā avasiṭṭhā.
Seventh: cha uposathā atikkantā, eko uposatho avasiṭṭho.
Eighth: satta uposathā atikkantā, aṭṭha uposathā paripuṇṇā.

During a normal rainy season:
First: cha ca uposathā ekā ca pavāraṇā avasiṭṭhā.
Second: eko uposatho atikkanto, pañca ca uposathā ekā ca pavāraṇā avasiṭṭhā.
Third: dve uposathā atikkantā, cattāro ca uposathā ekā ca pavāraṇā avasiṭṭhā.
Fourth: tayo uposathā atikkantā, tayo ca uposathā ekā ca pavāraṇā avasiṭṭhā.
Fifth: cattāro uposathā atikkantā, dve ca uposathā ekā ca pavāraṇā avasiṭṭhā.
Sixth: (see the separate section on the Pavāraṇā.)
Seventh: pañca ca uposathā ekā ca pavāraṇā atikkantā, eko uposatho avasiṭṭho.
Eighth: cha ca uposathā ekā ca pavāraṇā atikkantā, satta ca uposathā ekā ca pavāraṇā paripuṇṇā.

During a hot or cold season with an additional month:
First: nava uposathā avasiṭṭhā.
Second: eko uposatho atikkanto, aṭṭha uposathā avasiṭṭhā.
Third: dve uposathā atikkantā, satta uposathā avasiṭṭhā.
Fourth: tayo uposathā atikkantā, cha uposathā avasiṭṭhā.
Fifth: cattāro uposathā atikkantā, pañca uposathā avasiṭṭhā.
Sixth: pañca uposathā atikkantā, cattāro uposathā avasiṭṭhā.
Seventh: cha uposathā atikkantā, tayo uposathā avasiṭṭhā.
Eighth: satta uposathā atikkantā, dve uposathā avasiṭṭhā.
Ninth: aṭṭha uposathā atikkantā, eko uposatho avasiṭṭho.
Tenth: nava uposathā atikkantā, dasa uposathā paripuṇṇā.

During a rainy season with an additional month:
First: aṭṭha ca uposathā ekā ca pavāraṇā avasiṭṭhā.
Second: eko uposatho atikkanto, satta ca uposathā ekā ca pavāraṇā avasiṭṭhā.
Third: dve uposathā atikkantā, cha ca uposathā ekā ca pavāraṇā avasiṭṭhā.
Fourth: tayo uposathā atikkantā, pañca ca uposathā ekā ca pavāraṇā avasiṭṭhā.
Fifth: cattāro uposathā atikkantā, cattāro ca uposathā ekā ca pavāraṇā avasiṭṭhā.
Sixth: pañca uposathā atikkantā, tayo ca uposathā ekā ca pavāraṇā avasiṭṭhā.
Seventh: cha uposathā atikkantā, dve ca uposathā ekā ca pavāraṇā avasiṭṭhā.
Eighth: (see the separate section on the Pavāraṇā.)
Ninth: satta ca uposathā ekā ca pavāraṇā atikkantā, eko uposatho avasiṭṭho.
Tenth: aṭṭha ca uposathā ekā ca pavāraṇā atikkantā, nava ca uposathā ekā ca pavāraṇā paripuṇṇā.

6. Cattāro = four. This should be replaced with the actual number of bhikkhus present.

5 pañca 6 cha 7 satta 8 aṭṭha 9 nava 10 dasa 11 ekādasa 12
dvādasa, bārasa 13 terasa, teḷasa 14 catuddasa, cuddasa 15 paṇṇarasa,
pañcadasa 16 soḷasa 17 sattarasa 18 aṭṭhārasa, aṭṭhādasa 19
ekūnavīsati
20 vīsati, vīsa 21 ekavīsati 22 dvāvīsati, dvāvīsa, dvevīsati, bāvīsati,
bāvīsa 23 tevīsati 24 catuvīsati 25 pañcavīsati 26 chabbīsati 27
sattavīsati 28 aṭṭhavīsati 29 ekūnatiṁsa
30 tiṁsa, samatiṁsa, tiṁsati 31 ekatiṁsa, ekattiṁsa 32 dvattiṁsa
33 tettiṁsa 34 catuttiṁsa 35 pañcattiṁsa 36 chattiṁsa 37 sattattiṁsa
38 aṭṭhattiṁsa 39 ekūnacattāḷīsa
40 cattāḷīsa, cattārīsa 41 ekacattāḷīsa 42 dvacattāḷīsa, dvecattāḷīsa,
dvicattāḷīsa 43 tecattāḷīsa 44 catucattāḷīsa 45 pañcacattāḷīsa 46
chacattāḷīsa 47 sattacattāḷīsa 48 aṭṭhacattāḷīsa 49 ekūnapaññāsa
50 paññāsa 51 ekapaññāsa 52 dvapaññāsa, dvepaññāsa, dvipaññāsa
53 tepaññāsa 54 catupaññāsa 55 pañcapaññāsa 56 chapaññāsa 57
sattapaññāsa 58 aṭṭhapaññāsa 59 ekūnasaṭṭhī
60 saṭṭhī, saṭṭhi 61 ekasaṭṭhī 62 dvāsaṭṭhī, dvesaṭṭhī, dvisaṭṭhī 63
tesaṭṭhī 64 catusaṭṭhī 65 pañcasaṭṭhī 66 chasaṭṭhī 67 sattasaṭṭhī 68
aṭṭhasaṭṭhī 69 ekūnasattati
70 sattati 71 ekasattati 72 dvasattati, dvāsattati, dvesattati, dvisattati
73 tesattati 74 catusattati 75 pañcasattati 76 chasattati 77 sattasattati
78 aṭṭhasattati 79 ekūnāsīti
80 asīti 81 ekāsīti 82 dvāsīti 83 tayāsīti 84 caturāsīti 85 pañcāsīti 86
chaḷāsīti 87 sattāsīti 88 aṭṭhāsīti 89 ekūnanavuti
90 navuti 91 ekanavuti 92 dvanavuti, dvenavuti 93 tenavuti 94
catunavuti 95 pañcanavuti 96 chanavuti 97 sattanavuti 98 aṭṭhanavuti
99 ekūnasataṁ
100 bhikkhusataṁ 101 ekuttara-bhikkhusataṁ 102 dvayuttara-
bhikkhusataṁ 103 tayuttara-bhikkhusataṁ 104 catuttara-
bhikkhusataṁ 105 pañcuttara-bhikkhusataṁ 106 chaḷuttara-
bhikkhusataṁ 107 sattuttara-bhikkhusataṁ 108 aṭṭhuttara-
bhikkhusataṁ 109 navuttara-bhikkhusataṁ 110 dasuttara-
bhikkhusataṁ 120 vīsuttara-bhikkhusataṁ 130 tiṁsuttara-
bhikkhusataṁ 140 cattāḷīsuttara-bhikkhusataṁ 150 paññāsuttara-
bhikkhusataṁ 160 saṭṭhayuttara-bhikkhusataṁ 170 sattatyuttara-
bhikkhusataṁ 180 asītyuttara-bhikkhusataṁ 190 navutyuttara-
bhikkhusataṁ 199 ekūnasatuttara-bhikkhusataṁ
200 dve bhikkhu-satāni 201 ekuttarāni dve bhikkhu-satāni
300 tayo bhikkhu-satāni 400 cattāro bhikkhu-satāni 500 pañca
bhikkhu-satāni



Note:
All numbers ending with “bhikkhusataṁ” should be followed by “sannipatitaṁ hoti.”
All numbers ending with “bhikkhusatāni” should be followed by “sannipatitā honti.”