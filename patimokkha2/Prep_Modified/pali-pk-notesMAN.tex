NOTES TO THE PUBBA-KICCAṀ\\

1.  {\itshape If the recitation is held at night, change }“Tattha purimesu catūsu kiccesu padīpa-kiccaṁ idāni suriy’ālokassa atthitāya n’atthi. Aparāni tīṇi”{\itshape  to }“Tattha purimāni cattāri”\\

 {\itshape If there are bhikkhus outside of hatthapāsa but within the territory (sīmā) who have sent their consent and purity, then for a recitation during the day, the entire passage (between "Tattha...honti."): } “Tattha purimesu chasu kiccesu padīpa-kiccaṁ idāni suriy’ālokassa atthitāya n’atthi. Aparāni pañca
bhikkhūnaṁ vattaṁ jānantehi bhikkhūhi \textsuperscript{2} katāni pariniṭṭhitāni honti.”\\

{\itshape For a recitation at night in the same situation, the entire passage should be:}\\
“Tattha purimāni cha bhikkhūnaṁ vattaṁ jānantehi bhikkhūhi2 katāni pariniṭṭhitāni honti.”\\

2. {\itshape If sāmaṇeras help with the tasks, change} “bhikkhūhi” {\itshape to }“sāmaṇerehi-pi bhikkhūhi-pi”:{\itshape “Novices and bhikkhus….” If laypeople living in the monastery help with the tasks, change this to }“ārāmikehi-pi bhikkhūhi-pi”:{\itshape “Monastery dwellers and bhikkhus….”}\\


3. {\itshape Modify according to the season:}
\begin{itemize}[leftmargin=24pt, font=\scriptsize]
\item{\itshape winter season} - “hemantotu” 
\item{\itshape hot season} - “gimhotu.”
\item{\itshape rainy season} - “vassānotu.”
\end{itemize}

4. Please refer to the appropriate "Telling of the Season" on the opposite page.\\
A Pavarana section isn't complete at this time.\\
Additional info from dhammyut text here, not sure if consistent with the chart, but here for comparison during draft review...\\

(During a normal rainy season, change “aṭṭha uposathā” to “sattā ca uposathā ekā ca pavāraṇā”: “Seven uposathas and one pavāraṇā.”
During a hot or cold season with an additional month, change it to “adhikamāsa-vasena dasa uposathā”: “Because of the additional month, ten uposathās….”
During a rainy season with an additional month, change it to “adhikamāsa-vasena nava ca uposathā ekā ca pavāraṇā”: Because of the additional month, nine uposathas and one pavāraṇā….”)\\

5. If chanter is junior, "āvuso"

6. Substitute "cattāro" for the appropriate number of bhikkhus present, per the following page.

7. Substitute "cātuddaso" if it is a 14-day Uposatha


