% This is an auto-generated file. Do not edit!

\begin{samepage}
\ensurevspace{4\baselineskip}
\begin{leftcolumn*}
\item

\begingl[glneveryline={\EnglishGlossA,\EnglishGlossB}]
If[] any[] bhikkhu,[] not[] having[] asked[] (permission[] of)[] a[] bhikkhu[] who[] is[] present,[] should[] enter[] a[] village[] at[] the[] wrong[] time,[] except[] with[] an[] appropriate[] urgent[] duty,[] (this[] is[] a[] case)[] involving[] expiation.[]
\endgl
\end{leftcolumn*}
