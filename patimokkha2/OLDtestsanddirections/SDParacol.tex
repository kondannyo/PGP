\documentclass[11pt]{article}
\usepackage[margin=45pt,letter, landscape]{geometry}
\usepackage{titlesec}
\usepackage{paracol}
\usepackage{expex}
\usepackage[dvipsnames]{xcolor}
\usepackage{fontspec}
\setromanfont[BoldFont={Gentium Basic Bold},ItalicFont={Gentium Italic}]{Gentium}
\setcolumnwidth{310pt/60pt,310pt}

\sloppy
\raggedright

\newcommand{\vsp}{\vspace{2mm}}

%defining colors for grammatical highlighting

\newcommand{\NUL}[1]{\textcolor{Black}{#1}}
%noun
\newcommand{\NOM}[1]{\colorbox{pink}{#1}}
\newcommand{\ACC}[1]{\colorbox{Blue!20}{#1}}
\newcommand{\INS}[1]{\colorbox{yellow!20}{#1}}
\newcommand{\DAT}[1]{\colorbox{SpringGreen!20}{#1}}
\newcommand{\ABL}[1]{\colorbox{Orange!20}{#1}}
\newcommand{\GEN}[1]{\colorbox{Turquoise!20}{#1}}
\newcommand{\LOC}[1]{\colorbox{Purple!20}{#1}}
\newcommand{\VOC}[1]{\colorbox{Red!20}{#1}}

%verb
\newcommand{\ABS}[1]{\fcolorbox{Blue}{white}{#1}}
\newcommand{\OPT}[1]{\colorbox{green!30}{#1}}
\newcommand{\PRSPTCP}[1]{\colorbox{Apricot!20}{#1}}
\newcommand{\PRSIND}[1]{\colorbox{Melon!20}{#1}}


%Misc
\newcommand{\ADJ}[1]{\colorbox{Gray!20}{#1}}
\newcommand{\INDE}[1]{\colorbox{Thisle!20}{#1}}

%Glossing Definitions
\lingset{glstyle=nlevel,%sets the compile method for ExPex to the nlevel style (alternates words with []) as opposed to the wrap style uses gla glb separted lines
	glhangindent=10pt,% sets the indentation to the entire gloss after the first set of lines
	glossbreaking=true, %Allows for Glosses to break across pages
	glwordalign=left, %sets allignment on pairs of words, center is other option
	glneveryline={,\footnotesize\it}, %sets the font attributes per line {gla,glb...}
	glnabovelineskip={,-4pt}, %sets the space above the glossing {gla,glb,...}
	extraglskip=-2pt} %sets additional space between each line of glossing



\begin{document}

\begin{paracol}{2}
\begin{column}
{\itshape\footnotesize
Venerables, these thirteen cases (concerning) the community in the beginning and in the rest (of the procedure) come up for recitation.
}
\switchcolumn

\begin{flushleft}
Ime kho pan’āyasmanto terasa saṅghādisesā dhammā uddesaṁ āgacchanti.
\switchcolumn*
\end{flushleft}

{\itshape\footnotesize
1. The intentional emission of semen, except in a dream: (this is a case concerning) the community in the beginning and in the rest (of the procedure).
}
\switchcolumn

\begin{flushleft}
1. Sañcetanikā sukkavisaṭṭhi aññatra supinantā, saṅghādiseso.
\switchcolumn*
\end{flushleft}

{\itshape\footnotesize
2. If any bhikkhu, under the influence of an altered mind, should engage in (intimate) physical contact together with a woman (such as): the holding of a hand, or holding a braid (of hair), or caressing any limb: (this is a case concerning) the community in the beginning and in the rest (of the procedure).
}
\switchcolumn

\begin{flushleft}
2. Yo pana bhikkhu otiṇṇo vipariṇatena cittena mātugāmena saddhiṁ kāyasaṁsaggaṁ samāpajjeyya, hatthagāhaṁ vā veṇigāhaṁ vā aññatarassa vā aññatarassa vā aṅgassa parāmasanaṁ, saṅghādiseso.
\switchcolumn*
\end{flushleft}

{\itshape\footnotesize
3. If any bhikkhu, under the influence of an altered mind, should speak suggestively with depraved words to a woman, like a young man to a young woman, (with words) concerned with sexual intercourse: (this is a case concerning) the community in the beginning and in the rest (of the procedure).
}
\switchcolumn

\begin{flushleft}
3. Yo pana bhikkhu otiṇṇo vipariṇatena cittena mātugāmaṁ duṭṭhullāhi vācāhi obhāseyya, yathā taṁ yuvā yuvatiṁ methunūpasañhitāhi, saṅghādiseso.
\switchcolumn*
\end{flushleft}

{\itshape\footnotesize
4. If any bhikkhu, under the influence of an altered mind, (and) in the presence of a woman, should speak praise about the ministering to himself with sex: “Sister, this is the best of ministerings: she who would minister to a virtuous, good natured celibate like me with this act!,” (which is something) connected with sexual intercourse: (this is a case concerning) the community in the beginning and in the rest (of the procedure).
}
\switchcolumn

\begin{flushleft}
4. Yo pana bhikkhu otiṇṇo vipariṇatena cittena mātugāmassa santike attakāmapāricariyāya vaṇṇaṁ bhāseyya, “Etadaggaṁ bhagini pāricariyānaṁ, yā m’ādisaṁ sīlavantaṁ kalyāṇadhammaṁ brahmacāriṁ etena dhammena paricareyyā” ti, methunūpasañhitena, saṅghādiseso.
\switchcolumn*
\end{flushleft}

{\itshape\footnotesize
5. If any bhikkhu should engage in mediating a man's intention to a woman, or a woman's intention to a man, for being a wife or for being a mistress, even for being one on (just) that occasion: (this is a case concerning) the community in the beginning and in the rest (of the procedure).
}
\switchcolumn

\begin{flushleft}
5. Yo pana bhikkhu sañcarittaṁ samāpajjeyya, itthiyā vā purisamatiṁ, purisassa vā itthīmatiṁ, jāyattane vā jārattane vā antamaso taṁkhaṇikāya’pi, saṅghādiseso.
\switchcolumn*
\end{flushleft}

{\itshape\footnotesize
6. entailing harm (to creatures and which is) having a surrounding space. If a bhikkhu, having requested it himself, should have a hut built on a site entailing harm (to creatures), (and) not having a surrounding space, or if he should not bring bhikkhus to (it) for appointing the site, or if he should let (it) exceed the measure: (this is a case concerning) the community in the beginning and in the rest (of the procedure).
}
\switchcolumn

\begin{flushleft}
6. Saññācikāya pana bhikkhunā kuṭiṁ kārayamānena assāmikaṁ att’uddesaṁ pamāṇikā kāretabbā. Tatr’idaṁ pamāṇaṁ: dīghaso dvādasa vidatthiyo sugatavidatthiyā, tiriyaṁ satt’antarā. Bhikkhū abhinetabbā vatthudesanāya. Tehi bhikkhūhi vatthuṁ desetabbaṁ anārambhaṁ saparikkamanaṁ. Sārambhe ce bhikkhu vatthusmiṁ aparikkamane saññācikāya kuṭiṁ kāreyya, bhikkhū vā anabhineyya vatthudesanāya, pamāṇaṁ vā atikkāmeyya, saṅghādiseso.
\switchcolumn*
\end{flushleft}

{\itshape\footnotesize
7. By a bhikkhu who is having a large dwelling built, which has an owner, (and) is designated for himself, bhikkhus are to be brought to (it) for appointing the site. By those bhikkhus a site not entailing harm (to any creatures) (and) having a surrounding space is to be appointed. If a bhikkhu should have a hut built on a site entailing harm (to creatures), (and) not having a surrounding space, or if he should not bring bhikkhus to (it) for appointing the site, (this is a case concerning) the community in the beginning and in the rest (of the procedure).
}
\switchcolumn
\begin{flushleft}
7. Mahallakam pana bhikkhunā vihāraṁ kārayamānena, sassāmikaṁ att’uddesaṁ bhikkhū abhinetabbā vatthudesanāya. Tehi bhikkhūhi vatthuṁ desetabbaṁ anārambhaṁ saparikkamanaṁ. Sārambhe ce bhikkhu vatthusmiṁ aparikkamane mahallakaṁ vihāraṁ kāreyya, bhikkhū vā anabhineyya vatthudesanāya, saṅghādiseso.

\switchcolumn*
\end{flushleft}

{\itshape\footnotesize
8. If any bhikkhu, corrupted by malice (and) upset, should accuse a bhikkhu with a groundless case involving disqualification (thinking): “If only I could make him fall away from this holy life!,” (and) then, on another occasion, (whether) being interrogated or not being interrogated, if that legal issue is really groundless, and if the bhikkhu stands firm in malice: (this is a case concerning) the community in the beginning and in the rest (of the procedure).
}
\switchcolumn

\begin{flushleft}
8. Yo pana bhikkhu bhikkhuṁ duṭṭho doso appatīto amūlakena pārājikena dhammena anuddhaṁseyya, “App’eva nāma naṁ imamhā brahmacariyā cāveyyan” ti. Tato aparena samayena samanuggāhiyamāno vā asamanuggāhiyamāno vā, amūlakañc’eva taṁ adhikaraṇaṁ hoti, bhikkhu ca dosaṁ patiṭṭhāti, saṅghādiseso.
\switchcolumn*
\end{flushleft}

{\itshape\footnotesize
9. If any bhikkhu, corrupted by malice (and) upset, should accuse a bhikkhu with a case involving disqualification, having taken (it) up (with) some point, which is a mere pretext, of a legal issue belonging to another class (thinking): “If only I could make him fall away from this holy life!,” (and) then, on another occasion, (whether) being interrogated or not being interrogated, if that legal issue is really belonging to another class, (and) some point, which a mere pretext, has been taken up, and if the bhikkhu stands firm in malice: (this is a case concerning) the community in the beginning and in the rest (of the procedure).
}
\switchcolumn

\begin{flushleft}
9. Yo pana bhikkhu bhikkhuṁ duṭṭho doso appatīto aññabhāgiyassa adhikaraṇassa kiñci desaṁ lesamattaṁ upādāya pārājikena dhammena anuddhaṁseyya, “App’eva nāma naṁ imamhā brahmacariyā cāveyyan” ti. Tato aparena samayena samanuggāhiyamāno vā asamanuggāhiyamāno vā, aññabhāgiyañc’eva taṁ adhikaraṇaṁ hoti, koci deso lesamatto upādinno, bhikkhu ca dosaṁ patiṭṭhāti, saṅghādiseso.
\switchcolumn*
\end{flushleft}

{\itshape\footnotesize
10. If any bhikkhu should endeavor for the schism of a united community, or having undertaken, should persist in upholding a legal issue conducive to schism, (then) that bhikkhu should be spoken to thus by the bhikkhus: “Let the venerable one not endeavor for the schism of the united community, or having undertaken, persist in upholding a legal issue conducive to schism. Let the venerable one convene with the community, for a united community, which is on friendly terms, which is not disputing, which has a single recitation, dwells in comfort,” and (if) that bhikkhu being spoken to thus by the bhikkhus should persist in the same way (as before), (then) that bhikkhu is to be argued with up to three times by the bhikkhus for the relinquishing of that (course), (and if that bhikkhu,) being argued with up to three times, should relinquish that (course), then this is good, (but) if he should not relinquish (it): (this is a case concerning) the community in the beginning and in the rest (of the procedure).
}
\switchcolumn

\begin{flushleft}
10. Yo pana bhikkhu samaggassa saṅghassa bhedāya parakkameyya, bhedanasaṁvattanikaṁ vā adhikaraṇaṁ samādāya paggayha tiṭṭheyya, so bhikkhu bhikkhūhi evam assa vacanīyo, “Mā āyasmā samaggassa saṅghassa bhedāya parakkami. Bhedanasaṁvattanikaṁ vā adhikaraṇaṁ samādāya paggayha aṭṭhāsi. Samet’āyasmā saṅghena, samaggo hi saṅgho sammodamāno avivadamāno ek’uddeso phāsu viharatī” ti. Evañca so bhikkhu bhikkhūhi vuccamāno tath’eva paggaṇheyya, so bhikkhu bhikkhūhi yāvatatiyaṁ samanubhāsitabbo tassa paṭinissaggāya. Yāvatatiyañ’ce samanubhāsiyamāno taṁ paṭinissajjeyya, icc’etaṁ kusalaṁ. No ce paṭinissajjeyya, saṅghādiseso.
\switchcolumn*
\end{flushleft}

{\itshape\footnotesize
11. Now, there are bhikkhus who are followers of that same bhikkhu, (and) who are speaking for (his) faction: one, or two, or three, (and) they should say so: “Venerables, don't say anything to this bhikkhu! This bhikkhu is one who speaks in accordance with the Teaching and this bhikkhu is one who speaks in accordance the Discipline; this (bhikkhu), having received (our) consent and favour defines (the Teaching & Discipline). Knowing us, he speaks, (and) this suits us too.” (Then) those bhikkhus should be spoken to thus by the bhikkhus: “Venerables, don't say so! This bhikkhu does not speak in accordance with the Teaching, and this bhikkhu does not speak in accordance with the Discipline! Don't let the venerables too favour the schism of the community. Let there be convening with the community for the venerables, for a united community, which is on friendly terms, which is not disputing, which has a single recitation, dwells in comfort,” and (if) those bhikkhus being spoken to thus by the bhikkhus should persist in the same way (as before), (then) those bhikkhus are to be argued with up to three times by the bhikkhus for the relinquishing of that (course), (and if those bhikkhus) being argued with up to three times, should relinquish that (course), then this is good, (but) if they should not relinquish (it): (this is a case concerning) the community in the beginning and in the rest (of the procedure).
}
\switchcolumn

\begin{flushleft}
11. Tass’eva kho pana bhikkhussa bhikkhū honti anuvattakā vaggavādakā, eko vā dve vā tayo vā, te evaṁ vadeyyuṁ, “Mā āyasmanto etaṁ bhikkhuṁ kiñci avacuttha. Dhammavādī c’eso bhikkhu, vinayavādī c’eso bhikkhu, amhākañc’eso bhikkhu chandañca ruciñca ādāya voharati. Jānāti no bhāsati, amhākam’p’etaṁ khamatī” ti. Te bhikkhū bhikkhūhi evamassu vacanīyā, “Mā āyasmanto evaṁ avacuttha. Na c’eso bhikkhu dhammavādī, na c’eso bhikkhu vinayavādī. Mā āyasmantānam’pi saṅghabhedo rucittha. Samet’āyasmantānaṁ saṅghena, samaggo hi saṅgho sammodamāno avivadamāno ek’uddeso phāsu viharatī” ti. Evañca te bhikkhū bhikkhūhi vuccamānā tath’eva paggaṇheyyuṁ, te bhikkhū bhikkhūhi yāvatatiyaṁ samanubhāsitabbā tassa paṭinissaggāya. Yāvatatiyañce samanubhāsiyamānā taṁ paṭinissajjeyyuṁ, icc’etaṁ kusalaṁ. No ce paṭinissajjeyyuṁ, saṅghādiseso.
\switchcolumn*
\end{flushleft}

{\itshape\footnotesize
12. Now, a bhikkhu is of a nature difficult to be spoken to, (and when) being righteously spoken to by the bhikkhus about the training precepts included in the recitation, he makes himself (one) who can not be spoken to (saying): “Venerables, don't say anything good or bad to me, and I too shall not say anything good or bad to the venerables! Venerables, refrain from speaking to me!” (Then) that bhikkhu should be spoken to thus by the bhikkhus: “Let the venerable one one not make himself (one) who cannot be spoken to. Let the venerable one make himself (one) who can be spoken to. Let the venerable one speak to the bhikkhus with righteousness and the monks too will speak to the venerable one with righteousness. For the Blessed One's assembly has grown thus, that is, by the speaking of one to another, by the rehabilitating of one another,” and (if) that bhikkhu being spoken to thus by the bhikkhus should persist in the same way (as before), (then) that bhikkhu is to be argued with up to three times by the bhikkhus for the relinquishing of that (course), (and if that bhikkhu,) being argued with up to three times, should relinquish that (course), then this is good, (but) if he should not relinquish (it): (this is a case concerning) the community in the beginning and in the rest (of the procedure).
}
\switchcolumn

\begin{flushleft}
12. Bhikkhu pan’eva dubbacajātiko hoti, uddesapariyāpannesu sikkhāpadesu bhikkhūhi sahadhammikaṁ vuccamāno attānaṁ avacanīyaṁ karoti, “Mā maṁ āyasmanto kiñci avacuttha kalyāṇaṁ vā pāpakaṁ vā. Aham’p’āyasmante na kiñci vakkhāmi kalyāṇaṁ vā pāpakaṁ vā. Viramath’āyasmanto mama vacanāyā” ti. So bhikkhu bhikkhūhi evam’assa vacanīyo, “Mā āyasmā attānaṁ avacanīyaṁ akāsi. Vacanīyam’eva āyasmā attānaṁ karotu. Āyasmā’pi bhikkhū vadetu sahadhammena, bhikkhū’pi āyasmantaṁ vakkhanti sahadhammena. Evaṁ saṁvaḍḍhā hi tassa bhagavato parisā, yad’idaṁ aññamaññavacanena aññamaññavuṭṭhāpanenā” ti. Evañca so bhikkhu bhikkhūhi vuccamāno tath’eva paggaṇheyya, so bhikkhu bhikkhūhi yāvatatiyaṁ samanubhāsitabbo tassa paṭinissaggāya. Yāvatatiyañce samanubhāsiyamāno taṁ paṭinissajjeyya, icc’etaṁ kusalaṁ. No ce paṭinissajjeyya, saṅghādiseso.
\switchcolumn*
\end{flushleft}

{\itshape\footnotesize
13. Now, a bhikkhu lives dependent upon a certain village or town who is a spoiler of families, who is of bad behaviour. His bad behaviour is seen and is heard about, and the families spoilt by him are seen and heard about. That bhikkhu is to be spoken to thus by the bhikkhus: “The venerable one is a spoiler of families, one who is of bad behaviour. The bad behaviour of the venerable one is seen and is heard about, and the families spoilt by the venerable one are seen and are heard about. Let the venerable one depart from this dwelling-place! Enough of you dwelling here!” and (if) that bhikkhu being spoken to thus by the bhikkhus should say thus to those bhikkhus: “The bhikkhus are driven by desire; the bhikkhus are driven by anger; the bhikkhus are driven by delusion; the bhikkhus are driven by fear. They banish someone because of this kind of offence, (but) another one they do not banish.” (Then) that bhikkhu is to be spoken to thus by the bhikkhus: “Let the venerable one not speak thus! The bhikkhus are not driven by desire; and the bhikkhus are not driven by anger; and the bhikkhus are not driven by delusion; and the bhikkhus are not driven by fear. The venerable one is a spoiler of families, one who is of bad behaviour. The bad  behaviour  of  the venerable  one  is  seen and  is  heard  about,  and  the families   spoilt  by the venerable  one are seen and are heard  about.  Let  the venerable  one  depart  from  this  dwelling-place! Enough of you dwelling here!” and (if) that bhikkhu being spoken to thus by the bhikkhus should persist in the same way (as before), (then) that bhikkhu is to be argued with up to three times by the bhikkhus for the relinquishing of that (course), (and if that bhikkhu,) being argued with up to three times, should relinquish that (course), then this is good, (but) if he should not relinquish (it): (this is a case concerning) the community in the beginning and in the rest (of the procedure).
}
\switchcolumn

\begin{flushleft}
13. Bhikkhu pan’eva aññataraṁ gāmaṁ vā nigamaṁ vā upanissāya viharati kuladūsako pāpasamācāro. Tassa kho pāpakā samācārā dissanti c’eva suyyanti ca, kulāni ca tena duṭṭhāni dissanti c’eva suyyanti ca. So bhikkhu bhikkhūhi evam’assa vacanīyo, “Āyasmā kho kuladūsako pāpasamācāro. Āyasmato kho pāpakā samācārā dissanti c’eva suyyanti ca, kulāni c’āyasmatā duṭṭhāni dissanti c’eva suyyanti ca. Pakkamat’āyasmā imamhā āvāsā, alante idha vāsenā” ti. Evañca so bhikkhu bhikkhūhi vuccamāno te bhikkhū evaṁ vadeyya, “Chandagāmino ca bhikkhū, dosagāmino ca bhikkhū, mohagāmino ca bhikkhū, bhayagāmino ca bhikkhū, tādisikāya āpattiyā ekaccaṁ pabbājenti, ekaccaṁ na pabbājentī” ti. So bhikkhu bhikkhūhi evam’assa vacanīyo, “Mā āyasmā evaṁ avaca. Na ca bhikkhū chandagamino, na ca bhikkhū dosagāmino, na ca bhikkhū mohagāmino, na ca bhikkhū bhayagāmino. Āyasmā kho kuladūsako pāpasamācāro. Āyasmato kho pāpakā samācārā dissanti c’eva suyyanti ca, kulāni c’āyasmatā duṭṭhāni dissanti c’eva suyyanti ca. Pakkamat’āyasmā imamhā āvāsā, alan’te idha vāsenā” ti. Evañca so bhikkhu bhikkhūhi vuccamāno tath’eva paggaṇheyya, so bhikkhu bhikkhūhi yāvatatiyaṁ samanubhāsitabbo tassa paṭinissaggāya. Yāvatatiyañce samanubhāsiyamāno taṁ paṭinissajjeyya, icc’etaṁ kusalaṁ. No ce paṭinissajjeyya, saṅghādiseso.
\switchcolumn*
\end{flushleft}

{\itshape\footnotesize
Venerables, the thirteen cases (concerning) the community in the beginning and in the rest (of the procedure) have been recited, nine (cases) are of the offence-at-once (-class), four (cases) are of the up-to-the-third (time admonition-class). A bhikkhu who has committed any one of (these offenses), has to stay on probation with no choice (in the matter) for as many days as he knowingly conceals (it). Moreover, by a bhikkhu who has stayed on the probation, a six-night state of deference to (other) bhikkhus has to be entered upon. (When) the bhikkhu (is one by whom) the deference has been performed: wherever there may be a community of bhikkhus, which is a group of twenty (or more bhikkhus), there that bhikkhu should be reinstated. If a community of bhikkhus, which is a group of twenty deficient by even one (bhikkhu), should reinstate that bhikkhu (then) that bhikkhu is not reinstated, and those monks are blameworthy. This is the proper procedure here
}
\switchcolumn

\begin{flushleft}
Uddiṭṭhā kho āyasmanto terasa saṅghādisesā dhammā, nava paṭham’āpattikā cattāro yāvatatiyakā. Yesaṁ bhikkhu aññataraṁ vā aññataraṁ vā āpajjitvā yāvatihaṁ jānaṁ paṭicchādeti, tāvatihaṁ tena bhikkhunā akāmā parivatthabbaṁ. Parivutthaparivāsena bhikkhunā uttariṁ chārattaṁ, bhikkhumānattāya paṭipajjitabbaṁ. Ciṇṇamānatto bhikkhu, yattha siyā vīsatigaṇo bhikkhusaṅgho, tattha so bhikkhu abbhetabbo. Ekena’pi ce ūno vīsatigaṇo bhikkhusaṅgho taṁ bhikkhuṁ abbheyya, so ca bhikkhu anabbhito, te ca bhikkhū gārayhā. Ayaṁ tattha sāmīci.
\switchcolumn*
\end{flushleft}

{\itshape\footnotesize
Concerning that I ask the venerables: (Are you) pure in this? 
A second time again I ask: (Are you) pure in this? 
A third time again I ask: (Are you) pure in this? The venerables are pure in this, therefore there is silence, so do I bear this (in mind).
}
\switchcolumn

\begin{flushleft}
Tatth’āyasmante pucchāmi: Kacci’ttha parisuddhā?
Dutiyam’pi pucchāmi: Kacci’ttha parisuddhā?
Tatiyam’pi pucchāmi: Kacci’ttha parisuddhā?
Parisuddh’etth’āyasmanto, tasmā tuṇhī, evam’etaṁ dhārayāmi.
\switchcolumn*
\end{flushleft}

{\itshape\footnotesize
The recitation concerning the community in the beginning and the rest (of the procedure) is finished.
}
\switchcolumn

\begin{flushleft}
Saṅghādises’uddeso niṭṭhito
\switchcolumn*
\end{flushleft}


\end{column}
\end{paracol}

\end{document}
