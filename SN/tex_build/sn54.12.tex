
\begin{absolutelynopagebreak}
\setstretch{.7}
{\PaliGlossA{saṃyutta nikāya 54}}\\
\begin{addmargin}[1em]{2em}
\setstretch{.5}
{\PaliGlossB{Linked Discourses 54}}\\
\end{addmargin}
\end{absolutelynopagebreak}

\begin{absolutelynopagebreak}
\setstretch{.7}
{\PaliGlossA{2. dutiyavagga}}\\
\begin{addmargin}[1em]{2em}
\setstretch{.5}
{\PaliGlossB{2. The Second Chapter}}\\
\end{addmargin}
\end{absolutelynopagebreak}

\begin{absolutelynopagebreak}
\setstretch{.7}
{\PaliGlossA{12. kaṅkheyyasutta}}\\
\begin{addmargin}[1em]{2em}
\setstretch{.5}
{\PaliGlossB{12. In Doubt}}\\
\end{addmargin}
\end{absolutelynopagebreak}

\begin{absolutelynopagebreak}
\setstretch{.7}
{\PaliGlossA{ekaṃ samayaṃ āyasmā lomasakaṃbhiyo sakkesu viharati kapilavatthusmiṃ nigrodhārāme.}}\\
\begin{addmargin}[1em]{2em}
\setstretch{.5}
{\PaliGlossB{At one time Venerable Lomasavaṅgīsa was staying in the land of the Sakyans, near Kapilavatthu in the Banyan Tree Monastery.}}\\
\end{addmargin}
\end{absolutelynopagebreak}

\begin{absolutelynopagebreak}
\setstretch{.7}
{\PaliGlossA{atha kho mahānāmo sakko yenāyasmā lomasakaṃbhiyo tenupasaṅkami; upasaṅkamitvā āyasmantaṃ lomasakaṃbhiyaṃ abhivādetvā ekamantaṃ nisīdi. ekamantaṃ nisinno kho mahānāmo sakko āyasmantaṃ lomasakaṃbhiyaṃ etadavoca:}}\\
\begin{addmargin}[1em]{2em}
\setstretch{.5}
{\PaliGlossB{Then Mahānāma the Sakyan went up to Venerable Lomasavaṅgīsa, bowed, sat down to one side, and said to him,}}\\
\end{addmargin}
\end{absolutelynopagebreak}

\begin{absolutelynopagebreak}
\setstretch{.7}
{\PaliGlossA{“so eva nu kho, bhante, sekho vihāro so tathāgatavihāro, udāhu aññova sekho vihāro añño tathāgatavihāro”ti?}}\\
\begin{addmargin}[1em]{2em}
\setstretch{.5}
{\PaliGlossB{“Sir, is the meditation of a trainee just the same as the meditation of a Realized One? Or is the meditation of a trainee different from the meditation of a Realized One?”}}\\
\end{addmargin}
\end{absolutelynopagebreak}

\begin{absolutelynopagebreak}
\setstretch{.7}
{\PaliGlossA{“na kho, āvuso mahānāma, sveva sekho vihāro, so tathāgatavihāro.}}\\
\begin{addmargin}[1em]{2em}
\setstretch{.5}
{\PaliGlossB{“Reverend Mahānāma, the meditation of a trainee and a realized one are not the same;}}\\
\end{addmargin}
\end{absolutelynopagebreak}

\begin{absolutelynopagebreak}
\setstretch{.7}
{\PaliGlossA{añño kho, āvuso mahānāma, sekho vihāro, añño tathāgatavihāro.}}\\
\begin{addmargin}[1em]{2em}
\setstretch{.5}
{\PaliGlossB{they are different.}}\\
\end{addmargin}
\end{absolutelynopagebreak}

\begin{absolutelynopagebreak}
\setstretch{.7}
{\PaliGlossA{ye te, āvuso mahānāma, bhikkhū sekhā appattamānasā anuttaraṃ yogakkhemaṃ patthayamānā viharanti, te pañca nīvaraṇe pahāya viharanti.}}\\
\begin{addmargin}[1em]{2em}
\setstretch{.5}
{\PaliGlossB{Those mendicants who are trainees haven’t achieved their heart’s desire, but live aspiring for the supreme sanctuary. They meditate after giving up the five hindrances.}}\\
\end{addmargin}
\end{absolutelynopagebreak}

\begin{absolutelynopagebreak}
\setstretch{.7}
{\PaliGlossA{katame pañca?}}\\
\begin{addmargin}[1em]{2em}
\setstretch{.5}
{\PaliGlossB{What five?}}\\
\end{addmargin}
\end{absolutelynopagebreak}

\begin{absolutelynopagebreak}
\setstretch{.7}
{\PaliGlossA{kāmacchandanīvaraṇaṃ pahāya viharanti, byāpādanīvaraṇaṃ … pe …}}\\
\begin{addmargin}[1em]{2em}
\setstretch{.5}
{\PaliGlossB{The hindrances of sensual desire, ill will,}}\\
\end{addmargin}
\end{absolutelynopagebreak}

\begin{absolutelynopagebreak}
\setstretch{.7}
{\PaliGlossA{thinamiddhanīvaraṇaṃ …}}\\
\begin{addmargin}[1em]{2em}
\setstretch{.5}
{\PaliGlossB{dullness and drowsiness,}}\\
\end{addmargin}
\end{absolutelynopagebreak}

\begin{absolutelynopagebreak}
\setstretch{.7}
{\PaliGlossA{uddhaccakukkuccanīvaraṇaṃ …}}\\
\begin{addmargin}[1em]{2em}
\setstretch{.5}
{\PaliGlossB{restlessness and remorse,}}\\
\end{addmargin}
\end{absolutelynopagebreak}

\begin{absolutelynopagebreak}
\setstretch{.7}
{\PaliGlossA{vicikicchānīvaraṇaṃ pahāya viharanti.}}\\
\begin{addmargin}[1em]{2em}
\setstretch{.5}
{\PaliGlossB{and doubt.}}\\
\end{addmargin}
\end{absolutelynopagebreak}

\begin{absolutelynopagebreak}
\setstretch{.7}
{\PaliGlossA{yepi te, āvuso mahānāma, bhikkhū sekhā appattamānasā anuttaraṃ yogakkhemaṃ patthayamānā viharanti, te ime pañca nīvaraṇe pahāya viharanti.}}\\
\begin{addmargin}[1em]{2em}
\setstretch{.5}
{\PaliGlossB{Those who are trainee mendicants … meditate after giving up the five hindrances.}}\\
\end{addmargin}
\end{absolutelynopagebreak}

\begin{absolutelynopagebreak}
\setstretch{.7}
{\PaliGlossA{ye ca kho te, āvuso mahānāma, bhikkhū arahanto khīṇāsavā vusitavanto katakaraṇīyā ohitabhārā anuppattasadatthā parikkhīṇabhavasaṃyojanā sammadaññāvimuttā, tesaṃ pañca nīvaraṇā pahīnā ucchinnamūlā tālāvatthukatā anabhāvaṃkatā āyatiṃ anuppādadhammā.}}\\
\begin{addmargin}[1em]{2em}
\setstretch{.5}
{\PaliGlossB{Those mendicants who are perfected—who have ended the defilements, completed the spiritual journey, done what had to be done, laid down the burden, achieved their own goal, utterly ended the fetters of rebirth, and are rightly freed through enlightenment—for them, the five hindrances are cut off at the root, made like a palm stump, obliterated, and unable to arise in the future.}}\\
\end{addmargin}
\end{absolutelynopagebreak}

\begin{absolutelynopagebreak}
\setstretch{.7}
{\PaliGlossA{katame pañca?}}\\
\begin{addmargin}[1em]{2em}
\setstretch{.5}
{\PaliGlossB{What five?}}\\
\end{addmargin}
\end{absolutelynopagebreak}

\begin{absolutelynopagebreak}
\setstretch{.7}
{\PaliGlossA{kāmacchandanīvaraṇaṃ pahīnaṃ ucchinnamūlaṃ tālāvatthukataṃ anabhāvaṃkataṃ āyatiṃ anuppādadhammaṃ;}}\\
\begin{addmargin}[1em]{2em}
\setstretch{.5}
{\PaliGlossB{The hindrances of sensual desire,}}\\
\end{addmargin}
\end{absolutelynopagebreak}

\begin{absolutelynopagebreak}
\setstretch{.7}
{\PaliGlossA{byāpādanīvaraṇaṃ pahīnaṃ … pe …}}\\
\begin{addmargin}[1em]{2em}
\setstretch{.5}
{\PaliGlossB{ill will,}}\\
\end{addmargin}
\end{absolutelynopagebreak}

\begin{absolutelynopagebreak}
\setstretch{.7}
{\PaliGlossA{thinamiddhanīvaraṇaṃ …}}\\
\begin{addmargin}[1em]{2em}
\setstretch{.5}
{\PaliGlossB{dullness and drowsiness,}}\\
\end{addmargin}
\end{absolutelynopagebreak}

\begin{absolutelynopagebreak}
\setstretch{.7}
{\PaliGlossA{uddhaccakukkuccanīvaraṇaṃ …}}\\
\begin{addmargin}[1em]{2em}
\setstretch{.5}
{\PaliGlossB{restlessness and remorse,}}\\
\end{addmargin}
\end{absolutelynopagebreak}

\begin{absolutelynopagebreak}
\setstretch{.7}
{\PaliGlossA{vicikicchānīvaraṇaṃ pahīnaṃ ucchinnamūlaṃ tālāvatthukataṃ anabhāvaṃkataṃ āyatiṃ anuppādadhammaṃ.}}\\
\begin{addmargin}[1em]{2em}
\setstretch{.5}
{\PaliGlossB{and doubt.}}\\
\end{addmargin}
\end{absolutelynopagebreak}

\begin{absolutelynopagebreak}
\setstretch{.7}
{\PaliGlossA{ye te, āvuso mahānāma, bhikkhū arahanto khīṇāsavā vusitavanto katakaraṇīyā ohitabhārā anuppattasadatthā parikkhīṇabhavasaṃyojanā sammadaññāvimuttā, tesaṃ ime pañca nīvaraṇā pahīnā ucchinnamūlā tālāvatthukatā anabhāvaṅkatā āyatiṃ anuppādadhammā.}}\\
\begin{addmargin}[1em]{2em}
\setstretch{.5}
{\PaliGlossB{Those mendicants who are perfected—who have ended the defilements … for them, the five hindrances are cut off at the root … and unable to arise in the future.}}\\
\end{addmargin}
\end{absolutelynopagebreak}

\begin{absolutelynopagebreak}
\setstretch{.7}
{\PaliGlossA{tadamināpetaṃ, āvuso mahānāma, pariyāyena veditabbaṃ yathā—}}\\
\begin{addmargin}[1em]{2em}
\setstretch{.5}
{\PaliGlossB{And here’s another way to understand how}}\\
\end{addmargin}
\end{absolutelynopagebreak}

\begin{absolutelynopagebreak}
\setstretch{.7}
{\PaliGlossA{aññova sekho vihāro, añño tathāgatavihāro.}}\\
\begin{addmargin}[1em]{2em}
\setstretch{.5}
{\PaliGlossB{the meditation of a trainee and a realized one are different.}}\\
\end{addmargin}
\end{absolutelynopagebreak}

\begin{absolutelynopagebreak}
\setstretch{.7}
{\PaliGlossA{ekamidaṃ, āvuso mahānāma, samayaṃ bhagavā icchānaṅgale viharati icchānaṅgalavanasaṇḍe.}}\\
\begin{addmargin}[1em]{2em}
\setstretch{.5}
{\PaliGlossB{At one time the Buddha was staying in a forest near Icchānaṅgala.}}\\
\end{addmargin}
\end{absolutelynopagebreak}

\begin{absolutelynopagebreak}
\setstretch{.7}
{\PaliGlossA{tatra kho, āvuso mahānāma, bhagavā bhikkhū āmantesi:}}\\
\begin{addmargin}[1em]{2em}
\setstretch{.5}
{\PaliGlossB{There he addressed the mendicants,}}\\
\end{addmargin}
\end{absolutelynopagebreak}

\begin{absolutelynopagebreak}
\setstretch{.7}
{\PaliGlossA{‘icchāmahaṃ, bhikkhave, temāsaṃ paṭisallīyituṃ.}}\\
\begin{addmargin}[1em]{2em}
\setstretch{.5}
{\PaliGlossB{‘Mendicants, I wish to go on retreat for three months.}}\\
\end{addmargin}
\end{absolutelynopagebreak}

\begin{absolutelynopagebreak}
\setstretch{.7}
{\PaliGlossA{nāmhi kenaci upasaṅkamitabbo, aññatra ekena piṇḍapātanīhārakenā’ti.}}\\
\begin{addmargin}[1em]{2em}
\setstretch{.5}
{\PaliGlossB{No-one should approach me, except for the one who brings my alms-food.’}}\\
\end{addmargin}
\end{absolutelynopagebreak}

\begin{absolutelynopagebreak}
\setstretch{.7}
{\PaliGlossA{‘evaṃ, bhante’ti kho, āvuso mahānāma, te bhikkhū bhagavato paṭissutvā nāssudha koci bhagavantaṃ upasaṅkamati, aññatra ekena piṇḍapātanīhārakena.}}\\
\begin{addmargin}[1em]{2em}
\setstretch{.5}
{\PaliGlossB{‘Yes, sir,’ replied those mendicants. And no-one approached him, except for the one who brought the alms-food.}}\\
\end{addmargin}
\end{absolutelynopagebreak}

\begin{absolutelynopagebreak}
\setstretch{.7}
{\PaliGlossA{atha kho, āvuso, bhagavā tassa temāsassa accayena paṭisallānā vuṭṭhito bhikkhū āmantesi:}}\\
\begin{addmargin}[1em]{2em}
\setstretch{.5}
{\PaliGlossB{Then after three months had passed, the Buddha came out of retreat and addressed the mendicants:}}\\
\end{addmargin}
\end{absolutelynopagebreak}

\begin{absolutelynopagebreak}
\setstretch{.7}
{\PaliGlossA{‘sace kho, bhikkhave, aññatitthiyā paribbājakā evaṃ puccheyyuṃ:}}\\
\begin{addmargin}[1em]{2em}
\setstretch{.5}
{\PaliGlossB{‘Mendicants, if wanderers who follow another path were to ask you:}}\\
\end{addmargin}
\end{absolutelynopagebreak}

\begin{absolutelynopagebreak}
\setstretch{.7}
{\PaliGlossA{“katamenāvuso, vihārena samaṇo gotamo vassāvāsaṃ bahulaṃ vihāsī”ti, evaṃ puṭṭhā tumhe, bhikkhave, tesaṃ aññatitthiyānaṃ paribbājakānaṃ evaṃ byākareyyātha:}}\\
\begin{addmargin}[1em]{2em}
\setstretch{.5}
{\PaliGlossB{“Reverends, what was the ascetic Gotama’s usual meditation during the rainy season residence?” You should answer them like this:}}\\
\end{addmargin}
\end{absolutelynopagebreak}

\begin{absolutelynopagebreak}
\setstretch{.7}
{\PaliGlossA{“ānāpānassatisamādhinā kho, āvuso, bhagavā vassāvāsaṃ bahulaṃ vihāsī”’ti.}}\\
\begin{addmargin}[1em]{2em}
\setstretch{.5}
{\PaliGlossB{“Reverends, the ascetic Gotama’s usual meditation during the rainy season residence was immersion due to mindfulness of breathing.”}}\\
\end{addmargin}
\end{absolutelynopagebreak}

\begin{absolutelynopagebreak}
\setstretch{.7}
{\PaliGlossA{idhāhaṃ, bhikkhave, sato assasāmi, sato passasāmi.}}\\
\begin{addmargin}[1em]{2em}
\setstretch{.5}
{\PaliGlossB{In this regard: mindful, I breathe in. Mindful, I breathe out.}}\\
\end{addmargin}
\end{absolutelynopagebreak}

\begin{absolutelynopagebreak}
\setstretch{.7}
{\PaliGlossA{dīghaṃ assasanto dīghaṃ assasāmīti pajānāmi, dīghaṃ passasanto dīghaṃ passasāmīti pajānāmi … pe …}}\\
\begin{addmargin}[1em]{2em}
\setstretch{.5}
{\PaliGlossB{When breathing in heavily I know: ‘I’m breathing in heavily.’ When breathing out heavily I know: ‘I’m breathing out heavily.’ …}}\\
\end{addmargin}
\end{absolutelynopagebreak}

\begin{absolutelynopagebreak}
\setstretch{.7}
{\PaliGlossA{paṭinissaggānupassī assasissāmīti pajānāmi, paṭinissaggānupassī passasissāmīti pajānāmi.}}\\
\begin{addmargin}[1em]{2em}
\setstretch{.5}
{\PaliGlossB{I know: “I’ll breathe in observing letting go.” I know: “I’ll breathe out observing letting go.”}}\\
\end{addmargin}
\end{absolutelynopagebreak}

\begin{absolutelynopagebreak}
\setstretch{.7}
{\PaliGlossA{yañhi taṃ, bhikkhave, sammā vadamāno vadeyya—}}\\
\begin{addmargin}[1em]{2em}
\setstretch{.5}
{\PaliGlossB{For if anything should be rightly called}}\\
\end{addmargin}
\end{absolutelynopagebreak}

\begin{absolutelynopagebreak}
\setstretch{.7}
{\PaliGlossA{ariyavihāro itipi, brahmavihāro itipi, tathāgatavihāro itipi.}}\\
\begin{addmargin}[1em]{2em}
\setstretch{.5}
{\PaliGlossB{“the meditation of a noble one”, or else “the meditation of a Brahmā”, or else “the meditation of a realized one”,}}\\
\end{addmargin}
\end{absolutelynopagebreak}

\begin{absolutelynopagebreak}
\setstretch{.7}
{\PaliGlossA{ānāpānassatisamādhiṃ sammā vadamāno vadeyya—}}\\
\begin{addmargin}[1em]{2em}
\setstretch{.5}
{\PaliGlossB{it’s immersion due to mindfulness of breathing.}}\\
\end{addmargin}
\end{absolutelynopagebreak}

\begin{absolutelynopagebreak}
\setstretch{.7}
{\PaliGlossA{ariyavihāro itipi, brahmavihāro itipi, tathāgatavihāro itipi.}}\\
\begin{addmargin}[1em]{2em}
\setstretch{.5}
{\PaliGlossB{    -}}\\
\end{addmargin}
\end{absolutelynopagebreak}

\begin{absolutelynopagebreak}
\setstretch{.7}
{\PaliGlossA{ye te, bhikkhave, bhikkhū sekhā appattamānasā anuttaraṃ yogakkhemaṃ patthayamānā viharanti, tesaṃ ānāpānassatisamādhi bhāvito bahulīkato āsavānaṃ khayāya saṃvattati.}}\\
\begin{addmargin}[1em]{2em}
\setstretch{.5}
{\PaliGlossB{For those mendicants who are trainees—who haven’t achieved their heart’s desire, but live aspiring for the supreme sanctuary—the development and cultivation of immersion due to mindfulness of breathing leads to the ending of defilements.}}\\
\end{addmargin}
\end{absolutelynopagebreak}

\begin{absolutelynopagebreak}
\setstretch{.7}
{\PaliGlossA{ye ca kho te, bhikkhave, bhikkhū arahanto khīṇāsavā vusitavanto katakaraṇīyā ohitabhārā anuppattasadatthā parikkhīṇabhavasaṃyojanā sammadaññāvimuttā, tesaṃ ānāpānassatisamādhi bhāvito bahulīkato diṭṭheva dhamme sukhavihārāya ceva saṃvattati satisampajaññāya ca.}}\\
\begin{addmargin}[1em]{2em}
\setstretch{.5}
{\PaliGlossB{For those mendicants who are perfected—who have ended the defilements, completed the spiritual journey, done what had to be done, laid down the burden, achieved their own goal, utterly ended the fetters of rebirth, and are rightly freed through enlightenment—the development and cultivation of immersion due to mindfulness of breathing leads to blissful meditation in the present life, and to mindfulness and awareness.}}\\
\end{addmargin}
\end{absolutelynopagebreak}

\begin{absolutelynopagebreak}
\setstretch{.7}
{\PaliGlossA{yañhi taṃ, bhikkhave, sammā vadamāno vadeyya—}}\\
\begin{addmargin}[1em]{2em}
\setstretch{.5}
{\PaliGlossB{For if anything should be rightly called}}\\
\end{addmargin}
\end{absolutelynopagebreak}

\begin{absolutelynopagebreak}
\setstretch{.7}
{\PaliGlossA{ariyavihāro itipi, brahmavihāro itipi, tathāgatavihāro itipi.}}\\
\begin{addmargin}[1em]{2em}
\setstretch{.5}
{\PaliGlossB{“the meditation of a noble one”, or else “the meditation of a Brahmā”, or else “the meditation of a realized one”,}}\\
\end{addmargin}
\end{absolutelynopagebreak}

\begin{absolutelynopagebreak}
\setstretch{.7}
{\PaliGlossA{ānāpānassatisamādhiṃ sammā vadamāno vadeyya—}}\\
\begin{addmargin}[1em]{2em}
\setstretch{.5}
{\PaliGlossB{it’s immersion due to mindfulness of breathing.’}}\\
\end{addmargin}
\end{absolutelynopagebreak}

\begin{absolutelynopagebreak}
\setstretch{.7}
{\PaliGlossA{ariyavihāro itipi, brahmavihāro itipi, tathāgatavihāro itipīti.}}\\
\begin{addmargin}[1em]{2em}
\setstretch{.5}
{\PaliGlossB{    -}}\\
\end{addmargin}
\end{absolutelynopagebreak}

\begin{absolutelynopagebreak}
\setstretch{.7}
{\PaliGlossA{iminā kho etaṃ, āvuso mahānāma, pariyāyena veditabbaṃ, yathā—}}\\
\begin{addmargin}[1em]{2em}
\setstretch{.5}
{\PaliGlossB{This is another way to understand how}}\\
\end{addmargin}
\end{absolutelynopagebreak}

\begin{absolutelynopagebreak}
\setstretch{.7}
{\PaliGlossA{aññova sekho vihāro, añño tathāgatavihāro”ti.}}\\
\begin{addmargin}[1em]{2em}
\setstretch{.5}
{\PaliGlossB{the meditation of a trainee and a realized one are different.”}}\\
\end{addmargin}
\end{absolutelynopagebreak}

\begin{absolutelynopagebreak}
\setstretch{.7}
{\PaliGlossA{dutiyaṃ.}}\\
\begin{addmargin}[1em]{2em}
\setstretch{.5}
{\PaliGlossB{    -}}\\
\end{addmargin}
\end{absolutelynopagebreak}
