
\begin{absolutelynopagebreak}
\setstretch{.7}
{\PaliGlossA{saṃyutta nikāya 16}}\\
\begin{addmargin}[1em]{2em}
\setstretch{.5}
{\PaliGlossB{Linked Discourses 16}}\\
\end{addmargin}
\end{absolutelynopagebreak}

\begin{absolutelynopagebreak}
\setstretch{.7}
{\PaliGlossA{1. kassapavagga}}\\
\begin{addmargin}[1em]{2em}
\setstretch{.5}
{\PaliGlossB{1. Kassapa}}\\
\end{addmargin}
\end{absolutelynopagebreak}

\begin{absolutelynopagebreak}
\setstretch{.7}
{\PaliGlossA{6. ovādasutta}}\\
\begin{addmargin}[1em]{2em}
\setstretch{.5}
{\PaliGlossB{6. Advice}}\\
\end{addmargin}
\end{absolutelynopagebreak}

\begin{absolutelynopagebreak}
\setstretch{.7}
{\PaliGlossA{rājagahe veḷuvane.}}\\
\begin{addmargin}[1em]{2em}
\setstretch{.5}
{\PaliGlossB{Near Rājagaha, in the Bamboo Grove.}}\\
\end{addmargin}
\end{absolutelynopagebreak}

\begin{absolutelynopagebreak}
\setstretch{.7}
{\PaliGlossA{atha kho āyasmā mahākassapo yena bhagavā tenupasaṅkami; upasaṅkamitvā bhagavantaṃ abhivādetvā ekamantaṃ nisīdi. ekamantaṃ nisinnaṃ kho āyasmantaṃ mahākassapaṃ bhagavā etadavoca:}}\\
\begin{addmargin}[1em]{2em}
\setstretch{.5}
{\PaliGlossB{Then Venerable Mahākassapa went up to the Buddha, bowed, and sat down to one side. The Buddha said to him:}}\\
\end{addmargin}
\end{absolutelynopagebreak}

\begin{absolutelynopagebreak}
\setstretch{.7}
{\PaliGlossA{“ovada, kassapa, bhikkhū;}}\\
\begin{addmargin}[1em]{2em}
\setstretch{.5}
{\PaliGlossB{“Kassapa, advise the mendicants!}}\\
\end{addmargin}
\end{absolutelynopagebreak}

\begin{absolutelynopagebreak}
\setstretch{.7}
{\PaliGlossA{karohi, kassapa, bhikkhūnaṃ dhammiṃ kathaṃ.}}\\
\begin{addmargin}[1em]{2em}
\setstretch{.5}
{\PaliGlossB{Give them a Dhamma talk!}}\\
\end{addmargin}
\end{absolutelynopagebreak}

\begin{absolutelynopagebreak}
\setstretch{.7}
{\PaliGlossA{ahaṃ vā, kassapa, bhikkhū ovadeyyaṃ tvaṃ vā;}}\\
\begin{addmargin}[1em]{2em}
\setstretch{.5}
{\PaliGlossB{Either you or I should advise the mendicants}}\\
\end{addmargin}
\end{absolutelynopagebreak}

\begin{absolutelynopagebreak}
\setstretch{.7}
{\PaliGlossA{ahaṃ vā bhikkhūnaṃ dhammiṃ kathaṃ kareyyaṃ tvaṃ vā”ti.}}\\
\begin{addmargin}[1em]{2em}
\setstretch{.5}
{\PaliGlossB{and give them a Dhamma talk.”}}\\
\end{addmargin}
\end{absolutelynopagebreak}

\begin{absolutelynopagebreak}
\setstretch{.7}
{\PaliGlossA{“dubbacā kho, bhante, etarahi bhikkhū, dovacassakaraṇehi dhammehi samannāgatā, akkhamā, appadakkhiṇaggāhino anusāsaniṃ.}}\\
\begin{addmargin}[1em]{2em}
\setstretch{.5}
{\PaliGlossB{“Sir, the mendicants these days are hard to admonish, having qualities that make them hard to admonish. They’re impatient, and don’t take instruction respectfully.}}\\
\end{addmargin}
\end{absolutelynopagebreak}

\begin{absolutelynopagebreak}
\setstretch{.7}
{\PaliGlossA{idhāhaṃ, bhante, addasaṃ bhaṇḍañca nāma bhikkhuṃ ānandassa saddhivihāriṃ abhijikañca nāma bhikkhuṃ anuruddhassa saddhivihāriṃ aññamaññaṃ sutena accāvadante:}}\\
\begin{addmargin}[1em]{2em}
\setstretch{.5}
{\PaliGlossB{Take the monk called Bhaṇḍa, Ānanda’s pupil. He’s been competing in studies with the monk called Abhiñjika, Anuruddha’s pupil. They say:}}\\
\end{addmargin}
\end{absolutelynopagebreak}

\begin{absolutelynopagebreak}
\setstretch{.7}
{\PaliGlossA{‘ehi, bhikkhu, ko bahutaraṃ bhāsissati, ko sundarataraṃ bhāsissati, ko cirataraṃ bhāsissatī’”ti.}}\\
\begin{addmargin}[1em]{2em}
\setstretch{.5}
{\PaliGlossB{‘Come on, monk, who can recite more? Who can recite better? Who can recite longer?’”}}\\
\end{addmargin}
\end{absolutelynopagebreak}

\begin{absolutelynopagebreak}
\setstretch{.7}
{\PaliGlossA{atha kho bhagavā aññataraṃ bhikkhuṃ āmantesi:}}\\
\begin{addmargin}[1em]{2em}
\setstretch{.5}
{\PaliGlossB{So the Buddha said to a certain monk,}}\\
\end{addmargin}
\end{absolutelynopagebreak}

\begin{absolutelynopagebreak}
\setstretch{.7}
{\PaliGlossA{“ehi tvaṃ, bhikkhu, mama vacanena bhaṇḍañca bhikkhuṃ ānandassa saddhivihāriṃ abhijikañca bhikkhuṃ anuruddhassa saddhivihāriṃ āmantehi:}}\\
\begin{addmargin}[1em]{2em}
\setstretch{.5}
{\PaliGlossB{“Please, monk, in my name tell the monk called Bhaṇḍa, Ānanda’s pupil, and the monk called Abhiñjika, Anuruddha’s pupil that}}\\
\end{addmargin}
\end{absolutelynopagebreak}

\begin{absolutelynopagebreak}
\setstretch{.7}
{\PaliGlossA{‘satthā āyasmante āmantetī’”ti.}}\\
\begin{addmargin}[1em]{2em}
\setstretch{.5}
{\PaliGlossB{the teacher summons them.”}}\\
\end{addmargin}
\end{absolutelynopagebreak}

\begin{absolutelynopagebreak}
\setstretch{.7}
{\PaliGlossA{“evaṃ, bhante”ti kho so bhikkhu bhagavato paṭissutvā yena te bhikkhū tenupasaṅkami; upasaṅkamitvā te bhikkhū etadavoca:}}\\
\begin{addmargin}[1em]{2em}
\setstretch{.5}
{\PaliGlossB{“Yes, sir,” that monk replied. He went to those monks and said,}}\\
\end{addmargin}
\end{absolutelynopagebreak}

\begin{absolutelynopagebreak}
\setstretch{.7}
{\PaliGlossA{“satthā āyasmante āmantetī”ti.}}\\
\begin{addmargin}[1em]{2em}
\setstretch{.5}
{\PaliGlossB{“Venerables, the teacher summons you.”}}\\
\end{addmargin}
\end{absolutelynopagebreak}

\begin{absolutelynopagebreak}
\setstretch{.7}
{\PaliGlossA{“evamāvuso”ti kho te bhikkhū tassa bhikkhuno paṭissutvā yena bhagavā tenupasaṅkamiṃsu; upasaṅkamitvā bhagavantaṃ abhivādetvā ekamantaṃ nisīdiṃsu. ekamantaṃ nisinne kho te bhikkhū bhagavā etadavoca:}}\\
\begin{addmargin}[1em]{2em}
\setstretch{.5}
{\PaliGlossB{“Yes, reverend,” those monks replied. They went to the Buddha, bowed, and sat down to one side. The Buddha said to them:}}\\
\end{addmargin}
\end{absolutelynopagebreak}

\begin{absolutelynopagebreak}
\setstretch{.7}
{\PaliGlossA{“saccaṃ kira tumhe, bhikkhave, aññamaññaṃ sutena accāvadatha:}}\\
\begin{addmargin}[1em]{2em}
\setstretch{.5}
{\PaliGlossB{“Is it really true, monks, that you’ve been competing in studies, saying:}}\\
\end{addmargin}
\end{absolutelynopagebreak}

\begin{absolutelynopagebreak}
\setstretch{.7}
{\PaliGlossA{‘ehi, bhikkhu, ko bahutaraṃ bhāsissati, ko sundarataraṃ bhāsissati, ko cirataraṃ bhāsissatī’”ti?}}\\
\begin{addmargin}[1em]{2em}
\setstretch{.5}
{\PaliGlossB{‘Come on, monk, who can recite more? Who can recite better? Who can recite longer?’”}}\\
\end{addmargin}
\end{absolutelynopagebreak}

\begin{absolutelynopagebreak}
\setstretch{.7}
{\PaliGlossA{“evaṃ, bhante”.}}\\
\begin{addmargin}[1em]{2em}
\setstretch{.5}
{\PaliGlossB{“Yes, sir.”}}\\
\end{addmargin}
\end{absolutelynopagebreak}

\begin{absolutelynopagebreak}
\setstretch{.7}
{\PaliGlossA{“kiṃ nu kho me tumhe, bhikkhave, evaṃ dhammaṃ desitaṃ ājānātha:}}\\
\begin{addmargin}[1em]{2em}
\setstretch{.5}
{\PaliGlossB{“Have you ever known me to teach the Dhamma like this:}}\\
\end{addmargin}
\end{absolutelynopagebreak}

\begin{absolutelynopagebreak}
\setstretch{.7}
{\PaliGlossA{‘etha tumhe, bhikkhave, aññamaññaṃ sutena accāvadatha—}}\\
\begin{addmargin}[1em]{2em}
\setstretch{.5}
{\PaliGlossB{‘Please mendicants, compete in studies to}}\\
\end{addmargin}
\end{absolutelynopagebreak}

\begin{absolutelynopagebreak}
\setstretch{.7}
{\PaliGlossA{ehi, bhikkhu, ko bahutaraṃ bhāsissati, ko sundarataraṃ bhāsissati, ko cirataraṃ bhāsissatī’”ti?}}\\
\begin{addmargin}[1em]{2em}
\setstretch{.5}
{\PaliGlossB{see who can recite more and better and longer’?”}}\\
\end{addmargin}
\end{absolutelynopagebreak}

\begin{absolutelynopagebreak}
\setstretch{.7}
{\PaliGlossA{“no hetaṃ, bhante”.}}\\
\begin{addmargin}[1em]{2em}
\setstretch{.5}
{\PaliGlossB{“No, sir.”}}\\
\end{addmargin}
\end{absolutelynopagebreak}

\begin{absolutelynopagebreak}
\setstretch{.7}
{\PaliGlossA{“no ce kira me tumhe, bhikkhave, evaṃ dhammaṃ desitaṃ ājānātha, atha kiñcarahi tumhe, moghapurisā, kiṃ jānantā kiṃ passantā evaṃ svākkhāte dhammavinaye pabbajitā samānā aññamaññaṃ sutena accāvadatha:}}\\
\begin{addmargin}[1em]{2em}
\setstretch{.5}
{\PaliGlossB{“If you’ve never known me to teach the Dhamma like this, then what exactly do you know and see, you foolish men, that after going forth in such a well explained teaching and training you’d compete in studies to}}\\
\end{addmargin}
\end{absolutelynopagebreak}

\begin{absolutelynopagebreak}
\setstretch{.7}
{\PaliGlossA{‘ehi, bhikkhu, ko bahutaraṃ bhāsissati, ko sundarataraṃ bhāsissati, ko cirataraṃ bhāsissatī’”ti.}}\\
\begin{addmargin}[1em]{2em}
\setstretch{.5}
{\PaliGlossB{see who can recite more and better and longer?”}}\\
\end{addmargin}
\end{absolutelynopagebreak}

\begin{absolutelynopagebreak}
\setstretch{.7}
{\PaliGlossA{atha kho te bhikkhū bhagavato pādesu sirasā nipatitvā bhagavantaṃ etadavocuṃ:}}\\
\begin{addmargin}[1em]{2em}
\setstretch{.5}
{\PaliGlossB{Then those monks bowed with their heads at the Buddha’s feet and said,}}\\
\end{addmargin}
\end{absolutelynopagebreak}

\begin{absolutelynopagebreak}
\setstretch{.7}
{\PaliGlossA{“accayo no, bhante, accagamā, yathābāle yathāmūḷhe yathāakusale, ye mayaṃ evaṃ svākkhāte dhammavinaye pabbajitā samānā aññamaññaṃ sutena accāvadimha:}}\\
\begin{addmargin}[1em]{2em}
\setstretch{.5}
{\PaliGlossB{“We have made a mistake, sir. It was foolish, stupid, and unskillful of us in that after going forth in such a well explained teaching and training we competed in studies to}}\\
\end{addmargin}
\end{absolutelynopagebreak}

\begin{absolutelynopagebreak}
\setstretch{.7}
{\PaliGlossA{‘ehi, bhikkhu, ko bahutaraṃ bhāsissati, ko sundarataraṃ bhāsissati, ko cirataraṃ bhāsissatī’ti.}}\\
\begin{addmargin}[1em]{2em}
\setstretch{.5}
{\PaliGlossB{see who can recite more and better and longer.}}\\
\end{addmargin}
\end{absolutelynopagebreak}

\begin{absolutelynopagebreak}
\setstretch{.7}
{\PaliGlossA{tesaṃ no, bhante, bhagavā accayaṃ accayato paṭiggaṇhātu āyatiṃ saṃvarāyā”ti.}}\\
\begin{addmargin}[1em]{2em}
\setstretch{.5}
{\PaliGlossB{Please, sir, accept our mistake for what it is, so we will restrain ourselves in future.”}}\\
\end{addmargin}
\end{absolutelynopagebreak}

\begin{absolutelynopagebreak}
\setstretch{.7}
{\PaliGlossA{“taggha tumhe, bhikkhave, accayo accagamā yathābāle yathāmūḷhe yathāakusale, ye tumhe evaṃ svākkhāte dhammavinaye pabbajitā samānā aññamaññaṃ sutena accāvadittha:}}\\
\begin{addmargin}[1em]{2em}
\setstretch{.5}
{\PaliGlossB{“Indeed, monks, you made a mistake. It was foolish, stupid, and unskillful of you to act in that way.}}\\
\end{addmargin}
\end{absolutelynopagebreak}

\begin{absolutelynopagebreak}
\setstretch{.7}
{\PaliGlossA{‘ehi, bhikkhu, ko bahutaraṃ bhāsissati, ko sundarataraṃ bhāsissati, ko cirataraṃ bhāsissatī’ti.}}\\
\begin{addmargin}[1em]{2em}
\setstretch{.5}
{\PaliGlossB{    -}}\\
\end{addmargin}
\end{absolutelynopagebreak}

\begin{absolutelynopagebreak}
\setstretch{.7}
{\PaliGlossA{yato ca kho tumhe, bhikkhave, accayaṃ accayato disvā yathādhammaṃ paṭikarotha, taṃ vo mayaṃ paṭiggaṇhāma.}}\\
\begin{addmargin}[1em]{2em}
\setstretch{.5}
{\PaliGlossB{But since you have recognized your mistake for what it is, and have dealt with it properly, I accept it.}}\\
\end{addmargin}
\end{absolutelynopagebreak}

\begin{absolutelynopagebreak}
\setstretch{.7}
{\PaliGlossA{vuddhi hesā, bhikkhave, ariyassa vinaye yo accayaṃ accayato disvā yathādhammaṃ paṭikaroti āyatiñca saṃvaraṃ āpajjatī”ti.}}\\
\begin{addmargin}[1em]{2em}
\setstretch{.5}
{\PaliGlossB{For it is growth in the training of the noble one to recognize a mistake for what it is, deal with it properly, and commit to restraint in the future.”}}\\
\end{addmargin}
\end{absolutelynopagebreak}

\begin{absolutelynopagebreak}
\setstretch{.7}
{\PaliGlossA{chaṭṭhaṃ.}}\\
\begin{addmargin}[1em]{2em}
\setstretch{.5}
{\PaliGlossB{    -}}\\
\end{addmargin}
\end{absolutelynopagebreak}
