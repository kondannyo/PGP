
\begin{absolutelynopagebreak}
\setstretch{.7}
{\PaliGlossA{saṃyutta nikāya 48}}\\
\begin{addmargin}[1em]{2em}
\setstretch{.5}
{\PaliGlossB{Linked Discourses 48}}\\
\end{addmargin}
\end{absolutelynopagebreak}

\begin{absolutelynopagebreak}
\setstretch{.7}
{\PaliGlossA{4. sukhindriyavagga}}\\
\begin{addmargin}[1em]{2em}
\setstretch{.5}
{\PaliGlossB{4. The Pleasure Faculty}}\\
\end{addmargin}
\end{absolutelynopagebreak}

\begin{absolutelynopagebreak}
\setstretch{.7}
{\PaliGlossA{40. uppaṭipāṭikasutta}}\\
\begin{addmargin}[1em]{2em}
\setstretch{.5}
{\PaliGlossB{40. Irregular Order}}\\
\end{addmargin}
\end{absolutelynopagebreak}

\begin{absolutelynopagebreak}
\setstretch{.7}
{\PaliGlossA{“pañcimāni, bhikkhave, indriyāni.}}\\
\begin{addmargin}[1em]{2em}
\setstretch{.5}
{\PaliGlossB{“Mendicants, there are these five faculties.}}\\
\end{addmargin}
\end{absolutelynopagebreak}

\begin{absolutelynopagebreak}
\setstretch{.7}
{\PaliGlossA{katamāni pañca?}}\\
\begin{addmargin}[1em]{2em}
\setstretch{.5}
{\PaliGlossB{What five?}}\\
\end{addmargin}
\end{absolutelynopagebreak}

\begin{absolutelynopagebreak}
\setstretch{.7}
{\PaliGlossA{dukkhindriyaṃ, domanassindriyaṃ, sukhindriyaṃ, somanassindriyaṃ, upekkhindriyaṃ.}}\\
\begin{addmargin}[1em]{2em}
\setstretch{.5}
{\PaliGlossB{The faculties of pain, sadness, pleasure, happiness, and equanimity.}}\\
\end{addmargin}
\end{absolutelynopagebreak}

\begin{absolutelynopagebreak}
\setstretch{.7}
{\PaliGlossA{idha, bhikkhave, bhikkhuno appamattassa ātāpino pahitattassa viharato uppajjati dukkhindriyaṃ.}}\\
\begin{addmargin}[1em]{2em}
\setstretch{.5}
{\PaliGlossB{While a mendicant is meditating—diligent, keen, and resolute—the faculty of pain arises.}}\\
\end{addmargin}
\end{absolutelynopagebreak}

\begin{absolutelynopagebreak}
\setstretch{.7}
{\PaliGlossA{so evaṃ pajānāti:}}\\
\begin{addmargin}[1em]{2em}
\setstretch{.5}
{\PaliGlossB{They understand:}}\\
\end{addmargin}
\end{absolutelynopagebreak}

\begin{absolutelynopagebreak}
\setstretch{.7}
{\PaliGlossA{‘uppannaṃ kho me idaṃ dukkhindriyaṃ, tañca kho sanimittaṃ sanidānaṃ sasaṅkhāraṃ sappaccayaṃ.}}\\
\begin{addmargin}[1em]{2em}
\setstretch{.5}
{\PaliGlossB{‘The faculty of pain has arisen in me. And that has a foundation, a source, a condition, and a reason.}}\\
\end{addmargin}
\end{absolutelynopagebreak}

\begin{absolutelynopagebreak}
\setstretch{.7}
{\PaliGlossA{tañca animittaṃ anidānaṃ asaṅkhāraṃ appaccayaṃ dukkhindriyaṃ uppajjissatī’ti—netaṃ ṭhānaṃ vijjati.}}\\
\begin{addmargin}[1em]{2em}
\setstretch{.5}
{\PaliGlossB{It’s not possible for the faculty of pain to arise without a foundation, a source, a condition, or a reason.’}}\\
\end{addmargin}
\end{absolutelynopagebreak}

\begin{absolutelynopagebreak}
\setstretch{.7}
{\PaliGlossA{so dukkhindriyañca pajānāti, dukkhindriyasamudayañca pajānāti, dukkhindriyanirodhañca pajānāti, yattha cuppannaṃ dukkhindriyaṃ aparisesaṃ nirujjhati tañca pajānāti.}}\\
\begin{addmargin}[1em]{2em}
\setstretch{.5}
{\PaliGlossB{They understand the faculty of pain, its origin, its cessation, and where that faculty of pain that’s arisen ceases without anything left over.}}\\
\end{addmargin}
\end{absolutelynopagebreak}

\begin{absolutelynopagebreak}
\setstretch{.7}
{\PaliGlossA{kattha cuppannaṃ dukkhindriyaṃ aparisesaṃ nirujjhati?}}\\
\begin{addmargin}[1em]{2em}
\setstretch{.5}
{\PaliGlossB{And where does that faculty of pain that’s arisen cease without anything left over?}}\\
\end{addmargin}
\end{absolutelynopagebreak}

\begin{absolutelynopagebreak}
\setstretch{.7}
{\PaliGlossA{idha, bhikkhave, bhikkhu vivicceva kāmehi vivicca akusalehi dhammehi savitakkaṃ savicāraṃ vivekajaṃ pītisukhaṃ paṭhamaṃ jhānaṃ upasampajja viharati,}}\\
\begin{addmargin}[1em]{2em}
\setstretch{.5}
{\PaliGlossB{It’s when a mendicant, quite secluded from sensual pleasures, secluded from unskillful qualities, enters and remains in the first absorption, which has the rapture and bliss born of seclusion, while placing the mind and keeping it connected.}}\\
\end{addmargin}
\end{absolutelynopagebreak}

\begin{absolutelynopagebreak}
\setstretch{.7}
{\PaliGlossA{ettha cuppannaṃ dukkhindriyaṃ aparisesaṃ nirujjhati.}}\\
\begin{addmargin}[1em]{2em}
\setstretch{.5}
{\PaliGlossB{That’s where the faculty of pain that’s arisen ceases without anything left over.}}\\
\end{addmargin}
\end{absolutelynopagebreak}

\begin{absolutelynopagebreak}
\setstretch{.7}
{\PaliGlossA{ayaṃ vuccati, bhikkhave, ‘bhikkhu aññāsi dukkhindriyassa nirodhaṃ, tadatthāya cittaṃ upasaṃharati’.}}\\
\begin{addmargin}[1em]{2em}
\setstretch{.5}
{\PaliGlossB{They’re called a mendicant who understands the cessation of the faculty of pain, and who applies their mind to that end.}}\\
\end{addmargin}
\end{absolutelynopagebreak}

\begin{absolutelynopagebreak}
\setstretch{.7}
{\PaliGlossA{idha pana, bhikkhave, bhikkhuno appamattassa ātāpino pahitattassa viharato uppajjati domanassindriyaṃ.}}\\
\begin{addmargin}[1em]{2em}
\setstretch{.5}
{\PaliGlossB{While a mendicant is meditating—diligent, keen, and resolute—the faculty of sadness arises.}}\\
\end{addmargin}
\end{absolutelynopagebreak}

\begin{absolutelynopagebreak}
\setstretch{.7}
{\PaliGlossA{so evaṃ pajānāti:}}\\
\begin{addmargin}[1em]{2em}
\setstretch{.5}
{\PaliGlossB{They understand:}}\\
\end{addmargin}
\end{absolutelynopagebreak}

\begin{absolutelynopagebreak}
\setstretch{.7}
{\PaliGlossA{‘uppannaṃ kho me idaṃ domanassindriyaṃ, tañca kho sanimittaṃ sanidānaṃ sasaṅkhāraṃ sappaccayaṃ.}}\\
\begin{addmargin}[1em]{2em}
\setstretch{.5}
{\PaliGlossB{‘The faculty of sadness has arisen in me. And that has a foundation, a source, a condition, and a reason.}}\\
\end{addmargin}
\end{absolutelynopagebreak}

\begin{absolutelynopagebreak}
\setstretch{.7}
{\PaliGlossA{tañca animittaṃ anidānaṃ asaṅkhāraṃ appaccayaṃ domanassindriyaṃ uppajjissatī’ti—netaṃ ṭhānaṃ vijjati.}}\\
\begin{addmargin}[1em]{2em}
\setstretch{.5}
{\PaliGlossB{It’s not possible for the faculty of sadness to arise without a foundation, a source, a condition, or a reason.’}}\\
\end{addmargin}
\end{absolutelynopagebreak}

\begin{absolutelynopagebreak}
\setstretch{.7}
{\PaliGlossA{so domanassindriyañca pajānāti, domanassindriyasamudayañca pajānāti, domanassindriyanirodhañca pajānāti, yattha cuppannaṃ domanassindriyaṃ aparisesaṃ nirujjhati tañca pajānāti.}}\\
\begin{addmargin}[1em]{2em}
\setstretch{.5}
{\PaliGlossB{They understand the faculty of sadness, its origin, its cessation, and where that faculty of sadness that’s arisen ceases without anything left over.}}\\
\end{addmargin}
\end{absolutelynopagebreak}

\begin{absolutelynopagebreak}
\setstretch{.7}
{\PaliGlossA{kattha cuppannaṃ domanassindriyaṃ aparisesaṃ nirujjhati?}}\\
\begin{addmargin}[1em]{2em}
\setstretch{.5}
{\PaliGlossB{And where does that faculty of sadness that’s arisen cease without anything left over?}}\\
\end{addmargin}
\end{absolutelynopagebreak}

\begin{absolutelynopagebreak}
\setstretch{.7}
{\PaliGlossA{idha, bhikkhave, bhikkhu vitakkavicārānaṃ vūpasamā ajjhattaṃ sampasādanaṃ cetaso ekodibhāvaṃ avitakkaṃ avicāraṃ samādhijaṃ pītisukhaṃ dutiyaṃ jhānaṃ upasampajja viharati,}}\\
\begin{addmargin}[1em]{2em}
\setstretch{.5}
{\PaliGlossB{It’s when, as the placing of the mind and keeping it connected are stilled, a mendicant enters and remains in the second absorption, which has the rapture and bliss born of immersion, with internal clarity and confidence, and unified mind, without placing the mind and keeping it connected.}}\\
\end{addmargin}
\end{absolutelynopagebreak}

\begin{absolutelynopagebreak}
\setstretch{.7}
{\PaliGlossA{ettha cuppannaṃ domanassindriyaṃ aparisesaṃ nirujjhati.}}\\
\begin{addmargin}[1em]{2em}
\setstretch{.5}
{\PaliGlossB{That’s where the faculty of sadness that’s arisen ceases without anything left over.}}\\
\end{addmargin}
\end{absolutelynopagebreak}

\begin{absolutelynopagebreak}
\setstretch{.7}
{\PaliGlossA{ayaṃ vuccati, bhikkhave, ‘bhikkhu aññāsi domanassindriyassa nirodhaṃ, tadatthāya cittaṃ upasaṃharati’.}}\\
\begin{addmargin}[1em]{2em}
\setstretch{.5}
{\PaliGlossB{They’re called a mendicant who understands the cessation of the faculty of sadness, and who applies their mind to that end.}}\\
\end{addmargin}
\end{absolutelynopagebreak}

\begin{absolutelynopagebreak}
\setstretch{.7}
{\PaliGlossA{idha pana, bhikkhave, bhikkhuno appamattassa ātāpino pahitattassa viharato uppajjati sukhindriyaṃ.}}\\
\begin{addmargin}[1em]{2em}
\setstretch{.5}
{\PaliGlossB{While a mendicant is meditating—diligent, keen, and resolute—the faculty of pleasure arises.}}\\
\end{addmargin}
\end{absolutelynopagebreak}

\begin{absolutelynopagebreak}
\setstretch{.7}
{\PaliGlossA{so evaṃ pajānāti:}}\\
\begin{addmargin}[1em]{2em}
\setstretch{.5}
{\PaliGlossB{They understand:}}\\
\end{addmargin}
\end{absolutelynopagebreak}

\begin{absolutelynopagebreak}
\setstretch{.7}
{\PaliGlossA{‘uppannaṃ kho me idaṃ sukhindriyaṃ, tañca kho sanimittaṃ sanidānaṃ sasaṅkhāraṃ sappaccayaṃ.}}\\
\begin{addmargin}[1em]{2em}
\setstretch{.5}
{\PaliGlossB{‘The faculty of pleasure has arisen in me. And that has a foundation, a source, a condition, and a reason.}}\\
\end{addmargin}
\end{absolutelynopagebreak}

\begin{absolutelynopagebreak}
\setstretch{.7}
{\PaliGlossA{tañca animittaṃ anidānaṃ asaṅkhāraṃ appaccayaṃ sukhindriyaṃ uppajjissatī’ti—netaṃ ṭhānaṃ vijjati.}}\\
\begin{addmargin}[1em]{2em}
\setstretch{.5}
{\PaliGlossB{it’s not possible for the faculty of pleasure to arise without a foundation, a source, a condition, or a reason.’}}\\
\end{addmargin}
\end{absolutelynopagebreak}

\begin{absolutelynopagebreak}
\setstretch{.7}
{\PaliGlossA{so sukhindriyañca pajānāti, sukhindriyasamudayañca pajānāti, sukhindriyanirodhañca pajānāti, yattha cuppannaṃ sukhindriyaṃ aparisesaṃ nirujjhati tañca pajānāti.}}\\
\begin{addmargin}[1em]{2em}
\setstretch{.5}
{\PaliGlossB{They understand the faculty of pleasure, its origin, its cessation, and where that faculty of pleasure that’s arisen ceases without anything left over.}}\\
\end{addmargin}
\end{absolutelynopagebreak}

\begin{absolutelynopagebreak}
\setstretch{.7}
{\PaliGlossA{kattha cuppannaṃ sukhindriyaṃ aparisesaṃ nirujjhati?}}\\
\begin{addmargin}[1em]{2em}
\setstretch{.5}
{\PaliGlossB{And where does that faculty of pleasure that’s arisen cease without anything left over?}}\\
\end{addmargin}
\end{absolutelynopagebreak}

\begin{absolutelynopagebreak}
\setstretch{.7}
{\PaliGlossA{idha, bhikkhave, bhikkhu pītiyā ca virāgā upekkhako ca viharati sato ca sampajāno sukhañca kāyena paṭisaṃvedeti yaṃ taṃ ariyā ācikkhanti ‘upekkhako satimā sukhavihārī’ti tatiyaṃ jhānaṃ upasampajja viharati,}}\\
\begin{addmargin}[1em]{2em}
\setstretch{.5}
{\PaliGlossB{It’s when, with the fading away of rapture, a mendicant enters and remains in the third absorption, where they meditate with equanimity, mindful and aware, personally experiencing the bliss of which the noble ones declare, ‘Equanimous and mindful, one meditates in bliss.’}}\\
\end{addmargin}
\end{absolutelynopagebreak}

\begin{absolutelynopagebreak}
\setstretch{.7}
{\PaliGlossA{ettha cuppannaṃ sukhindriyaṃ aparisesaṃ nirujjhati.}}\\
\begin{addmargin}[1em]{2em}
\setstretch{.5}
{\PaliGlossB{That’s where the faculty of pleasure that’s arisen ceases without anything left over.}}\\
\end{addmargin}
\end{absolutelynopagebreak}

\begin{absolutelynopagebreak}
\setstretch{.7}
{\PaliGlossA{ayaṃ vuccati, bhikkhave, ‘bhikkhu aññāsi sukhindriyassa nirodhaṃ, tadatthāya cittaṃ upasaṃharati’.}}\\
\begin{addmargin}[1em]{2em}
\setstretch{.5}
{\PaliGlossB{They’re called a mendicant who understands the cessation of the faculty of pleasure, and who applies their mind to that end.}}\\
\end{addmargin}
\end{absolutelynopagebreak}

\begin{absolutelynopagebreak}
\setstretch{.7}
{\PaliGlossA{idha pana, bhikkhave, bhikkhuno appamattassa ātāpino pahitattassa viharato uppajjati somanassindriyaṃ.}}\\
\begin{addmargin}[1em]{2em}
\setstretch{.5}
{\PaliGlossB{While a mendicant is meditating—diligent, keen, and resolute—the faculty of happiness arises.}}\\
\end{addmargin}
\end{absolutelynopagebreak}

\begin{absolutelynopagebreak}
\setstretch{.7}
{\PaliGlossA{so evaṃ pajānāti:}}\\
\begin{addmargin}[1em]{2em}
\setstretch{.5}
{\PaliGlossB{They understand:}}\\
\end{addmargin}
\end{absolutelynopagebreak}

\begin{absolutelynopagebreak}
\setstretch{.7}
{\PaliGlossA{‘uppannaṃ kho me idaṃ somanassindriyaṃ, tañca kho sanimittaṃ sanidānaṃ sasaṅkhāraṃ sappaccayaṃ.}}\\
\begin{addmargin}[1em]{2em}
\setstretch{.5}
{\PaliGlossB{‘The faculty of happiness has arisen in me. And that has a foundation, a source, a condition, and a reason.}}\\
\end{addmargin}
\end{absolutelynopagebreak}

\begin{absolutelynopagebreak}
\setstretch{.7}
{\PaliGlossA{tañca animittaṃ anidānaṃ asaṅkhāraṃ appaccayaṃ somanassindriyaṃ uppajjissatī’ti—netaṃ ṭhānaṃ vijjati.}}\\
\begin{addmargin}[1em]{2em}
\setstretch{.5}
{\PaliGlossB{it’s not possible for the faculty of happiness to arise without a foundation, a source, a condition, or a reason.’}}\\
\end{addmargin}
\end{absolutelynopagebreak}

\begin{absolutelynopagebreak}
\setstretch{.7}
{\PaliGlossA{so somanassindriyañca pajānāti, somanassindriyasamudayañca pajānāti, somanassindriyanirodhañca pajānāti, yattha cuppannaṃ somanassindriyaṃ aparisesaṃ nirujjhati tañca pajānāti.}}\\
\begin{addmargin}[1em]{2em}
\setstretch{.5}
{\PaliGlossB{They understand the faculty of happiness, its origin, its cessation, and where that faculty of happiness that’s arisen ceases without anything left over.}}\\
\end{addmargin}
\end{absolutelynopagebreak}

\begin{absolutelynopagebreak}
\setstretch{.7}
{\PaliGlossA{kattha cuppannaṃ somanassindriyaṃ aparisesaṃ nirujjhati?}}\\
\begin{addmargin}[1em]{2em}
\setstretch{.5}
{\PaliGlossB{And where does that faculty of happiness that’s arisen cease without anything left over?}}\\
\end{addmargin}
\end{absolutelynopagebreak}

\begin{absolutelynopagebreak}
\setstretch{.7}
{\PaliGlossA{idha, bhikkhave, bhikkhu sukhassa ca pahānā dukkhassa ca pahānā pubbeva somanassadomanassānaṃ atthaṅgamā adukkhamasukhaṃ upekkhāsatipārisuddhiṃ catutthaṃ jhānaṃ upasampajja viharati,}}\\
\begin{addmargin}[1em]{2em}
\setstretch{.5}
{\PaliGlossB{It’s when, giving up pleasure and pain, and ending former happiness and sadness, a mendicant enters and remains in the fourth absorption, without pleasure or pain, with pure equanimity and mindfulness.}}\\
\end{addmargin}
\end{absolutelynopagebreak}

\begin{absolutelynopagebreak}
\setstretch{.7}
{\PaliGlossA{ettha cuppannaṃ somanassindriyaṃ aparisesaṃ nirujjhati.}}\\
\begin{addmargin}[1em]{2em}
\setstretch{.5}
{\PaliGlossB{That’s where the faculty of happiness that’s arisen ceases without anything left over.}}\\
\end{addmargin}
\end{absolutelynopagebreak}

\begin{absolutelynopagebreak}
\setstretch{.7}
{\PaliGlossA{ayaṃ vuccati, bhikkhave, ‘bhikkhu aññāsi somanassindriyassa nirodhaṃ, tadatthāya cittaṃ upasaṃharati’.}}\\
\begin{addmargin}[1em]{2em}
\setstretch{.5}
{\PaliGlossB{They’re called a mendicant who understands the cessation of the faculty of happiness, and who applies their mind to that end.}}\\
\end{addmargin}
\end{absolutelynopagebreak}

\begin{absolutelynopagebreak}
\setstretch{.7}
{\PaliGlossA{idha pana, bhikkhave, bhikkhuno appamattassa ātāpino pahitattassa viharato uppajjati upekkhindriyaṃ.}}\\
\begin{addmargin}[1em]{2em}
\setstretch{.5}
{\PaliGlossB{While a mendicant is meditating—diligent, keen, and resolute—the faculty of equanimity arises.}}\\
\end{addmargin}
\end{absolutelynopagebreak}

\begin{absolutelynopagebreak}
\setstretch{.7}
{\PaliGlossA{so evaṃ pajānāti:}}\\
\begin{addmargin}[1em]{2em}
\setstretch{.5}
{\PaliGlossB{They understand:}}\\
\end{addmargin}
\end{absolutelynopagebreak}

\begin{absolutelynopagebreak}
\setstretch{.7}
{\PaliGlossA{‘uppannaṃ kho me idaṃ upekkhindriyaṃ, tañca kho sanimittaṃ sanidānaṃ sasaṅkhāraṃ sappaccayaṃ.}}\\
\begin{addmargin}[1em]{2em}
\setstretch{.5}
{\PaliGlossB{‘The faculty of equanimity has arisen in me. And that has a foundation, a source, a condition, and a reason.}}\\
\end{addmargin}
\end{absolutelynopagebreak}

\begin{absolutelynopagebreak}
\setstretch{.7}
{\PaliGlossA{tañca animittaṃ anidānaṃ asaṅkhāraṃ appaccayaṃ upekkhindriyaṃ uppajjissatī’ti—netaṃ ṭhānaṃ vijjati.}}\\
\begin{addmargin}[1em]{2em}
\setstretch{.5}
{\PaliGlossB{It’s not possible for the faculty of equanimity to arise without a foundation, a source, a condition, or a reason.’}}\\
\end{addmargin}
\end{absolutelynopagebreak}

\begin{absolutelynopagebreak}
\setstretch{.7}
{\PaliGlossA{so upekkhindriyañca pajānāti, upekkhindriyasamudayañca pajānāti, upekkhindriyanirodhañca pajānāti, yattha cuppannaṃ upekkhindriyaṃ aparisesaṃ nirujjhati tañca pajānāti.}}\\
\begin{addmargin}[1em]{2em}
\setstretch{.5}
{\PaliGlossB{They understand the faculty of equanimity, its origin, its cessation, and where that faculty of equanimity that’s arisen ceases without anything left over.}}\\
\end{addmargin}
\end{absolutelynopagebreak}

\begin{absolutelynopagebreak}
\setstretch{.7}
{\PaliGlossA{kattha cuppannaṃ upekkhindriyaṃ aparisesaṃ nirujjhati?}}\\
\begin{addmargin}[1em]{2em}
\setstretch{.5}
{\PaliGlossB{And where does that faculty of equanimity that’s arisen cease without anything left over?}}\\
\end{addmargin}
\end{absolutelynopagebreak}

\begin{absolutelynopagebreak}
\setstretch{.7}
{\PaliGlossA{idha, bhikkhave, bhikkhu sabbaso nevasaññānāsaññāyatanaṃ samatikkamma saññāvedayitanirodhaṃ upasampajja viharati,}}\\
\begin{addmargin}[1em]{2em}
\setstretch{.5}
{\PaliGlossB{It’s when a mendicant, going totally beyond the dimension of neither perception nor non-perception, enters and remains in the cessation of perception and feeling.}}\\
\end{addmargin}
\end{absolutelynopagebreak}

\begin{absolutelynopagebreak}
\setstretch{.7}
{\PaliGlossA{ettha cuppannaṃ upekkhindriyaṃ aparisesaṃ nirujjhati.}}\\
\begin{addmargin}[1em]{2em}
\setstretch{.5}
{\PaliGlossB{That’s where the faculty of equanimity that’s arisen ceases without anything left over.}}\\
\end{addmargin}
\end{absolutelynopagebreak}

\begin{absolutelynopagebreak}
\setstretch{.7}
{\PaliGlossA{ayaṃ vuccati, bhikkhave, ‘bhikkhu aññāsi upekkhindriyassa nirodhaṃ, tadatthāya cittaṃ upasaṃharatī’”ti.}}\\
\begin{addmargin}[1em]{2em}
\setstretch{.5}
{\PaliGlossB{They’re called a mendicant who understands the cessation of the faculty of equanimity, and who applies their mind to that end.”}}\\
\end{addmargin}
\end{absolutelynopagebreak}

\begin{absolutelynopagebreak}
\setstretch{.7}
{\PaliGlossA{dasamaṃ.}}\\
\begin{addmargin}[1em]{2em}
\setstretch{.5}
{\PaliGlossB{    -}}\\
\end{addmargin}
\end{absolutelynopagebreak}

\begin{absolutelynopagebreak}
\setstretch{.7}
{\PaliGlossA{sukhindriyavaggo catuttho.}}\\
\begin{addmargin}[1em]{2em}
\setstretch{.5}
{\PaliGlossB{    -}}\\
\end{addmargin}
\end{absolutelynopagebreak}

\begin{absolutelynopagebreak}
\setstretch{.7}
{\PaliGlossA{suddhikañca soto arahā,}}\\
\begin{addmargin}[1em]{2em}
\setstretch{.5}
{\PaliGlossB{    -}}\\
\end{addmargin}
\end{absolutelynopagebreak}

\begin{absolutelynopagebreak}
\setstretch{.7}
{\PaliGlossA{duve samaṇabrāhmaṇā;}}\\
\begin{addmargin}[1em]{2em}
\setstretch{.5}
{\PaliGlossB{    -}}\\
\end{addmargin}
\end{absolutelynopagebreak}

\begin{absolutelynopagebreak}
\setstretch{.7}
{\PaliGlossA{vibhaṅgena tayo vuttā,}}\\
\begin{addmargin}[1em]{2em}
\setstretch{.5}
{\PaliGlossB{    -}}\\
\end{addmargin}
\end{absolutelynopagebreak}

\begin{absolutelynopagebreak}
\setstretch{.7}
{\PaliGlossA{kaṭṭho uppaṭipāṭikanti.}}\\
\begin{addmargin}[1em]{2em}
\setstretch{.5}
{\PaliGlossB{    -}}\\
\end{addmargin}
\end{absolutelynopagebreak}
