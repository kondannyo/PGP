
\begin{absolutelynopagebreak}
\setstretch{.7}
{\PaliGlossA{saṃyutta nikāya 22}}\\
\begin{addmargin}[1em]{2em}
\setstretch{.5}
{\PaliGlossB{Linked Discourses 22}}\\
\end{addmargin}
\end{absolutelynopagebreak}

\begin{absolutelynopagebreak}
\setstretch{.7}
{\PaliGlossA{5. attadīpavagga}}\\
\begin{addmargin}[1em]{2em}
\setstretch{.5}
{\PaliGlossB{5. Be Your Own Island}}\\
\end{addmargin}
\end{absolutelynopagebreak}

\begin{absolutelynopagebreak}
\setstretch{.7}
{\PaliGlossA{50. dutiyasoṇasutta}}\\
\begin{addmargin}[1em]{2em}
\setstretch{.5}
{\PaliGlossB{50. With Soṇa (2nd)}}\\
\end{addmargin}
\end{absolutelynopagebreak}

\begin{absolutelynopagebreak}
\setstretch{.7}
{\PaliGlossA{evaṃ me sutaṃ—}}\\
\begin{addmargin}[1em]{2em}
\setstretch{.5}
{\PaliGlossB{So I have heard.}}\\
\end{addmargin}
\end{absolutelynopagebreak}

\begin{absolutelynopagebreak}
\setstretch{.7}
{\PaliGlossA{ekaṃ samayaṃ bhagavā rājagahe viharati veḷuvane kalandakanivāpe.}}\\
\begin{addmargin}[1em]{2em}
\setstretch{.5}
{\PaliGlossB{At one time the Buddha was staying near Rājagaha, in the Bamboo Grove, the squirrels’ feeding ground.}}\\
\end{addmargin}
\end{absolutelynopagebreak}

\begin{absolutelynopagebreak}
\setstretch{.7}
{\PaliGlossA{atha kho soṇo gahapatiputto yena bhagavā tenupasaṅkami; upasaṅkamitvā bhagavantaṃ abhivādetvā ekamantaṃ nisīdi. ekamantaṃ nisinnaṃ kho soṇaṃ gahapatiputtaṃ bhagavā etadavoca:}}\\
\begin{addmargin}[1em]{2em}
\setstretch{.5}
{\PaliGlossB{Then the householder Soṇa went up to the Buddha, bowed, and sat down to one side. The Buddha said to him:}}\\
\end{addmargin}
\end{absolutelynopagebreak}

\begin{absolutelynopagebreak}
\setstretch{.7}
{\PaliGlossA{“ye hi keci, soṇa, samaṇā vā brāhmaṇā vā rūpaṃ nappajānanti, rūpasamudayaṃ nappajānanti, rūpanirodhaṃ nappajānanti, rūpanirodhagāminiṃ paṭipadaṃ nappajānanti;}}\\
\begin{addmargin}[1em]{2em}
\setstretch{.5}
{\PaliGlossB{“Soṇa, there are ascetics and brahmins who don’t understand form, its origin, its cessation, and the practice that leads to its cessation.}}\\
\end{addmargin}
\end{absolutelynopagebreak}

\begin{absolutelynopagebreak}
\setstretch{.7}
{\PaliGlossA{vedanaṃ nappajānanti, vedanāsamudayaṃ nappajānanti, vedanānirodhaṃ nappajānanti, vedanānirodhagāminiṃ paṭipadaṃ nappajānanti;}}\\
\begin{addmargin}[1em]{2em}
\setstretch{.5}
{\PaliGlossB{They don’t understand feeling …}}\\
\end{addmargin}
\end{absolutelynopagebreak}

\begin{absolutelynopagebreak}
\setstretch{.7}
{\PaliGlossA{saññaṃ nappajānanti … pe …}}\\
\begin{addmargin}[1em]{2em}
\setstretch{.5}
{\PaliGlossB{perception …}}\\
\end{addmargin}
\end{absolutelynopagebreak}

\begin{absolutelynopagebreak}
\setstretch{.7}
{\PaliGlossA{saṅkhāre nappajānanti, saṅkhārasamudayaṃ nappajānanti, saṅkhāranirodhaṃ nappajānanti, saṅkhāranirodhagāminiṃ paṭipadaṃ nappajānanti;}}\\
\begin{addmargin}[1em]{2em}
\setstretch{.5}
{\PaliGlossB{choices …}}\\
\end{addmargin}
\end{absolutelynopagebreak}

\begin{absolutelynopagebreak}
\setstretch{.7}
{\PaliGlossA{viññāṇaṃ nappajānanti, viññāṇasamudayaṃ nappajānanti, viññāṇanirodhaṃ nappajānanti, viññāṇanirodhagāminiṃ paṭipadaṃ nappajānanti.}}\\
\begin{addmargin}[1em]{2em}
\setstretch{.5}
{\PaliGlossB{consciousness, its origin, its cessation, and the practice that leads to its cessation.}}\\
\end{addmargin}
\end{absolutelynopagebreak}

\begin{absolutelynopagebreak}
\setstretch{.7}
{\PaliGlossA{na me te, soṇa, samaṇā vā brāhmaṇā vā samaṇesu vā samaṇasammatā brāhmaṇesu vā brāhmaṇasammatā, na ca pana te āyasmanto sāmaññatthaṃ vā brahmaññatthaṃ vā diṭṭheva dhamme sayaṃ abhiññā sacchikatvā upasampajja viharanti.}}\\
\begin{addmargin}[1em]{2em}
\setstretch{.5}
{\PaliGlossB{I don’t regard them as true ascetics and brahmins. Those venerables don’t realize the goal of life as an ascetic or brahmin, and don’t live having realized it with their own insight.}}\\
\end{addmargin}
\end{absolutelynopagebreak}

\begin{absolutelynopagebreak}
\setstretch{.7}
{\PaliGlossA{ye ca kho keci, soṇa, samaṇā vā brāhmaṇā vā rūpaṃ pajānanti, rūpasamudayaṃ pajānanti, rūpanirodhaṃ pajānanti, rūpanirodhagāminiṃ paṭipadaṃ pajānanti;}}\\
\begin{addmargin}[1em]{2em}
\setstretch{.5}
{\PaliGlossB{There are ascetics and brahmins who do understand form, its origin, its cessation, and the practice that leads to its cessation.}}\\
\end{addmargin}
\end{absolutelynopagebreak}

\begin{absolutelynopagebreak}
\setstretch{.7}
{\PaliGlossA{vedanaṃ pajānanti … pe …}}\\
\begin{addmargin}[1em]{2em}
\setstretch{.5}
{\PaliGlossB{They do understand feeling …}}\\
\end{addmargin}
\end{absolutelynopagebreak}

\begin{absolutelynopagebreak}
\setstretch{.7}
{\PaliGlossA{saññaṃ pajānanti …}}\\
\begin{addmargin}[1em]{2em}
\setstretch{.5}
{\PaliGlossB{perception …}}\\
\end{addmargin}
\end{absolutelynopagebreak}

\begin{absolutelynopagebreak}
\setstretch{.7}
{\PaliGlossA{saṅkhāre pajānanti …}}\\
\begin{addmargin}[1em]{2em}
\setstretch{.5}
{\PaliGlossB{choices …}}\\
\end{addmargin}
\end{absolutelynopagebreak}

\begin{absolutelynopagebreak}
\setstretch{.7}
{\PaliGlossA{viññāṇaṃ pajānanti, viññāṇasamudayaṃ pajānanti, viññāṇanirodhaṃ pajānanti, viññāṇanirodhagāminiṃ paṭipadaṃ pajānanti.}}\\
\begin{addmargin}[1em]{2em}
\setstretch{.5}
{\PaliGlossB{consciousness, its origin, its cessation, and the practice that leads to its cessation.}}\\
\end{addmargin}
\end{absolutelynopagebreak}

\begin{absolutelynopagebreak}
\setstretch{.7}
{\PaliGlossA{te ca kho me, soṇa, samaṇā vā brāhmaṇā vā samaṇesu ceva samaṇasammatā brāhmaṇesu ca brāhmaṇasammatā, te ca panāyasmanto sāmaññatthañca brahmaññatthañca diṭṭheva dhamme sayaṃ abhiññā sacchikatvā upasampajja viharantī”ti.}}\\
\begin{addmargin}[1em]{2em}
\setstretch{.5}
{\PaliGlossB{I regard them as true ascetics and brahmins. Those venerables realize the goal of life as an ascetic or brahmin, and live having realized it with their own insight.”}}\\
\end{addmargin}
\end{absolutelynopagebreak}

\begin{absolutelynopagebreak}
\setstretch{.7}
{\PaliGlossA{aṭṭhamaṃ.}}\\
\begin{addmargin}[1em]{2em}
\setstretch{.5}
{\PaliGlossB{    -}}\\
\end{addmargin}
\end{absolutelynopagebreak}
