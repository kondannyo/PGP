
\begin{absolutelynopagebreak}
\setstretch{.7}
{\PaliGlossA{saṃyutta nikāya 35}}\\
\begin{addmargin}[1em]{2em}
\setstretch{.5}
{\PaliGlossB{Linked Discourses 35}}\\
\end{addmargin}
\end{absolutelynopagebreak}

\begin{absolutelynopagebreak}
\setstretch{.7}
{\PaliGlossA{12. lokakāmaguṇavagga}}\\
\begin{addmargin}[1em]{2em}
\setstretch{.5}
{\PaliGlossB{12. The World and the Kinds of Sensual Stimulation}}\\
\end{addmargin}
\end{absolutelynopagebreak}

\begin{absolutelynopagebreak}
\setstretch{.7}
{\PaliGlossA{116. lokantagamanasutta}}\\
\begin{addmargin}[1em]{2em}
\setstretch{.5}
{\PaliGlossB{116. Traveling to the End of the World}}\\
\end{addmargin}
\end{absolutelynopagebreak}

\begin{absolutelynopagebreak}
\setstretch{.7}
{\PaliGlossA{“nāhaṃ, bhikkhave, gamanena lokassa antaṃ ñāteyyaṃ, daṭṭheyyaṃ, patteyyanti vadāmi.}}\\
\begin{addmargin}[1em]{2em}
\setstretch{.5}
{\PaliGlossB{“Mendicants, I say it’s not possible to know or see or reach the end of the world by traveling.}}\\
\end{addmargin}
\end{absolutelynopagebreak}

\begin{absolutelynopagebreak}
\setstretch{.7}
{\PaliGlossA{na ca panāhaṃ, bhikkhave, appatvā lokassa antaṃ dukkhassa antakiriyaṃ vadāmī”ti.}}\\
\begin{addmargin}[1em]{2em}
\setstretch{.5}
{\PaliGlossB{But I also say there’s no making an end of suffering without reaching the end of the world.”}}\\
\end{addmargin}
\end{absolutelynopagebreak}

\begin{absolutelynopagebreak}
\setstretch{.7}
{\PaliGlossA{idaṃ vatvā bhagavā uṭṭhāyāsanā vihāraṃ pāvisi.}}\\
\begin{addmargin}[1em]{2em}
\setstretch{.5}
{\PaliGlossB{When he had spoken, the Holy One got up from his seat and entered his dwelling.}}\\
\end{addmargin}
\end{absolutelynopagebreak}

\begin{absolutelynopagebreak}
\setstretch{.7}
{\PaliGlossA{atha kho tesaṃ bhikkhūnaṃ acirapakkantassa bhagavato etadahosi:}}\\
\begin{addmargin}[1em]{2em}
\setstretch{.5}
{\PaliGlossB{Soon after the Buddha left, those mendicants considered,}}\\
\end{addmargin}
\end{absolutelynopagebreak}

\begin{absolutelynopagebreak}
\setstretch{.7}
{\PaliGlossA{“idaṃ kho no, āvuso, bhagavā saṃkhittena uddesaṃ uddisitvā vitthārena atthaṃ avibhajitvā uṭṭhāyāsanā vihāraṃ paviṭṭho:}}\\
\begin{addmargin}[1em]{2em}
\setstretch{.5}
{\PaliGlossB{“The Buddha gave this brief passage for recitation, then entered his dwelling without explaining the meaning in detail. …}}\\
\end{addmargin}
\end{absolutelynopagebreak}

\begin{absolutelynopagebreak}
\setstretch{.7}
{\PaliGlossA{‘“nāhaṃ, bhikkhave, gamanena lokassa antaṃ ñāteyyaṃ, daṭṭheyyaṃ, patteyyan”ti vadāmi.}}\\
\begin{addmargin}[1em]{2em}
\setstretch{.5}
{\PaliGlossB{    -}}\\
\end{addmargin}
\end{absolutelynopagebreak}

\begin{absolutelynopagebreak}
\setstretch{.7}
{\PaliGlossA{na ca panāhaṃ, bhikkhave, appatvā lokassa antaṃ dukkhassa antakiriyaṃ vadāmī’ti.}}\\
\begin{addmargin}[1em]{2em}
\setstretch{.5}
{\PaliGlossB{    -}}\\
\end{addmargin}
\end{absolutelynopagebreak}

\begin{absolutelynopagebreak}
\setstretch{.7}
{\PaliGlossA{ko nu kho imassa bhagavatā saṃkhittena uddesassa uddiṭṭhassa vitthārena atthaṃ avibhattassa vitthārena atthaṃ vibhajeyyā”ti?}}\\
\begin{addmargin}[1em]{2em}
\setstretch{.5}
{\PaliGlossB{Who can explain in detail the meaning of this brief passage for recitation given by the Buddha?”}}\\
\end{addmargin}
\end{absolutelynopagebreak}

\begin{absolutelynopagebreak}
\setstretch{.7}
{\PaliGlossA{atha kho tesaṃ bhikkhūnaṃ etadahosi:}}\\
\begin{addmargin}[1em]{2em}
\setstretch{.5}
{\PaliGlossB{Then those mendicants thought,}}\\
\end{addmargin}
\end{absolutelynopagebreak}

\begin{absolutelynopagebreak}
\setstretch{.7}
{\PaliGlossA{“ayaṃ kho āyasmā ānando satthu ceva saṃvaṇṇito, sambhāvito ca viññūnaṃ sabrahmacārīnaṃ.}}\\
\begin{addmargin}[1em]{2em}
\setstretch{.5}
{\PaliGlossB{“This Venerable Ānanda is praised by the Buddha and esteemed by his sensible spiritual companions.}}\\
\end{addmargin}
\end{absolutelynopagebreak}

\begin{absolutelynopagebreak}
\setstretch{.7}
{\PaliGlossA{pahoti cāyasmā ānando imassa bhagavatā saṅkhittena uddesassa uddiṭṭhassa vitthārena atthaṃ avibhattassa vitthārena atthaṃ vibhajituṃ.}}\\
\begin{addmargin}[1em]{2em}
\setstretch{.5}
{\PaliGlossB{He is capable of explaining in detail the meaning of this brief passage for recitation given by the Buddha.}}\\
\end{addmargin}
\end{absolutelynopagebreak}

\begin{absolutelynopagebreak}
\setstretch{.7}
{\PaliGlossA{yannūna mayaṃ yenāyasmā ānando tenupasaṅkameyyāma; upasaṅkamitvā āyasmantaṃ ānandaṃ etamatthaṃ paṭipuccheyyāmā”ti.}}\\
\begin{addmargin}[1em]{2em}
\setstretch{.5}
{\PaliGlossB{Let’s go to him, and ask him about this matter.”}}\\
\end{addmargin}
\end{absolutelynopagebreak}

\begin{absolutelynopagebreak}
\setstretch{.7}
{\PaliGlossA{atha kho te bhikkhū yenāyasmā ānando tenupasaṅkamiṃsu; upasaṅkamitvā āyasmatā ānandena saddhiṃ sammodiṃsu.}}\\
\begin{addmargin}[1em]{2em}
\setstretch{.5}
{\PaliGlossB{Then those mendicants went to Ānanda and exchanged greetings with him.}}\\
\end{addmargin}
\end{absolutelynopagebreak}

\begin{absolutelynopagebreak}
\setstretch{.7}
{\PaliGlossA{sammodanīyaṃ kathaṃ sāraṇīyaṃ vītisāretvā ekamantaṃ nisīdiṃsu. ekamantaṃ nisinnā kho te bhikkhū āyasmantaṃ ānandaṃ etadavocuṃ:}}\\
\begin{addmargin}[1em]{2em}
\setstretch{.5}
{\PaliGlossB{When the greetings and polite conversation were over, they sat down to one side. They told him what had happened, and said,}}\\
\end{addmargin}
\end{absolutelynopagebreak}

\begin{absolutelynopagebreak}
\setstretch{.7}
{\PaliGlossA{“idaṃ kho no, āvuso ānanda, bhagavā saṅkhittena uddesaṃ uddisitvā vitthārena atthaṃ avibhajitvā uṭṭhāyāsanā vihāraṃ paviṭṭho:}}\\
\begin{addmargin}[1em]{2em}
\setstretch{.5}
{\PaliGlossB{    -}}\\
\end{addmargin}
\end{absolutelynopagebreak}

\begin{absolutelynopagebreak}
\setstretch{.7}
{\PaliGlossA{‘nāhaṃ, bhikkhave, gamanena lokassa antaṃ ñāteyyaṃ, daṭṭheyyaṃ, patteyyanti vadāmi.}}\\
\begin{addmargin}[1em]{2em}
\setstretch{.5}
{\PaliGlossB{    -}}\\
\end{addmargin}
\end{absolutelynopagebreak}

\begin{absolutelynopagebreak}
\setstretch{.7}
{\PaliGlossA{na ca panāhaṃ, bhikkhave, appatvā lokassa antaṃ dukkhassa antakiriyaṃ vadāmī’ti.}}\\
\begin{addmargin}[1em]{2em}
\setstretch{.5}
{\PaliGlossB{    -}}\\
\end{addmargin}
\end{absolutelynopagebreak}

\begin{absolutelynopagebreak}
\setstretch{.7}
{\PaliGlossA{tesaṃ no, āvuso, amhākaṃ acirapakkantassa bhagavato etadahosi:}}\\
\begin{addmargin}[1em]{2em}
\setstretch{.5}
{\PaliGlossB{    -}}\\
\end{addmargin}
\end{absolutelynopagebreak}

\begin{absolutelynopagebreak}
\setstretch{.7}
{\PaliGlossA{‘idaṃ kho no, āvuso, bhagavā saṅkhittena uddesaṃ uddisitvā vitthārena atthaṃ avibhajitvā uṭṭhāyāsanā vihāraṃ paviṭṭho—}}\\
\begin{addmargin}[1em]{2em}
\setstretch{.5}
{\PaliGlossB{    -}}\\
\end{addmargin}
\end{absolutelynopagebreak}

\begin{absolutelynopagebreak}
\setstretch{.7}
{\PaliGlossA{nāhaṃ, bhikkhave, gamanena lokassa antaṃ ñāteyyaṃ, daṭṭheyyaṃ, patteyyanti vadāmi.}}\\
\begin{addmargin}[1em]{2em}
\setstretch{.5}
{\PaliGlossB{    -}}\\
\end{addmargin}
\end{absolutelynopagebreak}

\begin{absolutelynopagebreak}
\setstretch{.7}
{\PaliGlossA{na ca panāhaṃ, bhikkhave, appatvā lokassa antaṃ dukkhassa antakiriyaṃ vadāmīti.}}\\
\begin{addmargin}[1em]{2em}
\setstretch{.5}
{\PaliGlossB{    -}}\\
\end{addmargin}
\end{absolutelynopagebreak}

\begin{absolutelynopagebreak}
\setstretch{.7}
{\PaliGlossA{ko nu kho imassa bhagavatā saṅkhittena uddesassa uddiṭṭhassa vitthārena atthaṃ avibhattassa vitthārena atthaṃ vibhajeyyā’ti?}}\\
\begin{addmargin}[1em]{2em}
\setstretch{.5}
{\PaliGlossB{    -}}\\
\end{addmargin}
\end{absolutelynopagebreak}

\begin{absolutelynopagebreak}
\setstretch{.7}
{\PaliGlossA{tesaṃ no, āvuso, amhākaṃ etadahosi:}}\\
\begin{addmargin}[1em]{2em}
\setstretch{.5}
{\PaliGlossB{    -}}\\
\end{addmargin}
\end{absolutelynopagebreak}

\begin{absolutelynopagebreak}
\setstretch{.7}
{\PaliGlossA{‘ayaṃ kho, āvuso, āyasmā ānando satthu ceva saṃvaṇṇito, sambhāvito ca viññūnaṃ sabrahmacārīnaṃ.}}\\
\begin{addmargin}[1em]{2em}
\setstretch{.5}
{\PaliGlossB{    -}}\\
\end{addmargin}
\end{absolutelynopagebreak}

\begin{absolutelynopagebreak}
\setstretch{.7}
{\PaliGlossA{pahoti cāyasmā ānando imassa bhagavatā saṅkhittena uddesassa uddiṭṭhassa vitthārena atthaṃ avibhattassa vitthārena atthaṃ vibhajituṃ.}}\\
\begin{addmargin}[1em]{2em}
\setstretch{.5}
{\PaliGlossB{    -}}\\
\end{addmargin}
\end{absolutelynopagebreak}

\begin{absolutelynopagebreak}
\setstretch{.7}
{\PaliGlossA{yannūna mayaṃ yenāyasmā ānando tenupasaṅkameyyāma; upasaṅkamitvā āyasmantaṃ ānandaṃ etamatthaṃ paṭipuccheyyāmā’ti.}}\\
\begin{addmargin}[1em]{2em}
\setstretch{.5}
{\PaliGlossB{    -}}\\
\end{addmargin}
\end{absolutelynopagebreak}

\begin{absolutelynopagebreak}
\setstretch{.7}
{\PaliGlossA{vibhajatāyasmā ānando”ti.}}\\
\begin{addmargin}[1em]{2em}
\setstretch{.5}
{\PaliGlossB{“May Venerable Ānanda please explain this.”}}\\
\end{addmargin}
\end{absolutelynopagebreak}

\begin{absolutelynopagebreak}
\setstretch{.7}
{\PaliGlossA{“seyyathāpi, āvuso, puriso sāratthiko sāragavesī sārapariyesanaṃ caramāno mahato rukkhassa tiṭṭhato sāravato atikkammeva, mūlaṃ atikkammeva, khandhaṃ sākhāpalāse sāraṃ pariyesitabbaṃ maññeyya;}}\\
\begin{addmargin}[1em]{2em}
\setstretch{.5}
{\PaliGlossB{“Reverends, suppose there was a person in need of heartwood. And while wandering in search of heartwood he’d come across a large tree standing with heartwood. But he’d pass over the roots and trunk, imagining that the heartwood should be sought in the branches and leaves.}}\\
\end{addmargin}
\end{absolutelynopagebreak}

\begin{absolutelynopagebreak}
\setstretch{.7}
{\PaliGlossA{evaṃ sampadamidaṃ āyasmantānaṃ satthari sammukhībhūte taṃ bhagavantaṃ atisitvā amhe etamatthaṃ paṭipucchitabbaṃ maññatha.}}\\
\begin{addmargin}[1em]{2em}
\setstretch{.5}
{\PaliGlossB{Such is the consequence for the venerables. Though you were face to face with the Buddha, you passed him by, imagining that you should ask me about this matter.}}\\
\end{addmargin}
\end{absolutelynopagebreak}

\begin{absolutelynopagebreak}
\setstretch{.7}
{\PaliGlossA{so hāvuso, bhagavā jānaṃ jānāti, passaṃ passati—}}\\
\begin{addmargin}[1em]{2em}
\setstretch{.5}
{\PaliGlossB{For he is the Buddha, who knows and sees. He is vision, he is knowledge, he is the truth, he is supreme. He is the teacher, the proclaimer, the elucidator of meaning, the bestower of the deathless, the lord of truth, the Realized One.}}\\
\end{addmargin}
\end{absolutelynopagebreak}

\begin{absolutelynopagebreak}
\setstretch{.7}
{\PaliGlossA{cakkhubhūto, ñāṇabhūto, dhammabhūto, brahmabhūto, vattā, pavattā, atthassa ninnetā, amatassa dātā, dhammassāmī, tathāgato.}}\\
\begin{addmargin}[1em]{2em}
\setstretch{.5}
{\PaliGlossB{    -}}\\
\end{addmargin}
\end{absolutelynopagebreak}

\begin{absolutelynopagebreak}
\setstretch{.7}
{\PaliGlossA{so ceva panetassa kālo ahosi yaṃ bhagavantaṃyeva etamatthaṃ paṭipuccheyyātha.}}\\
\begin{addmargin}[1em]{2em}
\setstretch{.5}
{\PaliGlossB{That was the time to approach the Buddha and ask about this matter.}}\\
\end{addmargin}
\end{absolutelynopagebreak}

\begin{absolutelynopagebreak}
\setstretch{.7}
{\PaliGlossA{yathā vo bhagavā byākareyya tathā vo dhāreyyāthā”ti.}}\\
\begin{addmargin}[1em]{2em}
\setstretch{.5}
{\PaliGlossB{You should have remembered it in line with the Buddha’s answer.”}}\\
\end{addmargin}
\end{absolutelynopagebreak}

\begin{absolutelynopagebreak}
\setstretch{.7}
{\PaliGlossA{“addhāvuso ānanda, bhagavā jānaṃ jānāti, passaṃ passati—}}\\
\begin{addmargin}[1em]{2em}
\setstretch{.5}
{\PaliGlossB{“Certainly he is the Buddha, who knows and sees. He is vision, he is knowledge, he is the truth, he is supreme. He is the teacher, the proclaimer, the elucidator of meaning, the bestower of the deathless, the lord of truth, the Realized One.}}\\
\end{addmargin}
\end{absolutelynopagebreak}

\begin{absolutelynopagebreak}
\setstretch{.7}
{\PaliGlossA{cakkhubhūto, ñāṇabhūto, dhammabhūto, brahmabhūto, vattā, pavattā, atthassa ninnetā, amatassa dātā, dhammassāmī, tathāgato.}}\\
\begin{addmargin}[1em]{2em}
\setstretch{.5}
{\PaliGlossB{    -}}\\
\end{addmargin}
\end{absolutelynopagebreak}

\begin{absolutelynopagebreak}
\setstretch{.7}
{\PaliGlossA{so ceva panetassa kālo ahosi yaṃ bhagavantaṃyeva etamatthaṃ paṭipuccheyyāma.}}\\
\begin{addmargin}[1em]{2em}
\setstretch{.5}
{\PaliGlossB{That was the time to approach the Buddha and ask about this matter.}}\\
\end{addmargin}
\end{absolutelynopagebreak}

\begin{absolutelynopagebreak}
\setstretch{.7}
{\PaliGlossA{yathā no bhagavā byākareyya tathā naṃ dhāreyyāma.}}\\
\begin{addmargin}[1em]{2em}
\setstretch{.5}
{\PaliGlossB{We should have remembered it in line with the Buddha’s answer.}}\\
\end{addmargin}
\end{absolutelynopagebreak}

\begin{absolutelynopagebreak}
\setstretch{.7}
{\PaliGlossA{api cāyasmā ānando satthu ceva saṃvaṇṇito, sambhāvito ca viññūnaṃ sabrahmacārīnaṃ.}}\\
\begin{addmargin}[1em]{2em}
\setstretch{.5}
{\PaliGlossB{Still, Venerable Ānanda is praised by the Buddha and esteemed by his sensible spiritual companions.}}\\
\end{addmargin}
\end{absolutelynopagebreak}

\begin{absolutelynopagebreak}
\setstretch{.7}
{\PaliGlossA{pahoti cāyasmā ānando imassa bhagavatā saṅkhittena uddesassa uddiṭṭhassa vitthārena atthaṃ avibhattassa vitthārena atthaṃ vibhajituṃ.}}\\
\begin{addmargin}[1em]{2em}
\setstretch{.5}
{\PaliGlossB{You are capable of explaining in detail the meaning of this brief passage for recitation given by the Buddha.}}\\
\end{addmargin}
\end{absolutelynopagebreak}

\begin{absolutelynopagebreak}
\setstretch{.7}
{\PaliGlossA{vibhajatāyasmā ānando agaruṃ karitvā”ti.}}\\
\begin{addmargin}[1em]{2em}
\setstretch{.5}
{\PaliGlossB{Please explain this, if it’s no trouble.”}}\\
\end{addmargin}
\end{absolutelynopagebreak}

\begin{absolutelynopagebreak}
\setstretch{.7}
{\PaliGlossA{“tenahāvuso, suṇātha, sādhukaṃ manasi karotha, bhāsissāmī”ti.}}\\
\begin{addmargin}[1em]{2em}
\setstretch{.5}
{\PaliGlossB{“Then listen and pay close attention, I will speak.”}}\\
\end{addmargin}
\end{absolutelynopagebreak}

\begin{absolutelynopagebreak}
\setstretch{.7}
{\PaliGlossA{“evamāvuso”ti kho te bhikkhū āyasmato ānandassa paccassosuṃ.}}\\
\begin{addmargin}[1em]{2em}
\setstretch{.5}
{\PaliGlossB{“Yes, reverend,” they replied.}}\\
\end{addmargin}
\end{absolutelynopagebreak}

\begin{absolutelynopagebreak}
\setstretch{.7}
{\PaliGlossA{āyasmā ānando etadavoca:}}\\
\begin{addmargin}[1em]{2em}
\setstretch{.5}
{\PaliGlossB{Ānanda said this:}}\\
\end{addmargin}
\end{absolutelynopagebreak}

\begin{absolutelynopagebreak}
\setstretch{.7}
{\PaliGlossA{“yaṃ kho vo, āvuso, bhagavā saṅkhittena uddesaṃ uddisitvā vitthārena atthaṃ avibhajitvā uṭṭhāyāsanā vihāraṃ paviṭṭho:}}\\
\begin{addmargin}[1em]{2em}
\setstretch{.5}
{\PaliGlossB{“Reverends, the Buddha gave this brief passage for recitation, then entered his dwelling without explaining the meaning in detail:}}\\
\end{addmargin}
\end{absolutelynopagebreak}

\begin{absolutelynopagebreak}
\setstretch{.7}
{\PaliGlossA{‘nāhaṃ, bhikkhave, gamanena lokassa antaṃ ñāteyyaṃ, daṭṭheyyaṃ, patteyyanti vadāmi.}}\\
\begin{addmargin}[1em]{2em}
\setstretch{.5}
{\PaliGlossB{‘Mendicants, I say it’s not possible to know or see or reach the end of the world by traveling.}}\\
\end{addmargin}
\end{absolutelynopagebreak}

\begin{absolutelynopagebreak}
\setstretch{.7}
{\PaliGlossA{na ca panāhaṃ, bhikkhave, appatvā lokassa antaṃ dukkhassa antakiriyaṃ vadāmī’ti,}}\\
\begin{addmargin}[1em]{2em}
\setstretch{.5}
{\PaliGlossB{But I also say there’s no making an end of suffering without reaching the end of the world.’}}\\
\end{addmargin}
\end{absolutelynopagebreak}

\begin{absolutelynopagebreak}
\setstretch{.7}
{\PaliGlossA{imassa khvāhaṃ, āvuso, bhagavatā saṅkhittena uddesassa uddiṭṭhassa vitthārena atthaṃ avibhattassa vitthārena atthaṃ ājānāmi.}}\\
\begin{addmargin}[1em]{2em}
\setstretch{.5}
{\PaliGlossB{This is how I understand the detailed meaning of this passage for recitation.}}\\
\end{addmargin}
\end{absolutelynopagebreak}

\begin{absolutelynopagebreak}
\setstretch{.7}
{\PaliGlossA{yena kho, āvuso, lokasmiṃ lokasaññī hoti lokamānī—}}\\
\begin{addmargin}[1em]{2em}
\setstretch{.5}
{\PaliGlossB{Whatever in the world through which you perceive the world and conceive the world}}\\
\end{addmargin}
\end{absolutelynopagebreak}

\begin{absolutelynopagebreak}
\setstretch{.7}
{\PaliGlossA{ayaṃ vuccati ariyassa vinaye loko.}}\\
\begin{addmargin}[1em]{2em}
\setstretch{.5}
{\PaliGlossB{is called the world in the training of the noble one.}}\\
\end{addmargin}
\end{absolutelynopagebreak}

\begin{absolutelynopagebreak}
\setstretch{.7}
{\PaliGlossA{kena cāvuso, lokasmiṃ lokasaññī hoti lokamānī?}}\\
\begin{addmargin}[1em]{2em}
\setstretch{.5}
{\PaliGlossB{And through what in the world do you perceive the world and conceive the world?}}\\
\end{addmargin}
\end{absolutelynopagebreak}

\begin{absolutelynopagebreak}
\setstretch{.7}
{\PaliGlossA{cakkhunā kho, āvuso, lokasmiṃ lokasaññī hoti lokamānī.}}\\
\begin{addmargin}[1em]{2em}
\setstretch{.5}
{\PaliGlossB{Through the eye in the world you perceive the world and conceive the world.}}\\
\end{addmargin}
\end{absolutelynopagebreak}

\begin{absolutelynopagebreak}
\setstretch{.7}
{\PaliGlossA{sotena kho, āvuso …}}\\
\begin{addmargin}[1em]{2em}
\setstretch{.5}
{\PaliGlossB{Through the ear …}}\\
\end{addmargin}
\end{absolutelynopagebreak}

\begin{absolutelynopagebreak}
\setstretch{.7}
{\PaliGlossA{ghānena kho, āvuso …}}\\
\begin{addmargin}[1em]{2em}
\setstretch{.5}
{\PaliGlossB{nose …}}\\
\end{addmargin}
\end{absolutelynopagebreak}

\begin{absolutelynopagebreak}
\setstretch{.7}
{\PaliGlossA{jivhāya kho, āvuso, lokasmiṃ lokasaññī hoti lokamānī.}}\\
\begin{addmargin}[1em]{2em}
\setstretch{.5}
{\PaliGlossB{tongue …}}\\
\end{addmargin}
\end{absolutelynopagebreak}

\begin{absolutelynopagebreak}
\setstretch{.7}
{\PaliGlossA{kāyena kho, āvuso …}}\\
\begin{addmargin}[1em]{2em}
\setstretch{.5}
{\PaliGlossB{body …}}\\
\end{addmargin}
\end{absolutelynopagebreak}

\begin{absolutelynopagebreak}
\setstretch{.7}
{\PaliGlossA{manena kho, āvuso, lokasmiṃ lokasaññī hoti lokamānī.}}\\
\begin{addmargin}[1em]{2em}
\setstretch{.5}
{\PaliGlossB{mind in the world you perceive the world and conceive the world.}}\\
\end{addmargin}
\end{absolutelynopagebreak}

\begin{absolutelynopagebreak}
\setstretch{.7}
{\PaliGlossA{yena kho, āvuso, lokasmiṃ lokasaññī hoti lokamānī—}}\\
\begin{addmargin}[1em]{2em}
\setstretch{.5}
{\PaliGlossB{Whatever in the world through which you perceive the world and conceive the world}}\\
\end{addmargin}
\end{absolutelynopagebreak}

\begin{absolutelynopagebreak}
\setstretch{.7}
{\PaliGlossA{ayaṃ vuccati ariyassa vinaye loko.}}\\
\begin{addmargin}[1em]{2em}
\setstretch{.5}
{\PaliGlossB{is called the world in the training of the noble one.}}\\
\end{addmargin}
\end{absolutelynopagebreak}

\begin{absolutelynopagebreak}
\setstretch{.7}
{\PaliGlossA{yaṃ kho vo, āvuso, bhagavā saṅkhittena uddesaṃ uddisitvā vitthārena atthaṃ avibhajitvā uṭṭhāyāsanā vihāraṃ paviṭṭho:}}\\
\begin{addmargin}[1em]{2em}
\setstretch{.5}
{\PaliGlossB{When the Buddha gave this brief passage for recitation, then entered his dwelling without explaining the meaning in detail:}}\\
\end{addmargin}
\end{absolutelynopagebreak}

\begin{absolutelynopagebreak}
\setstretch{.7}
{\PaliGlossA{‘nāhaṃ, bhikkhave, gamanena lokassa antaṃ ñāteyyaṃ, daṭṭheyyaṃ, patteyyanti vadāmi.}}\\
\begin{addmargin}[1em]{2em}
\setstretch{.5}
{\PaliGlossB{‘Mendicants, I say it’s not possible to know or see or reach the end of the world by traveling.}}\\
\end{addmargin}
\end{absolutelynopagebreak}

\begin{absolutelynopagebreak}
\setstretch{.7}
{\PaliGlossA{na ca panāhaṃ, bhikkhave, appatvā lokassa antaṃ dukkhassa antakiriyaṃ vadāmī’ti,}}\\
\begin{addmargin}[1em]{2em}
\setstretch{.5}
{\PaliGlossB{But I also say there’s no making an end of suffering without reaching the end of the world.’}}\\
\end{addmargin}
\end{absolutelynopagebreak}

\begin{absolutelynopagebreak}
\setstretch{.7}
{\PaliGlossA{imassa khvāhaṃ, āvuso, bhagavatā saṅkhittena uddesassa uddiṭṭhassa vitthārena atthaṃ avibhattassa evaṃ vitthārena atthaṃ ājānāmi.}}\\
\begin{addmargin}[1em]{2em}
\setstretch{.5}
{\PaliGlossB{That is how I understand the detailed meaning of this summary.}}\\
\end{addmargin}
\end{absolutelynopagebreak}

\begin{absolutelynopagebreak}
\setstretch{.7}
{\PaliGlossA{ākaṅkhamānā ca pana tumhe āyasmanto bhagavantaṃyeva upasaṅkamitvā etamatthaṃ paṭipuccheyyātha.}}\\
\begin{addmargin}[1em]{2em}
\setstretch{.5}
{\PaliGlossB{If you wish, you may go to the Buddha and ask him about this.}}\\
\end{addmargin}
\end{absolutelynopagebreak}

\begin{absolutelynopagebreak}
\setstretch{.7}
{\PaliGlossA{yathā vo bhagavā byākaroti tathā naṃ dhāreyyāthā”ti.}}\\
\begin{addmargin}[1em]{2em}
\setstretch{.5}
{\PaliGlossB{You should remember it in line with the Buddha’s answer.”}}\\
\end{addmargin}
\end{absolutelynopagebreak}

\begin{absolutelynopagebreak}
\setstretch{.7}
{\PaliGlossA{“evamāvuso”ti kho te bhikkhū āyasmato ānandassa paṭissutvā uṭṭhāyāsanā yena bhagavā tenupasaṅkamiṃsu; upasaṅkamitvā bhagavantaṃ abhivādetvā ekamantaṃ nisīdiṃsu. ekamantaṃ nisinnā kho te bhikkhū bhagavantaṃ etadavocuṃ:}}\\
\begin{addmargin}[1em]{2em}
\setstretch{.5}
{\PaliGlossB{“Yes, reverend,” replied those mendicants. Then they rose from their seats and went to the Buddha, bowed, sat down to one side, and told him what had happened.}}\\
\end{addmargin}
\end{absolutelynopagebreak}

\begin{absolutelynopagebreak}
\setstretch{.7}
{\PaliGlossA{“yaṃ kho no, bhante, bhagavā saṅkhittena uddesaṃ uddisitvā vitthārena atthaṃ avibhajitvā uṭṭhāyāsanā vihāraṃ paviṭṭho:}}\\
\begin{addmargin}[1em]{2em}
\setstretch{.5}
{\PaliGlossB{    -}}\\
\end{addmargin}
\end{absolutelynopagebreak}

\begin{absolutelynopagebreak}
\setstretch{.7}
{\PaliGlossA{‘nāhaṃ, bhikkhave, gamanena lokassa antaṃ ñāteyyaṃ, daṭṭheyyaṃ, patteyyanti vadāmi.}}\\
\begin{addmargin}[1em]{2em}
\setstretch{.5}
{\PaliGlossB{    -}}\\
\end{addmargin}
\end{absolutelynopagebreak}

\begin{absolutelynopagebreak}
\setstretch{.7}
{\PaliGlossA{na ca panāhaṃ, bhikkhave, appatvā lokassa antaṃ dukkhassa antakiriyaṃ vadāmī’ti.}}\\
\begin{addmargin}[1em]{2em}
\setstretch{.5}
{\PaliGlossB{    -}}\\
\end{addmargin}
\end{absolutelynopagebreak}

\begin{absolutelynopagebreak}
\setstretch{.7}
{\PaliGlossA{tesaṃ no, bhante, amhākaṃ acirapakkantassa bhagavato etadahosi:}}\\
\begin{addmargin}[1em]{2em}
\setstretch{.5}
{\PaliGlossB{    -}}\\
\end{addmargin}
\end{absolutelynopagebreak}

\begin{absolutelynopagebreak}
\setstretch{.7}
{\PaliGlossA{‘idaṃ kho no, āvuso, bhagavā saṅkhittena uddesaṃ uddisitvā vitthārena atthaṃ avibhajitvā uṭṭhāyāsanā vihāraṃ paviṭṭho—}}\\
\begin{addmargin}[1em]{2em}
\setstretch{.5}
{\PaliGlossB{    -}}\\
\end{addmargin}
\end{absolutelynopagebreak}

\begin{absolutelynopagebreak}
\setstretch{.7}
{\PaliGlossA{nāhaṃ, bhikkhave, gamanena lokassa antaṃ ñāteyyaṃ, daṭṭheyyaṃ, patteyyanti vadāmi.}}\\
\begin{addmargin}[1em]{2em}
\setstretch{.5}
{\PaliGlossB{    -}}\\
\end{addmargin}
\end{absolutelynopagebreak}

\begin{absolutelynopagebreak}
\setstretch{.7}
{\PaliGlossA{na ca panāhaṃ, bhikkhave, appatvā lokassa antaṃ dukkhassa antakiriyaṃ vadāmīti.}}\\
\begin{addmargin}[1em]{2em}
\setstretch{.5}
{\PaliGlossB{    -}}\\
\end{addmargin}
\end{absolutelynopagebreak}

\begin{absolutelynopagebreak}
\setstretch{.7}
{\PaliGlossA{ko nu kho imassa bhagavatā saṅkhittena uddesassa uddiṭṭhassa vitthārena atthaṃ avibhattassa vitthārena atthaṃ vibhajeyyā’ti?}}\\
\begin{addmargin}[1em]{2em}
\setstretch{.5}
{\PaliGlossB{    -}}\\
\end{addmargin}
\end{absolutelynopagebreak}

\begin{absolutelynopagebreak}
\setstretch{.7}
{\PaliGlossA{tesaṃ no, bhante, amhākaṃ etadahosi:}}\\
\begin{addmargin}[1em]{2em}
\setstretch{.5}
{\PaliGlossB{    -}}\\
\end{addmargin}
\end{absolutelynopagebreak}

\begin{absolutelynopagebreak}
\setstretch{.7}
{\PaliGlossA{‘ayaṃ kho āyasmā ānando satthu ceva saṃvaṇṇito, sambhāvito ca viññūnaṃ sabrahmacārīnaṃ.}}\\
\begin{addmargin}[1em]{2em}
\setstretch{.5}
{\PaliGlossB{    -}}\\
\end{addmargin}
\end{absolutelynopagebreak}

\begin{absolutelynopagebreak}
\setstretch{.7}
{\PaliGlossA{pahoti cāyasmā ānando imassa bhagavatā saṅkhittena uddesassa uddiṭṭhassa vitthārena atthaṃ avibhattassa vitthārena atthaṃ vibhajituṃ.}}\\
\begin{addmargin}[1em]{2em}
\setstretch{.5}
{\PaliGlossB{    -}}\\
\end{addmargin}
\end{absolutelynopagebreak}

\begin{absolutelynopagebreak}
\setstretch{.7}
{\PaliGlossA{yannūna mayaṃ yenāyasmā ānando tenupasaṅkameyyāma; upasaṅkamitvā āyasmantaṃ ānandaṃ etamatthaṃ paṭipuccheyyāmā’ti.}}\\
\begin{addmargin}[1em]{2em}
\setstretch{.5}
{\PaliGlossB{    -}}\\
\end{addmargin}
\end{absolutelynopagebreak}

\begin{absolutelynopagebreak}
\setstretch{.7}
{\PaliGlossA{atha kho mayaṃ, bhante, yenāyasmā ānando tenupasaṅkamimha; upasaṅkamitvā āyasmantaṃ ānandaṃ etamatthaṃ paṭipucchimha.}}\\
\begin{addmargin}[1em]{2em}
\setstretch{.5}
{\PaliGlossB{    -}}\\
\end{addmargin}
\end{absolutelynopagebreak}

\begin{absolutelynopagebreak}
\setstretch{.7}
{\PaliGlossA{tesaṃ no, bhante, āyasmatā ānandena imehi ākārehi imehi padehi imehi byañjanehi attho vibhatto”ti.}}\\
\begin{addmargin}[1em]{2em}
\setstretch{.5}
{\PaliGlossB{Then they said, “And Ānanda explained the meaning to us in this manner, with these words and phrases.”}}\\
\end{addmargin}
\end{absolutelynopagebreak}

\begin{absolutelynopagebreak}
\setstretch{.7}
{\PaliGlossA{“paṇḍito, bhikkhave, ānando; mahāpañño, bhikkhave, ānando.}}\\
\begin{addmargin}[1em]{2em}
\setstretch{.5}
{\PaliGlossB{“Mendicants, Ānanda is astute, he has great wisdom.}}\\
\end{addmargin}
\end{absolutelynopagebreak}

\begin{absolutelynopagebreak}
\setstretch{.7}
{\PaliGlossA{mañcepi tumhe, bhikkhave, etamatthaṃ paṭipuccheyyātha, ahampi taṃ evamevaṃ byākareyyaṃ yathā taṃ ānandena byākataṃ.}}\\
\begin{addmargin}[1em]{2em}
\setstretch{.5}
{\PaliGlossB{If you came to me and asked this question, I would answer it in exactly the same way as Ānanda.}}\\
\end{addmargin}
\end{absolutelynopagebreak}

\begin{absolutelynopagebreak}
\setstretch{.7}
{\PaliGlossA{eso cevetassa attho, evañca naṃ dhāreyyāthā”ti.}}\\
\begin{addmargin}[1em]{2em}
\setstretch{.5}
{\PaliGlossB{That is what it means, and that’s how you should remember it.”}}\\
\end{addmargin}
\end{absolutelynopagebreak}

\begin{absolutelynopagebreak}
\setstretch{.7}
{\PaliGlossA{tatiyaṃ.}}\\
\begin{addmargin}[1em]{2em}
\setstretch{.5}
{\PaliGlossB{    -}}\\
\end{addmargin}
\end{absolutelynopagebreak}
