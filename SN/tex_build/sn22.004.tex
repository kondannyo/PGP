
\begin{absolutelynopagebreak}
\setstretch{.7}
{\PaliGlossA{saṃyutta nikāya 22}}\\
\begin{addmargin}[1em]{2em}
\setstretch{.5}
{\PaliGlossB{Linked Discourses 22}}\\
\end{addmargin}
\end{absolutelynopagebreak}

\begin{absolutelynopagebreak}
\setstretch{.7}
{\PaliGlossA{1. nakulapituvagga}}\\
\begin{addmargin}[1em]{2em}
\setstretch{.5}
{\PaliGlossB{1. Nakula’s Father}}\\
\end{addmargin}
\end{absolutelynopagebreak}

\begin{absolutelynopagebreak}
\setstretch{.7}
{\PaliGlossA{4. dutiyahāliddikānisutta}}\\
\begin{addmargin}[1em]{2em}
\setstretch{.5}
{\PaliGlossB{4. Hāliddikāni (2nd)}}\\
\end{addmargin}
\end{absolutelynopagebreak}

\begin{absolutelynopagebreak}
\setstretch{.7}
{\PaliGlossA{evaṃ me sutaṃ—}}\\
\begin{addmargin}[1em]{2em}
\setstretch{.5}
{\PaliGlossB{So I have heard.}}\\
\end{addmargin}
\end{absolutelynopagebreak}

\begin{absolutelynopagebreak}
\setstretch{.7}
{\PaliGlossA{ekaṃ samayaṃ āyasmā mahākaccāno avantīsu viharati kuraraghare papāte pabbate.}}\\
\begin{addmargin}[1em]{2em}
\setstretch{.5}
{\PaliGlossB{At one time Venerable Mahākaccāna was staying in the land of the Avantis near Kuraraghara on Steep Mountain.}}\\
\end{addmargin}
\end{absolutelynopagebreak}

\begin{absolutelynopagebreak}
\setstretch{.7}
{\PaliGlossA{atha kho hāliddikāni gahapati yenāyasmā mahākaccāno … pe … ekamantaṃ nisinno kho hāliddikāni gahapati āyasmantaṃ mahākaccānaṃ etadavoca:}}\\
\begin{addmargin}[1em]{2em}
\setstretch{.5}
{\PaliGlossB{Then the householder Hāliddikāni went up to Venerable Mahākaccāna … and asked him,}}\\
\end{addmargin}
\end{absolutelynopagebreak}

\begin{absolutelynopagebreak}
\setstretch{.7}
{\PaliGlossA{“vuttamidaṃ, bhante, bhagavatā sakkapañhe:}}\\
\begin{addmargin}[1em]{2em}
\setstretch{.5}
{\PaliGlossB{“Sir, this was said by the Buddha in ‘The Questions of Sakka’:}}\\
\end{addmargin}
\end{absolutelynopagebreak}

\begin{absolutelynopagebreak}
\setstretch{.7}
{\PaliGlossA{‘ye te samaṇabrāhmaṇā taṇhāsaṅkhayavimuttā, te accantaniṭṭhā accantayogakkhemino accantabrahmacārino accantapariyosānā seṭṭhā devamanussānan’ti.}}\\
\begin{addmargin}[1em]{2em}
\setstretch{.5}
{\PaliGlossB{‘Those ascetics and brahmins who are freed due to the ending of craving have reached the ultimate goal, the ultimate sanctuary, the ultimate spiritual life, the ultimate end, and are best among gods and humans.’}}\\
\end{addmargin}
\end{absolutelynopagebreak}

\begin{absolutelynopagebreak}
\setstretch{.7}
{\PaliGlossA{imassa nu kho, bhante, bhagavatā saṅkhittena bhāsitassa kathaṃ vitthārena attho daṭṭhabbo”ti?}}\\
\begin{addmargin}[1em]{2em}
\setstretch{.5}
{\PaliGlossB{How should we see the detailed meaning of the Buddha’s brief statement?”}}\\
\end{addmargin}
\end{absolutelynopagebreak}

\begin{absolutelynopagebreak}
\setstretch{.7}
{\PaliGlossA{“rūpadhātuyā kho, gahapati, yo chando yo rāgo yā nandī yā taṇhā ye upayupādānā cetaso adhiṭṭhānābhinivesānusayā, tesaṃ khayā virāgā nirodhā cāgā paṭinissaggā ‘cittaṃ suvimuttanti’ vuccati.}}\\
\begin{addmargin}[1em]{2em}
\setstretch{.5}
{\PaliGlossB{“Householder, consider any desire, greed, relishing, and craving for the form element; any attraction, grasping, mental fixation, insistence, and underlying tendencies. With the ending, fading away, cessation, giving away, and letting go of that, the mind is said to be ‘well freed’.}}\\
\end{addmargin}
\end{absolutelynopagebreak}

\begin{absolutelynopagebreak}
\setstretch{.7}
{\PaliGlossA{vedanādhātuyā kho, gahapati …}}\\
\begin{addmargin}[1em]{2em}
\setstretch{.5}
{\PaliGlossB{Consider any desire, greed, relishing, and craving for the feeling element …}}\\
\end{addmargin}
\end{absolutelynopagebreak}

\begin{absolutelynopagebreak}
\setstretch{.7}
{\PaliGlossA{saññādhātuyā kho, gahapati …}}\\
\begin{addmargin}[1em]{2em}
\setstretch{.5}
{\PaliGlossB{the perception element …}}\\
\end{addmargin}
\end{absolutelynopagebreak}

\begin{absolutelynopagebreak}
\setstretch{.7}
{\PaliGlossA{saṅkhāradhātuyā kho, gahapati …}}\\
\begin{addmargin}[1em]{2em}
\setstretch{.5}
{\PaliGlossB{the choices element …}}\\
\end{addmargin}
\end{absolutelynopagebreak}

\begin{absolutelynopagebreak}
\setstretch{.7}
{\PaliGlossA{viññāṇadhātuyā kho, gahapati, yo chando yo rāgo yā nandī yā taṇhā ye upayupādānā cetaso adhiṭṭhānābhinivesānusayā, tesaṃ khayā virāgā nirodhā cāgā paṭinissaggā ‘cittaṃ suvimuttanti’ vuccati.}}\\
\begin{addmargin}[1em]{2em}
\setstretch{.5}
{\PaliGlossB{the consciousness element; any attraction, grasping, mental fixation, insistence, and underlying tendencies. With the ending, fading away, cessation, giving away, and letting go of that, the mind is said to be ‘well freed’.}}\\
\end{addmargin}
\end{absolutelynopagebreak}

\begin{absolutelynopagebreak}
\setstretch{.7}
{\PaliGlossA{iti kho, gahapati, yaṃ taṃ vuttaṃ bhagavatā sakkapañhe:}}\\
\begin{addmargin}[1em]{2em}
\setstretch{.5}
{\PaliGlossB{So, householder, that’s how to understand the detailed meaning of what the Buddha said in brief in ‘The Questions of Sakka’:}}\\
\end{addmargin}
\end{absolutelynopagebreak}

\begin{absolutelynopagebreak}
\setstretch{.7}
{\PaliGlossA{‘ye te samaṇabrāhmaṇā taṇhāsaṅkhayavimuttā te accantaniṭṭhā accantayogakkhemino accantabrahmacārino accantapariyosānā seṭṭhā devamanussānan’ti.}}\\
\begin{addmargin}[1em]{2em}
\setstretch{.5}
{\PaliGlossB{‘Those ascetics and brahmins who are freed due to the ending of craving have reached the ultimate goal, the ultimate sanctuary, the ultimate spiritual life, the ultimate end, and are best among gods and humans.’”}}\\
\end{addmargin}
\end{absolutelynopagebreak}

\begin{absolutelynopagebreak}
\setstretch{.7}
{\PaliGlossA{imassa kho, gahapati, bhagavatā saṅkhittena bhāsitassa evaṃ vitthārena attho daṭṭhabbo”ti.}}\\
\begin{addmargin}[1em]{2em}
\setstretch{.5}
{\PaliGlossB{    -}}\\
\end{addmargin}
\end{absolutelynopagebreak}

\begin{absolutelynopagebreak}
\setstretch{.7}
{\PaliGlossA{catutthaṃ.}}\\
\begin{addmargin}[1em]{2em}
\setstretch{.5}
{\PaliGlossB{    -}}\\
\end{addmargin}
\end{absolutelynopagebreak}
