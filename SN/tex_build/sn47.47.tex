
\begin{absolutelynopagebreak}
\setstretch{.7}
{\PaliGlossA{saṃyutta nikāya 47}}\\
\begin{addmargin}[1em]{2em}
\setstretch{.5}
{\PaliGlossB{Linked Discourses 47}}\\
\end{addmargin}
\end{absolutelynopagebreak}

\begin{absolutelynopagebreak}
\setstretch{.7}
{\PaliGlossA{5. amatavagga}}\\
\begin{addmargin}[1em]{2em}
\setstretch{.5}
{\PaliGlossB{5. The Deathless}}\\
\end{addmargin}
\end{absolutelynopagebreak}

\begin{absolutelynopagebreak}
\setstretch{.7}
{\PaliGlossA{47. duccaritasutta}}\\
\begin{addmargin}[1em]{2em}
\setstretch{.5}
{\PaliGlossB{47. Bad Conduct}}\\
\end{addmargin}
\end{absolutelynopagebreak}

\begin{absolutelynopagebreak}
\setstretch{.7}
{\PaliGlossA{atha kho aññataro bhikkhu yena bhagavā tenupasaṅkami … pe …}}\\
\begin{addmargin}[1em]{2em}
\setstretch{.5}
{\PaliGlossB{Then a mendicant went up to the Buddha … and said:}}\\
\end{addmargin}
\end{absolutelynopagebreak}

\begin{absolutelynopagebreak}
\setstretch{.7}
{\PaliGlossA{“sādhu me, bhante, bhagavā saṅkhittena dhammaṃ desetu, yamahaṃ bhagavato dhammaṃ sutvā eko vūpakaṭṭho appamatto ātāpī pahitatto vihareyyan”ti.}}\\
\begin{addmargin}[1em]{2em}
\setstretch{.5}
{\PaliGlossB{“Sir, may the Buddha please teach me Dhamma in brief. When I’ve heard it, I’ll live alone, withdrawn, diligent, keen, and resolute.”}}\\
\end{addmargin}
\end{absolutelynopagebreak}

\begin{absolutelynopagebreak}
\setstretch{.7}
{\PaliGlossA{“tasmātiha tvaṃ, bhikkhu, ādimeva visodhehi kusalesu dhammesu.}}\\
\begin{addmargin}[1em]{2em}
\setstretch{.5}
{\PaliGlossB{“Well then, mendicant, you should purify skillful qualities starting from the beginning.}}\\
\end{addmargin}
\end{absolutelynopagebreak}

\begin{absolutelynopagebreak}
\setstretch{.7}
{\PaliGlossA{ko cādi kusalānaṃ dhammānaṃ?}}\\
\begin{addmargin}[1em]{2em}
\setstretch{.5}
{\PaliGlossB{What is the beginning of skillful qualities?}}\\
\end{addmargin}
\end{absolutelynopagebreak}

\begin{absolutelynopagebreak}
\setstretch{.7}
{\PaliGlossA{idha tvaṃ, bhikkhu, kāyaduccaritaṃ pahāya kāyasucaritaṃ bhāvessasi.}}\\
\begin{addmargin}[1em]{2em}
\setstretch{.5}
{\PaliGlossB{Give up bad conduct by way of body, speech, and mind and develop good conduct by way of body, speech, and mind.}}\\
\end{addmargin}
\end{absolutelynopagebreak}

\begin{absolutelynopagebreak}
\setstretch{.7}
{\PaliGlossA{vacīduccaritaṃ pahāya vacīsucaritaṃ bhāvessasi.}}\\
\begin{addmargin}[1em]{2em}
\setstretch{.5}
{\PaliGlossB{    -}}\\
\end{addmargin}
\end{absolutelynopagebreak}

\begin{absolutelynopagebreak}
\setstretch{.7}
{\PaliGlossA{manoduccaritaṃ pahāya manosucaritaṃ bhāvessasi.}}\\
\begin{addmargin}[1em]{2em}
\setstretch{.5}
{\PaliGlossB{    -}}\\
\end{addmargin}
\end{absolutelynopagebreak}

\begin{absolutelynopagebreak}
\setstretch{.7}
{\PaliGlossA{yato kho tvaṃ, bhikkhu, kāyaduccaritaṃ pahāya kāyasucaritaṃ bhāvessasi, vacīduccaritaṃ pahāya vacīsucaritaṃ bhāvessasi, manoduccaritaṃ pahāya manosucaritaṃ bhāvessasi, tato tvaṃ, bhikkhu, sīlaṃ nissāya sīle patiṭṭhāya cattāro satipaṭṭhāne bhāveyyāsi.}}\\
\begin{addmargin}[1em]{2em}
\setstretch{.5}
{\PaliGlossB{When you’ve done this, you should develop the four kinds of mindfulness meditation, depending on and grounded on ethics.}}\\
\end{addmargin}
\end{absolutelynopagebreak}

\begin{absolutelynopagebreak}
\setstretch{.7}
{\PaliGlossA{katame cattāro?}}\\
\begin{addmargin}[1em]{2em}
\setstretch{.5}
{\PaliGlossB{What four?}}\\
\end{addmargin}
\end{absolutelynopagebreak}

\begin{absolutelynopagebreak}
\setstretch{.7}
{\PaliGlossA{idha tvaṃ, bhikkhu, kāye kāyānupassī viharāhi ātāpī sampajāno satimā, vineyya loke abhijjhādomanassaṃ;}}\\
\begin{addmargin}[1em]{2em}
\setstretch{.5}
{\PaliGlossB{Meditate observing an aspect of the body internally—keen, aware, and mindful, rid of desire and aversion for the world.}}\\
\end{addmargin}
\end{absolutelynopagebreak}

\begin{absolutelynopagebreak}
\setstretch{.7}
{\PaliGlossA{vedanāsu … pe …}}\\
\begin{addmargin}[1em]{2em}
\setstretch{.5}
{\PaliGlossB{Meditate observing an aspect of feelings …}}\\
\end{addmargin}
\end{absolutelynopagebreak}

\begin{absolutelynopagebreak}
\setstretch{.7}
{\PaliGlossA{citte … pe …}}\\
\begin{addmargin}[1em]{2em}
\setstretch{.5}
{\PaliGlossB{mind …}}\\
\end{addmargin}
\end{absolutelynopagebreak}

\begin{absolutelynopagebreak}
\setstretch{.7}
{\PaliGlossA{dhammesu dhammānupassī viharāhi ātāpī sampajāno satimā, vineyya loke abhijjhādomanassaṃ.}}\\
\begin{addmargin}[1em]{2em}
\setstretch{.5}
{\PaliGlossB{principles—keen, aware, and mindful, rid of desire and aversion for the world.}}\\
\end{addmargin}
\end{absolutelynopagebreak}

\begin{absolutelynopagebreak}
\setstretch{.7}
{\PaliGlossA{yato kho tvaṃ, bhikkhu, sīlaṃ nissāya sīle patiṭṭhāya ime cattāro satipaṭṭhāne evaṃ bhāvessasi, tato tuyhaṃ, bhikkhu, yā ratti vā divaso vā āgamissati vuddhiyeva pāṭikaṅkhā kusalesu dhammesu, no parihānī”ti … pe …}}\\
\begin{addmargin}[1em]{2em}
\setstretch{.5}
{\PaliGlossB{When you develop the four kinds of mindfulness meditation in this way, depending on and grounded on ethics, you can expect growth, not decline, in skillful qualities, whether by day or by night.” …}}\\
\end{addmargin}
\end{absolutelynopagebreak}

\begin{absolutelynopagebreak}
\setstretch{.7}
{\PaliGlossA{aññataro ca pana so bhikkhu arahataṃ ahosīti.}}\\
\begin{addmargin}[1em]{2em}
\setstretch{.5}
{\PaliGlossB{And that mendicant became one of the perfected.}}\\
\end{addmargin}
\end{absolutelynopagebreak}

\begin{absolutelynopagebreak}
\setstretch{.7}
{\PaliGlossA{sattamaṃ.}}\\
\begin{addmargin}[1em]{2em}
\setstretch{.5}
{\PaliGlossB{    -}}\\
\end{addmargin}
\end{absolutelynopagebreak}
