
\begin{absolutelynopagebreak}
\setstretch{.7}
{\PaliGlossA{saṃyutta nikāya 48}}\\
\begin{addmargin}[1em]{2em}
\setstretch{.5}
{\PaliGlossB{Linked Discourses 48}}\\
\end{addmargin}
\end{absolutelynopagebreak}

\begin{absolutelynopagebreak}
\setstretch{.7}
{\PaliGlossA{5. jarāvagga}}\\
\begin{addmargin}[1em]{2em}
\setstretch{.5}
{\PaliGlossB{5. Old Age}}\\
\end{addmargin}
\end{absolutelynopagebreak}

\begin{absolutelynopagebreak}
\setstretch{.7}
{\PaliGlossA{42. uṇṇābhabrāhmaṇasutta}}\\
\begin{addmargin}[1em]{2em}
\setstretch{.5}
{\PaliGlossB{42. The Brahmin Uṇṇābha}}\\
\end{addmargin}
\end{absolutelynopagebreak}

\begin{absolutelynopagebreak}
\setstretch{.7}
{\PaliGlossA{sāvatthinidānaṃ.}}\\
\begin{addmargin}[1em]{2em}
\setstretch{.5}
{\PaliGlossB{At Sāvatthī.}}\\
\end{addmargin}
\end{absolutelynopagebreak}

\begin{absolutelynopagebreak}
\setstretch{.7}
{\PaliGlossA{atha kho uṇṇābho brāhmaṇo yena bhagavā tenupasaṅkami; upasaṅkamitvā bhagavatā saddhiṃ sammodi.}}\\
\begin{addmargin}[1em]{2em}
\setstretch{.5}
{\PaliGlossB{Then Uṇṇābha the brahmin went up to the Buddha, and exchanged greetings with him.}}\\
\end{addmargin}
\end{absolutelynopagebreak}

\begin{absolutelynopagebreak}
\setstretch{.7}
{\PaliGlossA{sammodanīyaṃ kathaṃ sāraṇīyaṃ vītisāretvā ekamantaṃ nisīdi. ekamantaṃ nisinno kho uṇṇābho brāhmaṇo bhagavantaṃ etadavoca:}}\\
\begin{addmargin}[1em]{2em}
\setstretch{.5}
{\PaliGlossB{When the greetings and polite conversation were over, he sat down to one side and said to the Buddha:}}\\
\end{addmargin}
\end{absolutelynopagebreak}

\begin{absolutelynopagebreak}
\setstretch{.7}
{\PaliGlossA{“pañcimāni, bho gotama, indriyāni nānāvisayāni nānāgocarāni, na aññamaññassa gocaravisayaṃ paccanubhonti.}}\\
\begin{addmargin}[1em]{2em}
\setstretch{.5}
{\PaliGlossB{“Master Gotama, these five faculties have different scopes and different ranges, and don’t experience each others’ scope and range.}}\\
\end{addmargin}
\end{absolutelynopagebreak}

\begin{absolutelynopagebreak}
\setstretch{.7}
{\PaliGlossA{katamāni pañca?}}\\
\begin{addmargin}[1em]{2em}
\setstretch{.5}
{\PaliGlossB{What five?}}\\
\end{addmargin}
\end{absolutelynopagebreak}

\begin{absolutelynopagebreak}
\setstretch{.7}
{\PaliGlossA{cakkhundriyaṃ, sotindriyaṃ, ghānindriyaṃ, jivhindriyaṃ, kāyindriyaṃ.}}\\
\begin{addmargin}[1em]{2em}
\setstretch{.5}
{\PaliGlossB{The faculties of the eye, ear, nose, tongue, and body.}}\\
\end{addmargin}
\end{absolutelynopagebreak}

\begin{absolutelynopagebreak}
\setstretch{.7}
{\PaliGlossA{imesaṃ nu kho, bho gotama, pañcannaṃ indriyānaṃ nānāvisayānaṃ nānāgocarānaṃ na aññamaññassa gocaravisayaṃ paccanubhontānaṃ kiṃ paṭisaraṇaṃ, ko ca nesaṃ gocaravisayaṃ paccanubhotī”ti?}}\\
\begin{addmargin}[1em]{2em}
\setstretch{.5}
{\PaliGlossB{What do these five faculties, with their different scopes and ranges, have recourse to? What experiences their scopes and ranges?”}}\\
\end{addmargin}
\end{absolutelynopagebreak}

\begin{absolutelynopagebreak}
\setstretch{.7}
{\PaliGlossA{“pañcimāni, brāhmaṇa, indriyāni nānāvisayāni nānāgocarāni na aññamaññassa gocaravisayaṃ paccanubhonti.}}\\
\begin{addmargin}[1em]{2em}
\setstretch{.5}
{\PaliGlossB{“Brahmin, these five faculties have different scopes and different ranges, and don’t experience each others’ scope and range.}}\\
\end{addmargin}
\end{absolutelynopagebreak}

\begin{absolutelynopagebreak}
\setstretch{.7}
{\PaliGlossA{katamāni pañca?}}\\
\begin{addmargin}[1em]{2em}
\setstretch{.5}
{\PaliGlossB{What five?}}\\
\end{addmargin}
\end{absolutelynopagebreak}

\begin{absolutelynopagebreak}
\setstretch{.7}
{\PaliGlossA{cakkhundriyaṃ, sotindriyaṃ, ghānindriyaṃ, jivhindriyaṃ, kāyindriyaṃ.}}\\
\begin{addmargin}[1em]{2em}
\setstretch{.5}
{\PaliGlossB{The faculties of the eye, ear, nose, tongue, and body.}}\\
\end{addmargin}
\end{absolutelynopagebreak}

\begin{absolutelynopagebreak}
\setstretch{.7}
{\PaliGlossA{imesaṃ kho, brāhmaṇa, pañcannaṃ indriyānaṃ nānāvisayānaṃ nānāgocarānaṃ na aññamaññassa gocaravisayaṃ paccanubhontānaṃ mano paṭisaraṇaṃ, manova nesaṃ gocaravisayaṃ paccanubhotī”ti.}}\\
\begin{addmargin}[1em]{2em}
\setstretch{.5}
{\PaliGlossB{These five faculties, with their different scopes and ranges, have recourse to the mind. And the mind experiences their scopes and ranges.”}}\\
\end{addmargin}
\end{absolutelynopagebreak}

\begin{absolutelynopagebreak}
\setstretch{.7}
{\PaliGlossA{“manassa pana, bho gotama, kiṃ paṭisaraṇan”ti?}}\\
\begin{addmargin}[1em]{2em}
\setstretch{.5}
{\PaliGlossB{“But Master Gotama, what does the mind have recourse to?”}}\\
\end{addmargin}
\end{absolutelynopagebreak}

\begin{absolutelynopagebreak}
\setstretch{.7}
{\PaliGlossA{“manassa kho, brāhmaṇa, sati paṭisaraṇan”ti.}}\\
\begin{addmargin}[1em]{2em}
\setstretch{.5}
{\PaliGlossB{“The mind has recourse to mindfulness.”}}\\
\end{addmargin}
\end{absolutelynopagebreak}

\begin{absolutelynopagebreak}
\setstretch{.7}
{\PaliGlossA{“satiyā pana, bho gotama, kiṃ paṭisaraṇan”ti?}}\\
\begin{addmargin}[1em]{2em}
\setstretch{.5}
{\PaliGlossB{“But what does mindfulness have recourse to?”}}\\
\end{addmargin}
\end{absolutelynopagebreak}

\begin{absolutelynopagebreak}
\setstretch{.7}
{\PaliGlossA{“satiyā kho, brāhmaṇa, vimutti paṭisaraṇan”ti.}}\\
\begin{addmargin}[1em]{2em}
\setstretch{.5}
{\PaliGlossB{“Mindfulness has recourse to freedom.”}}\\
\end{addmargin}
\end{absolutelynopagebreak}

\begin{absolutelynopagebreak}
\setstretch{.7}
{\PaliGlossA{“vimuttiyā pana, bho gotama, kiṃ paṭisaraṇan”ti?}}\\
\begin{addmargin}[1em]{2em}
\setstretch{.5}
{\PaliGlossB{“But what does freedom have recourse to?”}}\\
\end{addmargin}
\end{absolutelynopagebreak}

\begin{absolutelynopagebreak}
\setstretch{.7}
{\PaliGlossA{“vimuttiyā kho, brāhmaṇa, nibbānaṃ paṭisaraṇan”ti.}}\\
\begin{addmargin}[1em]{2em}
\setstretch{.5}
{\PaliGlossB{“Freedom has recourse to extinguishment.”}}\\
\end{addmargin}
\end{absolutelynopagebreak}

\begin{absolutelynopagebreak}
\setstretch{.7}
{\PaliGlossA{“nibbānassa pana, bho gotama, kiṃ paṭisaraṇan”ti?}}\\
\begin{addmargin}[1em]{2em}
\setstretch{.5}
{\PaliGlossB{“But what does extinguishment have recourse to?”}}\\
\end{addmargin}
\end{absolutelynopagebreak}

\begin{absolutelynopagebreak}
\setstretch{.7}
{\PaliGlossA{“accayāsi, brāhmaṇa, pañhaṃ, nāsakkhi pañhassa pariyantaṃ gahetuṃ.}}\\
\begin{addmargin}[1em]{2em}
\setstretch{.5}
{\PaliGlossB{“This question goes too far, brahmin! You weren’t able to grasp the limit of questioning.}}\\
\end{addmargin}
\end{absolutelynopagebreak}

\begin{absolutelynopagebreak}
\setstretch{.7}
{\PaliGlossA{nibbānogadhañhi, brāhmaṇa, brahmacariyaṃ vussati nibbānaparāyaṇaṃ nibbānapariyosānan”ti.}}\\
\begin{addmargin}[1em]{2em}
\setstretch{.5}
{\PaliGlossB{For extinguishment is the culmination, destination, and end of the spiritual life.”}}\\
\end{addmargin}
\end{absolutelynopagebreak}

\begin{absolutelynopagebreak}
\setstretch{.7}
{\PaliGlossA{atha kho uṇṇābho brāhmaṇo bhagavato bhāsitaṃ abhinanditvā anumoditvā uṭṭhāyāsanā bhagavantaṃ abhivādetvā padakkhiṇaṃ katvā pakkāmi.}}\\
\begin{addmargin}[1em]{2em}
\setstretch{.5}
{\PaliGlossB{And then the brahmin Uṇṇābha approved and agreed with what the Buddha said. He got up from his seat, bowed, and respectfully circled the Buddha, keeping him on his right, before leaving.}}\\
\end{addmargin}
\end{absolutelynopagebreak}

\begin{absolutelynopagebreak}
\setstretch{.7}
{\PaliGlossA{atha kho bhagavā acirapakkante uṇṇābhe brāhmaṇe bhikkhū āmantesi:}}\\
\begin{addmargin}[1em]{2em}
\setstretch{.5}
{\PaliGlossB{Then, not long after he had left, the Buddha addressed the mendicants:}}\\
\end{addmargin}
\end{absolutelynopagebreak}

\begin{absolutelynopagebreak}
\setstretch{.7}
{\PaliGlossA{“seyyathāpi, bhikkhave, kūṭāgāre vā kūṭāgārasālāyaṃ vā pācīnavātapānā sūriye uggacchante vātapānena rasmi pavisitvā kvāssa patiṭṭhitā”ti?}}\\
\begin{addmargin}[1em]{2em}
\setstretch{.5}
{\PaliGlossB{“Suppose there was a bungalow or a hall with a peaked roof, with windows on the eastern side. When the sun rises and a ray of light enters through a window, where would it land?”}}\\
\end{addmargin}
\end{absolutelynopagebreak}

\begin{absolutelynopagebreak}
\setstretch{.7}
{\PaliGlossA{“pacchimāyaṃ, bhante, bhittiyan”ti.}}\\
\begin{addmargin}[1em]{2em}
\setstretch{.5}
{\PaliGlossB{“On the western wall, sir.”}}\\
\end{addmargin}
\end{absolutelynopagebreak}

\begin{absolutelynopagebreak}
\setstretch{.7}
{\PaliGlossA{“evameva kho, bhikkhave, uṇṇābhassa brāhmaṇassa tathāgate saddhā niviṭṭhā mūlajātā patiṭṭhitā daḷhā asaṃhāriyā samaṇena vā brāhmaṇena vā devena vā mārena vā brahmunā vā kenaci vā lokasmiṃ.}}\\
\begin{addmargin}[1em]{2em}
\setstretch{.5}
{\PaliGlossB{“In the same way, the brahmin Uṇṇābha’s faith in the Realized One is settled, rooted, and planted deep. It’s strong and can’t be shifted by any ascetic or brahmin or god or Māra or Brahmā or by anyone in the world.}}\\
\end{addmargin}
\end{absolutelynopagebreak}

\begin{absolutelynopagebreak}
\setstretch{.7}
{\PaliGlossA{imamhi ce, bhikkhave, samaye uṇṇābho brāhmaṇo kālaṃ kareyya, natthi taṃ saṃyojanaṃ yena saṃyojanena saṃyutto uṇṇābho brāhmaṇo puna imaṃ lokaṃ āgaccheyyā”ti.}}\\
\begin{addmargin}[1em]{2em}
\setstretch{.5}
{\PaliGlossB{If he were to pass away at this time, he would be bound by no fetter that might return him to this world.”}}\\
\end{addmargin}
\end{absolutelynopagebreak}

\begin{absolutelynopagebreak}
\setstretch{.7}
{\PaliGlossA{dutiyaṃ.}}\\
\begin{addmargin}[1em]{2em}
\setstretch{.5}
{\PaliGlossB{    -}}\\
\end{addmargin}
\end{absolutelynopagebreak}
