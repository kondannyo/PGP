
\begin{absolutelynopagebreak}
\setstretch{.7}
{\PaliGlossA{saṃyutta nikāya 56}}\\
\begin{addmargin}[1em]{2em}
\setstretch{.5}
{\PaliGlossB{Linked Discourses 56}}\\
\end{addmargin}
\end{absolutelynopagebreak}

\begin{absolutelynopagebreak}
\setstretch{.7}
{\PaliGlossA{1. samādhivagga}}\\
\begin{addmargin}[1em]{2em}
\setstretch{.5}
{\PaliGlossB{1. Immersion}}\\
\end{addmargin}
\end{absolutelynopagebreak}

\begin{absolutelynopagebreak}
\setstretch{.7}
{\PaliGlossA{3. paṭhamakulaputtasutta}}\\
\begin{addmargin}[1em]{2em}
\setstretch{.5}
{\PaliGlossB{3. A Gentleman (1st)}}\\
\end{addmargin}
\end{absolutelynopagebreak}

\begin{absolutelynopagebreak}
\setstretch{.7}
{\PaliGlossA{“ye hi keci, bhikkhave, atītamaddhānaṃ kulaputtā sammā agārasmā anagāriyaṃ pabbajiṃsu, sabbe te catunnaṃ ariyasaccānaṃ yathābhūtaṃ abhisamayāya.}}\\
\begin{addmargin}[1em]{2em}
\setstretch{.5}
{\PaliGlossB{“Mendicants, whatever gentlemen—past,}}\\
\end{addmargin}
\end{absolutelynopagebreak}

\begin{absolutelynopagebreak}
\setstretch{.7}
{\PaliGlossA{ye hi keci, bhikkhave, anāgatamaddhānaṃ kulaputtā sammā agārasmā anagāriyaṃ pabbajissanti, sabbe te catunnaṃ ariyasaccānaṃ yathābhūtaṃ abhisamayāya.}}\\
\begin{addmargin}[1em]{2em}
\setstretch{.5}
{\PaliGlossB{future,}}\\
\end{addmargin}
\end{absolutelynopagebreak}

\begin{absolutelynopagebreak}
\setstretch{.7}
{\PaliGlossA{ye hi keci, bhikkhave, etarahi kulaputtā sammā agārasmā anagāriyaṃ pabbajanti, sabbe te catunnaṃ ariyasaccānaṃ yathābhūtaṃ abhisamayāya.}}\\
\begin{addmargin}[1em]{2em}
\setstretch{.5}
{\PaliGlossB{or present—rightly go forth from the lay life to homelessness, all of them do so in order to truly comprehend the four noble truths.}}\\
\end{addmargin}
\end{absolutelynopagebreak}

\begin{absolutelynopagebreak}
\setstretch{.7}
{\PaliGlossA{katamesaṃ catunnaṃ?}}\\
\begin{addmargin}[1em]{2em}
\setstretch{.5}
{\PaliGlossB{What four?}}\\
\end{addmargin}
\end{absolutelynopagebreak}

\begin{absolutelynopagebreak}
\setstretch{.7}
{\PaliGlossA{dukkhassa ariyasaccassa dukkhasamudayassa ariyasaccassa dukkhanirodhassa ariyasaccassa dukkhanirodhagāminiyā paṭipadāya ariyasaccassa.}}\\
\begin{addmargin}[1em]{2em}
\setstretch{.5}
{\PaliGlossB{The noble truths of suffering, the origin of suffering, the cessation of suffering, and the practice that leads to the cessation of suffering.}}\\
\end{addmargin}
\end{absolutelynopagebreak}

\begin{absolutelynopagebreak}
\setstretch{.7}
{\PaliGlossA{ye hi keci, bhikkhave, atītamaddhānaṃ kulaputtā sammā agārasmā anagāriyaṃ pabbajiṃsu … pe …}}\\
\begin{addmargin}[1em]{2em}
\setstretch{.5}
{\PaliGlossB{Whatever gentlemen—past,}}\\
\end{addmargin}
\end{absolutelynopagebreak}

\begin{absolutelynopagebreak}
\setstretch{.7}
{\PaliGlossA{pabbajissanti … pe …}}\\
\begin{addmargin}[1em]{2em}
\setstretch{.5}
{\PaliGlossB{future,}}\\
\end{addmargin}
\end{absolutelynopagebreak}

\begin{absolutelynopagebreak}
\setstretch{.7}
{\PaliGlossA{pabbajanti, sabbe te imesaṃyeva catunnaṃ ariyasaccānaṃ yathābhūtaṃ abhisamayāya.}}\\
\begin{addmargin}[1em]{2em}
\setstretch{.5}
{\PaliGlossB{or present—rightly go forth from the lay life to homelessness, all of them do so in order to truly comprehend the four noble truths.}}\\
\end{addmargin}
\end{absolutelynopagebreak}

\begin{absolutelynopagebreak}
\setstretch{.7}
{\PaliGlossA{tasmātiha, bhikkhave, ‘idaṃ dukkhan’ti yogo karaṇīyo, ‘ayaṃ dukkhasamudayo’ti yogo karaṇīyo, ‘ayaṃ dukkhanirodho’ti yogo karaṇīyo, ‘ayaṃ dukkhanirodhagāminī paṭipadā’ti yogo karaṇīyo”ti.}}\\
\begin{addmargin}[1em]{2em}
\setstretch{.5}
{\PaliGlossB{That’s why you should practice meditation to understand: ‘This is suffering’ … ‘This is the origin of suffering’ … ‘This is the cessation of suffering’ … ‘This is the practice that leads to the cessation of suffering’.”}}\\
\end{addmargin}
\end{absolutelynopagebreak}

\begin{absolutelynopagebreak}
\setstretch{.7}
{\PaliGlossA{tatiyaṃ.}}\\
\begin{addmargin}[1em]{2em}
\setstretch{.5}
{\PaliGlossB{    -}}\\
\end{addmargin}
\end{absolutelynopagebreak}
