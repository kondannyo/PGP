
\begin{absolutelynopagebreak}
\setstretch{.7}
{\PaliGlossA{saṃyutta nikāya 45}}\\
\begin{addmargin}[1em]{2em}
\setstretch{.5}
{\PaliGlossB{Linked Discourses 45}}\\
\end{addmargin}
\end{absolutelynopagebreak}

\begin{absolutelynopagebreak}
\setstretch{.7}
{\PaliGlossA{12. balakaraṇīyavagga}}\\
\begin{addmargin}[1em]{2em}
\setstretch{.5}
{\PaliGlossB{12. Hard Work}}\\
\end{addmargin}
\end{absolutelynopagebreak}

\begin{absolutelynopagebreak}
\setstretch{.7}
{\PaliGlossA{160. nadīsutta}}\\
\begin{addmargin}[1em]{2em}
\setstretch{.5}
{\PaliGlossB{160. A River}}\\
\end{addmargin}
\end{absolutelynopagebreak}

\begin{absolutelynopagebreak}
\setstretch{.7}
{\PaliGlossA{“seyyathāpi, bhikkhave, gaṅgā nadī pācīnaninnā pācīnapoṇā pācīnapabbhārā.}}\\
\begin{addmargin}[1em]{2em}
\setstretch{.5}
{\PaliGlossB{“Mendicants, suppose that, although the Ganges river slants, slopes, and inclines to the east,}}\\
\end{addmargin}
\end{absolutelynopagebreak}

\begin{absolutelynopagebreak}
\setstretch{.7}
{\PaliGlossA{atha mahājanakāyo āgaccheyya kuddālapiṭakaṃ ādāya:}}\\
\begin{addmargin}[1em]{2em}
\setstretch{.5}
{\PaliGlossB{a large crowd were to come along with a spade and basket, saying:}}\\
\end{addmargin}
\end{absolutelynopagebreak}

\begin{absolutelynopagebreak}
\setstretch{.7}
{\PaliGlossA{‘mayaṃ imaṃ gaṅgaṃ nadiṃ pacchāninnaṃ karissāma pacchāpoṇaṃ pacchāpabbhāran’ti.}}\\
\begin{addmargin}[1em]{2em}
\setstretch{.5}
{\PaliGlossB{‘We’ll make this Ganges river slant, slope, and incline to the west!’}}\\
\end{addmargin}
\end{absolutelynopagebreak}

\begin{absolutelynopagebreak}
\setstretch{.7}
{\PaliGlossA{taṃ kiṃ maññatha, bhikkhave,}}\\
\begin{addmargin}[1em]{2em}
\setstretch{.5}
{\PaliGlossB{What do you think, mendicants?}}\\
\end{addmargin}
\end{absolutelynopagebreak}

\begin{absolutelynopagebreak}
\setstretch{.7}
{\PaliGlossA{api nu so mahājanakāyo gaṅgaṃ nadiṃ pacchāninnaṃ kareyya pacchāpoṇaṃ pacchāpabbhāran”ti?}}\\
\begin{addmargin}[1em]{2em}
\setstretch{.5}
{\PaliGlossB{Would they succeed?”}}\\
\end{addmargin}
\end{absolutelynopagebreak}

\begin{absolutelynopagebreak}
\setstretch{.7}
{\PaliGlossA{“no hetaṃ, bhante”.}}\\
\begin{addmargin}[1em]{2em}
\setstretch{.5}
{\PaliGlossB{“No, sir.}}\\
\end{addmargin}
\end{absolutelynopagebreak}

\begin{absolutelynopagebreak}
\setstretch{.7}
{\PaliGlossA{“taṃ kissa hetu”?}}\\
\begin{addmargin}[1em]{2em}
\setstretch{.5}
{\PaliGlossB{Why is that?}}\\
\end{addmargin}
\end{absolutelynopagebreak}

\begin{absolutelynopagebreak}
\setstretch{.7}
{\PaliGlossA{“gaṅgā, bhante, nadī pācīnaninnā pācīnapoṇā pācīnapabbhārā.}}\\
\begin{addmargin}[1em]{2em}
\setstretch{.5}
{\PaliGlossB{The Ganges river slants, slopes, and inclines to the east.}}\\
\end{addmargin}
\end{absolutelynopagebreak}

\begin{absolutelynopagebreak}
\setstretch{.7}
{\PaliGlossA{sā na sukarā pacchāninnaṃ kātuṃ pacchāpoṇaṃ pacchāpabbhāraṃ.}}\\
\begin{addmargin}[1em]{2em}
\setstretch{.5}
{\PaliGlossB{It’s not easy to make it slant, slope, and incline to the west.}}\\
\end{addmargin}
\end{absolutelynopagebreak}

\begin{absolutelynopagebreak}
\setstretch{.7}
{\PaliGlossA{yāvadeva pana so mahājanakāyo kilamathassa vighātassa bhāgī assā”ti.}}\\
\begin{addmargin}[1em]{2em}
\setstretch{.5}
{\PaliGlossB{That large crowd will eventually get weary and frustrated.”}}\\
\end{addmargin}
\end{absolutelynopagebreak}

\begin{absolutelynopagebreak}
\setstretch{.7}
{\PaliGlossA{“evameva kho, bhikkhave, bhikkhuṃ ariyaṃ aṭṭhaṅgikaṃ maggaṃ bhāventaṃ ariyaṃ aṭṭhaṅgikaṃ maggaṃ bahulīkarontaṃ rājāno vā rājamahāmattā vā mittā vā amaccā vā ñātī vā ñātisālohitā vā bhogehi abhihaṭṭhuṃ pavāreyyuṃ:}}\\
\begin{addmargin}[1em]{2em}
\setstretch{.5}
{\PaliGlossB{“In the same way, while a mendicant develops and cultivates the noble eightfold path, if rulers or their ministers, friends or colleagues, relatives or family should invite them to accept wealth, saying:}}\\
\end{addmargin}
\end{absolutelynopagebreak}

\begin{absolutelynopagebreak}
\setstretch{.7}
{\PaliGlossA{‘ehambho purisa, kiṃ te ime kāsāvā anudahanti, kiṃ muṇḍo kapālamanusaṃcarasi. ehi, hīnāyāvattitvā bhoge ca bhuñjassu, puññāni ca karohī’ti.}}\\
\begin{addmargin}[1em]{2em}
\setstretch{.5}
{\PaliGlossB{‘Please, mister, why let these ocher robes torment you? Why follow the practice of shaving your head and carrying an alms bowl? Come, return to a lesser life, enjoy wealth, and make merit!’}}\\
\end{addmargin}
\end{absolutelynopagebreak}

\begin{absolutelynopagebreak}
\setstretch{.7}
{\PaliGlossA{so vata, bhikkhave, bhikkhu ariyaṃ aṭṭhaṅgikaṃ maggaṃ bhāvento ariyaṃ aṭṭhaṅgikaṃ maggaṃ bahulīkaronto sikkhaṃ paccakkhāya hīnāyāvattissatīti—netaṃ ṭhānaṃ vijjati.}}\\
\begin{addmargin}[1em]{2em}
\setstretch{.5}
{\PaliGlossB{It’s simply impossible for a mendicant who develops and cultivates the noble eightfold path to reject the training and return to a lesser life.}}\\
\end{addmargin}
\end{absolutelynopagebreak}

\begin{absolutelynopagebreak}
\setstretch{.7}
{\PaliGlossA{taṃ kissa hetu?}}\\
\begin{addmargin}[1em]{2em}
\setstretch{.5}
{\PaliGlossB{Why is that?}}\\
\end{addmargin}
\end{absolutelynopagebreak}

\begin{absolutelynopagebreak}
\setstretch{.7}
{\PaliGlossA{yañhi taṃ, bhikkhave, cittaṃ dīgharattaṃ vivekaninnaṃ vivekapoṇaṃ vivekapabbhāraṃ taṃ vata hīnāyāvattissatīti—netaṃ ṭhānaṃ vijjati.}}\\
\begin{addmargin}[1em]{2em}
\setstretch{.5}
{\PaliGlossB{Because for a long time that mendicant’s mind has slanted, sloped, and inclined to seclusion. So it’s impossible for them to return to a lesser life.}}\\
\end{addmargin}
\end{absolutelynopagebreak}

\begin{absolutelynopagebreak}
\setstretch{.7}
{\PaliGlossA{kathañca, bhikkhave, bhikkhu ariyaṃ aṭṭhaṅgikaṃ maggaṃ bhāveti ariyaṃ aṭṭhaṅgikaṃ maggaṃ bahulīkaroti?}}\\
\begin{addmargin}[1em]{2em}
\setstretch{.5}
{\PaliGlossB{And how does a mendicant develop the noble eightfold path?}}\\
\end{addmargin}
\end{absolutelynopagebreak}

\begin{absolutelynopagebreak}
\setstretch{.7}
{\PaliGlossA{idha, bhikkhave, bhikkhu sammādiṭṭhiṃ bhāveti vivekanissitaṃ … pe … sammāsamādhiṃ bhāveti vivekanissitaṃ virāganissitaṃ nirodhanissitaṃ vossaggapariṇāmiṃ …}}\\
\begin{addmargin}[1em]{2em}
\setstretch{.5}
{\PaliGlossB{It’s when a mendicant develops right view, right thought, right speech, right action, right livelihood, right effort, right mindfulness, and right immersion, which rely on seclusion, fading away, and cessation, and ripen as letting go.}}\\
\end{addmargin}
\end{absolutelynopagebreak}

\begin{absolutelynopagebreak}
\setstretch{.7}
{\PaliGlossA{evaṃ kho, bhikkhave, bhikkhu ariyaṃ aṭṭhaṅgikaṃ maggaṃ bhāveti, ariyaṃ aṭṭhaṅgikaṃ maggaṃ bahulīkarotī”ti.}}\\
\begin{addmargin}[1em]{2em}
\setstretch{.5}
{\PaliGlossB{That’s how a mendicant develops and cultivates the noble eightfold path.”}}\\
\end{addmargin}
\end{absolutelynopagebreak}

\begin{absolutelynopagebreak}
\setstretch{.7}
{\PaliGlossA{(yadapi balakaraṇīyaṃ, tadapi vitthāretabbaṃ.)}}\\
\begin{addmargin}[1em]{2em}
\setstretch{.5}
{\PaliGlossB{    -}}\\
\end{addmargin}
\end{absolutelynopagebreak}

\begin{absolutelynopagebreak}
\setstretch{.7}
{\PaliGlossA{dvādasamaṃ.}}\\
\begin{addmargin}[1em]{2em}
\setstretch{.5}
{\PaliGlossB{    -}}\\
\end{addmargin}
\end{absolutelynopagebreak}

\begin{absolutelynopagebreak}
\setstretch{.7}
{\PaliGlossA{balakaraṇīyavaggo chaṭṭho.}}\\
\begin{addmargin}[1em]{2em}
\setstretch{.5}
{\PaliGlossB{    -}}\\
\end{addmargin}
\end{absolutelynopagebreak}

\begin{absolutelynopagebreak}
\setstretch{.7}
{\PaliGlossA{balaṃ bījañca nāgo ca,}}\\
\begin{addmargin}[1em]{2em}
\setstretch{.5}
{\PaliGlossB{    -}}\\
\end{addmargin}
\end{absolutelynopagebreak}

\begin{absolutelynopagebreak}
\setstretch{.7}
{\PaliGlossA{rukkho kumbhena sūkiyā;}}\\
\begin{addmargin}[1em]{2em}
\setstretch{.5}
{\PaliGlossB{    -}}\\
\end{addmargin}
\end{absolutelynopagebreak}

\begin{absolutelynopagebreak}
\setstretch{.7}
{\PaliGlossA{ākāsena ca dve meghā,}}\\
\begin{addmargin}[1em]{2em}
\setstretch{.5}
{\PaliGlossB{    -}}\\
\end{addmargin}
\end{absolutelynopagebreak}

\begin{absolutelynopagebreak}
\setstretch{.7}
{\PaliGlossA{nāvā āgantukā nadīti.}}\\
\begin{addmargin}[1em]{2em}
\setstretch{.5}
{\PaliGlossB{    -}}\\
\end{addmargin}
\end{absolutelynopagebreak}
