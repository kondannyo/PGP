
\begin{absolutelynopagebreak}
\setstretch{.7}
{\PaliGlossA{saṃyutta nikāya 54}}\\
\begin{addmargin}[1em]{2em}
\setstretch{.5}
{\PaliGlossB{Linked Discourses 54}}\\
\end{addmargin}
\end{absolutelynopagebreak}

\begin{absolutelynopagebreak}
\setstretch{.7}
{\PaliGlossA{2. dutiyavagga}}\\
\begin{addmargin}[1em]{2em}
\setstretch{.5}
{\PaliGlossB{2. The Second Chapter}}\\
\end{addmargin}
\end{absolutelynopagebreak}

\begin{absolutelynopagebreak}
\setstretch{.7}
{\PaliGlossA{11. icchānaṅgalasutta}}\\
\begin{addmargin}[1em]{2em}
\setstretch{.5}
{\PaliGlossB{11. Icchānaṅgala}}\\
\end{addmargin}
\end{absolutelynopagebreak}

\begin{absolutelynopagebreak}
\setstretch{.7}
{\PaliGlossA{ekaṃ samayaṃ bhagavā icchānaṅgale viharati icchānaṅgalavanasaṇḍe.}}\\
\begin{addmargin}[1em]{2em}
\setstretch{.5}
{\PaliGlossB{At one time the Buddha was staying in a forest near Icchānaṅgala.}}\\
\end{addmargin}
\end{absolutelynopagebreak}

\begin{absolutelynopagebreak}
\setstretch{.7}
{\PaliGlossA{tatra kho bhagavā bhikkhū āmantesi:}}\\
\begin{addmargin}[1em]{2em}
\setstretch{.5}
{\PaliGlossB{There he addressed the mendicants,}}\\
\end{addmargin}
\end{absolutelynopagebreak}

\begin{absolutelynopagebreak}
\setstretch{.7}
{\PaliGlossA{“icchāmahaṃ, bhikkhave, temāsaṃ paṭisallīyituṃ.}}\\
\begin{addmargin}[1em]{2em}
\setstretch{.5}
{\PaliGlossB{“Mendicants, I wish to go on retreat for three months.}}\\
\end{addmargin}
\end{absolutelynopagebreak}

\begin{absolutelynopagebreak}
\setstretch{.7}
{\PaliGlossA{nāmhi kenaci upasaṅkamitabbo, aññatra ekena piṇḍapātanīhārakenā”ti.}}\\
\begin{addmargin}[1em]{2em}
\setstretch{.5}
{\PaliGlossB{No-one should approach me, except for the one who brings my alms-food.”}}\\
\end{addmargin}
\end{absolutelynopagebreak}

\begin{absolutelynopagebreak}
\setstretch{.7}
{\PaliGlossA{“evaṃ, bhante”ti kho te bhikkhū bhagavato paṭissutvā nāssudha koci bhagavantaṃ upasaṅkamati, aññatra ekena piṇḍapātanīhārakena.}}\\
\begin{addmargin}[1em]{2em}
\setstretch{.5}
{\PaliGlossB{“Yes, sir,” replied those mendicants. And no-one approached him, except for the one who brought the alms-food.}}\\
\end{addmargin}
\end{absolutelynopagebreak}

\begin{absolutelynopagebreak}
\setstretch{.7}
{\PaliGlossA{atha kho bhagavā tassa temāsassa accayena paṭisallānā vuṭṭhito bhikkhū āmantesi:}}\\
\begin{addmargin}[1em]{2em}
\setstretch{.5}
{\PaliGlossB{Then after three months had passed, the Buddha came out of retreat and addressed the mendicants:}}\\
\end{addmargin}
\end{absolutelynopagebreak}

\begin{absolutelynopagebreak}
\setstretch{.7}
{\PaliGlossA{“sace kho, bhikkhave, aññatitthiyā paribbājakā evaṃ puccheyyuṃ:}}\\
\begin{addmargin}[1em]{2em}
\setstretch{.5}
{\PaliGlossB{“Mendicants, if wanderers who follow another path were to ask you:}}\\
\end{addmargin}
\end{absolutelynopagebreak}

\begin{absolutelynopagebreak}
\setstretch{.7}
{\PaliGlossA{‘katamenāvuso, vihārena samaṇo gotamo vassāvāsaṃ bahulaṃ vihāsī’ti, evaṃ puṭṭhā tumhe, bhikkhave, tesaṃ aññatitthiyānaṃ paribbājakānaṃ evaṃ byākareyyātha:}}\\
\begin{addmargin}[1em]{2em}
\setstretch{.5}
{\PaliGlossB{‘Reverends, what was the ascetic Gotama’s usual meditation during the rainy season residence?’ You should answer them like this.}}\\
\end{addmargin}
\end{absolutelynopagebreak}

\begin{absolutelynopagebreak}
\setstretch{.7}
{\PaliGlossA{‘ānāpānassatisamādhinā kho, āvuso, bhagavā vassāvāsaṃ bahulaṃ vihāsī’ti.}}\\
\begin{addmargin}[1em]{2em}
\setstretch{.5}
{\PaliGlossB{‘Reverends, the ascetic Gotama’s usual meditation during the rainy season residence was immersion due to mindfulness of breathing.’}}\\
\end{addmargin}
\end{absolutelynopagebreak}

\begin{absolutelynopagebreak}
\setstretch{.7}
{\PaliGlossA{idhāhaṃ, bhikkhave, sato assasāmi, sato passasāmi.}}\\
\begin{addmargin}[1em]{2em}
\setstretch{.5}
{\PaliGlossB{In this regard: mindful, I breathe in. Mindful, I breathe out.}}\\
\end{addmargin}
\end{absolutelynopagebreak}

\begin{absolutelynopagebreak}
\setstretch{.7}
{\PaliGlossA{dīghaṃ assasanto ‘dīghaṃ assasāmī’ti pajānāmi, dīghaṃ passasanto ‘dīghaṃ passasāmī’ti pajānāmi;}}\\
\begin{addmargin}[1em]{2em}
\setstretch{.5}
{\PaliGlossB{When breathing in heavily I know: ‘I’m breathing in heavily.’ When breathing out heavily I know: ‘I’m breathing out heavily.’}}\\
\end{addmargin}
\end{absolutelynopagebreak}

\begin{absolutelynopagebreak}
\setstretch{.7}
{\PaliGlossA{rassaṃ assasanto ‘rassaṃ assasāmī’ti pajānāmi, rassaṃ passasanto ‘rassaṃ passasāmī’ti pajānāmi;}}\\
\begin{addmargin}[1em]{2em}
\setstretch{.5}
{\PaliGlossB{When breathing in lightly I know: ‘I’m breathing in lightly.’ When breathing out lightly I know: ‘I’m breathing out lightly.’}}\\
\end{addmargin}
\end{absolutelynopagebreak}

\begin{absolutelynopagebreak}
\setstretch{.7}
{\PaliGlossA{‘sabbakāyappaṭisaṃvedī assasissāmī’ti pajānāmi … pe …}}\\
\begin{addmargin}[1em]{2em}
\setstretch{.5}
{\PaliGlossB{I know: ‘I’ll breathe in experiencing the whole body.’ …}}\\
\end{addmargin}
\end{absolutelynopagebreak}

\begin{absolutelynopagebreak}
\setstretch{.7}
{\PaliGlossA{‘paṭinissaggānupassī assasissāmī’ti pajānāmi, ‘paṭinissaggānupassī passasissāmī’ti pajānāmi.}}\\
\begin{addmargin}[1em]{2em}
\setstretch{.5}
{\PaliGlossB{I know: ‘I’ll breathe in observing letting go.’ I know: ‘I’ll breathe out observing letting go.’}}\\
\end{addmargin}
\end{absolutelynopagebreak}

\begin{absolutelynopagebreak}
\setstretch{.7}
{\PaliGlossA{yañhi taṃ, bhikkhave, sammā vadamāno vadeyya:}}\\
\begin{addmargin}[1em]{2em}
\setstretch{.5}
{\PaliGlossB{For if anything should be rightly called}}\\
\end{addmargin}
\end{absolutelynopagebreak}

\begin{absolutelynopagebreak}
\setstretch{.7}
{\PaliGlossA{‘ariyavihāro’ itipi, ‘brahmavihāro’ itipi, ‘tathāgatavihāro’ itipi.}}\\
\begin{addmargin}[1em]{2em}
\setstretch{.5}
{\PaliGlossB{‘the meditation of a noble one’, or else ‘the meditation of a Brahmā’, or else ‘the meditation of a realized one’,}}\\
\end{addmargin}
\end{absolutelynopagebreak}

\begin{absolutelynopagebreak}
\setstretch{.7}
{\PaliGlossA{ānāpānassatisamādhiṃ sammā vadamāno vadeyya:}}\\
\begin{addmargin}[1em]{2em}
\setstretch{.5}
{\PaliGlossB{it’s immersion due to mindfulness of breathing.}}\\
\end{addmargin}
\end{absolutelynopagebreak}

\begin{absolutelynopagebreak}
\setstretch{.7}
{\PaliGlossA{‘ariyavihāro’ itipi, ‘brahmavihāro’ itipi, ‘tathāgatavihāro’ itipi.}}\\
\begin{addmargin}[1em]{2em}
\setstretch{.5}
{\PaliGlossB{    -}}\\
\end{addmargin}
\end{absolutelynopagebreak}

\begin{absolutelynopagebreak}
\setstretch{.7}
{\PaliGlossA{ye te, bhikkhave, bhikkhū sekhā appattamānasā anuttaraṃ yogakkhemaṃ patthayamānā viharanti tesaṃ ānāpānassatisamādhi bhāvito bahulīkato āsavānaṃ khayāya saṃvattati.}}\\
\begin{addmargin}[1em]{2em}
\setstretch{.5}
{\PaliGlossB{For those mendicants who are trainees—who haven’t achieved their heart’s desire, but live aspiring to the supreme sanctuary—the development and cultivation of immersion due to mindfulness of breathing leads to the ending of defilements.}}\\
\end{addmargin}
\end{absolutelynopagebreak}

\begin{absolutelynopagebreak}
\setstretch{.7}
{\PaliGlossA{ye ca kho te, bhikkhave, bhikkhū arahanto khīṇāsavā vusitavanto katakaraṇīyā ohitabhārā anuppattasadatthā parikkhīṇabhavasaṃyojanā sammadaññāvimuttā, tesaṃ ānāpānassatisamādhi bhāvito bahulīkato diṭṭhadhammasukhavihārāya ceva saṃvattati satisampajaññāya ca.}}\\
\begin{addmargin}[1em]{2em}
\setstretch{.5}
{\PaliGlossB{For those mendicants who are perfected—who have ended the defilements, completed the spiritual journey, done what had to be done, laid down the burden, achieved their own goal, utterly ended the fetters of rebirth, and are rightly freed through enlightenment—the development and cultivation of immersion due to mindfulness of breathing leads to blissful meditation in the present life, and to mindfulness and awareness.}}\\
\end{addmargin}
\end{absolutelynopagebreak}

\begin{absolutelynopagebreak}
\setstretch{.7}
{\PaliGlossA{yañhi taṃ, bhikkhave, sammā vadamāno vadeyya:}}\\
\begin{addmargin}[1em]{2em}
\setstretch{.5}
{\PaliGlossB{For if anything should be rightly called}}\\
\end{addmargin}
\end{absolutelynopagebreak}

\begin{absolutelynopagebreak}
\setstretch{.7}
{\PaliGlossA{‘ariyavihāro’ itipi, ‘brahmavihāro’ itipi, ‘tathāgatavihāro’ itipi.}}\\
\begin{addmargin}[1em]{2em}
\setstretch{.5}
{\PaliGlossB{‘the meditation of a noble one’, or else ‘the meditation of a Brahmā’, or else ‘the meditation of a realized one’,}}\\
\end{addmargin}
\end{absolutelynopagebreak}

\begin{absolutelynopagebreak}
\setstretch{.7}
{\PaliGlossA{ānāpānassatisamādhiṃ sammā vadamāno vadeyya:}}\\
\begin{addmargin}[1em]{2em}
\setstretch{.5}
{\PaliGlossB{it’s immersion due to mindfulness of breathing.”}}\\
\end{addmargin}
\end{absolutelynopagebreak}

\begin{absolutelynopagebreak}
\setstretch{.7}
{\PaliGlossA{‘ariyavihāro’ itipi, ‘brahmavihāro’ itipi, ‘tathāgatavihāro’ itipī”ti.}}\\
\begin{addmargin}[1em]{2em}
\setstretch{.5}
{\PaliGlossB{    -}}\\
\end{addmargin}
\end{absolutelynopagebreak}

\begin{absolutelynopagebreak}
\setstretch{.7}
{\PaliGlossA{paṭhamaṃ.}}\\
\begin{addmargin}[1em]{2em}
\setstretch{.5}
{\PaliGlossB{    -}}\\
\end{addmargin}
\end{absolutelynopagebreak}
