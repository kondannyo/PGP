
\begin{absolutelynopagebreak}
\setstretch{.7}
{\PaliGlossA{saṃyutta nikāya 35}}\\
\begin{addmargin}[1em]{2em}
\setstretch{.5}
{\PaliGlossB{Linked Discourses 35}}\\
\end{addmargin}
\end{absolutelynopagebreak}

\begin{absolutelynopagebreak}
\setstretch{.7}
{\PaliGlossA{18. samuddavagga}}\\
\begin{addmargin}[1em]{2em}
\setstretch{.5}
{\PaliGlossB{18. The Ocean}}\\
\end{addmargin}
\end{absolutelynopagebreak}

\begin{absolutelynopagebreak}
\setstretch{.7}
{\PaliGlossA{234. udāyīsutta}}\\
\begin{addmargin}[1em]{2em}
\setstretch{.5}
{\PaliGlossB{234. With Udāyī}}\\
\end{addmargin}
\end{absolutelynopagebreak}

\begin{absolutelynopagebreak}
\setstretch{.7}
{\PaliGlossA{ekaṃ samayaṃ āyasmā ca ānando āyasmā ca udāyī kosambiyaṃ viharanti ghositārāme.}}\\
\begin{addmargin}[1em]{2em}
\setstretch{.5}
{\PaliGlossB{At one time the venerables Ānanda and Udāyī were staying near Kosambi, in Ghosita’s Monastery.}}\\
\end{addmargin}
\end{absolutelynopagebreak}

\begin{absolutelynopagebreak}
\setstretch{.7}
{\PaliGlossA{atha kho āyasmā udāyī sāyanhasamayaṃ paṭisallānā vuṭṭhito yenāyasmā ānando tenupasaṅkami; upasaṅkamitvā āyasmatā ānandena saddhiṃ sammodi.}}\\
\begin{addmargin}[1em]{2em}
\setstretch{.5}
{\PaliGlossB{Then in the late afternoon, Venerable Udāyī came out of retreat, went to Venerable Ānanda, and exchanged greetings with him.}}\\
\end{addmargin}
\end{absolutelynopagebreak}

\begin{absolutelynopagebreak}
\setstretch{.7}
{\PaliGlossA{sammodanīyaṃ kathaṃ sāraṇīyaṃ vītisāretvā ekamantaṃ nisīdi. ekamantaṃ nisinno kho āyasmā udāyī āyasmantaṃ ānandaṃ etadavoca:}}\\
\begin{addmargin}[1em]{2em}
\setstretch{.5}
{\PaliGlossB{When the greetings and polite conversation were over, he sat down to one side and said to Ānanda:}}\\
\end{addmargin}
\end{absolutelynopagebreak}

\begin{absolutelynopagebreak}
\setstretch{.7}
{\PaliGlossA{“yatheva nu kho, āvuso ānanda, ayaṃ kāyo bhagavatā anekapariyāyena akkhāto vivaṭo pakāsito:}}\\
\begin{addmargin}[1em]{2em}
\setstretch{.5}
{\PaliGlossB{“Reverend Ānanda, the Buddha has explained, opened, and illuminated in many ways how this body is not-self.}}\\
\end{addmargin}
\end{absolutelynopagebreak}

\begin{absolutelynopagebreak}
\setstretch{.7}
{\PaliGlossA{‘itipāyaṃ kāyo anattā’ti, sakkā evameva viññāṇaṃ pidaṃ ācikkhituṃ desetuṃ paññapetuṃ paṭṭhapetuṃ vivarituṃ vibhajituṃ uttānīkātuṃ: ‘itipidaṃ viññāṇaṃ anattā’”ti?}}\\
\begin{addmargin}[1em]{2em}
\setstretch{.5}
{\PaliGlossB{Is it possible to explain consciousness in the same way? To teach, assert, establish, open, analyze, and make it clear how consciousness is not-self?”}}\\
\end{addmargin}
\end{absolutelynopagebreak}

\begin{absolutelynopagebreak}
\setstretch{.7}
{\PaliGlossA{“yatheva kho, āvuso udāyī, ayaṃ kāyo bhagavatā anekapariyāyena akkhāto vivaṭo pakāsito:}}\\
\begin{addmargin}[1em]{2em}
\setstretch{.5}
{\PaliGlossB{    -}}\\
\end{addmargin}
\end{absolutelynopagebreak}

\begin{absolutelynopagebreak}
\setstretch{.7}
{\PaliGlossA{‘itipāyaṃ kāyo anattā’ti, sakkā evameva viññāṇaṃ pidaṃ ācikkhituṃ desetuṃ paññapetuṃ paṭṭhapetuṃ vivarituṃ vibhajituṃ uttānīkātuṃ: ‘itipidaṃ viññāṇaṃ anattā’”ti.}}\\
\begin{addmargin}[1em]{2em}
\setstretch{.5}
{\PaliGlossB{“It is possible, Reverend Udāyī.}}\\
\end{addmargin}
\end{absolutelynopagebreak}

\begin{absolutelynopagebreak}
\setstretch{.7}
{\PaliGlossA{“cakkhuñca, āvuso, paṭicca rūpe ca uppajjati cakkhuviññāṇan”ti?}}\\
\begin{addmargin}[1em]{2em}
\setstretch{.5}
{\PaliGlossB{Does eye consciousness arise dependent on the eye and sights?”}}\\
\end{addmargin}
\end{absolutelynopagebreak}

\begin{absolutelynopagebreak}
\setstretch{.7}
{\PaliGlossA{“evamāvuso”ti.}}\\
\begin{addmargin}[1em]{2em}
\setstretch{.5}
{\PaliGlossB{“Yes, reverend.”}}\\
\end{addmargin}
\end{absolutelynopagebreak}

\begin{absolutelynopagebreak}
\setstretch{.7}
{\PaliGlossA{“yo cāvuso, hetu, yo ca paccayo cakkhuviññāṇassa uppādāya, so ca hetu, so ca paccayo sabbena sabbaṃ sabbathā sabbaṃ aparisesaṃ nirujjheyya. api nu kho cakkhuviññāṇaṃ paññāyethā”ti?}}\\
\begin{addmargin}[1em]{2em}
\setstretch{.5}
{\PaliGlossB{“If the cause and condition that gives rise to eye consciousness were to totally and utterly cease without anything left over, would eye consciousness still be found?”}}\\
\end{addmargin}
\end{absolutelynopagebreak}

\begin{absolutelynopagebreak}
\setstretch{.7}
{\PaliGlossA{“no hetaṃ, āvuso”.}}\\
\begin{addmargin}[1em]{2em}
\setstretch{.5}
{\PaliGlossB{“No, reverend.”}}\\
\end{addmargin}
\end{absolutelynopagebreak}

\begin{absolutelynopagebreak}
\setstretch{.7}
{\PaliGlossA{“imināpi kho etaṃ, āvuso, pariyāyena bhagavatā akkhātaṃ vivaṭaṃ pakāsitaṃ: ‘itipidaṃ viññāṇaṃ anattā’”ti … pe ….}}\\
\begin{addmargin}[1em]{2em}
\setstretch{.5}
{\PaliGlossB{“In this way, too, it can be understood how consciousness is not-self.}}\\
\end{addmargin}
\end{absolutelynopagebreak}

\begin{absolutelynopagebreak}
\setstretch{.7}
{\PaliGlossA{“jivhañcāvuso, paṭicca rase ca uppajjati jivhāviññāṇan”ti?}}\\
\begin{addmargin}[1em]{2em}
\setstretch{.5}
{\PaliGlossB{Does ear … nose … tongue … body …}}\\
\end{addmargin}
\end{absolutelynopagebreak}

\begin{absolutelynopagebreak}
\setstretch{.7}
{\PaliGlossA{“evamāvuso”ti.}}\\
\begin{addmargin}[1em]{2em}
\setstretch{.5}
{\PaliGlossB{    -}}\\
\end{addmargin}
\end{absolutelynopagebreak}

\begin{absolutelynopagebreak}
\setstretch{.7}
{\PaliGlossA{“yo cāvuso, hetu yo ca paccayo jivhāviññāṇassa uppādāya, so ca hetu, so ca paccayo sabbena sabbaṃ sabbathā sabbaṃ aparisesaṃ nirujjheyya, api nu kho jivhāviññāṇaṃ paññāyethā”ti?}}\\
\begin{addmargin}[1em]{2em}
\setstretch{.5}
{\PaliGlossB{    -}}\\
\end{addmargin}
\end{absolutelynopagebreak}

\begin{absolutelynopagebreak}
\setstretch{.7}
{\PaliGlossA{“no hetaṃ, āvuso”.}}\\
\begin{addmargin}[1em]{2em}
\setstretch{.5}
{\PaliGlossB{    -}}\\
\end{addmargin}
\end{absolutelynopagebreak}

\begin{absolutelynopagebreak}
\setstretch{.7}
{\PaliGlossA{“imināpi kho etaṃ, āvuso, pariyāyena bhagavatā akkhātaṃ vivaṭaṃ pakāsitaṃ: ‘itipidaṃ viññāṇaṃ anattā’”ti … pe ….}}\\
\begin{addmargin}[1em]{2em}
\setstretch{.5}
{\PaliGlossB{    -}}\\
\end{addmargin}
\end{absolutelynopagebreak}

\begin{absolutelynopagebreak}
\setstretch{.7}
{\PaliGlossA{“manañcāvuso, paṭicca dhamme ca uppajjati manoviññāṇan”ti?}}\\
\begin{addmargin}[1em]{2em}
\setstretch{.5}
{\PaliGlossB{mind consciousness arise dependent on the mind and thoughts?”}}\\
\end{addmargin}
\end{absolutelynopagebreak}

\begin{absolutelynopagebreak}
\setstretch{.7}
{\PaliGlossA{“evamāvuso”ti.}}\\
\begin{addmargin}[1em]{2em}
\setstretch{.5}
{\PaliGlossB{“Yes, reverend.”}}\\
\end{addmargin}
\end{absolutelynopagebreak}

\begin{absolutelynopagebreak}
\setstretch{.7}
{\PaliGlossA{“yo cāvuso, hetu, yo ca paccayo manoviññāṇassa uppādāya, so ca hetu, so ca paccayo sabbena sabbaṃ sabbathā sabbaṃ aparisesaṃ nirujjheyya, api nu kho manoviññāṇaṃ paññāyethā”ti?}}\\
\begin{addmargin}[1em]{2em}
\setstretch{.5}
{\PaliGlossB{“If the cause and condition that gives rise to mind consciousness were to totally and utterly cease without anything left over, would mind consciousness still be found?”}}\\
\end{addmargin}
\end{absolutelynopagebreak}

\begin{absolutelynopagebreak}
\setstretch{.7}
{\PaliGlossA{“no hetaṃ, āvuso”.}}\\
\begin{addmargin}[1em]{2em}
\setstretch{.5}
{\PaliGlossB{“No, reverend.”}}\\
\end{addmargin}
\end{absolutelynopagebreak}

\begin{absolutelynopagebreak}
\setstretch{.7}
{\PaliGlossA{“imināpi kho etaṃ, āvuso, pariyāyena bhagavatā akkhātaṃ vivaṭaṃ pakāsitaṃ: ‘itipidaṃ viññāṇaṃ anattā’ti.}}\\
\begin{addmargin}[1em]{2em}
\setstretch{.5}
{\PaliGlossB{“In this way, too, it can be understood how consciousness is not-self.}}\\
\end{addmargin}
\end{absolutelynopagebreak}

\begin{absolutelynopagebreak}
\setstretch{.7}
{\PaliGlossA{seyyathāpi, āvuso, puriso sāratthiko sāragavesī sārapariyesanaṃ caramāno tiṇhaṃ kuṭhāriṃ ādāya vanaṃ paviseyya. so tattha passeyya mahantaṃ kadalikkhandhaṃ ujuṃ navaṃ akukkukajātaṃ. tamenaṃ mūle chindeyya; mūle chetvā agge chindeyya; agge chetvā pattavaṭṭiṃ vinibbhujeyya. so tattha pheggumpi nādhigaccheyya, kuto sāraṃ.}}\\
\begin{addmargin}[1em]{2em}
\setstretch{.5}
{\PaliGlossB{Suppose there was a person in need of heartwood. Wandering in search of heartwood, they’d take a sharp axe and enter a forest. There they’d see a big banana tree, straight and young and grown free of defects. They’d cut it down at the base, cut off the root, cut off the top, and unroll the coiled sheaths. But they wouldn’t even find sapwood, much less heartwood.}}\\
\end{addmargin}
\end{absolutelynopagebreak}

\begin{absolutelynopagebreak}
\setstretch{.7}
{\PaliGlossA{evameva kho, āvuso, bhikkhu chasu phassāyatanesu nevattānaṃ na attaniyaṃ samanupassati.}}\\
\begin{addmargin}[1em]{2em}
\setstretch{.5}
{\PaliGlossB{In the same way, a mendicant sees these six fields of contact as neither self nor belonging to self.}}\\
\end{addmargin}
\end{absolutelynopagebreak}

\begin{absolutelynopagebreak}
\setstretch{.7}
{\PaliGlossA{so evaṃ asamanupassanto na kiñci loke upādiyati.}}\\
\begin{addmargin}[1em]{2em}
\setstretch{.5}
{\PaliGlossB{So seeing, they don’t grasp anything in the world.}}\\
\end{addmargin}
\end{absolutelynopagebreak}

\begin{absolutelynopagebreak}
\setstretch{.7}
{\PaliGlossA{anupādiyaṃ na paritassati. aparitassaṃ paccattaññeva parinibbāyati.}}\\
\begin{addmargin}[1em]{2em}
\setstretch{.5}
{\PaliGlossB{Not grasping, they’re not anxious. Not being anxious, they personally become extinguished.}}\\
\end{addmargin}
\end{absolutelynopagebreak}

\begin{absolutelynopagebreak}
\setstretch{.7}
{\PaliGlossA{‘khīṇā jāti, vusitaṃ brahmacariyaṃ, kataṃ karaṇīyaṃ, nāparaṃ itthattāyā’ti pajānātī”ti.}}\\
\begin{addmargin}[1em]{2em}
\setstretch{.5}
{\PaliGlossB{They understand: ‘Rebirth is ended, the spiritual journey has been completed, what had to be done has been done, there is no return to any state of existence.’”}}\\
\end{addmargin}
\end{absolutelynopagebreak}

\begin{absolutelynopagebreak}
\setstretch{.7}
{\PaliGlossA{sattamaṃ.}}\\
\begin{addmargin}[1em]{2em}
\setstretch{.5}
{\PaliGlossB{    -}}\\
\end{addmargin}
\end{absolutelynopagebreak}
