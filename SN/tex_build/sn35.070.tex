
\begin{absolutelynopagebreak}
\setstretch{.7}
{\PaliGlossA{saṃyutta nikāya 35}}\\
\begin{addmargin}[1em]{2em}
\setstretch{.5}
{\PaliGlossB{Linked Discourses 35}}\\
\end{addmargin}
\end{absolutelynopagebreak}

\begin{absolutelynopagebreak}
\setstretch{.7}
{\PaliGlossA{7. migajālavagga}}\\
\begin{addmargin}[1em]{2em}
\setstretch{.5}
{\PaliGlossB{7. With Migajāla}}\\
\end{addmargin}
\end{absolutelynopagebreak}

\begin{absolutelynopagebreak}
\setstretch{.7}
{\PaliGlossA{70. upavāṇasandiṭṭhikasutta}}\\
\begin{addmargin}[1em]{2em}
\setstretch{.5}
{\PaliGlossB{70. Upavāṇa on What is Visible in This Very Life}}\\
\end{addmargin}
\end{absolutelynopagebreak}

\begin{absolutelynopagebreak}
\setstretch{.7}
{\PaliGlossA{atha kho āyasmā upavāṇo yena bhagavā tenupasaṅkami … pe …}}\\
\begin{addmargin}[1em]{2em}
\setstretch{.5}
{\PaliGlossB{Then Venerable Upavāṇa went up to the Buddha …}}\\
\end{addmargin}
\end{absolutelynopagebreak}

\begin{absolutelynopagebreak}
\setstretch{.7}
{\PaliGlossA{ekamantaṃ nisinno kho āyasmā upavāṇo bhagavantaṃ etadavoca:}}\\
\begin{addmargin}[1em]{2em}
\setstretch{.5}
{\PaliGlossB{and said to him:}}\\
\end{addmargin}
\end{absolutelynopagebreak}

\begin{absolutelynopagebreak}
\setstretch{.7}
{\PaliGlossA{“‘sandiṭṭhiko dhammo, sandiṭṭhiko dhammo’ti, bhante, vuccati.}}\\
\begin{addmargin}[1em]{2em}
\setstretch{.5}
{\PaliGlossB{“Sir, they speak of ‘a teaching visible in this very life’.}}\\
\end{addmargin}
\end{absolutelynopagebreak}

\begin{absolutelynopagebreak}
\setstretch{.7}
{\PaliGlossA{kittāvatā nu kho, bhante, sandiṭṭhiko dhammo hoti, akāliko ehipassiko opaneyyiko paccattaṃ veditabbo viññūhī”ti?}}\\
\begin{addmargin}[1em]{2em}
\setstretch{.5}
{\PaliGlossB{In what way is the teaching visible in this very life, immediately effective, inviting inspection, relevant, so that sensible people can know it for themselves?”}}\\
\end{addmargin}
\end{absolutelynopagebreak}

\begin{absolutelynopagebreak}
\setstretch{.7}
{\PaliGlossA{“idha pana, upavāṇa, bhikkhu cakkhunā rūpaṃ disvā rūpappaṭisaṃvedī ca hoti rūparāgappaṭisaṃvedī ca.}}\\
\begin{addmargin}[1em]{2em}
\setstretch{.5}
{\PaliGlossB{“Upavāṇa, take a mendicant who sees a sight with their eyes. They experience both the sight and the desire for the sight.}}\\
\end{addmargin}
\end{absolutelynopagebreak}

\begin{absolutelynopagebreak}
\setstretch{.7}
{\PaliGlossA{santañca ajjhattaṃ rūpesu rāgaṃ ‘atthi me ajjhattaṃ rūpesu rāgo’ti pajānāti.}}\\
\begin{addmargin}[1em]{2em}
\setstretch{.5}
{\PaliGlossB{There is desire for sights in them, and they understand that.}}\\
\end{addmargin}
\end{absolutelynopagebreak}

\begin{absolutelynopagebreak}
\setstretch{.7}
{\PaliGlossA{yaṃ taṃ, upavāṇa, bhikkhu cakkhunā rūpaṃ disvā rūpappaṭisaṃvedī ca hoti rūparāgappaṭisaṃvedī ca.}}\\
\begin{addmargin}[1em]{2em}
\setstretch{.5}
{\PaliGlossB{Since this is so,}}\\
\end{addmargin}
\end{absolutelynopagebreak}

\begin{absolutelynopagebreak}
\setstretch{.7}
{\PaliGlossA{santañca ajjhattaṃ rūpesu rāgaṃ ‘atthi me ajjhattaṃ rūpesu rāgo’ti pajānāti.}}\\
\begin{addmargin}[1em]{2em}
\setstretch{.5}
{\PaliGlossB{    -}}\\
\end{addmargin}
\end{absolutelynopagebreak}

\begin{absolutelynopagebreak}
\setstretch{.7}
{\PaliGlossA{evampi kho, upavāṇa, sandiṭṭhiko dhammo hoti akāliko ehipassiko opaneyyiko paccattaṃ veditabbo viññūhi … pe ….}}\\
\begin{addmargin}[1em]{2em}
\setstretch{.5}
{\PaliGlossB{this is how the teaching is visible in this very life, immediately effective, inviting inspection, relevant, so that sensible people can know it for themselves.}}\\
\end{addmargin}
\end{absolutelynopagebreak}

\begin{absolutelynopagebreak}
\setstretch{.7}
{\PaliGlossA{puna caparaṃ, upavāṇa, bhikkhu jivhāya rasaṃ sāyitvā rasappaṭisaṃvedī ca hoti rasarāgappaṭisaṃvedī ca.}}\\
\begin{addmargin}[1em]{2em}
\setstretch{.5}
{\PaliGlossB{Next, take a mendicant who hears … smells … tastes … touches …}}\\
\end{addmargin}
\end{absolutelynopagebreak}

\begin{absolutelynopagebreak}
\setstretch{.7}
{\PaliGlossA{santañca ajjhattaṃ rasesu rāgaṃ ‘atthi me ajjhattaṃ rasesu rāgo’ti pajānāti.}}\\
\begin{addmargin}[1em]{2em}
\setstretch{.5}
{\PaliGlossB{    -}}\\
\end{addmargin}
\end{absolutelynopagebreak}

\begin{absolutelynopagebreak}
\setstretch{.7}
{\PaliGlossA{yaṃ taṃ, upavāṇa, bhikkhu jivhāya rasaṃ sāyitvā rasappaṭisaṃvedī ca hoti rasarāgappaṭisaṃvedī ca.}}\\
\begin{addmargin}[1em]{2em}
\setstretch{.5}
{\PaliGlossB{    -}}\\
\end{addmargin}
\end{absolutelynopagebreak}

\begin{absolutelynopagebreak}
\setstretch{.7}
{\PaliGlossA{santañca ajjhattaṃ rasesu rāgaṃ ‘atthi me ajjhattaṃ rasesu rāgo’ti pajānāti. evampi kho, upavāṇa, sandiṭṭhiko dhammo hoti akāliko ehipassiko opaneyyiko paccattaṃ veditabbo viññūhi … pe ….}}\\
\begin{addmargin}[1em]{2em}
\setstretch{.5}
{\PaliGlossB{    -}}\\
\end{addmargin}
\end{absolutelynopagebreak}

\begin{absolutelynopagebreak}
\setstretch{.7}
{\PaliGlossA{puna caparaṃ, upavāṇa, bhikkhu manasā dhammaṃ viññāya dhammappaṭisaṃvedī ca hoti dhammarāgappaṭisaṃvedī ca.}}\\
\begin{addmargin}[1em]{2em}
\setstretch{.5}
{\PaliGlossB{Next, take a mendicant who knows a thought with their mind. They experience both the thought and the desire for the thought.}}\\
\end{addmargin}
\end{absolutelynopagebreak}

\begin{absolutelynopagebreak}
\setstretch{.7}
{\PaliGlossA{santañca ajjhattaṃ dhammesu rāgaṃ ‘atthi me ajjhattaṃ dhammesu rāgo’ti pajānāti.}}\\
\begin{addmargin}[1em]{2em}
\setstretch{.5}
{\PaliGlossB{There is desire for thoughts in them, and they understand that.}}\\
\end{addmargin}
\end{absolutelynopagebreak}

\begin{absolutelynopagebreak}
\setstretch{.7}
{\PaliGlossA{yaṃ taṃ, upavāṇa, bhikkhu manasā dhammaṃ viññāya dhammappaṭisaṃvedī ca hoti dhammarāgappaṭisaṃvedī ca.}}\\
\begin{addmargin}[1em]{2em}
\setstretch{.5}
{\PaliGlossB{Since this is so,}}\\
\end{addmargin}
\end{absolutelynopagebreak}

\begin{absolutelynopagebreak}
\setstretch{.7}
{\PaliGlossA{santañca ajjhattaṃ dhammesu rāgaṃ ‘atthi me ajjhattaṃ dhammesu rāgo’ti pajānāti.}}\\
\begin{addmargin}[1em]{2em}
\setstretch{.5}
{\PaliGlossB{    -}}\\
\end{addmargin}
\end{absolutelynopagebreak}

\begin{absolutelynopagebreak}
\setstretch{.7}
{\PaliGlossA{evampi kho, upavāṇa, sandiṭṭhiko dhammo hoti … pe … paccattaṃ veditabbo viññūhi … pe ….}}\\
\begin{addmargin}[1em]{2em}
\setstretch{.5}
{\PaliGlossB{this is how the teaching is visible in this very life, immediately effective, inviting inspection, relevant, so that sensible people can know it for themselves.}}\\
\end{addmargin}
\end{absolutelynopagebreak}

\begin{absolutelynopagebreak}
\setstretch{.7}
{\PaliGlossA{idha pana, upavāṇa, bhikkhu cakkhunā rūpaṃ disvā rūpappaṭisaṃvedī ca hoti, no ca rūparāgappaṭisaṃvedī.}}\\
\begin{addmargin}[1em]{2em}
\setstretch{.5}
{\PaliGlossB{Take a mendicant who sees a sight with their eyes. They experience the sight but no desire for the sight.}}\\
\end{addmargin}
\end{absolutelynopagebreak}

\begin{absolutelynopagebreak}
\setstretch{.7}
{\PaliGlossA{asantañca ajjhattaṃ rūpesu rāgaṃ ‘natthi me ajjhattaṃ rūpesu rāgo’ti pajānāti.}}\\
\begin{addmargin}[1em]{2em}
\setstretch{.5}
{\PaliGlossB{There is no desire for sights in them, and they understand that.}}\\
\end{addmargin}
\end{absolutelynopagebreak}

\begin{absolutelynopagebreak}
\setstretch{.7}
{\PaliGlossA{yaṃ taṃ, upavāṇa, bhikkhu cakkhunā rūpaṃ disvā rūpappaṭisaṃvedīhi kho hoti, no ca rūparāgappaṭisaṃvedī.}}\\
\begin{addmargin}[1em]{2em}
\setstretch{.5}
{\PaliGlossB{Since this is so,}}\\
\end{addmargin}
\end{absolutelynopagebreak}

\begin{absolutelynopagebreak}
\setstretch{.7}
{\PaliGlossA{asantañca ajjhattaṃ rūpesu rāgaṃ ‘natthi me ajjhattaṃ rūpesu rāgo’ti pajānāti.}}\\
\begin{addmargin}[1em]{2em}
\setstretch{.5}
{\PaliGlossB{    -}}\\
\end{addmargin}
\end{absolutelynopagebreak}

\begin{absolutelynopagebreak}
\setstretch{.7}
{\PaliGlossA{evampi kho, upavāṇa, sandiṭṭhiko dhammo hoti, akāliko ehipassiko opaneyyiko paccattaṃ veditabbo viññūhi … pe ….}}\\
\begin{addmargin}[1em]{2em}
\setstretch{.5}
{\PaliGlossB{this is how the teaching is visible in this very life, immediately effective, inviting inspection, relevant, so that sensible people can know it for themselves.}}\\
\end{addmargin}
\end{absolutelynopagebreak}

\begin{absolutelynopagebreak}
\setstretch{.7}
{\PaliGlossA{puna caparaṃ, upavāṇa, bhikkhu jivhāya rasaṃ sāyitvā rasappaṭisaṃvedīhi kho hoti, no ca rasarāgappaṭisaṃvedī. asantañca ajjhattaṃ rasesu rāgaṃ ‘natthi me ajjhattaṃ rasesu rāgo’ti pajānāti … pe ….}}\\
\begin{addmargin}[1em]{2em}
\setstretch{.5}
{\PaliGlossB{Next, take a mendicant who hears … smells … tastes … touches …}}\\
\end{addmargin}
\end{absolutelynopagebreak}

\begin{absolutelynopagebreak}
\setstretch{.7}
{\PaliGlossA{puna caparaṃ, upavāṇa, bhikkhu manasā dhammaṃ viññāya dhammappaṭisaṃvedīhi kho hoti, no ca dhammarāgappaṭisaṃvedī.}}\\
\begin{addmargin}[1em]{2em}
\setstretch{.5}
{\PaliGlossB{    -}}\\
\end{addmargin}
\end{absolutelynopagebreak}

\begin{absolutelynopagebreak}
\setstretch{.7}
{\PaliGlossA{asantañca ajjhattaṃ dhammesu rāgaṃ ‘natthi me ajjhattaṃ dhammesu rāgo’ti pajānāti.}}\\
\begin{addmargin}[1em]{2em}
\setstretch{.5}
{\PaliGlossB{    -}}\\
\end{addmargin}
\end{absolutelynopagebreak}

\begin{absolutelynopagebreak}
\setstretch{.7}
{\PaliGlossA{yaṃ taṃ, upavāṇa, bhikkhu manasā dhammaṃ viññāya dhammappaṭisaṃvedīhi kho hoti, no ca dhammarāgappaṭisaṃvedī.}}\\
\begin{addmargin}[1em]{2em}
\setstretch{.5}
{\PaliGlossB{Next, take a mendicant who knows a thought with their mind. They experience the thought but no desire for the thought.}}\\
\end{addmargin}
\end{absolutelynopagebreak}

\begin{absolutelynopagebreak}
\setstretch{.7}
{\PaliGlossA{asantañca ajjhattaṃ dhammesu rāgaṃ ‘natthi me ajjhattaṃ dhammesu rāgo’ti pajānāti.}}\\
\begin{addmargin}[1em]{2em}
\setstretch{.5}
{\PaliGlossB{There is no desire for thoughts in them, and they understand that.}}\\
\end{addmargin}
\end{absolutelynopagebreak}

\begin{absolutelynopagebreak}
\setstretch{.7}
{\PaliGlossA{evampi kho, upavāṇa, sandiṭṭhiko dhammo hoti, akāliko ehipassiko opaneyyiko paccattaṃ veditabbo viññūhī”ti.}}\\
\begin{addmargin}[1em]{2em}
\setstretch{.5}
{\PaliGlossB{Since this is so, this is how the teaching is visible in this very life, immediately effective, inviting inspection, relevant, so that sensible people can know it for themselves.”}}\\
\end{addmargin}
\end{absolutelynopagebreak}

\begin{absolutelynopagebreak}
\setstretch{.7}
{\PaliGlossA{aṭṭhamaṃ.}}\\
\begin{addmargin}[1em]{2em}
\setstretch{.5}
{\PaliGlossB{    -}}\\
\end{addmargin}
\end{absolutelynopagebreak}
