
\begin{absolutelynopagebreak}
\setstretch{.7}
{\PaliGlossA{saṃyutta nikāya 46}}\\
\begin{addmargin}[1em]{2em}
\setstretch{.5}
{\PaliGlossB{Linked Discourses 46}}\\
\end{addmargin}
\end{absolutelynopagebreak}

\begin{absolutelynopagebreak}
\setstretch{.7}
{\PaliGlossA{6. sākacchavagga}}\\
\begin{addmargin}[1em]{2em}
\setstretch{.5}
{\PaliGlossB{6. Discussion}}\\
\end{addmargin}
\end{absolutelynopagebreak}

\begin{absolutelynopagebreak}
\setstretch{.7}
{\PaliGlossA{52. pariyāyasutta}}\\
\begin{addmargin}[1em]{2em}
\setstretch{.5}
{\PaliGlossB{52. Is There a Way?}}\\
\end{addmargin}
\end{absolutelynopagebreak}

\begin{absolutelynopagebreak}
\setstretch{.7}
{\PaliGlossA{atha kho sambahulā bhikkhū pubbaṇhasamayaṃ nivāsetvā pattacīvaramādāya sāvatthiṃ piṇḍāya pavisiṃsu.}}\\
\begin{addmargin}[1em]{2em}
\setstretch{.5}
{\PaliGlossB{Then several mendicants robed up in the morning and, taking their bowls and robes, entered Sāvatthī for alms.}}\\
\end{addmargin}
\end{absolutelynopagebreak}

\begin{absolutelynopagebreak}
\setstretch{.7}
{\PaliGlossA{atha kho tesaṃ bhikkhūnaṃ etadahosi:}}\\
\begin{addmargin}[1em]{2em}
\setstretch{.5}
{\PaliGlossB{Then it occurred to him,}}\\
\end{addmargin}
\end{absolutelynopagebreak}

\begin{absolutelynopagebreak}
\setstretch{.7}
{\PaliGlossA{“atippago kho tāva sāvatthiyaṃ piṇḍāya carituṃ.}}\\
\begin{addmargin}[1em]{2em}
\setstretch{.5}
{\PaliGlossB{“It’s too early to wander for alms in Sāvatthī.}}\\
\end{addmargin}
\end{absolutelynopagebreak}

\begin{absolutelynopagebreak}
\setstretch{.7}
{\PaliGlossA{yannūna mayaṃ yena aññatitthiyānaṃ paribbājakānaṃ ārāmo tenupasaṅkameyyāmā”ti.}}\\
\begin{addmargin}[1em]{2em}
\setstretch{.5}
{\PaliGlossB{Why don’t we go to the monastery of the wanderers who follow other paths?”}}\\
\end{addmargin}
\end{absolutelynopagebreak}

\begin{absolutelynopagebreak}
\setstretch{.7}
{\PaliGlossA{atha kho te bhikkhū yena aññatitthiyānaṃ paribbājakānaṃ ārāmo tenupasaṅkamiṃsu; upasaṅkamitvā tehi aññatitthiyehi paribbājakehi saddhiṃ sammodiṃsu.}}\\
\begin{addmargin}[1em]{2em}
\setstretch{.5}
{\PaliGlossB{Then they went to the monastery of the wanderers who follow other paths, and exchanged greetings with the wanderers there.}}\\
\end{addmargin}
\end{absolutelynopagebreak}

\begin{absolutelynopagebreak}
\setstretch{.7}
{\PaliGlossA{sammodanīyaṃ kathaṃ sāraṇīyaṃ vītisāretvā ekamantaṃ nisīdiṃsu. ekamantaṃ nisinne kho te bhikkhū aññatitthiyā paribbājakā etadavocuṃ:}}\\
\begin{addmargin}[1em]{2em}
\setstretch{.5}
{\PaliGlossB{When the greetings and polite conversation were over, they sat down to one side. The wanderers said to them:}}\\
\end{addmargin}
\end{absolutelynopagebreak}

\begin{absolutelynopagebreak}
\setstretch{.7}
{\PaliGlossA{“samaṇo, āvuso, gotamo sāvakānaṃ evaṃ dhammaṃ deseti:}}\\
\begin{addmargin}[1em]{2em}
\setstretch{.5}
{\PaliGlossB{“Reverends, the ascetic Gotama teaches his disciples like this:}}\\
\end{addmargin}
\end{absolutelynopagebreak}

\begin{absolutelynopagebreak}
\setstretch{.7}
{\PaliGlossA{‘etha tumhe, bhikkhave, pañca nīvaraṇe pahāya cetaso upakkilese paññāya dubbalīkaraṇe satta bojjhaṅge yathābhūtaṃ bhāvethā’ti.}}\\
\begin{addmargin}[1em]{2em}
\setstretch{.5}
{\PaliGlossB{‘Mendicants, please give up the five hindrances—corruptions of the heart that weaken wisdom—and truly develop the seven awakening factors.’}}\\
\end{addmargin}
\end{absolutelynopagebreak}

\begin{absolutelynopagebreak}
\setstretch{.7}
{\PaliGlossA{mayampi kho, āvuso, sāvakānaṃ evaṃ dhammaṃ desema:}}\\
\begin{addmargin}[1em]{2em}
\setstretch{.5}
{\PaliGlossB{We too teach our disciples:}}\\
\end{addmargin}
\end{absolutelynopagebreak}

\begin{absolutelynopagebreak}
\setstretch{.7}
{\PaliGlossA{‘etha tumhe, āvuso, pañca nīvaraṇe pahāya cetaso upakkilese paññāya dubbalīkaraṇe satta bojjhaṅge yathābhūtaṃ bhāvethā’ti.}}\\
\begin{addmargin}[1em]{2em}
\setstretch{.5}
{\PaliGlossB{‘Reverends, please give up the five hindrances—corruptions of the heart that weaken wisdom—and truly develop the seven awakening factors.’}}\\
\end{addmargin}
\end{absolutelynopagebreak}

\begin{absolutelynopagebreak}
\setstretch{.7}
{\PaliGlossA{idha no, āvuso, ko viseso, ko adhippayāso, kiṃ nānākaraṇaṃ samaṇassa vā gotamassa amhākaṃ vā, yadidaṃ—dhammadesanāya vā dhammadesanaṃ, anusāsaniyā vā anusāsanin”ti?}}\\
\begin{addmargin}[1em]{2em}
\setstretch{.5}
{\PaliGlossB{What, then, is the difference between the ascetic Gotama’s teaching and instruction and ours?”}}\\
\end{addmargin}
\end{absolutelynopagebreak}

\begin{absolutelynopagebreak}
\setstretch{.7}
{\PaliGlossA{atha kho te bhikkhū tesaṃ aññatitthiyānaṃ paribbājakānaṃ bhāsitaṃ neva abhinandiṃsu nappaṭikkosiṃsu;}}\\
\begin{addmargin}[1em]{2em}
\setstretch{.5}
{\PaliGlossB{Those mendicants neither approved nor dismissed that statement of the wanderers who follow other paths.}}\\
\end{addmargin}
\end{absolutelynopagebreak}

\begin{absolutelynopagebreak}
\setstretch{.7}
{\PaliGlossA{anabhinanditvā appaṭikkositvā uṭṭhāyāsanā pakkamiṃsu:}}\\
\begin{addmargin}[1em]{2em}
\setstretch{.5}
{\PaliGlossB{They got up from their seat, thinking:}}\\
\end{addmargin}
\end{absolutelynopagebreak}

\begin{absolutelynopagebreak}
\setstretch{.7}
{\PaliGlossA{“bhagavato santike etassa bhāsitassa atthaṃ ājānissāmā”ti.}}\\
\begin{addmargin}[1em]{2em}
\setstretch{.5}
{\PaliGlossB{“We will learn the meaning of this statement from the Buddha himself.”}}\\
\end{addmargin}
\end{absolutelynopagebreak}

\begin{absolutelynopagebreak}
\setstretch{.7}
{\PaliGlossA{atha kho te bhikkhū sāvatthiṃ piṇḍāya caritvā pacchābhattaṃ piṇḍapātapaṭikkantā yena bhagavā tenupasaṅkamiṃsu; upasaṅkamitvā bhagavantaṃ abhivādetvā ekamantaṃ nisīdiṃsu. ekamantaṃ nisinnā kho te bhikkhū bhagavantaṃ etadavocuṃ:}}\\
\begin{addmargin}[1em]{2em}
\setstretch{.5}
{\PaliGlossB{Then, after the meal, when they returned from alms-round, they went up to the Buddha, bowed, sat down to one side, and told him what had happened.}}\\
\end{addmargin}
\end{absolutelynopagebreak}

\begin{absolutelynopagebreak}
\setstretch{.7}
{\PaliGlossA{“idha mayaṃ, bhante, pubbaṇhasamayaṃ nivāsetvā pattacīvaramādāya sāvatthiṃ piṇḍāya pavisimha.}}\\
\begin{addmargin}[1em]{2em}
\setstretch{.5}
{\PaliGlossB{    -}}\\
\end{addmargin}
\end{absolutelynopagebreak}

\begin{absolutelynopagebreak}
\setstretch{.7}
{\PaliGlossA{tesaṃ no, bhante, amhākaṃ etadahosi:}}\\
\begin{addmargin}[1em]{2em}
\setstretch{.5}
{\PaliGlossB{    -}}\\
\end{addmargin}
\end{absolutelynopagebreak}

\begin{absolutelynopagebreak}
\setstretch{.7}
{\PaliGlossA{‘atippago kho tāva sāvatthiyaṃ piṇḍāya carituṃ,}}\\
\begin{addmargin}[1em]{2em}
\setstretch{.5}
{\PaliGlossB{    -}}\\
\end{addmargin}
\end{absolutelynopagebreak}

\begin{absolutelynopagebreak}
\setstretch{.7}
{\PaliGlossA{yannūna mayaṃ yena aññatitthiyānaṃ paribbājakānaṃ ārāmo tenupasaṅkameyyāmā’ti.}}\\
\begin{addmargin}[1em]{2em}
\setstretch{.5}
{\PaliGlossB{    -}}\\
\end{addmargin}
\end{absolutelynopagebreak}

\begin{absolutelynopagebreak}
\setstretch{.7}
{\PaliGlossA{atha kho mayaṃ, bhante, yena aññatitthiyānaṃ paribbājakānaṃ ārāmo tenupasaṅkamimha; upasaṅkamitvā tehi aññatitthiyehi paribbājakehi saddhiṃ sammodimha.}}\\
\begin{addmargin}[1em]{2em}
\setstretch{.5}
{\PaliGlossB{    -}}\\
\end{addmargin}
\end{absolutelynopagebreak}

\begin{absolutelynopagebreak}
\setstretch{.7}
{\PaliGlossA{sammodanīyaṃ kathaṃ sāraṇīyaṃ vītisāretvā ekamantaṃ nisīdimha. ekamantaṃ nisinne kho amhe, bhante, aññatitthiyā paribbājakā etadavocuṃ:}}\\
\begin{addmargin}[1em]{2em}
\setstretch{.5}
{\PaliGlossB{    -}}\\
\end{addmargin}
\end{absolutelynopagebreak}

\begin{absolutelynopagebreak}
\setstretch{.7}
{\PaliGlossA{‘samaṇo, āvuso, gotamo sāvakānaṃ evaṃ dhammaṃ deseti “etha tumhe, bhikkhave, pañca nīvaraṇe pahāya cetaso upakkilese paññāya dubbalīkaraṇe satta bojjhaṅge yathābhūtaṃ bhāvethā”ti.}}\\
\begin{addmargin}[1em]{2em}
\setstretch{.5}
{\PaliGlossB{    -}}\\
\end{addmargin}
\end{absolutelynopagebreak}

\begin{absolutelynopagebreak}
\setstretch{.7}
{\PaliGlossA{mayampi kho, āvuso, sāvakānaṃ evaṃ dhammaṃ desema:}}\\
\begin{addmargin}[1em]{2em}
\setstretch{.5}
{\PaliGlossB{    -}}\\
\end{addmargin}
\end{absolutelynopagebreak}

\begin{absolutelynopagebreak}
\setstretch{.7}
{\PaliGlossA{“etha tumhe, āvuso, pañca nīvaraṇe pahāya cetaso upakkilese paññāya dubbalīkaraṇe satta bojjhaṅge yathābhūtaṃ bhāvethā”ti.}}\\
\begin{addmargin}[1em]{2em}
\setstretch{.5}
{\PaliGlossB{    -}}\\
\end{addmargin}
\end{absolutelynopagebreak}

\begin{absolutelynopagebreak}
\setstretch{.7}
{\PaliGlossA{idha no, āvuso, ko viseso, ko adhippayāso, kiṃ nānākaraṇaṃ samaṇassa vā gotamassa amhākaṃ vā, yadidaṃ—dhammadesanāya vā dhammadesanaṃ, anusāsaniyā vā anusāsanin’ti?}}\\
\begin{addmargin}[1em]{2em}
\setstretch{.5}
{\PaliGlossB{    -}}\\
\end{addmargin}
\end{absolutelynopagebreak}

\begin{absolutelynopagebreak}
\setstretch{.7}
{\PaliGlossA{atha kho mayaṃ, bhante, tesaṃ aññatitthiyānaṃ paribbājakānaṃ bhāsitaṃ neva abhinandimha nappaṭikkosimha, anabhinanditvā appaṭikkositvā uṭṭhāyāsanā pakkamimha:}}\\
\begin{addmargin}[1em]{2em}
\setstretch{.5}
{\PaliGlossB{    -}}\\
\end{addmargin}
\end{absolutelynopagebreak}

\begin{absolutelynopagebreak}
\setstretch{.7}
{\PaliGlossA{‘bhagavato santike etassa bhāsitassa atthaṃ ājānissāmā’”ti.}}\\
\begin{addmargin}[1em]{2em}
\setstretch{.5}
{\PaliGlossB{    -}}\\
\end{addmargin}
\end{absolutelynopagebreak}

\begin{absolutelynopagebreak}
\setstretch{.7}
{\PaliGlossA{“evaṃvādino, bhikkhave, aññatitthiyā paribbājakā evamassu vacanīyā:}}\\
\begin{addmargin}[1em]{2em}
\setstretch{.5}
{\PaliGlossB{“Mendicants, when wanderers who follow other paths say this, you should say to them:}}\\
\end{addmargin}
\end{absolutelynopagebreak}

\begin{absolutelynopagebreak}
\setstretch{.7}
{\PaliGlossA{‘atthi panāvuso, pariyāyo, yaṃ pariyāyaṃ āgamma pañca nīvaraṇā dasa honti, satta bojjhaṅgā catuddasā’ti.}}\\
\begin{addmargin}[1em]{2em}
\setstretch{.5}
{\PaliGlossB{‘But reverends, is there a way in which the five hindrances become ten and the seven awakening factors become fourteen?’}}\\
\end{addmargin}
\end{absolutelynopagebreak}

\begin{absolutelynopagebreak}
\setstretch{.7}
{\PaliGlossA{evaṃ puṭṭhā, bhikkhave, aññatitthiyā paribbājakā na ceva sampāyissanti, uttariñca vighātaṃ āpajjissanti.}}\\
\begin{addmargin}[1em]{2em}
\setstretch{.5}
{\PaliGlossB{Questioned like this, the wanderers who follow other paths would be stumped, and, in addition, would get frustrated.}}\\
\end{addmargin}
\end{absolutelynopagebreak}

\begin{absolutelynopagebreak}
\setstretch{.7}
{\PaliGlossA{taṃ kissa hetu?}}\\
\begin{addmargin}[1em]{2em}
\setstretch{.5}
{\PaliGlossB{Why is that?}}\\
\end{addmargin}
\end{absolutelynopagebreak}

\begin{absolutelynopagebreak}
\setstretch{.7}
{\PaliGlossA{yathā taṃ, bhikkhave, avisayasmiṃ.}}\\
\begin{addmargin}[1em]{2em}
\setstretch{.5}
{\PaliGlossB{Because they’re out of their element.}}\\
\end{addmargin}
\end{absolutelynopagebreak}

\begin{absolutelynopagebreak}
\setstretch{.7}
{\PaliGlossA{nāhaṃ taṃ, bhikkhave, passāmi sadevake loke samārake sabrahmake sassamaṇabrāhmaṇiyā pajāya sadevamanussāya, yo imesaṃ pañhānaṃ veyyākaraṇena cittaṃ ārādheyya, aññatra tathāgatena vā tathāgatasāvakena vā ito vā pana sutvā.}}\\
\begin{addmargin}[1em]{2em}
\setstretch{.5}
{\PaliGlossB{I don’t see anyone in this world—with its gods, Māras, and Brahmās, this population with its ascetics and brahmins, its gods and humans—who could provide a satisfying answer to these questions except for the Realized One or his disciple or someone who has heard it from them.}}\\
\end{addmargin}
\end{absolutelynopagebreak}

\begin{absolutelynopagebreak}
\setstretch{.7}
{\PaliGlossA{katamo ca, bhikkhave, pariyāyo, yaṃ pariyāyaṃ āgamma pañca nīvaraṇā dasa honti?}}\\
\begin{addmargin}[1em]{2em}
\setstretch{.5}
{\PaliGlossB{And what is the way in which the five hindrances become ten?}}\\
\end{addmargin}
\end{absolutelynopagebreak}

\begin{absolutelynopagebreak}
\setstretch{.7}
{\PaliGlossA{yadapi, bhikkhave, ajjhattaṃ kāmacchando tadapi nīvaraṇaṃ, yadapi bahiddhā kāmacchando tadapi nīvaraṇaṃ.}}\\
\begin{addmargin}[1em]{2em}
\setstretch{.5}
{\PaliGlossB{Sensual desire for what is internal is a hindrance; and sensual desire for what is external is also a hindrance.}}\\
\end{addmargin}
\end{absolutelynopagebreak}

\begin{absolutelynopagebreak}
\setstretch{.7}
{\PaliGlossA{‘kāmacchandanīvaraṇan’ti iti hidaṃ uddesaṃ gacchati. tadamināpetaṃ pariyāyena dvayaṃ hoti.}}\\
\begin{addmargin}[1em]{2em}
\setstretch{.5}
{\PaliGlossB{That’s how what is concisely referred to as ‘the hindrance of sensual desire’ becomes twofold.}}\\
\end{addmargin}
\end{absolutelynopagebreak}

\begin{absolutelynopagebreak}
\setstretch{.7}
{\PaliGlossA{yadapi, bhikkhave, ajjhattaṃ byāpādo tadapi nīvaraṇaṃ, yadapi bahiddhā byāpādo tadapi nīvaraṇaṃ.}}\\
\begin{addmargin}[1em]{2em}
\setstretch{.5}
{\PaliGlossB{Ill will for what is internal is a hindrance; and ill will for what is external is also a hindrance.}}\\
\end{addmargin}
\end{absolutelynopagebreak}

\begin{absolutelynopagebreak}
\setstretch{.7}
{\PaliGlossA{‘byāpādanīvaraṇan’ti iti hidaṃ uddesaṃ gacchati. tadamināpetaṃ pariyāyena dvayaṃ hoti.}}\\
\begin{addmargin}[1em]{2em}
\setstretch{.5}
{\PaliGlossB{That’s how what is concisely referred to as ‘the hindrance of ill will’ becomes twofold.}}\\
\end{addmargin}
\end{absolutelynopagebreak}

\begin{absolutelynopagebreak}
\setstretch{.7}
{\PaliGlossA{yadapi, bhikkhave, thinaṃ tadapi nīvaraṇaṃ, yadapi middhaṃ tadapi nīvaraṇaṃ.}}\\
\begin{addmargin}[1em]{2em}
\setstretch{.5}
{\PaliGlossB{Dullness is a hindrance; and drowsiness is also a hindrance.}}\\
\end{addmargin}
\end{absolutelynopagebreak}

\begin{absolutelynopagebreak}
\setstretch{.7}
{\PaliGlossA{‘thinamiddhanīvaraṇan’ti iti hidaṃ uddesaṃ gacchati. tadamināpetaṃ pariyāyena dvayaṃ hoti.}}\\
\begin{addmargin}[1em]{2em}
\setstretch{.5}
{\PaliGlossB{That’s how what is concisely referred to as ‘the hindrance of dullness and drowsiness’ becomes twofold.}}\\
\end{addmargin}
\end{absolutelynopagebreak}

\begin{absolutelynopagebreak}
\setstretch{.7}
{\PaliGlossA{yadapi, bhikkhave, uddhaccaṃ tadapi nīvaraṇaṃ, yadapi kukkuccaṃ tadapi nīvaraṇaṃ.}}\\
\begin{addmargin}[1em]{2em}
\setstretch{.5}
{\PaliGlossB{Restlessness is a hindrance; and remorse is also a hindrance.}}\\
\end{addmargin}
\end{absolutelynopagebreak}

\begin{absolutelynopagebreak}
\setstretch{.7}
{\PaliGlossA{‘uddhaccakukkuccanīvaraṇan’ti iti hidaṃ uddesaṃ gacchati. tadamināpetaṃ pariyāyena dvayaṃ hoti.}}\\
\begin{addmargin}[1em]{2em}
\setstretch{.5}
{\PaliGlossB{That’s how what is concisely referred to as ‘the hindrance of restlessness and remorse’ becomes twofold.}}\\
\end{addmargin}
\end{absolutelynopagebreak}

\begin{absolutelynopagebreak}
\setstretch{.7}
{\PaliGlossA{yadapi, bhikkhave, ajjhattaṃ dhammesu vicikicchā tadapi nīvaraṇaṃ, yadapi bahiddhā dhammesu vicikicchā tadapi nīvaraṇaṃ.}}\\
\begin{addmargin}[1em]{2em}
\setstretch{.5}
{\PaliGlossB{Doubt about internal things is a hindrance; and doubt about external things is also a hindrance.}}\\
\end{addmargin}
\end{absolutelynopagebreak}

\begin{absolutelynopagebreak}
\setstretch{.7}
{\PaliGlossA{‘vicikicchānīvaraṇan’ti iti hidaṃ uddesaṃ gacchati. tadamināpetaṃ pariyāyena dvayaṃ hoti.}}\\
\begin{addmargin}[1em]{2em}
\setstretch{.5}
{\PaliGlossB{That’s how what is concisely referred to as ‘the hindrance of doubt’ becomes twofold.}}\\
\end{addmargin}
\end{absolutelynopagebreak}

\begin{absolutelynopagebreak}
\setstretch{.7}
{\PaliGlossA{ayaṃ kho, bhikkhave, pariyāyo, yaṃ pariyāyaṃ āgamma pañca nīvaraṇā dasa honti.}}\\
\begin{addmargin}[1em]{2em}
\setstretch{.5}
{\PaliGlossB{This is the way in which the five hindrances become ten.}}\\
\end{addmargin}
\end{absolutelynopagebreak}

\begin{absolutelynopagebreak}
\setstretch{.7}
{\PaliGlossA{katamo ca, bhikkhave, pariyāyo, yaṃ pariyāyaṃ āgamma satta bojjhaṅgā catuddasa honti?}}\\
\begin{addmargin}[1em]{2em}
\setstretch{.5}
{\PaliGlossB{And what is the way in which the seven awakening factors become fourteen?}}\\
\end{addmargin}
\end{absolutelynopagebreak}

\begin{absolutelynopagebreak}
\setstretch{.7}
{\PaliGlossA{yadapi, bhikkhave, ajjhattaṃ dhammesu sati tadapi satisambojjhaṅgo, yadapi bahiddhā dhammesu sati tadapi satisambojjhaṅgo.}}\\
\begin{addmargin}[1em]{2em}
\setstretch{.5}
{\PaliGlossB{Mindfulness of internal things is the awakening factor of mindfulness; and mindfulness of external things is also the awakening factor of mindfulness.}}\\
\end{addmargin}
\end{absolutelynopagebreak}

\begin{absolutelynopagebreak}
\setstretch{.7}
{\PaliGlossA{‘satisambojjhaṅgo’ti iti hidaṃ uddesaṃ gacchati. tadamināpetaṃ pariyāyena dvayaṃ hoti.}}\\
\begin{addmargin}[1em]{2em}
\setstretch{.5}
{\PaliGlossB{That’s how what is concisely referred to as ‘the awakening factor of mindfulness’ becomes twofold.}}\\
\end{addmargin}
\end{absolutelynopagebreak}

\begin{absolutelynopagebreak}
\setstretch{.7}
{\PaliGlossA{yadapi, bhikkhave, ajjhattaṃ dhammesu paññāya pavicinati pavicarati parivīmaṃsamāpajjati tadapi dhammavicayasambojjhaṅgo, yadapi bahiddhā dhammesu paññāya pavicinati pavicarati parivīmaṃsamāpajjati tadapi dhammavicayasambojjhaṅgo.}}\\
\begin{addmargin}[1em]{2em}
\setstretch{.5}
{\PaliGlossB{Investigating, exploring, and inquiring into internal things with wisdom is the awakening factor of investigation of principles; and investigating, exploring, and inquiring into external things with wisdom is also the awakening factor of investigation of principles.}}\\
\end{addmargin}
\end{absolutelynopagebreak}

\begin{absolutelynopagebreak}
\setstretch{.7}
{\PaliGlossA{‘dhammavicayasambojjhaṅgo’ti iti hidaṃ uddesaṃ gacchati. tadamināpetaṃ pariyāyena dvayaṃ hoti.}}\\
\begin{addmargin}[1em]{2em}
\setstretch{.5}
{\PaliGlossB{That’s how what is concisely referred to as ‘the awakening factor of investigation of principles’ becomes twofold.}}\\
\end{addmargin}
\end{absolutelynopagebreak}

\begin{absolutelynopagebreak}
\setstretch{.7}
{\PaliGlossA{yadapi, bhikkhave, kāyikaṃ vīriyaṃ tadapi vīriyasambojjhaṅgo, yadapi cetasikaṃ vīriyaṃ tadapi vīriyasambojjhaṅgo.}}\\
\begin{addmargin}[1em]{2em}
\setstretch{.5}
{\PaliGlossB{Physical energy is the awakening factor of energy; and mental energy is also the awakening factor of energy.}}\\
\end{addmargin}
\end{absolutelynopagebreak}

\begin{absolutelynopagebreak}
\setstretch{.7}
{\PaliGlossA{‘vīriyasambojjhaṅgo’ti iti hidaṃ uddesaṃ gacchati. tadamināpetaṃ pariyāyena dvayaṃ hoti.}}\\
\begin{addmargin}[1em]{2em}
\setstretch{.5}
{\PaliGlossB{That’s how what is concisely referred to as ‘the awakening factor of energy’ becomes twofold.}}\\
\end{addmargin}
\end{absolutelynopagebreak}

\begin{absolutelynopagebreak}
\setstretch{.7}
{\PaliGlossA{yadapi, bhikkhave, savitakkasavicārā pīti tadapi pītisambojjhaṅgo, yadapi avitakkaavicārā pīti tadapi pītisambojjhaṅgo.}}\\
\begin{addmargin}[1em]{2em}
\setstretch{.5}
{\PaliGlossB{Rapture while placing the mind and keeping it connected is the awakening factor of rapture; and rapture without placing the mind and keeping it connected is also the awakening factor of rapture.}}\\
\end{addmargin}
\end{absolutelynopagebreak}

\begin{absolutelynopagebreak}
\setstretch{.7}
{\PaliGlossA{‘pītisambojjhaṅgo’ti iti hidaṃ uddesaṃ gacchati. tadamināpetaṃ pariyāyena dvayaṃ hoti.}}\\
\begin{addmargin}[1em]{2em}
\setstretch{.5}
{\PaliGlossB{In this way what is concisely referred to as ‘the awakening factor of rapture’ becomes twofold.}}\\
\end{addmargin}
\end{absolutelynopagebreak}

\begin{absolutelynopagebreak}
\setstretch{.7}
{\PaliGlossA{yadapi, bhikkhave, kāyappassaddhi tadapi passaddhisambojjhaṅgo, yadapi cittappassaddhi tadapi passaddhisambojjhaṅgo.}}\\
\begin{addmargin}[1em]{2em}
\setstretch{.5}
{\PaliGlossB{Physical tranquility is the awakening factor of tranquility; and mental tranquility is also the awakening factor of tranquility.}}\\
\end{addmargin}
\end{absolutelynopagebreak}

\begin{absolutelynopagebreak}
\setstretch{.7}
{\PaliGlossA{‘passaddhisambojjhaṅgo’ti iti hidaṃ uddesaṃ gacchati. tadamināpetaṃ pariyāyena dvayaṃ hoti.}}\\
\begin{addmargin}[1em]{2em}
\setstretch{.5}
{\PaliGlossB{In this way what is concisely referred to as ‘the awakening factor of tranquility’ becomes twofold.}}\\
\end{addmargin}
\end{absolutelynopagebreak}

\begin{absolutelynopagebreak}
\setstretch{.7}
{\PaliGlossA{yadapi, bhikkhave, savitakko savicāro samādhi tadapi samādhisambojjhaṅgo, yadapi avitakkaavicāro samādhi tadapi samādhisambojjhaṅgo.}}\\
\begin{addmargin}[1em]{2em}
\setstretch{.5}
{\PaliGlossB{Immersion while placing the mind and keeping it connected is the awakening factor of immersion; and immersion without placing the mind and keeping it connected is also the awakening factor of immersion.}}\\
\end{addmargin}
\end{absolutelynopagebreak}

\begin{absolutelynopagebreak}
\setstretch{.7}
{\PaliGlossA{‘samādhisambojjhaṅgo’ti iti hidaṃ uddesaṃ gacchati. tadamināpetaṃ pariyāyena dvayaṃ hoti.}}\\
\begin{addmargin}[1em]{2em}
\setstretch{.5}
{\PaliGlossB{In this way what is concisely referred to as ‘the awakening factor of immersion’ becomes twofold.}}\\
\end{addmargin}
\end{absolutelynopagebreak}

\begin{absolutelynopagebreak}
\setstretch{.7}
{\PaliGlossA{yadapi, bhikkhave, ajjhattaṃ dhammesu upekkhā tadapi upekkhāsambojjhaṅgo, yadapi bahiddhā dhammesu upekkhā tadapi upekkhāsambojjhaṅgo.}}\\
\begin{addmargin}[1em]{2em}
\setstretch{.5}
{\PaliGlossB{Equanimity for internal things is the awakening factor of equanimity; and equanimity for external things is also the awakening factor of equanimity.}}\\
\end{addmargin}
\end{absolutelynopagebreak}

\begin{absolutelynopagebreak}
\setstretch{.7}
{\PaliGlossA{‘upekkhāsambojjhaṅgo’ti iti hidaṃ uddesaṃ gacchati. tadamināpetaṃ pariyāyena dvayaṃ hoti.}}\\
\begin{addmargin}[1em]{2em}
\setstretch{.5}
{\PaliGlossB{In this way what is concisely referred to as ‘the awakening factor of equanimity’ becomes twofold.}}\\
\end{addmargin}
\end{absolutelynopagebreak}

\begin{absolutelynopagebreak}
\setstretch{.7}
{\PaliGlossA{ayaṃ kho, bhikkhave, pariyāyo, yaṃ pariyāyaṃ āgamma satta bojjhaṅgā catuddasā”ti.}}\\
\begin{addmargin}[1em]{2em}
\setstretch{.5}
{\PaliGlossB{This is the way in which the seven awakening factors become fourteen.”}}\\
\end{addmargin}
\end{absolutelynopagebreak}

\begin{absolutelynopagebreak}
\setstretch{.7}
{\PaliGlossA{dutiyaṃ.}}\\
\begin{addmargin}[1em]{2em}
\setstretch{.5}
{\PaliGlossB{    -}}\\
\end{addmargin}
\end{absolutelynopagebreak}
