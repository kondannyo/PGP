
\begin{absolutelynopagebreak}
\setstretch{.7}
{\PaliGlossA{saṃyutta nikāya 35}}\\
\begin{addmargin}[1em]{2em}
\setstretch{.5}
{\PaliGlossB{Linked Discourses 35}}\\
\end{addmargin}
\end{absolutelynopagebreak}

\begin{absolutelynopagebreak}
\setstretch{.7}
{\PaliGlossA{16. nandikkhayavagga}}\\
\begin{addmargin}[1em]{2em}
\setstretch{.5}
{\PaliGlossB{16. The End of Relishing}}\\
\end{addmargin}
\end{absolutelynopagebreak}

\begin{absolutelynopagebreak}
\setstretch{.7}
{\PaliGlossA{163. koṭṭhikadukkhasutta}}\\
\begin{addmargin}[1em]{2em}
\setstretch{.5}
{\PaliGlossB{163. With Koṭṭhita on Suffering}}\\
\end{addmargin}
\end{absolutelynopagebreak}

\begin{absolutelynopagebreak}
\setstretch{.7}
{\PaliGlossA{atha kho āyasmā mahākoṭṭhiko … pe … bhagavantaṃ etadavoca:}}\\
\begin{addmargin}[1em]{2em}
\setstretch{.5}
{\PaliGlossB{Then Venerable Mahākoṭṭhita … said to the Buddha:}}\\
\end{addmargin}
\end{absolutelynopagebreak}

\begin{absolutelynopagebreak}
\setstretch{.7}
{\PaliGlossA{“sādhu me, bhante … pe … vihareyyan”ti.}}\\
\begin{addmargin}[1em]{2em}
\setstretch{.5}
{\PaliGlossB{“Sir, may the Buddha please teach me Dhamma in brief. When I’ve heard it, I’ll live alone, withdrawn, diligent, keen, and resolute.”}}\\
\end{addmargin}
\end{absolutelynopagebreak}

\begin{absolutelynopagebreak}
\setstretch{.7}
{\PaliGlossA{“yaṃ kho, koṭṭhika, dukkhaṃ tatra te chando pahātabbo.}}\\
\begin{addmargin}[1em]{2em}
\setstretch{.5}
{\PaliGlossB{“Koṭṭhita, you should give up desire for what is suffering.}}\\
\end{addmargin}
\end{absolutelynopagebreak}

\begin{absolutelynopagebreak}
\setstretch{.7}
{\PaliGlossA{kiñca, koṭṭhika, dukkhaṃ?}}\\
\begin{addmargin}[1em]{2em}
\setstretch{.5}
{\PaliGlossB{And what is suffering?}}\\
\end{addmargin}
\end{absolutelynopagebreak}

\begin{absolutelynopagebreak}
\setstretch{.7}
{\PaliGlossA{cakkhu kho, koṭṭhika, dukkhaṃ; tatra te chando pahātabbo.}}\\
\begin{addmargin}[1em]{2em}
\setstretch{.5}
{\PaliGlossB{The eye,}}\\
\end{addmargin}
\end{absolutelynopagebreak}

\begin{absolutelynopagebreak}
\setstretch{.7}
{\PaliGlossA{rūpā dukkhā; tatra te chando pahātabbo.}}\\
\begin{addmargin}[1em]{2em}
\setstretch{.5}
{\PaliGlossB{sights,}}\\
\end{addmargin}
\end{absolutelynopagebreak}

\begin{absolutelynopagebreak}
\setstretch{.7}
{\PaliGlossA{cakkhuviññāṇaṃ dukkhaṃ; tatra te chando pahātabbo.}}\\
\begin{addmargin}[1em]{2em}
\setstretch{.5}
{\PaliGlossB{eye consciousness,}}\\
\end{addmargin}
\end{absolutelynopagebreak}

\begin{absolutelynopagebreak}
\setstretch{.7}
{\PaliGlossA{cakkhusamphasso dukkho; tatra te chando pahātabbo.}}\\
\begin{addmargin}[1em]{2em}
\setstretch{.5}
{\PaliGlossB{and eye contact are suffering: you should give up desire for them.}}\\
\end{addmargin}
\end{absolutelynopagebreak}

\begin{absolutelynopagebreak}
\setstretch{.7}
{\PaliGlossA{yampidaṃ cakkhusamphassapaccayā uppajjati vedayitaṃ sukhaṃ vā dukkhaṃ vā adukkhamasukhaṃ vā tampi dukkhaṃ; tatra te chando pahātabbo … pe …}}\\
\begin{addmargin}[1em]{2em}
\setstretch{.5}
{\PaliGlossB{The pleasant, painful, or neutral feeling that arises conditioned by eye contact is also suffering; you should give up desire for it.}}\\
\end{addmargin}
\end{absolutelynopagebreak}

\begin{absolutelynopagebreak}
\setstretch{.7}
{\PaliGlossA{jivhā dukkhā; tatra te chando pahātabbo … pe …}}\\
\begin{addmargin}[1em]{2em}
\setstretch{.5}
{\PaliGlossB{The ear … nose … tongue … body …}}\\
\end{addmargin}
\end{absolutelynopagebreak}

\begin{absolutelynopagebreak}
\setstretch{.7}
{\PaliGlossA{mano dukkho; tatra te chando pahātabbo.}}\\
\begin{addmargin}[1em]{2em}
\setstretch{.5}
{\PaliGlossB{The mind,}}\\
\end{addmargin}
\end{absolutelynopagebreak}

\begin{absolutelynopagebreak}
\setstretch{.7}
{\PaliGlossA{dhammā dukkhā; tatra te chando pahātabbo.}}\\
\begin{addmargin}[1em]{2em}
\setstretch{.5}
{\PaliGlossB{thoughts,}}\\
\end{addmargin}
\end{absolutelynopagebreak}

\begin{absolutelynopagebreak}
\setstretch{.7}
{\PaliGlossA{manoviññāṇaṃ dukkhaṃ; tatra te chando pahātabbo.}}\\
\begin{addmargin}[1em]{2em}
\setstretch{.5}
{\PaliGlossB{mind consciousness,}}\\
\end{addmargin}
\end{absolutelynopagebreak}

\begin{absolutelynopagebreak}
\setstretch{.7}
{\PaliGlossA{manosamphasso dukkho; tatra te chando pahātabbo.}}\\
\begin{addmargin}[1em]{2em}
\setstretch{.5}
{\PaliGlossB{and mind contact are suffering: you should give up desire for them.}}\\
\end{addmargin}
\end{absolutelynopagebreak}

\begin{absolutelynopagebreak}
\setstretch{.7}
{\PaliGlossA{yampidaṃ manosamphassapaccayā uppajjati vedayitaṃ sukhaṃ vā dukkhaṃ vā adukkhamasukhaṃ vā tampi dukkhaṃ; tatra te chando pahātabbo.}}\\
\begin{addmargin}[1em]{2em}
\setstretch{.5}
{\PaliGlossB{The pleasant, painful, or neutral feeling that arises conditioned by mind contact is also suffering: you should give up desire for it.}}\\
\end{addmargin}
\end{absolutelynopagebreak}

\begin{absolutelynopagebreak}
\setstretch{.7}
{\PaliGlossA{yaṃ kho, koṭṭhika, dukkhaṃ tatra te chando pahātabbo”ti.}}\\
\begin{addmargin}[1em]{2em}
\setstretch{.5}
{\PaliGlossB{Koṭṭhita, you should give up desire for what is suffering.”}}\\
\end{addmargin}
\end{absolutelynopagebreak}

\begin{absolutelynopagebreak}
\setstretch{.7}
{\PaliGlossA{aṭṭhamaṃ.}}\\
\begin{addmargin}[1em]{2em}
\setstretch{.5}
{\PaliGlossB{    -}}\\
\end{addmargin}
\end{absolutelynopagebreak}
