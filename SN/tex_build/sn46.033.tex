
\begin{absolutelynopagebreak}
\setstretch{.7}
{\PaliGlossA{saṃyutta nikāya 46}}\\
\begin{addmargin}[1em]{2em}
\setstretch{.5}
{\PaliGlossB{Linked Discourses 46}}\\
\end{addmargin}
\end{absolutelynopagebreak}

\begin{absolutelynopagebreak}
\setstretch{.7}
{\PaliGlossA{4. nīvaraṇavagga}}\\
\begin{addmargin}[1em]{2em}
\setstretch{.5}
{\PaliGlossB{4. Hindrances}}\\
\end{addmargin}
\end{absolutelynopagebreak}

\begin{absolutelynopagebreak}
\setstretch{.7}
{\PaliGlossA{33. upakkilesasutta}}\\
\begin{addmargin}[1em]{2em}
\setstretch{.5}
{\PaliGlossB{33. Corruptions}}\\
\end{addmargin}
\end{absolutelynopagebreak}

\begin{absolutelynopagebreak}
\setstretch{.7}
{\PaliGlossA{“pañcime, bhikkhave, jātarūpassa upakkilesā, yehi upakkilesehi upakkiliṭṭhaṃ jātarūpaṃ na ceva mudu hoti na ca kammaniyaṃ, na ca pabhassaraṃ pabhaṅgu ca, na ca sammā upeti kammāya.}}\\
\begin{addmargin}[1em]{2em}
\setstretch{.5}
{\PaliGlossB{“Mendicants, there are these five corruptions of gold. When gold is corrupted by these it’s not pliable, workable, or radiant, but is brittle and not completely ready for working.}}\\
\end{addmargin}
\end{absolutelynopagebreak}

\begin{absolutelynopagebreak}
\setstretch{.7}
{\PaliGlossA{katame pañca?}}\\
\begin{addmargin}[1em]{2em}
\setstretch{.5}
{\PaliGlossB{What five?}}\\
\end{addmargin}
\end{absolutelynopagebreak}

\begin{absolutelynopagebreak}
\setstretch{.7}
{\PaliGlossA{ayo, bhikkhave, jātarūpassa upakkileso, yena upakkilesena upakkiliṭṭhaṃ jātarūpaṃ na ceva mudu hoti na ca kammaniyaṃ, na ca pabhassaraṃ pabhaṅgu ca, na ca sammā upeti kammāya.}}\\
\begin{addmargin}[1em]{2em}
\setstretch{.5}
{\PaliGlossB{Iron,}}\\
\end{addmargin}
\end{absolutelynopagebreak}

\begin{absolutelynopagebreak}
\setstretch{.7}
{\PaliGlossA{lohaṃ, bhikkhave, jātarūpassa upakkileso, yena upakkilesena upakkiliṭṭhaṃ jātarūpaṃ … pe …}}\\
\begin{addmargin}[1em]{2em}
\setstretch{.5}
{\PaliGlossB{copper,}}\\
\end{addmargin}
\end{absolutelynopagebreak}

\begin{absolutelynopagebreak}
\setstretch{.7}
{\PaliGlossA{tipu, bhikkhave, jātarūpassa upakkileso … pe …}}\\
\begin{addmargin}[1em]{2em}
\setstretch{.5}
{\PaliGlossB{tin,}}\\
\end{addmargin}
\end{absolutelynopagebreak}

\begin{absolutelynopagebreak}
\setstretch{.7}
{\PaliGlossA{sīsaṃ, bhikkhave, jātarūpassa upakkileso … pe …}}\\
\begin{addmargin}[1em]{2em}
\setstretch{.5}
{\PaliGlossB{lead,}}\\
\end{addmargin}
\end{absolutelynopagebreak}

\begin{absolutelynopagebreak}
\setstretch{.7}
{\PaliGlossA{sajjhu, bhikkhave, jātarūpassa upakkileso, yena upakkilesena upakkiliṭṭhaṃ jātarūpaṃ na ceva mudu hoti na ca kammaniyaṃ, na ca pabhassaraṃ pabhaṅgu ca, na ca sammā upeti kammāya.}}\\
\begin{addmargin}[1em]{2em}
\setstretch{.5}
{\PaliGlossB{and silver.}}\\
\end{addmargin}
\end{absolutelynopagebreak}

\begin{absolutelynopagebreak}
\setstretch{.7}
{\PaliGlossA{ime kho, bhikkhave, pañca jātarūpassa upakkilesā, yehi upakkilesehi upakkiliṭṭhaṃ jātarūpaṃ na ceva mudu hoti na ca kammaniyaṃ, na ca pabhassaraṃ pabhaṅgu ca, na ca sammā upeti kammāya.}}\\
\begin{addmargin}[1em]{2em}
\setstretch{.5}
{\PaliGlossB{When gold is corrupted by these five corruptions it’s not pliable, workable, or radiant, but is brittle and not completely ready for working.}}\\
\end{addmargin}
\end{absolutelynopagebreak}

\begin{absolutelynopagebreak}
\setstretch{.7}
{\PaliGlossA{evameva kho, bhikkhave, pañcime cittassa upakkilesā, yehi upakkilesehi upakkiliṭṭhaṃ cittaṃ na ceva mudu hoti na ca kammaniyaṃ, na ca pabhassaraṃ pabhaṅgu ca, na ca sammā samādhiyati āsavānaṃ khayāya.}}\\
\begin{addmargin}[1em]{2em}
\setstretch{.5}
{\PaliGlossB{In the same way, there are these five corruptions of the mind. When the mind is corrupted by these it’s not pliable, workable, or radiant. It’s brittle, and not completely immersed in samādhi for the ending of defilements.}}\\
\end{addmargin}
\end{absolutelynopagebreak}

\begin{absolutelynopagebreak}
\setstretch{.7}
{\PaliGlossA{katame pañca?}}\\
\begin{addmargin}[1em]{2em}
\setstretch{.5}
{\PaliGlossB{What five?}}\\
\end{addmargin}
\end{absolutelynopagebreak}

\begin{absolutelynopagebreak}
\setstretch{.7}
{\PaliGlossA{kāmacchando, bhikkhave, cittassa upakkileso, yena upakkilesena upakkiliṭṭhaṃ cittaṃ na ceva mudu hoti na ca kammaniyaṃ, na ca pabhassaraṃ pabhaṅgu ca, na ca sammā samādhiyati āsavānaṃ khayāya … pe …}}\\
\begin{addmargin}[1em]{2em}
\setstretch{.5}
{\PaliGlossB{Sensual desire, ill will, dullness and drowsiness, restlessness and remorse, and doubt.}}\\
\end{addmargin}
\end{absolutelynopagebreak}

\begin{absolutelynopagebreak}
\setstretch{.7}
{\PaliGlossA{ime kho, bhikkhave, pañca cittassa upakkilesā, yehi upakkilesehi upakkiliṭṭhaṃ cittaṃ na ceva mudu hoti na ca kammaniyaṃ, na ca pabhassaraṃ pabhaṅgu ca, na ca sammā samādhiyati āsavānaṃ khayāyā”ti.}}\\
\begin{addmargin}[1em]{2em}
\setstretch{.5}
{\PaliGlossB{These are the five corruptions of the mind. When the mind is corrupted by these it’s not pliable, workable, or radiant. It’s brittle, and not completely immersed in samādhi for the ending of defilements.”}}\\
\end{addmargin}
\end{absolutelynopagebreak}

\begin{absolutelynopagebreak}
\setstretch{.7}
{\PaliGlossA{tatiyaṃ.}}\\
\begin{addmargin}[1em]{2em}
\setstretch{.5}
{\PaliGlossB{    -}}\\
\end{addmargin}
\end{absolutelynopagebreak}
