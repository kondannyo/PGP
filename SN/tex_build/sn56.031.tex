
\begin{absolutelynopagebreak}
\setstretch{.7}
{\PaliGlossA{saṃyutta nikāya 56}}\\
\begin{addmargin}[1em]{2em}
\setstretch{.5}
{\PaliGlossB{Linked Discourses 56}}\\
\end{addmargin}
\end{absolutelynopagebreak}

\begin{absolutelynopagebreak}
\setstretch{.7}
{\PaliGlossA{4. sīsapāvanavagga}}\\
\begin{addmargin}[1em]{2em}
\setstretch{.5}
{\PaliGlossB{4. In a Rosewood Forest}}\\
\end{addmargin}
\end{absolutelynopagebreak}

\begin{absolutelynopagebreak}
\setstretch{.7}
{\PaliGlossA{31. sīsapāvanasutta}}\\
\begin{addmargin}[1em]{2em}
\setstretch{.5}
{\PaliGlossB{31. In a Rosewood Forest}}\\
\end{addmargin}
\end{absolutelynopagebreak}

\begin{absolutelynopagebreak}
\setstretch{.7}
{\PaliGlossA{ekaṃ samayaṃ bhagavā kosambiyaṃ viharati sīsapāvane.}}\\
\begin{addmargin}[1em]{2em}
\setstretch{.5}
{\PaliGlossB{At one time the Buddha was staying near Kosambī in a rosewood forest.}}\\
\end{addmargin}
\end{absolutelynopagebreak}

\begin{absolutelynopagebreak}
\setstretch{.7}
{\PaliGlossA{atha kho bhagavā parittāni sīsapāpaṇṇāni pāṇinā gahetvā bhikkhū āmantesi:}}\\
\begin{addmargin}[1em]{2em}
\setstretch{.5}
{\PaliGlossB{Then the Buddha picked up a few rosewood leaves in his hand and addressed the mendicants:}}\\
\end{addmargin}
\end{absolutelynopagebreak}

\begin{absolutelynopagebreak}
\setstretch{.7}
{\PaliGlossA{“taṃ kiṃ maññatha, bhikkhave,}}\\
\begin{addmargin}[1em]{2em}
\setstretch{.5}
{\PaliGlossB{“What do you think, mendicants?}}\\
\end{addmargin}
\end{absolutelynopagebreak}

\begin{absolutelynopagebreak}
\setstretch{.7}
{\PaliGlossA{katamaṃ nu kho bahutaraṃ—}}\\
\begin{addmargin}[1em]{2em}
\setstretch{.5}
{\PaliGlossB{Which is more:}}\\
\end{addmargin}
\end{absolutelynopagebreak}

\begin{absolutelynopagebreak}
\setstretch{.7}
{\PaliGlossA{yāni vā mayā parittāni sīsapāpaṇṇāni pāṇinā gahitāni yadidaṃ upari sīsapāvane”ti?}}\\
\begin{addmargin}[1em]{2em}
\setstretch{.5}
{\PaliGlossB{the few leaves in my hand, or those in the forest above me?”}}\\
\end{addmargin}
\end{absolutelynopagebreak}

\begin{absolutelynopagebreak}
\setstretch{.7}
{\PaliGlossA{“appamattakāni, bhante, bhagavatā parittāni sīsapāpaṇṇāni pāṇinā gahitāni;}}\\
\begin{addmargin}[1em]{2em}
\setstretch{.5}
{\PaliGlossB{“Sir, the few leaves in your hand are a tiny amount.}}\\
\end{addmargin}
\end{absolutelynopagebreak}

\begin{absolutelynopagebreak}
\setstretch{.7}
{\PaliGlossA{atha kho etāneva bahutarāni yadidaṃ upari sīsapāvane”ti.}}\\
\begin{addmargin}[1em]{2em}
\setstretch{.5}
{\PaliGlossB{There are far more leaves in the forest above.”}}\\
\end{addmargin}
\end{absolutelynopagebreak}

\begin{absolutelynopagebreak}
\setstretch{.7}
{\PaliGlossA{“evameva kho, bhikkhave, etadeva bahutaraṃ yaṃ vo mayā abhiññāya anakkhātaṃ.}}\\
\begin{addmargin}[1em]{2em}
\setstretch{.5}
{\PaliGlossB{“In the same way, there is much more that I have directly known but have not explained to you. What I have explained is a tiny amount.}}\\
\end{addmargin}
\end{absolutelynopagebreak}

\begin{absolutelynopagebreak}
\setstretch{.7}
{\PaliGlossA{kasmā cetaṃ, bhikkhave, mayā anakkhātaṃ?}}\\
\begin{addmargin}[1em]{2em}
\setstretch{.5}
{\PaliGlossB{And why haven’t I explained it?}}\\
\end{addmargin}
\end{absolutelynopagebreak}

\begin{absolutelynopagebreak}
\setstretch{.7}
{\PaliGlossA{na hetaṃ, bhikkhave, atthasaṃhitaṃ nādibrahmacariyakaṃ na nibbidāya na virāgāya na nirodhāya na upasamāya na abhiññāya na sambodhāya na nibbānāya saṃvattati;}}\\
\begin{addmargin}[1em]{2em}
\setstretch{.5}
{\PaliGlossB{Because it’s not beneficial or relevant to the fundamentals of the spiritual life. It doesn’t lead to disillusionment, dispassion, cessation, peace, insight, awakening, and extinguishment.}}\\
\end{addmargin}
\end{absolutelynopagebreak}

\begin{absolutelynopagebreak}
\setstretch{.7}
{\PaliGlossA{tasmā taṃ mayā anakkhātaṃ.}}\\
\begin{addmargin}[1em]{2em}
\setstretch{.5}
{\PaliGlossB{That’s why I haven’t explained it.}}\\
\end{addmargin}
\end{absolutelynopagebreak}

\begin{absolutelynopagebreak}
\setstretch{.7}
{\PaliGlossA{kiñca, bhikkhave, mayā akkhātaṃ?}}\\
\begin{addmargin}[1em]{2em}
\setstretch{.5}
{\PaliGlossB{And what have I explained?}}\\
\end{addmargin}
\end{absolutelynopagebreak}

\begin{absolutelynopagebreak}
\setstretch{.7}
{\PaliGlossA{‘idaṃ dukkhan’ti, bhikkhave, mayā akkhātaṃ, ‘ayaṃ dukkhasamudayo’ti mayā akkhātaṃ, ‘ayaṃ dukkhanirodho’ti mayā akkhātaṃ, ‘ayaṃ dukkhanirodhagāminī paṭipadā’ti mayā akkhātaṃ.}}\\
\begin{addmargin}[1em]{2em}
\setstretch{.5}
{\PaliGlossB{I have explained: ‘This is suffering’ … ‘This is the origin of suffering’ … ‘This is the cessation of suffering’ … ‘This is the practice that leads to the cessation of suffering’.}}\\
\end{addmargin}
\end{absolutelynopagebreak}

\begin{absolutelynopagebreak}
\setstretch{.7}
{\PaliGlossA{kasmā cetaṃ, bhikkhave, mayā akkhātaṃ?}}\\
\begin{addmargin}[1em]{2em}
\setstretch{.5}
{\PaliGlossB{And why have I explained this?}}\\
\end{addmargin}
\end{absolutelynopagebreak}

\begin{absolutelynopagebreak}
\setstretch{.7}
{\PaliGlossA{etañhi, bhikkhave, atthasaṃhitaṃ etaṃ ādibrahmacariyakaṃ etaṃ nibbidāya virāgāya nirodhāya upasamāya abhiññāya sambodhāya nibbānāya saṃvattati;}}\\
\begin{addmargin}[1em]{2em}
\setstretch{.5}
{\PaliGlossB{Because it’s beneficial and relevant to the fundamentals of the spiritual life. It leads to disillusionment, dispassion, cessation, peace, insight, awakening, and extinguishment.}}\\
\end{addmargin}
\end{absolutelynopagebreak}

\begin{absolutelynopagebreak}
\setstretch{.7}
{\PaliGlossA{tasmā taṃ mayā akkhātaṃ.}}\\
\begin{addmargin}[1em]{2em}
\setstretch{.5}
{\PaliGlossB{That’s why I’ve explained it.}}\\
\end{addmargin}
\end{absolutelynopagebreak}

\begin{absolutelynopagebreak}
\setstretch{.7}
{\PaliGlossA{tasmātiha, bhikkhave, ‘idaṃ dukkhan’ti yogo karaṇīyo … pe … ‘ayaṃ dukkhanirodhagāminī paṭipadā’ti yogo karaṇīyo”ti.}}\\
\begin{addmargin}[1em]{2em}
\setstretch{.5}
{\PaliGlossB{That’s why you should practice meditation …”}}\\
\end{addmargin}
\end{absolutelynopagebreak}

\begin{absolutelynopagebreak}
\setstretch{.7}
{\PaliGlossA{paṭhamaṃ.}}\\
\begin{addmargin}[1em]{2em}
\setstretch{.5}
{\PaliGlossB{    -}}\\
\end{addmargin}
\end{absolutelynopagebreak}
