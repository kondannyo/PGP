
\begin{absolutelynopagebreak}
\setstretch{.7}
{\PaliGlossA{saṃyutta nikāya 12}}\\
\begin{addmargin}[1em]{2em}
\setstretch{.5}
{\PaliGlossB{Linked Discourses 12}}\\
\end{addmargin}
\end{absolutelynopagebreak}

\begin{absolutelynopagebreak}
\setstretch{.7}
{\PaliGlossA{8. samaṇabrāhmaṇavagga}}\\
\begin{addmargin}[1em]{2em}
\setstretch{.5}
{\PaliGlossB{8. Ascetics and Brahmins}}\\
\end{addmargin}
\end{absolutelynopagebreak}

\begin{absolutelynopagebreak}
\setstretch{.7}
{\PaliGlossA{71. jarāmaraṇasutta}}\\
\begin{addmargin}[1em]{2em}
\setstretch{.5}
{\PaliGlossB{71. Old Age and Death}}\\
\end{addmargin}
\end{absolutelynopagebreak}

\begin{absolutelynopagebreak}
\setstretch{.7}
{\PaliGlossA{evaṃ me sutaṃ—}}\\
\begin{addmargin}[1em]{2em}
\setstretch{.5}
{\PaliGlossB{So I have heard.}}\\
\end{addmargin}
\end{absolutelynopagebreak}

\begin{absolutelynopagebreak}
\setstretch{.7}
{\PaliGlossA{ekaṃ samayaṃ bhagavā sāvatthiyaṃ viharati jetavane anāthapiṇḍikassa ārāme.}}\\
\begin{addmargin}[1em]{2em}
\setstretch{.5}
{\PaliGlossB{At one time the Buddha was staying near Sāvatthī in Jeta’s Grove, Anāthapiṇḍika’s monastery.}}\\
\end{addmargin}
\end{absolutelynopagebreak}

\begin{absolutelynopagebreak}
\setstretch{.7}
{\PaliGlossA{tatra kho bhagavā … pe …}}\\
\begin{addmargin}[1em]{2em}
\setstretch{.5}
{\PaliGlossB{    -}}\\
\end{addmargin}
\end{absolutelynopagebreak}

\begin{absolutelynopagebreak}
\setstretch{.7}
{\PaliGlossA{“ye hi keci, bhikkhave, samaṇā vā brāhmaṇā vā jarāmaraṇaṃ nappajānanti, jarāmaraṇasamudayaṃ nappajānanti, jarāmaraṇanirodhaṃ nappajānanti, jarāmaraṇanirodhagāminiṃ paṭipadaṃ nappajānanti,}}\\
\begin{addmargin}[1em]{2em}
\setstretch{.5}
{\PaliGlossB{“Mendicants, there are ascetics and brahmins who don’t understand old age and death, their origin, their cessation, and the practice that leads to their cessation.}}\\
\end{addmargin}
\end{absolutelynopagebreak}

\begin{absolutelynopagebreak}
\setstretch{.7}
{\PaliGlossA{na me te, bhikkhave, samaṇā vā brāhmaṇā vā samaṇesu vā samaṇasammatā brāhmaṇesu vā brāhmaṇasammatā, na ca pana te āyasmanto sāmaññatthaṃ vā brahmaññatthaṃ vā diṭṭheva dhamme sayaṃ abhiññā sacchikatvā upasampajja viharanti.}}\\
\begin{addmargin}[1em]{2em}
\setstretch{.5}
{\PaliGlossB{I don’t regard them as true ascetics and brahmins. Those venerables don’t realize the goal of life as an ascetic or brahmin, and don’t live having realized it with their own insight.}}\\
\end{addmargin}
\end{absolutelynopagebreak}

\begin{absolutelynopagebreak}
\setstretch{.7}
{\PaliGlossA{ye ca kho keci, bhikkhave, samaṇā vā brāhmaṇā vā jarāmaraṇaṃ pajānanti … pe … paṭipadaṃ pajānanti,}}\\
\begin{addmargin}[1em]{2em}
\setstretch{.5}
{\PaliGlossB{There are ascetics and brahmins who do understand old age and death, their origin, their cessation, and the practice that leads to their cessation.}}\\
\end{addmargin}
\end{absolutelynopagebreak}

\begin{absolutelynopagebreak}
\setstretch{.7}
{\PaliGlossA{te kho me, bhikkhave, samaṇā vā brāhmaṇā vā samaṇesu ceva samaṇasammatā brāhmaṇesu ca brāhmaṇasammatā, te ca panāyasmanto sāmaññatthañca brahmaññatthañca diṭṭheva dhamme sayaṃ abhiññā sacchikatvā upasampajja viharantī”ti.}}\\
\begin{addmargin}[1em]{2em}
\setstretch{.5}
{\PaliGlossB{I regard them as true ascetics and brahmins. Those venerables realize the goal of life as an ascetic or brahmin, and live having realized it with their own insight.”}}\\
\end{addmargin}
\end{absolutelynopagebreak}

\begin{absolutelynopagebreak}
\setstretch{.7}
{\PaliGlossA{(suttanto eko.)}}\\
\begin{addmargin}[1em]{2em}
\setstretch{.5}
{\PaliGlossB{    -}}\\
\end{addmargin}
\end{absolutelynopagebreak}

\begin{absolutelynopagebreak}
\setstretch{.7}
{\PaliGlossA{paṭhamaṃ.}}\\
\begin{addmargin}[1em]{2em}
\setstretch{.5}
{\PaliGlossB{    -}}\\
\end{addmargin}
\end{absolutelynopagebreak}
