
\begin{absolutelynopagebreak}
\setstretch{.7}
{\PaliGlossA{saṃyutta nikāya 12}}\\
\begin{addmargin}[1em]{2em}
\setstretch{.5}
{\PaliGlossB{Linked Discourses 12}}\\
\end{addmargin}
\end{absolutelynopagebreak}

\begin{absolutelynopagebreak}
\setstretch{.7}
{\PaliGlossA{4. kaḷārakhattiyavagga}}\\
\begin{addmargin}[1em]{2em}
\setstretch{.5}
{\PaliGlossB{4. Kaḷāra the Aristocrat}}\\
\end{addmargin}
\end{absolutelynopagebreak}

\begin{absolutelynopagebreak}
\setstretch{.7}
{\PaliGlossA{35. avijjāpaccayasutta}}\\
\begin{addmargin}[1em]{2em}
\setstretch{.5}
{\PaliGlossB{35. Ignorance is a Condition}}\\
\end{addmargin}
\end{absolutelynopagebreak}

\begin{absolutelynopagebreak}
\setstretch{.7}
{\PaliGlossA{sāvatthiyaṃ viharati.}}\\
\begin{addmargin}[1em]{2em}
\setstretch{.5}
{\PaliGlossB{At Sāvatthī.}}\\
\end{addmargin}
\end{absolutelynopagebreak}

\begin{absolutelynopagebreak}
\setstretch{.7}
{\PaliGlossA{“avijjāpaccayā, bhikkhave, saṅkhārā;}}\\
\begin{addmargin}[1em]{2em}
\setstretch{.5}
{\PaliGlossB{“Ignorance is a condition for choices.}}\\
\end{addmargin}
\end{absolutelynopagebreak}

\begin{absolutelynopagebreak}
\setstretch{.7}
{\PaliGlossA{saṅkhārapaccayā viññāṇaṃ … pe …}}\\
\begin{addmargin}[1em]{2em}
\setstretch{.5}
{\PaliGlossB{Choices are a condition for consciousness. …}}\\
\end{addmargin}
\end{absolutelynopagebreak}

\begin{absolutelynopagebreak}
\setstretch{.7}
{\PaliGlossA{evametassa kevalassa dukkhakkhandhassa samudayo hotī”ti.}}\\
\begin{addmargin}[1em]{2em}
\setstretch{.5}
{\PaliGlossB{That is how this entire mass of suffering originates.”}}\\
\end{addmargin}
\end{absolutelynopagebreak}

\begin{absolutelynopagebreak}
\setstretch{.7}
{\PaliGlossA{evaṃ vutte, aññataro bhikkhu bhagavantaṃ etadavoca:}}\\
\begin{addmargin}[1em]{2em}
\setstretch{.5}
{\PaliGlossB{When this was said, one of the mendicants asked the Buddha,}}\\
\end{addmargin}
\end{absolutelynopagebreak}

\begin{absolutelynopagebreak}
\setstretch{.7}
{\PaliGlossA{“‘katamaṃ nu kho, bhante, jarāmaraṇaṃ, kassa ca panidaṃ jarāmaraṇan’ti?}}\\
\begin{addmargin}[1em]{2em}
\setstretch{.5}
{\PaliGlossB{“What are old age and death, sir, and who do they belong to?”}}\\
\end{addmargin}
\end{absolutelynopagebreak}

\begin{absolutelynopagebreak}
\setstretch{.7}
{\PaliGlossA{‘no kallo pañho’ti bhagavā avoca, ‘katamaṃ jarāmaraṇaṃ, kassa ca panidaṃ jarāmaraṇan’ti iti vā, bhikkhu, yo vadeyya, ‘aññaṃ jarāmaraṇaṃ aññassa ca panidaṃ jarāmaraṇan’ti, iti vā, bhikkhu, yo vadeyya, ubhayametaṃ ekatthaṃ byañjanameva nānaṃ.}}\\
\begin{addmargin}[1em]{2em}
\setstretch{.5}
{\PaliGlossB{“That’s not a fitting question,” said the Buddha. “You might say, ‘What are old age and death, and who do they belong to?’ Or you might say, ‘Old age and death are one thing, who they belong to is another.’ But both of these mean the same thing, only the phrasing differs.}}\\
\end{addmargin}
\end{absolutelynopagebreak}

\begin{absolutelynopagebreak}
\setstretch{.7}
{\PaliGlossA{taṃ jīvaṃ taṃ sarīranti vā, bhikkhu, diṭṭhiyā sati brahmacariyavāso na hoti.}}\\
\begin{addmargin}[1em]{2em}
\setstretch{.5}
{\PaliGlossB{Mendicant, if you have the view that the soul and the body are the same thing, there is no living of the spiritual life.}}\\
\end{addmargin}
\end{absolutelynopagebreak}

\begin{absolutelynopagebreak}
\setstretch{.7}
{\PaliGlossA{aññaṃ jīvaṃ aññaṃ sarīranti vā, bhikkhu, diṭṭhiyā sati brahmacariyavāso na hoti.}}\\
\begin{addmargin}[1em]{2em}
\setstretch{.5}
{\PaliGlossB{If you have the view that the soul and the body are different things, there is no living of the spiritual life.}}\\
\end{addmargin}
\end{absolutelynopagebreak}

\begin{absolutelynopagebreak}
\setstretch{.7}
{\PaliGlossA{ete te, bhikkhu, ubho ante anupagamma majjhena tathāgato dhammaṃ deseti:}}\\
\begin{addmargin}[1em]{2em}
\setstretch{.5}
{\PaliGlossB{Avoiding these two extremes, the Realized One teaches by the middle way:}}\\
\end{addmargin}
\end{absolutelynopagebreak}

\begin{absolutelynopagebreak}
\setstretch{.7}
{\PaliGlossA{‘jātipaccayā jarāmaraṇan’”ti.}}\\
\begin{addmargin}[1em]{2em}
\setstretch{.5}
{\PaliGlossB{‘Rebirth is a condition for old age and death.’”}}\\
\end{addmargin}
\end{absolutelynopagebreak}

\begin{absolutelynopagebreak}
\setstretch{.7}
{\PaliGlossA{“katamā nu kho, bhante, jāti, kassa ca panāyaṃ jātī”ti?}}\\
\begin{addmargin}[1em]{2em}
\setstretch{.5}
{\PaliGlossB{“What is rebirth, sir, and who does it belong to?”}}\\
\end{addmargin}
\end{absolutelynopagebreak}

\begin{absolutelynopagebreak}
\setstretch{.7}
{\PaliGlossA{“no kallo pañho”ti bhagavā avoca, “‘katamā jāti, kassa ca panāyaṃ jātī’ti iti vā, bhikkhu, yo vadeyya, ‘aññā jāti aññassa ca panāyaṃ jātī’ti iti vā, bhikkhu, yo vadeyya, ubhayametaṃ ekatthaṃ byañjanameva nānaṃ.}}\\
\begin{addmargin}[1em]{2em}
\setstretch{.5}
{\PaliGlossB{“That’s not a fitting question,” said the Buddha. “You might say, ‘What is rebirth, and who does it belong to?’ Or you might say, ‘Rebirth is one thing, who it belongs to is another.’ But both of these mean the same thing, only the phrasing differs.}}\\
\end{addmargin}
\end{absolutelynopagebreak}

\begin{absolutelynopagebreak}
\setstretch{.7}
{\PaliGlossA{taṃ jīvaṃ taṃ sarīranti vā, bhikkhu, diṭṭhiyā sati brahmacariyavāso na hoti.}}\\
\begin{addmargin}[1em]{2em}
\setstretch{.5}
{\PaliGlossB{Mendicant, if you have the view that the soul and the body are the same thing, there is no living of the spiritual life.}}\\
\end{addmargin}
\end{absolutelynopagebreak}

\begin{absolutelynopagebreak}
\setstretch{.7}
{\PaliGlossA{aññaṃ jīvaṃ aññaṃ sarīranti vā, bhikkhu, diṭṭhiyā sati brahmacariyavāso na hoti.}}\\
\begin{addmargin}[1em]{2em}
\setstretch{.5}
{\PaliGlossB{If you have the view that the soul and the body are different things, there is no living of the spiritual life.}}\\
\end{addmargin}
\end{absolutelynopagebreak}

\begin{absolutelynopagebreak}
\setstretch{.7}
{\PaliGlossA{ete te, bhikkhu, ubho ante anupagamma majjhena tathāgato dhammaṃ deseti:}}\\
\begin{addmargin}[1em]{2em}
\setstretch{.5}
{\PaliGlossB{Avoiding these two extremes, the Realized One teaches by the middle way:}}\\
\end{addmargin}
\end{absolutelynopagebreak}

\begin{absolutelynopagebreak}
\setstretch{.7}
{\PaliGlossA{‘bhavapaccayā jātī’”ti.}}\\
\begin{addmargin}[1em]{2em}
\setstretch{.5}
{\PaliGlossB{‘Continued existence is a condition for rebirth.’”}}\\
\end{addmargin}
\end{absolutelynopagebreak}

\begin{absolutelynopagebreak}
\setstretch{.7}
{\PaliGlossA{“katamo nu kho, bhante, bhavo, kassa ca panāyaṃ bhavo”ti?}}\\
\begin{addmargin}[1em]{2em}
\setstretch{.5}
{\PaliGlossB{“What is continued existence, sir, and who is it for?”}}\\
\end{addmargin}
\end{absolutelynopagebreak}

\begin{absolutelynopagebreak}
\setstretch{.7}
{\PaliGlossA{“no kallo pañho”ti bhagavā avoca, “‘katamo bhavo, kassa ca panāyaṃ bhavo’ti iti vā, bhikkhu, yo vadeyya, ‘añño bhavo aññassa ca panāyaṃ bhavo’ti iti vā, bhikkhu, yo vadeyya, ubhayametaṃ ekatthaṃ byañjanameva nānaṃ.}}\\
\begin{addmargin}[1em]{2em}
\setstretch{.5}
{\PaliGlossB{“That’s not a fitting question,” said the Buddha. “You might say, ‘What is continued existence, and who does it belong to?’ Or you might say, ‘Continued existence is one thing, who it belongs to is another.’ But both of these mean the same thing, only the phrasing differs.}}\\
\end{addmargin}
\end{absolutelynopagebreak}

\begin{absolutelynopagebreak}
\setstretch{.7}
{\PaliGlossA{taṃ jīvaṃ taṃ sarīranti vā, bhikkhu, diṭṭhiyā sati brahmacariyavāso na hoti;}}\\
\begin{addmargin}[1em]{2em}
\setstretch{.5}
{\PaliGlossB{Mendicant, if you have the view that the soul and the body are identical, there is no living of the spiritual life.}}\\
\end{addmargin}
\end{absolutelynopagebreak}

\begin{absolutelynopagebreak}
\setstretch{.7}
{\PaliGlossA{aññaṃ jīvaṃ aññaṃ sarīranti vā, bhikkhu, diṭṭhiyā sati brahmacariyavāso na hoti.}}\\
\begin{addmargin}[1em]{2em}
\setstretch{.5}
{\PaliGlossB{If you have the view that the soul and the body are different things, there is no living of the spiritual life.}}\\
\end{addmargin}
\end{absolutelynopagebreak}

\begin{absolutelynopagebreak}
\setstretch{.7}
{\PaliGlossA{ete te, bhikkhu, ubho ante anupagamma majjhena tathāgato dhammaṃ deseti:}}\\
\begin{addmargin}[1em]{2em}
\setstretch{.5}
{\PaliGlossB{Avoiding these two extremes, the Realized One teaches by the middle way:}}\\
\end{addmargin}
\end{absolutelynopagebreak}

\begin{absolutelynopagebreak}
\setstretch{.7}
{\PaliGlossA{‘upādānapaccayā bhavo’ti … pe …}}\\
\begin{addmargin}[1em]{2em}
\setstretch{.5}
{\PaliGlossB{‘Grasping is a condition for continued existence.’ …}}\\
\end{addmargin}
\end{absolutelynopagebreak}

\begin{absolutelynopagebreak}
\setstretch{.7}
{\PaliGlossA{‘taṇhāpaccayā upādānanti …}}\\
\begin{addmargin}[1em]{2em}
\setstretch{.5}
{\PaliGlossB{‘Craving is a condition for grasping.’ …}}\\
\end{addmargin}
\end{absolutelynopagebreak}

\begin{absolutelynopagebreak}
\setstretch{.7}
{\PaliGlossA{vedanāpaccayā taṇhāti …}}\\
\begin{addmargin}[1em]{2em}
\setstretch{.5}
{\PaliGlossB{‘Feeling is a condition for craving.’ …}}\\
\end{addmargin}
\end{absolutelynopagebreak}

\begin{absolutelynopagebreak}
\setstretch{.7}
{\PaliGlossA{phassapaccayā vedanāti …}}\\
\begin{addmargin}[1em]{2em}
\setstretch{.5}
{\PaliGlossB{‘Contact is a condition for feeling.’ …}}\\
\end{addmargin}
\end{absolutelynopagebreak}

\begin{absolutelynopagebreak}
\setstretch{.7}
{\PaliGlossA{saḷāyatanapaccayā phassoti …}}\\
\begin{addmargin}[1em]{2em}
\setstretch{.5}
{\PaliGlossB{‘The six sense fields are conditions for contact.’ …}}\\
\end{addmargin}
\end{absolutelynopagebreak}

\begin{absolutelynopagebreak}
\setstretch{.7}
{\PaliGlossA{nāmarūpapaccayā saḷāyatananti …}}\\
\begin{addmargin}[1em]{2em}
\setstretch{.5}
{\PaliGlossB{‘Name and form are conditions for the six sense fields.’ …}}\\
\end{addmargin}
\end{absolutelynopagebreak}

\begin{absolutelynopagebreak}
\setstretch{.7}
{\PaliGlossA{viññāṇapaccayā nāmarūpanti …}}\\
\begin{addmargin}[1em]{2em}
\setstretch{.5}
{\PaliGlossB{‘Consciousness is a condition for name and form.’ …}}\\
\end{addmargin}
\end{absolutelynopagebreak}

\begin{absolutelynopagebreak}
\setstretch{.7}
{\PaliGlossA{saṅkhārapaccayā viññāṇan’”ti.}}\\
\begin{addmargin}[1em]{2em}
\setstretch{.5}
{\PaliGlossB{‘Choices are a condition for consciousness.’”}}\\
\end{addmargin}
\end{absolutelynopagebreak}

\begin{absolutelynopagebreak}
\setstretch{.7}
{\PaliGlossA{“katame nu kho, bhante, saṅkhārā, kassa ca panime saṅkhārā”ti?}}\\
\begin{addmargin}[1em]{2em}
\setstretch{.5}
{\PaliGlossB{“What are choices, sir, and who do they belong to?”}}\\
\end{addmargin}
\end{absolutelynopagebreak}

\begin{absolutelynopagebreak}
\setstretch{.7}
{\PaliGlossA{“no kallo pañho”ti bhagavā avoca, “‘katame saṅkhārā kassa ca panime saṅkhārā’ti iti vā, bhikkhu, yo vadeyya, ‘aññe saṅkhārā aññassa ca panime saṅkhārā’ti iti vā, bhikkhu, yo vadeyya, ubhayametaṃ ekatthaṃ byañjanameva nānaṃ.}}\\
\begin{addmargin}[1em]{2em}
\setstretch{.5}
{\PaliGlossB{“That’s not a fitting question,” said the Buddha. “You might say, ‘What are choices, and who do they belong to?’ Or you might say, ‘Choices are one thing, who they belong to is another.’ But both of these mean the same thing, only the phrasing differs.}}\\
\end{addmargin}
\end{absolutelynopagebreak}

\begin{absolutelynopagebreak}
\setstretch{.7}
{\PaliGlossA{taṃ jīvaṃ taṃ sarīranti vā, bhikkhu, diṭṭhiyā sati brahmacariyavāso na hoti;}}\\
\begin{addmargin}[1em]{2em}
\setstretch{.5}
{\PaliGlossB{Mendicant, if you have the view that the soul and the body are the same thing, there is no living of the spiritual life.}}\\
\end{addmargin}
\end{absolutelynopagebreak}

\begin{absolutelynopagebreak}
\setstretch{.7}
{\PaliGlossA{aññaṃ jīvaṃ aññaṃ sarīranti vā, bhikkhu, diṭṭhiyā sati brahmacariyavāso na hoti.}}\\
\begin{addmargin}[1em]{2em}
\setstretch{.5}
{\PaliGlossB{If you have the view that the soul and the body are different things, there is no living of the spiritual life.}}\\
\end{addmargin}
\end{absolutelynopagebreak}

\begin{absolutelynopagebreak}
\setstretch{.7}
{\PaliGlossA{ete te, bhikkhu, ubho ante anupagamma majjhena tathāgato dhammaṃ deseti:}}\\
\begin{addmargin}[1em]{2em}
\setstretch{.5}
{\PaliGlossB{Avoiding these two extremes, the Realized One teaches by the middle way:}}\\
\end{addmargin}
\end{absolutelynopagebreak}

\begin{absolutelynopagebreak}
\setstretch{.7}
{\PaliGlossA{‘avijjāpaccayā saṅkhārā’”ti.}}\\
\begin{addmargin}[1em]{2em}
\setstretch{.5}
{\PaliGlossB{‘Ignorance is a condition for choices.’}}\\
\end{addmargin}
\end{absolutelynopagebreak}

\begin{absolutelynopagebreak}
\setstretch{.7}
{\PaliGlossA{“avijjāya tveva, bhikkhu, asesavirāganirodhā yānissa tāni visūkāyikāni visevitāni vipphanditāni kānici kānici.}}\\
\begin{addmargin}[1em]{2em}
\setstretch{.5}
{\PaliGlossB{When ignorance fades away and ceases with nothing left over, then any tricks, dodges, and evasions are given up:}}\\
\end{addmargin}
\end{absolutelynopagebreak}

\begin{absolutelynopagebreak}
\setstretch{.7}
{\PaliGlossA{‘katamaṃ jarāmaraṇaṃ, kassa ca panidaṃ jarāmaraṇaṃ’ iti vā, ‘aññaṃ jarāmaraṇaṃ, aññassa ca panidaṃ jarāmaraṇaṃ’ iti vā, ‘taṃ jīvaṃ taṃ sarīraṃ’ iti vā, ‘aññaṃ jīvaṃ aññaṃ sarīraṃ’ iti vā.}}\\
\begin{addmargin}[1em]{2em}
\setstretch{.5}
{\PaliGlossB{‘What are old age and death, and who do they belong to?’ or ‘old age and death are one thing, who they belong to is another’, or ‘the soul and the body are the same thing’, or ‘the soul and the body are different things.’}}\\
\end{addmargin}
\end{absolutelynopagebreak}

\begin{absolutelynopagebreak}
\setstretch{.7}
{\PaliGlossA{sabbānissa tāni pahīnāni bhavanti ucchinnamūlāni tālāvatthukatāni anabhāvaṅkatāni āyatiṃ anuppādadhammāni.}}\\
\begin{addmargin}[1em]{2em}
\setstretch{.5}
{\PaliGlossB{These are all cut off at the root, made like a palm stump, obliterated, and unable to arise in the future.}}\\
\end{addmargin}
\end{absolutelynopagebreak}

\begin{absolutelynopagebreak}
\setstretch{.7}
{\PaliGlossA{avijjāya tveva, bhikkhu, asesavirāganirodhā yānissa tāni visūkāyikāni visevitāni vipphanditāni kānici kānici.}}\\
\begin{addmargin}[1em]{2em}
\setstretch{.5}
{\PaliGlossB{When ignorance fades away and ceases with nothing left over, then any tricks, dodges, and evasions are given up:}}\\
\end{addmargin}
\end{absolutelynopagebreak}

\begin{absolutelynopagebreak}
\setstretch{.7}
{\PaliGlossA{‘katamā jāti, kassa ca panāyaṃ jāti’ iti vā, ‘aññā jāti, aññassa ca panāyaṃ jāti’ iti vā, ‘taṃ jīvaṃ taṃ sarīraṃ’ iti vā, ‘aññaṃ jīvaṃ aññaṃ sarīraṃ’ iti vā.}}\\
\begin{addmargin}[1em]{2em}
\setstretch{.5}
{\PaliGlossB{‘What is rebirth, and who does it belong to?’ or ‘rebirth is one thing, who it belongs to is another’, or ‘the soul and the body are the same thing’, or ‘the soul and the body are different things.’}}\\
\end{addmargin}
\end{absolutelynopagebreak}

\begin{absolutelynopagebreak}
\setstretch{.7}
{\PaliGlossA{sabbānissa tāni pahīnāni bhavanti ucchinnamūlāni tālāvatthukatāni anabhāvaṅkatāni āyatiṃ anuppādadhammāni.}}\\
\begin{addmargin}[1em]{2em}
\setstretch{.5}
{\PaliGlossB{These are all cut off at the root, made like a palm stump, obliterated, and unable to arise in the future.}}\\
\end{addmargin}
\end{absolutelynopagebreak}

\begin{absolutelynopagebreak}
\setstretch{.7}
{\PaliGlossA{avijjāya tveva, bhikkhu, asesavirāganirodhā yānissa tāni visūkāyikāni visevitāni vipphanditāni kānici kānici.}}\\
\begin{addmargin}[1em]{2em}
\setstretch{.5}
{\PaliGlossB{When ignorance fades away and ceases with nothing left over, then any tricks, dodges, and evasions are given up:}}\\
\end{addmargin}
\end{absolutelynopagebreak}

\begin{absolutelynopagebreak}
\setstretch{.7}
{\PaliGlossA{katamo bhavo … pe …}}\\
\begin{addmargin}[1em]{2em}
\setstretch{.5}
{\PaliGlossB{‘What is continued existence …’}}\\
\end{addmargin}
\end{absolutelynopagebreak}

\begin{absolutelynopagebreak}
\setstretch{.7}
{\PaliGlossA{katamaṃ upādānaṃ …}}\\
\begin{addmargin}[1em]{2em}
\setstretch{.5}
{\PaliGlossB{‘What is grasping …’}}\\
\end{addmargin}
\end{absolutelynopagebreak}

\begin{absolutelynopagebreak}
\setstretch{.7}
{\PaliGlossA{katamā taṇhā …}}\\
\begin{addmargin}[1em]{2em}
\setstretch{.5}
{\PaliGlossB{‘What is craving …’}}\\
\end{addmargin}
\end{absolutelynopagebreak}

\begin{absolutelynopagebreak}
\setstretch{.7}
{\PaliGlossA{katamā vedanā …}}\\
\begin{addmargin}[1em]{2em}
\setstretch{.5}
{\PaliGlossB{‘What is feeling …’}}\\
\end{addmargin}
\end{absolutelynopagebreak}

\begin{absolutelynopagebreak}
\setstretch{.7}
{\PaliGlossA{katamo phasso …}}\\
\begin{addmargin}[1em]{2em}
\setstretch{.5}
{\PaliGlossB{‘What is contact …’}}\\
\end{addmargin}
\end{absolutelynopagebreak}

\begin{absolutelynopagebreak}
\setstretch{.7}
{\PaliGlossA{katamaṃ saḷāyatanaṃ …}}\\
\begin{addmargin}[1em]{2em}
\setstretch{.5}
{\PaliGlossB{‘What are the six sense fields …’}}\\
\end{addmargin}
\end{absolutelynopagebreak}

\begin{absolutelynopagebreak}
\setstretch{.7}
{\PaliGlossA{katamaṃ nāmarūpaṃ …}}\\
\begin{addmargin}[1em]{2em}
\setstretch{.5}
{\PaliGlossB{‘What are name and form …’}}\\
\end{addmargin}
\end{absolutelynopagebreak}

\begin{absolutelynopagebreak}
\setstretch{.7}
{\PaliGlossA{katamaṃ viññāṇaṃ … pe ….}}\\
\begin{addmargin}[1em]{2em}
\setstretch{.5}
{\PaliGlossB{‘What is consciousness …’}}\\
\end{addmargin}
\end{absolutelynopagebreak}

\begin{absolutelynopagebreak}
\setstretch{.7}
{\PaliGlossA{avijjāya tveva, bhikkhu, asesavirāganirodhā yānissa tāni visūkāyikāni visevitāni vipphanditāni kānici kānici.}}\\
\begin{addmargin}[1em]{2em}
\setstretch{.5}
{\PaliGlossB{When ignorance fades away and ceases with nothing left over, then any tricks, dodges, and evasions are given up:}}\\
\end{addmargin}
\end{absolutelynopagebreak}

\begin{absolutelynopagebreak}
\setstretch{.7}
{\PaliGlossA{‘katame saṅkhārā, kassa ca panime saṅkhārā’ iti vā, ‘aññe saṅkhārā, aññassa ca panime saṅkhārā’ iti vā, ‘taṃ jīvaṃ taṃ sarīraṃ’ iti vā, ‘aññaṃ jīvaṃ, aññaṃ sarīraṃ’ iti vā.}}\\
\begin{addmargin}[1em]{2em}
\setstretch{.5}
{\PaliGlossB{‘What are choices, and who do they belong to?’ or ‘choices are one thing, who they belong to is another’, or ‘the soul and the body are the same thing’, or ‘the soul and the body are different things.’}}\\
\end{addmargin}
\end{absolutelynopagebreak}

\begin{absolutelynopagebreak}
\setstretch{.7}
{\PaliGlossA{sabbānissa tāni pahīnāni bhavanti ucchinnamūlāni tālāvatthukatāni anabhāvaṅkatāni āyatiṃ anuppādadhammānī”ti.}}\\
\begin{addmargin}[1em]{2em}
\setstretch{.5}
{\PaliGlossB{These are all cut off at the root, made like a palm stump, obliterated, and unable to arise in the future.”}}\\
\end{addmargin}
\end{absolutelynopagebreak}

\begin{absolutelynopagebreak}
\setstretch{.7}
{\PaliGlossA{pañcamaṃ.}}\\
\begin{addmargin}[1em]{2em}
\setstretch{.5}
{\PaliGlossB{    -}}\\
\end{addmargin}
\end{absolutelynopagebreak}
