
\begin{absolutelynopagebreak}
\setstretch{.7}
{\PaliGlossA{saṃyutta nikāya 45}}\\
\begin{addmargin}[1em]{2em}
\setstretch{.5}
{\PaliGlossB{Linked Discourses 45}}\\
\end{addmargin}
\end{absolutelynopagebreak}

\begin{absolutelynopagebreak}
\setstretch{.7}
{\PaliGlossA{11. appamādapeyyālavagga}}\\
\begin{addmargin}[1em]{2em}
\setstretch{.5}
{\PaliGlossB{11. Abbreviated Texts on Diligence}}\\
\end{addmargin}
\end{absolutelynopagebreak}

\begin{absolutelynopagebreak}
\setstretch{.7}
{\PaliGlossA{139. tathāgatasutta}}\\
\begin{addmargin}[1em]{2em}
\setstretch{.5}
{\PaliGlossB{139. The Realized One}}\\
\end{addmargin}
\end{absolutelynopagebreak}

\begin{absolutelynopagebreak}
\setstretch{.7}
{\PaliGlossA{sāvatthinidānaṃ.}}\\
\begin{addmargin}[1em]{2em}
\setstretch{.5}
{\PaliGlossB{At Sāvatthī.}}\\
\end{addmargin}
\end{absolutelynopagebreak}

\begin{absolutelynopagebreak}
\setstretch{.7}
{\PaliGlossA{“yāvatā, bhikkhave, sattā apadā vā dvipadā vā catuppadā vā bahuppadā vā rūpino vā arūpino vā saññino vā asaññino vā nevasaññīnāsaññino vā, tathāgato tesaṃ aggamakkhāyati arahaṃ sammāsambuddho;}}\\
\begin{addmargin}[1em]{2em}
\setstretch{.5}
{\PaliGlossB{“Mendicants, the Realized One, the perfected one, the fully awakened Buddha, is said to be the best of all sentient beings—be they footless, with two feet, four feet, or many feet; with form or formless; with perception or without perception or with neither perception nor non-perception.}}\\
\end{addmargin}
\end{absolutelynopagebreak}

\begin{absolutelynopagebreak}
\setstretch{.7}
{\PaliGlossA{evameva kho, bhikkhave, ye keci kusalā dhammā, sabbe te appamādamūlakā appamādasamosaraṇā;}}\\
\begin{addmargin}[1em]{2em}
\setstretch{.5}
{\PaliGlossB{In the same way, all skillful qualities are rooted in diligence and meet at diligence,}}\\
\end{addmargin}
\end{absolutelynopagebreak}

\begin{absolutelynopagebreak}
\setstretch{.7}
{\PaliGlossA{appamādo tesaṃ dhammānaṃ aggamakkhāyati.}}\\
\begin{addmargin}[1em]{2em}
\setstretch{.5}
{\PaliGlossB{and diligence is said to be the best of them.}}\\
\end{addmargin}
\end{absolutelynopagebreak}

\begin{absolutelynopagebreak}
\setstretch{.7}
{\PaliGlossA{appamattassetaṃ, bhikkhave, bhikkhuno pāṭikaṅkhaṃ—ariyaṃ aṭṭhaṅgikaṃ maggaṃ bhāvessati ariyaṃ aṭṭhaṅgikaṃ maggaṃ bahulīkarissati.}}\\
\begin{addmargin}[1em]{2em}
\setstretch{.5}
{\PaliGlossB{A mendicant who is diligent can expect to develop and cultivate the noble eightfold path.}}\\
\end{addmargin}
\end{absolutelynopagebreak}

\begin{absolutelynopagebreak}
\setstretch{.7}
{\PaliGlossA{kathañca, bhikkhave, bhikkhu appamatto ariyaṃ aṭṭhaṅgikaṃ maggaṃ bhāveti ariyaṃ aṭṭhaṅgikaṃ maggaṃ bahulīkaroti?}}\\
\begin{addmargin}[1em]{2em}
\setstretch{.5}
{\PaliGlossB{And how does a mendicant who is diligent develop and cultivate the noble eightfold path?}}\\
\end{addmargin}
\end{absolutelynopagebreak}

\begin{absolutelynopagebreak}
\setstretch{.7}
{\PaliGlossA{idha, bhikkhave, bhikkhu sammādiṭṭhiṃ bhāveti vivekanissitaṃ virāganissitaṃ nirodhanissitaṃ vossaggapariṇāmiṃ … pe … sammāsamādhiṃ bhāveti vivekanissitaṃ virāganissitaṃ nirodhanissitaṃ vossaggapariṇāmiṃ.}}\\
\begin{addmargin}[1em]{2em}
\setstretch{.5}
{\PaliGlossB{It’s when a mendicant develops right view, right thought, right speech, right action, right livelihood, right effort, right mindfulness, and right immersion, which rely on seclusion, fading away, and cessation, and ripen as letting go.}}\\
\end{addmargin}
\end{absolutelynopagebreak}

\begin{absolutelynopagebreak}
\setstretch{.7}
{\PaliGlossA{evaṃ kho, bhikkhave, bhikkhu appamatto ariyaṃ aṭṭhaṅgikaṃ maggaṃ bhāveti ariyaṃ aṭṭhaṅgikaṃ maggaṃ bahulīkarotīti.}}\\
\begin{addmargin}[1em]{2em}
\setstretch{.5}
{\PaliGlossB{That’s how a mendicant who is diligent develops and cultivates the noble eightfold path.}}\\
\end{addmargin}
\end{absolutelynopagebreak}

\begin{absolutelynopagebreak}
\setstretch{.7}
{\PaliGlossA{yāvatā, bhikkhave, sattā apadā vā dvipadā vā catuppadā vā bahuppadā vā rūpino vā arūpino vā saññino vā asaññino vā nevasaññīnāsaññino vā, tathāgato tesaṃ aggamakkhāyati arahaṃ sammāsambuddho;}}\\
\begin{addmargin}[1em]{2em}
\setstretch{.5}
{\PaliGlossB{Mendicants, the Realized One, the perfected one, the fully awakened Buddha, is said to be the best of all sentient beings—be they footless, with two feet, four feet, or many feet; with form or formless; with perception or without perception or with neither perception nor non-perception.}}\\
\end{addmargin}
\end{absolutelynopagebreak}

\begin{absolutelynopagebreak}
\setstretch{.7}
{\PaliGlossA{evameva kho, bhikkhave, ye keci kusalā dhammā, sabbe te appamādamūlakā appamādasamosaraṇā; appamādo tesaṃ dhammānaṃ aggamakkhāyati.}}\\
\begin{addmargin}[1em]{2em}
\setstretch{.5}
{\PaliGlossB{In the same way, all skillful qualities are rooted in diligence and meet at diligence, and diligence is said to be the best of them.}}\\
\end{addmargin}
\end{absolutelynopagebreak}

\begin{absolutelynopagebreak}
\setstretch{.7}
{\PaliGlossA{appamattassetaṃ, bhikkhave, bhikkhuno pāṭikaṅkhaṃ—ariyaṃ aṭṭhaṅgikaṃ maggaṃ bhāvessati, ariyaṃ aṭṭhaṅgikaṃ maggaṃ bahulīkarissati.}}\\
\begin{addmargin}[1em]{2em}
\setstretch{.5}
{\PaliGlossB{A mendicant who is diligent can expect to develop and cultivate the noble eightfold path.}}\\
\end{addmargin}
\end{absolutelynopagebreak}

\begin{absolutelynopagebreak}
\setstretch{.7}
{\PaliGlossA{kathañca, bhikkhave, bhikkhu appamatto ariyaṃ aṭṭhaṅgikaṃ maggaṃ bhāveti ariyaṃ aṭṭhaṅgikaṃ maggaṃ bahulīkaroti?}}\\
\begin{addmargin}[1em]{2em}
\setstretch{.5}
{\PaliGlossB{And how does a mendicant who is diligent develop and cultivate the noble eightfold path?}}\\
\end{addmargin}
\end{absolutelynopagebreak}

\begin{absolutelynopagebreak}
\setstretch{.7}
{\PaliGlossA{idha, bhikkhave, bhikkhu sammādiṭṭhiṃ bhāveti rāgavinayapariyosānaṃ dosavinayapariyosānaṃ mohavinayapariyosānaṃ … pe … sammāsamādhiṃ bhāveti rāgavinayapariyosānaṃ dosavinayapariyosānaṃ mohavinayapariyosānaṃ.}}\\
\begin{addmargin}[1em]{2em}
\setstretch{.5}
{\PaliGlossB{It’s when a mendicant develops right view, right thought, right speech, right action, right livelihood, right effort, right mindfulness, and right immersion, which culminate in the removal of greed, hate, and delusion.}}\\
\end{addmargin}
\end{absolutelynopagebreak}

\begin{absolutelynopagebreak}
\setstretch{.7}
{\PaliGlossA{evaṃ kho, bhikkhave, bhikkhu appamatto ariyaṃ aṭṭhaṅgikaṃ maggaṃ bhāveti ariyaṃ aṭṭhaṅgikaṃ maggaṃ bahulīkarotīti.}}\\
\begin{addmargin}[1em]{2em}
\setstretch{.5}
{\PaliGlossB{That’s how a mendicant who is diligent develops and cultivates the noble eightfold path.}}\\
\end{addmargin}
\end{absolutelynopagebreak}

\begin{absolutelynopagebreak}
\setstretch{.7}
{\PaliGlossA{yāvatā, bhikkhave, sattā apadā vā dvipadā vā catuppadā vā bahuppadā vā rūpino vā arūpino vā saññino vā asaññino vā nevasaññīnāsaññino vā, tathāgato tesaṃ aggamakkhāyati arahaṃ sammāsambuddho;}}\\
\begin{addmargin}[1em]{2em}
\setstretch{.5}
{\PaliGlossB{Mendicants, the Realized One, the perfected one, the fully awakened Buddha, is said to be the best of all sentient beings—be they footless, with two feet, four feet, or many feet; with form or formless; with perception or without perception or with neither perception nor non-perception.}}\\
\end{addmargin}
\end{absolutelynopagebreak}

\begin{absolutelynopagebreak}
\setstretch{.7}
{\PaliGlossA{evameva kho, bhikkhave, ye keci kusalā dhammā, sabbe te appamādamūlakā appamādasamosaraṇā; appamādo tesaṃ dhammānaṃ aggamakkhāyati.}}\\
\begin{addmargin}[1em]{2em}
\setstretch{.5}
{\PaliGlossB{In the same way, all skillful qualities are rooted in diligence and meet at diligence, and diligence is said to be the best of them.}}\\
\end{addmargin}
\end{absolutelynopagebreak}

\begin{absolutelynopagebreak}
\setstretch{.7}
{\PaliGlossA{appamattassetaṃ, bhikkhave, bhikkhuno pāṭikaṅkhaṃ—ariyaṃ aṭṭhaṅgikaṃ maggaṃ bhāvessati ariyaṃ aṭṭhaṅgikaṃ maggaṃ bahulīkarissati.}}\\
\begin{addmargin}[1em]{2em}
\setstretch{.5}
{\PaliGlossB{A mendicant who is diligent can expect to develop and cultivate the noble eightfold path.}}\\
\end{addmargin}
\end{absolutelynopagebreak}

\begin{absolutelynopagebreak}
\setstretch{.7}
{\PaliGlossA{kathañca, bhikkhave, bhikkhu appamatto ariyaṃ aṭṭhaṅgikaṃ maggaṃ bhāveti ariyaṃ aṭṭhaṅgikaṃ maggaṃ bahulīkaroti?}}\\
\begin{addmargin}[1em]{2em}
\setstretch{.5}
{\PaliGlossB{And how does a mendicant who is diligent develop and cultivate the noble eightfold path?}}\\
\end{addmargin}
\end{absolutelynopagebreak}

\begin{absolutelynopagebreak}
\setstretch{.7}
{\PaliGlossA{idha, bhikkhave, bhikkhu sammādiṭṭhiṃ bhāveti amatogadhaṃ amataparāyanaṃ amatapariyosānaṃ … pe … sammāsamādhiṃ bhāveti amatogadhaṃ amataparāyanaṃ amatapariyosānaṃ.}}\\
\begin{addmargin}[1em]{2em}
\setstretch{.5}
{\PaliGlossB{It’s when a mendicant develops right view, right thought, right speech, right action, right livelihood, right effort, right mindfulness, and right immersion, which culminate, finish, and end in the deathless.}}\\
\end{addmargin}
\end{absolutelynopagebreak}

\begin{absolutelynopagebreak}
\setstretch{.7}
{\PaliGlossA{evaṃ kho, bhikkhave, bhikkhu appamatto ariyaṃ aṭṭhaṅgikaṃ maggaṃ bhāveti ariyaṃ aṭṭhaṅgikaṃ maggaṃ bahulīkarotīti.}}\\
\begin{addmargin}[1em]{2em}
\setstretch{.5}
{\PaliGlossB{That’s how a mendicant who is diligent develops and cultivates the noble eightfold path.}}\\
\end{addmargin}
\end{absolutelynopagebreak}

\begin{absolutelynopagebreak}
\setstretch{.7}
{\PaliGlossA{yāvatā, bhikkhave, sattā apadā vā dvipadā vā catuppadā vā bahuppadā vā rūpino vā arūpino vā saññino vā asaññino vā nevasaññīnāsaññino vā, tathāgato tesaṃ aggamakkhāyati arahaṃ sammāsambuddho;}}\\
\begin{addmargin}[1em]{2em}
\setstretch{.5}
{\PaliGlossB{Mendicants, the Realized One, the perfected one, the fully awakened Buddha, is said to be the best of all sentient beings—be they footless, with two feet, four feet, or many feet; with form or formless; with perception or without perception or with neither perception nor non-perception.}}\\
\end{addmargin}
\end{absolutelynopagebreak}

\begin{absolutelynopagebreak}
\setstretch{.7}
{\PaliGlossA{evameva kho, bhikkhave, ye keci kusalā dhammā, sabbe te appamādamūlakā appamādasamosaraṇā;}}\\
\begin{addmargin}[1em]{2em}
\setstretch{.5}
{\PaliGlossB{In the same way, all skillful qualities are rooted in diligence and meet at diligence,}}\\
\end{addmargin}
\end{absolutelynopagebreak}

\begin{absolutelynopagebreak}
\setstretch{.7}
{\PaliGlossA{appamādo tesaṃ dhammānaṃ aggamakkhāyati.}}\\
\begin{addmargin}[1em]{2em}
\setstretch{.5}
{\PaliGlossB{and diligence is said to be the best of them.}}\\
\end{addmargin}
\end{absolutelynopagebreak}

\begin{absolutelynopagebreak}
\setstretch{.7}
{\PaliGlossA{appamattassetaṃ, bhikkhave, bhikkhuno pāṭikaṅkhaṃ—ariyaṃ aṭṭhaṅgikaṃ maggaṃ bhāvessati ariyaṃ aṭṭhaṅgikaṃ maggaṃ bahulīkarissati.}}\\
\begin{addmargin}[1em]{2em}
\setstretch{.5}
{\PaliGlossB{A mendicant who is diligent can expect to develop and cultivate the noble eightfold path.}}\\
\end{addmargin}
\end{absolutelynopagebreak}

\begin{absolutelynopagebreak}
\setstretch{.7}
{\PaliGlossA{kathañca, bhikkhave, bhikkhu appamatto ariyaṃ aṭṭhaṅgikaṃ maggaṃ bhāveti ariyaṃ aṭṭhaṅgikaṃ maggaṃ bahulīkaroti?}}\\
\begin{addmargin}[1em]{2em}
\setstretch{.5}
{\PaliGlossB{And how does a mendicant who is diligent develop and cultivate the noble eightfold path?}}\\
\end{addmargin}
\end{absolutelynopagebreak}

\begin{absolutelynopagebreak}
\setstretch{.7}
{\PaliGlossA{idha, bhikkhave, bhikkhu sammādiṭṭhiṃ bhāveti nibbānaninnaṃ nibbānapoṇaṃ nibbānapabbhāraṃ … pe … sammāsamādhiṃ bhāveti nibbānaninnaṃ nibbānapoṇaṃ nibbānapabbhāraṃ.}}\\
\begin{addmargin}[1em]{2em}
\setstretch{.5}
{\PaliGlossB{It’s when a mendicant develops right view, right thought, right speech, right action, right livelihood, right effort, right mindfulness, and right immersion, which slants, slopes, and inclines to extinguishment.}}\\
\end{addmargin}
\end{absolutelynopagebreak}

\begin{absolutelynopagebreak}
\setstretch{.7}
{\PaliGlossA{evaṃ kho, bhikkhave, bhikkhu appamatto ariyaṃ aṭṭhaṅgikaṃ maggaṃ bhāveti ariyaṃ aṭṭhaṅgikaṃ maggaṃ bahulīkarotī”ti.}}\\
\begin{addmargin}[1em]{2em}
\setstretch{.5}
{\PaliGlossB{That’s how a mendicant who is diligent develops and cultivates the noble eightfold path.”}}\\
\end{addmargin}
\end{absolutelynopagebreak}

\begin{absolutelynopagebreak}
\setstretch{.7}
{\PaliGlossA{paṭhamaṃ.}}\\
\begin{addmargin}[1em]{2em}
\setstretch{.5}
{\PaliGlossB{    -}}\\
\end{addmargin}
\end{absolutelynopagebreak}
