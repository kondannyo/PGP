
\begin{absolutelynopagebreak}
\setstretch{.7}
{\PaliGlossA{saṃyutta nikāya 42}}\\
\begin{addmargin}[1em]{2em}
\setstretch{.5}
{\PaliGlossB{Linked Discourses 42}}\\
\end{addmargin}
\end{absolutelynopagebreak}

\begin{absolutelynopagebreak}
\setstretch{.7}
{\PaliGlossA{1. gāmaṇivagga}}\\
\begin{addmargin}[1em]{2em}
\setstretch{.5}
{\PaliGlossB{1. Chiefs}}\\
\end{addmargin}
\end{absolutelynopagebreak}

\begin{absolutelynopagebreak}
\setstretch{.7}
{\PaliGlossA{7. khettūpamasutta}}\\
\begin{addmargin}[1em]{2em}
\setstretch{.5}
{\PaliGlossB{7. The Simile of the Field}}\\
\end{addmargin}
\end{absolutelynopagebreak}

\begin{absolutelynopagebreak}
\setstretch{.7}
{\PaliGlossA{ekaṃ samayaṃ bhagavā nāḷandāyaṃ viharati pāvārikambavane.}}\\
\begin{addmargin}[1em]{2em}
\setstretch{.5}
{\PaliGlossB{At one time the Buddha was staying near Nālandā in Pāvārika’s mango grove.}}\\
\end{addmargin}
\end{absolutelynopagebreak}

\begin{absolutelynopagebreak}
\setstretch{.7}
{\PaliGlossA{atha kho asibandhakaputto gāmaṇi yena bhagavā tenupasaṅkami; upasaṅkamitvā bhagavantaṃ abhivādetvā ekamantaṃ nisīdi. ekamantaṃ nisinno kho asibandhakaputto gāmaṇi bhagavantaṃ etadavoca:}}\\
\begin{addmargin}[1em]{2em}
\setstretch{.5}
{\PaliGlossB{Then Asibandhaka’s son the chief went up to the Buddha, bowed, sat down to one side, and said to him:}}\\
\end{addmargin}
\end{absolutelynopagebreak}

\begin{absolutelynopagebreak}
\setstretch{.7}
{\PaliGlossA{“nanu, bhante, bhagavā sabbapāṇabhūtahitānukampī viharatī”ti?}}\\
\begin{addmargin}[1em]{2em}
\setstretch{.5}
{\PaliGlossB{“Sir, doesn’t the Buddha live full of compassion for all living beings?”}}\\
\end{addmargin}
\end{absolutelynopagebreak}

\begin{absolutelynopagebreak}
\setstretch{.7}
{\PaliGlossA{“evaṃ, gāmaṇi, tathāgato sabbapāṇabhūtahitānukampī viharatī”ti.}}\\
\begin{addmargin}[1em]{2em}
\setstretch{.5}
{\PaliGlossB{“Yes, chief.”}}\\
\end{addmargin}
\end{absolutelynopagebreak}

\begin{absolutelynopagebreak}
\setstretch{.7}
{\PaliGlossA{“atha kiñcarahi, bhante, bhagavā ekaccānaṃ sakkaccaṃ dhammaṃ deseti, ekaccānaṃ no tathā sakkaccaṃ dhammaṃ desetī”ti?}}\\
\begin{addmargin}[1em]{2em}
\setstretch{.5}
{\PaliGlossB{“Well, sir, why exactly do you teach some people thoroughly and others less thoroughly?”}}\\
\end{addmargin}
\end{absolutelynopagebreak}

\begin{absolutelynopagebreak}
\setstretch{.7}
{\PaliGlossA{“tena hi, gāmaṇi, taññevettha paṭipucchissāmi. yathā te khameyya tathā naṃ byākareyyāsi.}}\\
\begin{addmargin}[1em]{2em}
\setstretch{.5}
{\PaliGlossB{“Well then, chief, I’ll ask you about this in return, and you can answer as you like.}}\\
\end{addmargin}
\end{absolutelynopagebreak}

\begin{absolutelynopagebreak}
\setstretch{.7}
{\PaliGlossA{taṃ kiṃ maññasi, gāmaṇi, idhassu kassakassa gahapatino tīṇi khettāni—ekaṃ khettaṃ aggaṃ, ekaṃ khettaṃ majjhimaṃ, ekaṃ khettaṃ hīnaṃ jaṅgalaṃ ūsaraṃ pāpabhūmi.}}\\
\begin{addmargin}[1em]{2em}
\setstretch{.5}
{\PaliGlossB{What do you think? Suppose a farmer has three fields: one’s good, one’s average, and one’s poor—bad ground of sand and salt.}}\\
\end{addmargin}
\end{absolutelynopagebreak}

\begin{absolutelynopagebreak}
\setstretch{.7}
{\PaliGlossA{taṃ kiṃ maññasi, gāmaṇi, asu kassako gahapati bījāni patiṭṭhāpetukāmo kattha paṭhamaṃ patiṭṭhāpeyya, yaṃ vā aduṃ khettaṃ aggaṃ, yaṃ vā aduṃ khettaṃ majjhimaṃ, yaṃ vā aduṃ khettaṃ hīnaṃ jaṅgalaṃ ūsaraṃ pāpabhūmī”ti?}}\\
\begin{addmargin}[1em]{2em}
\setstretch{.5}
{\PaliGlossB{What do you think? When that farmer wants to plant seeds, where would he plant them first: the good field, the average one, or the poor one?”}}\\
\end{addmargin}
\end{absolutelynopagebreak}

\begin{absolutelynopagebreak}
\setstretch{.7}
{\PaliGlossA{“asu, bhante, kassako gahapati bījāni patiṭṭhāpetukāmo yaṃ aduṃ khettaṃ aggaṃ tattha patiṭṭhāpeyya. tattha patiṭṭhāpetvā yaṃ aduṃ khettaṃ majjhimaṃ tattha patiṭṭhāpeyya. tattha patiṭṭhāpetvā yaṃ aduṃ khettaṃ hīnaṃ jaṅgalaṃ ūsaraṃ pāpabhūmi tattha patiṭṭhāpeyyapi, nopi patiṭṭhāpeyya.}}\\
\begin{addmargin}[1em]{2em}
\setstretch{.5}
{\PaliGlossB{“Sir, he’d plant them first in the good field, then the average, then he may or may not plant seed in the poor field.}}\\
\end{addmargin}
\end{absolutelynopagebreak}

\begin{absolutelynopagebreak}
\setstretch{.7}
{\PaliGlossA{taṃ kissa hetu?}}\\
\begin{addmargin}[1em]{2em}
\setstretch{.5}
{\PaliGlossB{Why is that?}}\\
\end{addmargin}
\end{absolutelynopagebreak}

\begin{absolutelynopagebreak}
\setstretch{.7}
{\PaliGlossA{antamaso gobhattampi bhavissatī”ti.}}\\
\begin{addmargin}[1em]{2em}
\setstretch{.5}
{\PaliGlossB{Because at least it can be fodder for the cattle.”}}\\
\end{addmargin}
\end{absolutelynopagebreak}

\begin{absolutelynopagebreak}
\setstretch{.7}
{\PaliGlossA{“seyyathāpi, gāmaṇi, yaṃ aduṃ khettaṃ aggaṃ; evameva mayhaṃ bhikkhubhikkhuniyo.}}\\
\begin{addmargin}[1em]{2em}
\setstretch{.5}
{\PaliGlossB{“To me, the monks and nuns are like the good field.}}\\
\end{addmargin}
\end{absolutelynopagebreak}

\begin{absolutelynopagebreak}
\setstretch{.7}
{\PaliGlossA{tesāhaṃ dhammaṃ desemi—ādikalyāṇaṃ majjhekalyāṇaṃ pariyosānakalyāṇaṃ, sātthaṃ sabyañjanaṃ kevalaparipuṇṇaṃ parisuddhaṃ brahmacariyaṃ pakāsemi.}}\\
\begin{addmargin}[1em]{2em}
\setstretch{.5}
{\PaliGlossB{I teach them the Dhamma that’s good in the beginning, good in the middle, and good in the end, meaningful and well-phrased. And I reveal a spiritual practice that’s entirely full and pure.}}\\
\end{addmargin}
\end{absolutelynopagebreak}

\begin{absolutelynopagebreak}
\setstretch{.7}
{\PaliGlossA{taṃ kissa hetu?}}\\
\begin{addmargin}[1em]{2em}
\setstretch{.5}
{\PaliGlossB{Why is that?}}\\
\end{addmargin}
\end{absolutelynopagebreak}

\begin{absolutelynopagebreak}
\setstretch{.7}
{\PaliGlossA{ete hi, gāmaṇi, maṃdīpā maṃleṇā maṃtāṇā maṃsaraṇā viharanti.}}\\
\begin{addmargin}[1em]{2em}
\setstretch{.5}
{\PaliGlossB{Because they live with me as their island, protection, shelter, and refuge.}}\\
\end{addmargin}
\end{absolutelynopagebreak}

\begin{absolutelynopagebreak}
\setstretch{.7}
{\PaliGlossA{seyyathāpi, gāmaṇi, yaṃ aduṃ khettaṃ majjhimaṃ; evameva mayhaṃ upāsakaupāsikāyo.}}\\
\begin{addmargin}[1em]{2em}
\setstretch{.5}
{\PaliGlossB{To me, the laymen and laywomen are like the average field.}}\\
\end{addmargin}
\end{absolutelynopagebreak}

\begin{absolutelynopagebreak}
\setstretch{.7}
{\PaliGlossA{tesaṃ pāhaṃ dhammaṃ desemi—ādikalyāṇaṃ majjhekalyāṇaṃ pariyosānakalyāṇaṃ, sātthaṃ sabyañjanaṃ kevalaparipuṇṇaṃ parisuddhaṃ brahmacariyaṃ pakāsemi.}}\\
\begin{addmargin}[1em]{2em}
\setstretch{.5}
{\PaliGlossB{I also teach them the Dhamma that’s good in the beginning, good in the middle, and good in the end, meaningful and well-phrased. And I reveal a spiritual practice that’s entirely full and pure.}}\\
\end{addmargin}
\end{absolutelynopagebreak}

\begin{absolutelynopagebreak}
\setstretch{.7}
{\PaliGlossA{taṃ kissa hetu?}}\\
\begin{addmargin}[1em]{2em}
\setstretch{.5}
{\PaliGlossB{Why is that?}}\\
\end{addmargin}
\end{absolutelynopagebreak}

\begin{absolutelynopagebreak}
\setstretch{.7}
{\PaliGlossA{ete hi, gāmaṇi, maṃdīpā maṃleṇā maṃtāṇā maṃsaraṇā viharanti.}}\\
\begin{addmargin}[1em]{2em}
\setstretch{.5}
{\PaliGlossB{Because they live with me as their island, protection, shelter, and refuge.}}\\
\end{addmargin}
\end{absolutelynopagebreak}

\begin{absolutelynopagebreak}
\setstretch{.7}
{\PaliGlossA{seyyathāpi, gāmaṇi, yaṃ aduṃ khettaṃ hīnaṃ jaṅgalaṃ ūsaraṃ pāpabhūmi; evameva mayhaṃ aññatitthiyā samaṇabrāhmaṇaparibbājakā.}}\\
\begin{addmargin}[1em]{2em}
\setstretch{.5}
{\PaliGlossB{To me, the ascetics, brahmins, and wanderers who follow other paths are like the poor field, the bad ground of sand and salt.}}\\
\end{addmargin}
\end{absolutelynopagebreak}

\begin{absolutelynopagebreak}
\setstretch{.7}
{\PaliGlossA{tesaṃ pāhaṃ dhammaṃ desemi—ādikalyāṇaṃ majjhekalyāṇaṃ pariyosānakalyāṇaṃ sātthaṃ sabyañjanaṃ, kevalaparipuṇṇaṃ parisuddhaṃ brahmacariyaṃ pakāsemi.}}\\
\begin{addmargin}[1em]{2em}
\setstretch{.5}
{\PaliGlossB{I also teach them the Dhamma that’s good in the beginning, good in the middle, and good in the end, meaningful and well-phrased. And I reveal a spiritual practice that’s entirely full and pure.}}\\
\end{addmargin}
\end{absolutelynopagebreak}

\begin{absolutelynopagebreak}
\setstretch{.7}
{\PaliGlossA{taṃ kissa hetu?}}\\
\begin{addmargin}[1em]{2em}
\setstretch{.5}
{\PaliGlossB{Why is that?}}\\
\end{addmargin}
\end{absolutelynopagebreak}

\begin{absolutelynopagebreak}
\setstretch{.7}
{\PaliGlossA{appeva nāma ekaṃ padampi ājāneyyuṃ taṃ nesaṃ assa dīgharattaṃ hitāya sukhāyāti.}}\\
\begin{addmargin}[1em]{2em}
\setstretch{.5}
{\PaliGlossB{Hopefully they might understand even a single sentence, which would be for their lasting welfare and happiness.}}\\
\end{addmargin}
\end{absolutelynopagebreak}

\begin{absolutelynopagebreak}
\setstretch{.7}
{\PaliGlossA{seyyathāpi, gāmaṇi, purisassa tayo udakamaṇikā—eko udakamaṇiko acchiddo ahārī aparihārī, eko udakamaṇiko acchiddo hārī parihārī, eko udakamaṇiko chiddo hārī parihārī.}}\\
\begin{addmargin}[1em]{2em}
\setstretch{.5}
{\PaliGlossB{Suppose a person had three water jars: one that’s uncracked and nonporous; one that’s uncracked but porous; and one that’s cracked and porous.}}\\
\end{addmargin}
\end{absolutelynopagebreak}

\begin{absolutelynopagebreak}
\setstretch{.7}
{\PaliGlossA{taṃ kiṃ maññasi, gāmaṇi, asu puriso udakaṃ nikkhipitukāmo kattha paṭhamaṃ nikkhipeyya, yo vā so udakamaṇiko acchiddo ahārī aparihārī, yo vā so udakamaṇiko acchiddo hārī parihārī, yo vā so udakamaṇiko chiddo hārī parihārī”ti?}}\\
\begin{addmargin}[1em]{2em}
\setstretch{.5}
{\PaliGlossB{What do you think? When that person wants to store water, where would they store it first: in the jar that’s uncracked and nonporous, the one that’s uncracked but porous, or the one that’s cracked and porous?”}}\\
\end{addmargin}
\end{absolutelynopagebreak}

\begin{absolutelynopagebreak}
\setstretch{.7}
{\PaliGlossA{“asu, bhante, puriso udakaṃ nikkhipitukāmo, yo so udakamaṇiko acchiddo ahārī aparihārī tattha nikkhipeyya, tattha nikkhipitvā, yo so udakamaṇiko acchiddo hārī parihārī tattha nikkhipeyya, tattha nikkhipitvā, yo so udakamaṇiko chiddo hārī parihārī tattha nikkhipeyyapi, nopi nikkhipeyya.}}\\
\begin{addmargin}[1em]{2em}
\setstretch{.5}
{\PaliGlossB{“Sir, they’d store water first in the jar that’s uncracked and nonporous, then the one that’s uncracked but porous, then they may or may not store water in the one that’s cracked and porous.}}\\
\end{addmargin}
\end{absolutelynopagebreak}

\begin{absolutelynopagebreak}
\setstretch{.7}
{\PaliGlossA{taṃ kissa hetu?}}\\
\begin{addmargin}[1em]{2em}
\setstretch{.5}
{\PaliGlossB{Why is that?}}\\
\end{addmargin}
\end{absolutelynopagebreak}

\begin{absolutelynopagebreak}
\setstretch{.7}
{\PaliGlossA{antamaso bhaṇḍadhovanampi bhavissatī”ti.}}\\
\begin{addmargin}[1em]{2em}
\setstretch{.5}
{\PaliGlossB{Because at least it can be used for washing the dishes.”}}\\
\end{addmargin}
\end{absolutelynopagebreak}

\begin{absolutelynopagebreak}
\setstretch{.7}
{\PaliGlossA{“seyyathāpi, gāmaṇi, yo so udakamaṇiko acchiddo ahārī aparihārī; evameva mayhaṃ bhikkhubhikkhuniyo.}}\\
\begin{addmargin}[1em]{2em}
\setstretch{.5}
{\PaliGlossB{“To me, the monks and nuns are like the water jar that’s uncracked and nonporous.}}\\
\end{addmargin}
\end{absolutelynopagebreak}

\begin{absolutelynopagebreak}
\setstretch{.7}
{\PaliGlossA{tesāhaṃ dhammaṃ desemi—ādikalyāṇaṃ majjhekalyāṇaṃ pariyosānakalyāṇaṃ sātthaṃ sabyañjanaṃ, kevalaparipuṇṇaṃ parisuddhaṃ brahmacariyaṃ pakāsemi.}}\\
\begin{addmargin}[1em]{2em}
\setstretch{.5}
{\PaliGlossB{I teach them the Dhamma that’s good in the beginning, good in the middle, and good in the end, meaningful and well-phrased. And I reveal a spiritual practice that’s entirely full and pure.}}\\
\end{addmargin}
\end{absolutelynopagebreak}

\begin{absolutelynopagebreak}
\setstretch{.7}
{\PaliGlossA{taṃ kissa hetu?}}\\
\begin{addmargin}[1em]{2em}
\setstretch{.5}
{\PaliGlossB{Why is that?}}\\
\end{addmargin}
\end{absolutelynopagebreak}

\begin{absolutelynopagebreak}
\setstretch{.7}
{\PaliGlossA{ete hi, gāmaṇi, maṃdīpā maṃleṇā maṃtāṇā maṃsaraṇā viharanti.}}\\
\begin{addmargin}[1em]{2em}
\setstretch{.5}
{\PaliGlossB{Because they live with me as their island, protection, shelter, and refuge.}}\\
\end{addmargin}
\end{absolutelynopagebreak}

\begin{absolutelynopagebreak}
\setstretch{.7}
{\PaliGlossA{seyyathāpi, gāmaṇi, yo so udakamaṇiko acchiddo hārī parihārī; evameva mayhaṃ upāsakaupāsikāyo.}}\\
\begin{addmargin}[1em]{2em}
\setstretch{.5}
{\PaliGlossB{To me, the laymen and laywomen are like the water jar that’s uncracked but porous.}}\\
\end{addmargin}
\end{absolutelynopagebreak}

\begin{absolutelynopagebreak}
\setstretch{.7}
{\PaliGlossA{tesāhaṃ dhammaṃ desemi—ādikalyāṇaṃ majjhekalyāṇaṃ pariyosānakalyāṇaṃ sātthaṃ sabyañjanaṃ, kevalaparipuṇṇaṃ parisuddhaṃ brahmacariyaṃ pakāsemi.}}\\
\begin{addmargin}[1em]{2em}
\setstretch{.5}
{\PaliGlossB{I teach them the Dhamma that’s good in the beginning, good in the middle, and good in the end, meaningful and well-phrased. And I reveal a spiritual practice that’s entirely full and pure.}}\\
\end{addmargin}
\end{absolutelynopagebreak}

\begin{absolutelynopagebreak}
\setstretch{.7}
{\PaliGlossA{taṃ kissa hetu?}}\\
\begin{addmargin}[1em]{2em}
\setstretch{.5}
{\PaliGlossB{Why is that?}}\\
\end{addmargin}
\end{absolutelynopagebreak}

\begin{absolutelynopagebreak}
\setstretch{.7}
{\PaliGlossA{ete hi, gāmaṇi, maṃdīpā maṃleṇā maṃtāṇā maṃsaraṇā viharanti.}}\\
\begin{addmargin}[1em]{2em}
\setstretch{.5}
{\PaliGlossB{Because they live with me as their island, protection, shelter, and refuge.}}\\
\end{addmargin}
\end{absolutelynopagebreak}

\begin{absolutelynopagebreak}
\setstretch{.7}
{\PaliGlossA{seyyathāpi, gāmaṇi, yo so udakamaṇiko chiddo hārī parihārī; evameva mayhaṃ aññatitthiyā samaṇabrāhmaṇaparibbājakā.}}\\
\begin{addmargin}[1em]{2em}
\setstretch{.5}
{\PaliGlossB{To me, the ascetics, brahmins, and wanderers who follow other paths are like the water jar that’s cracked and porous.}}\\
\end{addmargin}
\end{absolutelynopagebreak}

\begin{absolutelynopagebreak}
\setstretch{.7}
{\PaliGlossA{tesāhaṃ dhammaṃ desemi—ādikalyāṇaṃ majjhekalyāṇaṃ pariyosānakalyāṇaṃ sātthaṃ sabyañjanaṃ kevalaparipuṇṇaṃ parisuddhaṃ brahmacariyaṃ pakāsemi.}}\\
\begin{addmargin}[1em]{2em}
\setstretch{.5}
{\PaliGlossB{I also teach them the Dhamma that’s good in the beginning, good in the middle, and good in the end, meaningful and well-phrased. And I reveal a spiritual practice that’s entirely full and pure.}}\\
\end{addmargin}
\end{absolutelynopagebreak}

\begin{absolutelynopagebreak}
\setstretch{.7}
{\PaliGlossA{taṃ kissa hetu?}}\\
\begin{addmargin}[1em]{2em}
\setstretch{.5}
{\PaliGlossB{Why is that?}}\\
\end{addmargin}
\end{absolutelynopagebreak}

\begin{absolutelynopagebreak}
\setstretch{.7}
{\PaliGlossA{appeva nāma ekaṃ padampi ājāneyyuṃ, taṃ nesaṃ assa dīgharattaṃ hitāya sukhāyā”ti.}}\\
\begin{addmargin}[1em]{2em}
\setstretch{.5}
{\PaliGlossB{Hopefully they might understand even a single sentence, which would be for their lasting welfare and happiness.”}}\\
\end{addmargin}
\end{absolutelynopagebreak}

\begin{absolutelynopagebreak}
\setstretch{.7}
{\PaliGlossA{evaṃ vutte, asibandhakaputto gāmaṇi bhagavantaṃ etadavoca:}}\\
\begin{addmargin}[1em]{2em}
\setstretch{.5}
{\PaliGlossB{When he said this, Asibandhaka’s son the chief said to the Buddha,}}\\
\end{addmargin}
\end{absolutelynopagebreak}

\begin{absolutelynopagebreak}
\setstretch{.7}
{\PaliGlossA{“abhikkantaṃ, bhante, abhikkantaṃ, bhante … pe …}}\\
\begin{addmargin}[1em]{2em}
\setstretch{.5}
{\PaliGlossB{“Excellent, sir! Excellent! …}}\\
\end{addmargin}
\end{absolutelynopagebreak}

\begin{absolutelynopagebreak}
\setstretch{.7}
{\PaliGlossA{upāsakaṃ maṃ bhagavā dhāretu ajjatagge pāṇupetaṃ saraṇaṃ gatan”ti.}}\\
\begin{addmargin}[1em]{2em}
\setstretch{.5}
{\PaliGlossB{From this day forth, may the Buddha remember me as a lay follower who has gone for refuge for life.”}}\\
\end{addmargin}
\end{absolutelynopagebreak}

\begin{absolutelynopagebreak}
\setstretch{.7}
{\PaliGlossA{sattamaṃ.}}\\
\begin{addmargin}[1em]{2em}
\setstretch{.5}
{\PaliGlossB{    -}}\\
\end{addmargin}
\end{absolutelynopagebreak}
