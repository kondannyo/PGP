
\begin{absolutelynopagebreak}
\setstretch{.7}
{\PaliGlossA{saṃyutta nikāya 14}}\\
\begin{addmargin}[1em]{2em}
\setstretch{.5}
{\PaliGlossB{Linked Discourses 14}}\\
\end{addmargin}
\end{absolutelynopagebreak}

\begin{absolutelynopagebreak}
\setstretch{.7}
{\PaliGlossA{4. catutthavagga}}\\
\begin{addmargin}[1em]{2em}
\setstretch{.5}
{\PaliGlossB{4. The Fourth Chapter}}\\
\end{addmargin}
\end{absolutelynopagebreak}

\begin{absolutelynopagebreak}
\setstretch{.7}
{\PaliGlossA{37. samaṇabrāhmaṇasutta}}\\
\begin{addmargin}[1em]{2em}
\setstretch{.5}
{\PaliGlossB{37. Ascetics and Brahmins}}\\
\end{addmargin}
\end{absolutelynopagebreak}

\begin{absolutelynopagebreak}
\setstretch{.7}
{\PaliGlossA{sāvatthiyaṃ viharati.}}\\
\begin{addmargin}[1em]{2em}
\setstretch{.5}
{\PaliGlossB{At Sāvatthī.}}\\
\end{addmargin}
\end{absolutelynopagebreak}

\begin{absolutelynopagebreak}
\setstretch{.7}
{\PaliGlossA{“catasso imā, bhikkhave, dhātuyo.}}\\
\begin{addmargin}[1em]{2em}
\setstretch{.5}
{\PaliGlossB{“Mendicants, there are these four elements.}}\\
\end{addmargin}
\end{absolutelynopagebreak}

\begin{absolutelynopagebreak}
\setstretch{.7}
{\PaliGlossA{katamā catasso?}}\\
\begin{addmargin}[1em]{2em}
\setstretch{.5}
{\PaliGlossB{What four?}}\\
\end{addmargin}
\end{absolutelynopagebreak}

\begin{absolutelynopagebreak}
\setstretch{.7}
{\PaliGlossA{pathavīdhātu, āpodhātu, tejodhātu, vāyodhātu.}}\\
\begin{addmargin}[1em]{2em}
\setstretch{.5}
{\PaliGlossB{The elements of earth, water, fire, and air.}}\\
\end{addmargin}
\end{absolutelynopagebreak}

\begin{absolutelynopagebreak}
\setstretch{.7}
{\PaliGlossA{ye hi keci, bhikkhave, samaṇā vā brāhmaṇā vā imāsaṃ catunnaṃ dhātūnaṃ assādañca ādīnavañca nissaraṇañca yathābhūtaṃ nappajānanti,}}\\
\begin{addmargin}[1em]{2em}
\setstretch{.5}
{\PaliGlossB{There are ascetics and brahmins who don’t truly understand these four elements’ gratification, drawback, and escape for what they are.}}\\
\end{addmargin}
\end{absolutelynopagebreak}

\begin{absolutelynopagebreak}
\setstretch{.7}
{\PaliGlossA{na me te, bhikkhave, samaṇā vā brāhmaṇā vā samaṇesu vā samaṇasammatā brāhmaṇesu vā brāhmaṇasammatā; na ca pana te āyasmanto sāmaññatthaṃ vā brahmaññatthaṃ vā diṭṭheva dhamme sayaṃ abhiññā sacchikatvā upasampajja viharanti.}}\\
\begin{addmargin}[1em]{2em}
\setstretch{.5}
{\PaliGlossB{I don’t regard them as true ascetics and brahmins. Those venerables don’t realize the goal of life as an ascetic or brahmin, and don’t live having realized it with their own insight.}}\\
\end{addmargin}
\end{absolutelynopagebreak}

\begin{absolutelynopagebreak}
\setstretch{.7}
{\PaliGlossA{ye ca kho keci, bhikkhave, samaṇā vā brāhmaṇā vā imāsaṃ catunnaṃ dhātūnaṃ assādañca ādīnavañca nissaraṇañca yathābhūtaṃ pajānanti,}}\\
\begin{addmargin}[1em]{2em}
\setstretch{.5}
{\PaliGlossB{There are ascetics and brahmins who do truly understand these four elements’ gratification, drawback, and escape for what they are.}}\\
\end{addmargin}
\end{absolutelynopagebreak}

\begin{absolutelynopagebreak}
\setstretch{.7}
{\PaliGlossA{te ca kho me, bhikkhave, samaṇā vā brāhmaṇā vā samaṇesu ceva samaṇasammatā brāhmaṇesu ca brāhmaṇasammatā; te ca panāyasmanto sāmaññatthañca brahmaññatthañca diṭṭheva dhamme sayaṃ abhiññā sacchikatvā upasampajja viharantī”ti.}}\\
\begin{addmargin}[1em]{2em}
\setstretch{.5}
{\PaliGlossB{I regard them as true ascetics and brahmins. Those venerables realize the goal of life as an ascetic or brahmin, and live having realized it with their own insight.”}}\\
\end{addmargin}
\end{absolutelynopagebreak}

\begin{absolutelynopagebreak}
\setstretch{.7}
{\PaliGlossA{aṭṭhamaṃ.}}\\
\begin{addmargin}[1em]{2em}
\setstretch{.5}
{\PaliGlossB{    -}}\\
\end{addmargin}
\end{absolutelynopagebreak}
