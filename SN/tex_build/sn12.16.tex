
\begin{absolutelynopagebreak}
\setstretch{.7}
{\PaliGlossA{saṃyutta nikāya 12}}\\
\begin{addmargin}[1em]{2em}
\setstretch{.5}
{\PaliGlossB{Linked Discourses 12}}\\
\end{addmargin}
\end{absolutelynopagebreak}

\begin{absolutelynopagebreak}
\setstretch{.7}
{\PaliGlossA{2. āhāravagga}}\\
\begin{addmargin}[1em]{2em}
\setstretch{.5}
{\PaliGlossB{2. Fuel}}\\
\end{addmargin}
\end{absolutelynopagebreak}

\begin{absolutelynopagebreak}
\setstretch{.7}
{\PaliGlossA{16. dhammakathikasutta}}\\
\begin{addmargin}[1em]{2em}
\setstretch{.5}
{\PaliGlossB{16. A Dhamma Speaker}}\\
\end{addmargin}
\end{absolutelynopagebreak}

\begin{absolutelynopagebreak}
\setstretch{.7}
{\PaliGlossA{sāvatthiyaṃ …}}\\
\begin{addmargin}[1em]{2em}
\setstretch{.5}
{\PaliGlossB{At Sāvatthī.}}\\
\end{addmargin}
\end{absolutelynopagebreak}

\begin{absolutelynopagebreak}
\setstretch{.7}
{\PaliGlossA{atha kho aññataro bhikkhu yena bhagavā tenupasaṅkami; upasaṅkamitvā bhagavantaṃ abhivādetvā ekamantaṃ nisīdi. ekamantaṃ nisinno kho so bhikkhu bhagavantaṃ etadavoca:}}\\
\begin{addmargin}[1em]{2em}
\setstretch{.5}
{\PaliGlossB{Then a mendicant went up to the Buddha, bowed, sat down to one side, and said to him:}}\\
\end{addmargin}
\end{absolutelynopagebreak}

\begin{absolutelynopagebreak}
\setstretch{.7}
{\PaliGlossA{“‘dhammakathiko dhammakathiko’ti, bhante, vuccati.}}\\
\begin{addmargin}[1em]{2em}
\setstretch{.5}
{\PaliGlossB{“Sir, they speak of a ‘Dhamma speaker’.}}\\
\end{addmargin}
\end{absolutelynopagebreak}

\begin{absolutelynopagebreak}
\setstretch{.7}
{\PaliGlossA{kittāvatā nu kho, bhante, dhammakathiko hotī”ti?}}\\
\begin{addmargin}[1em]{2em}
\setstretch{.5}
{\PaliGlossB{How is a Dhamma speaker defined?”}}\\
\end{addmargin}
\end{absolutelynopagebreak}

\begin{absolutelynopagebreak}
\setstretch{.7}
{\PaliGlossA{“jarāmaraṇassa ce bhikkhu nibbidāya virāgāya nirodhāya dhammaṃ deseti, ‘dhammakathiko bhikkhū’ti alaṃvacanāya.}}\\
\begin{addmargin}[1em]{2em}
\setstretch{.5}
{\PaliGlossB{“If a mendicant teaches Dhamma for disillusionment, dispassion, and cessation regarding old age and death, they’re qualified to be called a ‘mendicant who speaks on Dhamma’.}}\\
\end{addmargin}
\end{absolutelynopagebreak}

\begin{absolutelynopagebreak}
\setstretch{.7}
{\PaliGlossA{jarāmaraṇassa ce bhikkhu nibbidāya virāgāya nirodhāya paṭipanno hoti, ‘dhammānudhammappaṭipanno bhikkhū’ti alaṃvacanāya.}}\\
\begin{addmargin}[1em]{2em}
\setstretch{.5}
{\PaliGlossB{If they practice for disillusionment, dispassion, and cessation regarding old age and death, they’re qualified to be called a ‘mendicant who practices in line with the teaching’.}}\\
\end{addmargin}
\end{absolutelynopagebreak}

\begin{absolutelynopagebreak}
\setstretch{.7}
{\PaliGlossA{jarāmaraṇassa ce bhikkhu nibbidā virāgā nirodhā anupādāvimutto hoti, ‘diṭṭhadhammanibbānappatto bhikkhū’ti alaṃvacanāya.}}\\
\begin{addmargin}[1em]{2em}
\setstretch{.5}
{\PaliGlossB{If they’re freed by not grasping, by disillusionment, dispassion, and cessation regarding old age and death, they’re qualified to be called a ‘mendicant who has attained extinguishment in this very life’.}}\\
\end{addmargin}
\end{absolutelynopagebreak}

\begin{absolutelynopagebreak}
\setstretch{.7}
{\PaliGlossA{jātiyā ce bhikkhu … pe …}}\\
\begin{addmargin}[1em]{2em}
\setstretch{.5}
{\PaliGlossB{If a mendicant teaches Dhamma for disillusionment regarding rebirth …}}\\
\end{addmargin}
\end{absolutelynopagebreak}

\begin{absolutelynopagebreak}
\setstretch{.7}
{\PaliGlossA{bhavassa ce bhikkhu …}}\\
\begin{addmargin}[1em]{2em}
\setstretch{.5}
{\PaliGlossB{continued existence …}}\\
\end{addmargin}
\end{absolutelynopagebreak}

\begin{absolutelynopagebreak}
\setstretch{.7}
{\PaliGlossA{upādānassa ce bhikkhu …}}\\
\begin{addmargin}[1em]{2em}
\setstretch{.5}
{\PaliGlossB{grasping …}}\\
\end{addmargin}
\end{absolutelynopagebreak}

\begin{absolutelynopagebreak}
\setstretch{.7}
{\PaliGlossA{taṇhāya ce bhikkhu …}}\\
\begin{addmargin}[1em]{2em}
\setstretch{.5}
{\PaliGlossB{craving …}}\\
\end{addmargin}
\end{absolutelynopagebreak}

\begin{absolutelynopagebreak}
\setstretch{.7}
{\PaliGlossA{vedanāya ce bhikkhu …}}\\
\begin{addmargin}[1em]{2em}
\setstretch{.5}
{\PaliGlossB{feeling …}}\\
\end{addmargin}
\end{absolutelynopagebreak}

\begin{absolutelynopagebreak}
\setstretch{.7}
{\PaliGlossA{phassassa ce bhikkhu …}}\\
\begin{addmargin}[1em]{2em}
\setstretch{.5}
{\PaliGlossB{contact …}}\\
\end{addmargin}
\end{absolutelynopagebreak}

\begin{absolutelynopagebreak}
\setstretch{.7}
{\PaliGlossA{saḷāyatanassa ce bhikkhu …}}\\
\begin{addmargin}[1em]{2em}
\setstretch{.5}
{\PaliGlossB{the six sense fields …}}\\
\end{addmargin}
\end{absolutelynopagebreak}

\begin{absolutelynopagebreak}
\setstretch{.7}
{\PaliGlossA{nāmarūpassa ce bhikkhu …}}\\
\begin{addmargin}[1em]{2em}
\setstretch{.5}
{\PaliGlossB{name and form …}}\\
\end{addmargin}
\end{absolutelynopagebreak}

\begin{absolutelynopagebreak}
\setstretch{.7}
{\PaliGlossA{viññāṇassa ce bhikkhu …}}\\
\begin{addmargin}[1em]{2em}
\setstretch{.5}
{\PaliGlossB{consciousness …}}\\
\end{addmargin}
\end{absolutelynopagebreak}

\begin{absolutelynopagebreak}
\setstretch{.7}
{\PaliGlossA{saṅkhārānañce bhikkhu …}}\\
\begin{addmargin}[1em]{2em}
\setstretch{.5}
{\PaliGlossB{choices …}}\\
\end{addmargin}
\end{absolutelynopagebreak}

\begin{absolutelynopagebreak}
\setstretch{.7}
{\PaliGlossA{avijjāya ce bhikkhu nibbidāya virāgāya nirodhāya dhammaṃ deseti, ‘dhammakathiko bhikkhū’ti alaṃvacanāya.}}\\
\begin{addmargin}[1em]{2em}
\setstretch{.5}
{\PaliGlossB{If a mendicant teaches Dhamma for disillusionment, dispassion, and cessation regarding ignorance, they’re qualified to be called a ‘mendicant who speaks on Dhamma’.}}\\
\end{addmargin}
\end{absolutelynopagebreak}

\begin{absolutelynopagebreak}
\setstretch{.7}
{\PaliGlossA{avijjāya ce bhikkhu nibbidāya virāgāya nirodhāya paṭipanno hoti, ‘dhammānudhammappaṭipanno bhikkhū’ti alaṃvacanāya.}}\\
\begin{addmargin}[1em]{2em}
\setstretch{.5}
{\PaliGlossB{If they practice for disillusionment, dispassion, and cessation regarding ignorance, they’re qualified to be called a ‘mendicant who practices in line with the teaching’.}}\\
\end{addmargin}
\end{absolutelynopagebreak}

\begin{absolutelynopagebreak}
\setstretch{.7}
{\PaliGlossA{avijjāya ce bhikkhu nibbidā virāgā nirodhā anupādāvimutto hoti, ‘diṭṭhadhammanibbānappatto bhikkhū’ti alaṃvacanāyā”ti.}}\\
\begin{addmargin}[1em]{2em}
\setstretch{.5}
{\PaliGlossB{If they’re freed by not grasping, by disillusionment, dispassion, and cessation regarding ignorance, they’re qualified to be called a ‘mendicant who has attained extinguishment in this very life’.”}}\\
\end{addmargin}
\end{absolutelynopagebreak}

\begin{absolutelynopagebreak}
\setstretch{.7}
{\PaliGlossA{chaṭṭhaṃ.}}\\
\begin{addmargin}[1em]{2em}
\setstretch{.5}
{\PaliGlossB{    -}}\\
\end{addmargin}
\end{absolutelynopagebreak}
