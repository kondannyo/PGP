
\begin{absolutelynopagebreak}
\setstretch{.7}
{\PaliGlossA{saṃyutta nikāya 12}}\\
\begin{addmargin}[1em]{2em}
\setstretch{.5}
{\PaliGlossB{Linked Discourses 12}}\\
\end{addmargin}
\end{absolutelynopagebreak}

\begin{absolutelynopagebreak}
\setstretch{.7}
{\PaliGlossA{3. dasabalavagga}}\\
\begin{addmargin}[1em]{2em}
\setstretch{.5}
{\PaliGlossB{3. The Ten Powers}}\\
\end{addmargin}
\end{absolutelynopagebreak}

\begin{absolutelynopagebreak}
\setstretch{.7}
{\PaliGlossA{25. bhūmijasutta}}\\
\begin{addmargin}[1em]{2em}
\setstretch{.5}
{\PaliGlossB{25. With Bhūmija}}\\
\end{addmargin}
\end{absolutelynopagebreak}

\begin{absolutelynopagebreak}
\setstretch{.7}
{\PaliGlossA{sāvatthiyaṃ viharati.}}\\
\begin{addmargin}[1em]{2em}
\setstretch{.5}
{\PaliGlossB{At Sāvatthī.}}\\
\end{addmargin}
\end{absolutelynopagebreak}

\begin{absolutelynopagebreak}
\setstretch{.7}
{\PaliGlossA{atha kho āyasmā bhūmijo sāyanhasamayaṃ paṭisallānā vuṭṭhito yenāyasmā sāriputto tenupasaṅkami; upasaṅkamitvā āyasmatā sāriputtena saddhiṃ sammodi.}}\\
\begin{addmargin}[1em]{2em}
\setstretch{.5}
{\PaliGlossB{Then in the late afternoon, Venerable Bhūmija came out of retreat, went to Venerable Sāriputta, and exchanged greetings with him.}}\\
\end{addmargin}
\end{absolutelynopagebreak}

\begin{absolutelynopagebreak}
\setstretch{.7}
{\PaliGlossA{sammodanīyaṃ kathaṃ sāraṇīyaṃ vītisāretvā ekamantaṃ nisīdi. ekamantaṃ nisinno kho āyasmā bhūmijo āyasmantaṃ sāriputtaṃ etadavoca:}}\\
\begin{addmargin}[1em]{2em}
\setstretch{.5}
{\PaliGlossB{When the greetings and polite conversation were over, he sat down to one side and said to him:}}\\
\end{addmargin}
\end{absolutelynopagebreak}

\begin{absolutelynopagebreak}
\setstretch{.7}
{\PaliGlossA{“santāvuso sāriputta, eke samaṇabrāhmaṇā kammavādā sayaṅkataṃ sukhadukkhaṃ paññapenti.}}\\
\begin{addmargin}[1em]{2em}
\setstretch{.5}
{\PaliGlossB{“Reverend Sāriputta, there are ascetics and brahmins who teach the efficacy of deeds. Some of them declare that pleasure and pain are made by oneself.}}\\
\end{addmargin}
\end{absolutelynopagebreak}

\begin{absolutelynopagebreak}
\setstretch{.7}
{\PaliGlossA{santi panāvuso sāriputta, eke samaṇabrāhmaṇā kammavādā paraṅkataṃ sukhadukkhaṃ paññapenti.}}\\
\begin{addmargin}[1em]{2em}
\setstretch{.5}
{\PaliGlossB{Some of them declare that pleasure and pain are made by another.}}\\
\end{addmargin}
\end{absolutelynopagebreak}

\begin{absolutelynopagebreak}
\setstretch{.7}
{\PaliGlossA{santāvuso sāriputta, eke samaṇabrāhmaṇā kammavādā sayaṅkatañca paraṅkatañca sukhadukkhaṃ paññapenti.}}\\
\begin{addmargin}[1em]{2em}
\setstretch{.5}
{\PaliGlossB{Some of them declare that pleasure and pain are made by both oneself and another.}}\\
\end{addmargin}
\end{absolutelynopagebreak}

\begin{absolutelynopagebreak}
\setstretch{.7}
{\PaliGlossA{santi panāvuso sāriputta, eke samaṇabrāhmaṇā kammavādā asayaṅkāraṃ aparaṅkāraṃ adhiccasamuppannaṃ sukhadukkhaṃ paññapenti.}}\\
\begin{addmargin}[1em]{2em}
\setstretch{.5}
{\PaliGlossB{Some of them declare that pleasure and pain arise by chance, not made by oneself or another.}}\\
\end{addmargin}
\end{absolutelynopagebreak}

\begin{absolutelynopagebreak}
\setstretch{.7}
{\PaliGlossA{idha no, āvuso sāriputta, bhagavā kiṃvādī kimakkhāyī,}}\\
\begin{addmargin}[1em]{2em}
\setstretch{.5}
{\PaliGlossB{What does the Buddha say about this? How does he explain it?}}\\
\end{addmargin}
\end{absolutelynopagebreak}

\begin{absolutelynopagebreak}
\setstretch{.7}
{\PaliGlossA{kathaṃ byākaramānā ca mayaṃ vuttavādino ceva bhagavato assāma, na ca bhagavantaṃ abhūtena abbhācikkheyyāma, dhammassa cānudhammaṃ byākareyyāma, na ca koci sahadhammiko vādānupāto gārayhaṃ ṭhānaṃ āgaccheyyā”ti?}}\\
\begin{addmargin}[1em]{2em}
\setstretch{.5}
{\PaliGlossB{How should we answer so as to repeat what the Buddha has said, and not misrepresent him with an untruth? How should we explain in line with his teaching, with no legitimate grounds for rebuke and criticism?”}}\\
\end{addmargin}
\end{absolutelynopagebreak}

\begin{absolutelynopagebreak}
\setstretch{.7}
{\PaliGlossA{“paṭiccasamuppannaṃ kho, āvuso, sukhadukkhaṃ vuttaṃ bhagavatā.}}\\
\begin{addmargin}[1em]{2em}
\setstretch{.5}
{\PaliGlossB{“Reverend, the Buddha said that suffering is dependently originated.}}\\
\end{addmargin}
\end{absolutelynopagebreak}

\begin{absolutelynopagebreak}
\setstretch{.7}
{\PaliGlossA{kiṃ paṭicca?}}\\
\begin{addmargin}[1em]{2em}
\setstretch{.5}
{\PaliGlossB{Dependent on what?}}\\
\end{addmargin}
\end{absolutelynopagebreak}

\begin{absolutelynopagebreak}
\setstretch{.7}
{\PaliGlossA{phassaṃ paṭicca.}}\\
\begin{addmargin}[1em]{2em}
\setstretch{.5}
{\PaliGlossB{Dependent on contact.}}\\
\end{addmargin}
\end{absolutelynopagebreak}

\begin{absolutelynopagebreak}
\setstretch{.7}
{\PaliGlossA{iti vadaṃ vuttavādī ceva bhagavato assa, na ca bhagavantaṃ abhūtena abbhācikkheyya, dhammassa cānudhammaṃ byākareyya, na ca koci sahadhammiko vādānupāto gārayhaṃ ṭhānaṃ āgaccheyya.}}\\
\begin{addmargin}[1em]{2em}
\setstretch{.5}
{\PaliGlossB{If you said this you would repeat what the Buddha has said, and not misrepresent him with an untruth. You would explain in line with his teaching, and there would be no legitimate grounds for rebuke and criticism.}}\\
\end{addmargin}
\end{absolutelynopagebreak}

\begin{absolutelynopagebreak}
\setstretch{.7}
{\PaliGlossA{tatrāvuso, ye te samaṇabrāhmaṇā kammavādā sayaṅkataṃ sukhadukkhaṃ paññapenti, tadapi phassapaccayā.}}\\
\begin{addmargin}[1em]{2em}
\setstretch{.5}
{\PaliGlossB{Consider the ascetics and brahmins who teach the efficacy of deeds. In the case of those who declare that pleasure and pain are made by oneself, that’s conditioned by contact. …}}\\
\end{addmargin}
\end{absolutelynopagebreak}

\begin{absolutelynopagebreak}
\setstretch{.7}
{\PaliGlossA{yepi te … pe …}}\\
\begin{addmargin}[1em]{2em}
\setstretch{.5}
{\PaliGlossB{    -}}\\
\end{addmargin}
\end{absolutelynopagebreak}

\begin{absolutelynopagebreak}
\setstretch{.7}
{\PaliGlossA{yepi te … pe …}}\\
\begin{addmargin}[1em]{2em}
\setstretch{.5}
{\PaliGlossB{    -}}\\
\end{addmargin}
\end{absolutelynopagebreak}

\begin{absolutelynopagebreak}
\setstretch{.7}
{\PaliGlossA{yepi te samaṇabrāhmaṇā kammavādā asayaṅkāraṃ aparaṅkāraṃ adhiccasamuppannaṃ sukhadukkhaṃ paññapenti, tadapi phassapaccayā.}}\\
\begin{addmargin}[1em]{2em}
\setstretch{.5}
{\PaliGlossB{In the case of those who declare that pleasure and pain arise by chance, not made by oneself or another, that’s also conditioned by contact.}}\\
\end{addmargin}
\end{absolutelynopagebreak}

\begin{absolutelynopagebreak}
\setstretch{.7}
{\PaliGlossA{tatrāvuso, ye te samaṇabrāhmaṇā kammavādā sayaṅkataṃ sukhadukkhaṃ paññapenti, te vata aññatra phassā paṭisaṃvedissantīti netaṃ ṭhānaṃ vijjati.}}\\
\begin{addmargin}[1em]{2em}
\setstretch{.5}
{\PaliGlossB{Consider the ascetics and brahmins who teach the efficacy of deeds. In the case of those who declare that pleasure and pain are made by oneself, it’s impossible that they will experience that without contact.}}\\
\end{addmargin}
\end{absolutelynopagebreak}

\begin{absolutelynopagebreak}
\setstretch{.7}
{\PaliGlossA{yepi te … pe …}}\\
\begin{addmargin}[1em]{2em}
\setstretch{.5}
{\PaliGlossB{    -}}\\
\end{addmargin}
\end{absolutelynopagebreak}

\begin{absolutelynopagebreak}
\setstretch{.7}
{\PaliGlossA{yepi te … pe …}}\\
\begin{addmargin}[1em]{2em}
\setstretch{.5}
{\PaliGlossB{    -}}\\
\end{addmargin}
\end{absolutelynopagebreak}

\begin{absolutelynopagebreak}
\setstretch{.7}
{\PaliGlossA{yepi te samaṇabrāhmaṇā kammavādā asayaṅkāraṃ aparaṅkāraṃ adhiccasamuppannaṃ sukhadukkhaṃ paññapenti, te vata aññatra phassā paṭisaṃvedissantīti netaṃ ṭhānaṃ vijjatī”ti.}}\\
\begin{addmargin}[1em]{2em}
\setstretch{.5}
{\PaliGlossB{In the case of those who declare that pleasure and pain arise by chance, not made by oneself or another, it’s impossible that they will experience that without contact.”}}\\
\end{addmargin}
\end{absolutelynopagebreak}

\begin{absolutelynopagebreak}
\setstretch{.7}
{\PaliGlossA{assosi kho āyasmā ānando āyasmato sāriputtassa āyasmatā bhūmijena saddhiṃ imaṃ kathāsallāpaṃ.}}\\
\begin{addmargin}[1em]{2em}
\setstretch{.5}
{\PaliGlossB{Venerable Ānanda heard this discussion between Venerable Sāriputta and Venerable Bhūmija.}}\\
\end{addmargin}
\end{absolutelynopagebreak}

\begin{absolutelynopagebreak}
\setstretch{.7}
{\PaliGlossA{atha kho āyasmā ānando yena bhagavā tenupasaṅkami; upasaṅkamitvā bhagavantaṃ abhivādetvā ekamantaṃ nisīdi.}}\\
\begin{addmargin}[1em]{2em}
\setstretch{.5}
{\PaliGlossB{Then Venerable Ānanda went up to the Buddha, bowed, sat down to one side,}}\\
\end{addmargin}
\end{absolutelynopagebreak}

\begin{absolutelynopagebreak}
\setstretch{.7}
{\PaliGlossA{ekamantaṃ nisinno kho āyasmā ānando yāvatako āyasmato sāriputtassa āyasmatā bhūmijena saddhiṃ ahosi kathāsallāpo taṃ sabbaṃ bhagavato ārocesi.}}\\
\begin{addmargin}[1em]{2em}
\setstretch{.5}
{\PaliGlossB{and informed the Buddha of all they had discussed.}}\\
\end{addmargin}
\end{absolutelynopagebreak}

\begin{absolutelynopagebreak}
\setstretch{.7}
{\PaliGlossA{“sādhu sādhu, ānanda, yathā taṃ sāriputto sammā byākaramāno byākareyya.}}\\
\begin{addmargin}[1em]{2em}
\setstretch{.5}
{\PaliGlossB{“Good, good, Ānanda! It’s just as Sāriputta has so rightly explained.}}\\
\end{addmargin}
\end{absolutelynopagebreak}

\begin{absolutelynopagebreak}
\setstretch{.7}
{\PaliGlossA{paṭiccasamuppannaṃ kho, ānanda, sukhadukkhaṃ vuttaṃ mayā.}}\\
\begin{addmargin}[1em]{2em}
\setstretch{.5}
{\PaliGlossB{I have said that pleasure and pain are dependently originated.}}\\
\end{addmargin}
\end{absolutelynopagebreak}

\begin{absolutelynopagebreak}
\setstretch{.7}
{\PaliGlossA{kiṃ paṭicca?}}\\
\begin{addmargin}[1em]{2em}
\setstretch{.5}
{\PaliGlossB{Dependent on what?}}\\
\end{addmargin}
\end{absolutelynopagebreak}

\begin{absolutelynopagebreak}
\setstretch{.7}
{\PaliGlossA{phassaṃ paṭicca.}}\\
\begin{addmargin}[1em]{2em}
\setstretch{.5}
{\PaliGlossB{Dependent on contact.}}\\
\end{addmargin}
\end{absolutelynopagebreak}

\begin{absolutelynopagebreak}
\setstretch{.7}
{\PaliGlossA{iti vadaṃ vuttavādī ceva me assa, na ca maṃ abhūtena abbhācikkheyya, dhammassa cānudhammaṃ byākareyya, na ca koci sahadhammiko vādānupāto gārayhaṃ ṭhānaṃ āgaccheyya.}}\\
\begin{addmargin}[1em]{2em}
\setstretch{.5}
{\PaliGlossB{Saying this you would repeat what I have said, and not misrepresent me with an untruth. You would explain in line with my teaching, and there would be no legitimate grounds for rebuke and criticism.}}\\
\end{addmargin}
\end{absolutelynopagebreak}

\begin{absolutelynopagebreak}
\setstretch{.7}
{\PaliGlossA{tatrānanda, ye te samaṇabrāhmaṇā kammavādā sayaṅkataṃ sukhadukkhaṃ paññapenti tadapi phassapaccayā.}}\\
\begin{addmargin}[1em]{2em}
\setstretch{.5}
{\PaliGlossB{Consider the ascetics and brahmins who teach the efficacy of deeds. In the case of those who declare that pleasure and pain are made by oneself, that’s conditioned by contact. …}}\\
\end{addmargin}
\end{absolutelynopagebreak}

\begin{absolutelynopagebreak}
\setstretch{.7}
{\PaliGlossA{yepi te … pe …}}\\
\begin{addmargin}[1em]{2em}
\setstretch{.5}
{\PaliGlossB{    -}}\\
\end{addmargin}
\end{absolutelynopagebreak}

\begin{absolutelynopagebreak}
\setstretch{.7}
{\PaliGlossA{yepi te … pe …}}\\
\begin{addmargin}[1em]{2em}
\setstretch{.5}
{\PaliGlossB{    -}}\\
\end{addmargin}
\end{absolutelynopagebreak}

\begin{absolutelynopagebreak}
\setstretch{.7}
{\PaliGlossA{yepi te samaṇabrāhmaṇā kammavādā asayaṅkāraṃ aparaṅkāraṃ adhiccasamuppannaṃ sukhadukkhaṃ paññapenti tadapi phassapaccayā.}}\\
\begin{addmargin}[1em]{2em}
\setstretch{.5}
{\PaliGlossB{In the case of those who declare that pleasure and pain arise by chance, not made by oneself or another, that’s also conditioned by contact.}}\\
\end{addmargin}
\end{absolutelynopagebreak}

\begin{absolutelynopagebreak}
\setstretch{.7}
{\PaliGlossA{tatrānanda, ye te samaṇabrāhmaṇā kammavādā sayaṅkataṃ sukhadukkhaṃ paññapenti, te vata aññatra phassā paṭisaṃvedissantīti netaṃ ṭhānaṃ vijjati.}}\\
\begin{addmargin}[1em]{2em}
\setstretch{.5}
{\PaliGlossB{Consider the ascetics and brahmins who teach the efficacy of deeds. In the case of those who declare that pleasure and pain are made by oneself, it’s impossible that they will experience that without contact.}}\\
\end{addmargin}
\end{absolutelynopagebreak}

\begin{absolutelynopagebreak}
\setstretch{.7}
{\PaliGlossA{yepi te … pe …}}\\
\begin{addmargin}[1em]{2em}
\setstretch{.5}
{\PaliGlossB{    -}}\\
\end{addmargin}
\end{absolutelynopagebreak}

\begin{absolutelynopagebreak}
\setstretch{.7}
{\PaliGlossA{yepi te … pe …}}\\
\begin{addmargin}[1em]{2em}
\setstretch{.5}
{\PaliGlossB{    -}}\\
\end{addmargin}
\end{absolutelynopagebreak}

\begin{absolutelynopagebreak}
\setstretch{.7}
{\PaliGlossA{yepi te samaṇabrāhmaṇā kammavādā asayaṅkāraṃ aparaṅkāraṃ adhiccasamuppannaṃ sukhadukkhaṃ paññapenti, te vata aññatra phassā paṭisaṃvedissantīti netaṃ ṭhānaṃ vijjati.}}\\
\begin{addmargin}[1em]{2em}
\setstretch{.5}
{\PaliGlossB{In the case of those who declare that pleasure and pain arise by chance, not made by oneself or another, it’s impossible that they will experience that without contact.}}\\
\end{addmargin}
\end{absolutelynopagebreak}

\begin{absolutelynopagebreak}
\setstretch{.7}
{\PaliGlossA{kāye vā hānanda, sati kāyasañcetanāhetu uppajjati ajjhattaṃ sukhadukkhaṃ.}}\\
\begin{addmargin}[1em]{2em}
\setstretch{.5}
{\PaliGlossB{Ānanda, as long as there’s a body, the intention that gives rise to bodily action causes pleasure and pain to arise in oneself.}}\\
\end{addmargin}
\end{absolutelynopagebreak}

\begin{absolutelynopagebreak}
\setstretch{.7}
{\PaliGlossA{vācāya vā hānanda, sati vacīsañcetanāhetu uppajjati ajjhattaṃ sukhadukkhaṃ.}}\\
\begin{addmargin}[1em]{2em}
\setstretch{.5}
{\PaliGlossB{As long as there’s a voice, the intention that gives rise to verbal action causes pleasure and pain to arise in oneself.}}\\
\end{addmargin}
\end{absolutelynopagebreak}

\begin{absolutelynopagebreak}
\setstretch{.7}
{\PaliGlossA{mane vā hānanda, sati manosañcetanāhetu uppajjati ajjhattaṃ sukhadukkhaṃ avijjāpaccayā ca.}}\\
\begin{addmargin}[1em]{2em}
\setstretch{.5}
{\PaliGlossB{As long as there’s a mind, the intention that gives rise to mental action causes pleasure and pain to arise in oneself. But these only apply when conditioned by ignorance.}}\\
\end{addmargin}
\end{absolutelynopagebreak}

\begin{absolutelynopagebreak}
\setstretch{.7}
{\PaliGlossA{sāmaṃ vā taṃ, ānanda, kāyasaṅkhāraṃ abhisaṅkharoti, yaṃpaccayāssa taṃ uppajjati ajjhattaṃ sukhadukkhaṃ.}}\\
\begin{addmargin}[1em]{2em}
\setstretch{.5}
{\PaliGlossB{By oneself one instigates the choice that gives rise to bodily, verbal, and mental action, conditioned by which that pleasure and pain arise in oneself.}}\\
\end{addmargin}
\end{absolutelynopagebreak}

\begin{absolutelynopagebreak}
\setstretch{.7}
{\PaliGlossA{pare vā taṃ, ānanda, kāyasaṅkhāraṃ abhisaṅkharonti, yaṃpaccayāssa taṃ uppajjati ajjhattaṃ sukhadukkhaṃ.}}\\
\begin{addmargin}[1em]{2em}
\setstretch{.5}
{\PaliGlossB{Or else others instigate the choice …}}\\
\end{addmargin}
\end{absolutelynopagebreak}

\begin{absolutelynopagebreak}
\setstretch{.7}
{\PaliGlossA{sampajāno vā taṃ, ānanda, kāyasaṅkhāraṃ abhisaṅkharoti yaṃpaccayāssa taṃ uppajjati ajjhattaṃ sukhadukkhaṃ.}}\\
\begin{addmargin}[1em]{2em}
\setstretch{.5}
{\PaliGlossB{One consciously instigates the choice …}}\\
\end{addmargin}
\end{absolutelynopagebreak}

\begin{absolutelynopagebreak}
\setstretch{.7}
{\PaliGlossA{asampajāno vā taṃ, ānanda, kāyasaṅkhāraṃ abhisaṅkharoti yaṃpaccayāssa taṃ uppajjati ajjhattaṃ sukhadukkhaṃ.}}\\
\begin{addmargin}[1em]{2em}
\setstretch{.5}
{\PaliGlossB{Or else one unconsciously instigates the choice …}}\\
\end{addmargin}
\end{absolutelynopagebreak}

\begin{absolutelynopagebreak}
\setstretch{.7}
{\PaliGlossA{sāmaṃ vā taṃ, ānanda, vacīsaṅkhāraṃ abhisaṅkharoti yaṃpaccayāssa taṃ uppajjati ajjhattaṃ sukhadukkhaṃ.}}\\
\begin{addmargin}[1em]{2em}
\setstretch{.5}
{\PaliGlossB{    -}}\\
\end{addmargin}
\end{absolutelynopagebreak}

\begin{absolutelynopagebreak}
\setstretch{.7}
{\PaliGlossA{pare vā taṃ, ānanda, vacīsaṅkhāraṃ abhisaṅkharonti yaṃpaccayāssa taṃ uppajjati ajjhattaṃ sukhadukkhaṃ.}}\\
\begin{addmargin}[1em]{2em}
\setstretch{.5}
{\PaliGlossB{    -}}\\
\end{addmargin}
\end{absolutelynopagebreak}

\begin{absolutelynopagebreak}
\setstretch{.7}
{\PaliGlossA{sampajāno vā taṃ, ānanda … pe …}}\\
\begin{addmargin}[1em]{2em}
\setstretch{.5}
{\PaliGlossB{    -}}\\
\end{addmargin}
\end{absolutelynopagebreak}

\begin{absolutelynopagebreak}
\setstretch{.7}
{\PaliGlossA{asampajāno vā taṃ, ānanda, vacīsaṅkhāraṃ abhisaṅkharoti yaṃpaccayāssa taṃ uppajjati ajjhattaṃ sukhadukkhaṃ.}}\\
\begin{addmargin}[1em]{2em}
\setstretch{.5}
{\PaliGlossB{    -}}\\
\end{addmargin}
\end{absolutelynopagebreak}

\begin{absolutelynopagebreak}
\setstretch{.7}
{\PaliGlossA{sāmaṃ vā taṃ, ānanda, manosaṅkhāraṃ abhisaṅkharoti yaṃpaccayāssa taṃ uppajjati ajjhattaṃ sukhadukkhaṃ.}}\\
\begin{addmargin}[1em]{2em}
\setstretch{.5}
{\PaliGlossB{    -}}\\
\end{addmargin}
\end{absolutelynopagebreak}

\begin{absolutelynopagebreak}
\setstretch{.7}
{\PaliGlossA{pare vā taṃ, ānanda, manosaṅkhāraṃ abhisaṅkharonti yaṃpaccayāssa taṃ uppajjati ajjhattaṃ sukhadukkhaṃ.}}\\
\begin{addmargin}[1em]{2em}
\setstretch{.5}
{\PaliGlossB{    -}}\\
\end{addmargin}
\end{absolutelynopagebreak}

\begin{absolutelynopagebreak}
\setstretch{.7}
{\PaliGlossA{sampajāno vā taṃ, ānanda … pe …}}\\
\begin{addmargin}[1em]{2em}
\setstretch{.5}
{\PaliGlossB{    -}}\\
\end{addmargin}
\end{absolutelynopagebreak}

\begin{absolutelynopagebreak}
\setstretch{.7}
{\PaliGlossA{asampajāno vā taṃ, ānanda, manosaṅkhāraṃ abhisaṅkharoti yaṃpaccayāssa taṃ uppajjati ajjhattaṃ sukhadukkhaṃ.}}\\
\begin{addmargin}[1em]{2em}
\setstretch{.5}
{\PaliGlossB{    -}}\\
\end{addmargin}
\end{absolutelynopagebreak}

\begin{absolutelynopagebreak}
\setstretch{.7}
{\PaliGlossA{imesu, ānanda, dhammesu avijjā anupatitā.}}\\
\begin{addmargin}[1em]{2em}
\setstretch{.5}
{\PaliGlossB{Ignorance is included in all these things.}}\\
\end{addmargin}
\end{absolutelynopagebreak}

\begin{absolutelynopagebreak}
\setstretch{.7}
{\PaliGlossA{avijjāya tveva, ānanda, asesavirāganirodhā so kāyo na hoti yaṃpaccayāssa taṃ uppajjati ajjhattaṃ sukhadukkhaṃ. sā vācā na hoti yaṃpaccayāssa taṃ uppajjati ajjhattaṃ sukhadukkhaṃ. so mano na hoti yaṃpaccayāssa taṃ uppajjati ajjhattaṃ sukhadukkhaṃ.}}\\
\begin{addmargin}[1em]{2em}
\setstretch{.5}
{\PaliGlossB{But when ignorance fades away and ceases with nothing left over, there is no body and no voice and no mind, conditioned by which that pleasure and pain arise in oneself.}}\\
\end{addmargin}
\end{absolutelynopagebreak}

\begin{absolutelynopagebreak}
\setstretch{.7}
{\PaliGlossA{khettaṃ taṃ na hoti … pe … vatthu taṃ na hoti … pe … āyatanaṃ taṃ na hoti … pe … adhikaraṇaṃ taṃ na hoti yaṃpaccayāssa taṃ uppajjati ajjhattaṃ sukhadukkhan”ti.}}\\
\begin{addmargin}[1em]{2em}
\setstretch{.5}
{\PaliGlossB{There is no field, no ground, no scope, no basis, conditioned by which that pleasure and pain arise in oneself.”}}\\
\end{addmargin}
\end{absolutelynopagebreak}

\begin{absolutelynopagebreak}
\setstretch{.7}
{\PaliGlossA{pañcamaṃ.}}\\
\begin{addmargin}[1em]{2em}
\setstretch{.5}
{\PaliGlossB{    -}}\\
\end{addmargin}
\end{absolutelynopagebreak}
