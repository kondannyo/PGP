
\begin{absolutelynopagebreak}
\setstretch{.7}
{\PaliGlossA{saṃyutta nikāya 47}}\\
\begin{addmargin}[1em]{2em}
\setstretch{.5}
{\PaliGlossB{Linked Discourses 47}}\\
\end{addmargin}
\end{absolutelynopagebreak}

\begin{absolutelynopagebreak}
\setstretch{.7}
{\PaliGlossA{2. nālandavagga}}\\
\begin{addmargin}[1em]{2em}
\setstretch{.5}
{\PaliGlossB{2. At Nālandā}}\\
\end{addmargin}
\end{absolutelynopagebreak}

\begin{absolutelynopagebreak}
\setstretch{.7}
{\PaliGlossA{12. nālandasutta}}\\
\begin{addmargin}[1em]{2em}
\setstretch{.5}
{\PaliGlossB{12. At Nālandā}}\\
\end{addmargin}
\end{absolutelynopagebreak}

\begin{absolutelynopagebreak}
\setstretch{.7}
{\PaliGlossA{ekaṃ samayaṃ bhagavā nālandāyaṃ viharati pāvārikambavane.}}\\
\begin{addmargin}[1em]{2em}
\setstretch{.5}
{\PaliGlossB{At one time the Buddha was staying near Nālandā in Pāvārika’s mango grove.}}\\
\end{addmargin}
\end{absolutelynopagebreak}

\begin{absolutelynopagebreak}
\setstretch{.7}
{\PaliGlossA{atha kho āyasmā sāriputto yena bhagavā tenupasaṅkami; upasaṅkamitvā bhagavantaṃ abhivādetvā ekamantaṃ nisīdi. ekamantaṃ nisinno kho āyasmā sāriputto bhagavantaṃ etadavoca:}}\\
\begin{addmargin}[1em]{2em}
\setstretch{.5}
{\PaliGlossB{Then Sāriputta went up to the Buddha, bowed, sat down to one side, and said to him:}}\\
\end{addmargin}
\end{absolutelynopagebreak}

\begin{absolutelynopagebreak}
\setstretch{.7}
{\PaliGlossA{“evaṃpasanno ahaṃ, bhante, bhagavati.}}\\
\begin{addmargin}[1em]{2em}
\setstretch{.5}
{\PaliGlossB{“Sir, I have such confidence in the Buddha that}}\\
\end{addmargin}
\end{absolutelynopagebreak}

\begin{absolutelynopagebreak}
\setstretch{.7}
{\PaliGlossA{na cāhu, na ca bhavissati, na cetarahi vijjati añño samaṇo vā brāhmaṇo vā bhagavatā bhiyyobhiññataro, yadidaṃ—sambodhiyan”ti.}}\\
\begin{addmargin}[1em]{2em}
\setstretch{.5}
{\PaliGlossB{I believe there’s no other ascetic or brahmin—whether past, future, or present—whose direct knowledge is superior to the Buddha when it comes to awakening.”}}\\
\end{addmargin}
\end{absolutelynopagebreak}

\begin{absolutelynopagebreak}
\setstretch{.7}
{\PaliGlossA{“uḷārā kho tyāyaṃ, sāriputta, āsabhī vācā bhāsitā, ekaṃso gahito, sīhanādo nadito:}}\\
\begin{addmargin}[1em]{2em}
\setstretch{.5}
{\PaliGlossB{“That’s a grand and dramatic statement, Sāriputta. You’ve roared a definitive, categorical lion’s roar, saying:}}\\
\end{addmargin}
\end{absolutelynopagebreak}

\begin{absolutelynopagebreak}
\setstretch{.7}
{\PaliGlossA{‘evaṃpasanno ahaṃ, bhante, bhagavati.}}\\
\begin{addmargin}[1em]{2em}
\setstretch{.5}
{\PaliGlossB{‘I have such confidence in the Buddha that}}\\
\end{addmargin}
\end{absolutelynopagebreak}

\begin{absolutelynopagebreak}
\setstretch{.7}
{\PaliGlossA{na cāhu, na ca bhavissati na cetarahi vijjati añño samaṇo vā brāhmaṇo vā bhagavatā bhiyyobhiññataro, yadidaṃ—sambodhiyan’ti.}}\\
\begin{addmargin}[1em]{2em}
\setstretch{.5}
{\PaliGlossB{I believe there’s no other ascetic or brahmin—whether past, future, or present—whose direct knowledge is superior to the Buddha when it comes to awakening.’}}\\
\end{addmargin}
\end{absolutelynopagebreak}

\begin{absolutelynopagebreak}
\setstretch{.7}
{\PaliGlossA{kiṃ nu te, sāriputta, ye te ahesuṃ atītamaddhānaṃ arahanto sammāsambuddhā, sabbe te bhagavanto cetasā ceto paricca viditā:}}\\
\begin{addmargin}[1em]{2em}
\setstretch{.5}
{\PaliGlossB{What about all the perfected ones, the fully awakened Buddhas who lived in the past? Have you comprehended their minds to know that}}\\
\end{addmargin}
\end{absolutelynopagebreak}

\begin{absolutelynopagebreak}
\setstretch{.7}
{\PaliGlossA{‘evaṃsīlā te bhagavanto ahesuṃ’ iti vā, ‘evaṃdhammā te bhagavanto ahesuṃ’ iti vā, ‘evaṃpaññā te bhagavanto ahesuṃ’ iti vā, ‘evaṃvihārino te bhagavanto ahesuṃ’ iti vā, ‘evaṃvimuttā te bhagavanto ahesuṃ’ iti vā”ti?}}\\
\begin{addmargin}[1em]{2em}
\setstretch{.5}
{\PaliGlossB{those Buddhas had such ethics, or such qualities, or such wisdom, or such meditation, or such freedom?”}}\\
\end{addmargin}
\end{absolutelynopagebreak}

\begin{absolutelynopagebreak}
\setstretch{.7}
{\PaliGlossA{“no hetaṃ, bhante”.}}\\
\begin{addmargin}[1em]{2em}
\setstretch{.5}
{\PaliGlossB{“No, sir.”}}\\
\end{addmargin}
\end{absolutelynopagebreak}

\begin{absolutelynopagebreak}
\setstretch{.7}
{\PaliGlossA{“kiṃ pana te, sāriputta, ye te bhavissanti anāgatamaddhānaṃ arahanto sammāsambuddhā, sabbe te bhagavanto cetasā ceto paricca viditā:}}\\
\begin{addmargin}[1em]{2em}
\setstretch{.5}
{\PaliGlossB{“And what about all the perfected ones, the fully awakened Buddhas who will live in the future? Have you comprehended their minds to know that}}\\
\end{addmargin}
\end{absolutelynopagebreak}

\begin{absolutelynopagebreak}
\setstretch{.7}
{\PaliGlossA{‘evaṃsīlā te bhagavanto bhavissanti’ iti vā, ‘evaṃdhammā te bhagavanto bhavissanti’ iti vā, ‘evaṃpaññā te bhagavanto bhavissanti’ iti vā, ‘evaṃvihārino te bhagavanto bhavissanti’ iti vā, ‘evaṃvimuttā te bhagavanto bhavissanti’ iti vā”ti?}}\\
\begin{addmargin}[1em]{2em}
\setstretch{.5}
{\PaliGlossB{those Buddhas will have such ethics, or such qualities, or such wisdom, or such meditation, or such freedom?”}}\\
\end{addmargin}
\end{absolutelynopagebreak}

\begin{absolutelynopagebreak}
\setstretch{.7}
{\PaliGlossA{“no hetaṃ, bhante”.}}\\
\begin{addmargin}[1em]{2em}
\setstretch{.5}
{\PaliGlossB{“No, sir.”}}\\
\end{addmargin}
\end{absolutelynopagebreak}

\begin{absolutelynopagebreak}
\setstretch{.7}
{\PaliGlossA{“kiṃ pana tyāhaṃ, sāriputta, etarahi, arahaṃ sammāsambuddho cetasā ceto paricca vidito:}}\\
\begin{addmargin}[1em]{2em}
\setstretch{.5}
{\PaliGlossB{“And what about me, the perfected one, the fully awakened Buddha at present? Have you comprehended my mind to know that}}\\
\end{addmargin}
\end{absolutelynopagebreak}

\begin{absolutelynopagebreak}
\setstretch{.7}
{\PaliGlossA{‘evaṃsīlo bhagavā’ iti vā, ‘evaṃdhammo bhagavā’ iti vā, ‘evaṃpañño bhagavā’ iti vā, ‘evaṃvihārī bhagavā’ iti vā, ‘evaṃvimutto bhagavā’ iti vā”ti?}}\\
\begin{addmargin}[1em]{2em}
\setstretch{.5}
{\PaliGlossB{I have such ethics, or such qualities, or such wisdom, or such meditation, or such freedom?”}}\\
\end{addmargin}
\end{absolutelynopagebreak}

\begin{absolutelynopagebreak}
\setstretch{.7}
{\PaliGlossA{“no hetaṃ, bhante”.}}\\
\begin{addmargin}[1em]{2em}
\setstretch{.5}
{\PaliGlossB{“No, sir.”}}\\
\end{addmargin}
\end{absolutelynopagebreak}

\begin{absolutelynopagebreak}
\setstretch{.7}
{\PaliGlossA{“ettha ca te, sāriputta, atītānāgatapaccuppannesu arahantesu sammāsambuddhesu cetopariyañāṇaṃ natthi.}}\\
\begin{addmargin}[1em]{2em}
\setstretch{.5}
{\PaliGlossB{“Well then, Sāriputta, given that you don’t comprehend the minds of Buddhas past, future, or present,}}\\
\end{addmargin}
\end{absolutelynopagebreak}

\begin{absolutelynopagebreak}
\setstretch{.7}
{\PaliGlossA{atha kiñcarahi tyāyaṃ, sāriputta, uḷārā āsabhī vācā bhāsitā, ekaṃso gahito, sīhanādo nadito:}}\\
\begin{addmargin}[1em]{2em}
\setstretch{.5}
{\PaliGlossB{what exactly are you doing, making such a grand and dramatic statement, roaring such a definitive, categorical lion’s roar?”}}\\
\end{addmargin}
\end{absolutelynopagebreak}

\begin{absolutelynopagebreak}
\setstretch{.7}
{\PaliGlossA{‘evaṃpasanno ahaṃ, bhante, bhagavati.}}\\
\begin{addmargin}[1em]{2em}
\setstretch{.5}
{\PaliGlossB{    -}}\\
\end{addmargin}
\end{absolutelynopagebreak}

\begin{absolutelynopagebreak}
\setstretch{.7}
{\PaliGlossA{na cāhu, na ca bhavissati, na cetarahi vijjati añño samaṇo vā brāhmaṇo vā bhagavatā’ bhiyyobhiññataro, yadidaṃ—sambodhiyan”ti?}}\\
\begin{addmargin}[1em]{2em}
\setstretch{.5}
{\PaliGlossB{    -}}\\
\end{addmargin}
\end{absolutelynopagebreak}

\begin{absolutelynopagebreak}
\setstretch{.7}
{\PaliGlossA{“na kho me, bhante, atītānāgatapaccuppannesu arahantesu sammāsambuddhesu cetopariyañāṇaṃ atthi,}}\\
\begin{addmargin}[1em]{2em}
\setstretch{.5}
{\PaliGlossB{“Sir, though I don’t comprehend the minds of Buddhas past, future, and present,}}\\
\end{addmargin}
\end{absolutelynopagebreak}

\begin{absolutelynopagebreak}
\setstretch{.7}
{\PaliGlossA{api ca me dhammanvayo vidito.}}\\
\begin{addmargin}[1em]{2em}
\setstretch{.5}
{\PaliGlossB{still I understand this by inference from the teaching.}}\\
\end{addmargin}
\end{absolutelynopagebreak}

\begin{absolutelynopagebreak}
\setstretch{.7}
{\PaliGlossA{seyyathāpi, bhante, rañño paccantimaṃ nagaraṃ daḷhuddhāpaṃ daḷhapākāratoraṇaṃ ekadvāraṃ.}}\\
\begin{addmargin}[1em]{2em}
\setstretch{.5}
{\PaliGlossB{Suppose there was a king’s frontier citadel with fortified embankments, ramparts, and arches, and a single gate.}}\\
\end{addmargin}
\end{absolutelynopagebreak}

\begin{absolutelynopagebreak}
\setstretch{.7}
{\PaliGlossA{tatrassa dovāriko paṇḍito byatto medhāvī aññātānaṃ nivāretā ñātānaṃ pavesetā.}}\\
\begin{addmargin}[1em]{2em}
\setstretch{.5}
{\PaliGlossB{And it has a gatekeeper who is astute, competent, and intelligent. He keeps strangers out and lets known people in.}}\\
\end{addmargin}
\end{absolutelynopagebreak}

\begin{absolutelynopagebreak}
\setstretch{.7}
{\PaliGlossA{so tassa nagarassa samantā anupariyāyapathaṃ anukkamamāno na passeyya pākārasandhiṃ vā pākāravivaraṃ vā, antamaso biḷāranikkhamanamattampi.}}\\
\begin{addmargin}[1em]{2em}
\setstretch{.5}
{\PaliGlossB{As he walks around the patrol path, he doesn’t see a hole or cleft in the wall, not even one big enough for a cat to slip out.}}\\
\end{addmargin}
\end{absolutelynopagebreak}

\begin{absolutelynopagebreak}
\setstretch{.7}
{\PaliGlossA{tassa evamassa:}}\\
\begin{addmargin}[1em]{2em}
\setstretch{.5}
{\PaliGlossB{He thinks,}}\\
\end{addmargin}
\end{absolutelynopagebreak}

\begin{absolutelynopagebreak}
\setstretch{.7}
{\PaliGlossA{‘ye kho keci oḷārikā pāṇā imaṃ nagaraṃ pavisanti vā nikkhamanti vā, sabbe te imināva dvārena pavisanti vā nikkhamanti vā’ti.}}\\
\begin{addmargin}[1em]{2em}
\setstretch{.5}
{\PaliGlossB{‘Whatever sizable creatures enter or leave the citadel, all of them do so via this gate.’}}\\
\end{addmargin}
\end{absolutelynopagebreak}

\begin{absolutelynopagebreak}
\setstretch{.7}
{\PaliGlossA{evameva kho me, bhante, dhammanvayo vidito:}}\\
\begin{addmargin}[1em]{2em}
\setstretch{.5}
{\PaliGlossB{In the same way, I understand this by inference from the teaching:}}\\
\end{addmargin}
\end{absolutelynopagebreak}

\begin{absolutelynopagebreak}
\setstretch{.7}
{\PaliGlossA{‘yepi te, bhante, ahesuṃ atītamaddhānaṃ arahanto sammāsambuddhā, sabbe te bhagavanto pañca nīvaraṇe pahāya, cetaso upakkilese paññāya dubbalīkaraṇe, catūsu satipaṭṭhānesu suppatiṭṭhitacittā, satta bojjhaṅge yathābhūtaṃ bhāvetvā, anuttaraṃ sammāsambodhiṃ abhisambujjhiṃsu.}}\\
\begin{addmargin}[1em]{2em}
\setstretch{.5}
{\PaliGlossB{‘All the perfected ones, fully awakened Buddhas—whether past, future, or present—give up the five hindrances, corruptions of the heart that weaken wisdom. Their mind is firmly established in the four kinds of mindfulness meditation. They correctly develop the seven awakening factors. And they wake up to the supreme perfect awakening.’”}}\\
\end{addmargin}
\end{absolutelynopagebreak}

\begin{absolutelynopagebreak}
\setstretch{.7}
{\PaliGlossA{yepi te, bhante, bhavissanti anāgatamaddhānaṃ arahanto sammāsambuddhā, sabbe te bhagavanto pañca nīvaraṇe pahāya, cetaso upakkilese paññāya dubbalīkaraṇe, catūsu satipaṭṭhānesu suppatiṭṭhitacittā, satta bojjhaṅge yathābhūtaṃ bhāvetvā, anuttaraṃ sammāsambodhiṃ abhisambujjhissanti.}}\\
\begin{addmargin}[1em]{2em}
\setstretch{.5}
{\PaliGlossB{    -}}\\
\end{addmargin}
\end{absolutelynopagebreak}

\begin{absolutelynopagebreak}
\setstretch{.7}
{\PaliGlossA{bhagavāpi, bhante, etarahi arahaṃ sammāsambuddho pañca nīvaraṇe pahāya, cetaso upakkilese paññāya dubbalīkaraṇe, catūsu satipaṭṭhānesu suppatiṭṭhitacitto, satta bojjhaṅge yathābhūtaṃ bhāvetvā, anuttaraṃ sammāsambodhiṃ abhisambuddho’”ti.}}\\
\begin{addmargin}[1em]{2em}
\setstretch{.5}
{\PaliGlossB{    -}}\\
\end{addmargin}
\end{absolutelynopagebreak}

\begin{absolutelynopagebreak}
\setstretch{.7}
{\PaliGlossA{“sādhu sādhu, sāriputta.}}\\
\begin{addmargin}[1em]{2em}
\setstretch{.5}
{\PaliGlossB{“Good, good, Sāriputta!}}\\
\end{addmargin}
\end{absolutelynopagebreak}

\begin{absolutelynopagebreak}
\setstretch{.7}
{\PaliGlossA{tasmātiha tvaṃ, sāriputta, imaṃ dhammapariyāyaṃ abhikkhaṇaṃ bhāseyyāsi bhikkhūnaṃ bhikkhunīnaṃ upāsakānaṃ upāsikānaṃ.}}\\
\begin{addmargin}[1em]{2em}
\setstretch{.5}
{\PaliGlossB{So Sāriputta, you should frequently speak this exposition of the teaching to the monks, nuns, laymen, and laywomen.}}\\
\end{addmargin}
\end{absolutelynopagebreak}

\begin{absolutelynopagebreak}
\setstretch{.7}
{\PaliGlossA{yesampi hi, sāriputta, moghapurisānaṃ bhavissati tathāgate kaṅkhā vā vimati vā, tesampimaṃ dhammapariyāyaṃ sutvā yā tathāgate kaṅkhā vā vimati vā sā pahīyissatī”ti.}}\\
\begin{addmargin}[1em]{2em}
\setstretch{.5}
{\PaliGlossB{Though there will be some foolish people who have doubt or uncertainty regarding the Realized One, when they hear this exposition of the teaching they’ll give up that doubt or uncertainty.”}}\\
\end{addmargin}
\end{absolutelynopagebreak}

\begin{absolutelynopagebreak}
\setstretch{.7}
{\PaliGlossA{dutiyaṃ.}}\\
\begin{addmargin}[1em]{2em}
\setstretch{.5}
{\PaliGlossB{    -}}\\
\end{addmargin}
\end{absolutelynopagebreak}
