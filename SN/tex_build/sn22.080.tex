
\begin{absolutelynopagebreak}
\setstretch{.7}
{\PaliGlossA{saṃyutta nikāya 22}}\\
\begin{addmargin}[1em]{2em}
\setstretch{.5}
{\PaliGlossB{Linked Discourses 22}}\\
\end{addmargin}
\end{absolutelynopagebreak}

\begin{absolutelynopagebreak}
\setstretch{.7}
{\PaliGlossA{8. khajjanīyavagga}}\\
\begin{addmargin}[1em]{2em}
\setstretch{.5}
{\PaliGlossB{8. Itchy}}\\
\end{addmargin}
\end{absolutelynopagebreak}

\begin{absolutelynopagebreak}
\setstretch{.7}
{\PaliGlossA{80. piṇḍolyasutta}}\\
\begin{addmargin}[1em]{2em}
\setstretch{.5}
{\PaliGlossB{80. Beggars}}\\
\end{addmargin}
\end{absolutelynopagebreak}

\begin{absolutelynopagebreak}
\setstretch{.7}
{\PaliGlossA{ekaṃ samayaṃ bhagavā sakkesu viharati kapilavatthusmiṃ nigrodhārāme.}}\\
\begin{addmargin}[1em]{2em}
\setstretch{.5}
{\PaliGlossB{At one time the Buddha was staying in the land of the Sakyans, near Kapilavatthu in the Banyan Tree Monastery.}}\\
\end{addmargin}
\end{absolutelynopagebreak}

\begin{absolutelynopagebreak}
\setstretch{.7}
{\PaliGlossA{atha kho bhagavā kismiñcideva pakaraṇe bhikkhusaṅghaṃ paṇāmetvā pubbaṇhasamayaṃ nivāsetvā pattacīvaramādāya kapilavatthuṃ piṇḍāya pāvisi.}}\\
\begin{addmargin}[1em]{2em}
\setstretch{.5}
{\PaliGlossB{Then the Buddha, having dismissed the mendicant Saṅgha for some reason, robed up in the morning and, taking his bowl and robe, entered Kapilavatthu for alms.}}\\
\end{addmargin}
\end{absolutelynopagebreak}

\begin{absolutelynopagebreak}
\setstretch{.7}
{\PaliGlossA{kapilavatthusmiṃ piṇḍāya caritvā pacchābhattaṃ piṇḍapātapaṭikkanto yena mahāvanaṃ tenupasaṅkami divāvihārāya.}}\\
\begin{addmargin}[1em]{2em}
\setstretch{.5}
{\PaliGlossB{He wandered for alms in Kapilavatthu. After the meal, on his return from alms-round, he went to the Great Wood,}}\\
\end{addmargin}
\end{absolutelynopagebreak}

\begin{absolutelynopagebreak}
\setstretch{.7}
{\PaliGlossA{mahāvanaṃ ajjhogāhetvā beluvalaṭṭhikāya mūle divāvihāraṃ nisīdi.}}\\
\begin{addmargin}[1em]{2em}
\setstretch{.5}
{\PaliGlossB{plunged deep into it, and sat at the root of a young wood apple tree for the day’s meditation.}}\\
\end{addmargin}
\end{absolutelynopagebreak}

\begin{absolutelynopagebreak}
\setstretch{.7}
{\PaliGlossA{atha kho bhagavato rahogatassa paṭisallīnassa evaṃ cetaso parivitakko udapādi:}}\\
\begin{addmargin}[1em]{2em}
\setstretch{.5}
{\PaliGlossB{Then as he was in private retreat this thought came to his mind,}}\\
\end{addmargin}
\end{absolutelynopagebreak}

\begin{absolutelynopagebreak}
\setstretch{.7}
{\PaliGlossA{“mayā kho bhikkhusaṅgho pabāḷho.}}\\
\begin{addmargin}[1em]{2em}
\setstretch{.5}
{\PaliGlossB{“I’ve sent the mendicant Saṅgha away.}}\\
\end{addmargin}
\end{absolutelynopagebreak}

\begin{absolutelynopagebreak}
\setstretch{.7}
{\PaliGlossA{santettha bhikkhū navā acirapabbajitā adhunāgatā imaṃ dhammavinayaṃ.}}\\
\begin{addmargin}[1em]{2em}
\setstretch{.5}
{\PaliGlossB{But there are mendicants here who are junior, recently gone forth, newly come to this teaching and training.}}\\
\end{addmargin}
\end{absolutelynopagebreak}

\begin{absolutelynopagebreak}
\setstretch{.7}
{\PaliGlossA{tesaṃ mamaṃ apassantānaṃ siyā aññathattaṃ siyā vipariṇāmo.}}\\
\begin{addmargin}[1em]{2em}
\setstretch{.5}
{\PaliGlossB{Not seeing me they may change and fall apart.}}\\
\end{addmargin}
\end{absolutelynopagebreak}

\begin{absolutelynopagebreak}
\setstretch{.7}
{\PaliGlossA{seyyathāpi nāma vacchassa taruṇassa mātaraṃ apassantassa siyā aññathattaṃ siyā vipariṇāmo;}}\\
\begin{addmargin}[1em]{2em}
\setstretch{.5}
{\PaliGlossB{If a young calf doesn’t see its mother it may change and fall apart. …}}\\
\end{addmargin}
\end{absolutelynopagebreak}

\begin{absolutelynopagebreak}
\setstretch{.7}
{\PaliGlossA{evameva santettha bhikkhū navā acirapabbajitā adhunāgatā imaṃ dhammavinayaṃ tesaṃ mamaṃ apassantānaṃ siyā aññathattaṃ siyā vipariṇāmo.}}\\
\begin{addmargin}[1em]{2em}
\setstretch{.5}
{\PaliGlossB{    -}}\\
\end{addmargin}
\end{absolutelynopagebreak}

\begin{absolutelynopagebreak}
\setstretch{.7}
{\PaliGlossA{seyyathāpi nāma bījānaṃ taruṇānaṃ udakaṃ alabhantānaṃ siyā aññathattaṃ siyā vipariṇāmo;}}\\
\begin{addmargin}[1em]{2em}
\setstretch{.5}
{\PaliGlossB{Or if young seedlings don’t get water they may change and fall apart.}}\\
\end{addmargin}
\end{absolutelynopagebreak}

\begin{absolutelynopagebreak}
\setstretch{.7}
{\PaliGlossA{evameva santettha … pe …}}\\
\begin{addmargin}[1em]{2em}
\setstretch{.5}
{\PaliGlossB{In the same way, there are mendicants here who are junior, recently gone forth, newly come to this teaching and training.}}\\
\end{addmargin}
\end{absolutelynopagebreak}

\begin{absolutelynopagebreak}
\setstretch{.7}
{\PaliGlossA{tesaṃ mamaṃ alabhantānaṃ dassanāya siyā aññathattaṃ siyā vipariṇāmo.}}\\
\begin{addmargin}[1em]{2em}
\setstretch{.5}
{\PaliGlossB{Not seeing me they may change and fall apart.}}\\
\end{addmargin}
\end{absolutelynopagebreak}

\begin{absolutelynopagebreak}
\setstretch{.7}
{\PaliGlossA{yannūnāhaṃ yatheva mayā pubbe bhikkhusaṅgho anuggahito, evameva etarahi anuggaṇheyyaṃ bhikkhusaṅghan”ti.}}\\
\begin{addmargin}[1em]{2em}
\setstretch{.5}
{\PaliGlossB{Why don’t I support the mendicant Saṅgha now as I did in the past?”}}\\
\end{addmargin}
\end{absolutelynopagebreak}

\begin{absolutelynopagebreak}
\setstretch{.7}
{\PaliGlossA{atha kho brahmā sahampati bhagavato cetasā cetoparivitakkamaññāya—seyyathāpi nāma balavā puriso samiñjitaṃ vā bāhaṃ pasāreyya pasāritaṃ vā bāhaṃ samiñjeyya; evameva—brahmaloke antarahito bhagavato purato pāturahosi.}}\\
\begin{addmargin}[1em]{2em}
\setstretch{.5}
{\PaliGlossB{Then Brahmā Sahampati knew what the Buddha was thinking. As easily as a strong person would extend or contract their arm, he vanished from the Brahmā realm and reappeared in front of the Buddha.}}\\
\end{addmargin}
\end{absolutelynopagebreak}

\begin{absolutelynopagebreak}
\setstretch{.7}
{\PaliGlossA{atha kho brahmā sahampati ekaṃsaṃ uttarāsaṅgaṃ karitvā yena bhagavā tenañjaliṃ paṇāmetvā bhagavantaṃ etadavoca:}}\\
\begin{addmargin}[1em]{2em}
\setstretch{.5}
{\PaliGlossB{He arranged his robe over one shoulder, raised his joined palms toward the Buddha, and said:}}\\
\end{addmargin}
\end{absolutelynopagebreak}

\begin{absolutelynopagebreak}
\setstretch{.7}
{\PaliGlossA{“evametaṃ, bhagavā, evametaṃ, sugata.}}\\
\begin{addmargin}[1em]{2em}
\setstretch{.5}
{\PaliGlossB{“That’s so true, Blessed One! That’s so true, Holy One!}}\\
\end{addmargin}
\end{absolutelynopagebreak}

\begin{absolutelynopagebreak}
\setstretch{.7}
{\PaliGlossA{bhagavatā, bhante, bhikkhusaṃgho pabāḷho.}}\\
\begin{addmargin}[1em]{2em}
\setstretch{.5}
{\PaliGlossB{The Buddha has sent the mendicant Saṅgha away.}}\\
\end{addmargin}
\end{absolutelynopagebreak}

\begin{absolutelynopagebreak}
\setstretch{.7}
{\PaliGlossA{santettha bhikkhū navā acirapabbajitā adhunāgatā imaṃ dhammavinayaṃ.}}\\
\begin{addmargin}[1em]{2em}
\setstretch{.5}
{\PaliGlossB{But there are mendicants who are junior, recently gone forth, newly come to this teaching and training. …}}\\
\end{addmargin}
\end{absolutelynopagebreak}

\begin{absolutelynopagebreak}
\setstretch{.7}
{\PaliGlossA{tesaṃ bhagavantaṃ apassantānaṃ siyā aññathattaṃ siyā vipariṇāmo.}}\\
\begin{addmargin}[1em]{2em}
\setstretch{.5}
{\PaliGlossB{    -}}\\
\end{addmargin}
\end{absolutelynopagebreak}

\begin{absolutelynopagebreak}
\setstretch{.7}
{\PaliGlossA{seyyathāpi nāma vacchassa taruṇassa mātaraṃ apassantassa siyā aññathattaṃ siyā vipariṇāmo;}}\\
\begin{addmargin}[1em]{2em}
\setstretch{.5}
{\PaliGlossB{    -}}\\
\end{addmargin}
\end{absolutelynopagebreak}

\begin{absolutelynopagebreak}
\setstretch{.7}
{\PaliGlossA{evameva santettha bhikkhū navā acirapabbajitā adhunāgatā imaṃ dhammavinayaṃ tesaṃ bhagavantaṃ apassantānaṃ siyā aññathattaṃ siyā vipariṇāmo.}}\\
\begin{addmargin}[1em]{2em}
\setstretch{.5}
{\PaliGlossB{    -}}\\
\end{addmargin}
\end{absolutelynopagebreak}

\begin{absolutelynopagebreak}
\setstretch{.7}
{\PaliGlossA{seyyathāpi nāma bījānaṃ taruṇānaṃ udakaṃ alabhantānaṃ siyā aññathattaṃ siyā vipariṇāmo;}}\\
\begin{addmargin}[1em]{2em}
\setstretch{.5}
{\PaliGlossB{    -}}\\
\end{addmargin}
\end{absolutelynopagebreak}

\begin{absolutelynopagebreak}
\setstretch{.7}
{\PaliGlossA{evameva santettha bhikkhū navā acirapabbajitā adhunāgatā imaṃ dhammavinayaṃ, tesaṃ bhagavantaṃ alabhantānaṃ dassanāya siyā aññathattaṃ siyā vipariṇāmo.}}\\
\begin{addmargin}[1em]{2em}
\setstretch{.5}
{\PaliGlossB{    -}}\\
\end{addmargin}
\end{absolutelynopagebreak}

\begin{absolutelynopagebreak}
\setstretch{.7}
{\PaliGlossA{abhinandatu, bhante, bhagavā bhikkhusaṃghaṃ;}}\\
\begin{addmargin}[1em]{2em}
\setstretch{.5}
{\PaliGlossB{May the Buddha be happy with the mendicant Saṅgha!}}\\
\end{addmargin}
\end{absolutelynopagebreak}

\begin{absolutelynopagebreak}
\setstretch{.7}
{\PaliGlossA{abhivadatu, bhante, bhagavā bhikkhusaṃghaṃ.}}\\
\begin{addmargin}[1em]{2em}
\setstretch{.5}
{\PaliGlossB{May the Buddha welcome the mendicant Saṅgha!}}\\
\end{addmargin}
\end{absolutelynopagebreak}

\begin{absolutelynopagebreak}
\setstretch{.7}
{\PaliGlossA{yatheva bhagavatā pubbe bhikkhusaṃgho anuggahito, evameva etarahi anuggaṇhātu bhikkhusaṃghan”ti.}}\\
\begin{addmargin}[1em]{2em}
\setstretch{.5}
{\PaliGlossB{May the Buddha support the mendicant Saṅgha now as he did in the past!”}}\\
\end{addmargin}
\end{absolutelynopagebreak}

\begin{absolutelynopagebreak}
\setstretch{.7}
{\PaliGlossA{adhivāsesi bhagavā tuṇhībhāvena.}}\\
\begin{addmargin}[1em]{2em}
\setstretch{.5}
{\PaliGlossB{The Buddha consented in silence.}}\\
\end{addmargin}
\end{absolutelynopagebreak}

\begin{absolutelynopagebreak}
\setstretch{.7}
{\PaliGlossA{atha kho brahmā sahampati bhagavato adhivāsanaṃ viditvā bhagavantaṃ abhivādetvā padakkhiṇaṃ katvā tatthevantaradhāyi.}}\\
\begin{addmargin}[1em]{2em}
\setstretch{.5}
{\PaliGlossB{Then Brahmā Sahampati, knowing that the Buddha had consented, bowed, and respectfully circled the Buddha, keeping him on his right, before vanishing right there.}}\\
\end{addmargin}
\end{absolutelynopagebreak}

\begin{absolutelynopagebreak}
\setstretch{.7}
{\PaliGlossA{atha kho bhagavā sāyanhasamayaṃ paṭisallānā vuṭṭhito yena nigrodhārāmo tenupasaṅkami; upasaṅkamitvā paññatte āsane nisīdi.}}\\
\begin{addmargin}[1em]{2em}
\setstretch{.5}
{\PaliGlossB{Then in the late afternoon, the Buddha came out of retreat and went to the Banyan Tree Monastery, where he sat on the seat spread out.}}\\
\end{addmargin}
\end{absolutelynopagebreak}

\begin{absolutelynopagebreak}
\setstretch{.7}
{\PaliGlossA{nisajja kho bhagavā tathārūpaṃ iddhābhisaṅkhāraṃ abhisaṅkhāsi yathā te bhikkhū ekadvīhikāya sārajjamānarūpā yenāhaṃ tenupasaṅkameyyuṃ.}}\\
\begin{addmargin}[1em]{2em}
\setstretch{.5}
{\PaliGlossB{Then he used his psychic power to will that the mendicants would come to him timidly, alone or in pairs.}}\\
\end{addmargin}
\end{absolutelynopagebreak}

\begin{absolutelynopagebreak}
\setstretch{.7}
{\PaliGlossA{tepi bhikkhū ekadvīhikāya sārajjamānarūpā yena bhagavā tenupasaṅkamiṃsu; upasaṅkamitvā bhagavantaṃ abhivādetvā ekamantaṃ nisīdiṃsu. ekamantaṃ nisinne kho te bhikkhū bhagavā etadavoca:}}\\
\begin{addmargin}[1em]{2em}
\setstretch{.5}
{\PaliGlossB{Those mendicants approached the Buddha timidly, bowed, and sat down to one side. The Buddha said to them:}}\\
\end{addmargin}
\end{absolutelynopagebreak}

\begin{absolutelynopagebreak}
\setstretch{.7}
{\PaliGlossA{“antamidaṃ, bhikkhave, jīvikānaṃ yadidaṃ piṇḍolyaṃ.}}\\
\begin{addmargin}[1em]{2em}
\setstretch{.5}
{\PaliGlossB{“Mendicants, this relying on alms is an extreme way to live.}}\\
\end{addmargin}
\end{absolutelynopagebreak}

\begin{absolutelynopagebreak}
\setstretch{.7}
{\PaliGlossA{abhisāpoyaṃ, bhikkhave, lokasmiṃ piṇḍolo vicarasi pattapāṇīti.}}\\
\begin{addmargin}[1em]{2em}
\setstretch{.5}
{\PaliGlossB{The world curses you: ‘You beggar, walking bowl in hand!’}}\\
\end{addmargin}
\end{absolutelynopagebreak}

\begin{absolutelynopagebreak}
\setstretch{.7}
{\PaliGlossA{tañca kho etaṃ, bhikkhave, kulaputtā upenti atthavasikā, atthavasaṃ paṭicca;}}\\
\begin{addmargin}[1em]{2em}
\setstretch{.5}
{\PaliGlossB{Yet earnest and gentlemen take it up for a good reason.}}\\
\end{addmargin}
\end{absolutelynopagebreak}

\begin{absolutelynopagebreak}
\setstretch{.7}
{\PaliGlossA{neva rājābhinītā, na corābhinītā, na iṇaṭṭā, na bhayaṭṭā, na ājīvikāpakatā;}}\\
\begin{addmargin}[1em]{2em}
\setstretch{.5}
{\PaliGlossB{Not because they’ve been forced to by kings or bandits, or because they’re in debt or threatened, or to earn a living.}}\\
\end{addmargin}
\end{absolutelynopagebreak}

\begin{absolutelynopagebreak}
\setstretch{.7}
{\PaliGlossA{api ca kho otiṇṇāmha jātiyā jarāya maraṇena sokehi paridevehi dukkhehi domanassehi upāyāsehi dukkhotiṇṇā dukkhaparetā}}\\
\begin{addmargin}[1em]{2em}
\setstretch{.5}
{\PaliGlossB{But because they’re swamped by rebirth, old age, and death; by sorrow, lamentation, pain, sadness, and distress. They’re swamped by suffering, mired in suffering.}}\\
\end{addmargin}
\end{absolutelynopagebreak}

\begin{absolutelynopagebreak}
\setstretch{.7}
{\PaliGlossA{appeva nāma imassa kevalassa dukkhakkhandhassa antakiriyā paññāyethāti.}}\\
\begin{addmargin}[1em]{2em}
\setstretch{.5}
{\PaliGlossB{And they think, ‘Hopefully I can find an end to this entire mass of suffering.’}}\\
\end{addmargin}
\end{absolutelynopagebreak}

\begin{absolutelynopagebreak}
\setstretch{.7}
{\PaliGlossA{evaṃ pabbajito cāyaṃ, bhikkhave, kulaputto.}}\\
\begin{addmargin}[1em]{2em}
\setstretch{.5}
{\PaliGlossB{That’s how this gentleman has gone forth.}}\\
\end{addmargin}
\end{absolutelynopagebreak}

\begin{absolutelynopagebreak}
\setstretch{.7}
{\PaliGlossA{so ca hoti abhijjhālu kāmesu tibbasārāgo byāpannacitto paduṭṭhamanasaṅkappo muṭṭhassati asampajāno asamāhito vibbhantacitto pākatindriyo.}}\\
\begin{addmargin}[1em]{2em}
\setstretch{.5}
{\PaliGlossB{Yet they covet sensual pleasures; they’re infatuated, full of ill will and hateful intent. They are unmindful, lacking situational awareness and immersion, with straying mind and undisciplined faculties.}}\\
\end{addmargin}
\end{absolutelynopagebreak}

\begin{absolutelynopagebreak}
\setstretch{.7}
{\PaliGlossA{seyyathāpi, bhikkhave, chavālātaṃ ubhatopadittaṃ majjhe gūthagataṃ, neva gāme kaṭṭhatthaṃ pharati, nāraññe kaṭṭhatthaṃ pharati.}}\\
\begin{addmargin}[1em]{2em}
\setstretch{.5}
{\PaliGlossB{Suppose there was a firebrand for lighting a funeral pyre, burning at both ends, and smeared with dung in the middle. It couldn’t be used as timber either in the village or the wilderness.}}\\
\end{addmargin}
\end{absolutelynopagebreak}

\begin{absolutelynopagebreak}
\setstretch{.7}
{\PaliGlossA{tathūpamāhaṃ, bhikkhave, imaṃ puggalaṃ vadāmi gihibhogā ca parihīno, sāmaññatthañca na paripūreti.}}\\
\begin{addmargin}[1em]{2em}
\setstretch{.5}
{\PaliGlossB{I say that person is just like this. They’ve missed out on the pleasures of the lay life, and haven’t fulfilled the goal of the ascetic life.}}\\
\end{addmargin}
\end{absolutelynopagebreak}

\begin{absolutelynopagebreak}
\setstretch{.7}
{\PaliGlossA{tayome, bhikkhave, akusalavitakkā—}}\\
\begin{addmargin}[1em]{2em}
\setstretch{.5}
{\PaliGlossB{There are these three unskillful thoughts.}}\\
\end{addmargin}
\end{absolutelynopagebreak}

\begin{absolutelynopagebreak}
\setstretch{.7}
{\PaliGlossA{kāmavitakko, byāpādavitakko, vihiṃsāvitakko.}}\\
\begin{addmargin}[1em]{2em}
\setstretch{.5}
{\PaliGlossB{Sensual, malicious, and cruel thoughts.}}\\
\end{addmargin}
\end{absolutelynopagebreak}

\begin{absolutelynopagebreak}
\setstretch{.7}
{\PaliGlossA{ime ca bhikkhave, tayo akusalavitakkā kva aparisesā nirujjhanti?}}\\
\begin{addmargin}[1em]{2em}
\setstretch{.5}
{\PaliGlossB{And where do these three unskillful thoughts cease without anything left over?}}\\
\end{addmargin}
\end{absolutelynopagebreak}

\begin{absolutelynopagebreak}
\setstretch{.7}
{\PaliGlossA{catūsu vā satipaṭṭhānesu suppatiṭṭhitacittassa viharato animittaṃ vā samādhiṃ bhāvayato.}}\\
\begin{addmargin}[1em]{2em}
\setstretch{.5}
{\PaliGlossB{In those who meditate with their mind firmly established in the four kinds of mindfulness meditation; or who develop signless immersion.}}\\
\end{addmargin}
\end{absolutelynopagebreak}

\begin{absolutelynopagebreak}
\setstretch{.7}
{\PaliGlossA{yāvañcidaṃ, bhikkhave, alameva animitto samādhi bhāvetuṃ.}}\\
\begin{addmargin}[1em]{2em}
\setstretch{.5}
{\PaliGlossB{Just this much is quite enough motivation to develop signless immersion.}}\\
\end{addmargin}
\end{absolutelynopagebreak}

\begin{absolutelynopagebreak}
\setstretch{.7}
{\PaliGlossA{animitto, bhikkhave, samādhi bhāvito bahulīkato mahapphalo hoti mahānisaṃso.}}\\
\begin{addmargin}[1em]{2em}
\setstretch{.5}
{\PaliGlossB{When signless immersion is developed and cultivated it is very fruitful and beneficial.}}\\
\end{addmargin}
\end{absolutelynopagebreak}

\begin{absolutelynopagebreak}
\setstretch{.7}
{\PaliGlossA{dvemā, bhikkhave, diṭṭhiyo—}}\\
\begin{addmargin}[1em]{2em}
\setstretch{.5}
{\PaliGlossB{There are these two views.}}\\
\end{addmargin}
\end{absolutelynopagebreak}

\begin{absolutelynopagebreak}
\setstretch{.7}
{\PaliGlossA{bhavadiṭṭhi ca vibhavadiṭṭhi ca.}}\\
\begin{addmargin}[1em]{2em}
\setstretch{.5}
{\PaliGlossB{Views favoring continued existence and views favoring ending existence.}}\\
\end{addmargin}
\end{absolutelynopagebreak}

\begin{absolutelynopagebreak}
\setstretch{.7}
{\PaliGlossA{tatra kho, bhikkhave, sutavā ariyasāvako iti paṭisañcikkhati:}}\\
\begin{addmargin}[1em]{2em}
\setstretch{.5}
{\PaliGlossB{A noble disciple reflects on this:}}\\
\end{addmargin}
\end{absolutelynopagebreak}

\begin{absolutelynopagebreak}
\setstretch{.7}
{\PaliGlossA{‘atthi nu kho taṃ kiñci lokasmiṃ yamahaṃ upādiyamāno na vajjavā assan’ti?}}\\
\begin{addmargin}[1em]{2em}
\setstretch{.5}
{\PaliGlossB{‘Is there anything in the world that I could grasp without fault?’}}\\
\end{addmargin}
\end{absolutelynopagebreak}

\begin{absolutelynopagebreak}
\setstretch{.7}
{\PaliGlossA{so evaṃ pajānāti:}}\\
\begin{addmargin}[1em]{2em}
\setstretch{.5}
{\PaliGlossB{They understand:}}\\
\end{addmargin}
\end{absolutelynopagebreak}

\begin{absolutelynopagebreak}
\setstretch{.7}
{\PaliGlossA{‘natthi nu kho taṃ kiñci lokasmiṃ yamahaṃ upādiyamāno na vajjavā assaṃ.}}\\
\begin{addmargin}[1em]{2em}
\setstretch{.5}
{\PaliGlossB{‘There’s nothing in the world that I could grasp without fault.}}\\
\end{addmargin}
\end{absolutelynopagebreak}

\begin{absolutelynopagebreak}
\setstretch{.7}
{\PaliGlossA{ahañhi rūpaññeva upādiyamāno upādiyeyyaṃ vedanaññeva …}}\\
\begin{addmargin}[1em]{2em}
\setstretch{.5}
{\PaliGlossB{For in grasping I would grasp only at form, feeling,}}\\
\end{addmargin}
\end{absolutelynopagebreak}

\begin{absolutelynopagebreak}
\setstretch{.7}
{\PaliGlossA{saññaññeva …}}\\
\begin{addmargin}[1em]{2em}
\setstretch{.5}
{\PaliGlossB{perception,}}\\
\end{addmargin}
\end{absolutelynopagebreak}

\begin{absolutelynopagebreak}
\setstretch{.7}
{\PaliGlossA{saṅkhāreyeva viññāṇaññeva upādiyamāno upādiyeyyaṃ.}}\\
\begin{addmargin}[1em]{2em}
\setstretch{.5}
{\PaliGlossB{choices, or consciousness.}}\\
\end{addmargin}
\end{absolutelynopagebreak}

\begin{absolutelynopagebreak}
\setstretch{.7}
{\PaliGlossA{tassa me assa upādānapaccayā bhavo;}}\\
\begin{addmargin}[1em]{2em}
\setstretch{.5}
{\PaliGlossB{That grasping of mine would be a condition for continued existence.}}\\
\end{addmargin}
\end{absolutelynopagebreak}

\begin{absolutelynopagebreak}
\setstretch{.7}
{\PaliGlossA{bhavapaccayā jāti;}}\\
\begin{addmargin}[1em]{2em}
\setstretch{.5}
{\PaliGlossB{Continued existence is a condition for rebirth.}}\\
\end{addmargin}
\end{absolutelynopagebreak}

\begin{absolutelynopagebreak}
\setstretch{.7}
{\PaliGlossA{jātipaccayā jarāmaraṇaṃ sokaparidevadukkhadomanassupāyāsā sambhaveyyuṃ.}}\\
\begin{addmargin}[1em]{2em}
\setstretch{.5}
{\PaliGlossB{Rebirth is a condition that gives rise to old age and death, sorrow, lamentation, pain, sadness, and distress.}}\\
\end{addmargin}
\end{absolutelynopagebreak}

\begin{absolutelynopagebreak}
\setstretch{.7}
{\PaliGlossA{evametassa kevalassa dukkhakkhandhassa samudayo assā’ti.}}\\
\begin{addmargin}[1em]{2em}
\setstretch{.5}
{\PaliGlossB{That is how this entire mass of suffering originates.}}\\
\end{addmargin}
\end{absolutelynopagebreak}

\begin{absolutelynopagebreak}
\setstretch{.7}
{\PaliGlossA{taṃ kiṃ maññatha, bhikkhave,}}\\
\begin{addmargin}[1em]{2em}
\setstretch{.5}
{\PaliGlossB{What do you think, mendicants?}}\\
\end{addmargin}
\end{absolutelynopagebreak}

\begin{absolutelynopagebreak}
\setstretch{.7}
{\PaliGlossA{rūpaṃ niccaṃ vā aniccaṃ vā”ti?}}\\
\begin{addmargin}[1em]{2em}
\setstretch{.5}
{\PaliGlossB{Is form permanent or impermanent?”}}\\
\end{addmargin}
\end{absolutelynopagebreak}

\begin{absolutelynopagebreak}
\setstretch{.7}
{\PaliGlossA{“aniccaṃ, bhante”.}}\\
\begin{addmargin}[1em]{2em}
\setstretch{.5}
{\PaliGlossB{“Impermanent, sir.”}}\\
\end{addmargin}
\end{absolutelynopagebreak}

\begin{absolutelynopagebreak}
\setstretch{.7}
{\PaliGlossA{“yaṃ panāniccaṃ dukkhaṃ vā taṃ sukhaṃ vā”ti?}}\\
\begin{addmargin}[1em]{2em}
\setstretch{.5}
{\PaliGlossB{“But if it’s impermanent, is it suffering or happiness?”}}\\
\end{addmargin}
\end{absolutelynopagebreak}

\begin{absolutelynopagebreak}
\setstretch{.7}
{\PaliGlossA{“dukkhaṃ, bhante”.}}\\
\begin{addmargin}[1em]{2em}
\setstretch{.5}
{\PaliGlossB{“Suffering, sir.”}}\\
\end{addmargin}
\end{absolutelynopagebreak}

\begin{absolutelynopagebreak}
\setstretch{.7}
{\PaliGlossA{“yaṃ panāniccaṃ dukkhaṃ vipariṇāmadhammaṃ kallaṃ nu taṃ samanupassituṃ:}}\\
\begin{addmargin}[1em]{2em}
\setstretch{.5}
{\PaliGlossB{“But if it’s impermanent, suffering, and perishable, is it fit to be regarded thus:}}\\
\end{addmargin}
\end{absolutelynopagebreak}

\begin{absolutelynopagebreak}
\setstretch{.7}
{\PaliGlossA{‘etaṃ mama, esohamasmi, eso me attā’”ti?}}\\
\begin{addmargin}[1em]{2em}
\setstretch{.5}
{\PaliGlossB{‘This is mine, I am this, this is my self’?”}}\\
\end{addmargin}
\end{absolutelynopagebreak}

\begin{absolutelynopagebreak}
\setstretch{.7}
{\PaliGlossA{“no hetaṃ, bhante”.}}\\
\begin{addmargin}[1em]{2em}
\setstretch{.5}
{\PaliGlossB{“No, sir.”}}\\
\end{addmargin}
\end{absolutelynopagebreak}

\begin{absolutelynopagebreak}
\setstretch{.7}
{\PaliGlossA{“vedanā …}}\\
\begin{addmargin}[1em]{2em}
\setstretch{.5}
{\PaliGlossB{“Is feeling …}}\\
\end{addmargin}
\end{absolutelynopagebreak}

\begin{absolutelynopagebreak}
\setstretch{.7}
{\PaliGlossA{saññā …}}\\
\begin{addmargin}[1em]{2em}
\setstretch{.5}
{\PaliGlossB{perception …}}\\
\end{addmargin}
\end{absolutelynopagebreak}

\begin{absolutelynopagebreak}
\setstretch{.7}
{\PaliGlossA{saṅkhārā …}}\\
\begin{addmargin}[1em]{2em}
\setstretch{.5}
{\PaliGlossB{choices …}}\\
\end{addmargin}
\end{absolutelynopagebreak}

\begin{absolutelynopagebreak}
\setstretch{.7}
{\PaliGlossA{viññāṇaṃ … pe …}}\\
\begin{addmargin}[1em]{2em}
\setstretch{.5}
{\PaliGlossB{consciousness permanent or impermanent?” …}}\\
\end{addmargin}
\end{absolutelynopagebreak}

\begin{absolutelynopagebreak}
\setstretch{.7}
{\PaliGlossA{tasmātiha, bhikkhave,}}\\
\begin{addmargin}[1em]{2em}
\setstretch{.5}
{\PaliGlossB{“So you should truly see …}}\\
\end{addmargin}
\end{absolutelynopagebreak}

\begin{absolutelynopagebreak}
\setstretch{.7}
{\PaliGlossA{evaṃ passaṃ …}}\\
\begin{addmargin}[1em]{2em}
\setstretch{.5}
{\PaliGlossB{Seeing this …}}\\
\end{addmargin}
\end{absolutelynopagebreak}

\begin{absolutelynopagebreak}
\setstretch{.7}
{\PaliGlossA{nāparaṃ itthattāyāti pajānātī”ti.}}\\
\begin{addmargin}[1em]{2em}
\setstretch{.5}
{\PaliGlossB{They understand: ‘… there is no return to any state of existence.’”}}\\
\end{addmargin}
\end{absolutelynopagebreak}

\begin{absolutelynopagebreak}
\setstretch{.7}
{\PaliGlossA{aṭṭhamaṃ.}}\\
\begin{addmargin}[1em]{2em}
\setstretch{.5}
{\PaliGlossB{    -}}\\
\end{addmargin}
\end{absolutelynopagebreak}
