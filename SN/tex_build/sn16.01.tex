
\begin{absolutelynopagebreak}
\setstretch{.7}
{\PaliGlossA{saṃyutta nikāya 16}}\\
\begin{addmargin}[1em]{2em}
\setstretch{.5}
{\PaliGlossB{Linked Discourses 16}}\\
\end{addmargin}
\end{absolutelynopagebreak}

\begin{absolutelynopagebreak}
\setstretch{.7}
{\PaliGlossA{1. kassapavagga}}\\
\begin{addmargin}[1em]{2em}
\setstretch{.5}
{\PaliGlossB{1. Kassapa}}\\
\end{addmargin}
\end{absolutelynopagebreak}

\begin{absolutelynopagebreak}
\setstretch{.7}
{\PaliGlossA{1. santuṭṭhasutta}}\\
\begin{addmargin}[1em]{2em}
\setstretch{.5}
{\PaliGlossB{1. Content}}\\
\end{addmargin}
\end{absolutelynopagebreak}

\begin{absolutelynopagebreak}
\setstretch{.7}
{\PaliGlossA{sāvatthiyaṃ viharati.}}\\
\begin{addmargin}[1em]{2em}
\setstretch{.5}
{\PaliGlossB{At Sāvatthī.}}\\
\end{addmargin}
\end{absolutelynopagebreak}

\begin{absolutelynopagebreak}
\setstretch{.7}
{\PaliGlossA{“santuṭṭhāyaṃ, bhikkhave, kassapo itarītarena cīvarena, itarītaracīvarasantuṭṭhiyā ca vaṇṇavādī;}}\\
\begin{addmargin}[1em]{2em}
\setstretch{.5}
{\PaliGlossB{“Mendicants, Kassapa is content with any kind of robe, and praises such contentment.}}\\
\end{addmargin}
\end{absolutelynopagebreak}

\begin{absolutelynopagebreak}
\setstretch{.7}
{\PaliGlossA{na ca cīvarahetu anesanaṃ appatirūpaṃ āpajjati; aladdhā ca cīvaraṃ na paritassati; laddhā ca cīvaraṃ agadhito amucchito anajjhāpanno ādīnavadassāvī nissaraṇapañño paribhuñjati.}}\\
\begin{addmargin}[1em]{2em}
\setstretch{.5}
{\PaliGlossB{He doesn’t try to get hold of a robe in an improper way. He doesn’t get upset if he doesn’t get a robe. And if he does get a robe, he uses it untied, uninfatuated, unattached, seeing the drawback, and understanding the escape.}}\\
\end{addmargin}
\end{absolutelynopagebreak}

\begin{absolutelynopagebreak}
\setstretch{.7}
{\PaliGlossA{santuṭṭhāyaṃ, bhikkhave, kassapo itarītarena piṇḍapātena, itarītarapiṇḍapātasantuṭṭhiyā ca vaṇṇavādī; na ca piṇḍapātahetu anesanaṃ appatirūpaṃ āpajjati; aladdhā ca piṇḍapātaṃ na paritassati; laddhā ca piṇḍapātaṃ agadhito amucchito anajjhāpanno ādīnavadassāvī nissaraṇapañño paribhuñjati.}}\\
\begin{addmargin}[1em]{2em}
\setstretch{.5}
{\PaliGlossB{Kassapa is content with any kind of alms-food …}}\\
\end{addmargin}
\end{absolutelynopagebreak}

\begin{absolutelynopagebreak}
\setstretch{.7}
{\PaliGlossA{santuṭṭhāyaṃ, bhikkhave, kassapo itarītarena senāsanena, itarītarasenāsanasantuṭṭhiyā ca vaṇṇavādī; na ca senāsanahetu anesanaṃ appatirūpaṃ āpajjati; aladdhā ca senāsanaṃ na paritassati; laddhā ca senāsanaṃ agadhito amucchito anajjhāpanno ādīnavadassāvī nissaraṇapañño paribhuñjati.}}\\
\begin{addmargin}[1em]{2em}
\setstretch{.5}
{\PaliGlossB{Kassapa is content with any kind of lodging …}}\\
\end{addmargin}
\end{absolutelynopagebreak}

\begin{absolutelynopagebreak}
\setstretch{.7}
{\PaliGlossA{santuṭṭhāyaṃ, bhikkhave, kassapo itarītarena gilānappaccayabhesajjaparikkhārena, itarītaragilānappaccayabhesajjaparikkhārasantuṭṭhiyā ca vaṇṇavādī; na ca gilānappaccayabhesajjaparikkhārahetu anesanaṃ appatirūpaṃ āpajjati; aladdhā ca gilānappaccayabhesajjaparikkhāraṃ na paritassati; laddhā ca gilānappaccayabhesajjaparikkhāraṃ agadhito amucchito anajjhāpanno ādīnavadassāvī nissaraṇapañño paribhuñjati.}}\\
\begin{addmargin}[1em]{2em}
\setstretch{.5}
{\PaliGlossB{Kassapa is content with any kind of medicines and supplies for the sick …}}\\
\end{addmargin}
\end{absolutelynopagebreak}

\begin{absolutelynopagebreak}
\setstretch{.7}
{\PaliGlossA{tasmātiha, bhikkhave, evaṃ sikkhitabbaṃ: ‘santuṭṭhā bhavissāma itarītarena cīvarena, itarītaracīvarasantuṭṭhiyā ca vaṇṇavādino; na ca cīvarahetu anesanaṃ appatirūpaṃ āpajjissāma;}}\\
\begin{addmargin}[1em]{2em}
\setstretch{.5}
{\PaliGlossB{So you should train like this: ‘We will be content with any kind of robe, and praise such contentment. We won’t try to get hold of a robe in an improper way.}}\\
\end{addmargin}
\end{absolutelynopagebreak}

\begin{absolutelynopagebreak}
\setstretch{.7}
{\PaliGlossA{aladdhā ca cīvaraṃ na ca paritassissāma; laddhā ca cīvaraṃ agadhitā amucchitā anajjhāpannā ādīnavadassāvino nissaraṇapaññā paribhuñjissāma’.}}\\
\begin{addmargin}[1em]{2em}
\setstretch{.5}
{\PaliGlossB{We won’t get upset if we don’t get a robe. And if we do get a robe, we’ll use it untied, uninfatuated, unattached, seeing the drawback, and understanding the escape.’}}\\
\end{addmargin}
\end{absolutelynopagebreak}

\begin{absolutelynopagebreak}
\setstretch{.7}
{\PaliGlossA{(evaṃ sabbaṃ kātabbaṃ.)}}\\
\begin{addmargin}[1em]{2em}
\setstretch{.5}
{\PaliGlossB{(All should be treated the same way.)}}\\
\end{addmargin}
\end{absolutelynopagebreak}

\begin{absolutelynopagebreak}
\setstretch{.7}
{\PaliGlossA{‘santuṭṭhā bhavissāma itarītarena piṇḍapātena … pe …}}\\
\begin{addmargin}[1em]{2em}
\setstretch{.5}
{\PaliGlossB{‘We will be content with any kind of alms-food …’}}\\
\end{addmargin}
\end{absolutelynopagebreak}

\begin{absolutelynopagebreak}
\setstretch{.7}
{\PaliGlossA{santuṭṭhā bhavissāma itarītarena senāsanena … pe …}}\\
\begin{addmargin}[1em]{2em}
\setstretch{.5}
{\PaliGlossB{‘We will be content with any kind of lodging …’}}\\
\end{addmargin}
\end{absolutelynopagebreak}

\begin{absolutelynopagebreak}
\setstretch{.7}
{\PaliGlossA{santuṭṭhā bhavissāma itarītarena gilānappaccayabhesajjaparikkhārena, itarītaragilānappaccayabhesajjaparikkhārasantuṭṭhiyā ca vaṇṇavādino; na ca gilānappaccayabhesajjaparikkhārahetu anesanaṃ appatirūpaṃ āpajjissāma aladdhā ca gilānappaccayabhesajjaparikkhāraṃ na paritassissāma; laddhā ca gilānappaccayabhesajjaparikkhāraṃ agadhitā amucchitā anajjhāpannā ādīnavadassāvino nissaraṇapaññā paribhuñjissāmā’ti.}}\\
\begin{addmargin}[1em]{2em}
\setstretch{.5}
{\PaliGlossB{‘We will be content with any kind of medicines and supplies for the sick …’}}\\
\end{addmargin}
\end{absolutelynopagebreak}

\begin{absolutelynopagebreak}
\setstretch{.7}
{\PaliGlossA{evañhi vo, bhikkhave, sikkhitabbaṃ.}}\\
\begin{addmargin}[1em]{2em}
\setstretch{.5}
{\PaliGlossB{That’s how you should train.}}\\
\end{addmargin}
\end{absolutelynopagebreak}

\begin{absolutelynopagebreak}
\setstretch{.7}
{\PaliGlossA{kassapena vā hi vo, bhikkhave, ovadissāmi yo vā panassa kassapasadiso, ovaditehi ca pana vo tathattāya paṭipajjitabban”ti.}}\\
\begin{addmargin}[1em]{2em}
\setstretch{.5}
{\PaliGlossB{I will exhort you with the example of Kassapa or someone like him. You should practice accordingly.”}}\\
\end{addmargin}
\end{absolutelynopagebreak}

\begin{absolutelynopagebreak}
\setstretch{.7}
{\PaliGlossA{paṭhamaṃ.}}\\
\begin{addmargin}[1em]{2em}
\setstretch{.5}
{\PaliGlossB{    -}}\\
\end{addmargin}
\end{absolutelynopagebreak}
