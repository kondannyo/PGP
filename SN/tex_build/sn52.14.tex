
\begin{absolutelynopagebreak}
\setstretch{.7}
{\PaliGlossA{saṃyutta nikāya 52}}\\
\begin{addmargin}[1em]{2em}
\setstretch{.5}
{\PaliGlossB{Linked Discourses 52}}\\
\end{addmargin}
\end{absolutelynopagebreak}

\begin{absolutelynopagebreak}
\setstretch{.7}
{\PaliGlossA{2. dutiyavagga}}\\
\begin{addmargin}[1em]{2em}
\setstretch{.5}
{\PaliGlossB{2. A Thousand}}\\
\end{addmargin}
\end{absolutelynopagebreak}

\begin{absolutelynopagebreak}
\setstretch{.7}
{\PaliGlossA{14. cetopariyasutta}}\\
\begin{addmargin}[1em]{2em}
\setstretch{.5}
{\PaliGlossB{14. Comprehending the Mind}}\\
\end{addmargin}
\end{absolutelynopagebreak}

\begin{absolutelynopagebreak}
\setstretch{.7}
{\PaliGlossA{“imesañca panāhaṃ, āvuso, catunnaṃ satipaṭṭhānānaṃ bhāvitattā bahulīkatattā parasattānaṃ parapuggalānaṃ cetasā ceto paricca pajānāmi—sarāgaṃ vā cittaṃ ‘sarāgaṃ cittan’ti pajānāmi … pe … avimuttaṃ vā cittaṃ ‘avimuttaṃ cittan’ti pajānāmī”ti.}}\\
\begin{addmargin}[1em]{2em}
\setstretch{.5}
{\PaliGlossB{“… And it’s because of developing and cultivating these four kinds of mindfulness meditation that I understand the minds of other beings and individuals, having comprehended them with my mind. I understand mind with greed as ‘mind with greed’ … I understand unfreed mind as ‘unfreed mind’.”}}\\
\end{addmargin}
\end{absolutelynopagebreak}

\begin{absolutelynopagebreak}
\setstretch{.7}
{\PaliGlossA{catutthaṃ.}}\\
\begin{addmargin}[1em]{2em}
\setstretch{.5}
{\PaliGlossB{    -}}\\
\end{addmargin}
\end{absolutelynopagebreak}
