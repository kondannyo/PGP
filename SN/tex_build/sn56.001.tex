
\begin{absolutelynopagebreak}
\setstretch{.7}
{\PaliGlossA{saṃyutta nikāya 56}}\\
\begin{addmargin}[1em]{2em}
\setstretch{.5}
{\PaliGlossB{Linked Discourses 56}}\\
\end{addmargin}
\end{absolutelynopagebreak}

\begin{absolutelynopagebreak}
\setstretch{.7}
{\PaliGlossA{1. samādhivagga}}\\
\begin{addmargin}[1em]{2em}
\setstretch{.5}
{\PaliGlossB{1. Immersion}}\\
\end{addmargin}
\end{absolutelynopagebreak}

\begin{absolutelynopagebreak}
\setstretch{.7}
{\PaliGlossA{1. samādhisutta}}\\
\begin{addmargin}[1em]{2em}
\setstretch{.5}
{\PaliGlossB{1. Immersion}}\\
\end{addmargin}
\end{absolutelynopagebreak}

\begin{absolutelynopagebreak}
\setstretch{.7}
{\PaliGlossA{sāvatthinidānaṃ.}}\\
\begin{addmargin}[1em]{2em}
\setstretch{.5}
{\PaliGlossB{At Sāvatthī.}}\\
\end{addmargin}
\end{absolutelynopagebreak}

\begin{absolutelynopagebreak}
\setstretch{.7}
{\PaliGlossA{… pe …}}\\
\begin{addmargin}[1em]{2em}
\setstretch{.5}
{\PaliGlossB{    -}}\\
\end{addmargin}
\end{absolutelynopagebreak}

\begin{absolutelynopagebreak}
\setstretch{.7}
{\PaliGlossA{“samādhiṃ, bhikkhave, bhāvetha.}}\\
\begin{addmargin}[1em]{2em}
\setstretch{.5}
{\PaliGlossB{“Mendicants, develop immersion.}}\\
\end{addmargin}
\end{absolutelynopagebreak}

\begin{absolutelynopagebreak}
\setstretch{.7}
{\PaliGlossA{samāhito, bhikkhave, bhikkhu yathābhūtaṃ pajānāti.}}\\
\begin{addmargin}[1em]{2em}
\setstretch{.5}
{\PaliGlossB{A mendicant who has immersion truly understands.}}\\
\end{addmargin}
\end{absolutelynopagebreak}

\begin{absolutelynopagebreak}
\setstretch{.7}
{\PaliGlossA{kiñca yathābhūtaṃ pajānāti?}}\\
\begin{addmargin}[1em]{2em}
\setstretch{.5}
{\PaliGlossB{What do they truly understand?}}\\
\end{addmargin}
\end{absolutelynopagebreak}

\begin{absolutelynopagebreak}
\setstretch{.7}
{\PaliGlossA{‘idaṃ dukkhan’ti yathābhūtaṃ pajānāti, ‘ayaṃ dukkhasamudayo’ti yathābhūtaṃ pajānāti, ‘ayaṃ dukkhanirodho’ti yathābhūtaṃ pajānāti, ‘ayaṃ dukkhanirodhagāminī paṭipadā’ti yathābhūtaṃ pajānāti.}}\\
\begin{addmargin}[1em]{2em}
\setstretch{.5}
{\PaliGlossB{They truly understand: ‘This is suffering’ … ‘This is the origin of suffering’ … ‘This is the cessation of suffering’ … ‘This is the practice that leads to the cessation of suffering’.}}\\
\end{addmargin}
\end{absolutelynopagebreak}

\begin{absolutelynopagebreak}
\setstretch{.7}
{\PaliGlossA{samādhiṃ, bhikkhave, bhāvetha.}}\\
\begin{addmargin}[1em]{2em}
\setstretch{.5}
{\PaliGlossB{Develop immersion.}}\\
\end{addmargin}
\end{absolutelynopagebreak}

\begin{absolutelynopagebreak}
\setstretch{.7}
{\PaliGlossA{samāhito, bhikkhave, bhikkhu yathābhūtaṃ pajānāti.}}\\
\begin{addmargin}[1em]{2em}
\setstretch{.5}
{\PaliGlossB{A mendicant who has immersion truly understands.}}\\
\end{addmargin}
\end{absolutelynopagebreak}

\begin{absolutelynopagebreak}
\setstretch{.7}
{\PaliGlossA{tasmātiha, bhikkhave, ‘idaṃ dukkhan’ti yogo karaṇīyo, ‘ayaṃ dukkhasamudayo’ti yogo karaṇīyo, ‘ayaṃ dukkhanirodho’ti yogo karaṇīyo, ‘ayaṃ dukkhanirodhagāminī paṭipadā’ti yogo karaṇīyo”ti.}}\\
\begin{addmargin}[1em]{2em}
\setstretch{.5}
{\PaliGlossB{That’s why you should practice meditation to understand: ‘This is suffering’ … ‘This is the origin of suffering’ … ‘This is the cessation of suffering’ … ‘This is the practice that leads to the cessation of suffering’.”}}\\
\end{addmargin}
\end{absolutelynopagebreak}

\begin{absolutelynopagebreak}
\setstretch{.7}
{\PaliGlossA{paṭhamaṃ.}}\\
\begin{addmargin}[1em]{2em}
\setstretch{.5}
{\PaliGlossB{    -}}\\
\end{addmargin}
\end{absolutelynopagebreak}
