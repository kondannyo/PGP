
\begin{absolutelynopagebreak}
\setstretch{.7}
{\PaliGlossA{saṃyutta nikāya 35}}\\
\begin{addmargin}[1em]{2em}
\setstretch{.5}
{\PaliGlossB{Linked Discourses 35}}\\
\end{addmargin}
\end{absolutelynopagebreak}

\begin{absolutelynopagebreak}
\setstretch{.7}
{\PaliGlossA{14. devadahavagga}}\\
\begin{addmargin}[1em]{2em}
\setstretch{.5}
{\PaliGlossB{14. At Devadaha}}\\
\end{addmargin}
\end{absolutelynopagebreak}

\begin{absolutelynopagebreak}
\setstretch{.7}
{\PaliGlossA{134. devadahasutta}}\\
\begin{addmargin}[1em]{2em}
\setstretch{.5}
{\PaliGlossB{134. At Devadaha}}\\
\end{addmargin}
\end{absolutelynopagebreak}

\begin{absolutelynopagebreak}
\setstretch{.7}
{\PaliGlossA{ekaṃ samayaṃ bhagavā sakkesu viharati devadahaṃ nāma sakyānaṃ nigamo.}}\\
\begin{addmargin}[1em]{2em}
\setstretch{.5}
{\PaliGlossB{At one time the Buddha was staying in the land of the Sakyans, near the Sakyan town named Devadaha.}}\\
\end{addmargin}
\end{absolutelynopagebreak}

\begin{absolutelynopagebreak}
\setstretch{.7}
{\PaliGlossA{tatra kho bhagavā bhikkhū āmantesi:}}\\
\begin{addmargin}[1em]{2em}
\setstretch{.5}
{\PaliGlossB{There the Buddha addressed the mendicants:}}\\
\end{addmargin}
\end{absolutelynopagebreak}

\begin{absolutelynopagebreak}
\setstretch{.7}
{\PaliGlossA{“nāhaṃ, bhikkhave, sabbesaṃyeva bhikkhūnaṃ chasu phassāyatanesu appamādena karaṇīyanti vadāmi, na ca panāhaṃ, bhikkhave, sabbesaṃyeva bhikkhūnaṃ chasu phassāyatanesu nāppamādena karaṇīyanti vadāmi.}}\\
\begin{addmargin}[1em]{2em}
\setstretch{.5}
{\PaliGlossB{“When it comes to the six fields of contact, mendicants, I don’t say that all mendicants have work to do with diligence, nor do I say that none of them have work to do with diligence.}}\\
\end{addmargin}
\end{absolutelynopagebreak}

\begin{absolutelynopagebreak}
\setstretch{.7}
{\PaliGlossA{ye te, bhikkhave, bhikkhū arahanto khīṇāsavā vusitavanto katakaraṇīyā ohitabhārā anuppattasadatthā parikkhīṇabhavasaṃyojanā sammadaññāvimuttā, tesāhaṃ, bhikkhave, bhikkhūnaṃ chasu phassāyatanesu nāppamādena karaṇīyanti vadāmi.}}\\
\begin{addmargin}[1em]{2em}
\setstretch{.5}
{\PaliGlossB{I say that, when it comes to the six fields of contact, mendicants don’t have work to do with diligence if they are perfected, with defilements ended, having completed the spiritual journey, done what had to be done, laid down the burden, achieved their own goal, utterly ended the fetters of rebirth, and become rightly freed through enlightenment.}}\\
\end{addmargin}
\end{absolutelynopagebreak}

\begin{absolutelynopagebreak}
\setstretch{.7}
{\PaliGlossA{taṃ kissa hetu?}}\\
\begin{addmargin}[1em]{2em}
\setstretch{.5}
{\PaliGlossB{Why is that?}}\\
\end{addmargin}
\end{absolutelynopagebreak}

\begin{absolutelynopagebreak}
\setstretch{.7}
{\PaliGlossA{kataṃ tesaṃ appamādena, abhabbā te pamajjituṃ.}}\\
\begin{addmargin}[1em]{2em}
\setstretch{.5}
{\PaliGlossB{They’ve done their work with diligence, and are incapable of negligence.}}\\
\end{addmargin}
\end{absolutelynopagebreak}

\begin{absolutelynopagebreak}
\setstretch{.7}
{\PaliGlossA{ye ca kho te, bhikkhave, bhikkhū sekkhā appattamānasā anuttaraṃ yogakkhemaṃ patthayamānā viharanti, tesāhaṃ, bhikkhave, bhikkhūnaṃ chasu phassāyatanesu appamādena karaṇīyanti vadāmi.}}\\
\begin{addmargin}[1em]{2em}
\setstretch{.5}
{\PaliGlossB{I say that, when it comes to the six fields of contact, mendicants do have work to do with diligence if they are trainees, who haven’t achieved their heart’s desire, but live aspiring to the supreme sanctuary.}}\\
\end{addmargin}
\end{absolutelynopagebreak}

\begin{absolutelynopagebreak}
\setstretch{.7}
{\PaliGlossA{taṃ kissa hetu?}}\\
\begin{addmargin}[1em]{2em}
\setstretch{.5}
{\PaliGlossB{Why is that?}}\\
\end{addmargin}
\end{absolutelynopagebreak}

\begin{absolutelynopagebreak}
\setstretch{.7}
{\PaliGlossA{santi, bhikkhave, cakkhuviññeyyā rūpā manoramāpi, amanoramāpi.}}\\
\begin{addmargin}[1em]{2em}
\setstretch{.5}
{\PaliGlossB{There are sights known by the eye that are pleasant and also those that are unpleasant.}}\\
\end{addmargin}
\end{absolutelynopagebreak}

\begin{absolutelynopagebreak}
\setstretch{.7}
{\PaliGlossA{tyāssa phussa phussa cittaṃ na pariyādāya tiṭṭhanti.}}\\
\begin{addmargin}[1em]{2em}
\setstretch{.5}
{\PaliGlossB{Though experiencing them again and again they don’t occupy the mind.}}\\
\end{addmargin}
\end{absolutelynopagebreak}

\begin{absolutelynopagebreak}
\setstretch{.7}
{\PaliGlossA{cetaso apariyādānā āraddhaṃ hoti vīriyaṃ asallīnaṃ, upaṭṭhitā sati asammuṭṭhā, passaddho kāyo asāraddho, samāhitaṃ cittaṃ ekaggaṃ.}}\\
\begin{addmargin}[1em]{2em}
\setstretch{.5}
{\PaliGlossB{Their energy is roused up and unflagging, their mindfulness is established and lucid, their body is tranquil and undisturbed, and their mind is immersed in samādhi.}}\\
\end{addmargin}
\end{absolutelynopagebreak}

\begin{absolutelynopagebreak}
\setstretch{.7}
{\PaliGlossA{imaṃ khvāhaṃ, bhikkhave, appamādaphalaṃ sampassamāno tesaṃ bhikkhūnaṃ chasu phassāyatanesu appamādena karaṇīyanti vadāmi … pe …}}\\
\begin{addmargin}[1em]{2em}
\setstretch{.5}
{\PaliGlossB{Seeing this fruit of diligence, I say that those mendicants have work to do with diligence when it comes to the six fields of contact. …}}\\
\end{addmargin}
\end{absolutelynopagebreak}

\begin{absolutelynopagebreak}
\setstretch{.7}
{\PaliGlossA{santi, bhikkhave, manoviññeyyā dhammā manoramāpi amanoramāpi.}}\\
\begin{addmargin}[1em]{2em}
\setstretch{.5}
{\PaliGlossB{There are thoughts known by the mind that are pleasant and also those that are unpleasant.}}\\
\end{addmargin}
\end{absolutelynopagebreak}

\begin{absolutelynopagebreak}
\setstretch{.7}
{\PaliGlossA{tyāssa phussa phussa cittaṃ na pariyādāya tiṭṭhanti.}}\\
\begin{addmargin}[1em]{2em}
\setstretch{.5}
{\PaliGlossB{Though experiencing them again and again they don’t occupy the mind.}}\\
\end{addmargin}
\end{absolutelynopagebreak}

\begin{absolutelynopagebreak}
\setstretch{.7}
{\PaliGlossA{cetaso apariyādānā āraddhaṃ hoti vīriyaṃ asallīnaṃ, upaṭṭhitā sati asammuṭṭhā, passaddho kāyo asāraddho, samāhitaṃ cittaṃ ekaggaṃ.}}\\
\begin{addmargin}[1em]{2em}
\setstretch{.5}
{\PaliGlossB{Their energy is roused up and unflagging, their mindfulness is established and lucid, their body is tranquil and undisturbed, and their mind is immersed in samādhi.}}\\
\end{addmargin}
\end{absolutelynopagebreak}

\begin{absolutelynopagebreak}
\setstretch{.7}
{\PaliGlossA{imaṃ khvāhaṃ, bhikkhave, appamādaphalaṃ sampassamāno tesaṃ bhikkhūnaṃ chasu phassāyatanesu appamādena karaṇīyanti vadāmī”ti.}}\\
\begin{addmargin}[1em]{2em}
\setstretch{.5}
{\PaliGlossB{Seeing this fruit of diligence, I say that those mendicants have work to do with diligence when it comes to the six fields of contact.”}}\\
\end{addmargin}
\end{absolutelynopagebreak}

\begin{absolutelynopagebreak}
\setstretch{.7}
{\PaliGlossA{paṭhamaṃ.}}\\
\begin{addmargin}[1em]{2em}
\setstretch{.5}
{\PaliGlossB{    -}}\\
\end{addmargin}
\end{absolutelynopagebreak}
