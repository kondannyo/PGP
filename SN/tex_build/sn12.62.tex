
\begin{absolutelynopagebreak}
\setstretch{.7}
{\PaliGlossA{saṃyutta nikāya 12}}\\
\begin{addmargin}[1em]{2em}
\setstretch{.5}
{\PaliGlossB{Linked Discourses 12}}\\
\end{addmargin}
\end{absolutelynopagebreak}

\begin{absolutelynopagebreak}
\setstretch{.7}
{\PaliGlossA{7. mahāvagga}}\\
\begin{addmargin}[1em]{2em}
\setstretch{.5}
{\PaliGlossB{7. The Great Chapter}}\\
\end{addmargin}
\end{absolutelynopagebreak}

\begin{absolutelynopagebreak}
\setstretch{.7}
{\PaliGlossA{62. dutiyaassutavāsutta}}\\
\begin{addmargin}[1em]{2em}
\setstretch{.5}
{\PaliGlossB{62. Uneducated (2nd)}}\\
\end{addmargin}
\end{absolutelynopagebreak}

\begin{absolutelynopagebreak}
\setstretch{.7}
{\PaliGlossA{sāvatthiyaṃ viharati.}}\\
\begin{addmargin}[1em]{2em}
\setstretch{.5}
{\PaliGlossB{At Sāvatthī.}}\\
\end{addmargin}
\end{absolutelynopagebreak}

\begin{absolutelynopagebreak}
\setstretch{.7}
{\PaliGlossA{“assutavā, bhikkhave, puthujjano imasmiṃ cātumahābhūtikasmiṃ kāyasmiṃ nibbindeyyapi virajjeyyapi vimucceyyapi.}}\\
\begin{addmargin}[1em]{2em}
\setstretch{.5}
{\PaliGlossB{“Mendicants, when it comes to this body made up of the four primary elements, an uneducated ordinary person might become disillusioned, dispassionate, and freed.}}\\
\end{addmargin}
\end{absolutelynopagebreak}

\begin{absolutelynopagebreak}
\setstretch{.7}
{\PaliGlossA{taṃ kissa hetu?}}\\
\begin{addmargin}[1em]{2em}
\setstretch{.5}
{\PaliGlossB{Why is that?}}\\
\end{addmargin}
\end{absolutelynopagebreak}

\begin{absolutelynopagebreak}
\setstretch{.7}
{\PaliGlossA{dissati, bhikkhave, imassa cātumahābhūtikassa kāyassa ācayopi apacayopi ādānampi nikkhepanampi.}}\\
\begin{addmargin}[1em]{2em}
\setstretch{.5}
{\PaliGlossB{This body made up of the four primary elements is seen to accumulate and disperse, to be taken up and laid to rest.}}\\
\end{addmargin}
\end{absolutelynopagebreak}

\begin{absolutelynopagebreak}
\setstretch{.7}
{\PaliGlossA{tasmā tatrāssutavā puthujjano nibbindeyyapi virajjeyyapi vimucceyyapi.}}\\
\begin{addmargin}[1em]{2em}
\setstretch{.5}
{\PaliGlossB{That’s why, when it comes to this body, an uneducated ordinary person might become disillusioned, dispassionate, and freed.}}\\
\end{addmargin}
\end{absolutelynopagebreak}

\begin{absolutelynopagebreak}
\setstretch{.7}
{\PaliGlossA{yañca kho etaṃ, bhikkhave, vuccati cittaṃ itipi, mano itipi, viññāṇaṃ itipi, tatrāssutavā puthujjano nālaṃ nibbindituṃ nālaṃ virajjituṃ nālaṃ vimuccituṃ.}}\\
\begin{addmargin}[1em]{2em}
\setstretch{.5}
{\PaliGlossB{But when it comes to that which is called ‘mind’ or ‘sentience’ or ‘consciousness’, an uneducated ordinary person is unable to become disillusioned, dispassionate, or freed.}}\\
\end{addmargin}
\end{absolutelynopagebreak}

\begin{absolutelynopagebreak}
\setstretch{.7}
{\PaliGlossA{taṃ kissa hetu?}}\\
\begin{addmargin}[1em]{2em}
\setstretch{.5}
{\PaliGlossB{Why is that?}}\\
\end{addmargin}
\end{absolutelynopagebreak}

\begin{absolutelynopagebreak}
\setstretch{.7}
{\PaliGlossA{dīgharattañhetaṃ, bhikkhave, assutavato puthujjanassa ajjhositaṃ mamāyitaṃ parāmaṭṭhaṃ:}}\\
\begin{addmargin}[1em]{2em}
\setstretch{.5}
{\PaliGlossB{Because for a long time they’ve been attached to it, thought of it as their own, and mistaken it:}}\\
\end{addmargin}
\end{absolutelynopagebreak}

\begin{absolutelynopagebreak}
\setstretch{.7}
{\PaliGlossA{‘etaṃ mama, esohamasmi, eso me attā’ti.}}\\
\begin{addmargin}[1em]{2em}
\setstretch{.5}
{\PaliGlossB{‘This is mine, I am this, this is my self.’}}\\
\end{addmargin}
\end{absolutelynopagebreak}

\begin{absolutelynopagebreak}
\setstretch{.7}
{\PaliGlossA{tasmā tatrāssutavā puthujjano nālaṃ nibbindituṃ nālaṃ virajjituṃ nālaṃ vimuccituṃ.}}\\
\begin{addmargin}[1em]{2em}
\setstretch{.5}
{\PaliGlossB{That’s why, when it comes to this mind, an uneducated ordinary person is unable to become disillusioned, dispassionate, and freed.}}\\
\end{addmargin}
\end{absolutelynopagebreak}

\begin{absolutelynopagebreak}
\setstretch{.7}
{\PaliGlossA{varaṃ, bhikkhave, assutavā puthujjano imaṃ cātumahābhūtikaṃ kāyaṃ attato upagaccheyya, na tveva cittaṃ.}}\\
\begin{addmargin}[1em]{2em}
\setstretch{.5}
{\PaliGlossB{But an uneducated ordinary person would be better off taking this body made up of the four primary elements to be their self, rather than the mind.}}\\
\end{addmargin}
\end{absolutelynopagebreak}

\begin{absolutelynopagebreak}
\setstretch{.7}
{\PaliGlossA{taṃ kissa hetu?}}\\
\begin{addmargin}[1em]{2em}
\setstretch{.5}
{\PaliGlossB{Why is that?}}\\
\end{addmargin}
\end{absolutelynopagebreak}

\begin{absolutelynopagebreak}
\setstretch{.7}
{\PaliGlossA{dissatāyaṃ, bhikkhave, cātumahābhūtiko kāyo ekampi vassaṃ tiṭṭhamāno dvepi vassāni tiṭṭhamāno tīṇipi vassāni tiṭṭhamāno cattāripi vassāni tiṭṭhamāno pañcapi vassāni tiṭṭhamāno dasapi vassāni tiṭṭhamāno vīsatipi vassāni tiṭṭhamāno tiṃsampi vassāni tiṭṭhamāno cattārīsampi vassāni tiṭṭhamāno paññāsampi vassāni tiṭṭhamāno vassasatampi tiṭṭhamāno, bhiyyopi tiṭṭhamāno.}}\\
\begin{addmargin}[1em]{2em}
\setstretch{.5}
{\PaliGlossB{This body made up of the four primary elements is seen to last for a year, or for two, three, four, five, ten, twenty, thirty, forty, fifty, or a hundred years, or even longer.}}\\
\end{addmargin}
\end{absolutelynopagebreak}

\begin{absolutelynopagebreak}
\setstretch{.7}
{\PaliGlossA{yañca kho etaṃ, bhikkhave, vuccati cittaṃ itipi, mano itipi, viññāṇaṃ itipi, taṃ rattiyā ca divasassa ca aññadeva uppajjati aññaṃ nirujjhati.}}\\
\begin{addmargin}[1em]{2em}
\setstretch{.5}
{\PaliGlossB{But that which is called ‘mind’ or ‘sentience’ or ‘consciousness’ arises as one thing and ceases as another all day and all night.}}\\
\end{addmargin}
\end{absolutelynopagebreak}

\begin{absolutelynopagebreak}
\setstretch{.7}
{\PaliGlossA{tatra, bhikkhave, sutavā ariyasāvako paṭiccasamuppādaṃyeva sādhukaṃ yoniso manasi karoti:}}\\
\begin{addmargin}[1em]{2em}
\setstretch{.5}
{\PaliGlossB{In this case, a learned noble disciple carefully and properly attends to dependent origination itself:}}\\
\end{addmargin}
\end{absolutelynopagebreak}

\begin{absolutelynopagebreak}
\setstretch{.7}
{\PaliGlossA{‘iti imasmiṃ sati idaṃ hoti, imassuppādā idaṃ uppajjati;}}\\
\begin{addmargin}[1em]{2em}
\setstretch{.5}
{\PaliGlossB{‘When this exists, that is; due to the arising of this, that arises.}}\\
\end{addmargin}
\end{absolutelynopagebreak}

\begin{absolutelynopagebreak}
\setstretch{.7}
{\PaliGlossA{imasmiṃ asati idaṃ na hoti, imassa nirodhā idaṃ nirujjhatī’ti.}}\\
\begin{addmargin}[1em]{2em}
\setstretch{.5}
{\PaliGlossB{When this doesn’t exist, that is not; due to the cessation of this, that ceases. That is:}}\\
\end{addmargin}
\end{absolutelynopagebreak}

\begin{absolutelynopagebreak}
\setstretch{.7}
{\PaliGlossA{sukhavedaniyaṃ, bhikkhave, phassaṃ paṭicca uppajjati sukhavedanā.}}\\
\begin{addmargin}[1em]{2em}
\setstretch{.5}
{\PaliGlossB{Pleasant feeling arises dependent on a contact to be experienced as pleasant.}}\\
\end{addmargin}
\end{absolutelynopagebreak}

\begin{absolutelynopagebreak}
\setstretch{.7}
{\PaliGlossA{tasseva sukhavedaniyassa phassassa nirodhā yaṃ tajjaṃ vedayitaṃ sukhavedaniyaṃ phassaṃ paṭicca uppannā sukhavedanā sā nirujjhati sā vūpasammati.}}\\
\begin{addmargin}[1em]{2em}
\setstretch{.5}
{\PaliGlossB{With the cessation of that contact to be experienced as pleasant, the corresponding pleasant feeling ceases and stops.}}\\
\end{addmargin}
\end{absolutelynopagebreak}

\begin{absolutelynopagebreak}
\setstretch{.7}
{\PaliGlossA{dukkhavedaniyaṃ, bhikkhave, phassaṃ paṭicca uppajjati dukkhavedanā.}}\\
\begin{addmargin}[1em]{2em}
\setstretch{.5}
{\PaliGlossB{Painful feeling arises dependent on a contact to be experienced as painful.}}\\
\end{addmargin}
\end{absolutelynopagebreak}

\begin{absolutelynopagebreak}
\setstretch{.7}
{\PaliGlossA{tasseva dukkhavedaniyassa phassassa nirodhā yaṃ tajjaṃ vedayitaṃ dukkhavedaniyaṃ phassaṃ paṭicca uppannā dukkhavedanā sā nirujjhati sā vūpasammati.}}\\
\begin{addmargin}[1em]{2em}
\setstretch{.5}
{\PaliGlossB{With the cessation of that contact to be experienced as painful, the corresponding painful feeling ceases and stops.}}\\
\end{addmargin}
\end{absolutelynopagebreak}

\begin{absolutelynopagebreak}
\setstretch{.7}
{\PaliGlossA{adukkhamasukhavedaniyaṃ, bhikkhave, phassaṃ paṭicca uppajjati adukkhamasukhavedanā.}}\\
\begin{addmargin}[1em]{2em}
\setstretch{.5}
{\PaliGlossB{Neutral feeling arises dependent on a contact to be experienced as neutral.}}\\
\end{addmargin}
\end{absolutelynopagebreak}

\begin{absolutelynopagebreak}
\setstretch{.7}
{\PaliGlossA{tasseva adukkhamasukhavedaniyassa phassassa nirodhā yaṃ tajjaṃ vedayitaṃ adukkhamasukhavedaniyaṃ phassaṃ paṭicca uppannā adukkhamasukhavedanā sā nirujjhati sā vūpasammati.}}\\
\begin{addmargin}[1em]{2em}
\setstretch{.5}
{\PaliGlossB{With the cessation of that contact to be experienced as neutral, the corresponding neutral feeling ceases and stops.}}\\
\end{addmargin}
\end{absolutelynopagebreak}

\begin{absolutelynopagebreak}
\setstretch{.7}
{\PaliGlossA{seyyathāpi, bhikkhave, dvinnaṃ kaṭṭhānaṃ saṅghaṭṭanasamodhānā usmā jāyati tejo abhinibbattati. tesaṃyeva dvinnaṃ kaṭṭhānaṃ nānākatavinibbhogā yā tajjā usmā sā nirujjhati sā vūpasammati;}}\\
\begin{addmargin}[1em]{2em}
\setstretch{.5}
{\PaliGlossB{When you rub two sticks together, heat is generated and fire is produced. But when you part the sticks and lay them aside, any corresponding heat ceases and stops.}}\\
\end{addmargin}
\end{absolutelynopagebreak}

\begin{absolutelynopagebreak}
\setstretch{.7}
{\PaliGlossA{evameva kho, bhikkhave, sukhavedaniyaṃ phassaṃ paṭicca uppajjati sukhavedanā.}}\\
\begin{addmargin}[1em]{2em}
\setstretch{.5}
{\PaliGlossB{In the same way, pleasant feeling arises dependent on a contact to be experienced as pleasant.}}\\
\end{addmargin}
\end{absolutelynopagebreak}

\begin{absolutelynopagebreak}
\setstretch{.7}
{\PaliGlossA{tasseva sukhavedaniyassa phassassa nirodhā yaṃ tajjaṃ vedayitaṃ sukhavedaniyaṃ phassaṃ paṭicca uppannā sukhavedanā sā nirujjhati sā vūpasammati … pe …}}\\
\begin{addmargin}[1em]{2em}
\setstretch{.5}
{\PaliGlossB{With the cessation of that contact to be experienced as pleasant, the corresponding pleasant feeling ceases and stops.}}\\
\end{addmargin}
\end{absolutelynopagebreak}

\begin{absolutelynopagebreak}
\setstretch{.7}
{\PaliGlossA{dukkhavedaniyaṃ phassaṃ paṭicca …}}\\
\begin{addmargin}[1em]{2em}
\setstretch{.5}
{\PaliGlossB{Painful feeling …}}\\
\end{addmargin}
\end{absolutelynopagebreak}

\begin{absolutelynopagebreak}
\setstretch{.7}
{\PaliGlossA{adukkhamasukhavedaniyaṃ phassaṃ paṭicca uppajjati adukkhamasukhavedanā.}}\\
\begin{addmargin}[1em]{2em}
\setstretch{.5}
{\PaliGlossB{Neutral feeling arises dependent on a contact to be experienced as neutral.}}\\
\end{addmargin}
\end{absolutelynopagebreak}

\begin{absolutelynopagebreak}
\setstretch{.7}
{\PaliGlossA{tasseva adukkhamasukhavedaniyassa phassassa nirodhā yaṃ tajjaṃ vedayitaṃ adukkhamasukhavedaniyaṃ phassaṃ paṭicca uppannā adukkhamasukhavedanā sā nirujjhati sā vūpasammati.}}\\
\begin{addmargin}[1em]{2em}
\setstretch{.5}
{\PaliGlossB{With the cessation of that contact to be experienced as neutral, the corresponding neutral feeling ceases and stops.}}\\
\end{addmargin}
\end{absolutelynopagebreak}

\begin{absolutelynopagebreak}
\setstretch{.7}
{\PaliGlossA{evaṃ passaṃ, bhikkhave, sutavā ariyasāvako phassepi nibbindati, vedanāyapi nibbindati, saññāyapi nibbindati, saṅkhāresupi nibbindati, viññāṇasmimpi nibbindati;}}\\
\begin{addmargin}[1em]{2em}
\setstretch{.5}
{\PaliGlossB{Seeing this, a learned noble disciple grows disillusioned with form, feeling, perception, choices, and consciousness.}}\\
\end{addmargin}
\end{absolutelynopagebreak}

\begin{absolutelynopagebreak}
\setstretch{.7}
{\PaliGlossA{nibbindaṃ virajjati, virāgā vimuccati, vimuttasmiṃ vimuttamiti ñāṇaṃ hoti.}}\\
\begin{addmargin}[1em]{2em}
\setstretch{.5}
{\PaliGlossB{Being disillusioned, desire fades away. When desire fades away they’re freed. When they’re freed, they know they’re freed.}}\\
\end{addmargin}
\end{absolutelynopagebreak}

\begin{absolutelynopagebreak}
\setstretch{.7}
{\PaliGlossA{‘khīṇā jāti, vusitaṃ brahmacariyaṃ, kataṃ karaṇīyaṃ, nāparaṃ itthattāyā’ti pajānātī”ti.}}\\
\begin{addmargin}[1em]{2em}
\setstretch{.5}
{\PaliGlossB{They understand: ‘Rebirth is ended, the spiritual journey has been completed, what had to be done has been done, there is no return to any state of existence.’”}}\\
\end{addmargin}
\end{absolutelynopagebreak}

\begin{absolutelynopagebreak}
\setstretch{.7}
{\PaliGlossA{dutiyaṃ.}}\\
\begin{addmargin}[1em]{2em}
\setstretch{.5}
{\PaliGlossB{    -}}\\
\end{addmargin}
\end{absolutelynopagebreak}
