
\begin{absolutelynopagebreak}
\setstretch{.7}
{\PaliGlossA{saṃyutta nikāya 35}}\\
\begin{addmargin}[1em]{2em}
\setstretch{.5}
{\PaliGlossB{Linked Discourses 35}}\\
\end{addmargin}
\end{absolutelynopagebreak}

\begin{absolutelynopagebreak}
\setstretch{.7}
{\PaliGlossA{19. āsīvisavagga}}\\
\begin{addmargin}[1em]{2em}
\setstretch{.5}
{\PaliGlossB{19. The Simile of the Vipers}}\\
\end{addmargin}
\end{absolutelynopagebreak}

\begin{absolutelynopagebreak}
\setstretch{.7}
{\PaliGlossA{247. chappāṇakopamasutta}}\\
\begin{addmargin}[1em]{2em}
\setstretch{.5}
{\PaliGlossB{247. The Simile of Six Animals}}\\
\end{addmargin}
\end{absolutelynopagebreak}

\begin{absolutelynopagebreak}
\setstretch{.7}
{\PaliGlossA{“seyyathāpi, bhikkhave, puriso arugatto pakkagatto saravanaṃ paviseyya.}}\\
\begin{addmargin}[1em]{2em}
\setstretch{.5}
{\PaliGlossB{“Mendicants, suppose a person with wounded and festering limbs was to enter a thicket of thorny reeds.}}\\
\end{addmargin}
\end{absolutelynopagebreak}

\begin{absolutelynopagebreak}
\setstretch{.7}
{\PaliGlossA{tassa kusakaṇṭakā ceva pāde vijjheyyuṃ, sarapattāni ca gattāni vilekheyyuṃ.}}\\
\begin{addmargin}[1em]{2em}
\setstretch{.5}
{\PaliGlossB{The kusa thorns would pierce their feet, and the reed leaves would scratch their limbs.}}\\
\end{addmargin}
\end{absolutelynopagebreak}

\begin{absolutelynopagebreak}
\setstretch{.7}
{\PaliGlossA{evañhi so, bhikkhave, puriso bhiyyoso mattāya tatonidānaṃ dukkhaṃ domanassaṃ paṭisaṃvediyetha.}}\\
\begin{addmargin}[1em]{2em}
\setstretch{.5}
{\PaliGlossB{And that would cause that person to experience even more pain and distress.}}\\
\end{addmargin}
\end{absolutelynopagebreak}

\begin{absolutelynopagebreak}
\setstretch{.7}
{\PaliGlossA{evameva kho, bhikkhave, idhekacco bhikkhu gāmagato vā araññagato vā labhati vattāraṃ:}}\\
\begin{addmargin}[1em]{2em}
\setstretch{.5}
{\PaliGlossB{In the same way, some mendicant goes to a village or a wilderness and gets scolded,}}\\
\end{addmargin}
\end{absolutelynopagebreak}

\begin{absolutelynopagebreak}
\setstretch{.7}
{\PaliGlossA{‘ayañca so āyasmā evaṃkārī evaṃsamācāro asucigāmakaṇṭako’ti.}}\\
\begin{addmargin}[1em]{2em}
\setstretch{.5}
{\PaliGlossB{‘This venerable, acting like this, behaving like this, is a filthy village thorn.’}}\\
\end{addmargin}
\end{absolutelynopagebreak}

\begin{absolutelynopagebreak}
\setstretch{.7}
{\PaliGlossA{taṃ kaṇṭakoti iti viditvā saṃvaro ca asaṃvaro ca veditabbo.}}\\
\begin{addmargin}[1em]{2em}
\setstretch{.5}
{\PaliGlossB{Understanding that they’re a thorn, they should understand restraint and lack of restraint.}}\\
\end{addmargin}
\end{absolutelynopagebreak}

\begin{absolutelynopagebreak}
\setstretch{.7}
{\PaliGlossA{kathañca, bhikkhave, asaṃvaro hoti?}}\\
\begin{addmargin}[1em]{2em}
\setstretch{.5}
{\PaliGlossB{And how is someone unrestrained?}}\\
\end{addmargin}
\end{absolutelynopagebreak}

\begin{absolutelynopagebreak}
\setstretch{.7}
{\PaliGlossA{idha, bhikkhave, bhikkhu cakkhunā rūpaṃ disvā piyarūpe rūpe adhimuccati, appiyarūpe rūpe byāpajjati, anupaṭṭhitakāyassati ca viharati parittacetaso.}}\\
\begin{addmargin}[1em]{2em}
\setstretch{.5}
{\PaliGlossB{Take a mendicant who sees a sight with their eyes. If it’s pleasant they hold on to it, but if it’s unpleasant they dislike it. They live with mindfulness of the body unestablished and their heart restricted.}}\\
\end{addmargin}
\end{absolutelynopagebreak}

\begin{absolutelynopagebreak}
\setstretch{.7}
{\PaliGlossA{tañca cetovimuttiṃ paññāvimuttiṃ yathābhūtaṃ nappajānāti, yatthassa te uppannā pāpakā akusalā dhammā aparisesā nirujjhanti.}}\\
\begin{addmargin}[1em]{2em}
\setstretch{.5}
{\PaliGlossB{And they don’t truly understand the freedom of heart and freedom by wisdom where those arisen bad, unskillful qualities cease without anything left over.}}\\
\end{addmargin}
\end{absolutelynopagebreak}

\begin{absolutelynopagebreak}
\setstretch{.7}
{\PaliGlossA{sotena saddaṃ sutvā …}}\\
\begin{addmargin}[1em]{2em}
\setstretch{.5}
{\PaliGlossB{When they hear a sound with their ears …}}\\
\end{addmargin}
\end{absolutelynopagebreak}

\begin{absolutelynopagebreak}
\setstretch{.7}
{\PaliGlossA{ghānena gandhaṃ ghāyitvā …}}\\
\begin{addmargin}[1em]{2em}
\setstretch{.5}
{\PaliGlossB{When they smell an odor with their nose …}}\\
\end{addmargin}
\end{absolutelynopagebreak}

\begin{absolutelynopagebreak}
\setstretch{.7}
{\PaliGlossA{jivhāya rasaṃ sāyitvā …}}\\
\begin{addmargin}[1em]{2em}
\setstretch{.5}
{\PaliGlossB{When they taste a flavor with their tongue …}}\\
\end{addmargin}
\end{absolutelynopagebreak}

\begin{absolutelynopagebreak}
\setstretch{.7}
{\PaliGlossA{kāyena phoṭṭhabbaṃ phusitvā …}}\\
\begin{addmargin}[1em]{2em}
\setstretch{.5}
{\PaliGlossB{When they feel a touch with their body …}}\\
\end{addmargin}
\end{absolutelynopagebreak}

\begin{absolutelynopagebreak}
\setstretch{.7}
{\PaliGlossA{manasā dhammaṃ viññāya piyarūpe dhamme adhimuccati, appiyarūpe dhamme byāpajjati, anupaṭṭhitakāyassati ca viharati parittacetaso,}}\\
\begin{addmargin}[1em]{2em}
\setstretch{.5}
{\PaliGlossB{When they know a thought with their mind, if it’s pleasant they hold on to it, but if it’s unpleasant they dislike it. They live with mindfulness of the body unestablished and a limited heart.}}\\
\end{addmargin}
\end{absolutelynopagebreak}

\begin{absolutelynopagebreak}
\setstretch{.7}
{\PaliGlossA{tañca cetovimuttiṃ paññāvimuttiṃ yathābhūtaṃ nappajānāti, yatthassa te uppannā pāpakā akusalā dhammā aparisesā nirujjhanti.}}\\
\begin{addmargin}[1em]{2em}
\setstretch{.5}
{\PaliGlossB{And they don’t truly understand the freedom of heart and freedom by wisdom where those arisen bad, unskillful qualities cease without anything left over.}}\\
\end{addmargin}
\end{absolutelynopagebreak}

\begin{absolutelynopagebreak}
\setstretch{.7}
{\PaliGlossA{seyyathāpi, bhikkhave, puriso chappāṇake gahetvā nānāvisaye nānāgocare daḷhāya rajjuyā bandheyya.}}\\
\begin{addmargin}[1em]{2em}
\setstretch{.5}
{\PaliGlossB{Suppose a person was to catch six animals, with diverse territories and feeding grounds, and tie them up with a strong rope.}}\\
\end{addmargin}
\end{absolutelynopagebreak}

\begin{absolutelynopagebreak}
\setstretch{.7}
{\PaliGlossA{ahiṃ gahetvā daḷhāya rajjuyā bandheyya.}}\\
\begin{addmargin}[1em]{2em}
\setstretch{.5}
{\PaliGlossB{They’d catch a snake,}}\\
\end{addmargin}
\end{absolutelynopagebreak}

\begin{absolutelynopagebreak}
\setstretch{.7}
{\PaliGlossA{susumāraṃ gahetvā daḷhāya rajjuyā bandheyya.}}\\
\begin{addmargin}[1em]{2em}
\setstretch{.5}
{\PaliGlossB{a crocodile,}}\\
\end{addmargin}
\end{absolutelynopagebreak}

\begin{absolutelynopagebreak}
\setstretch{.7}
{\PaliGlossA{pakkhiṃ gahetvā daḷhāya rajjuyā bandheyya.}}\\
\begin{addmargin}[1em]{2em}
\setstretch{.5}
{\PaliGlossB{a bird,}}\\
\end{addmargin}
\end{absolutelynopagebreak}

\begin{absolutelynopagebreak}
\setstretch{.7}
{\PaliGlossA{kukkuraṃ gahetvā daḷhāya rajjuyā bandheyya.}}\\
\begin{addmargin}[1em]{2em}
\setstretch{.5}
{\PaliGlossB{a dog,}}\\
\end{addmargin}
\end{absolutelynopagebreak}

\begin{absolutelynopagebreak}
\setstretch{.7}
{\PaliGlossA{siṅgālaṃ gahetvā daḷhāya rajjuyā bandheyya.}}\\
\begin{addmargin}[1em]{2em}
\setstretch{.5}
{\PaliGlossB{a jackal,}}\\
\end{addmargin}
\end{absolutelynopagebreak}

\begin{absolutelynopagebreak}
\setstretch{.7}
{\PaliGlossA{makkaṭaṃ gahetvā daḷhāya rajjuyā bandheyya.}}\\
\begin{addmargin}[1em]{2em}
\setstretch{.5}
{\PaliGlossB{and a monkey,}}\\
\end{addmargin}
\end{absolutelynopagebreak}

\begin{absolutelynopagebreak}
\setstretch{.7}
{\PaliGlossA{daḷhāya rajjuyā bandhitvā majjhe gaṇṭhiṃ karitvā ossajjeyya.}}\\
\begin{addmargin}[1em]{2em}
\setstretch{.5}
{\PaliGlossB{tie each up with a strong rope, then tie a knot in the middle and let them loose.}}\\
\end{addmargin}
\end{absolutelynopagebreak}

\begin{absolutelynopagebreak}
\setstretch{.7}
{\PaliGlossA{atha kho, te, bhikkhave, chappāṇakā nānāvisayā nānāgocarā sakaṃ sakaṃ gocaravisayaṃ āviñcheyyuṃ—}}\\
\begin{addmargin}[1em]{2em}
\setstretch{.5}
{\PaliGlossB{Then those six animals with diverse domains and territories would each pull towards their own domain and territory.}}\\
\end{addmargin}
\end{absolutelynopagebreak}

\begin{absolutelynopagebreak}
\setstretch{.7}
{\PaliGlossA{ahi āviñcheyya ‘vammikaṃ pavekkhāmī’ti, susumāro āviñcheyya ‘udakaṃ pavekkhāmī’ti, pakkhī āviñcheyya ‘ākāsaṃ ḍessāmī’ti, kukkuro āviñcheyya ‘gāmaṃ pavekkhāmī’ti, siṅgālo āviñcheyya ‘sīvathikaṃ pavekkhāmī’ti, makkaṭo āviñcheyya ‘vanaṃ pavekkhāmī’ti.}}\\
\begin{addmargin}[1em]{2em}
\setstretch{.5}
{\PaliGlossB{The snake would pull one way, thinking ‘I’m going into an anthill!’ The crocodile would pull another way, thinking ‘I’m going into the water!’ The bird would pull another way, thinking ‘I’m flying into the sky!’ The dog would pull another way, thinking ‘I’m going into the village!’ The jackal would pull another way, thinking ‘I’m going into the charnel ground!’ The monkey would pull another way, thinking ‘I’m going into the jungle!’}}\\
\end{addmargin}
\end{absolutelynopagebreak}

\begin{absolutelynopagebreak}
\setstretch{.7}
{\PaliGlossA{yadā kho te, bhikkhave, chappāṇakā jhattā assu kilantā, atha kho yo nesaṃ pāṇakānaṃ balavataro assa tassa te anuvatteyyuṃ, anuvidhāyeyyuṃ vasaṃ gaccheyyuṃ.}}\\
\begin{addmargin}[1em]{2em}
\setstretch{.5}
{\PaliGlossB{When those six animals became exhausted and worn out, the strongest of them would get their way, and they’d all have to submit to their control.}}\\
\end{addmargin}
\end{absolutelynopagebreak}

\begin{absolutelynopagebreak}
\setstretch{.7}
{\PaliGlossA{evameva kho, bhikkhave, yassa kassaci bhikkhuno kāyagatāsati abhāvitā abahulīkatā, taṃ cakkhu āviñchati manāpiyesu rūpesu, amanāpiyā rūpā paṭikūlā honti … pe …}}\\
\begin{addmargin}[1em]{2em}
\setstretch{.5}
{\PaliGlossB{In the same way, when a mendicant has not developed or cultivated mindfulness of the body, their eye pulls towards pleasant sights, but is put off by unpleasant sights. Their ear … nose … tongue … body …}}\\
\end{addmargin}
\end{absolutelynopagebreak}

\begin{absolutelynopagebreak}
\setstretch{.7}
{\PaliGlossA{mano āviñchati manāpiyesu dhammesu, amanāpiyā dhammā paṭikūlā honti.}}\\
\begin{addmargin}[1em]{2em}
\setstretch{.5}
{\PaliGlossB{mind pulls towards pleasant thoughts, but is put off by unpleasant thoughts.}}\\
\end{addmargin}
\end{absolutelynopagebreak}

\begin{absolutelynopagebreak}
\setstretch{.7}
{\PaliGlossA{evaṃ kho, bhikkhave, asaṃvaro hoti.}}\\
\begin{addmargin}[1em]{2em}
\setstretch{.5}
{\PaliGlossB{This is how someone is unrestrained.}}\\
\end{addmargin}
\end{absolutelynopagebreak}

\begin{absolutelynopagebreak}
\setstretch{.7}
{\PaliGlossA{kathañca, bhikkhave, saṃvaro hoti?}}\\
\begin{addmargin}[1em]{2em}
\setstretch{.5}
{\PaliGlossB{And how is someone restrained?}}\\
\end{addmargin}
\end{absolutelynopagebreak}

\begin{absolutelynopagebreak}
\setstretch{.7}
{\PaliGlossA{idha, bhikkhave, bhikkhu cakkhunā rūpaṃ disvā piyarūpe rūpe nādhimuccati, appiyarūpe rūpe na byāpajjati, upaṭṭhitakāyassati ca viharati appamāṇacetaso,}}\\
\begin{addmargin}[1em]{2em}
\setstretch{.5}
{\PaliGlossB{Take a mendicant who sees a sight with their eyes. If it’s pleasant they don’t hold on to it, and if it’s unpleasant they don’t dislike it. They live with mindfulness of the body established and a limitless heart.}}\\
\end{addmargin}
\end{absolutelynopagebreak}

\begin{absolutelynopagebreak}
\setstretch{.7}
{\PaliGlossA{tañca cetovimuttiṃ paññāvimuttiṃ yathābhūtaṃ pajānāti, yatthassa te uppannā pāpakā akusalā dhammā aparisesā nirujjhanti … pe …}}\\
\begin{addmargin}[1em]{2em}
\setstretch{.5}
{\PaliGlossB{And they truly understand the freedom of heart and freedom by wisdom where those arisen bad, unskillful qualities cease without anything left over.}}\\
\end{addmargin}
\end{absolutelynopagebreak}

\begin{absolutelynopagebreak}
\setstretch{.7}
{\PaliGlossA{jivhāya rasaṃ sāyitvā … pe …}}\\
\begin{addmargin}[1em]{2em}
\setstretch{.5}
{\PaliGlossB{They hear a sound … smell an odor … taste a flavor … feel a touch …}}\\
\end{addmargin}
\end{absolutelynopagebreak}

\begin{absolutelynopagebreak}
\setstretch{.7}
{\PaliGlossA{manasā dhammaṃ viññāya piyarūpe dhamme nādhimuccati, appiyarūpe dhamme na byāpajjati, upaṭṭhitakāyassati ca viharati appamāṇacetaso,}}\\
\begin{addmargin}[1em]{2em}
\setstretch{.5}
{\PaliGlossB{know a thought with their mind. If it’s pleasant they don’t hold on to it, and if it’s unpleasant they don’t dislike it. They live with mindfulness of the body established and a limitless heart.}}\\
\end{addmargin}
\end{absolutelynopagebreak}

\begin{absolutelynopagebreak}
\setstretch{.7}
{\PaliGlossA{tañca cetovimuttiṃ paññāvimuttiṃ yathābhūtaṃ pajānāti yatthassa te uppannā pāpakā akusalā dhammā aparisesā nirujjhanti.}}\\
\begin{addmargin}[1em]{2em}
\setstretch{.5}
{\PaliGlossB{And they truly understand the freedom of heart and freedom by wisdom where those arisen bad, unskillful qualities cease without anything left over.}}\\
\end{addmargin}
\end{absolutelynopagebreak}

\begin{absolutelynopagebreak}
\setstretch{.7}
{\PaliGlossA{seyyathāpi, bhikkhave, puriso chappāṇake gahetvā nānāvisaye nānāgocare daḷhāya rajjuyā bandheyya.}}\\
\begin{addmargin}[1em]{2em}
\setstretch{.5}
{\PaliGlossB{Suppose a person was to catch six animals, with diverse territories and feeding grounds, and tie them up with a strong rope.}}\\
\end{addmargin}
\end{absolutelynopagebreak}

\begin{absolutelynopagebreak}
\setstretch{.7}
{\PaliGlossA{ahiṃ gahetvā daḷhāya rajjuyā bandheyya.}}\\
\begin{addmargin}[1em]{2em}
\setstretch{.5}
{\PaliGlossB{They’d catch a snake,}}\\
\end{addmargin}
\end{absolutelynopagebreak}

\begin{absolutelynopagebreak}
\setstretch{.7}
{\PaliGlossA{susumāraṃ gahetvā daḷhāya rajjuyā bandheyya.}}\\
\begin{addmargin}[1em]{2em}
\setstretch{.5}
{\PaliGlossB{a crocodile,}}\\
\end{addmargin}
\end{absolutelynopagebreak}

\begin{absolutelynopagebreak}
\setstretch{.7}
{\PaliGlossA{pakkhiṃ gahetvā … pe …}}\\
\begin{addmargin}[1em]{2em}
\setstretch{.5}
{\PaliGlossB{a bird,}}\\
\end{addmargin}
\end{absolutelynopagebreak}

\begin{absolutelynopagebreak}
\setstretch{.7}
{\PaliGlossA{kukkuraṃ gahetvā …}}\\
\begin{addmargin}[1em]{2em}
\setstretch{.5}
{\PaliGlossB{a dog,}}\\
\end{addmargin}
\end{absolutelynopagebreak}

\begin{absolutelynopagebreak}
\setstretch{.7}
{\PaliGlossA{siṅgālaṃ gahetvā …}}\\
\begin{addmargin}[1em]{2em}
\setstretch{.5}
{\PaliGlossB{a jackal,}}\\
\end{addmargin}
\end{absolutelynopagebreak}

\begin{absolutelynopagebreak}
\setstretch{.7}
{\PaliGlossA{makkaṭaṃ gahetvā daḷhāya rajjuyā bandheyya.}}\\
\begin{addmargin}[1em]{2em}
\setstretch{.5}
{\PaliGlossB{and a monkey,}}\\
\end{addmargin}
\end{absolutelynopagebreak}

\begin{absolutelynopagebreak}
\setstretch{.7}
{\PaliGlossA{daḷhāya rajjuyā bandhitvā daḷhe khīle vā thambhe vā upanibandheyya.}}\\
\begin{addmargin}[1em]{2em}
\setstretch{.5}
{\PaliGlossB{tie each up with a strong rope, then tether them to a strong post or pillar.}}\\
\end{addmargin}
\end{absolutelynopagebreak}

\begin{absolutelynopagebreak}
\setstretch{.7}
{\PaliGlossA{atha kho te, bhikkhave, chappāṇakā nānāvisayā nānāgocarā sakaṃ sakaṃ gocaravisayaṃ āviñcheyyuṃ—}}\\
\begin{addmargin}[1em]{2em}
\setstretch{.5}
{\PaliGlossB{Then those six animals with diverse domains and territories would each pull towards their own domain and territory.}}\\
\end{addmargin}
\end{absolutelynopagebreak}

\begin{absolutelynopagebreak}
\setstretch{.7}
{\PaliGlossA{ahi āviñcheyya ‘vammikaṃ pavekkhāmī’ti, susumāro āviñcheyya ‘udakaṃ pavekkhāmī’ti, pakkhī āviñcheyya ‘ākāsaṃ ḍessāmī’ti, kukkuro āviñcheyya ‘gāmaṃ pavekkhāmī’ti, siṅgālo āviñcheyya ‘sīvathikaṃ pavekkhāmī’ti, makkaṭo āviñcheyya ‘vanaṃ pavekkhāmī’ti.}}\\
\begin{addmargin}[1em]{2em}
\setstretch{.5}
{\PaliGlossB{The snake would pull one way, thinking ‘I’m going into an anthill!’ The crocodile would pull another way, thinking ‘I’m going into the water!’ The bird would pull another way, thinking ‘I’m flying into the sky!’ The dog would pull another way, thinking ‘I’m going into the village!’ The jackal would pull another way, thinking ‘I’m going into the charnel ground!’ The monkey would pull another way, thinking ‘I’m going into the jungle!’}}\\
\end{addmargin}
\end{absolutelynopagebreak}

\begin{absolutelynopagebreak}
\setstretch{.7}
{\PaliGlossA{yadā kho te, bhikkhave, chappāṇakā jhattā assu kilantā, atha tameva khīlaṃ vā thambhaṃ vā upatiṭṭheyyuṃ, upanisīdeyyuṃ, upanipajjeyyuṃ.}}\\
\begin{addmargin}[1em]{2em}
\setstretch{.5}
{\PaliGlossB{When those six animals became exhausted and worn out, they’d stand or sit or lie down right by that post or pillar.}}\\
\end{addmargin}
\end{absolutelynopagebreak}

\begin{absolutelynopagebreak}
\setstretch{.7}
{\PaliGlossA{evameva kho, bhikkhave, yassa kassaci bhikkhuno kāyagatāsati bhāvitā bahulīkatā, taṃ cakkhu nāviñchati manāpiyesu rūpesu, amanāpiyā rūpā nappaṭikūlā honti … pe … jivhā nāviñchati manāpiyesu rasesu … pe …}}\\
\begin{addmargin}[1em]{2em}
\setstretch{.5}
{\PaliGlossB{In the same way, when a mendicant has developed and cultivated mindfulness of the body, their eye doesn’t pull towards pleasant sights, and isn’t put off by unpleasant sights. Their ear … nose … tongue … body …}}\\
\end{addmargin}
\end{absolutelynopagebreak}

\begin{absolutelynopagebreak}
\setstretch{.7}
{\PaliGlossA{mano nāviñchati manāpiyesu dhammesu, amanāpiyā dhammā nappaṭikūlā honti.}}\\
\begin{addmargin}[1em]{2em}
\setstretch{.5}
{\PaliGlossB{mind doesn’t pull towards pleasant thoughts, and isn’t put off by unpleasant thoughts.}}\\
\end{addmargin}
\end{absolutelynopagebreak}

\begin{absolutelynopagebreak}
\setstretch{.7}
{\PaliGlossA{evaṃ kho, bhikkhave, saṃvaro hoti.}}\\
\begin{addmargin}[1em]{2em}
\setstretch{.5}
{\PaliGlossB{This is how someone is restrained.}}\\
\end{addmargin}
\end{absolutelynopagebreak}

\begin{absolutelynopagebreak}
\setstretch{.7}
{\PaliGlossA{‘daḷhe khīle vā thambhe vā’ti kho, bhikkhave, kāyagatāya satiyā etaṃ adhivacanaṃ.}}\\
\begin{addmargin}[1em]{2em}
\setstretch{.5}
{\PaliGlossB{‘A strong post or pillar’ is a term for mindfulness of the body.}}\\
\end{addmargin}
\end{absolutelynopagebreak}

\begin{absolutelynopagebreak}
\setstretch{.7}
{\PaliGlossA{tasmātiha vo, bhikkhave, evaṃ sikkhitabbaṃ:}}\\
\begin{addmargin}[1em]{2em}
\setstretch{.5}
{\PaliGlossB{So you should train like this:}}\\
\end{addmargin}
\end{absolutelynopagebreak}

\begin{absolutelynopagebreak}
\setstretch{.7}
{\PaliGlossA{‘kāyagatā no sati bhāvitā bhavissati bahulīkatā yānīkatā vatthukatā anuṭṭhitā paricitā susamāraddhā’ti.}}\\
\begin{addmargin}[1em]{2em}
\setstretch{.5}
{\PaliGlossB{‘We will develop mindfulness of the body. We’ll cultivate it, make it our vehicle and our basis, keep it up, consolidate it, and properly implement it.’}}\\
\end{addmargin}
\end{absolutelynopagebreak}

\begin{absolutelynopagebreak}
\setstretch{.7}
{\PaliGlossA{evañhi kho, bhikkhave, sikkhitabban”ti.}}\\
\begin{addmargin}[1em]{2em}
\setstretch{.5}
{\PaliGlossB{That’s how you should train.”}}\\
\end{addmargin}
\end{absolutelynopagebreak}

\begin{absolutelynopagebreak}
\setstretch{.7}
{\PaliGlossA{dasamaṃ.}}\\
\begin{addmargin}[1em]{2em}
\setstretch{.5}
{\PaliGlossB{    -}}\\
\end{addmargin}
\end{absolutelynopagebreak}
