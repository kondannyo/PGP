
\begin{absolutelynopagebreak}
\setstretch{.7}
{\PaliGlossA{saṃyutta nikāya 35}}\\
\begin{addmargin}[1em]{2em}
\setstretch{.5}
{\PaliGlossB{Linked Discourses 35}}\\
\end{addmargin}
\end{absolutelynopagebreak}

\begin{absolutelynopagebreak}
\setstretch{.7}
{\PaliGlossA{15. navapurāṇavagga}}\\
\begin{addmargin}[1em]{2em}
\setstretch{.5}
{\PaliGlossB{15. The Old and the New}}\\
\end{addmargin}
\end{absolutelynopagebreak}

\begin{absolutelynopagebreak}
\setstretch{.7}
{\PaliGlossA{153. atthinukhopariyāyasutta}}\\
\begin{addmargin}[1em]{2em}
\setstretch{.5}
{\PaliGlossB{153. Is There a Method?}}\\
\end{addmargin}
\end{absolutelynopagebreak}

\begin{absolutelynopagebreak}
\setstretch{.7}
{\PaliGlossA{“atthi nu kho, bhikkhave, pariyāyo yaṃ pariyāyaṃ āgamma bhikkhu aññatreva saddhāya, aññatra ruciyā, aññatra anussavā, aññatra ākāraparivitakkā, aññatra diṭṭhinijjhānakkhantiyā aññaṃ byākareyya:}}\\
\begin{addmargin}[1em]{2em}
\setstretch{.5}
{\PaliGlossB{“Mendicants, is there a method—apart from faith, preference, oral tradition, reasoned contemplation, or acceptance of a view after consideration—that a mendicant can rely on to declare their enlightenment? That is:}}\\
\end{addmargin}
\end{absolutelynopagebreak}

\begin{absolutelynopagebreak}
\setstretch{.7}
{\PaliGlossA{‘khīṇā jāti, vusitaṃ brahmacariyaṃ, kataṃ karaṇīyaṃ, nāparaṃ itthattāyā’ti pajānāmī”ti?}}\\
\begin{addmargin}[1em]{2em}
\setstretch{.5}
{\PaliGlossB{‘I understand: “Rebirth is ended, the spiritual journey has been completed, what had to be done has been done, there is no return to any state of existence.”’”}}\\
\end{addmargin}
\end{absolutelynopagebreak}

\begin{absolutelynopagebreak}
\setstretch{.7}
{\PaliGlossA{“bhagavaṃmūlakā no, bhante, dhammā, bhagavaṃnettikā bhagavaṃpaṭisaraṇā. sādhu vata, bhante, bhagavantaṃyeva paṭibhātu etassa bhāsitassa attho. bhagavato sutvā bhikkhū dhāressantī”ti.}}\\
\begin{addmargin}[1em]{2em}
\setstretch{.5}
{\PaliGlossB{“Our teachings are rooted in the Buddha. He is our guide and our refuge. Sir, may the Buddha himself please clarify the meaning of this. The mendicants will listen and remember it.”}}\\
\end{addmargin}
\end{absolutelynopagebreak}

\begin{absolutelynopagebreak}
\setstretch{.7}
{\PaliGlossA{“tena hi, bhikkhave, suṇātha, sādhukaṃ manasi karotha, bhāsissāmī”ti.}}\\
\begin{addmargin}[1em]{2em}
\setstretch{.5}
{\PaliGlossB{“Well then, mendicants, listen and pay close attention, I will speak.”}}\\
\end{addmargin}
\end{absolutelynopagebreak}

\begin{absolutelynopagebreak}
\setstretch{.7}
{\PaliGlossA{“evaṃ, bhante”ti kho te bhikkhū bhagavato paccassosuṃ.}}\\
\begin{addmargin}[1em]{2em}
\setstretch{.5}
{\PaliGlossB{“Yes, sir,” they replied.}}\\
\end{addmargin}
\end{absolutelynopagebreak}

\begin{absolutelynopagebreak}
\setstretch{.7}
{\PaliGlossA{bhagavā etadavoca:}}\\
\begin{addmargin}[1em]{2em}
\setstretch{.5}
{\PaliGlossB{The Buddha said this:}}\\
\end{addmargin}
\end{absolutelynopagebreak}

\begin{absolutelynopagebreak}
\setstretch{.7}
{\PaliGlossA{“atthi, bhikkhave, pariyāyo yaṃ pariyāyaṃ āgamma bhikkhu aññatreva saddhāya, aññatra ruciyā, aññatra anussavā, aññatra ākāraparivitakkā, aññatra diṭṭhinijjhānakkhantiyā aññaṃ byākareyya:}}\\
\begin{addmargin}[1em]{2em}
\setstretch{.5}
{\PaliGlossB{“There is a method—apart from faith, preference, oral tradition, reasoned contemplation, or acceptance of a view after consideration—that a mendicant can rely on to declare their enlightenment. That is:}}\\
\end{addmargin}
\end{absolutelynopagebreak}

\begin{absolutelynopagebreak}
\setstretch{.7}
{\PaliGlossA{‘“khīṇā jāti, vusitaṃ brahmacariyaṃ, kataṃ karaṇīyaṃ, nāparaṃ itthattāyā”ti pajānāmī’ti.}}\\
\begin{addmargin}[1em]{2em}
\setstretch{.5}
{\PaliGlossB{‘I understand: “Rebirth is ended, the spiritual journey has been completed, what had to be done has been done, there is no return to any state of existence.”’}}\\
\end{addmargin}
\end{absolutelynopagebreak}

\begin{absolutelynopagebreak}
\setstretch{.7}
{\PaliGlossA{katamo ca, bhikkhave, pariyāyo, yaṃ pariyāyaṃ āgamma bhikkhu aññatreva saddhāya … pe … aññatra diṭṭhinijjhānakkhantiyā aññaṃ byākaroti: ‘khīṇā jāti, vusitaṃ brahmacariyaṃ, kataṃ karaṇīyaṃ, nāparaṃ itthattāyāti pajānāmī’ti?}}\\
\begin{addmargin}[1em]{2em}
\setstretch{.5}
{\PaliGlossB{And what is that method?}}\\
\end{addmargin}
\end{absolutelynopagebreak}

\begin{absolutelynopagebreak}
\setstretch{.7}
{\PaliGlossA{idha, bhikkhave, bhikkhu cakkhunā rūpaṃ disvā santaṃ vā ajjhattaṃ rāgadosamohaṃ, atthi me ajjhattaṃ rāgadosamohoti pajānāti;}}\\
\begin{addmargin}[1em]{2em}
\setstretch{.5}
{\PaliGlossB{Take a mendicant who sees a sight with the eye. When they have greed, hate, and delusion in them, they understand ‘I have greed, hate, and delusion in me.’}}\\
\end{addmargin}
\end{absolutelynopagebreak}

\begin{absolutelynopagebreak}
\setstretch{.7}
{\PaliGlossA{asantaṃ vā ajjhattaṃ rāgadosamohaṃ, natthi me ajjhattaṃ rāgadosamohoti pajānāti.}}\\
\begin{addmargin}[1em]{2em}
\setstretch{.5}
{\PaliGlossB{When they don’t have greed, hate, and delusion in them, they understand ‘I don’t have greed, hate, and delusion in me.’}}\\
\end{addmargin}
\end{absolutelynopagebreak}

\begin{absolutelynopagebreak}
\setstretch{.7}
{\PaliGlossA{yaṃ taṃ, bhikkhave, bhikkhu cakkhunā rūpaṃ disvā santaṃ vā ajjhattaṃ rāgadosamohaṃ, atthi me ajjhattaṃ rāgadosamohoti pajānāti; asantaṃ vā ajjhattaṃ rāgadosamohaṃ, natthi me ajjhattaṃ rāgadosamohoti pajānāti. api nu me, bhikkhave, dhammā saddhāya vā veditabbā, ruciyā vā veditabbā, anussavena vā veditabbā, ākāraparivitakkena vā veditabbā, diṭṭhinijjhānakkhantiyā vā veditabbā”ti?}}\\
\begin{addmargin}[1em]{2em}
\setstretch{.5}
{\PaliGlossB{Since this is so, are these things understood by faith, preference, oral tradition, reasoned contemplation, or acceptance of a view after consideration?”}}\\
\end{addmargin}
\end{absolutelynopagebreak}

\begin{absolutelynopagebreak}
\setstretch{.7}
{\PaliGlossA{“no hetaṃ, bhante”.}}\\
\begin{addmargin}[1em]{2em}
\setstretch{.5}
{\PaliGlossB{“No, sir.”}}\\
\end{addmargin}
\end{absolutelynopagebreak}

\begin{absolutelynopagebreak}
\setstretch{.7}
{\PaliGlossA{“nanume, bhikkhave, dhammā paññāya disvā veditabbā”ti?}}\\
\begin{addmargin}[1em]{2em}
\setstretch{.5}
{\PaliGlossB{“Aren’t they understood by seeing them with wisdom?”}}\\
\end{addmargin}
\end{absolutelynopagebreak}

\begin{absolutelynopagebreak}
\setstretch{.7}
{\PaliGlossA{“evaṃ, bhante”.}}\\
\begin{addmargin}[1em]{2em}
\setstretch{.5}
{\PaliGlossB{“Yes, sir.”}}\\
\end{addmargin}
\end{absolutelynopagebreak}

\begin{absolutelynopagebreak}
\setstretch{.7}
{\PaliGlossA{“ayaṃ kho, bhikkhave, pariyāyo yaṃ pariyāyaṃ āgamma bhikkhu aññatreva saddhāya, aññatra ruciyā, aññatra anussavā, aññatra ākāraparivitakkā, aññatra diṭṭhinijjhānakkhantiyā aññaṃ byākaroti:}}\\
\begin{addmargin}[1em]{2em}
\setstretch{.5}
{\PaliGlossB{“This is a method—apart from faith, preference, oral tradition, reasoned contemplation, or acceptance of a view after consideration—that a mendicant can rely on to declare their enlightenment. That is:}}\\
\end{addmargin}
\end{absolutelynopagebreak}

\begin{absolutelynopagebreak}
\setstretch{.7}
{\PaliGlossA{‘“khīṇā jāti, vusitaṃ brahmacariyaṃ, kataṃ karaṇīyaṃ, nāparaṃ itthattāyā”ti pajānāmī’ti … pe ….}}\\
\begin{addmargin}[1em]{2em}
\setstretch{.5}
{\PaliGlossB{‘I understand: “Rebirth is ended, the spiritual journey has been completed, what had to be done has been done, there is no return to any state of existence.”’}}\\
\end{addmargin}
\end{absolutelynopagebreak}

\begin{absolutelynopagebreak}
\setstretch{.7}
{\PaliGlossA{puna caparaṃ, bhikkhave, bhikkhu jivhāya rasaṃ sāyitvā santaṃ vā ajjhattaṃ … pe … rāgadosamohoti pajānāti; asantaṃ vā ajjhattaṃ rāgadosamohaṃ, natthi me ajjhattaṃ rāgadosamohoti pajānāti.}}\\
\begin{addmargin}[1em]{2em}
\setstretch{.5}
{\PaliGlossB{Furthermore, a mendicant hears a sound … smells an odor … tastes a flavor … feels a touch …}}\\
\end{addmargin}
\end{absolutelynopagebreak}

\begin{absolutelynopagebreak}
\setstretch{.7}
{\PaliGlossA{yaṃ taṃ, bhikkhave, jivhāya rasaṃ sāyitvā santaṃ vā ajjhattaṃ rāgadosamohaṃ, atthi me ajjhattaṃ rāgadosamohoti pajānāti; asantaṃ vā ajjhattaṃ rāgadosamohaṃ, natthi me ajjhattaṃ rāgadosamohoti pajānāti; api nu me, bhikkhave, dhammā saddhāya vā veditabbā, ruciyā vā veditabbā, anussavena vā veditabbā, ākāraparivitakkena vā veditabbā, diṭṭhinijjhānakkhantiyā vā veditabbā”ti?}}\\
\begin{addmargin}[1em]{2em}
\setstretch{.5}
{\PaliGlossB{    -}}\\
\end{addmargin}
\end{absolutelynopagebreak}

\begin{absolutelynopagebreak}
\setstretch{.7}
{\PaliGlossA{“no hetaṃ, bhante”.}}\\
\begin{addmargin}[1em]{2em}
\setstretch{.5}
{\PaliGlossB{    -}}\\
\end{addmargin}
\end{absolutelynopagebreak}

\begin{absolutelynopagebreak}
\setstretch{.7}
{\PaliGlossA{“nanume, bhikkhave, dhammā paññāya disvā veditabbā”ti?}}\\
\begin{addmargin}[1em]{2em}
\setstretch{.5}
{\PaliGlossB{    -}}\\
\end{addmargin}
\end{absolutelynopagebreak}

\begin{absolutelynopagebreak}
\setstretch{.7}
{\PaliGlossA{“evaṃ, bhante”.}}\\
\begin{addmargin}[1em]{2em}
\setstretch{.5}
{\PaliGlossB{    -}}\\
\end{addmargin}
\end{absolutelynopagebreak}

\begin{absolutelynopagebreak}
\setstretch{.7}
{\PaliGlossA{“ayampi kho, bhikkhave, pariyāyo yaṃ pariyāyaṃ āgamma bhikkhu aññatreva saddhāya, aññatra ruciyā, aññatra anussavā, aññatra ākāraparivitakkā, aññatra diṭṭhinijjhānakkhantiyā aññaṃ byākaroti: ‘“khīṇā jāti, vusitaṃ brahmacariyaṃ, kataṃ karaṇīyaṃ, nāparaṃ itthattāyā”ti pajānāmī’ti … pe ….}}\\
\begin{addmargin}[1em]{2em}
\setstretch{.5}
{\PaliGlossB{    -}}\\
\end{addmargin}
\end{absolutelynopagebreak}

\begin{absolutelynopagebreak}
\setstretch{.7}
{\PaliGlossA{puna caparaṃ, bhikkhave, bhikkhu manasā dhammaṃ viññāya santaṃ vā ajjhattaṃ rāgadosamohaṃ, atthi me ajjhattaṃ rāgadosamohoti pajānāti;}}\\
\begin{addmargin}[1em]{2em}
\setstretch{.5}
{\PaliGlossB{knows a thought with the mind. When they have greed, hate, and delusion in them, they understand ‘I have greed, hate, and delusion in me.’}}\\
\end{addmargin}
\end{absolutelynopagebreak}

\begin{absolutelynopagebreak}
\setstretch{.7}
{\PaliGlossA{asantaṃ vā ajjhattaṃ rāgadosamohaṃ, natthi me ajjhattaṃ rāgadosamohoti pajānāti.}}\\
\begin{addmargin}[1em]{2em}
\setstretch{.5}
{\PaliGlossB{When they don’t have greed, hate, and delusion in them, they understand ‘I don’t have greed, hate, and delusion in me.’}}\\
\end{addmargin}
\end{absolutelynopagebreak}

\begin{absolutelynopagebreak}
\setstretch{.7}
{\PaliGlossA{yaṃ taṃ, bhikkhave, bhikkhu manasā dhammaṃ viññāya santaṃ vā ajjhattaṃ rāgadosamohaṃ, atthi me ajjhattaṃ rāgadosamohoti pajānāti; asantaṃ vā ajjhattaṃ rāgadosamohaṃ, natthi me ajjhattaṃ rāgadosamohoti pajānāti; api nu me, bhikkhave, dhammā saddhāya vā veditabbā, ruciyā vā veditabbā, anussavena vā veditabbā, ākāraparivitakkena vā veditabbā, diṭṭhinijjhānakkhantiyā vā veditabbā”ti?}}\\
\begin{addmargin}[1em]{2em}
\setstretch{.5}
{\PaliGlossB{Since this is so, are these things understood by faith, preference, oral tradition, reasoned contemplation, or acceptance of a view after consideration?”}}\\
\end{addmargin}
\end{absolutelynopagebreak}

\begin{absolutelynopagebreak}
\setstretch{.7}
{\PaliGlossA{“no hetaṃ, bhante”.}}\\
\begin{addmargin}[1em]{2em}
\setstretch{.5}
{\PaliGlossB{“No, sir.”}}\\
\end{addmargin}
\end{absolutelynopagebreak}

\begin{absolutelynopagebreak}
\setstretch{.7}
{\PaliGlossA{“nanume, bhikkhave, dhammā paññāya disvā veditabbā”ti?}}\\
\begin{addmargin}[1em]{2em}
\setstretch{.5}
{\PaliGlossB{“Aren’t they understood by seeing them with wisdom?”}}\\
\end{addmargin}
\end{absolutelynopagebreak}

\begin{absolutelynopagebreak}
\setstretch{.7}
{\PaliGlossA{“evaṃ, bhante”.}}\\
\begin{addmargin}[1em]{2em}
\setstretch{.5}
{\PaliGlossB{“Yes, sir.”}}\\
\end{addmargin}
\end{absolutelynopagebreak}

\begin{absolutelynopagebreak}
\setstretch{.7}
{\PaliGlossA{“ayampi kho, bhikkhave, pariyāyo yaṃ pariyāyaṃ āgamma bhikkhu aññatreva saddhāya, aññatra ruciyā, aññatra anussavā, aññatra ākāraparivitakkā, aññatra diṭṭhinijjhānakkhantiyā aññaṃ byākaroti:}}\\
\begin{addmargin}[1em]{2em}
\setstretch{.5}
{\PaliGlossB{“This too is a method—apart from faith, preference, oral tradition, reasoned contemplation, or acceptance of a view after consideration—that a mendicant can rely on to declare their enlightenment. That is:}}\\
\end{addmargin}
\end{absolutelynopagebreak}

\begin{absolutelynopagebreak}
\setstretch{.7}
{\PaliGlossA{‘“khīṇā jāti, vusitaṃ brahmacariyaṃ, kataṃ karaṇīyaṃ, nāparaṃ itthattāyā”ti pajānāmī’ti.}}\\
\begin{addmargin}[1em]{2em}
\setstretch{.5}
{\PaliGlossB{‘I understand: “Rebirth is ended, the spiritual journey has been completed, what had to be done has been done, there is no return to any state of existence.”’”}}\\
\end{addmargin}
\end{absolutelynopagebreak}

\begin{absolutelynopagebreak}
\setstretch{.7}
{\PaliGlossA{aṭṭhamaṃ.}}\\
\begin{addmargin}[1em]{2em}
\setstretch{.5}
{\PaliGlossB{    -}}\\
\end{addmargin}
\end{absolutelynopagebreak}
