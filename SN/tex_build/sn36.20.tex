
\begin{absolutelynopagebreak}
\setstretch{.7}
{\PaliGlossA{saṃyutta nikāya 36}}\\
\begin{addmargin}[1em]{2em}
\setstretch{.5}
{\PaliGlossB{Linked Discourses 36}}\\
\end{addmargin}
\end{absolutelynopagebreak}

\begin{absolutelynopagebreak}
\setstretch{.7}
{\PaliGlossA{2. rahogatavagga}}\\
\begin{addmargin}[1em]{2em}
\setstretch{.5}
{\PaliGlossB{2. In Private}}\\
\end{addmargin}
\end{absolutelynopagebreak}

\begin{absolutelynopagebreak}
\setstretch{.7}
{\PaliGlossA{20. bhikkhusutta}}\\
\begin{addmargin}[1em]{2em}
\setstretch{.5}
{\PaliGlossB{20. A Mendicant}}\\
\end{addmargin}
\end{absolutelynopagebreak}

\begin{absolutelynopagebreak}
\setstretch{.7}
{\PaliGlossA{“dvepi mayā, bhikkhave, vedanā vuttā pariyāyena, tissopi mayā vedanā vuttā pariyāyena, pañcapi mayā vedanā vuttā pariyāyena, chapi mayā vedanā vuttā pariyāyena, aṭṭhārasāpi mayā vedanā vuttā pariyāyena, chattiṃsāpi mayā vedanā vuttā pariyāyena, aṭṭhasatampi mayā vedanā vuttā pariyāyena.}}\\
\begin{addmargin}[1em]{2em}
\setstretch{.5}
{\PaliGlossB{“Mendicants, in one explanation I’ve spoken of two feelings. In another explanation I’ve spoken of three feelings, or five, six, eighteen, thirty-six, or a hundred and eight feelings.}}\\
\end{addmargin}
\end{absolutelynopagebreak}

\begin{absolutelynopagebreak}
\setstretch{.7}
{\PaliGlossA{evaṃ pariyāyadesito, bhikkhave, mayā dhammo.}}\\
\begin{addmargin}[1em]{2em}
\setstretch{.5}
{\PaliGlossB{I’ve taught the Dhamma with all these explanations.}}\\
\end{addmargin}
\end{absolutelynopagebreak}

\begin{absolutelynopagebreak}
\setstretch{.7}
{\PaliGlossA{evaṃ pariyāyadesite kho, bhikkhave, mayā dhamme ye aññamaññassa subhāsitaṃ sulapitaṃ na samanumaññissanti, na samanujānissanti, na samanumodissanti, tesaṃ etaṃ pāṭikaṅkhaṃ—bhaṇḍanajātā kalahajātā vivādāpannā aññamaññaṃ mukhasattīhi vitudantā viharissantīti.}}\\
\begin{addmargin}[1em]{2em}
\setstretch{.5}
{\PaliGlossB{This being so, you can expect that those who don’t concede, approve, or agree with what has been well spoken will argue, quarrel, and fight, continually wounding each other with barbed words.}}\\
\end{addmargin}
\end{absolutelynopagebreak}

\begin{absolutelynopagebreak}
\setstretch{.7}
{\PaliGlossA{evaṃ pariyāyadesito, bhikkhave, mayā dhammo.}}\\
\begin{addmargin}[1em]{2em}
\setstretch{.5}
{\PaliGlossB{I’ve taught the Dhamma with all these explanations.}}\\
\end{addmargin}
\end{absolutelynopagebreak}

\begin{absolutelynopagebreak}
\setstretch{.7}
{\PaliGlossA{evaṃ pariyāyadesite kho, bhikkhave, mayā dhamme ye aññamaññassa subhāsitaṃ sulapitaṃ samanumaññissanti samanujānissanti samanumodissanti, tesaṃ etaṃ pāṭikaṅkhaṃ—samaggā sammodamānā avivadamānā khīrodakībhūtā aññamaññaṃ piyacakkhūhi sampassantā viharissantīti.}}\\
\begin{addmargin}[1em]{2em}
\setstretch{.5}
{\PaliGlossB{This being so, you can expect that those who do concede, approve, or agree with what has been well spoken will live in harmony, appreciating each other, without quarreling, blending like milk and water, and regarding each other with kindly eyes.}}\\
\end{addmargin}
\end{absolutelynopagebreak}

\begin{absolutelynopagebreak}
\setstretch{.7}
{\PaliGlossA{pañcime, bhikkhave, kāmaguṇā … pe …}}\\
\begin{addmargin}[1em]{2em}
\setstretch{.5}
{\PaliGlossB{There are these five kinds of sensual stimulation. …}}\\
\end{addmargin}
\end{absolutelynopagebreak}

\begin{absolutelynopagebreak}
\setstretch{.7}
{\PaliGlossA{ṭhānaṃ kho panetaṃ, bhikkhave, vijjati yaṃ aññatitthiyā paribbājakā evaṃ vadeyyuṃ:}}\\
\begin{addmargin}[1em]{2em}
\setstretch{.5}
{\PaliGlossB{It’s possible that wanderers who follow other paths might say:}}\\
\end{addmargin}
\end{absolutelynopagebreak}

\begin{absolutelynopagebreak}
\setstretch{.7}
{\PaliGlossA{‘saññāvedayitanirodhaṃ samaṇo gotamo āha, tañca sukhasmiṃ paññapeti.}}\\
\begin{addmargin}[1em]{2em}
\setstretch{.5}
{\PaliGlossB{‘The ascetic Gotama spoke of the cessation of perception and feeling, and he includes it in happiness.}}\\
\end{addmargin}
\end{absolutelynopagebreak}

\begin{absolutelynopagebreak}
\setstretch{.7}
{\PaliGlossA{tayidaṃ kiṃsu, tayidaṃ kathaṃsū’ti?}}\\
\begin{addmargin}[1em]{2em}
\setstretch{.5}
{\PaliGlossB{What’s up with that?’}}\\
\end{addmargin}
\end{absolutelynopagebreak}

\begin{absolutelynopagebreak}
\setstretch{.7}
{\PaliGlossA{evaṃvādino, bhikkhave, aññatitthiyā paribbājakā evamassu vacanīyā:}}\\
\begin{addmargin}[1em]{2em}
\setstretch{.5}
{\PaliGlossB{Mendicants, when wanderers who follow other paths say this, you should say to them:}}\\
\end{addmargin}
\end{absolutelynopagebreak}

\begin{absolutelynopagebreak}
\setstretch{.7}
{\PaliGlossA{‘na kho, āvuso, bhagavā sukhaññeva vedanaṃ sandhāya sukhasmiṃ paññapeti.}}\\
\begin{addmargin}[1em]{2em}
\setstretch{.5}
{\PaliGlossB{‘Reverends, when the Buddha describes what’s included in happiness, he’s not just referring to pleasant feeling.}}\\
\end{addmargin}
\end{absolutelynopagebreak}

\begin{absolutelynopagebreak}
\setstretch{.7}
{\PaliGlossA{yattha yattha, āvuso, sukhaṃ upalabbhati yahiṃ yahiṃ, taṃ taṃ tathāgato sukhasmiṃ paññapetī’”ti.}}\\
\begin{addmargin}[1em]{2em}
\setstretch{.5}
{\PaliGlossB{The Realized One describes pleasure as included in happiness wherever it’s found, and in whatever context.’”}}\\
\end{addmargin}
\end{absolutelynopagebreak}

\begin{absolutelynopagebreak}
\setstretch{.7}
{\PaliGlossA{dasamaṃ.}}\\
\begin{addmargin}[1em]{2em}
\setstretch{.5}
{\PaliGlossB{    -}}\\
\end{addmargin}
\end{absolutelynopagebreak}

\begin{absolutelynopagebreak}
\setstretch{.7}
{\PaliGlossA{rahogatavaggo dutiyo.}}\\
\begin{addmargin}[1em]{2em}
\setstretch{.5}
{\PaliGlossB{    -}}\\
\end{addmargin}
\end{absolutelynopagebreak}

\begin{absolutelynopagebreak}
\setstretch{.7}
{\PaliGlossA{rahogataṃ dve ākāsaṃ,}}\\
\begin{addmargin}[1em]{2em}
\setstretch{.5}
{\PaliGlossB{    -}}\\
\end{addmargin}
\end{absolutelynopagebreak}

\begin{absolutelynopagebreak}
\setstretch{.7}
{\PaliGlossA{agāraṃ dve ca ānandā;}}\\
\begin{addmargin}[1em]{2em}
\setstretch{.5}
{\PaliGlossB{    -}}\\
\end{addmargin}
\end{absolutelynopagebreak}

\begin{absolutelynopagebreak}
\setstretch{.7}
{\PaliGlossA{sambahulā duve vuttā,}}\\
\begin{addmargin}[1em]{2em}
\setstretch{.5}
{\PaliGlossB{    -}}\\
\end{addmargin}
\end{absolutelynopagebreak}

\begin{absolutelynopagebreak}
\setstretch{.7}
{\PaliGlossA{pañcakaṅgo ca bhikkhunāti.}}\\
\begin{addmargin}[1em]{2em}
\setstretch{.5}
{\PaliGlossB{    -}}\\
\end{addmargin}
\end{absolutelynopagebreak}
