
\begin{absolutelynopagebreak}
\setstretch{.7}
{\PaliGlossA{saṃyutta nikāya 24}}\\
\begin{addmargin}[1em]{2em}
\setstretch{.5}
{\PaliGlossB{Linked Discourses 24}}\\
\end{addmargin}
\end{absolutelynopagebreak}

\begin{absolutelynopagebreak}
\setstretch{.7}
{\PaliGlossA{1. sotāpattivagga}}\\
\begin{addmargin}[1em]{2em}
\setstretch{.5}
{\PaliGlossB{1. Stream-Entry}}\\
\end{addmargin}
\end{absolutelynopagebreak}

\begin{absolutelynopagebreak}
\setstretch{.7}
{\PaliGlossA{4. nocamesiyāsutta}}\\
\begin{addmargin}[1em]{2em}
\setstretch{.5}
{\PaliGlossB{4. It Might Not Be Mine}}\\
\end{addmargin}
\end{absolutelynopagebreak}

\begin{absolutelynopagebreak}
\setstretch{.7}
{\PaliGlossA{sāvatthinidānaṃ.}}\\
\begin{addmargin}[1em]{2em}
\setstretch{.5}
{\PaliGlossB{At Sāvatthī.}}\\
\end{addmargin}
\end{absolutelynopagebreak}

\begin{absolutelynopagebreak}
\setstretch{.7}
{\PaliGlossA{“kismiṃ nu kho, bhikkhave, sati, kiṃ upādāya, kiṃ abhinivissa evaṃ diṭṭhi uppajjati:}}\\
\begin{addmargin}[1em]{2em}
\setstretch{.5}
{\PaliGlossB{“Mendicants, when what exists, because of grasping what and insisting on what, does the view arise:}}\\
\end{addmargin}
\end{absolutelynopagebreak}

\begin{absolutelynopagebreak}
\setstretch{.7}
{\PaliGlossA{‘no cassaṃ, no ca me siyā, nābhavissa, na me bhavissatī’”ti?}}\\
\begin{addmargin}[1em]{2em}
\setstretch{.5}
{\PaliGlossB{‘I might not be, and it might not be mine. I will not be, and it will not be mine’?”}}\\
\end{addmargin}
\end{absolutelynopagebreak}

\begin{absolutelynopagebreak}
\setstretch{.7}
{\PaliGlossA{“bhagavaṃmūlakā no, bhante, dhammā … pe ….}}\\
\begin{addmargin}[1em]{2em}
\setstretch{.5}
{\PaliGlossB{“Our teachings are rooted in the Buddha. …”}}\\
\end{addmargin}
\end{absolutelynopagebreak}

\begin{absolutelynopagebreak}
\setstretch{.7}
{\PaliGlossA{“rūpe kho, bhikkhave, sati, rūpaṃ upādāya, rūpaṃ abhinivissa evaṃ diṭṭhi uppajjati:}}\\
\begin{addmargin}[1em]{2em}
\setstretch{.5}
{\PaliGlossB{“When form exists, because of grasping form and insisting on form, the view arises:}}\\
\end{addmargin}
\end{absolutelynopagebreak}

\begin{absolutelynopagebreak}
\setstretch{.7}
{\PaliGlossA{‘no cassaṃ, no ca me siyā, nābhavissa, na me bhavissatī’ti.}}\\
\begin{addmargin}[1em]{2em}
\setstretch{.5}
{\PaliGlossB{‘It might not be, and it might not be mine. It will not be, and it will not be mine.’}}\\
\end{addmargin}
\end{absolutelynopagebreak}

\begin{absolutelynopagebreak}
\setstretch{.7}
{\PaliGlossA{vedanāya sati …}}\\
\begin{addmargin}[1em]{2em}
\setstretch{.5}
{\PaliGlossB{When feeling …}}\\
\end{addmargin}
\end{absolutelynopagebreak}

\begin{absolutelynopagebreak}
\setstretch{.7}
{\PaliGlossA{saññāya sati …}}\\
\begin{addmargin}[1em]{2em}
\setstretch{.5}
{\PaliGlossB{perception …}}\\
\end{addmargin}
\end{absolutelynopagebreak}

\begin{absolutelynopagebreak}
\setstretch{.7}
{\PaliGlossA{saṅkhāresu sati …}}\\
\begin{addmargin}[1em]{2em}
\setstretch{.5}
{\PaliGlossB{choices …}}\\
\end{addmargin}
\end{absolutelynopagebreak}

\begin{absolutelynopagebreak}
\setstretch{.7}
{\PaliGlossA{viññāṇe sati, viññāṇaṃ upādāya, viññāṇaṃ abhinivissa evaṃ diṭṭhi uppajjati:}}\\
\begin{addmargin}[1em]{2em}
\setstretch{.5}
{\PaliGlossB{consciousness exists, because of grasping consciousness and insisting on consciousness, the view arises:}}\\
\end{addmargin}
\end{absolutelynopagebreak}

\begin{absolutelynopagebreak}
\setstretch{.7}
{\PaliGlossA{‘no cassaṃ, no ca me siyā, nābhavissa, na me bhavissatī’ti.}}\\
\begin{addmargin}[1em]{2em}
\setstretch{.5}
{\PaliGlossB{‘It might not be, and it might not be mine. It will not be, and it will not be mine.’}}\\
\end{addmargin}
\end{absolutelynopagebreak}

\begin{absolutelynopagebreak}
\setstretch{.7}
{\PaliGlossA{taṃ kiṃ maññatha, bhikkhave,}}\\
\begin{addmargin}[1em]{2em}
\setstretch{.5}
{\PaliGlossB{What do you think, mendicants?}}\\
\end{addmargin}
\end{absolutelynopagebreak}

\begin{absolutelynopagebreak}
\setstretch{.7}
{\PaliGlossA{rūpaṃ niccaṃ vā aniccaṃ vā”ti?}}\\
\begin{addmargin}[1em]{2em}
\setstretch{.5}
{\PaliGlossB{Is form permanent or impermanent?”}}\\
\end{addmargin}
\end{absolutelynopagebreak}

\begin{absolutelynopagebreak}
\setstretch{.7}
{\PaliGlossA{“aniccaṃ, bhante” … pe …}}\\
\begin{addmargin}[1em]{2em}
\setstretch{.5}
{\PaliGlossB{“Impermanent, sir.” …}}\\
\end{addmargin}
\end{absolutelynopagebreak}

\begin{absolutelynopagebreak}
\setstretch{.7}
{\PaliGlossA{api nu taṃ anupādāya evaṃ diṭṭhi uppajjeyya:}}\\
\begin{addmargin}[1em]{2em}
\setstretch{.5}
{\PaliGlossB{    -}}\\
\end{addmargin}
\end{absolutelynopagebreak}

\begin{absolutelynopagebreak}
\setstretch{.7}
{\PaliGlossA{‘no cassaṃ, no ca me siyā, nābhavissa, na me bhavissatī’”ti?}}\\
\begin{addmargin}[1em]{2em}
\setstretch{.5}
{\PaliGlossB{    -}}\\
\end{addmargin}
\end{absolutelynopagebreak}

\begin{absolutelynopagebreak}
\setstretch{.7}
{\PaliGlossA{“no hetaṃ, bhante”.}}\\
\begin{addmargin}[1em]{2em}
\setstretch{.5}
{\PaliGlossB{    -}}\\
\end{addmargin}
\end{absolutelynopagebreak}

\begin{absolutelynopagebreak}
\setstretch{.7}
{\PaliGlossA{“vedanā …}}\\
\begin{addmargin}[1em]{2em}
\setstretch{.5}
{\PaliGlossB{“Is feeling …}}\\
\end{addmargin}
\end{absolutelynopagebreak}

\begin{absolutelynopagebreak}
\setstretch{.7}
{\PaliGlossA{saññā …}}\\
\begin{addmargin}[1em]{2em}
\setstretch{.5}
{\PaliGlossB{perception …}}\\
\end{addmargin}
\end{absolutelynopagebreak}

\begin{absolutelynopagebreak}
\setstretch{.7}
{\PaliGlossA{saṅkhārā …}}\\
\begin{addmargin}[1em]{2em}
\setstretch{.5}
{\PaliGlossB{choices …}}\\
\end{addmargin}
\end{absolutelynopagebreak}

\begin{absolutelynopagebreak}
\setstretch{.7}
{\PaliGlossA{viññāṇaṃ niccaṃ vā aniccaṃ vā”ti?}}\\
\begin{addmargin}[1em]{2em}
\setstretch{.5}
{\PaliGlossB{consciousness permanent or impermanent?”}}\\
\end{addmargin}
\end{absolutelynopagebreak}

\begin{absolutelynopagebreak}
\setstretch{.7}
{\PaliGlossA{“aniccaṃ, bhante … pe …}}\\
\begin{addmargin}[1em]{2em}
\setstretch{.5}
{\PaliGlossB{“Impermanent, sir.” …}}\\
\end{addmargin}
\end{absolutelynopagebreak}

\begin{absolutelynopagebreak}
\setstretch{.7}
{\PaliGlossA{api nu taṃ anupādāya evaṃ diṭṭhi uppajjeyya:}}\\
\begin{addmargin}[1em]{2em}
\setstretch{.5}
{\PaliGlossB{    -}}\\
\end{addmargin}
\end{absolutelynopagebreak}

\begin{absolutelynopagebreak}
\setstretch{.7}
{\PaliGlossA{‘no cassaṃ, no ca me siyā, nābhavissa, na me bhavissatī’”ti?}}\\
\begin{addmargin}[1em]{2em}
\setstretch{.5}
{\PaliGlossB{    -}}\\
\end{addmargin}
\end{absolutelynopagebreak}

\begin{absolutelynopagebreak}
\setstretch{.7}
{\PaliGlossA{“no hetaṃ, bhante”.}}\\
\begin{addmargin}[1em]{2em}
\setstretch{.5}
{\PaliGlossB{    -}}\\
\end{addmargin}
\end{absolutelynopagebreak}

\begin{absolutelynopagebreak}
\setstretch{.7}
{\PaliGlossA{“yampidaṃ diṭṭhaṃ sutaṃ mutaṃ viññātaṃ pattaṃ pariyesitaṃ anuvicaritaṃ manasā tampi niccaṃ vā aniccaṃ vā”ti?}}\\
\begin{addmargin}[1em]{2em}
\setstretch{.5}
{\PaliGlossB{“That which is seen, heard, thought, known, sought, and explored by the mind: is that permanent or impermanent?”}}\\
\end{addmargin}
\end{absolutelynopagebreak}

\begin{absolutelynopagebreak}
\setstretch{.7}
{\PaliGlossA{“aniccaṃ, bhante … pe …}}\\
\begin{addmargin}[1em]{2em}
\setstretch{.5}
{\PaliGlossB{“Impermanent, sir.” …}}\\
\end{addmargin}
\end{absolutelynopagebreak}

\begin{absolutelynopagebreak}
\setstretch{.7}
{\PaliGlossA{api nu taṃ anupādāya evaṃ diṭṭhi uppajjeyya:}}\\
\begin{addmargin}[1em]{2em}
\setstretch{.5}
{\PaliGlossB{“But by not grasping what’s impermanent, suffering, and perishable, would such a view arise?”}}\\
\end{addmargin}
\end{absolutelynopagebreak}

\begin{absolutelynopagebreak}
\setstretch{.7}
{\PaliGlossA{‘no cassaṃ, no ca me siyā, nābhavissa, na me bhavissatī’”ti?}}\\
\begin{addmargin}[1em]{2em}
\setstretch{.5}
{\PaliGlossB{    -}}\\
\end{addmargin}
\end{absolutelynopagebreak}

\begin{absolutelynopagebreak}
\setstretch{.7}
{\PaliGlossA{“no hetaṃ, bhante”.}}\\
\begin{addmargin}[1em]{2em}
\setstretch{.5}
{\PaliGlossB{“No, sir.”}}\\
\end{addmargin}
\end{absolutelynopagebreak}

\begin{absolutelynopagebreak}
\setstretch{.7}
{\PaliGlossA{“yato kho, bhikkhave, ariyasāvakassa imesu ca ṭhānesu kaṅkhā pahīnā hoti, dukkhepissa kaṅkhā pahīnā hoti … pe … dukkhanirodhagāminiyā paṭipadāyapissa kaṅkhā pahīnā hoti—}}\\
\begin{addmargin}[1em]{2em}
\setstretch{.5}
{\PaliGlossB{“When a noble disciple has given up doubt in these six cases, and has given up doubt in suffering, its origin, its cessation, and the practice that leads to its cessation,}}\\
\end{addmargin}
\end{absolutelynopagebreak}

\begin{absolutelynopagebreak}
\setstretch{.7}
{\PaliGlossA{ayaṃ vuccati, bhikkhave, ariyasāvako sotāpanno avinipātadhammo niyato sambodhiparāyano”ti.}}\\
\begin{addmargin}[1em]{2em}
\setstretch{.5}
{\PaliGlossB{they’re called a noble disciple who is a stream-enterer, not liable to be reborn in the underworld, bound for awakening.”}}\\
\end{addmargin}
\end{absolutelynopagebreak}

\begin{absolutelynopagebreak}
\setstretch{.7}
{\PaliGlossA{catutthaṃ.}}\\
\begin{addmargin}[1em]{2em}
\setstretch{.5}
{\PaliGlossB{    -}}\\
\end{addmargin}
\end{absolutelynopagebreak}
