
\begin{absolutelynopagebreak}
\setstretch{.7}
{\PaliGlossA{saṃyutta nikāya 35}}\\
\begin{addmargin}[1em]{2em}
\setstretch{.5}
{\PaliGlossB{Linked Discourses 35}}\\
\end{addmargin}
\end{absolutelynopagebreak}

\begin{absolutelynopagebreak}
\setstretch{.7}
{\PaliGlossA{8. gilānavagga}}\\
\begin{addmargin}[1em]{2em}
\setstretch{.5}
{\PaliGlossB{8. Sick}}\\
\end{addmargin}
\end{absolutelynopagebreak}

\begin{absolutelynopagebreak}
\setstretch{.7}
{\PaliGlossA{79. paṭhamaavijjāpahānasutta}}\\
\begin{addmargin}[1em]{2em}
\setstretch{.5}
{\PaliGlossB{79. Giving Up Ignorance (1st)}}\\
\end{addmargin}
\end{absolutelynopagebreak}

\begin{absolutelynopagebreak}
\setstretch{.7}
{\PaliGlossA{atha kho aññataro bhikkhu yena bhagavā tenupasaṅkami … pe …}}\\
\begin{addmargin}[1em]{2em}
\setstretch{.5}
{\PaliGlossB{Then a mendicant went up to the Buddha …}}\\
\end{addmargin}
\end{absolutelynopagebreak}

\begin{absolutelynopagebreak}
\setstretch{.7}
{\PaliGlossA{ekamantaṃ nisinno kho so bhikkhu bhagavantaṃ etadavoca:}}\\
\begin{addmargin}[1em]{2em}
\setstretch{.5}
{\PaliGlossB{and said to him:}}\\
\end{addmargin}
\end{absolutelynopagebreak}

\begin{absolutelynopagebreak}
\setstretch{.7}
{\PaliGlossA{“atthi nu kho, bhante, eko dhammo yassa pahānā bhikkhuno avijjā pahīyati, vijjā uppajjatī”ti?}}\\
\begin{addmargin}[1em]{2em}
\setstretch{.5}
{\PaliGlossB{“Sir, is there one thing such that by giving it up a mendicant gives up ignorance and gives rise to knowledge?”}}\\
\end{addmargin}
\end{absolutelynopagebreak}

\begin{absolutelynopagebreak}
\setstretch{.7}
{\PaliGlossA{“atthi kho, bhikkhu, eko dhammo yassa pahānā bhikkhuno avijjā pahīyati, vijjā uppajjatī”ti.}}\\
\begin{addmargin}[1em]{2em}
\setstretch{.5}
{\PaliGlossB{“There is, mendicant.”}}\\
\end{addmargin}
\end{absolutelynopagebreak}

\begin{absolutelynopagebreak}
\setstretch{.7}
{\PaliGlossA{“katamo pana, bhante, eko dhammo yassa pahānā bhikkhuno avijjā pahīyati, vijjā uppajjatī”ti?}}\\
\begin{addmargin}[1em]{2em}
\setstretch{.5}
{\PaliGlossB{“But what is that one thing?”}}\\
\end{addmargin}
\end{absolutelynopagebreak}

\begin{absolutelynopagebreak}
\setstretch{.7}
{\PaliGlossA{“avijjā kho, bhikkhu, eko dhammo yassa pahānā bhikkhuno avijjā pahīyati, vijjā uppajjatī”ti.}}\\
\begin{addmargin}[1em]{2em}
\setstretch{.5}
{\PaliGlossB{“Ignorance is one thing such that by giving it up a mendicant gives up ignorance and gives rise to knowledge.”}}\\
\end{addmargin}
\end{absolutelynopagebreak}

\begin{absolutelynopagebreak}
\setstretch{.7}
{\PaliGlossA{“kathaṃ pana, bhante, jānato, kathaṃ passato bhikkhuno avijjā pahīyati, vijjā uppajjatī”ti?}}\\
\begin{addmargin}[1em]{2em}
\setstretch{.5}
{\PaliGlossB{“But how does a mendicant know and see so as to give up ignorance and give rise to knowledge?”}}\\
\end{addmargin}
\end{absolutelynopagebreak}

\begin{absolutelynopagebreak}
\setstretch{.7}
{\PaliGlossA{“cakkhuṃ kho, bhikkhu, aniccato jānato passato bhikkhuno avijjā pahīyati, vijjā uppajjati. rūpe … cakkhuviññāṇaṃ … cakkhusamphassaṃ …}}\\
\begin{addmargin}[1em]{2em}
\setstretch{.5}
{\PaliGlossB{“When a mendicant knows and sees the eye, sights, eye consciousness, and eye contact as impermanent, ignorance is given up and knowledge arises.}}\\
\end{addmargin}
\end{absolutelynopagebreak}

\begin{absolutelynopagebreak}
\setstretch{.7}
{\PaliGlossA{yampidaṃ, cakkhusamphassapaccayā uppajjati vedayitaṃ sukhaṃ vā dukkhaṃ vā adukkhamasukhaṃ vā tampi aniccato jānato passato bhikkhuno avijjā pahīyati, vijjā uppajjati … pe …}}\\
\begin{addmargin}[1em]{2em}
\setstretch{.5}
{\PaliGlossB{And also knowing and seeing the pleasant, painful, or neutral feeling that arises conditioned by eye contact as impermanent, ignorance is given up and knowledge arises. …}}\\
\end{addmargin}
\end{absolutelynopagebreak}

\begin{absolutelynopagebreak}
\setstretch{.7}
{\PaliGlossA{manaṃ aniccato jānato passato bhikkhuno avijjā pahīyati, vijjā uppajjati. dhamme … manoviññāṇaṃ … manosamphassaṃ …}}\\
\begin{addmargin}[1em]{2em}
\setstretch{.5}
{\PaliGlossB{Knowing and seeing the mind, thoughts, mind consciousness, and mind contact as impermanent, ignorance is given up and knowledge arises.}}\\
\end{addmargin}
\end{absolutelynopagebreak}

\begin{absolutelynopagebreak}
\setstretch{.7}
{\PaliGlossA{yampidaṃ manosamphassapaccayā uppajjati vedayitaṃ sukhaṃ vā dukkhaṃ vā adukkhamasukhaṃ vā tampi aniccato jānato passato bhikkhuno avijjā pahīyati, vijjā uppajjati.}}\\
\begin{addmargin}[1em]{2em}
\setstretch{.5}
{\PaliGlossB{And also knowing and seeing the pleasant, painful, or neutral feeling that arises conditioned by mind contact as impermanent, ignorance is given up and knowledge arises.}}\\
\end{addmargin}
\end{absolutelynopagebreak}

\begin{absolutelynopagebreak}
\setstretch{.7}
{\PaliGlossA{evaṃ kho, bhikkhu, jānato evaṃ passato bhikkhuno avijjā pahīyati, vijjā uppajjatī”ti.}}\\
\begin{addmargin}[1em]{2em}
\setstretch{.5}
{\PaliGlossB{That’s how a mendicant knows and sees so as to give up ignorance and give rise to knowledge.”}}\\
\end{addmargin}
\end{absolutelynopagebreak}

\begin{absolutelynopagebreak}
\setstretch{.7}
{\PaliGlossA{chaṭṭhaṃ.}}\\
\begin{addmargin}[1em]{2em}
\setstretch{.5}
{\PaliGlossB{    -}}\\
\end{addmargin}
\end{absolutelynopagebreak}
