
\begin{absolutelynopagebreak}
\setstretch{.7}
{\PaliGlossA{saṃyutta nikāya 12}}\\
\begin{addmargin}[1em]{2em}
\setstretch{.5}
{\PaliGlossB{Linked Discourses 12}}\\
\end{addmargin}
\end{absolutelynopagebreak}

\begin{absolutelynopagebreak}
\setstretch{.7}
{\PaliGlossA{7. mahāvagga}}\\
\begin{addmargin}[1em]{2em}
\setstretch{.5}
{\PaliGlossB{7. The Great Chapter}}\\
\end{addmargin}
\end{absolutelynopagebreak}

\begin{absolutelynopagebreak}
\setstretch{.7}
{\PaliGlossA{67. naḷakalāpīsutta}}\\
\begin{addmargin}[1em]{2em}
\setstretch{.5}
{\PaliGlossB{67. Bundles of Reeds}}\\
\end{addmargin}
\end{absolutelynopagebreak}

\begin{absolutelynopagebreak}
\setstretch{.7}
{\PaliGlossA{ekaṃ samayaṃ āyasmā ca sāriputto āyasmā ca mahākoṭṭhiko bārāṇasiyaṃ viharanti isipatane migadāye.}}\\
\begin{addmargin}[1em]{2em}
\setstretch{.5}
{\PaliGlossB{At one time Venerable Sāriputta and Venerable Mahākoṭṭhita were staying near Benares, in the deer park at Isipatana.}}\\
\end{addmargin}
\end{absolutelynopagebreak}

\begin{absolutelynopagebreak}
\setstretch{.7}
{\PaliGlossA{atha kho āyasmā mahākoṭṭhiko sāyanhasamayaṃ paṭisallānā vuṭṭhito yenāyasmā sāriputto tenupasaṅkami; upasaṅkamitvā āyasmatā sāriputtena saddhiṃ sammodi.}}\\
\begin{addmargin}[1em]{2em}
\setstretch{.5}
{\PaliGlossB{Then in the late afternoon, Venerable Mahākoṭṭhita came out of retreat, went to Venerable Sāriputta, and exchanged greetings with him.}}\\
\end{addmargin}
\end{absolutelynopagebreak}

\begin{absolutelynopagebreak}
\setstretch{.7}
{\PaliGlossA{sammodanīyaṃ kathaṃ sāraṇīyaṃ vītisāretvā ekamantaṃ nisīdi. ekamantaṃ nisinno kho āyasmā mahākoṭṭhiko āyasmantaṃ sāriputtaṃ etadavoca:}}\\
\begin{addmargin}[1em]{2em}
\setstretch{.5}
{\PaliGlossB{When the greetings and polite conversation were over, he sat down to one side and said to Sāriputta:}}\\
\end{addmargin}
\end{absolutelynopagebreak}

\begin{absolutelynopagebreak}
\setstretch{.7}
{\PaliGlossA{“kiṃ nu kho, āvuso sāriputta, sayaṃkataṃ jarāmaraṇaṃ, paraṃkataṃ jarāmaraṇaṃ, sayaṃkatañca paraṃkatañca jarāmaraṇaṃ, udāhu asayaṃkāraṃ aparaṅkāraṃ adhiccasamuppannaṃ jarāmaraṇan”ti?}}\\
\begin{addmargin}[1em]{2em}
\setstretch{.5}
{\PaliGlossB{“Well, Reverend Sāriputta, are old age and death made by oneself? Or by another? Or by both oneself and another? Or do they arise by chance, not made by oneself or another?”}}\\
\end{addmargin}
\end{absolutelynopagebreak}

\begin{absolutelynopagebreak}
\setstretch{.7}
{\PaliGlossA{“na kho, āvuso koṭṭhika, sayaṃkataṃ jarāmaraṇaṃ, na paraṃkataṃ jarāmaraṇaṃ, na sayaṃkatañca paraṃkatañca jarāmaraṇaṃ, nāpi asayaṃkāraṃ aparaṅkāraṃ adhiccasamuppannaṃ jarāmaraṇaṃ.}}\\
\begin{addmargin}[1em]{2em}
\setstretch{.5}
{\PaliGlossB{“No, Reverend Koṭṭhita, old age and death are not made by oneself, nor by another, nor by both oneself and another, nor do they arise by chance, not made by oneself or another.}}\\
\end{addmargin}
\end{absolutelynopagebreak}

\begin{absolutelynopagebreak}
\setstretch{.7}
{\PaliGlossA{api ca jātipaccayā jarāmaraṇan”ti.}}\\
\begin{addmargin}[1em]{2em}
\setstretch{.5}
{\PaliGlossB{Rather, rebirth is a condition for old age and death.”}}\\
\end{addmargin}
\end{absolutelynopagebreak}

\begin{absolutelynopagebreak}
\setstretch{.7}
{\PaliGlossA{“kiṃ nu kho, āvuso sāriputta, sayaṅkatā jāti, paraṅkatā jāti, sayaṅkatā ca paraṅkatā ca jāti, udāhu asayaṅkārā aparaṅkārā adhiccasamuppannā jātī”ti?}}\\
\begin{addmargin}[1em]{2em}
\setstretch{.5}
{\PaliGlossB{“Well, Reverend Sāriputta, is rebirth made by oneself? Or by another? Or by both oneself and another? Or does it arise by chance, not made by oneself or another?”}}\\
\end{addmargin}
\end{absolutelynopagebreak}

\begin{absolutelynopagebreak}
\setstretch{.7}
{\PaliGlossA{“na kho, āvuso koṭṭhika, sayaṅkatā jāti, na paraṅkatā jāti, na sayaṅkatā ca paraṅkatā ca jāti, nāpi asayaṅkārā aparaṅkārā adhiccasamuppannā jāti.}}\\
\begin{addmargin}[1em]{2em}
\setstretch{.5}
{\PaliGlossB{“No, Reverend Koṭṭhita, rebirth is not made by oneself, nor by another, nor by both oneself and another, nor does it arise by chance, not made by oneself or another.}}\\
\end{addmargin}
\end{absolutelynopagebreak}

\begin{absolutelynopagebreak}
\setstretch{.7}
{\PaliGlossA{api ca bhavapaccayā jātī”ti.}}\\
\begin{addmargin}[1em]{2em}
\setstretch{.5}
{\PaliGlossB{Rather, continued existence is a condition for rebirth.”}}\\
\end{addmargin}
\end{absolutelynopagebreak}

\begin{absolutelynopagebreak}
\setstretch{.7}
{\PaliGlossA{“kiṃ nu kho, āvuso sāriputta, sayaṅkato bhavo … pe …}}\\
\begin{addmargin}[1em]{2em}
\setstretch{.5}
{\PaliGlossB{“Well, Reverend Sāriputta, is continued existence made by oneself? …” …}}\\
\end{addmargin}
\end{absolutelynopagebreak}

\begin{absolutelynopagebreak}
\setstretch{.7}
{\PaliGlossA{sayaṅkataṃ upādānaṃ …}}\\
\begin{addmargin}[1em]{2em}
\setstretch{.5}
{\PaliGlossB{“Is grasping made by oneself? …” …}}\\
\end{addmargin}
\end{absolutelynopagebreak}

\begin{absolutelynopagebreak}
\setstretch{.7}
{\PaliGlossA{sayaṅkatā taṇhā …}}\\
\begin{addmargin}[1em]{2em}
\setstretch{.5}
{\PaliGlossB{“Is craving made by oneself? …” …}}\\
\end{addmargin}
\end{absolutelynopagebreak}

\begin{absolutelynopagebreak}
\setstretch{.7}
{\PaliGlossA{sayaṅkatā vedanā …}}\\
\begin{addmargin}[1em]{2em}
\setstretch{.5}
{\PaliGlossB{“Is feeling made by oneself? …” …}}\\
\end{addmargin}
\end{absolutelynopagebreak}

\begin{absolutelynopagebreak}
\setstretch{.7}
{\PaliGlossA{sayaṅkato phasso …}}\\
\begin{addmargin}[1em]{2em}
\setstretch{.5}
{\PaliGlossB{“Is contact made by oneself? …” …}}\\
\end{addmargin}
\end{absolutelynopagebreak}

\begin{absolutelynopagebreak}
\setstretch{.7}
{\PaliGlossA{sayaṅkataṃ saḷāyatanaṃ …}}\\
\begin{addmargin}[1em]{2em}
\setstretch{.5}
{\PaliGlossB{“Are the six sense fields made by oneself? …” …}}\\
\end{addmargin}
\end{absolutelynopagebreak}

\begin{absolutelynopagebreak}
\setstretch{.7}
{\PaliGlossA{sayaṅkataṃ nāmarūpaṃ, paraṅkataṃ nāmarūpaṃ, sayaṅkatañca paraṅkatañca nāmarūpaṃ, udāhu asayaṅkāraṃ aparaṅkāraṃ adhiccasamuppannaṃ nāmarūpan”ti?}}\\
\begin{addmargin}[1em]{2em}
\setstretch{.5}
{\PaliGlossB{“Well, Reverend Sāriputta, are name and form made by oneself? Or by another? Or by both oneself and another? Or do they arise by chance, not made by oneself or another?”}}\\
\end{addmargin}
\end{absolutelynopagebreak}

\begin{absolutelynopagebreak}
\setstretch{.7}
{\PaliGlossA{“na kho, āvuso koṭṭhika, sayaṅkataṃ nāmarūpaṃ, na paraṅkataṃ nāmarūpaṃ, na sayaṅkatañca paraṅkatañca nāmarūpaṃ, nāpi asayaṅkāraṃ aparaṅkāraṃ, adhiccasamuppannaṃ nāmarūpaṃ.}}\\
\begin{addmargin}[1em]{2em}
\setstretch{.5}
{\PaliGlossB{“No, Reverend Koṭṭhita, name and form are not made by oneself, nor by another, nor by both oneself and another, nor do they arise by chance, not made by oneself or another.}}\\
\end{addmargin}
\end{absolutelynopagebreak}

\begin{absolutelynopagebreak}
\setstretch{.7}
{\PaliGlossA{api ca viññāṇapaccayā nāmarūpan”ti.}}\\
\begin{addmargin}[1em]{2em}
\setstretch{.5}
{\PaliGlossB{Rather, consciousness is a condition for name and form.”}}\\
\end{addmargin}
\end{absolutelynopagebreak}

\begin{absolutelynopagebreak}
\setstretch{.7}
{\PaliGlossA{“kiṃ nu kho, āvuso sāriputta, sayaṅkataṃ viññāṇaṃ, paraṅkataṃ viññāṇaṃ, sayaṅkatañca paraṅkatañca viññāṇaṃ, udāhu asayaṅkāraṃ aparaṅkāraṃ adhiccasamuppannaṃ viññāṇan”ti?}}\\
\begin{addmargin}[1em]{2em}
\setstretch{.5}
{\PaliGlossB{“Well, Reverend Sāriputta, is consciousness made by oneself? Or by another? Or by both oneself and another? Or does it arise by chance, not made by oneself or another?”}}\\
\end{addmargin}
\end{absolutelynopagebreak}

\begin{absolutelynopagebreak}
\setstretch{.7}
{\PaliGlossA{“na kho, āvuso koṭṭhika, sayaṅkataṃ viññāṇaṃ, na paraṅkataṃ viññāṇaṃ, na sayaṅkatañca paraṅkatañca viññāṇaṃ, nāpi asayaṅkāraṃ aparaṅkāraṃ adhiccasamuppannaṃ viññāṇaṃ.}}\\
\begin{addmargin}[1em]{2em}
\setstretch{.5}
{\PaliGlossB{“No, Reverend Koṭṭhita, consciousness is not made by oneself, nor by another, nor by both oneself and another, nor does it arise by chance, not made by oneself or another.}}\\
\end{addmargin}
\end{absolutelynopagebreak}

\begin{absolutelynopagebreak}
\setstretch{.7}
{\PaliGlossA{api ca nāmarūpapaccayā viññāṇan”ti.}}\\
\begin{addmargin}[1em]{2em}
\setstretch{.5}
{\PaliGlossB{Rather, name and form are conditions for consciousness.”}}\\
\end{addmargin}
\end{absolutelynopagebreak}

\begin{absolutelynopagebreak}
\setstretch{.7}
{\PaliGlossA{“idāneva kho mayaṃ āyasmato sāriputtassa bhāsitaṃ evaṃ ājānāma:}}\\
\begin{addmargin}[1em]{2em}
\setstretch{.5}
{\PaliGlossB{“Just now I understood you to say:}}\\
\end{addmargin}
\end{absolutelynopagebreak}

\begin{absolutelynopagebreak}
\setstretch{.7}
{\PaliGlossA{‘na khvāvuso koṭṭhika, sayaṅkataṃ nāmarūpaṃ, na paraṅkataṃ nāmarūpaṃ, na sayaṅkatañca paraṅkatañca nāmarūpaṃ, nāpi asayaṅkāraṃ aparaṅkāraṃ adhiccasamuppannaṃ nāmarūpaṃ.}}\\
\begin{addmargin}[1em]{2em}
\setstretch{.5}
{\PaliGlossB{‘No, Reverend Koṭṭhita, name and form are not made by oneself, nor by another, nor by both oneself and another, nor do they arise by chance, not made by oneself or another.}}\\
\end{addmargin}
\end{absolutelynopagebreak}

\begin{absolutelynopagebreak}
\setstretch{.7}
{\PaliGlossA{api ca viññāṇapaccayā nāmarūpan’ti.}}\\
\begin{addmargin}[1em]{2em}
\setstretch{.5}
{\PaliGlossB{Rather, consciousness is a condition for name and form.’}}\\
\end{addmargin}
\end{absolutelynopagebreak}

\begin{absolutelynopagebreak}
\setstretch{.7}
{\PaliGlossA{idāneva ca pana mayaṃ āyasmato sāriputtassa bhāsitaṃ evaṃ ājānāma:}}\\
\begin{addmargin}[1em]{2em}
\setstretch{.5}
{\PaliGlossB{But I also understood you to say:}}\\
\end{addmargin}
\end{absolutelynopagebreak}

\begin{absolutelynopagebreak}
\setstretch{.7}
{\PaliGlossA{‘na khvāvuso koṭṭhika, sayaṅkataṃ viññāṇaṃ, na paraṅkataṃ viññāṇaṃ, na sayaṅkatañca paraṅkatañca viññāṇaṃ, nāpi asayaṅkāraṃ aparaṅkāraṃ adhiccasamuppannaṃ viññāṇaṃ.}}\\
\begin{addmargin}[1em]{2em}
\setstretch{.5}
{\PaliGlossB{‘No, Reverend Koṭṭhita, consciousness is not made by oneself, nor by another, nor by both oneself and another, nor does it arise by chance, not made by oneself or another.}}\\
\end{addmargin}
\end{absolutelynopagebreak}

\begin{absolutelynopagebreak}
\setstretch{.7}
{\PaliGlossA{api ca nāmarūpapaccayā viññāṇan’ti.}}\\
\begin{addmargin}[1em]{2em}
\setstretch{.5}
{\PaliGlossB{Rather, name and form are conditions for consciousness.’}}\\
\end{addmargin}
\end{absolutelynopagebreak}

\begin{absolutelynopagebreak}
\setstretch{.7}
{\PaliGlossA{yathā kathaṃ panāvuso sāriputta, imassa bhāsitassa attho daṭṭhabbo”ti?}}\\
\begin{addmargin}[1em]{2em}
\setstretch{.5}
{\PaliGlossB{How then should we see the meaning of this statement?”}}\\
\end{addmargin}
\end{absolutelynopagebreak}

\begin{absolutelynopagebreak}
\setstretch{.7}
{\PaliGlossA{“tenahāvuso, upamaṃ te karissāmi.}}\\
\begin{addmargin}[1em]{2em}
\setstretch{.5}
{\PaliGlossB{“Well then, reverend, I shall give you a simile.}}\\
\end{addmargin}
\end{absolutelynopagebreak}

\begin{absolutelynopagebreak}
\setstretch{.7}
{\PaliGlossA{upamāyapidhekacce viññū purisā bhāsitassa atthaṃ jānanti.}}\\
\begin{addmargin}[1em]{2em}
\setstretch{.5}
{\PaliGlossB{For by means of a simile some sensible people understand the meaning of what is said.}}\\
\end{addmargin}
\end{absolutelynopagebreak}

\begin{absolutelynopagebreak}
\setstretch{.7}
{\PaliGlossA{seyyathāpi, āvuso, dve naḷakalāpiyo aññamaññaṃ nissāya tiṭṭheyyuṃ.}}\\
\begin{addmargin}[1em]{2em}
\setstretch{.5}
{\PaliGlossB{Suppose there were two bundles of reeds leaning up against each other.}}\\
\end{addmargin}
\end{absolutelynopagebreak}

\begin{absolutelynopagebreak}
\setstretch{.7}
{\PaliGlossA{evameva kho, āvuso, nāmarūpapaccayā viññāṇaṃ;}}\\
\begin{addmargin}[1em]{2em}
\setstretch{.5}
{\PaliGlossB{In the same way, name and form are conditions for consciousness.}}\\
\end{addmargin}
\end{absolutelynopagebreak}

\begin{absolutelynopagebreak}
\setstretch{.7}
{\PaliGlossA{viññāṇapaccayā nāmarūpaṃ;}}\\
\begin{addmargin}[1em]{2em}
\setstretch{.5}
{\PaliGlossB{Consciousness is a condition for name and form.}}\\
\end{addmargin}
\end{absolutelynopagebreak}

\begin{absolutelynopagebreak}
\setstretch{.7}
{\PaliGlossA{nāmarūpapaccayā saḷāyatanaṃ;}}\\
\begin{addmargin}[1em]{2em}
\setstretch{.5}
{\PaliGlossB{Name and form are conditions for the six sense fields.}}\\
\end{addmargin}
\end{absolutelynopagebreak}

\begin{absolutelynopagebreak}
\setstretch{.7}
{\PaliGlossA{saḷāyatanapaccayā phasso … pe …}}\\
\begin{addmargin}[1em]{2em}
\setstretch{.5}
{\PaliGlossB{The six sense fields are conditions for contact. …}}\\
\end{addmargin}
\end{absolutelynopagebreak}

\begin{absolutelynopagebreak}
\setstretch{.7}
{\PaliGlossA{evametassa kevalassa dukkhakkhandhassa samudayo hoti.}}\\
\begin{addmargin}[1em]{2em}
\setstretch{.5}
{\PaliGlossB{That is how this entire mass of suffering originates.}}\\
\end{addmargin}
\end{absolutelynopagebreak}

\begin{absolutelynopagebreak}
\setstretch{.7}
{\PaliGlossA{tāsañce, āvuso, naḷakalāpīnaṃ ekaṃ ākaḍḍheyya, ekā papateyya;}}\\
\begin{addmargin}[1em]{2em}
\setstretch{.5}
{\PaliGlossB{If the first of those bundles of reeds were to be pulled away, the other would collapse.}}\\
\end{addmargin}
\end{absolutelynopagebreak}

\begin{absolutelynopagebreak}
\setstretch{.7}
{\PaliGlossA{aparañce ākaḍḍheyya, aparā papateyya.}}\\
\begin{addmargin}[1em]{2em}
\setstretch{.5}
{\PaliGlossB{And if the other were to be pulled away, the first would collapse.}}\\
\end{addmargin}
\end{absolutelynopagebreak}

\begin{absolutelynopagebreak}
\setstretch{.7}
{\PaliGlossA{evameva kho, āvuso, nāmarūpanirodhā viññāṇanirodho;}}\\
\begin{addmargin}[1em]{2em}
\setstretch{.5}
{\PaliGlossB{In the same way, when name and form cease, consciousness ceases.}}\\
\end{addmargin}
\end{absolutelynopagebreak}

\begin{absolutelynopagebreak}
\setstretch{.7}
{\PaliGlossA{viññāṇanirodhā nāmarūpanirodho;}}\\
\begin{addmargin}[1em]{2em}
\setstretch{.5}
{\PaliGlossB{When consciousness ceases, name and form cease.}}\\
\end{addmargin}
\end{absolutelynopagebreak}

\begin{absolutelynopagebreak}
\setstretch{.7}
{\PaliGlossA{nāmarūpanirodhā saḷāyatananirodho;}}\\
\begin{addmargin}[1em]{2em}
\setstretch{.5}
{\PaliGlossB{When name and form cease, the six sense fields cease.}}\\
\end{addmargin}
\end{absolutelynopagebreak}

\begin{absolutelynopagebreak}
\setstretch{.7}
{\PaliGlossA{saḷāyatananirodhā phassanirodho … pe …}}\\
\begin{addmargin}[1em]{2em}
\setstretch{.5}
{\PaliGlossB{When the six sense fields cease, contact ceases. …}}\\
\end{addmargin}
\end{absolutelynopagebreak}

\begin{absolutelynopagebreak}
\setstretch{.7}
{\PaliGlossA{evametassa kevalassa dukkhakkhandhassa nirodho hotī”ti.}}\\
\begin{addmargin}[1em]{2em}
\setstretch{.5}
{\PaliGlossB{That is how this entire mass of suffering ceases.”}}\\
\end{addmargin}
\end{absolutelynopagebreak}

\begin{absolutelynopagebreak}
\setstretch{.7}
{\PaliGlossA{“acchariyaṃ, āvuso sāriputta;}}\\
\begin{addmargin}[1em]{2em}
\setstretch{.5}
{\PaliGlossB{“It’s incredible, Reverend Sāriputta, it’s amazing!}}\\
\end{addmargin}
\end{absolutelynopagebreak}

\begin{absolutelynopagebreak}
\setstretch{.7}
{\PaliGlossA{abbhutaṃ, āvuso sāriputta.}}\\
\begin{addmargin}[1em]{2em}
\setstretch{.5}
{\PaliGlossB{    -}}\\
\end{addmargin}
\end{absolutelynopagebreak}

\begin{absolutelynopagebreak}
\setstretch{.7}
{\PaliGlossA{yāvasubhāsitañcidaṃ āyasmatā sāriputtena.}}\\
\begin{addmargin}[1em]{2em}
\setstretch{.5}
{\PaliGlossB{How well spoken this was by Venerable Sāriputta!}}\\
\end{addmargin}
\end{absolutelynopagebreak}

\begin{absolutelynopagebreak}
\setstretch{.7}
{\PaliGlossA{idañca pana mayaṃ āyasmato sāriputtassa bhāsitaṃ imehi chattiṃsāya vatthūhi anumodāma:}}\\
\begin{addmargin}[1em]{2em}
\setstretch{.5}
{\PaliGlossB{And we can express our agreement with Venerable Sāriputta’s statement on these thirty-six grounds.}}\\
\end{addmargin}
\end{absolutelynopagebreak}

\begin{absolutelynopagebreak}
\setstretch{.7}
{\PaliGlossA{‘jarāmaraṇassa ce, āvuso, bhikkhu nibbidāya virāgāya nirodhāya dhammaṃ deseti, dhammakathiko bhikkhūti alaṃvacanāya.}}\\
\begin{addmargin}[1em]{2em}
\setstretch{.5}
{\PaliGlossB{If a mendicant teaches Dhamma for disillusionment, dispassion, and cessation regarding old age and death, they’re qualified to be called a ‘mendicant who speaks on Dhamma’.}}\\
\end{addmargin}
\end{absolutelynopagebreak}

\begin{absolutelynopagebreak}
\setstretch{.7}
{\PaliGlossA{jarāmaraṇassa ce, āvuso, bhikkhu nibbidāya virāgāya nirodhāya paṭipanno hoti, dhammānudhammappaṭipanno bhikkhūti alaṃvacanāya.}}\\
\begin{addmargin}[1em]{2em}
\setstretch{.5}
{\PaliGlossB{If they practice for disillusionment, dispassion, and cessation regarding old age and death, they’re qualified to be called a ‘mendicant who practices in line with the teaching’.}}\\
\end{addmargin}
\end{absolutelynopagebreak}

\begin{absolutelynopagebreak}
\setstretch{.7}
{\PaliGlossA{jarāmaraṇassa ce, āvuso, bhikkhu nibbidā virāgā nirodhā anupādā vimutto hoti, diṭṭhadhammanibbānappatto bhikkhūti alaṃvacanāya.}}\\
\begin{addmargin}[1em]{2em}
\setstretch{.5}
{\PaliGlossB{If they’re freed by not grasping by disillusionment, dispassion, and cessation regarding old age and death, they’re qualified to be called a ‘mendicant who has attained extinguishment in this very life’.}}\\
\end{addmargin}
\end{absolutelynopagebreak}

\begin{absolutelynopagebreak}
\setstretch{.7}
{\PaliGlossA{jātiyā ce …}}\\
\begin{addmargin}[1em]{2em}
\setstretch{.5}
{\PaliGlossB{If a mendicant teaches Dhamma for disillusionment regarding rebirth …}}\\
\end{addmargin}
\end{absolutelynopagebreak}

\begin{absolutelynopagebreak}
\setstretch{.7}
{\PaliGlossA{bhavassa ce …}}\\
\begin{addmargin}[1em]{2em}
\setstretch{.5}
{\PaliGlossB{continued existence …}}\\
\end{addmargin}
\end{absolutelynopagebreak}

\begin{absolutelynopagebreak}
\setstretch{.7}
{\PaliGlossA{upādānassa ce …}}\\
\begin{addmargin}[1em]{2em}
\setstretch{.5}
{\PaliGlossB{grasping …}}\\
\end{addmargin}
\end{absolutelynopagebreak}

\begin{absolutelynopagebreak}
\setstretch{.7}
{\PaliGlossA{taṇhāya ce …}}\\
\begin{addmargin}[1em]{2em}
\setstretch{.5}
{\PaliGlossB{craving …}}\\
\end{addmargin}
\end{absolutelynopagebreak}

\begin{absolutelynopagebreak}
\setstretch{.7}
{\PaliGlossA{vedanāya ce …}}\\
\begin{addmargin}[1em]{2em}
\setstretch{.5}
{\PaliGlossB{feeling …}}\\
\end{addmargin}
\end{absolutelynopagebreak}

\begin{absolutelynopagebreak}
\setstretch{.7}
{\PaliGlossA{phassassa ce …}}\\
\begin{addmargin}[1em]{2em}
\setstretch{.5}
{\PaliGlossB{contact …}}\\
\end{addmargin}
\end{absolutelynopagebreak}

\begin{absolutelynopagebreak}
\setstretch{.7}
{\PaliGlossA{saḷāyatanassa ce …}}\\
\begin{addmargin}[1em]{2em}
\setstretch{.5}
{\PaliGlossB{the six sense fields …}}\\
\end{addmargin}
\end{absolutelynopagebreak}

\begin{absolutelynopagebreak}
\setstretch{.7}
{\PaliGlossA{nāmarūpassa ce …}}\\
\begin{addmargin}[1em]{2em}
\setstretch{.5}
{\PaliGlossB{name and form …}}\\
\end{addmargin}
\end{absolutelynopagebreak}

\begin{absolutelynopagebreak}
\setstretch{.7}
{\PaliGlossA{viññāṇassa ce …}}\\
\begin{addmargin}[1em]{2em}
\setstretch{.5}
{\PaliGlossB{consciousness …}}\\
\end{addmargin}
\end{absolutelynopagebreak}

\begin{absolutelynopagebreak}
\setstretch{.7}
{\PaliGlossA{saṅkhārānañce …}}\\
\begin{addmargin}[1em]{2em}
\setstretch{.5}
{\PaliGlossB{choices …}}\\
\end{addmargin}
\end{absolutelynopagebreak}

\begin{absolutelynopagebreak}
\setstretch{.7}
{\PaliGlossA{avijjāya ce, āvuso, bhikkhu nibbidāya virāgāya nirodhāya dhammaṃ deseti, dhammakathiko bhikkhūti alaṃvacanāya.}}\\
\begin{addmargin}[1em]{2em}
\setstretch{.5}
{\PaliGlossB{If a mendicant teaches Dhamma for disillusionment, dispassion, and cessation regarding ignorance, they’re qualified to be called a ‘mendicant who speaks on Dhamma’.}}\\
\end{addmargin}
\end{absolutelynopagebreak}

\begin{absolutelynopagebreak}
\setstretch{.7}
{\PaliGlossA{avijjāya ce, āvuso, bhikkhu nibbidāya virāgāya nirodhāya paṭipanno hoti, dhammānudhammappaṭipanno bhikkhūti alaṃvacanāya.}}\\
\begin{addmargin}[1em]{2em}
\setstretch{.5}
{\PaliGlossB{If they practice for disillusionment, dispassion, and cessation regarding ignorance, they’re qualified to be called a ‘mendicant who practices in line with the teaching’.}}\\
\end{addmargin}
\end{absolutelynopagebreak}

\begin{absolutelynopagebreak}
\setstretch{.7}
{\PaliGlossA{avijjāya ce, āvuso, bhikkhu nibbidā virāgā nirodhā anupādā vimutto hoti, diṭṭhadhammanibbānappatto bhikkhūti alaṃvacanāyā’”ti.}}\\
\begin{addmargin}[1em]{2em}
\setstretch{.5}
{\PaliGlossB{If they’re freed by not grasping by disillusionment, dispassion, and cessation regarding ignorance, they’re qualified to be called a ‘mendicant who has attained extinguishment in this very life’.”}}\\
\end{addmargin}
\end{absolutelynopagebreak}

\begin{absolutelynopagebreak}
\setstretch{.7}
{\PaliGlossA{sattamaṃ.}}\\
\begin{addmargin}[1em]{2em}
\setstretch{.5}
{\PaliGlossB{    -}}\\
\end{addmargin}
\end{absolutelynopagebreak}
