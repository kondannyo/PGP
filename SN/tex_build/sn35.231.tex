
\begin{absolutelynopagebreak}
\setstretch{.7}
{\PaliGlossA{saṃyutta nikāya 35}}\\
\begin{addmargin}[1em]{2em}
\setstretch{.5}
{\PaliGlossB{Linked Discourses 35}}\\
\end{addmargin}
\end{absolutelynopagebreak}

\begin{absolutelynopagebreak}
\setstretch{.7}
{\PaliGlossA{18. samuddavagga}}\\
\begin{addmargin}[1em]{2em}
\setstretch{.5}
{\PaliGlossB{18. The Ocean}}\\
\end{addmargin}
\end{absolutelynopagebreak}

\begin{absolutelynopagebreak}
\setstretch{.7}
{\PaliGlossA{231. khīrarukkhopamasutta}}\\
\begin{addmargin}[1em]{2em}
\setstretch{.5}
{\PaliGlossB{231. The Simile of the Latex-Producing Tree}}\\
\end{addmargin}
\end{absolutelynopagebreak}

\begin{absolutelynopagebreak}
\setstretch{.7}
{\PaliGlossA{“yassa kassaci, bhikkhave, bhikkhussa vā bhikkhuniyā vā cakkhuviññeyyesu rūpesu yo rāgo so atthi, yo doso so atthi, yo moho so atthi, yo rāgo so appahīno, yo doso so appahīno, yo moho so appahīno tassa parittā cepi cakkhuviññeyyā rūpā cakkhussa āpāthaṃ āgacchanti pariyādiyantevassa cittaṃ; ko pana vādo adhimattānaṃ.}}\\
\begin{addmargin}[1em]{2em}
\setstretch{.5}
{\PaliGlossB{“Mendicants, take any monk or nun who, when it comes to sights known by the eye, still has greed, hate, and delusion, and has not given them up. If even trivial sights come into their range of vision they overcome their mind, let alone those that are compelling.}}\\
\end{addmargin}
\end{absolutelynopagebreak}

\begin{absolutelynopagebreak}
\setstretch{.7}
{\PaliGlossA{taṃ kissa hetu?}}\\
\begin{addmargin}[1em]{2em}
\setstretch{.5}
{\PaliGlossB{Why is that?}}\\
\end{addmargin}
\end{absolutelynopagebreak}

\begin{absolutelynopagebreak}
\setstretch{.7}
{\PaliGlossA{yo, bhikkhave, rāgo, so atthi, yo doso so atthi, yo moho so atthi, yo rāgo so appahīno, yo doso so appahīno, yo moho so appahīno … pe ….}}\\
\begin{addmargin}[1em]{2em}
\setstretch{.5}
{\PaliGlossB{Because they still have greed, hate, and delusion, and have not given them up.}}\\
\end{addmargin}
\end{absolutelynopagebreak}

\begin{absolutelynopagebreak}
\setstretch{.7}
{\PaliGlossA{yassa kassaci, bhikkhave, bhikkhussa vā bhikkhuniyā vā jivhāviññeyyesu rasesu yo rāgo so atthi … pe ….}}\\
\begin{addmargin}[1em]{2em}
\setstretch{.5}
{\PaliGlossB{When it comes to sounds … smells … tastes … touches …}}\\
\end{addmargin}
\end{absolutelynopagebreak}

\begin{absolutelynopagebreak}
\setstretch{.7}
{\PaliGlossA{yassa kassaci, bhikkhave, bhikkhussa vā bhikkhuniyā vā manoviññeyyesu dhammesu yo rāgo so atthi, yo doso so atthi, yo moho so atthi, yo rāgo so appahīno, yo doso so appahīno, yo moho so appahīno, tassa parittā cepi manoviññeyyā dhammā manassa āpāthaṃ āgacchanti pariyādiyantevassa cittaṃ; ko pana vādo adhimattānaṃ.}}\\
\begin{addmargin}[1em]{2em}
\setstretch{.5}
{\PaliGlossB{thoughts known by the mind, they still have greed, hate, and delusion, and have not given them up. If even trivial thoughts come into the range of the mind they overcome their mind, let alone those that are compelling.}}\\
\end{addmargin}
\end{absolutelynopagebreak}

\begin{absolutelynopagebreak}
\setstretch{.7}
{\PaliGlossA{taṃ kissa hetu?}}\\
\begin{addmargin}[1em]{2em}
\setstretch{.5}
{\PaliGlossB{Why is that?}}\\
\end{addmargin}
\end{absolutelynopagebreak}

\begin{absolutelynopagebreak}
\setstretch{.7}
{\PaliGlossA{yo, bhikkhave, rāgo, so atthi, yo doso so atthi, yo moho so atthi, yo rāgo so appahīno, yo doso so appahīno, yo moho so appahīno.}}\\
\begin{addmargin}[1em]{2em}
\setstretch{.5}
{\PaliGlossB{Because they still have greed, hate, and delusion, and have not given them up.}}\\
\end{addmargin}
\end{absolutelynopagebreak}

\begin{absolutelynopagebreak}
\setstretch{.7}
{\PaliGlossA{seyyathāpi, bhikkhave, khīrarukkho assattho vā nigrodho vā pilakkho vā udumbaro vā daharo taruṇo komārako.}}\\
\begin{addmargin}[1em]{2em}
\setstretch{.5}
{\PaliGlossB{Suppose there was a latex-producing tree—such as a bodhi, a banyan, a wavy leaf fig, or a cluster fig—that’s a tender young sapling.}}\\
\end{addmargin}
\end{absolutelynopagebreak}

\begin{absolutelynopagebreak}
\setstretch{.7}
{\PaliGlossA{tamenaṃ puriso tiṇhāya kuṭhāriyā yato yato ābhindeyya āgaccheyya khīran”ti?}}\\
\begin{addmargin}[1em]{2em}
\setstretch{.5}
{\PaliGlossB{If a man were to chop it here and there with a sharp axe, would latex come out?”}}\\
\end{addmargin}
\end{absolutelynopagebreak}

\begin{absolutelynopagebreak}
\setstretch{.7}
{\PaliGlossA{“evaṃ, bhante”.}}\\
\begin{addmargin}[1em]{2em}
\setstretch{.5}
{\PaliGlossB{“Yes, sir.”}}\\
\end{addmargin}
\end{absolutelynopagebreak}

\begin{absolutelynopagebreak}
\setstretch{.7}
{\PaliGlossA{“taṃ kissa hetu”?}}\\
\begin{addmargin}[1em]{2em}
\setstretch{.5}
{\PaliGlossB{Why is that?}}\\
\end{addmargin}
\end{absolutelynopagebreak}

\begin{absolutelynopagebreak}
\setstretch{.7}
{\PaliGlossA{“yañhi, bhante, khīraṃ taṃ atthī”ti.}}\\
\begin{addmargin}[1em]{2em}
\setstretch{.5}
{\PaliGlossB{Because it still has latex.”}}\\
\end{addmargin}
\end{absolutelynopagebreak}

\begin{absolutelynopagebreak}
\setstretch{.7}
{\PaliGlossA{“evameva kho, bhikkhave, yassa kassaci bhikkhussa vā bhikkhuniyā vā cakkhuviññeyyesu rūpesu yo rāgo so atthi, yo doso so atthi, yo moho so atthi, yo rāgo so appahīno, yo doso so appahīno, yo moho so appahīno, tassa parittā cepi cakkhuviññeyyā rūpā cakkhussa āpāthaṃ āgacchanti pariyādiyantevassa cittaṃ; ko pana vādo adhimattānaṃ.}}\\
\begin{addmargin}[1em]{2em}
\setstretch{.5}
{\PaliGlossB{“In the same way, take any monk or nun who, when it comes to sights known by the eye, still has greed, hate, and delusion, and has not given them up. If even trivial sights come into their range of vision they overcome their mind, let alone those that are compelling.}}\\
\end{addmargin}
\end{absolutelynopagebreak}

\begin{absolutelynopagebreak}
\setstretch{.7}
{\PaliGlossA{taṃ kissa hetu?}}\\
\begin{addmargin}[1em]{2em}
\setstretch{.5}
{\PaliGlossB{Why is that?}}\\
\end{addmargin}
\end{absolutelynopagebreak}

\begin{absolutelynopagebreak}
\setstretch{.7}
{\PaliGlossA{yo, bhikkhave, rāgo so atthi, yo doso so atthi, yo moho so atthi, yo rāgo so appahīno, yo doso so appahīno, yo moho so appahīno … pe ….}}\\
\begin{addmargin}[1em]{2em}
\setstretch{.5}
{\PaliGlossB{Because they still have greed, hate, and delusion, and have not given them up.}}\\
\end{addmargin}
\end{absolutelynopagebreak}

\begin{absolutelynopagebreak}
\setstretch{.7}
{\PaliGlossA{yassa kassaci, bhikkhave, bhikkhussa vā bhikkhuniyā vā jivhāviññeyyesu rasesu yo rāgo so atthi … pe ….}}\\
\begin{addmargin}[1em]{2em}
\setstretch{.5}
{\PaliGlossB{When it comes to sounds … smells … tastes … touches …}}\\
\end{addmargin}
\end{absolutelynopagebreak}

\begin{absolutelynopagebreak}
\setstretch{.7}
{\PaliGlossA{yassa kassaci, bhikkhave, bhikkhussa vā bhikkhuniyā vā manoviññeyyesu dhammesu yo rāgo so atthi, yo doso so atthi, yo moho so atthi, yo rāgo so appahīno, yo doso so appahīno, yo moho so appahīno, tassa parittā cepi manoviññeyyā dhammā manassa āpāthaṃ āgacchanti pariyādiyantevassa cittaṃ; ko pana vādo adhimattānaṃ.}}\\
\begin{addmargin}[1em]{2em}
\setstretch{.5}
{\PaliGlossB{thoughts known by the mind, they still have greed, hate, and delusion, and have not given them up. If even trivial thoughts come into the range of the mind they overcome their mind, let alone those that are compelling.}}\\
\end{addmargin}
\end{absolutelynopagebreak}

\begin{absolutelynopagebreak}
\setstretch{.7}
{\PaliGlossA{taṃ kissa hetu?}}\\
\begin{addmargin}[1em]{2em}
\setstretch{.5}
{\PaliGlossB{Why is that?}}\\
\end{addmargin}
\end{absolutelynopagebreak}

\begin{absolutelynopagebreak}
\setstretch{.7}
{\PaliGlossA{yo, bhikkhave, rāgo so atthi, yo doso so atthi, yo moho so atthi, yo rāgo so appahīno, yo doso so appahīno, yo moho so appahīno.}}\\
\begin{addmargin}[1em]{2em}
\setstretch{.5}
{\PaliGlossB{Because they still have greed, hate, and delusion, and have not given them up.}}\\
\end{addmargin}
\end{absolutelynopagebreak}

\begin{absolutelynopagebreak}
\setstretch{.7}
{\PaliGlossA{yassa kassaci, bhikkhave, bhikkhussa vā bhikkhuniyā vā cakkhuviññeyyesu rūpesu yo rāgo so natthi, yo doso so natthi, yo moho so natthi, yo rāgo so pahīno, yo doso so pahīno, yo moho so pahīno, tassa adhimattā cepi cakkhuviññeyyā rūpā cakkhussa āpāthaṃ āgacchanti nevassa cittaṃ pariyādiyanti; ko pana vādo parittānaṃ.}}\\
\begin{addmargin}[1em]{2em}
\setstretch{.5}
{\PaliGlossB{Take any monk or nun who, when it comes to sights known by the eye, has no greed, hate, and delusion left, and has given them up. If even compelling sights come into their range of vision they don’t overcome their mind, let alone those that are trivial.}}\\
\end{addmargin}
\end{absolutelynopagebreak}

\begin{absolutelynopagebreak}
\setstretch{.7}
{\PaliGlossA{taṃ kissa hetu?}}\\
\begin{addmargin}[1em]{2em}
\setstretch{.5}
{\PaliGlossB{Why is that?}}\\
\end{addmargin}
\end{absolutelynopagebreak}

\begin{absolutelynopagebreak}
\setstretch{.7}
{\PaliGlossA{yo, bhikkhave, rāgo so natthi, yo doso so natthi, yo moho so natthi, yo rāgo so pahīno, yo doso so pahīno, yo moho so pahīno … pe ….}}\\
\begin{addmargin}[1em]{2em}
\setstretch{.5}
{\PaliGlossB{Because they have no greed, hate, and delusion left, and have given them up.}}\\
\end{addmargin}
\end{absolutelynopagebreak}

\begin{absolutelynopagebreak}
\setstretch{.7}
{\PaliGlossA{yassa kassaci, bhikkhave, bhikkhussa vā bhikkhuniyā vā jivhāviññeyyesu rasesu … pe … manoviññeyyesu dhammesu yo rāgo so natthi, yo doso so natthi, yo moho so natthi, yo rāgo so pahīno, yo doso so pahīno, yo moho so pahīno, tassa adhimattā cepi manoviññeyyā dhammā manassa āpāthaṃ āgacchanti nevassa cittaṃ pariyādiyanti; ko pana vādo parittānaṃ.}}\\
\begin{addmargin}[1em]{2em}
\setstretch{.5}
{\PaliGlossB{When it comes to sounds … smells … tastes … touches … thoughts known by the mind, they have no greed, hate, and delusion left, and have given them up. If even compelling thoughts come into the range of the mind they don’t overcome their mind, let alone those that are trivial.}}\\
\end{addmargin}
\end{absolutelynopagebreak}

\begin{absolutelynopagebreak}
\setstretch{.7}
{\PaliGlossA{taṃ kissa hetu?}}\\
\begin{addmargin}[1em]{2em}
\setstretch{.5}
{\PaliGlossB{Why is that?}}\\
\end{addmargin}
\end{absolutelynopagebreak}

\begin{absolutelynopagebreak}
\setstretch{.7}
{\PaliGlossA{yo, bhikkhave, rāgo so natthi, yo doso so natthi, yo moho so natthi, yo rāgo so pahīno, yo doso so pahīno, yo moho so pahīno.}}\\
\begin{addmargin}[1em]{2em}
\setstretch{.5}
{\PaliGlossB{Because they have no greed, hate, and delusion left, and have given them up.}}\\
\end{addmargin}
\end{absolutelynopagebreak}

\begin{absolutelynopagebreak}
\setstretch{.7}
{\PaliGlossA{seyyathāpi, bhikkhave, khīrarukkho assattho vā nigrodho vā pilakkho vā udumbaro vā sukkho kolāpo terovassiko.}}\\
\begin{addmargin}[1em]{2em}
\setstretch{.5}
{\PaliGlossB{Suppose there was a latex-producing tree—such as a bodhi, a banyan, a wavy leaf fig, or a cluster fig—that’s dried up, withered, and decrepit.}}\\
\end{addmargin}
\end{absolutelynopagebreak}

\begin{absolutelynopagebreak}
\setstretch{.7}
{\PaliGlossA{tamenaṃ puriso tiṇhāya kuṭhāriyā yato yato ābhindeyya āgaccheyya khīran”ti?}}\\
\begin{addmargin}[1em]{2em}
\setstretch{.5}
{\PaliGlossB{If a man were to chop it here and there with a sharp axe, would latex come out?”}}\\
\end{addmargin}
\end{absolutelynopagebreak}

\begin{absolutelynopagebreak}
\setstretch{.7}
{\PaliGlossA{“no hetaṃ, bhante”.}}\\
\begin{addmargin}[1em]{2em}
\setstretch{.5}
{\PaliGlossB{“No, sir.}}\\
\end{addmargin}
\end{absolutelynopagebreak}

\begin{absolutelynopagebreak}
\setstretch{.7}
{\PaliGlossA{“taṃ kissa hetu”?}}\\
\begin{addmargin}[1em]{2em}
\setstretch{.5}
{\PaliGlossB{Why is that?}}\\
\end{addmargin}
\end{absolutelynopagebreak}

\begin{absolutelynopagebreak}
\setstretch{.7}
{\PaliGlossA{“yañhi, bhante, khīraṃ taṃ natthī”ti.}}\\
\begin{addmargin}[1em]{2em}
\setstretch{.5}
{\PaliGlossB{Because it has no latex left.”}}\\
\end{addmargin}
\end{absolutelynopagebreak}

\begin{absolutelynopagebreak}
\setstretch{.7}
{\PaliGlossA{“evameva kho, bhikkhave, yassa kassaci bhikkhussa vā bhikkhuniyā vā cakkhuviññeyyesu rūpesu yo rāgo so natthi, yo doso so natthi, yo moho so natthi, yo rāgo so pahīno, yo doso so pahīno, yo moho so pahīno, tassa adhimattā cepi cakkhuviññeyyā rūpā cakkhussa āpāthaṃ āgacchanti nevassa cittaṃ pariyādiyanti; ko pana vādo parittānaṃ.}}\\
\begin{addmargin}[1em]{2em}
\setstretch{.5}
{\PaliGlossB{“In the same way, take any monk or nun who, when it comes to sights known by the eye, has no greed, hate, and delusion left, and has given them up. If even compelling sights come into their range of vision they don’t overcome their mind, let alone those that are trivial.}}\\
\end{addmargin}
\end{absolutelynopagebreak}

\begin{absolutelynopagebreak}
\setstretch{.7}
{\PaliGlossA{taṃ kissa hetu?}}\\
\begin{addmargin}[1em]{2em}
\setstretch{.5}
{\PaliGlossB{Why is that?}}\\
\end{addmargin}
\end{absolutelynopagebreak}

\begin{absolutelynopagebreak}
\setstretch{.7}
{\PaliGlossA{yo, bhikkhave, rāgo so natthi, yo doso so natthi, yo moho so natthi, yo rāgo so pahīno, yo doso so pahīno, yo moho so pahīno … pe ….}}\\
\begin{addmargin}[1em]{2em}
\setstretch{.5}
{\PaliGlossB{Because they have no greed, hate, and delusion left, and have given them up.}}\\
\end{addmargin}
\end{absolutelynopagebreak}

\begin{absolutelynopagebreak}
\setstretch{.7}
{\PaliGlossA{yassa kassaci, bhikkhave, bhikkhussa vā bhikkhuniyā vā jivhāviññeyyesu rasesu … pe ….}}\\
\begin{addmargin}[1em]{2em}
\setstretch{.5}
{\PaliGlossB{When it comes to sounds … smells … tastes … touches …}}\\
\end{addmargin}
\end{absolutelynopagebreak}

\begin{absolutelynopagebreak}
\setstretch{.7}
{\PaliGlossA{yassa kassaci, bhikkhave, bhikkhussa vā bhikkhuniyā vā manoviññeyyesu dhammesu yo rāgo so natthi, yo doso so natthi, yo moho so natthi, yo rāgo so pahīno, yo doso so pahīno, yo moho so pahīno, tassa adhimattā cepi manoviññeyyā dhammā manassa āpāthaṃ āgacchanti, nevassa cittaṃ pariyādiyanti; ko pana vādo parittānaṃ.}}\\
\begin{addmargin}[1em]{2em}
\setstretch{.5}
{\PaliGlossB{thoughts known by the mind, they have no greed, hate, and delusion left, and have given them up. If even compelling thoughts come into the range of the mind they don’t overcome their mind, let alone those that are trivial.}}\\
\end{addmargin}
\end{absolutelynopagebreak}

\begin{absolutelynopagebreak}
\setstretch{.7}
{\PaliGlossA{taṃ kissa hetu?}}\\
\begin{addmargin}[1em]{2em}
\setstretch{.5}
{\PaliGlossB{Why is that?}}\\
\end{addmargin}
\end{absolutelynopagebreak}

\begin{absolutelynopagebreak}
\setstretch{.7}
{\PaliGlossA{yo, bhikkhave, rāgo so natthi, yo doso so natthi, yo moho so natthi, yo rāgo so pahīno, yo doso so pahīno, yo moho so pahīno”ti.}}\\
\begin{addmargin}[1em]{2em}
\setstretch{.5}
{\PaliGlossB{Because they have no greed, hate, and delusion left, and have given them up.”}}\\
\end{addmargin}
\end{absolutelynopagebreak}

\begin{absolutelynopagebreak}
\setstretch{.7}
{\PaliGlossA{catutthaṃ.}}\\
\begin{addmargin}[1em]{2em}
\setstretch{.5}
{\PaliGlossB{    -}}\\
\end{addmargin}
\end{absolutelynopagebreak}
