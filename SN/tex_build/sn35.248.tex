
\begin{absolutelynopagebreak}
\setstretch{.7}
{\PaliGlossA{saṃyutta nikāya 35}}\\
\begin{addmargin}[1em]{2em}
\setstretch{.5}
{\PaliGlossB{Linked Discourses 35}}\\
\end{addmargin}
\end{absolutelynopagebreak}

\begin{absolutelynopagebreak}
\setstretch{.7}
{\PaliGlossA{19. āsīvisavagga}}\\
\begin{addmargin}[1em]{2em}
\setstretch{.5}
{\PaliGlossB{19. The Simile of the Vipers}}\\
\end{addmargin}
\end{absolutelynopagebreak}

\begin{absolutelynopagebreak}
\setstretch{.7}
{\PaliGlossA{248. yavakalāpisutta}}\\
\begin{addmargin}[1em]{2em}
\setstretch{.5}
{\PaliGlossB{248. The Sheaf of Barley}}\\
\end{addmargin}
\end{absolutelynopagebreak}

\begin{absolutelynopagebreak}
\setstretch{.7}
{\PaliGlossA{“seyyathāpi, bhikkhave, yavakalāpī cātumahāpathe nikkhittā assa.}}\\
\begin{addmargin}[1em]{2em}
\setstretch{.5}
{\PaliGlossB{“Mendicants, suppose a sheaf of barley was placed at a crossroads.}}\\
\end{addmargin}
\end{absolutelynopagebreak}

\begin{absolutelynopagebreak}
\setstretch{.7}
{\PaliGlossA{atha cha purisā āgaccheyyuṃ byābhaṅgihatthā.}}\\
\begin{addmargin}[1em]{2em}
\setstretch{.5}
{\PaliGlossB{Then six people would come along carrying flails,}}\\
\end{addmargin}
\end{absolutelynopagebreak}

\begin{absolutelynopagebreak}
\setstretch{.7}
{\PaliGlossA{te yavakalāpiṃ chahi byābhaṅgīhi haneyyuṃ.}}\\
\begin{addmargin}[1em]{2em}
\setstretch{.5}
{\PaliGlossB{and started threshing the sheaf of barley.}}\\
\end{addmargin}
\end{absolutelynopagebreak}

\begin{absolutelynopagebreak}
\setstretch{.7}
{\PaliGlossA{evañhi sā, bhikkhave, yavakalāpī suhatā assa chahi byābhaṅgīhi haññamānā.}}\\
\begin{addmargin}[1em]{2em}
\setstretch{.5}
{\PaliGlossB{So that sheaf of barley would be thoroughly threshed by those six flails.}}\\
\end{addmargin}
\end{absolutelynopagebreak}

\begin{absolutelynopagebreak}
\setstretch{.7}
{\PaliGlossA{atha sattamo puriso āgaccheyya byābhaṅgihattho.}}\\
\begin{addmargin}[1em]{2em}
\setstretch{.5}
{\PaliGlossB{Then a seventh person would come along carrying a flail,}}\\
\end{addmargin}
\end{absolutelynopagebreak}

\begin{absolutelynopagebreak}
\setstretch{.7}
{\PaliGlossA{so taṃ yavakalāpiṃ sattamāya byābhaṅgiyā haneyya.}}\\
\begin{addmargin}[1em]{2em}
\setstretch{.5}
{\PaliGlossB{and they’d give the sheaf of barley a seventh threshing.}}\\
\end{addmargin}
\end{absolutelynopagebreak}

\begin{absolutelynopagebreak}
\setstretch{.7}
{\PaliGlossA{evañhi sā bhikkhave, yavakalāpī suhatatarā assa, sattamāya byābhaṅgiyā haññamānā.}}\\
\begin{addmargin}[1em]{2em}
\setstretch{.5}
{\PaliGlossB{So that sheaf of barley would be even more thoroughly threshed by that seventh flail.}}\\
\end{addmargin}
\end{absolutelynopagebreak}

\begin{absolutelynopagebreak}
\setstretch{.7}
{\PaliGlossA{evameva kho, bhikkhave, assutavā puthujjano cakkhusmiṃ haññati manāpāmanāpehi rūpehi … pe …}}\\
\begin{addmargin}[1em]{2em}
\setstretch{.5}
{\PaliGlossB{In the same way, an uneducated ordinary person is struck in the eye by both pleasant and unpleasant sights.}}\\
\end{addmargin}
\end{absolutelynopagebreak}

\begin{absolutelynopagebreak}
\setstretch{.7}
{\PaliGlossA{jivhāya haññati manāpāmanāpehi rasehi … pe …}}\\
\begin{addmargin}[1em]{2em}
\setstretch{.5}
{\PaliGlossB{They’re struck in the ear … nose … tongue … body …}}\\
\end{addmargin}
\end{absolutelynopagebreak}

\begin{absolutelynopagebreak}
\setstretch{.7}
{\PaliGlossA{manasmiṃ haññati manāpāmanāpehi dhammehi.}}\\
\begin{addmargin}[1em]{2em}
\setstretch{.5}
{\PaliGlossB{mind by both pleasant and unpleasant thoughts.}}\\
\end{addmargin}
\end{absolutelynopagebreak}

\begin{absolutelynopagebreak}
\setstretch{.7}
{\PaliGlossA{sace so, bhikkhave, assutavā puthujjano āyatiṃ punabbhavāya ceteti, evañhi so, bhikkhave, moghapuriso suhatataro hoti, seyyathāpi sā yavakalāpī sattamāya byābhaṅgiyā haññamānā.}}\\
\begin{addmargin}[1em]{2em}
\setstretch{.5}
{\PaliGlossB{And if that uneducated ordinary person has intentions regarding rebirth into a new state of existence in the future, that foolish person is even more thoroughly struck, like that sheaf of barley threshed by the seventh person.}}\\
\end{addmargin}
\end{absolutelynopagebreak}

\begin{absolutelynopagebreak}
\setstretch{.7}
{\PaliGlossA{bhūtapubbaṃ, bhikkhave, devāsurasaṅgāmo samupabyūḷho ahosi.}}\\
\begin{addmargin}[1em]{2em}
\setstretch{.5}
{\PaliGlossB{Once upon a time, a battle was fought between the gods and the demons.}}\\
\end{addmargin}
\end{absolutelynopagebreak}

\begin{absolutelynopagebreak}
\setstretch{.7}
{\PaliGlossA{atha kho, bhikkhave, vepacitti asurindo asure āmantesi:}}\\
\begin{addmargin}[1em]{2em}
\setstretch{.5}
{\PaliGlossB{Then Vepacitti, lord of demons, addressed the demons,}}\\
\end{addmargin}
\end{absolutelynopagebreak}

\begin{absolutelynopagebreak}
\setstretch{.7}
{\PaliGlossA{‘sace, mārisā, devāsurasaṅgāme samupabyūḷhe asurā jineyyuṃ devā parājineyyuṃ, yena naṃ sakkaṃ devānamindaṃ kaṇṭhapañcamehi bandhanehi bandhitvā mama santike āneyyātha asurapuran’ti.}}\\
\begin{addmargin}[1em]{2em}
\setstretch{.5}
{\PaliGlossB{‘My good sirs, if the demons defeat the gods in this battle, bind Sakka, the lord of gods, by his limbs and neck and bring him to my presence in the castle of demons.’}}\\
\end{addmargin}
\end{absolutelynopagebreak}

\begin{absolutelynopagebreak}
\setstretch{.7}
{\PaliGlossA{sakkopi kho, bhikkhave, devānamindo deve tāvatiṃse āmantesi:}}\\
\begin{addmargin}[1em]{2em}
\setstretch{.5}
{\PaliGlossB{Meanwhile, Sakka, lord of gods, addressed the gods of the Thirty-Three,}}\\
\end{addmargin}
\end{absolutelynopagebreak}

\begin{absolutelynopagebreak}
\setstretch{.7}
{\PaliGlossA{‘sace, mārisā, devāsurasaṅgāme samupabyūḷhe devā jineyyuṃ asurā parājineyyuṃ, yena naṃ vepacittiṃ asurindaṃ kaṇṭhapañcamehi bandhanehi bandhitvā mama santike āneyyātha sudhammaṃ devasabhan’ti.}}\\
\begin{addmargin}[1em]{2em}
\setstretch{.5}
{\PaliGlossB{‘My good sirs, if the gods defeat the demons in this battle, bind Vepacitti by his limbs and neck and bring him to my presence in the Sudhamma hall of the gods.’}}\\
\end{addmargin}
\end{absolutelynopagebreak}

\begin{absolutelynopagebreak}
\setstretch{.7}
{\PaliGlossA{tasmiṃ kho pana, bhikkhave, saṅgāme devā jiniṃsu, asurā parājiniṃsu.}}\\
\begin{addmargin}[1em]{2em}
\setstretch{.5}
{\PaliGlossB{In that battle the gods won and the demons lost.}}\\
\end{addmargin}
\end{absolutelynopagebreak}

\begin{absolutelynopagebreak}
\setstretch{.7}
{\PaliGlossA{atha kho, bhikkhave, devā tāvatiṃsā vepacittiṃ asurindaṃ kaṇṭhapañcamehi bandhanehi bandhitvā sakkassa devānamindassa santike ānesuṃ sudhammaṃ devasabhaṃ.}}\\
\begin{addmargin}[1em]{2em}
\setstretch{.5}
{\PaliGlossB{So the gods of the Thirty-Three bound Vepacitti by his limbs and neck and brought him to Sakka’s presence in the Sudhamma hall of the gods.}}\\
\end{addmargin}
\end{absolutelynopagebreak}

\begin{absolutelynopagebreak}
\setstretch{.7}
{\PaliGlossA{tatra sudaṃ, bhikkhave, vepacitti asurindo kaṇṭhapañcamehi bandhanehi baddho hoti.}}\\
\begin{addmargin}[1em]{2em}
\setstretch{.5}
{\PaliGlossB{And there Vepacitti remained bound by his limbs and neck.}}\\
\end{addmargin}
\end{absolutelynopagebreak}

\begin{absolutelynopagebreak}
\setstretch{.7}
{\PaliGlossA{yadā kho, bhikkhave, vepacittissa asurindassa evaṃ hoti:}}\\
\begin{addmargin}[1em]{2em}
\setstretch{.5}
{\PaliGlossB{That is, until he thought,}}\\
\end{addmargin}
\end{absolutelynopagebreak}

\begin{absolutelynopagebreak}
\setstretch{.7}
{\PaliGlossA{‘dhammikā kho devā, adhammikā asurā, idheva dānāhaṃ devapuraṃ gacchāmī’ti.}}\\
\begin{addmargin}[1em]{2em}
\setstretch{.5}
{\PaliGlossB{‘It’s the gods who are principled, while the demons are unprincipled. Now I belong right here in the castle of the gods.’}}\\
\end{addmargin}
\end{absolutelynopagebreak}

\begin{absolutelynopagebreak}
\setstretch{.7}
{\PaliGlossA{atha kaṇṭhapañcamehi bandhanehi muttaṃ attānaṃ samanupassati, dibbehi ca pañcahi kāmaguṇehi samappito samaṅgībhūto paricāreti.}}\\
\begin{addmargin}[1em]{2em}
\setstretch{.5}
{\PaliGlossB{Then he found himself freed from the bonds on his limbs and neck. He entertained himself, supplied and provided with the five kinds of heavenly sensual stimulation.}}\\
\end{addmargin}
\end{absolutelynopagebreak}

\begin{absolutelynopagebreak}
\setstretch{.7}
{\PaliGlossA{yadā ca kho, bhikkhave, vepacittissa asurindassa evaṃ hoti:}}\\
\begin{addmargin}[1em]{2em}
\setstretch{.5}
{\PaliGlossB{But when he thought,}}\\
\end{addmargin}
\end{absolutelynopagebreak}

\begin{absolutelynopagebreak}
\setstretch{.7}
{\PaliGlossA{‘dhammikā kho asurā, adhammikā devā, tattheva dānāhaṃ asurapuraṃ gamissāmī’ti.}}\\
\begin{addmargin}[1em]{2em}
\setstretch{.5}
{\PaliGlossB{‘It’s the demons who are principled, while the gods are unprincipled. Now I will go over there to the castle of the demons,’}}\\
\end{addmargin}
\end{absolutelynopagebreak}

\begin{absolutelynopagebreak}
\setstretch{.7}
{\PaliGlossA{atha kaṇṭhapañcamehi bandhanehi baddhaṃ attānaṃ samanupassati, dibbehi ca pañcahi kāmaguṇehi parihāyati.}}\\
\begin{addmargin}[1em]{2em}
\setstretch{.5}
{\PaliGlossB{he found himself bound by his limbs and neck, and the five kinds of heavenly sensual stimulation disappeared.}}\\
\end{addmargin}
\end{absolutelynopagebreak}

\begin{absolutelynopagebreak}
\setstretch{.7}
{\PaliGlossA{evaṃ sukhumaṃ kho, bhikkhave, vepacittibandhanaṃ.}}\\
\begin{addmargin}[1em]{2em}
\setstretch{.5}
{\PaliGlossB{That’s how subtly Vepacitti was bound.}}\\
\end{addmargin}
\end{absolutelynopagebreak}

\begin{absolutelynopagebreak}
\setstretch{.7}
{\PaliGlossA{tato sukhumataraṃ mārabandhanaṃ.}}\\
\begin{addmargin}[1em]{2em}
\setstretch{.5}
{\PaliGlossB{But the bonds of Māra are even more subtle than that.}}\\
\end{addmargin}
\end{absolutelynopagebreak}

\begin{absolutelynopagebreak}
\setstretch{.7}
{\PaliGlossA{maññamāno kho, bhikkhave, baddho mārassa, amaññamāno mutto pāpimato.}}\\
\begin{addmargin}[1em]{2em}
\setstretch{.5}
{\PaliGlossB{When you identify, you’re bound by Māra. Not identifying, you’re free from the Wicked One.}}\\
\end{addmargin}
\end{absolutelynopagebreak}

\begin{absolutelynopagebreak}
\setstretch{.7}
{\PaliGlossA{‘asmī’ti, bhikkhave, maññitametaṃ, ‘ayamahamasmī’ti maññitametaṃ, ‘bhavissan’ti maññitametaṃ, ‘na bhavissan’ti maññitametaṃ, ‘rūpī bhavissan’ti maññitametaṃ, ‘arūpī bhavissan’ti maññitametaṃ, ‘saññī bhavissan’ti maññitametaṃ, ‘asaññī bhavissan’ti maññitametaṃ, ‘nevasaññīnāsaññī bhavissan’ti maññitametaṃ.}}\\
\begin{addmargin}[1em]{2em}
\setstretch{.5}
{\PaliGlossB{These are all forms of identifying: ‘I am’, ‘I am this’, ‘I will be’, ‘I will not be’, ‘I will have form’, ‘I will be formless’, ‘I will be percipient’, ‘I will be non-percipient’, ‘I will be neither percipient nor non-percipient.’}}\\
\end{addmargin}
\end{absolutelynopagebreak}

\begin{absolutelynopagebreak}
\setstretch{.7}
{\PaliGlossA{maññitaṃ, bhikkhave, rogo, maññitaṃ gaṇḍo, maññitaṃ sallaṃ.}}\\
\begin{addmargin}[1em]{2em}
\setstretch{.5}
{\PaliGlossB{Conceit is a disease, a boil, a dart.}}\\
\end{addmargin}
\end{absolutelynopagebreak}

\begin{absolutelynopagebreak}
\setstretch{.7}
{\PaliGlossA{tasmātiha, bhikkhave, ‘amaññamānena cetasā viharissāmā’ti—}}\\
\begin{addmargin}[1em]{2em}
\setstretch{.5}
{\PaliGlossB{So mendicants, you should train yourselves like this: ‘We will live with a heart that does not identify.’}}\\
\end{addmargin}
\end{absolutelynopagebreak}

\begin{absolutelynopagebreak}
\setstretch{.7}
{\PaliGlossA{evañhi vo, bhikkhave, sikkhitabbaṃ.}}\\
\begin{addmargin}[1em]{2em}
\setstretch{.5}
{\PaliGlossB{    -}}\\
\end{addmargin}
\end{absolutelynopagebreak}

\begin{absolutelynopagebreak}
\setstretch{.7}
{\PaliGlossA{‘asmī’ti, bhikkhave, iñjitametaṃ, ‘ayamahamasmī’ti iñjitametaṃ, ‘bhavissan’ti iñjitametaṃ, ‘na bhavissan’ti iñjitametaṃ, ‘rūpī bhavissan’ti iñjitametaṃ, ‘arūpī bhavissan’ti iñjitametaṃ, ‘saññī bhavissan’ti iñjitametaṃ, ‘asaññī bhavissan’ti iñjitametaṃ, ‘nevasaññīnāsaññī bhavissan’ti iñjitametaṃ.}}\\
\begin{addmargin}[1em]{2em}
\setstretch{.5}
{\PaliGlossB{These are all disturbances: ‘I am’, ‘I am this’, ‘I will be’, ‘I will not be’, ‘I will have form’, ‘I will be formless’, ‘I will be percipient’, ‘I will be non-percipient’, ‘I will be neither percipient nor non-percipient.’}}\\
\end{addmargin}
\end{absolutelynopagebreak}

\begin{absolutelynopagebreak}
\setstretch{.7}
{\PaliGlossA{iñjitaṃ, bhikkhave, rogo, iñjitaṃ gaṇḍo, iñjitaṃ sallaṃ.}}\\
\begin{addmargin}[1em]{2em}
\setstretch{.5}
{\PaliGlossB{Disturbances are a disease, a boil, a dart.}}\\
\end{addmargin}
\end{absolutelynopagebreak}

\begin{absolutelynopagebreak}
\setstretch{.7}
{\PaliGlossA{tasmātiha, bhikkhave, ‘aniñjamānena cetasā viharissāmā’ti—}}\\
\begin{addmargin}[1em]{2em}
\setstretch{.5}
{\PaliGlossB{So mendicants, you should train yourselves like this: ‘We will live with a heart free of disturbances.’}}\\
\end{addmargin}
\end{absolutelynopagebreak}

\begin{absolutelynopagebreak}
\setstretch{.7}
{\PaliGlossA{evañhi vo, bhikkhave, sikkhitabbaṃ.}}\\
\begin{addmargin}[1em]{2em}
\setstretch{.5}
{\PaliGlossB{    -}}\\
\end{addmargin}
\end{absolutelynopagebreak}

\begin{absolutelynopagebreak}
\setstretch{.7}
{\PaliGlossA{‘asmī’ti, bhikkhave, phanditametaṃ, ‘ayamahamasmī’ti phanditametaṃ, ‘bhavissan’ti … pe … ‘na bhavissan’ti … ‘rūpī bhavissan’ti … ‘arūpī bhavissan’ti … ‘saññī bhavissan’ti … ‘asaññī bhavissan’ti … ‘nevasaññīnāsaññī bhavissan’ti phanditametaṃ.}}\\
\begin{addmargin}[1em]{2em}
\setstretch{.5}
{\PaliGlossB{These are all tremblings: ‘I am’, ‘I am this’, ‘I will be’, ‘I will not be’, ‘I will have form’, ‘I will be formless’, ‘I will be percipient’, ‘I will be non-percipient’, ‘I will be neither percipient nor non-percipient.’}}\\
\end{addmargin}
\end{absolutelynopagebreak}

\begin{absolutelynopagebreak}
\setstretch{.7}
{\PaliGlossA{phanditaṃ, bhikkhave, rogo, phanditaṃ gaṇḍo, phanditaṃ sallaṃ.}}\\
\begin{addmargin}[1em]{2em}
\setstretch{.5}
{\PaliGlossB{Trembling is a disease, a boil, a dart.}}\\
\end{addmargin}
\end{absolutelynopagebreak}

\begin{absolutelynopagebreak}
\setstretch{.7}
{\PaliGlossA{tasmātiha, bhikkhave, ‘aphandamānena cetasā viharissāmā’ti—}}\\
\begin{addmargin}[1em]{2em}
\setstretch{.5}
{\PaliGlossB{So mendicants, you should train yourselves like this: ‘We will live with a heart free of tremblings.’}}\\
\end{addmargin}
\end{absolutelynopagebreak}

\begin{absolutelynopagebreak}
\setstretch{.7}
{\PaliGlossA{evañhi vo, bhikkhave, sikkhitabbaṃ.}}\\
\begin{addmargin}[1em]{2em}
\setstretch{.5}
{\PaliGlossB{    -}}\\
\end{addmargin}
\end{absolutelynopagebreak}

\begin{absolutelynopagebreak}
\setstretch{.7}
{\PaliGlossA{‘asmī’ti, bhikkhave, papañcitametaṃ, ‘ayamahamasmī’ti papañcitametaṃ, ‘bhavissan’ti … pe … ‘na bhavissan’ti … ‘rūpī bhavissan’ti … ‘arūpī bhavissan’ti … ‘saññī bhavissan’ti … ‘asaññī bhavissan’ti … ‘nevasaññīnāsaññī bhavissan’ti papañcitametaṃ.}}\\
\begin{addmargin}[1em]{2em}
\setstretch{.5}
{\PaliGlossB{These are all proliferations: ‘I am’, ‘I am this’, ‘I will be’, ‘I will not be’, ‘I will have form’, ‘I will be formless’, ‘I will be percipient’, ‘I will be non-percipient’, ‘I will be neither percipient nor non-percipient.’}}\\
\end{addmargin}
\end{absolutelynopagebreak}

\begin{absolutelynopagebreak}
\setstretch{.7}
{\PaliGlossA{papañcitaṃ, bhikkhave, rogo, papañcitaṃ gaṇḍo, papañcitaṃ sallaṃ.}}\\
\begin{addmargin}[1em]{2em}
\setstretch{.5}
{\PaliGlossB{Proliferation is a disease, a boil, a dart.}}\\
\end{addmargin}
\end{absolutelynopagebreak}

\begin{absolutelynopagebreak}
\setstretch{.7}
{\PaliGlossA{tasmātiha, bhikkhave, ‘nippapañcena cetasā viharissāmā’ti—}}\\
\begin{addmargin}[1em]{2em}
\setstretch{.5}
{\PaliGlossB{So mendicants, you should train yourselves like this: ‘We will live with a heart free of proliferation.’}}\\
\end{addmargin}
\end{absolutelynopagebreak}

\begin{absolutelynopagebreak}
\setstretch{.7}
{\PaliGlossA{evañhi vo, bhikkhave, sikkhitabbaṃ.}}\\
\begin{addmargin}[1em]{2em}
\setstretch{.5}
{\PaliGlossB{    -}}\\
\end{addmargin}
\end{absolutelynopagebreak}

\begin{absolutelynopagebreak}
\setstretch{.7}
{\PaliGlossA{‘asmī’ti, bhikkhave, mānagatametaṃ, ‘ayamahamasmī’ti mānagatametaṃ, ‘bhavissan’ti mānagatametaṃ, ‘na bhavissan’ti mānagatametaṃ, ‘rūpī bhavissan’ti mānagatametaṃ, ‘arūpī bhavissan’ti mānagatametaṃ, ‘saññī bhavissan’ti mānagatametaṃ, ‘asaññī bhavissan’ti mānagatametaṃ, ‘nevasaññīnāsaññī bhavissan’ti mānagatametaṃ.}}\\
\begin{addmargin}[1em]{2em}
\setstretch{.5}
{\PaliGlossB{These are all conceits: ‘I am’, ‘I am this’, ‘I will be’, ‘I will not be’, ‘I will have form’, ‘I will be formless’, ‘I will be percipient’, ‘I will be non-percipient’, ‘I will be neither percipient nor non-percipient.’}}\\
\end{addmargin}
\end{absolutelynopagebreak}

\begin{absolutelynopagebreak}
\setstretch{.7}
{\PaliGlossA{mānagataṃ, bhikkhave, rogo, mānagataṃ gaṇḍo, mānagataṃ sallaṃ.}}\\
\begin{addmargin}[1em]{2em}
\setstretch{.5}
{\PaliGlossB{Conceit is a disease, a boil, a dart.}}\\
\end{addmargin}
\end{absolutelynopagebreak}

\begin{absolutelynopagebreak}
\setstretch{.7}
{\PaliGlossA{tasmātiha, bhikkhave, ‘nihatamānena cetasā viharissāmā’ti—}}\\
\begin{addmargin}[1em]{2em}
\setstretch{.5}
{\PaliGlossB{So mendicants, you should train yourselves like this: ‘We will live with a heart that has struck down conceit.’”}}\\
\end{addmargin}
\end{absolutelynopagebreak}

\begin{absolutelynopagebreak}
\setstretch{.7}
{\PaliGlossA{evañhi vo, bhikkhave, sikkhitabban”ti.}}\\
\begin{addmargin}[1em]{2em}
\setstretch{.5}
{\PaliGlossB{    -}}\\
\end{addmargin}
\end{absolutelynopagebreak}

\begin{absolutelynopagebreak}
\setstretch{.7}
{\PaliGlossA{ekādasamaṃ.}}\\
\begin{addmargin}[1em]{2em}
\setstretch{.5}
{\PaliGlossB{    -}}\\
\end{addmargin}
\end{absolutelynopagebreak}

\begin{absolutelynopagebreak}
\setstretch{.7}
{\PaliGlossA{āsīvisavaggo catuttho.}}\\
\begin{addmargin}[1em]{2em}
\setstretch{.5}
{\PaliGlossB{    -}}\\
\end{addmargin}
\end{absolutelynopagebreak}

\begin{absolutelynopagebreak}
\setstretch{.7}
{\PaliGlossA{āsīviso ratho kummo,}}\\
\begin{addmargin}[1em]{2em}
\setstretch{.5}
{\PaliGlossB{    -}}\\
\end{addmargin}
\end{absolutelynopagebreak}

\begin{absolutelynopagebreak}
\setstretch{.7}
{\PaliGlossA{dve dārukkhandhā avassuto;}}\\
\begin{addmargin}[1em]{2em}
\setstretch{.5}
{\PaliGlossB{    -}}\\
\end{addmargin}
\end{absolutelynopagebreak}

\begin{absolutelynopagebreak}
\setstretch{.7}
{\PaliGlossA{dukkhadhammā kiṃsukā vīṇā,}}\\
\begin{addmargin}[1em]{2em}
\setstretch{.5}
{\PaliGlossB{    -}}\\
\end{addmargin}
\end{absolutelynopagebreak}

\begin{absolutelynopagebreak}
\setstretch{.7}
{\PaliGlossA{chappāṇā yavakalāpīti.}}\\
\begin{addmargin}[1em]{2em}
\setstretch{.5}
{\PaliGlossB{    -}}\\
\end{addmargin}
\end{absolutelynopagebreak}

\begin{absolutelynopagebreak}
\setstretch{.7}
{\PaliGlossA{saḷāyatanavagge catutthapaṇṇāsako samatto.}}\\
\begin{addmargin}[1em]{2em}
\setstretch{.5}
{\PaliGlossB{    -}}\\
\end{addmargin}
\end{absolutelynopagebreak}

\begin{absolutelynopagebreak}
\setstretch{.7}
{\PaliGlossA{nandikkhayo saṭṭhinayo,}}\\
\begin{addmargin}[1em]{2em}
\setstretch{.5}
{\PaliGlossB{    -}}\\
\end{addmargin}
\end{absolutelynopagebreak}

\begin{absolutelynopagebreak}
\setstretch{.7}
{\PaliGlossA{samuddo uragena ca;}}\\
\begin{addmargin}[1em]{2em}
\setstretch{.5}
{\PaliGlossB{    -}}\\
\end{addmargin}
\end{absolutelynopagebreak}

\begin{absolutelynopagebreak}
\setstretch{.7}
{\PaliGlossA{catupaṇṇāsakā ete,}}\\
\begin{addmargin}[1em]{2em}
\setstretch{.5}
{\PaliGlossB{    -}}\\
\end{addmargin}
\end{absolutelynopagebreak}

\begin{absolutelynopagebreak}
\setstretch{.7}
{\PaliGlossA{nipātesu pakāsitāti.}}\\
\begin{addmargin}[1em]{2em}
\setstretch{.5}
{\PaliGlossB{    -}}\\
\end{addmargin}
\end{absolutelynopagebreak}

\begin{absolutelynopagebreak}
\setstretch{.7}
{\PaliGlossA{saḷāyatanasaṃyuttaṃ samattaṃ.}}\\
\begin{addmargin}[1em]{2em}
\setstretch{.5}
{\PaliGlossB{The Linked Discourses on the six sense fields are complete.}}\\
\end{addmargin}
\end{absolutelynopagebreak}
