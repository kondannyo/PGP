
\begin{absolutelynopagebreak}
\setstretch{.7}
{\PaliGlossA{saṃyutta nikāya 35}}\\
\begin{addmargin}[1em]{2em}
\setstretch{.5}
{\PaliGlossB{Linked Discourses 35}}\\
\end{addmargin}
\end{absolutelynopagebreak}

\begin{absolutelynopagebreak}
\setstretch{.7}
{\PaliGlossA{8. gilānavagga}}\\
\begin{addmargin}[1em]{2em}
\setstretch{.5}
{\PaliGlossB{8. Sick}}\\
\end{addmargin}
\end{absolutelynopagebreak}

\begin{absolutelynopagebreak}
\setstretch{.7}
{\PaliGlossA{81. sambahulabhikkhusutta}}\\
\begin{addmargin}[1em]{2em}
\setstretch{.5}
{\PaliGlossB{81. Several Mendicants}}\\
\end{addmargin}
\end{absolutelynopagebreak}

\begin{absolutelynopagebreak}
\setstretch{.7}
{\PaliGlossA{atha kho sambahulā bhikkhū yena bhagavā tenupasaṅkamiṃsu … pe … ekamantaṃ nisinnā kho te bhikkhū bhagavantaṃ etadavocuṃ:}}\\
\begin{addmargin}[1em]{2em}
\setstretch{.5}
{\PaliGlossB{Then several mendicants went up to the Buddha … and asked him,}}\\
\end{addmargin}
\end{absolutelynopagebreak}

\begin{absolutelynopagebreak}
\setstretch{.7}
{\PaliGlossA{“idha no, bhante, aññatitthiyā paribbājakā amhe evaṃ pucchanti:}}\\
\begin{addmargin}[1em]{2em}
\setstretch{.5}
{\PaliGlossB{“Sir, sometimes wanderers who follow other paths ask us:}}\\
\end{addmargin}
\end{absolutelynopagebreak}

\begin{absolutelynopagebreak}
\setstretch{.7}
{\PaliGlossA{‘kimatthiyaṃ, āvuso, samaṇe gotame brahmacariyaṃ vussatī’ti?}}\\
\begin{addmargin}[1em]{2em}
\setstretch{.5}
{\PaliGlossB{‘Reverends, what’s the purpose of living the spiritual life with the ascetic Gotama?’}}\\
\end{addmargin}
\end{absolutelynopagebreak}

\begin{absolutelynopagebreak}
\setstretch{.7}
{\PaliGlossA{evaṃ puṭṭhā mayaṃ, bhante, tesaṃ aññatitthiyānaṃ paribbājakānaṃ evaṃ byākaroma:}}\\
\begin{addmargin}[1em]{2em}
\setstretch{.5}
{\PaliGlossB{We answer them like this:}}\\
\end{addmargin}
\end{absolutelynopagebreak}

\begin{absolutelynopagebreak}
\setstretch{.7}
{\PaliGlossA{‘dukkhassa kho, āvuso, pariññatthaṃ bhagavati brahmacariyaṃ vussatī’ti.}}\\
\begin{addmargin}[1em]{2em}
\setstretch{.5}
{\PaliGlossB{‘The purpose of living the spiritual life under the Buddha is to completely understand suffering.’}}\\
\end{addmargin}
\end{absolutelynopagebreak}

\begin{absolutelynopagebreak}
\setstretch{.7}
{\PaliGlossA{kacci mayaṃ, bhante, evaṃ puṭṭhā evaṃ byākaramānā vuttavādino ceva bhagavato homa, na ca bhagavantaṃ abhūtena abbhācikkhāma, dhammassa cānudhammaṃ byākaroma, na ca koci sahadhammiko vādānuvādo gārayhaṃ ṭhānaṃ āgacchatī”ti?}}\\
\begin{addmargin}[1em]{2em}
\setstretch{.5}
{\PaliGlossB{Answering this way, we trust that we repeat what the Buddha has said, and don’t misrepresent him with an untruth. We trust our explanation is in line with the teaching, and that there are no legitimate grounds for rebuke or criticism.”}}\\
\end{addmargin}
\end{absolutelynopagebreak}

\begin{absolutelynopagebreak}
\setstretch{.7}
{\PaliGlossA{“taggha tumhe, bhikkhave, evaṃ puṭṭhā evaṃ byākaramānā vuttavādino ceva me hotha, na ca maṃ abhūtena abbhācikkhatha, dhammassa cānudhammaṃ byākarotha, na ca koci sahadhammiko vādānuvādo gārayhaṃ ṭhānaṃ āgacchati.}}\\
\begin{addmargin}[1em]{2em}
\setstretch{.5}
{\PaliGlossB{“Indeed, in answering this way you repeat what I’ve said, and don’t misrepresent me with an untruth. Your explanation is in line with the teaching, and there are no legitimate grounds for rebuke or criticism.}}\\
\end{addmargin}
\end{absolutelynopagebreak}

\begin{absolutelynopagebreak}
\setstretch{.7}
{\PaliGlossA{dukkhassa hi, bhikkhave, pariññatthaṃ mayi brahmacariyaṃ vussati.}}\\
\begin{addmargin}[1em]{2em}
\setstretch{.5}
{\PaliGlossB{For the purpose of living the spiritual life under me is to completely understand suffering.}}\\
\end{addmargin}
\end{absolutelynopagebreak}

\begin{absolutelynopagebreak}
\setstretch{.7}
{\PaliGlossA{sace pana vo, bhikkhave, aññatitthiyā paribbājakā evaṃ puccheyyuṃ:}}\\
\begin{addmargin}[1em]{2em}
\setstretch{.5}
{\PaliGlossB{If wanderers who follow other paths were to ask you:}}\\
\end{addmargin}
\end{absolutelynopagebreak}

\begin{absolutelynopagebreak}
\setstretch{.7}
{\PaliGlossA{‘katamaṃ pana taṃ, āvuso, dukkhaṃ, yassa pariññāya samaṇe gotame brahmacariyaṃ vussatī’ti?}}\\
\begin{addmargin}[1em]{2em}
\setstretch{.5}
{\PaliGlossB{‘Reverends, what is that suffering?’}}\\
\end{addmargin}
\end{absolutelynopagebreak}

\begin{absolutelynopagebreak}
\setstretch{.7}
{\PaliGlossA{evaṃ puṭṭhā tumhe, bhikkhave, tesaṃ aññatitthiyānaṃ paribbājakānaṃ evaṃ byākareyyātha:}}\\
\begin{addmargin}[1em]{2em}
\setstretch{.5}
{\PaliGlossB{You should answer them:}}\\
\end{addmargin}
\end{absolutelynopagebreak}

\begin{absolutelynopagebreak}
\setstretch{.7}
{\PaliGlossA{‘cakkhu kho, āvuso, dukkhaṃ, tassa pariññāya bhagavati brahmacariyaṃ vussati.}}\\
\begin{addmargin}[1em]{2em}
\setstretch{.5}
{\PaliGlossB{‘Reverends, the eye is suffering. The purpose of living the spiritual life under the Buddha is to completely understand this.}}\\
\end{addmargin}
\end{absolutelynopagebreak}

\begin{absolutelynopagebreak}
\setstretch{.7}
{\PaliGlossA{rūpā … pe …}}\\
\begin{addmargin}[1em]{2em}
\setstretch{.5}
{\PaliGlossB{Sights … Eye consciousness … Eye contact …}}\\
\end{addmargin}
\end{absolutelynopagebreak}

\begin{absolutelynopagebreak}
\setstretch{.7}
{\PaliGlossA{yampidaṃ cakkhusamphassapaccayā uppajjati vedayitaṃ sukhaṃ vā dukkhaṃ vā adukkhamasukhaṃ vā tampi dukkhaṃ.}}\\
\begin{addmargin}[1em]{2em}
\setstretch{.5}
{\PaliGlossB{The pleasant, painful, or neutral feeling that arises conditioned by eye contact is also suffering.}}\\
\end{addmargin}
\end{absolutelynopagebreak}

\begin{absolutelynopagebreak}
\setstretch{.7}
{\PaliGlossA{tassa pariññāya bhagavati brahmacariyaṃ vussati … pe …}}\\
\begin{addmargin}[1em]{2em}
\setstretch{.5}
{\PaliGlossB{The purpose of living the spiritual life under the Buddha is to completely understand this.}}\\
\end{addmargin}
\end{absolutelynopagebreak}

\begin{absolutelynopagebreak}
\setstretch{.7}
{\PaliGlossA{mano dukkho … pe …}}\\
\begin{addmargin}[1em]{2em}
\setstretch{.5}
{\PaliGlossB{Ear … Nose … Tongue … Body … Mind …}}\\
\end{addmargin}
\end{absolutelynopagebreak}

\begin{absolutelynopagebreak}
\setstretch{.7}
{\PaliGlossA{yampidaṃ manosamphassapaccayā uppajjati vedayitaṃ sukhaṃ vā dukkhaṃ vā adukkhamasukhaṃ vā tampi dukkhaṃ.}}\\
\begin{addmargin}[1em]{2em}
\setstretch{.5}
{\PaliGlossB{The pleasant, painful, or neutral feeling that arises conditioned by mind contact is also suffering.}}\\
\end{addmargin}
\end{absolutelynopagebreak}

\begin{absolutelynopagebreak}
\setstretch{.7}
{\PaliGlossA{tassa pariññāya bhagavati brahmacariyaṃ vussati.}}\\
\begin{addmargin}[1em]{2em}
\setstretch{.5}
{\PaliGlossB{The purpose of living the spiritual life under the Buddha is to completely understand this.}}\\
\end{addmargin}
\end{absolutelynopagebreak}

\begin{absolutelynopagebreak}
\setstretch{.7}
{\PaliGlossA{idaṃ kho taṃ, āvuso, dukkhaṃ, tassa pariññāya bhagavati brahmacariyaṃ vussatī’ti.}}\\
\begin{addmargin}[1em]{2em}
\setstretch{.5}
{\PaliGlossB{This is that suffering. The purpose of living the spiritual life under the Buddha is to completely understand this.’}}\\
\end{addmargin}
\end{absolutelynopagebreak}

\begin{absolutelynopagebreak}
\setstretch{.7}
{\PaliGlossA{evaṃ puṭṭhā tumhe, bhikkhave, tesaṃ aññatitthiyānaṃ paribbājakānaṃ evaṃ byākareyyāthā”ti.}}\\
\begin{addmargin}[1em]{2em}
\setstretch{.5}
{\PaliGlossB{When questioned by wanderers who follow other paths, that’s how you should answer them.”}}\\
\end{addmargin}
\end{absolutelynopagebreak}

\begin{absolutelynopagebreak}
\setstretch{.7}
{\PaliGlossA{aṭṭhamaṃ.}}\\
\begin{addmargin}[1em]{2em}
\setstretch{.5}
{\PaliGlossB{    -}}\\
\end{addmargin}
\end{absolutelynopagebreak}
