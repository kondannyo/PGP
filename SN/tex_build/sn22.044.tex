
\begin{absolutelynopagebreak}
\setstretch{.7}
{\PaliGlossA{saṃyutta nikāya 22}}\\
\begin{addmargin}[1em]{2em}
\setstretch{.5}
{\PaliGlossB{Linked Discourses 22}}\\
\end{addmargin}
\end{absolutelynopagebreak}

\begin{absolutelynopagebreak}
\setstretch{.7}
{\PaliGlossA{5. attadīpavagga}}\\
\begin{addmargin}[1em]{2em}
\setstretch{.5}
{\PaliGlossB{5. Be Your Own Island}}\\
\end{addmargin}
\end{absolutelynopagebreak}

\begin{absolutelynopagebreak}
\setstretch{.7}
{\PaliGlossA{44. paṭipadāsutta}}\\
\begin{addmargin}[1em]{2em}
\setstretch{.5}
{\PaliGlossB{44. Practice}}\\
\end{addmargin}
\end{absolutelynopagebreak}

\begin{absolutelynopagebreak}
\setstretch{.7}
{\PaliGlossA{sāvatthinidānaṃ.}}\\
\begin{addmargin}[1em]{2em}
\setstretch{.5}
{\PaliGlossB{At Sāvatthī.}}\\
\end{addmargin}
\end{absolutelynopagebreak}

\begin{absolutelynopagebreak}
\setstretch{.7}
{\PaliGlossA{“sakkāyasamudayagāminiñca vo, bhikkhave, paṭipadaṃ desessāmi, sakkāyanirodhagāminiñca paṭipadaṃ.}}\\
\begin{addmargin}[1em]{2em}
\setstretch{.5}
{\PaliGlossB{“Mendicants, I will teach you the practice that leads to the origin of identity and the practice that leads to the cessation of identity.}}\\
\end{addmargin}
\end{absolutelynopagebreak}

\begin{absolutelynopagebreak}
\setstretch{.7}
{\PaliGlossA{taṃ suṇātha.}}\\
\begin{addmargin}[1em]{2em}
\setstretch{.5}
{\PaliGlossB{Listen …}}\\
\end{addmargin}
\end{absolutelynopagebreak}

\begin{absolutelynopagebreak}
\setstretch{.7}
{\PaliGlossA{katamā ca, bhikkhave, sakkāyasamudayagāminī paṭipadā?}}\\
\begin{addmargin}[1em]{2em}
\setstretch{.5}
{\PaliGlossB{And what is the practice that leads to the origin of identity?}}\\
\end{addmargin}
\end{absolutelynopagebreak}

\begin{absolutelynopagebreak}
\setstretch{.7}
{\PaliGlossA{idha, bhikkhave, assutavā puthujjano ariyānaṃ adassāvī ariyadhammassa akovido ariyadhamme avinīto, sappurisānaṃ adassāvī sappurisadhammassa akovido sappurisadhamme avinīto,}}\\
\begin{addmargin}[1em]{2em}
\setstretch{.5}
{\PaliGlossB{It’s when an uneducated ordinary person has not seen the noble ones, and is neither skilled nor trained in the teaching of the noble ones. They’ve not seen good persons, and are neither skilled nor trained in the teaching of the good persons.}}\\
\end{addmargin}
\end{absolutelynopagebreak}

\begin{absolutelynopagebreak}
\setstretch{.7}
{\PaliGlossA{rūpaṃ attato samanupassati, rūpavantaṃ vā attānaṃ; attani vā rūpaṃ, rūpasmiṃ vā attānaṃ.}}\\
\begin{addmargin}[1em]{2em}
\setstretch{.5}
{\PaliGlossB{They regard form as self, self as having form, form in self, or self in form.}}\\
\end{addmargin}
\end{absolutelynopagebreak}

\begin{absolutelynopagebreak}
\setstretch{.7}
{\PaliGlossA{vedanaṃ attato …}}\\
\begin{addmargin}[1em]{2em}
\setstretch{.5}
{\PaliGlossB{They regard feeling as self …}}\\
\end{addmargin}
\end{absolutelynopagebreak}

\begin{absolutelynopagebreak}
\setstretch{.7}
{\PaliGlossA{saññaṃ …}}\\
\begin{addmargin}[1em]{2em}
\setstretch{.5}
{\PaliGlossB{They regard perception as self …}}\\
\end{addmargin}
\end{absolutelynopagebreak}

\begin{absolutelynopagebreak}
\setstretch{.7}
{\PaliGlossA{saṅkhāre …}}\\
\begin{addmargin}[1em]{2em}
\setstretch{.5}
{\PaliGlossB{They regard choices as self …}}\\
\end{addmargin}
\end{absolutelynopagebreak}

\begin{absolutelynopagebreak}
\setstretch{.7}
{\PaliGlossA{viññāṇaṃ attato samanupassati, viññāṇavantaṃ vā attānaṃ; attani vā viññāṇaṃ, viññāṇasmiṃ vā attānaṃ.}}\\
\begin{addmargin}[1em]{2em}
\setstretch{.5}
{\PaliGlossB{They regard consciousness as self, self as having consciousness, consciousness in self, or self in consciousness.}}\\
\end{addmargin}
\end{absolutelynopagebreak}

\begin{absolutelynopagebreak}
\setstretch{.7}
{\PaliGlossA{ayaṃ vuccati, bhikkhave, ‘sakkāyasamudayagāminī paṭipadā, sakkāyasamudayagāminī paṭipadā’ti.}}\\
\begin{addmargin}[1em]{2em}
\setstretch{.5}
{\PaliGlossB{This is called the practice that leads to the origin of identity.}}\\
\end{addmargin}
\end{absolutelynopagebreak}

\begin{absolutelynopagebreak}
\setstretch{.7}
{\PaliGlossA{iti hidaṃ, bhikkhave, vuccati ‘dukkhasamudayagāminī samanupassanā’ti.}}\\
\begin{addmargin}[1em]{2em}
\setstretch{.5}
{\PaliGlossB{And that’s why it’s called a way of regarding things that leads to the origin of suffering.}}\\
\end{addmargin}
\end{absolutelynopagebreak}

\begin{absolutelynopagebreak}
\setstretch{.7}
{\PaliGlossA{ayamevettha attho.}}\\
\begin{addmargin}[1em]{2em}
\setstretch{.5}
{\PaliGlossB{    -}}\\
\end{addmargin}
\end{absolutelynopagebreak}

\begin{absolutelynopagebreak}
\setstretch{.7}
{\PaliGlossA{katamā ca, bhikkhave, sakkāyanirodhagāminī paṭipadā?}}\\
\begin{addmargin}[1em]{2em}
\setstretch{.5}
{\PaliGlossB{And what is the practice that leads to the cessation of identity?}}\\
\end{addmargin}
\end{absolutelynopagebreak}

\begin{absolutelynopagebreak}
\setstretch{.7}
{\PaliGlossA{idha, bhikkhave, sutavā ariyasāvako ariyānaṃ dassāvī ariyadhammassa kovido ariyadhamme suvinīto, sappurisānaṃ dassāvī sappurisadhammassa kovido sappurisadhamme suvinīto,}}\\
\begin{addmargin}[1em]{2em}
\setstretch{.5}
{\PaliGlossB{It’s when an educated noble disciple has seen the noble ones, and is skilled and trained in the teaching of the noble ones. They’ve seen good persons, and are skilled and trained in the teaching of the good persons.}}\\
\end{addmargin}
\end{absolutelynopagebreak}

\begin{absolutelynopagebreak}
\setstretch{.7}
{\PaliGlossA{na rūpaṃ attato samanupassati, na rūpavantaṃ vā attānaṃ; na attani vā rūpaṃ, na rūpasmiṃ vā attānaṃ.}}\\
\begin{addmargin}[1em]{2em}
\setstretch{.5}
{\PaliGlossB{They don’t regard form as self, self as having form, form in self, or self in form.}}\\
\end{addmargin}
\end{absolutelynopagebreak}

\begin{absolutelynopagebreak}
\setstretch{.7}
{\PaliGlossA{na vedanaṃ attato …}}\\
\begin{addmargin}[1em]{2em}
\setstretch{.5}
{\PaliGlossB{They don’t regard feeling as self …}}\\
\end{addmargin}
\end{absolutelynopagebreak}

\begin{absolutelynopagebreak}
\setstretch{.7}
{\PaliGlossA{na saññaṃ …}}\\
\begin{addmargin}[1em]{2em}
\setstretch{.5}
{\PaliGlossB{They don’t regard perception as self …}}\\
\end{addmargin}
\end{absolutelynopagebreak}

\begin{absolutelynopagebreak}
\setstretch{.7}
{\PaliGlossA{na saṅkhāre …}}\\
\begin{addmargin}[1em]{2em}
\setstretch{.5}
{\PaliGlossB{They don’t regard choices as self …}}\\
\end{addmargin}
\end{absolutelynopagebreak}

\begin{absolutelynopagebreak}
\setstretch{.7}
{\PaliGlossA{na viññāṇaṃ attato samanupassati, na viññāṇavantaṃ vā attānaṃ; na attani vā viññāṇaṃ, na viññāṇasmiṃ vā attānaṃ.}}\\
\begin{addmargin}[1em]{2em}
\setstretch{.5}
{\PaliGlossB{They don’t regard consciousness as self, self as having consciousness, consciousness in self, or self in consciousness.}}\\
\end{addmargin}
\end{absolutelynopagebreak}

\begin{absolutelynopagebreak}
\setstretch{.7}
{\PaliGlossA{ayaṃ vuccati, bhikkhave, ‘sakkāyanirodhagāminī paṭipadā, sakkāyanirodhagāminī paṭipadā’ti.}}\\
\begin{addmargin}[1em]{2em}
\setstretch{.5}
{\PaliGlossB{This is called the practice that leads to the cessation of identity.}}\\
\end{addmargin}
\end{absolutelynopagebreak}

\begin{absolutelynopagebreak}
\setstretch{.7}
{\PaliGlossA{iti hidaṃ, bhikkhave, vuccati ‘dukkhanirodhagāminī samanupassanā’ti.}}\\
\begin{addmargin}[1em]{2em}
\setstretch{.5}
{\PaliGlossB{And that’s why it’s called a way of regarding things that leads to the cessation of suffering.”}}\\
\end{addmargin}
\end{absolutelynopagebreak}

\begin{absolutelynopagebreak}
\setstretch{.7}
{\PaliGlossA{ayamevettha attho”ti.}}\\
\begin{addmargin}[1em]{2em}
\setstretch{.5}
{\PaliGlossB{    -}}\\
\end{addmargin}
\end{absolutelynopagebreak}

\begin{absolutelynopagebreak}
\setstretch{.7}
{\PaliGlossA{dutiyaṃ.}}\\
\begin{addmargin}[1em]{2em}
\setstretch{.5}
{\PaliGlossB{    -}}\\
\end{addmargin}
\end{absolutelynopagebreak}
