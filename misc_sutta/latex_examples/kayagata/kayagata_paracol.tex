
\begin{samepage}
\begin{leftcolumn*}
\EnglishColumn{Majjhima Nikāya, uparipaṇṇāsapāḷi, 2. anupadavaggo, 9. kāyagatāsatisuttaṃ (MN 119)}
\hspace{0pt}\end{leftcolumn*}

\begin{rightcolumn}\PaliColumn{Mindfulness of the Body}
\hspace{0pt}
\end{rightcolumn}
\end{samepage}
\vskip 0.05in
\begin{samepage}
\begin{leftcolumn*}
\EnglishColumn{153. evaṁ me sutaṁ. ekaṁ samayaṁ bhagavā sāvatthiyaṁ viharati jetavane anāthapiṇḍikassa ārāme.}
\hspace{0pt}\end{leftcolumn*}

\begin{rightcolumn}\PaliColumn{Thus have I heard. On one occasion the Blessed One was living at S̄āvattī in Jeta’s Grove, Anāthapiṇḍika’s Park.}
\hspace{0pt}
\end{rightcolumn}
\end{samepage}
\begin{samepage}
\begin{leftcolumn*}
\EnglishColumn{atha kho sambahulānaṁ bhikkhūnaṁ pacchābhattaṁ piṇḍapātapaṭikkantānaṁ upaṭṭhānasālāyaṁ sannisinnānaṁ sannipatitānaṁ ayamantarākathā udapādi;}
\hspace{0pt}\end{leftcolumn*}

\begin{rightcolumn}\PaliColumn{Now a number of bhikkhus were sitting in the assembly hall, where they had met together on returning from their almsround, after their meal, when this discussion arose among them:}
\hspace{0pt}
\end{rightcolumn}
\end{samepage}
\begin{samepage}
\begin{leftcolumn*}
\EnglishColumn{“acchariyaṁ, āvuso, abbhutaṁ, āvuso! yāvañcidaṁ tena bhagavatā jānatā passatā arahatā sammāsambuddhena kāyagatāsati bhāvitā bahulīkatā mahapphalā vuttā mahānisaṁsā”ti.}
\hspace{0pt}\end{leftcolumn*}

\begin{rightcolumn}\PaliColumn{“It is wonderful, friends, it is marvellous, how it has been said by the Blessed One who knows and sees, accomplished and fully enlightened, that mindfulness of the body, when developed and cultivated, is of great fruit and great benefit.”}
\hspace{0pt}
\end{rightcolumn}
\end{samepage}
\begin{samepage}
\begin{leftcolumn*}
\EnglishColumn{ayañca hidaṁ tesaṁ bhikkhūnaṁ antarākathā vippakatā hoti, atha kho bhagavā sāyanhasamayaṁ paṭisallānā vuṭṭhito yena upaṭṭhānasālā tenupasaṅkami; upasaṅkamitvā paññatte āsane nisīdi.}
\hspace{0pt}\end{leftcolumn*}

\begin{rightcolumn}\PaliColumn{However, their discussion was interrupted; for the Blessed One rose from meditation when it was evening, went to the assembly hall, and sat down on a seat made ready.}
\hspace{0pt}
\end{rightcolumn}
\end{samepage}
\begin{samepage}
\begin{leftcolumn*}
\EnglishColumn{nisajja kho bhagavā bhikkhū āmantesi; “kāya nuttha, bhikkhave, etarahi kathāya sannisinnā, kā ca pana vo antarākathā vippakatā”ti?}
\hspace{0pt}\end{leftcolumn*}

\begin{rightcolumn}\PaliColumn{Then he addressed the bhikkhus thus: “Bhikkhus, for what discussion are you sitting together here now? And what was your discussion that was interrupted?”}
\hspace{0pt}
\end{rightcolumn}
\end{samepage}
\begin{samepage}
\begin{leftcolumn*}
\EnglishColumn{“idha, bhante, amhākaṁ pacchābhattaṁ piṇḍapātapaṭikkantānaṁ upaṭṭhānasālāyaṁ sannisinnānaṁ sannipatitānaṁ ayamantarākathā udapādi;}
\hspace{0pt}\end{leftcolumn*}

\begin{rightcolumn}\PaliColumn{“Here, venerable sir, we were sitting in the assembly hall, where we had met together on returning from our almsround, after our meal, when this discussion arose among us:}
\hspace{0pt}
\end{rightcolumn}
\end{samepage}
\begin{samepage}
\begin{leftcolumn*}
\EnglishColumn{‘acchariyaṁ, āvuso, abbhutaṁ, āvuso! yāvañcidaṁ tena bhagavatā jānatā passatā arahatā sammāsambuddhena kāyagatāsati bhāvitā bahulīkatā mahapphalā vuttā mahānisaṁsā’ti.}
\hspace{0pt}\end{leftcolumn*}

\begin{rightcolumn}\PaliColumn{‘It is wonderful, friends, it is marvellous, how it has been said by the Blessed One who knows and sees, accomplished and fully enlightened, that mindfulness of the body, when developed and cultivated, is of great fruit and great benefit.’}
\hspace{0pt}
\end{rightcolumn}
\end{samepage}
\begin{samepage}
\begin{leftcolumn*}
\EnglishColumn{ayaṁ kho no, bhante, antarākathā vippakatā, atha bhagavā anuppatto”ti.}
\hspace{0pt}\end{leftcolumn*}

\begin{rightcolumn}\PaliColumn{This was our discussion, venerable sir, that was interrupted when the Blessed One arrived.”}
\hspace{0pt}
\end{rightcolumn}
\end{samepage}
\vskip 0.05in
\begin{samepage}
\begin{leftcolumn*}
\EnglishColumn{154. “kathaṁ bhāvitā ca, bhikkhave, kāyagatāsati kathaṁ bahulīkatā mahapphalā hoti mahānisaṁsā?}
\hspace{0pt}\end{leftcolumn*}

\begin{rightcolumn}\PaliColumn{“And how, bhikkhus, is mindfulness of the body developed and cultivated so that it is of great fruit and great benefit?}
\hspace{0pt}
\end{rightcolumn}
\end{samepage}
\begin{samepage}
\begin{leftcolumn*}
\EnglishColumn{idha, bhikkhave, bhikkhu araññagato vā rukkhamūlagato vā suññāgāragato vā nisīdati pallaṅkaṁ ābhujitvā ujuṁ kāyaṁ paṇidhāya parimukhaṁ satiṁ upaṭṭhapetvā.}
\hspace{0pt}\end{leftcolumn*}

\begin{rightcolumn}\PaliColumn{“Here a bhikkhu, gone to the forest or to the root of a tree or to an empty hut, sits down; having folded his legs crosswise, set his body erect, and established mindfulness in front of him,}
\hspace{0pt}
\end{rightcolumn}
\end{samepage}
\begin{samepage}
\begin{leftcolumn*}
\EnglishColumn{so satova assasati satova passasati;}
\hspace{0pt}\end{leftcolumn*}

\begin{rightcolumn}\PaliColumn{ever mindful he breathes in, mindful he breathes out.}
\hspace{0pt}
\end{rightcolumn}
\end{samepage}
\begin{samepage}
\begin{leftcolumn*}
\EnglishColumn{dīghaṁ vā assasanto ‘dīghaṁ assasāmī’ti pajānāti,}
\hspace{0pt}\end{leftcolumn*}

\begin{rightcolumn}\PaliColumn{Breathing in long, he understands: ‘I breathe in long’;}
\hspace{0pt}
\end{rightcolumn}
\end{samepage}
\begin{samepage}
\begin{leftcolumn*}
\EnglishColumn{dīghaṁ vā passasanto ‘dīghaṁ passasāmī’ti pajānāti;}
\hspace{0pt}\end{leftcolumn*}

\begin{rightcolumn}\PaliColumn{or breathing out long, he understands: ‘I breathe out long.’}
\hspace{0pt}
\end{rightcolumn}
\end{samepage}
\begin{samepage}
\begin{leftcolumn*}
\EnglishColumn{rassaṁ vā assasanto ‘rassaṁ assasāmī’ti pajānāti,}
\hspace{0pt}\end{leftcolumn*}

\begin{rightcolumn}\PaliColumn{Breathing in short, he understands: ‘I breathe in short’;}
\hspace{0pt}
\end{rightcolumn}
\end{samepage}
\begin{samepage}
\begin{leftcolumn*}
\EnglishColumn{rassaṁ vā passasanto ‘rassaṁ passasāmī’ti pajānāti;}
\hspace{0pt}\end{leftcolumn*}

\begin{rightcolumn}\PaliColumn{or breathing out short, he understands: ‘I breathe out short.’}
\hspace{0pt}
\end{rightcolumn}
\end{samepage}
\begin{samepage}
\begin{leftcolumn*}
\EnglishColumn{‘sabbakāyapaṭisaṁvedī assasissāmī’ti sikkhati,}
\hspace{0pt}\end{leftcolumn*}

\begin{rightcolumn}\PaliColumn{He trains thus: ‘I shall breathe in experiencing the whole body’;}
\hspace{0pt}
\end{rightcolumn}
\end{samepage}
\begin{samepage}
\begin{leftcolumn*}
\EnglishColumn{‘sabbakāyapaṭisaṁvedī passasissāmī’ti sikkhati;}
\hspace{0pt}\end{leftcolumn*}

\begin{rightcolumn}\PaliColumn{he trains thus: ‘I shall breathe out experiencing the whole body.}
\hspace{0pt}
\end{rightcolumn}
\end{samepage}
\begin{samepage}
\begin{leftcolumn*}
\EnglishColumn{‘passambhayaṁ kāyasaṅkhāraṁ assasissāmī’ti sikkhati,}
\hspace{0pt}\end{leftcolumn*}

\begin{rightcolumn}\PaliColumn{He trains thus: ‘I shall breathe in tranquillising the bodily formation’;}
\hspace{0pt}
\end{rightcolumn}
\end{samepage}
\begin{samepage}
\begin{leftcolumn*}
\EnglishColumn{‘passambhayaṁ kāyasaṅkhāraṁ passasissāmī’ti sikkhati.}
\hspace{0pt}\end{leftcolumn*}

\begin{rightcolumn}\PaliColumn{he trains thus: ‘I shall breathe out tranquillising the bodily formation.’}
\hspace{0pt}
\end{rightcolumn}
\end{samepage}
\begin{samepage}
\begin{leftcolumn*}
\EnglishColumn{tassa evaṁ appamattassa ātāpino pahitattassa viharato ye gehasitā sarasaṅkappā te pahīyanti.}
\hspace{0pt}\end{leftcolumn*}

\begin{rightcolumn}\PaliColumn{As he abides thus diligent, ardent, and resolute, his memories and intentions based on the household life are abandoned;}
\hspace{0pt}
\end{rightcolumn}
\end{samepage}
\begin{samepage}
\begin{leftcolumn*}
\EnglishColumn{tesaṁ pahānā ajjhattameva cittaṁ santiṭṭhati sannisīdati ekodi hoti samādhiyati.}
\hspace{0pt}\end{leftcolumn*}

\begin{rightcolumn}\PaliColumn{with their abandoning his mind becomes steadied internally, quieted, brought to singleness, and concentrated.}
\hspace{0pt}
\end{rightcolumn}
\end{samepage}
\begin{samepage}
\begin{leftcolumn*}
\EnglishColumn{evaṁ, bhikkhave, bhikkhu kāyagatāsatiṁ bhāveti.}
\hspace{0pt}\end{leftcolumn*}

\begin{rightcolumn}\PaliColumn{That is how a bhikkhu develops mindfulness of the body.}
\hspace{0pt}
\end{rightcolumn}
\end{samepage}
\vskip 0.05in
\begin{samepage}
\begin{leftcolumn*}
\EnglishColumn{“puna caparaṁ, bhikkhave, bhikkhu gacchanto vā ‘gacchāmī’ti pajānāti,}
\hspace{0pt}\end{leftcolumn*}

\begin{rightcolumn}\PaliColumn{“Again, bhikkhus, when walking, a bhikkhu understands: ‘I am walking’;}
\hspace{0pt}
\end{rightcolumn}
\end{samepage}
\begin{samepage}
\begin{leftcolumn*}
\EnglishColumn{ṭhito vā ‘ṭhitomhī’ti pajānāti,}
\hspace{0pt}\end{leftcolumn*}

\begin{rightcolumn}\PaliColumn{when standing, he understands: ‘I am standing’;}
\hspace{0pt}
\end{rightcolumn}
\end{samepage}
\begin{samepage}
\begin{leftcolumn*}
\EnglishColumn{nisinno vā ‘nisinnomhī’ti pajānāti,}
\hspace{0pt}\end{leftcolumn*}

\begin{rightcolumn}\PaliColumn{when sitting, he understands: ‘I am sitting’;}
\hspace{0pt}
\end{rightcolumn}
\end{samepage}
\begin{samepage}
\begin{leftcolumn*}
\EnglishColumn{sayāno vā ‘sayānomhī’ti pajānāti.}
\hspace{0pt}\end{leftcolumn*}

\begin{rightcolumn}\PaliColumn{when lying down, he understands: ‘I am lying down’;}
\hspace{0pt}
\end{rightcolumn}
\end{samepage}
\begin{samepage}
\begin{leftcolumn*}
\EnglishColumn{yathā yathā vā panassa kāyo paṇihito hoti, tathā tathā naṁ pajānāti.}
\hspace{0pt}\end{leftcolumn*}

\begin{rightcolumn}\PaliColumn{or he understands accordingly however his body is disposed.}
\hspace{0pt}
\end{rightcolumn}
\end{samepage}
\begin{samepage}
\begin{leftcolumn*}
\EnglishColumn{tassa evaṁ appamattassa ātāpino pahitattassa viharato ye gehasitā sarasaṅkappā te pahīyanti.}
\hspace{0pt}\end{leftcolumn*}

\begin{rightcolumn}\PaliColumn{As he abides thus diligent, ardent, and resolute, his memories and intentions based on the household life are abandoned;}
\hspace{0pt}
\end{rightcolumn}
\end{samepage}
\begin{samepage}
\begin{leftcolumn*}
\EnglishColumn{tesaṁ pahānā ajjhattameva cittaṁ santiṭṭhati sannisīdati ekodi hoti samādhiyati.}
\hspace{0pt}\end{leftcolumn*}

\begin{rightcolumn}\PaliColumn{with their abandoning his mind becomes steadied internally, quieted, brought to singleness, and concentrated.}
\hspace{0pt}
\end{rightcolumn}
\end{samepage}
\begin{samepage}
\begin{leftcolumn*}
\EnglishColumn{evampi, bhikkhave, bhikkhu kāyagatāsatiṁ bhāveti.}
\hspace{0pt}\end{leftcolumn*}

\begin{rightcolumn}\PaliColumn{That too is how a bhikkhu develops mindfulness of the body.}
\hspace{0pt}
\end{rightcolumn}
\end{samepage}
\vskip 0.05in
\begin{samepage}
\begin{leftcolumn*}
\EnglishColumn{“puna caparaṁ, bhikkhave, bhikkhu abhikkante paṭikkante sampajānakārī hoti,}
\hspace{0pt}\end{leftcolumn*}

\begin{rightcolumn}\PaliColumn{“Again, bhikkhus, a bhikkhu is one who acts in full awareness when going forward and returning;}
\hspace{0pt}
\end{rightcolumn}
\end{samepage}
\begin{samepage}
\begin{leftcolumn*}
\EnglishColumn{ālokite vilokite sampajānakārī hoti,}
\hspace{0pt}\end{leftcolumn*}

\begin{rightcolumn}\PaliColumn{who acts in full awareness when looking ahead and looking away;}
\hspace{0pt}
\end{rightcolumn}
\end{samepage}
\begin{samepage}
\begin{leftcolumn*}
\EnglishColumn{samiñjite pasārite sampajānakārī hoti,}
\hspace{0pt}\end{leftcolumn*}

\begin{rightcolumn}\PaliColumn{who acts in full awareness when flexing and extending his limbs;}
\hspace{0pt}
\end{rightcolumn}
\end{samepage}
\begin{samepage}
\begin{leftcolumn*}
\EnglishColumn{saṅghāṭipattacīvaradhāraṇe sampajānakārī hoti,}
\hspace{0pt}\end{leftcolumn*}

\begin{rightcolumn}\PaliColumn{who acts in full awareness when wearing his robes and carrying his outer robe and bowl;}
\hspace{0pt}
\end{rightcolumn}
\end{samepage}
\begin{samepage}
\begin{leftcolumn*}
\EnglishColumn{asite pīte khāyite sāyite sampajānakārī hoti,}
\hspace{0pt}\end{leftcolumn*}

\begin{rightcolumn}\PaliColumn{who acts in full awareness when eating, drinking, consuming food, and tasting;}
\hspace{0pt}
\end{rightcolumn}
\end{samepage}
\begin{samepage}
\begin{leftcolumn*}
\EnglishColumn{uccārapassāvakamme sampajānakārī hoti,}
\hspace{0pt}\end{leftcolumn*}

\begin{rightcolumn}\PaliColumn{who acts in full awareness when defecating or urinating;}
\hspace{0pt}
\end{rightcolumn}
\end{samepage}
\begin{samepage}
\begin{leftcolumn*}
\EnglishColumn{gate ṭhite nisinne sutte jāgarite bhāsite tuṇhībhāve sampajānakārī hoti.}
\hspace{0pt}\end{leftcolumn*}

\begin{rightcolumn}\PaliColumn{who acts in full awareness when walking, standing, sitting, falling asleep, waking up, talking, and keeping silent.}
\hspace{0pt}
\end{rightcolumn}
\end{samepage}
\begin{samepage}
\begin{leftcolumn*}
\EnglishColumn{tassa evaṁ appamattassa ātāpino pahitattassa viharato ye gehasitā sarasaṅkappā te pahīyanti.}
\hspace{0pt}\end{leftcolumn*}

\begin{rightcolumn}\PaliColumn{As he abides thus diligent, ardent, and resolute, his memories and intentions based on the household life are abandoned;}
\hspace{0pt}
\end{rightcolumn}
\end{samepage}
\begin{samepage}
\begin{leftcolumn*}
\EnglishColumn{tesaṁ pahānā ajjhattameva cittaṁ santiṭṭhati sannisīdati ekodi hoti samādhiyati. evampi, bhikkhave, bhikkhu kāyagatāsatiṁ bhāveti.}
\hspace{0pt}\end{leftcolumn*}

\begin{rightcolumn}\PaliColumn{with their abandoning his mind becomes steadied internally, quieted, brought to singleness, and concentrated. That too is how a bhikkhu develops mindfulness of the body.}
\hspace{0pt}
\end{rightcolumn}
\end{samepage}
\vskip 0.05in
\begin{samepage}
\begin{leftcolumn*}
\EnglishColumn{“puna caparaṁ, bhikkhave, bhikkhu imameva kāyaṁ uddhaṁ pādatalā adho kesamatthakā tacapariyantaṁ pūraṁ nānappakārassa asucino paccavekkhati;}
\hspace{0pt}\end{leftcolumn*}

\begin{rightcolumn}\PaliColumn{“Again, bhikkhus, a bhikkhu reviews this same body up from the soles of the feet and down from the top of the hair, bounded by skin, as full of many kinds of impurity thus:}
\hspace{0pt}
\end{rightcolumn}
\end{samepage}
\begin{samepage}
\begin{leftcolumn*}
\EnglishColumn{‘atthi imasmiṁ kāye}
\hspace{0pt}\end{leftcolumn*}

\begin{rightcolumn}\PaliColumn{‘In this body there are}
\hspace{0pt}
\end{rightcolumn}
\end{samepage}
\begin{samepage}
\begin{leftcolumn*}
\EnglishColumn{kesā lomā nakhā dantā taco}
\hspace{0pt}\end{leftcolumn*}

\begin{rightcolumn}\PaliColumn{head-hairs, body-hairs, nails, teeth, skin,}
\hspace{0pt}
\end{rightcolumn}
\end{samepage}
\begin{samepage}
\begin{leftcolumn*}
\EnglishColumn{maṁsaṁ nhāru aṭṭhi aṭṭhimiñjaṁ vakkaṁ}
\hspace{0pt}\end{leftcolumn*}

\begin{rightcolumn}\PaliColumn{flesh, sinews, bones, bone-marrow, kidneys,}
\hspace{0pt}
\end{rightcolumn}
\end{samepage}
\begin{samepage}
\begin{leftcolumn*}
\EnglishColumn{hadayaṁ yakanaṁ kilomakaṁ pihakaṁ papphāsaṁ}
\hspace{0pt}\end{leftcolumn*}

\begin{rightcolumn}\PaliColumn{heart, liver, diaphragm, spleen, lungs,}
\hspace{0pt}
\end{rightcolumn}
\end{samepage}
\begin{samepage}
\begin{leftcolumn*}
\EnglishColumn{antaṁ antaguṇaṁ udariyaṁ karīsaṁ pittaṁ}
\hspace{0pt}\end{leftcolumn*}

\begin{rightcolumn}\PaliColumn{intestines, mesentery, contents of the stomach, feces, bile,}
\hspace{0pt}
\end{rightcolumn}
\end{samepage}
\begin{samepage}
\begin{leftcolumn*}
\EnglishColumn{semhaṁ pubbo lohitaṁ sedo medo assu}
\hspace{0pt}\end{leftcolumn*}

\begin{rightcolumn}\PaliColumn{phlegm, pus, blood, sweat, fat, tears,}
\hspace{0pt}
\end{rightcolumn}
\end{samepage}
\begin{samepage}
\begin{leftcolumn*}
\EnglishColumn{vasā kheḷo siṅghāṇikā lasikā muttan’ti.}
\hspace{0pt}\end{leftcolumn*}

\begin{rightcolumn}\PaliColumn{grease, spittle, snot, oil of the joints, and urine.’}
\hspace{0pt}
\end{rightcolumn}
\end{samepage}
\begin{samepage}
\begin{leftcolumn*}
\EnglishColumn{“seyyathāpi, bhikkhave, ubhatomukhā putoḷi pūrā nānāvihitassa dhaññassa, seyyathidaṁ;}
\hspace{0pt}\end{leftcolumn*}

\begin{rightcolumn}\PaliColumn{Just as though there were a bag with an opening at both ends full of many sorts of grain, such as}
\hspace{0pt}
\end{rightcolumn}
\end{samepage}
\begin{samepage}
\begin{leftcolumn*}
\EnglishColumn{sālīnaṁ vīhīnaṁ muggānaṁ māsānaṁ tilānaṁ taṇḍulānaṁ, tamenaṁ cakkhumā puriso muñcitvā paccavekkheyya;}
\hspace{0pt}\end{leftcolumn*}

\begin{rightcolumn}\PaliColumn{hill rice, red rice, beans, peas, millet, and white rice, and a man with good eyes were to open it and review it thus:}
\hspace{0pt}
\end{rightcolumn}
\end{samepage}
\begin{samepage}
\begin{leftcolumn*}
\EnglishColumn{‘ime sālī ime vīhī ime muggā ime māsā ime tilā ime taṇḍulā’ti;}
\hspace{0pt}\end{leftcolumn*}

\begin{rightcolumn}\PaliColumn{‘This is hill rice, this is red rice, these are beans, these are peas, this is millet, this is white rice’;}
\hspace{0pt}
\end{rightcolumn}
\end{samepage}
\begin{samepage}
\begin{leftcolumn*}
\EnglishColumn{evameva kho, bhikkhave, bhikkhu imameva kāyaṁ uddhaṁ pādatalā adho kesamatthakā tacapariyantaṁ pūraṁ nānappakārassa asucino paccavekkhati;}
\hspace{0pt}\end{leftcolumn*}

\begin{rightcolumn}\PaliColumn{so too, a bhikkhu reviews this same body as full of many kinds of impurity thus:}
\hspace{0pt}
\end{rightcolumn}
\end{samepage}
\begin{samepage}
\begin{leftcolumn*}
\EnglishColumn{‘atthi imasmiṁ kāye}
\hspace{0pt}\end{leftcolumn*}

\begin{rightcolumn}\PaliColumn{‘In this body there are}
\hspace{0pt}
\end{rightcolumn}
\end{samepage}
\begin{samepage}
\begin{leftcolumn*}
\EnglishColumn{kesā lomā nakhā dantā taco}
\hspace{0pt}\end{leftcolumn*}

\begin{rightcolumn}\PaliColumn{head-hairs, body-hairs, nails, teeth, skin,}
\hspace{0pt}
\end{rightcolumn}
\end{samepage}
\begin{samepage}
\begin{leftcolumn*}
\EnglishColumn{maṁsaṁ nhāru aṭṭhi aṭṭhimiñjaṁ vakkaṁ}
\hspace{0pt}\end{leftcolumn*}

\begin{rightcolumn}\PaliColumn{flesh, sinews, bones, bone-marrow, kidneys,}
\hspace{0pt}
\end{rightcolumn}
\end{samepage}
\begin{samepage}
\begin{leftcolumn*}
\EnglishColumn{hadayaṁ yakanaṁ kilomakaṁ pihakaṁ papphāsaṁ}
\hspace{0pt}\end{leftcolumn*}

\begin{rightcolumn}\PaliColumn{heart, liver, diaphragm, spleen, lungs,}
\hspace{0pt}
\end{rightcolumn}
\end{samepage}
\begin{samepage}
\begin{leftcolumn*}
\EnglishColumn{antaṁ antaguṇaṁ udariyaṁ karīsaṁ pittaṁ}
\hspace{0pt}\end{leftcolumn*}

\begin{rightcolumn}\PaliColumn{intestines, mesentery, contents of the stomach, feces, bile,}
\hspace{0pt}
\end{rightcolumn}
\end{samepage}
\begin{samepage}
\begin{leftcolumn*}
\EnglishColumn{semhaṁ pubbo lohitaṁ sedo medo assu}
\hspace{0pt}\end{leftcolumn*}

\begin{rightcolumn}\PaliColumn{phlegm, pus, blood, sweat, fat, tears,}
\hspace{0pt}
\end{rightcolumn}
\end{samepage}
\begin{samepage}
\begin{leftcolumn*}
\EnglishColumn{vasā kheḷo siṅghāṇikā lasikā muttan’ti.}
\hspace{0pt}\end{leftcolumn*}

\begin{rightcolumn}\PaliColumn{grease, spittle, snot, oil of the joints, and urine.’}
\hspace{0pt}
\end{rightcolumn}
\end{samepage}
\begin{samepage}
\begin{leftcolumn*}
\EnglishColumn{tassa evaṁ appamattassa ātāpino pahitattassa viharato ye gehasitā sarasaṅkappā te pahīyanti.}
\hspace{0pt}\end{leftcolumn*}

\begin{rightcolumn}\PaliColumn{As he abides thus diligent, ardent, and resolute, his memories and intentions based on the household life are abandoned;}
\hspace{0pt}
\end{rightcolumn}
\end{samepage}
\begin{samepage}
\begin{leftcolumn*}
\EnglishColumn{tesaṁ pahānā ajjhattameva cittaṁ santiṭṭhati sannisīdati ekodi hoti samādhiyati.}
\hspace{0pt}\end{leftcolumn*}

\begin{rightcolumn}\PaliColumn{with their abandoning his mind becomes steadied internally, quieted, brought to singleness, and concentrated.}
\hspace{0pt}
\end{rightcolumn}
\end{samepage}
\begin{samepage}
\begin{leftcolumn*}
\EnglishColumn{evampi, bhikkhave, bhikkhu kāyagatāsatiṁ bhāveti.}
\hspace{0pt}\end{leftcolumn*}

\begin{rightcolumn}\PaliColumn{That too is how a bhikkhu develops mindfulness of the body.}
\hspace{0pt}
\end{rightcolumn}
\end{samepage}
\vskip 0.05in
\begin{samepage}
\begin{leftcolumn*}
\EnglishColumn{“puna caparaṁ, bhikkhave, bhikkhu imameva kāyaṁ yathāṭhitaṁ yathāpaṇihitaṁ dhātuso paccavekkhati;}
\hspace{0pt}\end{leftcolumn*}

\begin{rightcolumn}\PaliColumn{“Again, bhikkhus, a bhikkhu reviews this same body, however it is placed, however disposed, as consisting of elements thus:}
\hspace{0pt}
\end{rightcolumn}
\end{samepage}
\begin{samepage}
\begin{leftcolumn*}
\EnglishColumn{‘atthi imasmiṁ kāye pathavīdhātu āpodhātu tejodhātu vāyodhātū’ti.}
\hspace{0pt}\end{leftcolumn*}

\begin{rightcolumn}\PaliColumn{‘In this body there are the earth element, the water element, the fire element, and the air element.’}
\hspace{0pt}
\end{rightcolumn}
\end{samepage}
\begin{samepage}
\begin{leftcolumn*}
\EnglishColumn{“seyyathāpi, bhikkhave, dakkho goghātako vā goghātakantevāsī vā gāviṁ vadhitvā catumahāpathe bilaso vibhajitvā nisinno assa;}
\hspace{0pt}\end{leftcolumn*}

\begin{rightcolumn}\PaliColumn{Just as though a skilled butcher or his apprentice had killed a cow and were seated at the crossroads with it cut up into pieces;}
\hspace{0pt}
\end{rightcolumn}
\end{samepage}
\begin{samepage}
\begin{leftcolumn*}
\EnglishColumn{evameva kho, bhikkhave, bhikkhu imameva kāyaṁ yathāṭhitaṁ yathāpaṇihitaṁ dhātuso paccavekkhati;}
\hspace{0pt}\end{leftcolumn*}

\begin{rightcolumn}\PaliColumn{so too, a bhikkhu reviews this same body however it is placed, however disposed, as consisting of elements thus:}
\hspace{0pt}
\end{rightcolumn}
\end{samepage}
\begin{samepage}
\begin{leftcolumn*}
\EnglishColumn{‘atthi imasmiṁ kāye pathavīdhātu āpodhātu tejodhātu vāyodhātū’ti.}
\hspace{0pt}\end{leftcolumn*}

\begin{rightcolumn}\PaliColumn{‘In this body there are the earth element, the water element, the fire element, and the air element.’}
\hspace{0pt}
\end{rightcolumn}
\end{samepage}
\begin{samepage}
\begin{leftcolumn*}
\EnglishColumn{tassa evaṁ appamattassa ātāpino pahitattassa viharato ye gehasitā sarasaṅkappā te pahīyanti.}
\hspace{0pt}\end{leftcolumn*}

\begin{rightcolumn}\PaliColumn{As he abides thus diligent, ardent, and resolute, his memories and intentions connected with the household life are abandoned;}
\hspace{0pt}
\end{rightcolumn}
\end{samepage}
\begin{samepage}
\begin{leftcolumn*}
\EnglishColumn{tesaṁ pahānā ajjhattameva cittaṁ santiṭṭhati sannisīdati ekodi hoti samādhiyati.}
\hspace{0pt}\end{leftcolumn*}

\begin{rightcolumn}\PaliColumn{with their abandoning his mind becomes steadied internally, quieted, brought to singleness, and concentrated.}
\hspace{0pt}
\end{rightcolumn}
\end{samepage}
\begin{samepage}
\begin{leftcolumn*}
\EnglishColumn{evampi, bhikkhave, bhikkhu kāyagatāsatiṁ bhāveti.}
\hspace{0pt}\end{leftcolumn*}

\begin{rightcolumn}\PaliColumn{That too is how a bhikkhu develops mindfulness of the body.}
\hspace{0pt}
\end{rightcolumn}
\end{samepage}
\vskip 0.05in
\begin{samepage}
\begin{leftcolumn*}
\EnglishColumn{“puna caparaṁ, bhikkhave, bhikkhu seyyathāpi passeyya sarīraṁ sivathikāya chaḍḍitaṁ ekāhamataṁ vā dvīhamataṁ vā tīhamataṁ vā uddhumātakaṁ vinīlakaṁ vipubbakajātaṁ.}
\hspace{0pt}\end{leftcolumn*}

\begin{rightcolumn}\PaliColumn{“Again, bhikkhus, as though he were to see a corpse thrown aside in a charnel ground, one, two, or three days dead, bloated, livid, and oozing matter,}
\hspace{0pt}
\end{rightcolumn}
\end{samepage}
\begin{samepage}
\begin{leftcolumn*}
\EnglishColumn{so imameva kāyaṁ upasaṁharati;}
\hspace{0pt}\end{leftcolumn*}

\begin{rightcolumn}\PaliColumn{a bhikkhu compares this same body with it thus:}
\hspace{0pt}
\end{rightcolumn}
\end{samepage}
\begin{samepage}
\begin{leftcolumn*}
\EnglishColumn{‘ayampi kho kāyo evaṁdhammo evaṁbhāvī evaṁanatīto’ti.}
\hspace{0pt}\end{leftcolumn*}

\begin{rightcolumn}\PaliColumn{‘This body too is of the same nature, it will be like that, it is not exempt from that fate.’}
\hspace{0pt}
\end{rightcolumn}
\end{samepage}
\begin{samepage}
\begin{leftcolumn*}
\EnglishColumn{tassa evaṁ appamattassa ātāpino pahitattassa viharato ye gehasitā sarasaṅkappā te pahīyanti.}
\hspace{0pt}\end{leftcolumn*}

\begin{rightcolumn}\PaliColumn{As he abides thus diligent, ardent, and resolute, his memories and intentions connected with the household life are abandoned;}
\hspace{0pt}
\end{rightcolumn}
\end{samepage}
\begin{samepage}
\begin{leftcolumn*}
\EnglishColumn{tesaṁ pahānā ajjhattameva cittaṁ santiṭṭhati sannisīdati ekodi hoti samādhiyati.}
\hspace{0pt}\end{leftcolumn*}

\begin{rightcolumn}\PaliColumn{with their abandoning his mind becomes steadied internally, quieted, brought to singleness, and concentrated.}
\hspace{0pt}
\end{rightcolumn}
\end{samepage}
\begin{samepage}
\begin{leftcolumn*}
\EnglishColumn{evampi, bhikkhave, bhikkhu kāyagatāsatiṁ bhāveti.}
\hspace{0pt}\end{leftcolumn*}

\begin{rightcolumn}\PaliColumn{That too is how a bhikkhu develops mindfulness of the body.}
\hspace{0pt}
\end{rightcolumn}
\end{samepage}
\vskip 0.05in
\begin{samepage}
\begin{leftcolumn*}
\EnglishColumn{“puna caparaṁ, bhikkhave, bhikkhu seyyathāpi passeyya sarīraṁ sivathikāya chaḍḍitaṁ kākehi vā khajjamānaṁ kulalehi vā khajjamānaṁ gijjhehi vā khajjamānaṁ kaṅkehi vā khajjamānaṁ sunakhehi vā khajjamānaṁ byagghehi vā khajjamānaṁ dīpīhi vā khajjamānaṁ siṅgālehi vā khajjamānaṁ vividhehi vā pāṇakajātehi khajjamānaṁ.}
\hspace{0pt}\end{leftcolumn*}

\begin{rightcolumn}\PaliColumn{“Again, as though he were to see a corpse thrown aside in a charnel ground, being devoured by crows, hawks, vultures, dogs, jackals, or various kinds of worms,}
\hspace{0pt}
\end{rightcolumn}
\end{samepage}
\begin{samepage}
\begin{leftcolumn*}
\EnglishColumn{so imameva kāyaṁ upasaṁharati;}
\hspace{0pt}\end{leftcolumn*}

\begin{rightcolumn}\PaliColumn{a bhikkhu compares this same body with it thus:}
\hspace{0pt}
\end{rightcolumn}
\end{samepage}
\begin{samepage}
\begin{leftcolumn*}
\EnglishColumn{‘ayampi kho kāyo evaṁdhammo evaṁbhāvī evaṁanatīto’ti.}
\hspace{0pt}\end{leftcolumn*}

\begin{rightcolumn}\PaliColumn{‘This body too is of the same nature, it will be like that, it is not exempt from that fate.’}
\hspace{0pt}
\end{rightcolumn}
\end{samepage}
\begin{samepage}
\begin{leftcolumn*}
\EnglishColumn{tassa evaṁ appamattassa ātāpino pahitattassa viharato ye gehasitā sarasaṅkappā te pahīyanti.}
\hspace{0pt}\end{leftcolumn*}

\begin{rightcolumn}\PaliColumn{As he abides thus diligent, ardent, and resolute, his memories and intentions connected with the household life are abandoned;}
\hspace{0pt}
\end{rightcolumn}
\end{samepage}
\begin{samepage}
\begin{leftcolumn*}
\EnglishColumn{tesaṁ pahānā ajjhattameva cittaṁ santiṭṭhati sannisīdati ekodi hoti samādhiyati.}
\hspace{0pt}\end{leftcolumn*}

\begin{rightcolumn}\PaliColumn{with their abandoning his mind becomes steadied internally, quieted, brought to singleness, and concentrated.}
\hspace{0pt}
\end{rightcolumn}
\end{samepage}
\begin{samepage}
\begin{leftcolumn*}
\EnglishColumn{evampi, bhikkhave, bhikkhu kāyagatāsatiṁ bhāveti.}
\hspace{0pt}\end{leftcolumn*}

\begin{rightcolumn}\PaliColumn{That too is how a bhikkhu develops mindfulness of the body.}
\hspace{0pt}
\end{rightcolumn}
\end{samepage}
\vskip 0.05in
\begin{samepage}
\begin{leftcolumn*}
\EnglishColumn{“puna caparaṁ, bhikkhave, bhikkhu seyyathāpi passeyya sarīraṁ sivathikāya chaḍḍitaṁ aṭṭhikasaṅkhalikaṁ samaṁsalohitaṁ nhārusambandhaṁ.}
\hspace{0pt}\end{leftcolumn*}

\begin{rightcolumn}\PaliColumn{Again, as though he were to see a corpse thrown aside in a charnel ground, a skeleton with flesh and blood, held together with sinews,}
\hspace{0pt}
\end{rightcolumn}
\end{samepage}
\begin{samepage}
\begin{leftcolumn*}
\EnglishColumn{so imameva kāyaṁ upasaṁharati;}
\hspace{0pt}\end{leftcolumn*}

\begin{rightcolumn}\PaliColumn{a bhikkhu compares this same body with it thus:}
\hspace{0pt}
\end{rightcolumn}
\end{samepage}
\begin{samepage}
\begin{leftcolumn*}
\EnglishColumn{‘ayampi kho kāyo evaṁdhammo evaṁbhāvī evaṁanatīto’ti.}
\hspace{0pt}\end{leftcolumn*}

\begin{rightcolumn}\PaliColumn{‘This body too is of the same nature, it will be like that, it is not exempt from that fate.’}
\hspace{0pt}
\end{rightcolumn}
\end{samepage}
\begin{samepage}
\begin{leftcolumn*}
\EnglishColumn{tassa evaṁ appamattassa ātāpino pahitattassa viharato ye gehasitā sarasaṅkappā te pahīyanti.}
\hspace{0pt}\end{leftcolumn*}

\begin{rightcolumn}\PaliColumn{As he abides thus diligent, ardent, and resolute, his memories and intentions connected with the household life are abandoned;}
\hspace{0pt}
\end{rightcolumn}
\end{samepage}
\begin{samepage}
\begin{leftcolumn*}
\EnglishColumn{tesaṁ pahānā ajjhattameva cittaṁ santiṭṭhati sannisīdati ekodi hoti samādhiyati.}
\hspace{0pt}\end{leftcolumn*}

\begin{rightcolumn}\PaliColumn{with their abandoning his mind becomes steadied internally, quieted, brought to singleness, and concentrated.}
\hspace{0pt}
\end{rightcolumn}
\end{samepage}
\begin{samepage}
\begin{leftcolumn*}
\EnglishColumn{evampi, bhikkhave, bhikkhu kāyagatāsatiṁ bhāveti.}
\hspace{0pt}\end{leftcolumn*}

\begin{rightcolumn}\PaliColumn{That too is how a bhikkhu develops mindfulness of the body.}
\hspace{0pt}
\end{rightcolumn}
\end{samepage}
\vskip 0.05in
\begin{samepage}
\begin{leftcolumn*}
\EnglishColumn{“puna caparaṁ, bhikkhave, bhikkhu seyyathāpi passeyya aṭṭhikasaṅkhalikaṁ nimmaṁsalohitamakkhitaṁ nhārusambandhaṁ}
\hspace{0pt}\end{leftcolumn*}

\begin{rightcolumn}\PaliColumn{Again, as though he were to see a fleshless skeleton smeared with blood, held together with sinews,}
\hspace{0pt}
\end{rightcolumn}
\end{samepage}
\begin{samepage}
\begin{leftcolumn*}
\EnglishColumn{so imameva kāyaṁ upasaṁharati;}
\hspace{0pt}\end{leftcolumn*}

\begin{rightcolumn}\PaliColumn{a bhikkhu compares this same body with it thus:}
\hspace{0pt}
\end{rightcolumn}
\end{samepage}
\begin{samepage}
\begin{leftcolumn*}
\EnglishColumn{‘ayampi kho kāyo evaṁdhammo evaṁbhāvī evaṁanatīto’ti.}
\hspace{0pt}\end{leftcolumn*}

\begin{rightcolumn}\PaliColumn{‘This body too is of the same nature, it will be like that, it is not exempt from that fate.’}
\hspace{0pt}
\end{rightcolumn}
\end{samepage}
\begin{samepage}
\begin{leftcolumn*}
\EnglishColumn{tassa evaṁ appamattassa ātāpino pahitattassa viharato ye gehasitā sarasaṅkappā te pahīyanti.}
\hspace{0pt}\end{leftcolumn*}

\begin{rightcolumn}\PaliColumn{As he abides thus diligent, ardent, and resolute, his memories and intentions connected with the household life are abandoned;}
\hspace{0pt}
\end{rightcolumn}
\end{samepage}
\begin{samepage}
\begin{leftcolumn*}
\EnglishColumn{tesaṁ pahānā ajjhattameva cittaṁ santiṭṭhati sannisīdati ekodi hoti samādhiyati.}
\hspace{0pt}\end{leftcolumn*}

\begin{rightcolumn}\PaliColumn{with their abandoning his mind becomes steadied internally, quieted, brought to singleness, and concentrated.}
\hspace{0pt}
\end{rightcolumn}
\end{samepage}
\begin{samepage}
\begin{leftcolumn*}
\EnglishColumn{evampi, bhikkhave, bhikkhu kāyagatāsatiṁ bhāveti.}
\hspace{0pt}\end{leftcolumn*}

\begin{rightcolumn}\PaliColumn{That too is how a bhikkhu develops mindfulness of the body.}
\hspace{0pt}
\end{rightcolumn}
\end{samepage}
\vskip 0.05in
\begin{samepage}
\begin{leftcolumn*}
\EnglishColumn{“puna caparaṁ, bhikkhave, bhikkhu seyyathāpi passeyya aṭṭhikasaṅkhalikaṁ apagatamaṁsalohitaṁ nhārusambandhaṁ.}
\hspace{0pt}\end{leftcolumn*}

\begin{rightcolumn}\PaliColumn{"Again, as though he were to see a skeleton without flesh and blood, held together with sinews,}
\hspace{0pt}
\end{rightcolumn}
\end{samepage}
\begin{samepage}
\begin{leftcolumn*}
\EnglishColumn{so imameva kāyaṁ upasaṁharati;}
\hspace{0pt}\end{leftcolumn*}

\begin{rightcolumn}\PaliColumn{a bhikkhu compares this same body with it thus:}
\hspace{0pt}
\end{rightcolumn}
\end{samepage}
\begin{samepage}
\begin{leftcolumn*}
\EnglishColumn{‘ayampi kho kāyo evaṁdhammo evaṁbhāvī evaṁanatīto’ti.}
\hspace{0pt}\end{leftcolumn*}

\begin{rightcolumn}\PaliColumn{‘This body too is of the same nature, it will be like that, it is not exempt from that fate.’}
\hspace{0pt}
\end{rightcolumn}
\end{samepage}
\begin{samepage}
\begin{leftcolumn*}
\EnglishColumn{tassa evaṁ appamattassa ātāpino pahitattassa viharato ye gehasitā sarasaṅkappā te pahīyanti.}
\hspace{0pt}\end{leftcolumn*}

\begin{rightcolumn}\PaliColumn{As he abides thus diligent, ardent, and resolute, his memories and intentions connected with the household life are abandoned;}
\hspace{0pt}
\end{rightcolumn}
\end{samepage}
\begin{samepage}
\begin{leftcolumn*}
\EnglishColumn{tesaṁ pahānā ajjhattameva cittaṁ santiṭṭhati sannisīdati ekodi hoti samādhiyati.}
\hspace{0pt}\end{leftcolumn*}

\begin{rightcolumn}\PaliColumn{with their abandoning his mind becomes steadied internally, quieted, brought to singleness, and concentrated.}
\hspace{0pt}
\end{rightcolumn}
\end{samepage}
\begin{samepage}
\begin{leftcolumn*}
\EnglishColumn{evampi, bhikkhave, bhikkhu kāyagatāsatiṁ bhāveti.}
\hspace{0pt}\end{leftcolumn*}

\begin{rightcolumn}\PaliColumn{That too is how a bhikkhu develops mindfulness of the body.}
\hspace{0pt}
\end{rightcolumn}
\end{samepage}
\vskip 0.05in
\begin{samepage}
\begin{leftcolumn*}
\EnglishColumn{“puna caparaṁ, bhikkhave, bhikkhu seyyathāpi passeyya aṭṭhikāni apagatasambandhāni disāvidisāvikkhittāni aññena hatthaṭṭhikaṁ aññena pādaṭṭhikaṁ aññena gopphakaṭṭhikaṁ aññena jaṅghaṭṭhikaṁ aññena ūruṭṭhikaṁ aññena kaṭiṭṭhikaṁ aññena phāsukaṭṭhikaṁ aññena piṭṭhiṭṭhikaṁ aññena khandhaṭṭhikaṁ aññena gīvaṭṭhikaṁ aññena hanukaṭṭhikaṁ aññena dantaṭṭhikaṁ aññena sīsakaṭāhaṁ.}
\hspace{0pt}\end{leftcolumn*}

\begin{rightcolumn}\PaliColumn{"Again, as though he were to see disconnected bones scattered in all directions—here a hand-bone, there a foot-bone, here a shin-bone, there a thigh-bone, here a hip-bone, there a back-bone, here a rib-bone, there a breast-bone, here an arm-bone, there a shoulder-bone, here a neck-bone, there a jaw-bone, here a tooth, there the skull,}
\hspace{0pt}
\end{rightcolumn}
\end{samepage}
\begin{samepage}
\begin{leftcolumn*}
\EnglishColumn{so imameva kāyaṁ upasaṁharati;}
\hspace{0pt}\end{leftcolumn*}

\begin{rightcolumn}\PaliColumn{a bhikkhu compares this same body with it thus:}
\hspace{0pt}
\end{rightcolumn}
\end{samepage}
\begin{samepage}
\begin{leftcolumn*}
\EnglishColumn{‘ayampi kho kāyo evaṁdhammo evaṁbhāvī evaṁanatīto’ti.}
\hspace{0pt}\end{leftcolumn*}

\begin{rightcolumn}\PaliColumn{‘This body too is of the same nature, it will be like that, it is not exempt from that fate.’}
\hspace{0pt}
\end{rightcolumn}
\end{samepage}
\begin{samepage}
\begin{leftcolumn*}
\EnglishColumn{tassa evaṁ appamattassa ātāpino pahitattassa viharato ye gehasitā sarasaṅkappā te pahīyanti.}
\hspace{0pt}\end{leftcolumn*}

\begin{rightcolumn}\PaliColumn{As he abides thus diligent, ardent, and resolute, his memories and intentions connected with the household life are abandoned;}
\hspace{0pt}
\end{rightcolumn}
\end{samepage}
\begin{samepage}
\begin{leftcolumn*}
\EnglishColumn{tesaṁ pahānā ajjhattameva cittaṁ santiṭṭhati sannisīdati ekodi hoti samādhiyati.}
\hspace{0pt}\end{leftcolumn*}

\begin{rightcolumn}\PaliColumn{with their abandoning his mind becomes steadied internally, quieted, brought to singleness, and concentrated.}
\hspace{0pt}
\end{rightcolumn}
\end{samepage}
\begin{samepage}
\begin{leftcolumn*}
\EnglishColumn{evampi, bhikkhave, bhikkhu kāyagatāsatiṁ bhāveti.}
\hspace{0pt}\end{leftcolumn*}

\begin{rightcolumn}\PaliColumn{That too is how a bhikkhu develops mindfulness of the body.}
\hspace{0pt}
\end{rightcolumn}
\end{samepage}
\vskip 0.05in
\begin{samepage}
\begin{leftcolumn*}
\EnglishColumn{“puna caparaṁ, bhikkhave, bhikkhu seyyathāpi passeyya sarīraṁ sivathikāya chaḍḍitaṁ; aṭṭhikāni setāni saṅkhavaṇṇapaṭibhāgāni.}
\hspace{0pt}\end{leftcolumn*}

\begin{rightcolumn}\PaliColumn{“Again, as though he were to see a corpse thrown aside in a charnel ground, bones bleached white, the colour of shells,}
\hspace{0pt}
\end{rightcolumn}
\end{samepage}
\begin{samepage}
\begin{leftcolumn*}
\EnglishColumn{so imameva kāyaṁ upasaṁharati;}
\hspace{0pt}\end{leftcolumn*}

\begin{rightcolumn}\PaliColumn{a bhikkhu compares this same body with it thus:}
\hspace{0pt}
\end{rightcolumn}
\end{samepage}
\begin{samepage}
\begin{leftcolumn*}
\EnglishColumn{‘ayampi kho kāyo evaṁdhammo evaṁbhāvī evaṁanatīto’ti.}
\hspace{0pt}\end{leftcolumn*}

\begin{rightcolumn}\PaliColumn{‘This body too is of the same nature, it will be like that, it is not exempt from that fate.’}
\hspace{0pt}
\end{rightcolumn}
\end{samepage}
\begin{samepage}
\begin{leftcolumn*}
\EnglishColumn{tassa evaṁ appamattassa ātāpino pahitattassa viharato ye gehasitā sarasaṅkappā te pahīyanti.}
\hspace{0pt}\end{leftcolumn*}

\begin{rightcolumn}\PaliColumn{As he abides thus diligent, ardent, and resolute, his memories and intentions connected with the household life are abandoned;}
\hspace{0pt}
\end{rightcolumn}
\end{samepage}
\begin{samepage}
\begin{leftcolumn*}
\EnglishColumn{tesaṁ pahānā ajjhattameva cittaṁ santiṭṭhati sannisīdati ekodi hoti samādhiyati.}
\hspace{0pt}\end{leftcolumn*}

\begin{rightcolumn}\PaliColumn{with their abandoning his mind becomes steadied internally, quieted, brought to singleness, and concentrated.}
\hspace{0pt}
\end{rightcolumn}
\end{samepage}
\begin{samepage}
\begin{leftcolumn*}
\EnglishColumn{evampi, bhikkhave, bhikkhu kāyagatāsatiṁ bhāveti.}
\hspace{0pt}\end{leftcolumn*}

\begin{rightcolumn}\PaliColumn{That too is how a bhikkhu develops mindfulness of the body.}
\hspace{0pt}
\end{rightcolumn}
\end{samepage}
\vskip 0.05in
\begin{samepage}
\begin{leftcolumn*}
\EnglishColumn{“puna caparaṁ, bhikkhave, bhikkhu seyyathāpi passeyya aṭṭhikāni puñjakitāni terovassikāni}
\hspace{0pt}\end{leftcolumn*}

\begin{rightcolumn}\PaliColumn{"Again, as though he were to see bones heaped up,}
\hspace{0pt}
\end{rightcolumn}
\end{samepage}
\begin{samepage}
\begin{leftcolumn*}
\EnglishColumn{so imameva kāyaṁ upasaṁharati;}
\hspace{0pt}\end{leftcolumn*}

\begin{rightcolumn}\PaliColumn{a bhikkhu compares this same body with it thus:}
\hspace{0pt}
\end{rightcolumn}
\end{samepage}
\begin{samepage}
\begin{leftcolumn*}
\EnglishColumn{‘ayampi kho kāyo evaṁdhammo evaṁbhāvī evaṁanatīto’ti.}
\hspace{0pt}\end{leftcolumn*}

\begin{rightcolumn}\PaliColumn{‘This body too is of the same nature, it will be like that, it is not exempt from that fate.’}
\hspace{0pt}
\end{rightcolumn}
\end{samepage}
\begin{samepage}
\begin{leftcolumn*}
\EnglishColumn{tassa evaṁ appamattassa ātāpino pahitattassa viharato ye gehasitā sarasaṅkappā te pahīyanti.}
\hspace{0pt}\end{leftcolumn*}

\begin{rightcolumn}\PaliColumn{As he abides thus diligent, ardent, and resolute, his memories and intentions connected with the household life are abandoned;}
\hspace{0pt}
\end{rightcolumn}
\end{samepage}
\begin{samepage}
\begin{leftcolumn*}
\EnglishColumn{tesaṁ pahānā ajjhattameva cittaṁ santiṭṭhati sannisīdati ekodi hoti samādhiyati.}
\hspace{0pt}\end{leftcolumn*}

\begin{rightcolumn}\PaliColumn{with their abandoning his mind becomes steadied internally, quieted, brought to singleness, and concentrated.}
\hspace{0pt}
\end{rightcolumn}
\end{samepage}
\begin{samepage}
\begin{leftcolumn*}
\EnglishColumn{evampi, bhikkhave, bhikkhu kāyagatāsatiṁ bhāveti.}
\hspace{0pt}\end{leftcolumn*}

\begin{rightcolumn}\PaliColumn{That too is how a bhikkhu develops mindfulness of the body.}
\hspace{0pt}
\end{rightcolumn}
\end{samepage}
\vskip 0.05in
\begin{samepage}
\begin{leftcolumn*}
\EnglishColumn{“puna caparaṁ, bhikkhave, bhikkhu seyyathāpi passeyya aṭṭhikāni pūtīni cuṇṇakajātāni.}
\hspace{0pt}\end{leftcolumn*}

\begin{rightcolumn}\PaliColumn{“Again, as though he were to see bones more than a year old, rotted and crumbled to dust,}
\hspace{0pt}
\end{rightcolumn}
\end{samepage}
\begin{samepage}
\begin{leftcolumn*}
\EnglishColumn{so imameva kāyaṁ upasaṁharati;}
\hspace{0pt}\end{leftcolumn*}

\begin{rightcolumn}\PaliColumn{a bhikkhu compares this same body with it thus:}
\hspace{0pt}
\end{rightcolumn}
\end{samepage}
\begin{samepage}
\begin{leftcolumn*}
\EnglishColumn{‘ayampi kho kāyo evaṁdhammo evaṁbhāvī evaṁanatīto’ti.}
\hspace{0pt}\end{leftcolumn*}

\begin{rightcolumn}\PaliColumn{‘This body too is of the same nature, it will be like that, it is not exempt from that fate.’}
\hspace{0pt}
\end{rightcolumn}
\end{samepage}
\begin{samepage}
\begin{leftcolumn*}
\EnglishColumn{tassa evaṁ appamattassa ātāpino pahitattassa viharato ye gehasitā sarasaṅkappā te pahīyanti.}
\hspace{0pt}\end{leftcolumn*}

\begin{rightcolumn}\PaliColumn{As he abides thus diligent, ardent, and resolute, his memories and intentions connected with the household life are abandoned;}
\hspace{0pt}
\end{rightcolumn}
\end{samepage}
\begin{samepage}
\begin{leftcolumn*}
\EnglishColumn{tesaṁ pahānā ajjhattameva cittaṁ santiṭṭhati sannisīdati ekodi hoti samādhiyati.}
\hspace{0pt}\end{leftcolumn*}

\begin{rightcolumn}\PaliColumn{with their abandoning his mind becomes steadied internally, quieted, brought to singleness, and concentrated.}
\hspace{0pt}
\end{rightcolumn}
\end{samepage}
\begin{samepage}
\begin{leftcolumn*}
\EnglishColumn{evampi, bhikkhave, bhikkhu kāyagatāsatiṁ bhāveti.}
\hspace{0pt}\end{leftcolumn*}

\begin{rightcolumn}\PaliColumn{That too is how a bhikkhu develops mindfulness of the body.}
\hspace{0pt}
\end{rightcolumn}
\end{samepage}
\vskip 0.05in
\begin{samepage}
\begin{leftcolumn*}
\EnglishColumn{55. “puna caparaṁ, bhikkhave, bhikkhu vivicceva kāmehi, vivicca akusalehi dhammehi savitakkaṁ savicāraṁ vivekajaṁ pītisukhaṁ paṭhamaṁ jhānaṁ upasampajja viharati.}
\hspace{0pt}\end{leftcolumn*}

\begin{rightcolumn}\PaliColumn{“Again, bhikkhus, quite secluded from sensual pleasures, secluded from unwholesome states, a bhikkhu enters upon and abides in the first jhāna, which is accompanied by applied and sustained thought, with rapture and pleasure born of seclusion.}
\hspace{0pt}
\end{rightcolumn}
\end{samepage}
\begin{samepage}
\begin{leftcolumn*}
\EnglishColumn{so imameva kāyaṁ vivekajena pītisukhena abhisandeti parisandeti paripūreti parippharati, nāssa kiñci sabbāvato kāyassa vivekajena pītisukhena apphuṭaṁ hoti.}
\hspace{0pt}\end{leftcolumn*}

\begin{rightcolumn}\PaliColumn{He makes the rapture and pleasure born of seclusion drench, steep, fill, and pervade this body, so that there is no part of his whole body unpervaded by the rapture and pleasure born of seclusion.}
\hspace{0pt}
\end{rightcolumn}
\end{samepage}
\begin{samepage}
\begin{leftcolumn*}
\EnglishColumn{seyyathāpi, bhikkhave, dakkho nhāpako vā nhāpakantevāsī vā kaṁsathāle nhānīyacuṇṇāni ākiritvā udakena paripphosakaṁ paripphosakaṁ sanneyya, sāyaṁ nhānīyapiṇḍi snehānugatā snehaparetā santarabāhirā phuṭā snehena na ca pagghariṇī;}
\hspace{0pt}\end{leftcolumn*}

\begin{rightcolumn}\PaliColumn{Just as a skilled bath man or a bath man’s apprentice  heaps bath powder in a metal basin and, sprinkling it gradually with water, kneads it till the moisture wets his ball of bath powder, soaks it and pervades it inside and out, yet the ball itself does not ooze;}
\hspace{0pt}
\end{rightcolumn}
\end{samepage}
\begin{samepage}
\begin{leftcolumn*}
\EnglishColumn{evameva kho, bhikkhave, bhikkhu imameva kāyaṁ vivekajena pītisukhena abhisandeti parisandeti paripūreti parippharati; nāssa kiñci sabbāvato kāyassa vivekajena pītisukhena apphuṭaṁ hoti.}
\hspace{0pt}\end{leftcolumn*}

\begin{rightcolumn}\PaliColumn{so too, a bhikkhu makes the rapture and pleasure born of seclusion  drench, steep, fill, and pervade this body, so that there is no part of his whole body unpervaded by the rapture and pleasure born of seclusion.}
\hspace{0pt}
\end{rightcolumn}
\end{samepage}
\begin{samepage}
\begin{leftcolumn*}
\EnglishColumn{tassa evaṁ appamattassa ātāpino pahitattassa viharato ye gehasitā sarasaṅkappā te pahīyanti.}
\hspace{0pt}\end{leftcolumn*}

\begin{rightcolumn}\PaliColumn{As he abides thus diligent, ardent, and resolute, his memories and intentions connected with the household life are abandoned;}
\hspace{0pt}
\end{rightcolumn}
\end{samepage}
\begin{samepage}
\begin{leftcolumn*}
\EnglishColumn{tesaṁ pahānā ajjhattameva cittaṁ santiṭṭhati sannisīdati ekodi hoti samādhiyati.}
\hspace{0pt}\end{leftcolumn*}

\begin{rightcolumn}\PaliColumn{with their abandoning his mind becomes steadied internally, quieted, brought to singleness, and concentrated.}
\hspace{0pt}
\end{rightcolumn}
\end{samepage}
\begin{samepage}
\begin{leftcolumn*}
\EnglishColumn{evampi, bhikkhave, bhikkhu kāyagatāsatiṁ bhāveti.}
\hspace{0pt}\end{leftcolumn*}

\begin{rightcolumn}\PaliColumn{That too is how a bhikkhu develops mindfulness of the body.}
\hspace{0pt}
\end{rightcolumn}
\end{samepage}
\vskip 0.05in
\begin{samepage}
\begin{leftcolumn*}
\EnglishColumn{“puna caparaṁ, bhikkhave, bhikkhu vitakkavicārānaṁ vūpasamā ajjhattaṁ sampasādanaṁ cetaso ekodibhāvaṁ avitakkaṁ avicāraṁ samādhijaṁ pītisukhaṁ dutiyaṁ jhānaṁ upasampajja viharati.}
\hspace{0pt}\end{leftcolumn*}

\begin{rightcolumn}\PaliColumn{“Again, bhikkhus, with the stilling of applied and sustained thought, a bhikkhu enters upon and abides in the second jhāna, which has self-confidence and singleness of mind without applied and sustained thought, with rapture and pleasure born of concentration.}
\hspace{0pt}
\end{rightcolumn}
\end{samepage}
\begin{samepage}
\begin{leftcolumn*}
\EnglishColumn{so imameva kāyaṁ samādhijena pītisukhena abhisandeti parisandeti paripūreti parippharati; nāssa kiñci sabbāvato kāyassa samādhijena pītisukhena apphuṭaṁ hoti.}
\hspace{0pt}\end{leftcolumn*}

\begin{rightcolumn}\PaliColumn{He makes the rapture and pleasure born of concentration drench, steep, fill, and pervade this body, so that there is no part of his whole body unpervaded by the rapture and pleasure born of concentration.}
\hspace{0pt}
\end{rightcolumn}
\end{samepage}
\begin{samepage}
\begin{leftcolumn*}
\EnglishColumn{seyyathāpi, bhikkhave, udakarahado gambhīro ubbhidodako.}
\hspace{0pt}\end{leftcolumn*}

\begin{rightcolumn}\PaliColumn{Just as though there were a lake whose waters welled up from below;}
\hspace{0pt}
\end{rightcolumn}
\end{samepage}
\begin{samepage}
\begin{leftcolumn*}
\EnglishColumn{tassa nevassa puratthimāya disāya udakassa āyamukhaṁ na pacchimāya disāya udakassa āyamukhaṁ na uttarāya disāya udakassa āyamukhaṁ na dakkhiṇāya disāya udakassa āyamukhaṁ;}
\hspace{0pt}\end{leftcolumn*}

\begin{rightcolumn}\PaliColumn{and it had no inflow from east, west, north, or south;}
\hspace{0pt}
\end{rightcolumn}
\end{samepage}
\begin{samepage}
\begin{leftcolumn*}
\EnglishColumn{devo ca na kālena kālaṁ sammā dhāraṁ anuppaveccheyya; atha kho tamhāva udakarahadā sītā vāridhārā ubbhijjitvā tameva udakarahadaṁ sītena vārinā abhisandeyya parisandeyya paripūreyya paripphareyya, nāssa kiñci sabbāvato udakarahadassa sītena vārinā apphuṭaṁ assa;}
\hspace{0pt}\end{leftcolumn*}

\begin{rightcolumn}\PaliColumn{and would not be replenished from time to time by showers of rain, then the cool fount of water welling up in the lake would make the cool water drench, steep, fill, and pervade the lake, so that there would be no part of the whole lake unpervaded by cool water;}
\hspace{0pt}
\end{rightcolumn}
\end{samepage}
\begin{samepage}
\begin{leftcolumn*}
\EnglishColumn{evameva kho, bhikkhave, bhikkhu imameva kāyaṁ samādhijena pītisukhena abhisandeti parisandeti paripūreti parippharati, nāssa kiñci sabbāvato kāyassa samādhijena pītisukhena apphuṭaṁ hoti.}
\hspace{0pt}\end{leftcolumn*}

\begin{rightcolumn}\PaliColumn{so too, a bhikkhu makes the rapture and pleasure born of concentration drench, steep, fill, and pervade this body, so that there is no part of his whole body unpervaded by the rapture and pleasure born of concentration.}
\hspace{0pt}
\end{rightcolumn}
\end{samepage}
\begin{samepage}
\begin{leftcolumn*}
\EnglishColumn{tassa evaṁ appamattassa ātāpino pahitattassa viharato ye gehasitā sarasaṅkappā te pahīyanti.}
\hspace{0pt}\end{leftcolumn*}

\begin{rightcolumn}\PaliColumn{As he abides thus diligent, ardent, and resolute, his memories and intentions connected with the household life are abandoned;}
\hspace{0pt}
\end{rightcolumn}
\end{samepage}
\begin{samepage}
\begin{leftcolumn*}
\EnglishColumn{tesaṁ pahānā ajjhattameva cittaṁ santiṭṭhati sannisīdati ekodi hoti samādhiyati.}
\hspace{0pt}\end{leftcolumn*}

\begin{rightcolumn}\PaliColumn{with their abandoning his mind becomes steadied internally, quieted, brought to singleness, and concentrated.}
\hspace{0pt}
\end{rightcolumn}
\end{samepage}
\begin{samepage}
\begin{leftcolumn*}
\EnglishColumn{evampi, bhikkhave, bhikkhu kāyagatāsatiṁ bhāveti.}
\hspace{0pt}\end{leftcolumn*}

\begin{rightcolumn}\PaliColumn{That too is how a bhikkhu develops mindfulness of the body.}
\hspace{0pt}
\end{rightcolumn}
\end{samepage}
\vskip 0.05in
\begin{samepage}
\begin{leftcolumn*}
\EnglishColumn{“puna caparaṁ, bhikkhave, bhikkhu pītiyā ca virāgā upekkhako ca viharati sato ca sampajāno, sukhañca kāyena paṭisaṁvedeti, yaṁ taṁ ariyā ācikkhanti: “upekkhako satimā sukhavihārī”ti, tatiyaṁ jhānaṁ upasampajja viharati.}
\hspace{0pt}\end{leftcolumn*}

\begin{rightcolumn}\PaliColumn{“Again, bhikkhus, with the fading away as well of rapture, a bhikkhu abides in equanimity, and mindful and fully aware, still feeling pleasure with the body, he enters upon and abides in the third jhāna, on account of which noble ones announce: ‘He has a pleasant abiding who has equanimity and is mindful.’}
\hspace{0pt}
\end{rightcolumn}
\end{samepage}
\begin{samepage}
\begin{leftcolumn*}
\EnglishColumn{so imameva kāyaṁ nippītikena sukhena abhisandeti parisandeti paripūreti parippharati, nāssa kiñci sabbāvato kāyassa nippītikena sukhena apphuṭaṁ hoti.}
\hspace{0pt}\end{leftcolumn*}

\begin{rightcolumn}\PaliColumn{He makes the pleasure divested of rapture drench, steep, fill, and pervade this body, so that there is no part of his whole body unpervaded by the pleasure divested of rapture.}
\hspace{0pt}
\end{rightcolumn}
\end{samepage}
\begin{samepage}
\begin{leftcolumn*}
\EnglishColumn{seyyathāpi, bhikkhave, uppaliniyaṁ vā paduminiyaṁ vā puṇḍarīkiniyaṁ vā appekaccāni uppalāni vā padumāni vā puṇḍarīkāni vā udake jātāni udake saṁvaḍḍhāni udakānuggatāni antonimuggaposīni, tāni yāva caggā yāva ca mūlā sītena vārinā abhisannāni parisannāni paripūrāni paripphuṭāni, nāssa kiñci sabbāvataṁ uppalānaṁ vā padumānaṁ vā puṇḍarīkānaṁ vā sītena vārinā apphuṭaṁ assa;}
\hspace{0pt}\end{leftcolumn*}

\begin{rightcolumn}\PaliColumn{Just as in a pond of blue or white or red lotuses, some lotuses that are born and grow in the water thrive immersed in the water without rising out of it, and cool water drenches, steeps, fills, and pervades them to their tips and their roots, so that there is no part of all those lotuses unpervaded by cool water;}
\hspace{0pt}
\end{rightcolumn}
\end{samepage}
\begin{samepage}
\begin{leftcolumn*}
\EnglishColumn{evameva kho, bhikkhave, bhikkhu imameva kāyaṁ nippītikena sukhena abhisandeti parisandeti paripūreti parippharati, nāssa kiñci sabbāvato kāyassa nippītikena sukhena apphuṭaṁ hoti.}
\hspace{0pt}\end{leftcolumn*}

\begin{rightcolumn}\PaliColumn{so too, a bhikkhu makes the pleasure divested of rapture drench, steep, fill, and pervade this body, so that there is no part of his whole body unpervaded by the pleasure divested of rapture.}
\hspace{0pt}
\end{rightcolumn}
\end{samepage}
\begin{samepage}
\begin{leftcolumn*}
\EnglishColumn{tassa evaṁ appamattassa ātāpino pahitattassa viharato ye gehasitā sarasaṅkappā te pahīyanti.}
\hspace{0pt}\end{leftcolumn*}

\begin{rightcolumn}\PaliColumn{As he abides thus diligent, ardent, and resolute, his memories and intentions connected with the household life are abandoned;}
\hspace{0pt}
\end{rightcolumn}
\end{samepage}
\begin{samepage}
\begin{leftcolumn*}
\EnglishColumn{tesaṁ pahānā ajjhattameva cittaṁ santiṭṭhati sannisīdati ekodi hoti samādhiyati.}
\hspace{0pt}\end{leftcolumn*}

\begin{rightcolumn}\PaliColumn{with their abandoning his mind becomes steadied internally, quieted, brought to singleness, and concentrated.}
\hspace{0pt}
\end{rightcolumn}
\end{samepage}
\begin{samepage}
\begin{leftcolumn*}
\EnglishColumn{evampi, bhikkhave, bhikkhu kāyagatāsatiṁ bhāveti.}
\hspace{0pt}\end{leftcolumn*}

\begin{rightcolumn}\PaliColumn{That too is how a bhikkhu develops mindfulness of the body.}
\hspace{0pt}
\end{rightcolumn}
\end{samepage}
\vskip 0.05in
\begin{samepage}
\begin{leftcolumn*}
\EnglishColumn{“puna caparaṁ, bhikkhave, bhikkhu sukhassa ca pahānā pubbeva somanassadomanassānaṁ atthaṅgamā adukkhamasukhaṁ upekkhāsatipārisuddhiṁ catutthaṁ jhānaṁ upasampajja viharati.}
\hspace{0pt}\end{leftcolumn*}

\begin{rightcolumn}\PaliColumn{“Again, bhikkhus, with the abandoning of pleasure and pain, and with the previous disappearance of joy and grief, a bhikkhu enters upon and abides in the fourth jhāna, which has neither-pain-nor-pleasure and purity of mindfulness due to equanimity.}
\hspace{0pt}
\end{rightcolumn}
\end{samepage}
\begin{samepage}
\begin{leftcolumn*}
\EnglishColumn{so imameva kāyaṁ parisuddhena cetasā pariyodātena pharitvā nisinno hoti; nāssa kiñci sabbāvato kāyassa parisuddhena cetasā pariyodātena apphuṭaṁ hoti.}
\hspace{0pt}\end{leftcolumn*}

\begin{rightcolumn}\PaliColumn{He sits pervading this body with a pure bright mind, so that there is no part of his whole body unpervaded by the pure bright mind.}
\hspace{0pt}
\end{rightcolumn}
\end{samepage}
\begin{samepage}
\begin{leftcolumn*}
\EnglishColumn{seyyathāpi, bhikkhave, puriso odātena vatthena sasīsaṁ pārupitvā nisinno assa, nāssa kiñci sabbāvato kāyassa odātena vatthena apphuṭaṁ assa;}
\hspace{0pt}\end{leftcolumn*}

\begin{rightcolumn}\PaliColumn{Just as though a man were sitting covered from head down with a white cloth, so that there would be no part of his whole body not covered by the white cloth;}
\hspace{0pt}
\end{rightcolumn}
\end{samepage}
\begin{samepage}
\begin{leftcolumn*}
\EnglishColumn{evameva kho, bhikkhave, bhikkhu imameva kāyaṁ parisuddhena cetasā pariyodātena pharitvā nisinno hoti, nāssa kiñci sabbāvato kāyassa parisuddhena cetasā pariyodātena apphuṭaṁ hoti.}
\hspace{0pt}\end{leftcolumn*}

\begin{rightcolumn}\PaliColumn{so too, a bhikkhu sits pervading this body with a pure bright mind, so that there is no part of his whole body unpervaded by the pure bright mind.}
\hspace{0pt}
\end{rightcolumn}
\end{samepage}
\begin{samepage}
\begin{leftcolumn*}
\EnglishColumn{tassa evaṁ appamattassa ātāpino pahitattassa viharato ye gehasitā sarasaṅkappā te pahīyanti.}
\hspace{0pt}\end{leftcolumn*}

\begin{rightcolumn}\PaliColumn{As he abides thus diligent, ardent, and resolute, his memories and intentions based on the household life are abandoned;}
\hspace{0pt}
\end{rightcolumn}
\end{samepage}
\begin{samepage}
\begin{leftcolumn*}
\EnglishColumn{tesaṁ pahānā ajjhattameva cittaṁ santiṭṭhati sannisīdati ekodi hoti samādhiyati.}
\hspace{0pt}\end{leftcolumn*}

\begin{rightcolumn}\PaliColumn{with their abandoning his mind becomes steadied internally, quieted, brought to singleness, and concentrated.}
\hspace{0pt}
\end{rightcolumn}
\end{samepage}
\begin{samepage}
\begin{leftcolumn*}
\EnglishColumn{evampi, bhikkhave, bhikkhu kāyagatāsatiṁ bhāveti.}
\hspace{0pt}\end{leftcolumn*}

\begin{rightcolumn}\PaliColumn{That too is how a bhikkhu develops mindfulness of the body.}
\hspace{0pt}
\end{rightcolumn}
\end{samepage}
\vskip 0.05in
\begin{samepage}
\begin{leftcolumn*}
\EnglishColumn{156. “yassa kassaci, bhikkhave, kāyagatāsati bhāvitā bahulīkatā, antogadhāvāssa kusalā dhammā ye keci vijjābhāgiyā.}
\hspace{0pt}\end{leftcolumn*}

\begin{rightcolumn}\PaliColumn{“Bhikkhus, anyone who has developed and cultivated mindfulness of the body has included within himself whatever wholesome states there are that partake of true knowledge.}
\hspace{0pt}
\end{rightcolumn}
\end{samepage}
\begin{samepage}
\begin{leftcolumn*}
\EnglishColumn{seyyathāpi, bhikkhave, yassa kassaci mahāsamuddo cetasā phuṭo, antogadhāvāssa kunnadiyo yā kāci samuddaṅgamā;}
\hspace{0pt}\end{leftcolumn*}

\begin{rightcolumn}\PaliColumn{Just as anyone who has extended his mind over the great ocean has included within it whatever streams there are that flow into the ocean;}
\hspace{0pt}
\end{rightcolumn}
\end{samepage}
\begin{samepage}
\begin{leftcolumn*}
\EnglishColumn{evameva kho, bhikkhave, yassa kassaci kāyagatāsati bhāvitā bahulīkatā, antogadhāvāssa kusalā dhammā ye keci vijjābhāgiyā.}
\hspace{0pt}\end{leftcolumn*}

\begin{rightcolumn}\PaliColumn{so too, anyone who has developed and cultivated mindfulness of the body has included within himself whatever wholesome states there are that partake of true knowledge.}
\hspace{0pt}
\end{rightcolumn}
\end{samepage}
\begin{samepage}
\begin{leftcolumn*}
\EnglishColumn{“yassa kassaci, bhikkhave, kāyagatāsati abhāvitā abahulīkatā, labhati tassa māro otāraṁ, labhati tassa māro ārammaṇaṁ.}
\hspace{0pt}\end{leftcolumn*}

\begin{rightcolumn}\PaliColumn{“Bhikkhus, when anyone has not developed and cultivated mindfulness of the body, Māra finds an opportunity and a support in him.}
\hspace{0pt}
\end{rightcolumn}
\end{samepage}
\begin{samepage}
\begin{leftcolumn*}
\EnglishColumn{seyyathāpi, bhikkhave, puriso garukaṁ silāguḷaṁ allamattikāpuñje pakkhipeyya.}
\hspace{0pt}\end{leftcolumn*}

\begin{rightcolumn}\PaliColumn{Suppose a man were to throw a heavy stone ball upon a mound of wet clay.}
\hspace{0pt}
\end{rightcolumn}
\end{samepage}
\begin{samepage}
\begin{leftcolumn*}
\EnglishColumn{taṁ kiṁ maññatha, bhikkhave,}
\hspace{0pt}\end{leftcolumn*}

\begin{rightcolumn}\PaliColumn{What do you think, bhikkhus?}
\hspace{0pt}
\end{rightcolumn}
\end{samepage}
\begin{samepage}
\begin{leftcolumn*}
\EnglishColumn{api nu taṁ garukaṁ silāguḷaṁ allamattikāpuñje labhetha otāran”ti?}
\hspace{0pt}\end{leftcolumn*}

\begin{rightcolumn}\PaliColumn{Would that heavy ball find entry into that mound of wet clay?”}
\hspace{0pt}
\end{rightcolumn}
\end{samepage}
\begin{samepage}
\begin{leftcolumn*}
\EnglishColumn{“evaṁ, bhante”.}
\hspace{0pt}\end{leftcolumn*}

\begin{rightcolumn}\PaliColumn{“Yes, venerable sir.”}
\hspace{0pt}
\end{rightcolumn}
\end{samepage}
\begin{samepage}
\begin{leftcolumn*}
\EnglishColumn{“evameva kho, bhikkhave, yassa kassaci kāyagatāsati abhāvitā abahulīkatā, labhati tassa māro otāraṁ, labhati tassa māro ārammaṇaṁ.}
\hspace{0pt}\end{leftcolumn*}

\begin{rightcolumn}\PaliColumn{“So too, bhikkhus, when anyone has not developed and cultivated mindfulness of the body, Māra finds an opportunity and a support in him.}
\hspace{0pt}
\end{rightcolumn}
\end{samepage}
\begin{samepage}
\begin{leftcolumn*}
\EnglishColumn{seyyathāpi, bhikkhave, sukkhaṁ kaṭṭhaṁ koḷāpaṁ; atha puriso āgaccheyya uttarāraṇiṁ ādāya}
\hspace{0pt}\end{leftcolumn*}

\begin{rightcolumn}\PaliColumn{“Suppose there were a dry sapless piece of wood, and a man came with an upper fire-stick, thinking:}
\hspace{0pt}
\end{rightcolumn}
\end{samepage}
\begin{samepage}
\begin{leftcolumn*}
\EnglishColumn{‘aggiṁ abhinibbattessāmi, tejo pātukarissāmī’ti.}
\hspace{0pt}\end{leftcolumn*}

\begin{rightcolumn}\PaliColumn{‘I shall light a fire, I shall produce heat.’}
\hspace{0pt}
\end{rightcolumn}
\end{samepage}
\begin{samepage}
\begin{leftcolumn*}
\EnglishColumn{taṁ kiṁ maññatha, bhikkhave,}
\hspace{0pt}\end{leftcolumn*}

\begin{rightcolumn}\PaliColumn{What do you think, bhikkhus?}
\hspace{0pt}
\end{rightcolumn}
\end{samepage}
\begin{samepage}
\begin{leftcolumn*}
\EnglishColumn{api nu so puriso amuṁ sukkhaṁ kaṭṭhaṁ koḷāpaṁ uttarāraṇiṁ ādāya abhimanthento aggiṁ abhinibbatteyya, tejo pātukareyyā”ti?}
\hspace{0pt}\end{leftcolumn*}

\begin{rightcolumn}\PaliColumn{Could the man light a fire and produce heat by rubbing the dry sapless piece of wood with an upper fire-stick?”}
\hspace{0pt}
\end{rightcolumn}
\end{samepage}
\begin{samepage}
\begin{leftcolumn*}
\EnglishColumn{“evaṁ, bhante”.}
\hspace{0pt}\end{leftcolumn*}

\begin{rightcolumn}\PaliColumn{“Yes, venerable sir.”}
\hspace{0pt}
\end{rightcolumn}
\end{samepage}
\begin{samepage}
\begin{leftcolumn*}
\EnglishColumn{“evameva kho, bhikkhave, yassa kassaci kāyagatāsati abhāvitā abahulīkatā, labhati tassa māro otāraṁ, labhati tassa māro ārammaṇaṁ.}
\hspace{0pt}\end{leftcolumn*}

\begin{rightcolumn}\PaliColumn{“So too, bhikkhus, when anyone has not developed and cultivated mindfulness of the body, Māra finds an opportunity and a support in him.}
\hspace{0pt}
\end{rightcolumn}
\end{samepage}
\begin{samepage}
\begin{leftcolumn*}
\EnglishColumn{seyyathāpi, bhikkhave, udakamaṇiko ritto tuccho ādhāre ṭhapito; atha puriso āgaccheyya udakabhāraṁ ādāya.}
\hspace{0pt}\end{leftcolumn*}

\begin{rightcolumn}\PaliColumn{“Suppose there were a hollow empty water jug set out on a stand, and a man came with a supply of water.}
\hspace{0pt}
\end{rightcolumn}
\end{samepage}
\begin{samepage}
\begin{leftcolumn*}
\EnglishColumn{taṁ kiṁ maññatha, bhikkhave,}
\hspace{0pt}\end{leftcolumn*}

\begin{rightcolumn}\PaliColumn{What do you think, bhikkhus?}
\hspace{0pt}
\end{rightcolumn}
\end{samepage}
\begin{samepage}
\begin{leftcolumn*}
\EnglishColumn{api nu so puriso labhetha udakassa nikkhepanan”ti?}
\hspace{0pt}\end{leftcolumn*}

\begin{rightcolumn}\PaliColumn{Could the man pour the water into the jug?”}
\hspace{0pt}
\end{rightcolumn}
\end{samepage}
\begin{samepage}
\begin{leftcolumn*}
\EnglishColumn{“evaṁ, bhante”.}
\hspace{0pt}\end{leftcolumn*}

\begin{rightcolumn}\PaliColumn{“Yes, venerable sir.”}
\hspace{0pt}
\end{rightcolumn}
\end{samepage}
\begin{samepage}
\begin{leftcolumn*}
\EnglishColumn{“evameva kho, bhikkhave, yassa kassaci kāyagatāsati abhāvitā abahulīkatā, labhati tassa māro otāraṁ, labhati tassa māro ārammaṇaṁ”.}
\hspace{0pt}\end{leftcolumn*}

\begin{rightcolumn}\PaliColumn{“So too, bhikkhus, when anyone has not developed and cultivated mindfulness of the body, Māra finds an opportunity and a support in him.}
\hspace{0pt}
\end{rightcolumn}
\end{samepage}
\begin{samepage}
\begin{leftcolumn*}
\EnglishColumn{157. “yassa kassaci, bhikkhave, kāyagatāsati bhāvitā bahulīkatā, na tassa labhati māro otāraṁ, na tassa labhati māro ārammaṇaṁ.}
\hspace{0pt}\end{leftcolumn*}

\begin{rightcolumn}\PaliColumn{“Bhikkhus, when anyone has developed and cultivated mindfulness of the body, Māra cannot find an opportunity or a support in him.}
\hspace{0pt}
\end{rightcolumn}
\end{samepage}
\begin{samepage}
\begin{leftcolumn*}
\EnglishColumn{seyyathāpi, bhikkhave, puriso lahukaṁ suttaguḷaṁ sabbasāramaye aggaḷaphalake pakkhipeyya.}
\hspace{0pt}\end{leftcolumn*}

\begin{rightcolumn}\PaliColumn{Suppose a man were to throw a light ball of string at a door-panel made entirely of heartwood.}
\hspace{0pt}
\end{rightcolumn}
\end{samepage}
\begin{samepage}
\begin{leftcolumn*}
\EnglishColumn{taṁ kiṁ maññatha, bhikkhave,}
\hspace{0pt}\end{leftcolumn*}

\begin{rightcolumn}\PaliColumn{What do you think, bhikkhus?}
\hspace{0pt}
\end{rightcolumn}
\end{samepage}
\begin{samepage}
\begin{leftcolumn*}
\EnglishColumn{api nu so puriso taṁ lahukaṁ suttaguḷaṁ sabbasāramaye aggaḷaphalake labhetha otāran”ti?}
\hspace{0pt}\end{leftcolumn*}

\begin{rightcolumn}\PaliColumn{Would that light ball of string find entry through that door-panel made entirely of heartwood?”}
\hspace{0pt}
\end{rightcolumn}
\end{samepage}
\begin{samepage}
\begin{leftcolumn*}
\EnglishColumn{“no hetaṁ, bhante”.}
\hspace{0pt}\end{leftcolumn*}

\begin{rightcolumn}\PaliColumn{“No, venerable sir.”}
\hspace{0pt}
\end{rightcolumn}
\end{samepage}
\begin{samepage}
\begin{leftcolumn*}
\EnglishColumn{“evameva kho, bhikkhave, yassa kassaci kāyagatāsati bhāvitā bahulīkatā, na tassa labhati māro otāraṁ, na tassa labhati māro ārammaṇaṁ.}
\hspace{0pt}\end{leftcolumn*}

\begin{rightcolumn}\PaliColumn{“So too, bhikkhus, when anyone has developed and cultivated mindfulness of the body, Māra cannot find an opportunity or a support in him.}
\hspace{0pt}
\end{rightcolumn}
\end{samepage}
\begin{samepage}
\begin{leftcolumn*}
\EnglishColumn{seyyathāpi, bhikkhave, allaṁ kaṭṭhaṁ sasnehaṁ; atha puriso āgaccheyya uttarāraṇiṁ ādāya;}
\hspace{0pt}\end{leftcolumn*}

\begin{rightcolumn}\PaliColumn{“Suppose there were a wet sappy piece of wood, and a man came with an upper fire-stick, thinking:}
\hspace{0pt}
\end{rightcolumn}
\end{samepage}
\begin{samepage}
\begin{leftcolumn*}
\EnglishColumn{‘aggiṁ abhinibbattessāmi, tejo pātukarissāmī’ti.}
\hspace{0pt}\end{leftcolumn*}

\begin{rightcolumn}\PaliColumn{‘I shall light a fire, I shall produce heat.’}
\hspace{0pt}
\end{rightcolumn}
\end{samepage}
\begin{samepage}
\begin{leftcolumn*}
\EnglishColumn{taṁ kiṁ maññatha, bhikkhave,}
\hspace{0pt}\end{leftcolumn*}

\begin{rightcolumn}\PaliColumn{What do you think, bhikkhus?}
\hspace{0pt}
\end{rightcolumn}
\end{samepage}
\begin{samepage}
\begin{leftcolumn*}
\EnglishColumn{api nu so puriso amuṁ allaṁ kaṭṭhaṁ sasnehaṁ uttarāraṇiṁ ādāya abhimanthento aggiṁ abhinibbatteyya, tejo pātukareyyā”ti?}
\hspace{0pt}\end{leftcolumn*}

\begin{rightcolumn}\PaliColumn{Could the man light a fire and produce heat by taking the upper fire-stick and rubbing it against the wet sappy piece of wood?}
\hspace{0pt}
\end{rightcolumn}
\end{samepage}
\begin{samepage}
\begin{leftcolumn*}
\EnglishColumn{“no hetaṁ, bhante”.}
\hspace{0pt}\end{leftcolumn*}

\begin{rightcolumn}\PaliColumn{—“No, venerable sir.”}
\hspace{0pt}
\end{rightcolumn}
\end{samepage}
\begin{samepage}
\begin{leftcolumn*}
\EnglishColumn{“evameva kho, bhikkhave, yassa kassaci kāyagatāsati bhāvitā bahulīkatā, na tassa labhati māro otāraṁ, na tassa labhati māro ārammaṇaṁ.}
\hspace{0pt}\end{leftcolumn*}

\begin{rightcolumn}\PaliColumn{“So too, bhikkhus, when anyone has developed and cultivated mindfulness of the body, Māra cannot find an opportunity or a support in him.}
\hspace{0pt}
\end{rightcolumn}
\end{samepage}
\begin{samepage}
\begin{leftcolumn*}
\EnglishColumn{seyyathāpi, bhikkhave, udakamaṇiko pūro udakassa samatittiko kākapeyyo ādhāre ṭhapito; atha puriso āgaccheyya udakabhāraṁ ādāya.}
\hspace{0pt}\end{leftcolumn*}

\begin{rightcolumn}\PaliColumn{“Suppose, set out on a stand, there were a water jug full of water right up to the brim so that crows could drink from it, and a man came with a supply of water.}
\hspace{0pt}
\end{rightcolumn}
\end{samepage}
\begin{samepage}
\begin{leftcolumn*}
\EnglishColumn{taṁ kiṁ maññatha, bhikkhave,}
\hspace{0pt}\end{leftcolumn*}

\begin{rightcolumn}\PaliColumn{What do you think, bhikkhus?}
\hspace{0pt}
\end{rightcolumn}
\end{samepage}
\begin{samepage}
\begin{leftcolumn*}
\EnglishColumn{api nu so puriso labhetha udakassa nikkhepanan”ti?}
\hspace{0pt}\end{leftcolumn*}

\begin{rightcolumn}\PaliColumn{Could the man pour the water into the jug?”}
\hspace{0pt}
\end{rightcolumn}
\end{samepage}
\begin{samepage}
\begin{leftcolumn*}
\EnglishColumn{“no hetaṁ, bhante”.}
\hspace{0pt}\end{leftcolumn*}

\begin{rightcolumn}\PaliColumn{“No, venerable sir.”}
\hspace{0pt}
\end{rightcolumn}
\end{samepage}
\begin{samepage}
\begin{leftcolumn*}
\EnglishColumn{“evameva kho, bhikkhave, yassa kassaci kāyagatāsati bhāvitā bahulīkatā, na tassa labhati māro otāraṁ, na tassa labhati māro ārammaṇaṁ”.}
\hspace{0pt}\end{leftcolumn*}

\begin{rightcolumn}\PaliColumn{“So too, bhikkhus, when anyone has developed and cultivated mindfulness of the body, Māra cannot find an opportunity or a support in him.}
\hspace{0pt}
\end{rightcolumn}
\end{samepage}
\begin{samepage}
\begin{leftcolumn*}
\EnglishColumn{158. “yassa kassaci, bhikkhave, kāyagatāsati bhāvitā bahulīkatā, so yassa yassa abhiññāsacchikaraṇīyassa dhammassa cittaṁ abhininnāmeti abhiññāsacchikiriyāya, ta tatre sakkhibhabbataṁ pāpuṇāti sati satiāyatane.}
\hspace{0pt}\end{leftcolumn*}

\begin{rightcolumn}\PaliColumn{“Bhikkhus, when anyone has developed and cultivated mindfulness of the body, then when he inclines his mind towards realising any state that may be realised by direct knowledge, he attains the ability to witness any aspect therein, there being a suitable basis.}
\hspace{0pt}
\end{rightcolumn}
\end{samepage}
\begin{samepage}
\begin{leftcolumn*}
\EnglishColumn{seyyathāpi, bhikkhave, udakamaṇiko pūro udakassa samatittiko kākapeyyo ādhāre ṭhapito.}
\hspace{0pt}\end{leftcolumn*}

\begin{rightcolumn}\PaliColumn{Suppose, set out on a stand, there were a water jug full of water right up to the brim so that crows could drink from it.}
\hspace{0pt}
\end{rightcolumn}
\end{samepage}
\begin{samepage}
\begin{leftcolumn*}
\EnglishColumn{tamenaṁ balavā puriso yato yato āviñcheyya, āgaccheyya udakan”ti?}
\hspace{0pt}\end{leftcolumn*}

\begin{rightcolumn}\PaliColumn{Whenever a strong man tips it, would water come out?”}
\hspace{0pt}
\end{rightcolumn}
\end{samepage}
\begin{samepage}
\begin{leftcolumn*}
\EnglishColumn{“evaṁ, bhante”.}
\hspace{0pt}\end{leftcolumn*}

\begin{rightcolumn}\PaliColumn{“Yes, venerable sir.”}
\hspace{0pt}
\end{rightcolumn}
\end{samepage}
\begin{samepage}
\begin{leftcolumn*}
\EnglishColumn{“evameva kho, bhikkhave, yassa kassaci kāyagatāsati bhāvitā bahulīkatā so, yassa yassa abhiññāsacchikaraṇīyassa dhammassa cittaṁ abhininnāmeti abhiññāsacchikiriyāya, tatra tatreva sakkhibhabbataṁ pāpuṇāti sati satiāyatane.}
\hspace{0pt}\end{leftcolumn*}

\begin{rightcolumn}\PaliColumn{“So too, bhikkhus, when anyone has developed and cultivated mindfulness of the body, then when he inclines his mind towards realising any state that may be realised by direct knowledge, he attains the ability to witness any aspect therein, there being a suitable basis.}
\hspace{0pt}
\end{rightcolumn}
\end{samepage}
\begin{samepage}
\begin{leftcolumn*}
\EnglishColumn{seyyathāpi, bhikkhave, same bhūmibhāge caturassā pokkharaṇī assa āḷibandhā pūrā udakassa samatittikā kākapeyyā.}
\hspace{0pt}\end{leftcolumn*}

\begin{rightcolumn}\PaliColumn{“Suppose there were a square pond on level ground, surrounded by an embankment, full of water right up to the brim so that crows could drink from it.}
\hspace{0pt}
\end{rightcolumn}
\end{samepage}
\begin{samepage}
\begin{leftcolumn*}
\EnglishColumn{tamenaṁ balavā puriso yato yato āḷiṁ muñceyya āgaccheyya udakan”ti?}
\hspace{0pt}\end{leftcolumn*}

\begin{rightcolumn}\PaliColumn{Whenever a strong man loosens the embankment, would water come out?}
\hspace{0pt}
\end{rightcolumn}
\end{samepage}
\begin{samepage}
\begin{leftcolumn*}
\EnglishColumn{“evaṁ, bhante”.}
\hspace{0pt}\end{leftcolumn*}

\begin{rightcolumn}\PaliColumn{“Yes, venerable sir.”}
\hspace{0pt}
\end{rightcolumn}
\end{samepage}
\begin{samepage}
\begin{leftcolumn*}
\EnglishColumn{“evameva kho, bhikkhave, yassa kassaci kāyagatāsati bhāvitā bahulīkatā, so yassa yassa abhiññāsacchikaraṇīyassa dhammassa cittaṁ abhininnāmeti abhiññāsacchikiriyāya, tatra tatreva sakkhibhabbataṁ pāpuṇāti sati satiāyatane.}
\hspace{0pt}\end{leftcolumn*}

\begin{rightcolumn}\PaliColumn{“So too, bhikkhus, when anyone has developed and cultivated mindfulness of the body, then when he inclines his mind towards realising any state that may be realised by direct knowledge, he attains the ability to witness any aspect therein, there being a suitable basis.}
\hspace{0pt}
\end{rightcolumn}
\end{samepage}
\begin{samepage}
\begin{leftcolumn*}
\EnglishColumn{seyyathāpi, bhikkhave, subhūmiyaṁ catumahāpathe ājaññaratho yutto assa ṭhito odhastapatodo; tamenaṁ dakkho yoggācariyo assadammasārathi abhiruhitvā vāmena hatthena rasmiyo gahetvā dakkhiṇena hatthena patodaṁ gahetvā yenicchakaṁ yadicchakaṁ sāreyyāpi paccāsāreyyāpi;}
\hspace{0pt}\end{leftcolumn*}

\begin{rightcolumn}\PaliColumn{“Suppose there were a chariot on even ground at the crossroads, harnessed to thoroughbreds, waiting with goad lying ready, so that a skilled trainer, a charioteer of horses to be tamed, might mount it, and taking the reins in his left hand and the goad in his right hand, might drive out and back by any road whenever he likes.}
\hspace{0pt}
\end{rightcolumn}
\end{samepage}
\begin{samepage}
\begin{leftcolumn*}
\EnglishColumn{evameva kho, bhikkhave, yassa kassaci kāyagatāsati bhāvitā bahulīkatā, so yassa yassa abhiññāsacchikaraṇīyassa dhammassa cittaṁ abhininnāmeti abhiññāsacchikiriyāya, tatra tatreva sakkhibhabbataṁ pāpuṇāti sati satiāyatane”.}
\hspace{0pt}\end{leftcolumn*}

\begin{rightcolumn}\PaliColumn{So too, bhikkhus, when anyone has developed and cultivated mindfulness of the body, then when he inclines his mind towards realising any state that may be realised by direct knowledge, he attains the ability to witness any aspect therein, there being a suitable basis.}
\hspace{0pt}
\end{rightcolumn}
\end{samepage}
\begin{samepage}
\begin{leftcolumn*}
\EnglishColumn{159. “kāyagatāya, bhikkhave, satiyā āsevitāya bhāvitāya bahulīkatāya yānīkatāya vatthukatāya anuṭṭhitāya paricitāya susamāraddhāya dasānisaṁsā pāṭikaṅkhā.}
\hspace{0pt}\end{leftcolumn*}

\begin{rightcolumn}\PaliColumn{“Bhikkhus, when mindfulness of the body has been repeatedly practised, developed, cultivated, used as a vehicle, used as a basis, established, consolidated, and well undertaken, these ten benefits may be expected. What ten?}
\hspace{0pt}
\end{rightcolumn}
\end{samepage}
\begin{samepage}
\begin{leftcolumn*}
\EnglishColumn{(i) "aratiratisaho hoti, na ca taṁ arati sahati, uppannaṁ aratiṁ abhibhuyya viharati.}
\hspace{0pt}\end{leftcolumn*}

\begin{rightcolumn}\PaliColumn{(i) “One becomes a conqueror of discontent and delight, and discontent does not conquer oneself; one abides overcoming discontent whenever it arises.}
\hspace{0pt}
\end{rightcolumn}
\end{samepage}
\begin{samepage}
\begin{leftcolumn*}
\EnglishColumn{(ii) “bhayabheravasaho hoti, na ca taṁ bhayabheravaṁ sahati, uppannaṁ bhayabheravaṁ abhibhuyya viharati.}
\hspace{0pt}\end{leftcolumn*}

\begin{rightcolumn}\PaliColumn{(ii) “One becomes a conqueror of fear and dread, and fear and dread do not conquer oneself; one abides overcoming fear and dread whenever they arise.}
\hspace{0pt}
\end{rightcolumn}
\end{samepage}
\begin{samepage}
\begin{leftcolumn*}
\EnglishColumn{(iii) “khamo hoti sītassa uṇhassa jighacchāya pipāsāya ḍaṁsamakasavātātapasarīsapasamphassānaṁ duruttānaṁ durāgatānaṁ vacanapathānaṁ, uppannānaṁ sārīrikānaṁ vedanānaṁ dukkhānaṁ tibbānaṁ kharānaṁ kaṭukānaṁ asātānaṁ amanāpānaṁ pāṇaharānaṁ adhivāsakajātiko hoti.}
\hspace{0pt}\end{leftcolumn*}

\begin{rightcolumn}\PaliColumn{(iii) “One bears cold and heat, hunger and thirst, and contact with gadflies, mosquitoes, wind, the sun, and creeping things; one endures ill-spoken, unwelcome words and arisen bodily feelings that are painful, racking, sharp, piercing, disagreeable, distressing, and menacing to life.}
\hspace{0pt}
\end{rightcolumn}
\end{samepage}
\begin{samepage}
\begin{leftcolumn*}
\EnglishColumn{(iv) “catunnaṁ jhānānaṁ ābhicetasikānaṁ diṭṭhadhammasukhavihārānaṁ nikāmalābhī hoti akicchalābhī akasiralābhī.}
\hspace{0pt}\end{leftcolumn*}

\begin{rightcolumn}\PaliColumn{(iv) “One obtains at will, without trouble or difficulty, the four jhānas that constitute the higher mind and  provide a pleasant abiding here and now.}
\hspace{0pt}
\end{rightcolumn}
\end{samepage}
\begin{samepage}
\begin{leftcolumn*}
\EnglishColumn{(v) “so anekavihitaṁ iddhividhaṁ paccānubhoti.}
\hspace{0pt}\end{leftcolumn*}

\begin{rightcolumn}\PaliColumn{(v) “One wields the various kinds of supernormal power:}
\hspace{0pt}
\end{rightcolumn}
\end{samepage}
\begin{samepage}
\begin{leftcolumn*}
\EnglishColumn{ekopi hutvā bahudhā hoti, bahudhāpi hutvā eko hoti,}
\hspace{0pt}\end{leftcolumn*}

\begin{rightcolumn}\PaliColumn{having been one, he becomes many; having been many, he becomes one;}
\hspace{0pt}
\end{rightcolumn}
\end{samepage}
\begin{samepage}
\begin{leftcolumn*}
\EnglishColumn{āvibhāvaṁ tirobhāvaṁ; tirokuṭṭaṁ tiropākāraṁ tiropabbataṁ asajjamāno gacchati, seyyathāpi ākāse;}
\hspace{0pt}\end{leftcolumn*}

\begin{rightcolumn}\PaliColumn{one appears and vanishes; one goes unhindered through a wall, through an enclosure, through a mountain as though through space;}
\hspace{0pt}
\end{rightcolumn}
\end{samepage}
\begin{samepage}
\begin{leftcolumn*}
\EnglishColumn{pathaviyāpi ummujjanimujjaṁ karoti, seyyathāpi udake;}
\hspace{0pt}\end{leftcolumn*}

\begin{rightcolumn}\PaliColumn{one dives in and out of the earth as though it were water;}
\hspace{0pt}
\end{rightcolumn}
\end{samepage}
\begin{samepage}
\begin{leftcolumn*}
\EnglishColumn{udakepi abhijjamāne gacchati, seyyathāpi pathaviyaṁ;}
\hspace{0pt}\end{leftcolumn*}

\begin{rightcolumn}\PaliColumn{one walks on water without sinking as though it were earth;}
\hspace{0pt}
\end{rightcolumn}
\end{samepage}
\begin{samepage}
\begin{leftcolumn*}
\EnglishColumn{ākāsepi pallaṅkena kamati, seyyathāpi pakkhī sakuṇo;}
\hspace{0pt}\end{leftcolumn*}

\begin{rightcolumn}\PaliColumn{seated cross-legged, one travels in space like a bird;}
\hspace{0pt}
\end{rightcolumn}
\end{samepage}
\begin{samepage}
\begin{leftcolumn*}
\EnglishColumn{imepi candimasūriye evaṁmahiddhike evaṁmahānubhāve pāṇinā parimasati parimajjati,}
\hspace{0pt}\end{leftcolumn*}

\begin{rightcolumn}\PaliColumn{with his hand one touches and strokes the moon and sun so powerful and mighty;}
\hspace{0pt}
\end{rightcolumn}
\end{samepage}
\begin{samepage}
\begin{leftcolumn*}
\EnglishColumn{yāva brahmalokāpi kāyena vasaṁ vatteti.}
\hspace{0pt}\end{leftcolumn*}

\begin{rightcolumn}\PaliColumn{one wields bodily mastery even as far as the Brahma-world.}
\hspace{0pt}
\end{rightcolumn}
\end{samepage}
\begin{samepage}
\begin{leftcolumn*}
\EnglishColumn{(vi) “dibbāya sotadhātuyā visuddhāya atikkantamānusikāya ubho sadde suṇāti dibbe ca mānuse ca, ye dūre santike ca.}
\hspace{0pt}\end{leftcolumn*}

\begin{rightcolumn}\PaliColumn{(vi) “With the divine ear element, which is purified and surpasses the human, one hears both kinds of sounds, the divine and the human, those that are far as well as near.}
\hspace{0pt}
\end{rightcolumn}
\end{samepage}
\begin{samepage}
\begin{leftcolumn*}
\EnglishColumn{(vii) “parasattānaṁ parapuggalānaṁ cetasā ceto paricca pajānāti.}
\hspace{0pt}\end{leftcolumn*}

\begin{rightcolumn}\PaliColumn{(vii) “One understands the minds of other beings, of other persons, having encompassed them with one’s own mind.}
\hspace{0pt}
\end{rightcolumn}
\end{samepage}
\begin{samepage}
\begin{leftcolumn*}
\EnglishColumn{sarāgaṁ vā cittaṁ ‘sarāgaṁ cittan’ti pajānāti, vītarāgaṁ vā cittaṁ ‘vītarāgaṁ cittan’ti pajānāti,}
\hspace{0pt}\end{leftcolumn*}

\begin{rightcolumn}\PaliColumn{One understands a mind affected by lust as affected by lust and a mind unaffected by lust as unaffected by lust;}
\hspace{0pt}
\end{rightcolumn}
\end{samepage}
\begin{samepage}
\begin{leftcolumn*}
\EnglishColumn{sadosaṁ vā cittaṁ ‘sadosaṁ cittan’ti pajānāti, vītadosaṁ vā cittaṁ ‘vītadosaṁ cittan’ti pajānāti,}
\hspace{0pt}\end{leftcolumn*}

\begin{rightcolumn}\PaliColumn{one understands a mind affected by hate as affected by hate and a mind unaffected by hate as unaffected by hate;}
\hspace{0pt}
\end{rightcolumn}
\end{samepage}
\begin{samepage}
\begin{leftcolumn*}
\EnglishColumn{samohaṁ vā cittaṁ ‘samohaṁ cittan’ti pajānāti, vītamohaṁ vā cittaṁ ‘vītamohaṁ cittan’ti pajānāti,}
\hspace{0pt}\end{leftcolumn*}

\begin{rightcolumn}\PaliColumn{one understands a mind affected by delusion as affected by delusion and a mind unaffected by delusion as unaffected by delusion;}
\hspace{0pt}
\end{rightcolumn}
\end{samepage}
\begin{samepage}
\begin{leftcolumn*}
\EnglishColumn{saṁkhittaṁ vā cittaṁ ‘saṁkhittaṁ cittan’ti pajānāti, vikkhittaṁ vā cittaṁ ‘vikkhittaṁ cittan’ti pajānāti,}
\hspace{0pt}\end{leftcolumn*}

\begin{rightcolumn}\PaliColumn{one understands a contracted mind as contracted and a distracted mind as distracted;}
\hspace{0pt}
\end{rightcolumn}
\end{samepage}
\begin{samepage}
\begin{leftcolumn*}
\EnglishColumn{mahaggataṁ vā cittaṁ ‘mahaggataṁ cittan’ti pajānāti, amahaggataṁ vā cittaṁ ‘amahaggataṁ cittan’ti pajānāti,}
\hspace{0pt}\end{leftcolumn*}

\begin{rightcolumn}\PaliColumn{one understands an exalted mind as exalted and an unexalted mind as unexalted;}
\hspace{0pt}
\end{rightcolumn}
\end{samepage}
\begin{samepage}
\begin{leftcolumn*}
\EnglishColumn{sauttaraṁ vā cittaṁ ‘sauttaraṁ cittan’ti pajānāti, anuttaraṁ vā cittaṁ ‘anuttaraṁ cittan’ti pajānāti,}
\hspace{0pt}\end{leftcolumn*}

\begin{rightcolumn}\PaliColumn{one understands a surpassed mind as surpassed and an unsurpassed mind as unsurpassed;}
\hspace{0pt}
\end{rightcolumn}
\end{samepage}
\begin{samepage}
\begin{leftcolumn*}
\EnglishColumn{samāhitaṁ vā cittaṁ ‘samāhitaṁ cittan’ti pajānāti, asamāhitaṁ vā cittaṁ ‘asamāhitaṁ cittan’ti pajānāti,}
\hspace{0pt}\end{leftcolumn*}

\begin{rightcolumn}\PaliColumn{one understands a concentrated mind as concentrated and an unconcentrated mind as unconcentrated;}
\hspace{0pt}
\end{rightcolumn}
\end{samepage}
\begin{samepage}
\begin{leftcolumn*}
\EnglishColumn{vimuttaṁ vā cittaṁ ‘vimuttaṁ cittan’ti pajānāti, avimuttaṁ vā cittaṁ ‘avimuttaṁ cittan’ti pajānāti.}
\hspace{0pt}\end{leftcolumn*}

\begin{rightcolumn}\PaliColumn{one understands a liberated mind as liberated and an unliberated mind as unliberated.}
\hspace{0pt}
\end{rightcolumn}
\end{samepage}
\begin{samepage}
\begin{leftcolumn*}
\EnglishColumn{(viii) “so anekavihitaṁ pubbenivāsaṁ anussarati, seyyathidaṁ — ekampi jātiṁ dvepi jātiyo tissopi jātiyo catassopi jātiyo pañcapi jātiyo dasapi jātiyo vīsampi jātiyo tiṁsampi jātiyo cattārīsampi jātiyo paññāsampi jātiyo jātisatampi jātisahassampi jātisatasahassampi}
\hspace{0pt}\end{leftcolumn*}

\begin{rightcolumn}\PaliColumn{(viii) “One recollects ones manifold past lives, that is, one birth, two births, three births, four births, five births, ten births, twenty births, thirty births, forty births, fifty births, a hundred births, a thousand births, a hundred thousand births,}
\hspace{0pt}
\end{rightcolumn}
\end{samepage}
\begin{samepage}
\begin{leftcolumn*}
\EnglishColumn{anekepi saṁvaṭṭakappe anekepi vivaṭṭakappe anekepi saṁvaṭṭavivaṭṭakappe;}
\hspace{0pt}\end{leftcolumn*}

\begin{rightcolumn}\PaliColumn{many aeons of world-contraction, many aeons of world-expansion, many aeons of world-contraction and expansion:}
\hspace{0pt}
\end{rightcolumn}
\end{samepage}
\begin{samepage}
\begin{leftcolumn*}
\EnglishColumn{‘amutrāsiṁ evaṁnāmo evaṁgotto evaṁvaṇṇo evamāhāro evaṁsukhadukkhappaṭisaṁvedī evamāyupariyanto,}
\hspace{0pt}\end{leftcolumn*}

\begin{rightcolumn}\PaliColumn{‘There I was so named, of such a clan, with such an appearance, such was my nutriment, such my experience of pleasure and pain, such my life-term;}
\hspace{0pt}
\end{rightcolumn}
\end{samepage}
\begin{samepage}
\begin{leftcolumn*}
\EnglishColumn{so tato cuto amutra udapādiṁ;}
\hspace{0pt}\end{leftcolumn*}

\begin{rightcolumn}\PaliColumn{and passing away from there, I reappeared elsewhere;}
\hspace{0pt}
\end{rightcolumn}
\end{samepage}
\begin{samepage}
\begin{leftcolumn*}
\EnglishColumn{tatrāpāsiṁ evaṁnāmo evaṁgotto evaṁvaṇṇo evamāhāro evaṁsukhadukkhappaṭisaṁvedī evamāyupariyanto,}
\hspace{0pt}\end{leftcolumn*}

\begin{rightcolumn}\PaliColumn{and there too I was so named, of such a clan, with such an appearance, such was my nutriment, such my experience of pleasure and pain, such my life-term;}
\hspace{0pt}
\end{rightcolumn}
\end{samepage}
\begin{samepage}
\begin{leftcolumn*}
\EnglishColumn{so tato cuto idhūpapanno’ti.}
\hspace{0pt}\end{leftcolumn*}

\begin{rightcolumn}\PaliColumn{and passing away from there, I reappeared here.’}
\hspace{0pt}
\end{rightcolumn}
\end{samepage}
\begin{samepage}
\begin{leftcolumn*}
\EnglishColumn{iti sākāraṁ sauddesaṁ anekavihitaṁ pubbenivāsaṁ anussarati.}
\hspace{0pt}\end{leftcolumn*}

\begin{rightcolumn}\PaliColumn{Thus with their aspects and particulars one recollects ones manifold past lives.}
\hspace{0pt}
\end{rightcolumn}
\end{samepage}
\begin{samepage}
\begin{leftcolumn*}
\EnglishColumn{(ix) “dibbena cakkhunā visuddhena atikkantamānusakena satte passati cavamāne upapajjamāne hīne paṇīte suvaṇṇe dubbaṇṇe, sugate duggate yathākammūpage satte pajānāti.}
\hspace{0pt}\end{leftcolumn*}

\begin{rightcolumn}\PaliColumn{(ix) “With the divine eye, which is purified and surpasses the human, one sees beings passing away and reappearing, inferior and superior, fair and ugly, fortunate and unfortunate, and one understands how beings pass on according to their actions.}
\hspace{0pt}
\end{rightcolumn}
\end{samepage}
\begin{samepage}
\begin{leftcolumn*}
\EnglishColumn{(x) “āsavānaṁ khayā anāsavaṁ cetovimuttiṁ paññāvimuttiṁ diṭṭheva dhamme sayaṁ abhiññā sacchikatvā upasampajja viharati.}
\hspace{0pt}\end{leftcolumn*}

\begin{rightcolumn}\PaliColumn{(x) “By realising for oneself with direct knowledge, one here and now enters upon and abides in the deliverance of mind and deliverance by wisdom that are taintless with the destruction of the taints.}
\hspace{0pt}
\end{rightcolumn}
\end{samepage}
\begin{samepage}
\begin{leftcolumn*}
\EnglishColumn{“kāyagatāya, bhikkhave, satiyā āsevitāya bhāvitāya bahulīkatāya yānīkatāya vatthukatāya anuṭṭhitāya paricitāya susamāraddhāya ime dasānisaṁsā pāṭikaṅkhā”ti.}
\hspace{0pt}\end{leftcolumn*}

\begin{rightcolumn}\PaliColumn{“Bhikkhus, when mindfulness of the body has been repeatedly practised, developed, cultivated, used as a vehicle, used as a basis, established, consolidated, and well undertaken, these ten benefits may be expected.”}
\hspace{0pt}
\end{rightcolumn}
\end{samepage}
\begin{samepage}
\begin{leftcolumn*}
\EnglishColumn{idamavoca bhagavā. attamanā te bhikkhū bhagavato bhāsitaṁ abhinandunti.}
\hspace{0pt}\end{leftcolumn*}

\begin{rightcolumn}\PaliColumn{That is what the Blessed One said. The bhikkhus were satisfied and delighted in the Blessed One’s words.}
\hspace{0pt}
\end{rightcolumn}
\end{samepage}
\begin{samepage}
\begin{leftcolumn*}
\EnglishColumn{kāyagatāsatisuttaṁ niṭṭhitaṁ navamaṁ.}
\hspace{0pt}\end{leftcolumn*}

\begin{rightcolumn}\PaliColumn{Contemplation of the body, concludes, 11(9)}
\hspace{0pt}
\end{rightcolumn}
\end{samepage}