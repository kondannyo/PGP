
\begin{samepage}
\begingl[glneveryline={\PaliGlossA,\PaliGlossB}]
Majjhima[middle] Nikāya,[collection] uparipaṇṇāsapāḷi,[-] 2.[-] anupadavaggo,[-] 9.[-] kāyagatāsatisuttaṃ[-] (MN[-] 119)[-]
\endgl
\nopagebreak
\linespread{0.5}
\begin{spacin}{0.2}
{\PaliGlossFT Mindfulness of the Body}
\end{spacin}
\vskip 12pt
\end{samepage}
\vskip 0.2in
\begin{samepage}
\begingl[glneveryline={\PaliGlossA,\PaliGlossB}]
153.[-] evaṁ[thus] me[to me] sutaṁ.[hear] ekaṁ[one] samayaṁ[time] bhagavā[blessed] sāvatthiyaṁ[in Kosala] viharati[dwells] jetavane[jetagrove] anāthapiṇḍikassa[anāthapiṇḍika] ārāme.[park]
\endgl
\nopagebreak
\linespread{0.5}
\begin{spacin}{0.2}
{\PaliGlossFT Thus have I heard. On one occasion the Blessed One was living at S̄āvattī in Jeta’s Grove, Anāthapiṇḍika’s Park.}
\end{spacin}
\vskip 12pt
\end{samepage}
\begin{samepage}
\begingl[glneveryline={\PaliGlossA,\PaliGlossB}]
atha[then] kho[indeed] sambahulānaṁ[many] bhikkhūnaṁ[-] pacchābhattaṁ[afternoon] piṇḍapātapaṭikkantānaṁ[alms.back from] upaṭṭhānasālāyaṁ[assembly hall] sannisinnānaṁ[settled] sannipatitānaṁ[assembled] ayamantarākathā[them.discussion] udapādi;[arose]
\endgl
\nopagebreak
\linespread{0.5}
\begin{spacin}{0.2}
{\PaliGlossFT Now a number of bhikkhus were sitting in the assembly hall, where they had met together on returning from their almsround, after their meal, when this discussion arose among them:}
\end{spacin}
\vskip 12pt
\end{samepage}
\begin{samepage}
\begingl[glneveryline={\PaliGlossA,\PaliGlossB}]
“acchariyaṁ,[wonderful] āvuso,[friend] abbhutaṁ,[marvelous] āvuso![friend] yāvañcidaṁ[as far as] tena[because of] bhagavatā[blessed] jānatā[know] passatā[see] arahatā[worthy] sammāsambuddhena[perfect.enlightened] kāyagatāsati[body.direct.mindful] bhāvitā[developed] bahulīkatā[increased] mahapphalā[great.fruit] vuttā[habit] mahānisaṁsā”ti.[great.benifit]
\endgl
\nopagebreak
\linespread{0.5}
\begin{spacin}{0.2}
{\PaliGlossFT “It is wonderful, friends, it is marvellous, how it has been said by the Blessed One who knows and sees, accomplished and fully enlightened, that mindfulness of the body, when developed and cultivated, is of great fruit and great benefit.”}
\end{spacin}
\vskip 12pt
\end{samepage}
\begin{samepage}
\begingl[glneveryline={\PaliGlossA,\PaliGlossB}]
ayañca[then.and] hidaṁ[this] tesaṁ[that] bhikkhūnaṁ[-] antarākathā[between talk] vippakatā[unfinished] hoti,[to be] atha[then] kho[indeed] bhagavā[blessed] sāyanhasamayaṁ[evening] paṭisallānā[seclusion] vuṭṭhito[emerged from] yena[because of] upaṭṭhānasālā[assembly hall] tenupasaṅkami;[approach] upasaṅkamitvā[approached] paññatte[prepared] āsane[seat] nisīdi.[sat down]
\endgl
\nopagebreak
\linespread{0.5}
\begin{spacin}{0.2}
{\PaliGlossFT However, their discussion was interrupted; for the Blessed One rose from meditation when it was evening, went to the assembly hall, and sat down on a seat made ready.}
\end{spacin}
\vskip 12pt
\end{samepage}
\begin{samepage}
\begingl[glneveryline={\PaliGlossA,\PaliGlossB}]
nisajja[having sat] kho[indeed] bhagavā[blessed] bhikkhū[-] āmantesi;[address] “kāya[body] nuttha,[(affirm. part.)] bhikkhave,[-] etarahi[at present] kathāya[talk] sannisinnā,[settled] kā[which] ca[and] pana[yet] vo[to you] antarākathā[between talk] vippakatā”ti?[unfinished]
\endgl
\nopagebreak
\linespread{0.5}
\begin{spacin}{0.2}
{\PaliGlossFT Then he addressed the bhikkhus thus: “Bhikkhus, for what discussion are you sitting together here now? And what was your discussion that was interrupted?”}
\end{spacin}
\vskip 12pt
\end{samepage}
\begin{samepage}
\begingl[glneveryline={\PaliGlossA,\PaliGlossB}]
“idha,[here] bhante,[Sir] amhākaṁ[we are] pacchābhattaṁ[afternoon] piṇḍapātapaṭikkantānaṁ[alms.back from] upaṭṭhānasālāyaṁ[assembly hall] sannisinnānaṁ[settled] sannipatitānaṁ[assembled] ayamantarākathā[them.discussion] udapādi;[arose]
\endgl
\nopagebreak
\linespread{0.5}
\begin{spacin}{0.2}
{\PaliGlossFT “Here, venerable sir, we were sitting in the assembly hall, where we had met together on returning from our almsround, after our meal, when this discussion arose among us:}
\end{spacin}
\vskip 12pt
\end{samepage}
\begin{samepage}
\begingl[glneveryline={\PaliGlossA,\PaliGlossB}]
‘acchariyaṁ,[wonderful] āvuso,[friend] abbhutaṁ,[marvelous] āvuso![friend] yāvañcidaṁ[as far as] tena[because of] bhagavatā[blessed] jānatā[know] passatā[see] arahatā[worthy] sammāsambuddhena[perfect.enlightened] kāyagatāsati[body.direct.mindful] bhāvitā[developed] bahulīkatā[increased] mahapphalā[great.fruit] vuttā[habit] mahānisaṁsā’ti.[great.benifit]
\endgl
\nopagebreak
\linespread{0.5}
\begin{spacin}{0.2}
{\PaliGlossFT ‘It is wonderful, friends, it is marvellous, how it has been said by the Blessed One who knows and sees, accomplished and fully enlightened, that mindfulness of the body, when developed and cultivated, is of great fruit and great benefit.’}
\end{spacin}
\vskip 12pt
\end{samepage}
\begin{samepage}
\begingl[glneveryline={\PaliGlossA,\PaliGlossB}]
ayaṁ[this] kho[indeed] no,[(neg)] bhante,[Sir] antarākathā[between talk] vippakatā,[unfinished] atha[then] bhagavā[blessed] anuppatto”ti.[arrived]
\endgl
\nopagebreak
\linespread{0.5}
\begin{spacin}{0.2}
{\PaliGlossFT This was our discussion, venerable sir, that was interrupted when the Blessed One arrived.”}
\end{spacin}
\vskip 12pt
\end{samepage}
\vskip 0.2in
\begin{samepage}
\begingl[glneveryline={\PaliGlossA,\PaliGlossB}]
154.[-] “kathaṁ[and how] bhāvitā[developed] ca,[and] bhikkhave,[-] kāyagatāsati[body.direct.mindful] kathaṁ[and how] bahulīkatā[increased] mahapphalā[great.fruit] hoti[to be] mahānisaṁsā?[great.benifit]
\endgl
\nopagebreak
\linespread{0.5}
\begin{spacin}{0.2}
{\PaliGlossFT “And how, bhikkhus, is mindfulness of the body developed and cultivated so that it is of great fruit and great benefit?}
\end{spacin}
\vskip 12pt
\end{samepage}
\begin{samepage}
\begingl[glneveryline={\PaliGlossA,\PaliGlossB}]
idha,[here] bhikkhave,[-] bhikkhu[-] araññagato[go forest] vā[or] rukkhamūlagato[foot tree] vā[or] suññāgāragato[empty.house.go] vā[or] nisīdati[sit down] pallaṅkaṁ[cross-leg] ābhujitvā[bent] ujuṁ[up-right] kāyaṁ[the body] paṇidhāya[have aspired] parimukhaṁ[round.mouth] satiṁ[mindful] upaṭṭhapetvā.[put forth]
\endgl
\nopagebreak
\linespread{0.5}
\begin{spacin}{0.2}
{\PaliGlossFT “Here a bhikkhu, gone to the forest or to the root of a tree or to an empty hut, sits down; having folded his legs crosswise, set his body erect, and established mindfulness in front of him,}
\end{spacin}
\vskip 12pt
\end{samepage}
\begin{samepage}
\begingl[glneveryline={\PaliGlossA,\PaliGlossB}]
so[he] satova[mindful] assasati[inhales] satova[mindful] passasati;[exhales]
\endgl
\nopagebreak
\linespread{0.5}
\begin{spacin}{0.2}
{\PaliGlossFT ever mindful he breathes in, mindful he breathes out.}
\end{spacin}
\vskip 12pt
\end{samepage}
\begin{samepage}
\begingl[glneveryline={\PaliGlossA,\PaliGlossB}]
dīghaṁ[long] vā[or] assasanto[inhale] ‘dīghaṁ[long] assasāmī’ti[inhale] pajānāti,[know clearly]
\endgl
\nopagebreak
\linespread{0.5}
\begin{spacin}{0.2}
{\PaliGlossFT Breathing in long, he understands: ‘I breathe in long’;}
\end{spacin}
\vskip 12pt
\end{samepage}
\begin{samepage}
\begingl[glneveryline={\PaliGlossA,\PaliGlossB}]
dīghaṁ[long] vā[or] passasanto[exhale] ‘dīghaṁ[long] passasāmī’ti[exhale] pajānāti;[know clearly]
\endgl
\nopagebreak
\linespread{0.5}
\begin{spacin}{0.2}
{\PaliGlossFT or breathing out long, he understands: ‘I breathe out long.’}
\end{spacin}
\vskip 12pt
\end{samepage}
\begin{samepage}
\begingl[glneveryline={\PaliGlossA,\PaliGlossB}]
rassaṁ[short] vā[or] assasanto[inhale] ‘rassaṁ[short] assasāmī’ti[inhale] pajānāti,[know clearly]
\endgl
\nopagebreak
\linespread{0.5}
\begin{spacin}{0.2}
{\PaliGlossFT Breathing in short, he understands: ‘I breathe in short’;}
\end{spacin}
\vskip 12pt
\end{samepage}
\begin{samepage}
\begingl[glneveryline={\PaliGlossA,\PaliGlossB}]
rassaṁ[short] vā[or] passasanto[exhale] ‘rassaṁ[short] passasāmī’ti[exhale] pajānāti;[know clearly]
\endgl
\nopagebreak
\linespread{0.5}
\begin{spacin}{0.2}
{\PaliGlossFT or breathing out short, he understands: ‘I breathe out short.’}
\end{spacin}
\vskip 12pt
\end{samepage}
\begin{samepage}
\begingl[glneveryline={\PaliGlossA,\PaliGlossB}]
‘sabbakāyapaṭisaṁvedī[all.body.experience] assasissāmī’ti[inhales] sikkhati,[trains]
\endgl
\nopagebreak
\linespread{0.5}
\begin{spacin}{0.2}
{\PaliGlossFT He trains thus: ‘I shall breathe in experiencing the whole body’;}
\end{spacin}
\vskip 12pt
\end{samepage}
\begin{samepage}
\begingl[glneveryline={\PaliGlossA,\PaliGlossB}]
‘sabbakāyapaṭisaṁvedī[all.body.experience] passasissāmī’ti[exhales] sikkhati;[trains]
\endgl
\nopagebreak
\linespread{0.5}
\begin{spacin}{0.2}
{\PaliGlossFT he trains thus: ‘I shall breathe out experiencing the whole body.}
\end{spacin}
\vskip 12pt
\end{samepage}
\begin{samepage}
\begingl[glneveryline={\PaliGlossA,\PaliGlossB}]
‘passambhayaṁ[calms down] kāyasaṅkhāraṁ[body.formation] assasissāmī’ti[inhales] sikkhati,[trains]
\endgl
\nopagebreak
\linespread{0.5}
\begin{spacin}{0.2}
{\PaliGlossFT He trains thus: ‘I shall breathe in tranquillising the bodily formation’;}
\end{spacin}
\vskip 12pt
\end{samepage}
\begin{samepage}
\begingl[glneveryline={\PaliGlossA,\PaliGlossB}]
‘passambhayaṁ[calms down] kāyasaṅkhāraṁ[body.formation] passasissāmī’ti[exhales] sikkhati.[trains]
\endgl
\nopagebreak
\linespread{0.5}
\begin{spacin}{0.2}
{\PaliGlossFT he trains thus: ‘I shall breathe out tranquillising the bodily formation.’}
\end{spacin}
\vskip 12pt
\end{samepage}
\begin{samepage}
\begingl[glneveryline={\PaliGlossA,\PaliGlossB}]
tassa[he] evaṁ[thus] appamattassa[careful] ātāpino[ardent] pahitattassa[able.truth] viharato[abides] ye[whatever] gehasitā[family life] sarasaṅkappā[memory.intention] te[his] pahīyanti.[abandoned]
\endgl
\nopagebreak
\linespread{0.5}
\begin{spacin}{0.2}
{\PaliGlossFT As he abides thus diligent, ardent, and resolute, his memories and intentions based on the household life are abandoned;}
\end{spacin}
\vskip 12pt
\end{samepage}
\begin{samepage}
\begingl[glneveryline={\PaliGlossA,\PaliGlossB}]
tesaṁ[that] pahānā[removal] ajjhattameva[internally] cittaṁ[mind] santiṭṭhati[steadied] sannisīdati[settled] ekodi[single] hoti[to be] samādhiyati.[concentrated]
\endgl
\nopagebreak
\linespread{0.5}
\begin{spacin}{0.2}
{\PaliGlossFT with their abandoning his mind becomes steadied internally, quieted, brought to singleness, and concentrated.}
\end{spacin}
\vskip 12pt
\end{samepage}
\begin{samepage}
\begingl[glneveryline={\PaliGlossA,\PaliGlossB}]
evaṁ,[thus] bhikkhave,[-] bhikkhu[-] kāyagatāsatiṁ[relate to body] bhāveti.[develops]
\endgl
\nopagebreak
\linespread{0.5}
\begin{spacin}{0.2}
{\PaliGlossFT That is how a bhikkhu develops mindfulness of the body.}
\end{spacin}
\vskip 12pt
\end{samepage}
\vskip 0.2in
\begin{samepage}
\begingl[glneveryline={\PaliGlossA,\PaliGlossB}]
“puna[again] caparaṁ,[then] bhikkhave,[-] bhikkhu[-] gacchanto[walking] vā[or] ‘gacchāmī’ti[walk.I] pajānāti,[know clearly]
\endgl
\nopagebreak
\linespread{0.5}
\begin{spacin}{0.2}
{\PaliGlossFT “Again, bhikkhus, when walking, a bhikkhu understands: ‘I am walking’;}
\end{spacin}
\vskip 12pt
\end{samepage}
\begin{samepage}
\begingl[glneveryline={\PaliGlossA,\PaliGlossB}]
ṭhito[stand] vā[or] ‘ṭhitomhī’ti[stand.I] pajānāti,[know clearly]
\endgl
\nopagebreak
\linespread{0.5}
\begin{spacin}{0.2}
{\PaliGlossFT when standing, he understands: ‘I am standing’;}
\end{spacin}
\vskip 12pt
\end{samepage}
\begin{samepage}
\begingl[glneveryline={\PaliGlossA,\PaliGlossB}]
nisinno[sat] vā[or] ‘nisinnomhī’ti[sit.I] pajānāti,[know clearly]
\endgl
\nopagebreak
\linespread{0.5}
\begin{spacin}{0.2}
{\PaliGlossFT when sitting, he understands: ‘I am sitting’;}
\end{spacin}
\vskip 12pt
\end{samepage}
\begin{samepage}
\begingl[glneveryline={\PaliGlossA,\PaliGlossB}]
sayāno[sleeping] vā[or] ‘sayānomhī’ti[sleep.I] pajānāti.[know clearly]
\endgl
\nopagebreak
\linespread{0.5}
\begin{spacin}{0.2}
{\PaliGlossFT when lying down, he understands: ‘I am lying down’;}
\end{spacin}
\vskip 12pt
\end{samepage}
\begin{samepage}
\begingl[glneveryline={\PaliGlossA,\PaliGlossB}]
yathā[as] yathā[as] vā[or] panassa[-] kāyo[body] paṇihito[directed] hoti,[to be] tathā[so] tathā[so] naṁ[not] pajānāti.[know clearly]
\endgl
\nopagebreak
\linespread{0.5}
\begin{spacin}{0.2}
{\PaliGlossFT or he understands accordingly however his body is disposed.}
\end{spacin}
\vskip 12pt
\end{samepage}
\begin{samepage}
\begingl[glneveryline={\PaliGlossA,\PaliGlossB}]
tassa[he] evaṁ[thus] appamattassa[careful] ātāpino[ardent] pahitattassa[able.truth] viharato[abides] ye[whatever] gehasitā[family life] sarasaṅkappā[memory.intention] te[his] pahīyanti.[abandoned]
\endgl
\nopagebreak
\linespread{0.5}
\begin{spacin}{0.2}
{\PaliGlossFT As he abides thus diligent, ardent, and resolute, his memories and intentions based on the household life are abandoned;}
\end{spacin}
\vskip 12pt
\end{samepage}
\begin{samepage}
\begingl[glneveryline={\PaliGlossA,\PaliGlossB}]
tesaṁ[that] pahānā[removal] ajjhattameva[internally] cittaṁ[mind] santiṭṭhati[steadied] sannisīdati[settled] ekodi[single] hoti[to be] samādhiyati.[concentrated]
\endgl
\nopagebreak
\linespread{0.5}
\begin{spacin}{0.2}
{\PaliGlossFT with their abandoning his mind becomes steadied internally, quieted, brought to singleness, and concentrated.}
\end{spacin}
\vskip 12pt
\end{samepage}
\begin{samepage}
\begingl[glneveryline={\PaliGlossA,\PaliGlossB}]
evampi,[that.to] bhikkhave,[-] bhikkhu[-] kāyagatāsatiṁ[relate to body] bhāveti.[develops]
\endgl
\nopagebreak
\linespread{0.5}
\begin{spacin}{0.2}
{\PaliGlossFT That too is how a bhikkhu develops mindfulness of the body.}
\end{spacin}
\vskip 12pt
\end{samepage}
\vskip 0.2in
\begin{samepage}
\begingl[glneveryline={\PaliGlossA,\PaliGlossB}]
“puna[again] caparaṁ,[then] bhikkhave,[-] bhikkhu[-] abhikkante[approaching] paṭikkante[gone back] sampajānakārī[mindful] hoti,[to be]
\endgl
\nopagebreak
\linespread{0.5}
\begin{spacin}{0.2}
{\PaliGlossFT “Again, bhikkhus, a bhikkhu is one who acts in full awareness when going forward and returning;}
\end{spacin}
\vskip 12pt
\end{samepage}
\begin{samepage}
\begingl[glneveryline={\PaliGlossA,\PaliGlossB}]
ālokite[look ahead] vilokite[look back] sampajānakārī[mindful] hoti,[to be]
\endgl
\nopagebreak
\linespread{0.5}
\begin{spacin}{0.2}
{\PaliGlossFT who acts in full awareness when looking ahead and looking away;}
\end{spacin}
\vskip 12pt
\end{samepage}
\begin{samepage}
\begingl[glneveryline={\PaliGlossA,\PaliGlossB}]
samiñjite[moves] pasārite[stretch] sampajānakārī[mindful] hoti,[to be]
\endgl
\nopagebreak
\linespread{0.5}
\begin{spacin}{0.2}
{\PaliGlossFT who acts in full awareness when flexing and extending his limbs;}
\end{spacin}
\vskip 12pt
\end{samepage}
\begin{samepage}
\begingl[glneveryline={\PaliGlossA,\PaliGlossB}]
saṅghāṭipattacīvaradhāraṇe[outrobe.bowl.robe.carry] sampajānakārī[mindful] hoti,[to be]
\endgl
\nopagebreak
\linespread{0.5}
\begin{spacin}{0.2}
{\PaliGlossFT who acts in full awareness when wearing his robes and carrying his outer robe and bowl;}
\end{spacin}
\vskip 12pt
\end{samepage}
\begin{samepage}
\begingl[glneveryline={\PaliGlossA,\PaliGlossB}]
asite[eat] pīte[drink] khāyite[consume] sāyite[taste] sampajānakārī[mindful] hoti,[to be]
\endgl
\nopagebreak
\linespread{0.5}
\begin{spacin}{0.2}
{\PaliGlossFT who acts in full awareness when eating, drinking, consuming food, and tasting;}
\end{spacin}
\vskip 12pt
\end{samepage}
\begin{samepage}
\begingl[glneveryline={\PaliGlossA,\PaliGlossB}]
uccārapassāvakamme[excrete.urinate] sampajānakārī[mindful] hoti,[to be]
\endgl
\nopagebreak
\linespread{0.5}
\begin{spacin}{0.2}
{\PaliGlossFT who acts in full awareness when defecating or urinating;}
\end{spacin}
\vskip 12pt
\end{samepage}
\begin{samepage}
\begingl[glneveryline={\PaliGlossA,\PaliGlossB}]
gate[walk] ṭhite[stand] nisinne[sat down] sutte[sleep] jāgarite[awake] bhāsite[speak] tuṇhībhāve[silent] sampajānakārī[mindful] hoti.[to be]
\endgl
\nopagebreak
\linespread{0.5}
\begin{spacin}{0.2}
{\PaliGlossFT who acts in full awareness when walking, standing, sitting, falling asleep, waking up, talking, and keeping silent.}
\end{spacin}
\vskip 12pt
\end{samepage}
\begin{samepage}
\begingl[glneveryline={\PaliGlossA,\PaliGlossB}]
tassa[he] evaṁ[thus] appamattassa[careful] ātāpino[ardent] pahitattassa[able.truth] viharato[abides] ye[whatever] gehasitā[family life] sarasaṅkappā[memory.intention] te[his] pahīyanti.[abandoned]
\endgl
\nopagebreak
\linespread{0.5}
\begin{spacin}{0.2}
{\PaliGlossFT As he abides thus diligent, ardent, and resolute, his memories and intentions based on the household life are abandoned;}
\end{spacin}
\vskip 12pt
\end{samepage}
\begin{samepage}
\begingl[glneveryline={\PaliGlossA,\PaliGlossB}]
tesaṁ[that] pahānā[removal] ajjhattameva[internally] cittaṁ[mind] santiṭṭhati[steadied] sannisīdati[settled] ekodi[single] hoti[to be] samādhiyati.[concentrated] evampi,[that.to] bhikkhave,[-] bhikkhu[-] kāyagatāsatiṁ[relate to body] bhāveti.[develops]
\endgl
\nopagebreak
\linespread{0.5}
\begin{spacin}{0.2}
{\PaliGlossFT with their abandoning his mind becomes steadied internally, quieted, brought to singleness, and concentrated. That too is how a bhikkhu develops mindfulness of the body.}
\end{spacin}
\vskip 12pt
\end{samepage}
\vskip 0.2in
\begin{samepage}
\begingl[glneveryline={\PaliGlossA,\PaliGlossB}]
“puna[again] caparaṁ,[then] bhikkhave,[-] bhikkhu[-] imameva[this] kāyaṁ[the body] uddhaṁ[upward] pādatalā[sole foot] adho[below] kesamatthakā[hair.head] tacapariyantaṁ[skin.bound] pūraṁ[full] nānappakārassa[of many kind] asucino[unclean] paccavekkhati;[contemplate]
\endgl
\nopagebreak
\linespread{0.5}
\begin{spacin}{0.2}
{\PaliGlossFT “Again, bhikkhus, a bhikkhu reviews this same body up from the soles of the feet and down from the top of the hair, bounded by skin, as full of many kinds of impurity thus:}
\end{spacin}
\vskip 12pt
\end{samepage}
\begin{samepage}
\begingl[glneveryline={\PaliGlossA,\PaliGlossB}]
‘atthi[exist] imasmiṁ[this] kāye[body]
\endgl
\nopagebreak
\linespread{0.5}
\begin{spacin}{0.2}
{\PaliGlossFT ‘In this body there are}
\end{spacin}
\vskip 12pt
\end{samepage}
\begin{samepage}
\begingl[glneveryline={\PaliGlossA,\PaliGlossB}]
kesā[headhair] lomā[bodyhair] nakhā[nail] dantā[teeth] taco[skin]
\endgl
\nopagebreak
\linespread{0.5}
\begin{spacin}{0.2}
{\PaliGlossFT head-hairs, body-hairs, nails, teeth, skin,}
\end{spacin}
\vskip 12pt
\end{samepage}
\begin{samepage}
\begingl[glneveryline={\PaliGlossA,\PaliGlossB}]
maṁsaṁ[flesh] nhāru[sinew] aṭṭhi[bone] aṭṭhimiñjaṁ[bonemarrow] vakkaṁ[kidney]
\endgl
\nopagebreak
\linespread{0.5}
\begin{spacin}{0.2}
{\PaliGlossFT flesh, sinews, bones, bone-marrow, kidneys,}
\end{spacin}
\vskip 12pt
\end{samepage}
\begin{samepage}
\begingl[glneveryline={\PaliGlossA,\PaliGlossB}]
hadayaṁ[heart] yakanaṁ[liver] kilomakaṁ[pleura] pihakaṁ[spleen] papphāsaṁ[lungs]
\endgl
\nopagebreak
\linespread{0.5}
\begin{spacin}{0.2}
{\PaliGlossFT heart, liver, diaphragm, spleen, lungs,}
\end{spacin}
\vskip 12pt
\end{samepage}
\begin{samepage}
\begingl[glneveryline={\PaliGlossA,\PaliGlossB}]
antaṁ[intestine] antaguṇaṁ[mesentery] udariyaṁ[undig. food] karīsaṁ[excrement] pittaṁ[bile]
\endgl
\nopagebreak
\linespread{0.5}
\begin{spacin}{0.2}
{\PaliGlossFT intestines, mesentery, contents of the stomach, feces, bile,}
\end{spacin}
\vskip 12pt
\end{samepage}
\begin{samepage}
\begingl[glneveryline={\PaliGlossA,\PaliGlossB}]
semhaṁ[phlegm] pubbo[pus] lohitaṁ[blood] sedo[sweat] medo[fat] assu[tears]
\endgl
\nopagebreak
\linespread{0.5}
\begin{spacin}{0.2}
{\PaliGlossFT phlegm, pus, blood, sweat, fat, tears,}
\end{spacin}
\vskip 12pt
\end{samepage}
\begin{samepage}
\begingl[glneveryline={\PaliGlossA,\PaliGlossB}]
vasā[grease] kheḷo[saliva] siṅghāṇikā[mucus] lasikā[synovic fluid] muttan’ti.[urine]
\endgl
\nopagebreak
\linespread{0.5}
\begin{spacin}{0.2}
{\PaliGlossFT grease, spittle, snot, oil of the joints, and urine.’}
\end{spacin}
\vskip 12pt
\end{samepage}
\begin{samepage}
\begingl[glneveryline={\PaliGlossA,\PaliGlossB}]
“seyyathāpi,[just as] bhikkhave,[-] ubhatomukhā[double mouthed] putoḷi[bag] pūrā[full] nānāvihitassa[various] dhaññassa,[grain] seyyathidaṁ;[such as]
\endgl
\nopagebreak
\linespread{0.5}
\begin{spacin}{0.2}
{\PaliGlossFT Just as though there were a bag with an opening at both ends full of many sorts of grain, such as}
\end{spacin}
\vskip 12pt
\end{samepage}
\begin{samepage}
\begingl[glneveryline={\PaliGlossA,\PaliGlossB}]
sālīnaṁ[fine rice] vīhīnaṁ[paddy] muggānaṁ[green pea] māsānaṁ[bean] tilānaṁ[sesamum] taṇḍulānaṁ,[rice-grain] tamenaṁ[-] cakkhumā[with.eyes] puriso[man] muñcitvā[loosened] paccavekkheyya;[reviews]
\endgl
\nopagebreak
\linespread{0.5}
\begin{spacin}{0.2}
{\PaliGlossFT hill rice, red rice, beans, peas, millet, and white rice, and a man with good eyes were to open it and review it thus:}
\end{spacin}
\vskip 12pt
\end{samepage}
\begin{samepage}
\begingl[glneveryline={\PaliGlossA,\PaliGlossB}]
‘ime[this] sālī[fine rice] ime[this] vīhī[paddy] ime[this] muggā[green pea] ime[this] māsā[bean] ime[this] tilā[sesamum ] ime[this] taṇḍulā’ti;[rice-grain]
\endgl
\nopagebreak
\linespread{0.5}
\begin{spacin}{0.2}
{\PaliGlossFT ‘This is hill rice, this is red rice, these are beans, these are peas, this is millet, this is white rice’;}
\end{spacin}
\vskip 12pt
\end{samepage}
\begin{samepage}
\begingl[glneveryline={\PaliGlossA,\PaliGlossB}]
evameva[only] kho,[indeed] bhikkhave,[-] bhikkhu[-] imameva[this] kāyaṁ[the body] uddhaṁ[upward] pādatalā[sole foot] adho[below] kesamatthakā[hair.head] tacapariyantaṁ[skin.bound] pūraṁ[full] nānappakārassa[of many kind] asucino[unclean] paccavekkhati;[contemplate]
\endgl
\nopagebreak
\linespread{0.5}
\begin{spacin}{0.2}
{\PaliGlossFT so too, a bhikkhu reviews this same body as full of many kinds of impurity thus:}
\end{spacin}
\vskip 12pt
\end{samepage}
\begin{samepage}
\begingl[glneveryline={\PaliGlossA,\PaliGlossB}]
‘atthi[exist] imasmiṁ[this] kāye[body]
\endgl
\nopagebreak
\linespread{0.5}
\begin{spacin}{0.2}
{\PaliGlossFT ‘In this body there are}
\end{spacin}
\vskip 12pt
\end{samepage}
\begin{samepage}
\begingl[glneveryline={\PaliGlossA,\PaliGlossB}]
kesā[headhair] lomā[bodyhair] nakhā[nail] dantā[teeth] taco[skin]
\endgl
\nopagebreak
\linespread{0.5}
\begin{spacin}{0.2}
{\PaliGlossFT head-hairs, body-hairs, nails, teeth, skin,}
\end{spacin}
\vskip 12pt
\end{samepage}
\begin{samepage}
\begingl[glneveryline={\PaliGlossA,\PaliGlossB}]
maṁsaṁ[flesh] nhāru[sinew] aṭṭhi[bone] aṭṭhimiñjaṁ[bonemarrow] vakkaṁ[kidney]
\endgl
\nopagebreak
\linespread{0.5}
\begin{spacin}{0.2}
{\PaliGlossFT flesh, sinews, bones, bone-marrow, kidneys,}
\end{spacin}
\vskip 12pt
\end{samepage}
\begin{samepage}
\begingl[glneveryline={\PaliGlossA,\PaliGlossB}]
hadayaṁ[heart] yakanaṁ[liver] kilomakaṁ[pleura] pihakaṁ[spleen] papphāsaṁ[lungs]
\endgl
\nopagebreak
\linespread{0.5}
\begin{spacin}{0.2}
{\PaliGlossFT heart, liver, diaphragm, spleen, lungs,}
\end{spacin}
\vskip 12pt
\end{samepage}
\begin{samepage}
\begingl[glneveryline={\PaliGlossA,\PaliGlossB}]
antaṁ[intestine] antaguṇaṁ[mesentery] udariyaṁ[undig. food] karīsaṁ[excrement] pittaṁ[bile]
\endgl
\nopagebreak
\linespread{0.5}
\begin{spacin}{0.2}
{\PaliGlossFT intestines, mesentery, contents of the stomach, feces, bile,}
\end{spacin}
\vskip 12pt
\end{samepage}
\begin{samepage}
\begingl[glneveryline={\PaliGlossA,\PaliGlossB}]
semhaṁ[phlegm] pubbo[pus] lohitaṁ[blood] sedo[sweat] medo[fat] assu[tears]
\endgl
\nopagebreak
\linespread{0.5}
\begin{spacin}{0.2}
{\PaliGlossFT phlegm, pus, blood, sweat, fat, tears,}
\end{spacin}
\vskip 12pt
\end{samepage}
\begin{samepage}
\begingl[glneveryline={\PaliGlossA,\PaliGlossB}]
vasā[grease] kheḷo[saliva] siṅghāṇikā[mucus] lasikā[synovic fluid] muttan’ti.[urine]
\endgl
\nopagebreak
\linespread{0.5}
\begin{spacin}{0.2}
{\PaliGlossFT grease, spittle, snot, oil of the joints, and urine.’}
\end{spacin}
\vskip 12pt
\end{samepage}
\begin{samepage}
\begingl[glneveryline={\PaliGlossA,\PaliGlossB}]
tassa[he] evaṁ[thus] appamattassa[careful] ātāpino[ardent] pahitattassa[able.truth] viharato[abides] ye[whatever] gehasitā[family life] sarasaṅkappā[memory.intention] te[his] pahīyanti.[abandoned]
\endgl
\nopagebreak
\linespread{0.5}
\begin{spacin}{0.2}
{\PaliGlossFT As he abides thus diligent, ardent, and resolute, his memories and intentions based on the household life are abandoned;}
\end{spacin}
\vskip 12pt
\end{samepage}
\begin{samepage}
\begingl[glneveryline={\PaliGlossA,\PaliGlossB}]
tesaṁ[that] pahānā[removal] ajjhattameva[internally] cittaṁ[mind] santiṭṭhati[steadied] sannisīdati[settled] ekodi[single] hoti[to be] samādhiyati.[concentrated]
\endgl
\nopagebreak
\linespread{0.5}
\begin{spacin}{0.2}
{\PaliGlossFT with their abandoning his mind becomes steadied internally, quieted, brought to singleness, and concentrated.}
\end{spacin}
\vskip 12pt
\end{samepage}
\begin{samepage}
\begingl[glneveryline={\PaliGlossA,\PaliGlossB}]
evampi,[that.to] bhikkhave,[-] bhikkhu[-] kāyagatāsatiṁ[relate to body] bhāveti.[develops]
\endgl
\nopagebreak
\linespread{0.5}
\begin{spacin}{0.2}
{\PaliGlossFT That too is how a bhikkhu develops mindfulness of the body.}
\end{spacin}
\vskip 12pt
\end{samepage}
\vskip 0.2in
\begin{samepage}
\begingl[glneveryline={\PaliGlossA,\PaliGlossB}]
“puna[again] caparaṁ,[then] bhikkhave,[-] bhikkhu[-] imameva[this] kāyaṁ[the body] yathāṭhitaṁ[as it stand] yathāpaṇihitaṁ[as it directed] dhātuso[element] paccavekkhati;[contemplate]
\endgl
\nopagebreak
\linespread{0.5}
\begin{spacin}{0.2}
{\PaliGlossFT “Again, bhikkhus, a bhikkhu reviews this same body, however it is placed, however disposed, as consisting of elements thus:}
\end{spacin}
\vskip 12pt
\end{samepage}
\begin{samepage}
\begingl[glneveryline={\PaliGlossA,\PaliGlossB}]
‘atthi[exist] imasmiṁ[this] kāye[body] pathavīdhātu[earth.elem] āpodhātu[water.elem] tejodhātu[fire.elem] vāyodhātū’ti.[air.elem]
\endgl
\nopagebreak
\linespread{0.5}
\begin{spacin}{0.2}
{\PaliGlossFT ‘In this body there are the earth element, the water element, the fire element, and the air element.’}
\end{spacin}
\vskip 12pt
\end{samepage}
\begin{samepage}
\begingl[glneveryline={\PaliGlossA,\PaliGlossB}]
“seyyathāpi,[just as] bhikkhave,[-] dakkho[skilled] goghātako[butcher] vā[or] goghātakantevāsī[butcher.pupil] vā[or] gāviṁ[cow] vadhitvā[have killed] catumahāpathe[4.road] bilaso[portions] vibhajitvā[have dissected] nisinno[sat] assa;[to be]
\endgl
\nopagebreak
\linespread{0.5}
\begin{spacin}{0.2}
{\PaliGlossFT Just as though a skilled butcher or his apprentice had killed a cow and were seated at the crossroads with it cut up into pieces;}
\end{spacin}
\vskip 12pt
\end{samepage}
\begin{samepage}
\begingl[glneveryline={\PaliGlossA,\PaliGlossB}]
evameva[only] kho,[indeed] bhikkhave,[-] bhikkhu[-] imameva[this] kāyaṁ[the body] yathāṭhitaṁ[as it stand] yathāpaṇihitaṁ[as it directed] dhātuso[element] paccavekkhati;[contemplate]
\endgl
\nopagebreak
\linespread{0.5}
\begin{spacin}{0.2}
{\PaliGlossFT so too, a bhikkhu reviews this same body however it is placed, however disposed, as consisting of elements thus:}
\end{spacin}
\vskip 12pt
\end{samepage}
\begin{samepage}
\begingl[glneveryline={\PaliGlossA,\PaliGlossB}]
‘atthi[exist] imasmiṁ[this] kāye[body] pathavīdhātu[earth.elem] āpodhātu[water.elem] tejodhātu[fire.elem] vāyodhātū’ti.[air.elem]
\endgl
\nopagebreak
\linespread{0.5}
\begin{spacin}{0.2}
{\PaliGlossFT ‘In this body there are the earth element, the water element, the fire element, and the air element.’}
\end{spacin}
\vskip 12pt
\end{samepage}
\begin{samepage}
\begingl[glneveryline={\PaliGlossA,\PaliGlossB}]
tassa[he] evaṁ[thus] appamattassa[careful] ātāpino[ardent] pahitattassa[able.truth] viharato[abides] ye[whatever] gehasitā[family life] sarasaṅkappā[memory.intention] te[his] pahīyanti.[abandoned]
\endgl
\nopagebreak
\linespread{0.5}
\begin{spacin}{0.2}
{\PaliGlossFT As he abides thus diligent, ardent, and resolute, his memories and intentions connected with the household life are abandoned;}
\end{spacin}
\vskip 12pt
\end{samepage}
\begin{samepage}
\begingl[glneveryline={\PaliGlossA,\PaliGlossB}]
tesaṁ[that] pahānā[removal] ajjhattameva[internally] cittaṁ[mind] santiṭṭhati[steadied] sannisīdati[settled] ekodi[single] hoti[to be] samādhiyati.[concentrated]
\endgl
\nopagebreak
\linespread{0.5}
\begin{spacin}{0.2}
{\PaliGlossFT with their abandoning his mind becomes steadied internally, quieted, brought to singleness, and concentrated.}
\end{spacin}
\vskip 12pt
\end{samepage}
\begin{samepage}
\begingl[glneveryline={\PaliGlossA,\PaliGlossB}]
evampi,[that.to] bhikkhave,[-] bhikkhu[-] kāyagatāsatiṁ[relate to body] bhāveti.[develops]
\endgl
\nopagebreak
\linespread{0.5}
\begin{spacin}{0.2}
{\PaliGlossFT That too is how a bhikkhu develops mindfulness of the body.}
\end{spacin}
\vskip 12pt
\end{samepage}
\vskip 0.2in
\begin{samepage}
\begingl[glneveryline={\PaliGlossA,\PaliGlossB}]
“puna[again] caparaṁ,[then] bhikkhave,[-] bhikkhu[-] seyyathāpi[just as] passeyya[aside] sarīraṁ[the body] sivathikāya[safe.body] chaḍḍitaṁ[abandoned] ekāhamataṁ[one day] vā[or] dvīhamataṁ[two days] vā[or] tīhamataṁ[three days] vā[or] uddhumātakaṁ[bloated] vinīlakaṁ[discolored] vipubbakajātaṁ.[festering]
\endgl
\nopagebreak
\linespread{0.5}
\begin{spacin}{0.2}
{\PaliGlossFT “Again, bhikkhus, as though he were to see a corpse thrown aside in a charnel ground, one, two, or three days dead, bloated, livid, and oozing matter,}
\end{spacin}
\vskip 12pt
\end{samepage}
\begin{samepage}
\begingl[glneveryline={\PaliGlossA,\PaliGlossB}]
so[he] imameva[this] kāyaṁ[the body] upasaṁharati;[compare]
\endgl
\nopagebreak
\linespread{0.5}
\begin{spacin}{0.2}
{\PaliGlossFT a bhikkhu compares this same body with it thus:}
\end{spacin}
\vskip 12pt
\end{samepage}
\begin{samepage}
\begingl[glneveryline={\PaliGlossA,\PaliGlossB}]
‘ayampi[this.also] kho[indeed] kāyo[body] evaṁdhammo[this.nature] evaṁbhāvī[this.become] evaṁanatīto’ti.[this.not.overcome]
\endgl
\nopagebreak
\linespread{0.5}
\begin{spacin}{0.2}
{\PaliGlossFT ‘This body too is of the same nature, it will be like that, it is not exempt from that fate.’}
\end{spacin}
\vskip 12pt
\end{samepage}
\begin{samepage}
\begingl[glneveryline={\PaliGlossA,\PaliGlossB}]
tassa[he] evaṁ[thus] appamattassa[careful] ātāpino[ardent] pahitattassa[able.truth] viharato[abides] ye[whatever] gehasitā[family life] sarasaṅkappā[memory.intention] te[his] pahīyanti.[abandoned]
\endgl
\nopagebreak
\linespread{0.5}
\begin{spacin}{0.2}
{\PaliGlossFT As he abides thus diligent, ardent, and resolute, his memories and intentions connected with the household life are abandoned;}
\end{spacin}
\vskip 12pt
\end{samepage}
\begin{samepage}
\begingl[glneveryline={\PaliGlossA,\PaliGlossB}]
tesaṁ[that] pahānā[removal] ajjhattameva[internally] cittaṁ[mind] santiṭṭhati[steadied] sannisīdati[settled] ekodi[single] hoti[to be] samādhiyati.[concentrated]
\endgl
\nopagebreak
\linespread{0.5}
\begin{spacin}{0.2}
{\PaliGlossFT with their abandoning his mind becomes steadied internally, quieted, brought to singleness, and concentrated.}
\end{spacin}
\vskip 12pt
\end{samepage}
\begin{samepage}
\begingl[glneveryline={\PaliGlossA,\PaliGlossB}]
evampi,[that.to] bhikkhave,[-] bhikkhu[-] kāyagatāsatiṁ[relate to body] bhāveti.[develops]
\endgl
\nopagebreak
\linespread{0.5}
\begin{spacin}{0.2}
{\PaliGlossFT That too is how a bhikkhu develops mindfulness of the body.}
\end{spacin}
\vskip 12pt
\end{samepage}
\vskip 0.2in
\begin{samepage}
\begingl[glneveryline={\PaliGlossA,\PaliGlossB}]
“puna[again] caparaṁ,[then] bhikkhave,[-] bhikkhu[-] seyyathāpi[just as] passeyya[aside] sarīraṁ[the body] sivathikāya[safe.body] chaḍḍitaṁ[abandoned] kākehi[crow] vā[or] khajjamānaṁ[consumed ] kulalehi[hawk] vā[or] khajjamānaṁ[consumed ] gijjhehi[vulture] vā[or] khajjamānaṁ[consumed ] kaṅkehi[heron] vā[or] khajjamānaṁ[consumed ] sunakhehi[ dog] vā[or] khajjamānaṁ[consumed ] byagghehi[tiger] vā[or] khajjamānaṁ[consumed ] dīpīhi[panther] vā[or] khajjamānaṁ[consumed ] siṅgālehi[jackel] vā[or] khajjamānaṁ[consumed ] vividhehi[various] vā[or] pāṇakajātehi[insect] khajjamānaṁ.[consumed ]
\endgl
\nopagebreak
\linespread{0.5}
\begin{spacin}{0.2}
{\PaliGlossFT “Again, as though he were to see a corpse thrown aside in a charnel ground, being devoured by crows, hawks, vultures, dogs, jackals, or various kinds of worms,}
\end{spacin}
\vskip 12pt
\end{samepage}
\begin{samepage}
\begingl[glneveryline={\PaliGlossA,\PaliGlossB}]
so[he] imameva[this] kāyaṁ[the body] upasaṁharati;[compare]
\endgl
\nopagebreak
\linespread{0.5}
\begin{spacin}{0.2}
{\PaliGlossFT a bhikkhu compares this same body with it thus:}
\end{spacin}
\vskip 12pt
\end{samepage}
\begin{samepage}
\begingl[glneveryline={\PaliGlossA,\PaliGlossB}]
‘ayampi[this.also] kho[indeed] kāyo[body] evaṁdhammo[this.nature] evaṁbhāvī[this.become] evaṁanatīto’ti.[this.not.overcome]
\endgl
\nopagebreak
\linespread{0.5}
\begin{spacin}{0.2}
{\PaliGlossFT ‘This body too is of the same nature, it will be like that, it is not exempt from that fate.’}
\end{spacin}
\vskip 12pt
\end{samepage}
\begin{samepage}
\begingl[glneveryline={\PaliGlossA,\PaliGlossB}]
tassa[he] evaṁ[thus] appamattassa[careful] ātāpino[ardent] pahitattassa[able.truth] viharato[abides] ye[whatever] gehasitā[family life] sarasaṅkappā[memory.intention] te[his] pahīyanti.[abandoned]
\endgl
\nopagebreak
\linespread{0.5}
\begin{spacin}{0.2}
{\PaliGlossFT As he abides thus diligent, ardent, and resolute, his memories and intentions connected with the household life are abandoned;}
\end{spacin}
\vskip 12pt
\end{samepage}
\begin{samepage}
\begingl[glneveryline={\PaliGlossA,\PaliGlossB}]
tesaṁ[that] pahānā[removal] ajjhattameva[internally] cittaṁ[mind] santiṭṭhati[steadied] sannisīdati[settled] ekodi[single] hoti[to be] samādhiyati.[concentrated]
\endgl
\nopagebreak
\linespread{0.5}
\begin{spacin}{0.2}
{\PaliGlossFT with their abandoning his mind becomes steadied internally, quieted, brought to singleness, and concentrated.}
\end{spacin}
\vskip 12pt
\end{samepage}
\begin{samepage}
\begingl[glneveryline={\PaliGlossA,\PaliGlossB}]
evampi,[that.to] bhikkhave,[-] bhikkhu[-] kāyagatāsatiṁ[relate to body] bhāveti.[develops]
\endgl
\nopagebreak
\linespread{0.5}
\begin{spacin}{0.2}
{\PaliGlossFT That too is how a bhikkhu develops mindfulness of the body.}
\end{spacin}
\vskip 12pt
\end{samepage}
\vskip 0.2in
\begin{samepage}
\begingl[glneveryline={\PaliGlossA,\PaliGlossB}]
“puna[again] caparaṁ,[then] bhikkhave,[-] bhikkhu[-] seyyathāpi[just as] passeyya[aside] sarīraṁ[the body] sivathikāya[safe.body] chaḍḍitaṁ[abandoned] aṭṭhikasaṅkhalikaṁ[bone.chain] samaṁsalohitaṁ[with.flesh.blood] nhārusambandhaṁ.[sinew.connect]
\endgl
\nopagebreak
\linespread{0.5}
\begin{spacin}{0.2}
{\PaliGlossFT Again, as though he were to see a corpse thrown aside in a charnel ground, a skeleton with flesh and blood, held together with sinews,}
\end{spacin}
\vskip 12pt
\end{samepage}
\begin{samepage}
\begingl[glneveryline={\PaliGlossA,\PaliGlossB}]
so[he] imameva[this] kāyaṁ[the body] upasaṁharati;[compare]
\endgl
\nopagebreak
\linespread{0.5}
\begin{spacin}{0.2}
{\PaliGlossFT a bhikkhu compares this same body with it thus:}
\end{spacin}
\vskip 12pt
\end{samepage}
\begin{samepage}
\begingl[glneveryline={\PaliGlossA,\PaliGlossB}]
‘ayampi[this.also] kho[indeed] kāyo[body] evaṁdhammo[this.nature] evaṁbhāvī[this.become] evaṁanatīto’ti.[this.not.overcome]
\endgl
\nopagebreak
\linespread{0.5}
\begin{spacin}{0.2}
{\PaliGlossFT ‘This body too is of the same nature, it will be like that, it is not exempt from that fate.’}
\end{spacin}
\vskip 12pt
\end{samepage}
\begin{samepage}
\begingl[glneveryline={\PaliGlossA,\PaliGlossB}]
tassa[he] evaṁ[thus] appamattassa[careful] ātāpino[ardent] pahitattassa[able.truth] viharato[abides] ye[whatever] gehasitā[family life] sarasaṅkappā[memory.intention] te[his] pahīyanti.[abandoned]
\endgl
\nopagebreak
\linespread{0.5}
\begin{spacin}{0.2}
{\PaliGlossFT As he abides thus diligent, ardent, and resolute, his memories and intentions connected with the household life are abandoned;}
\end{spacin}
\vskip 12pt
\end{samepage}
\begin{samepage}
\begingl[glneveryline={\PaliGlossA,\PaliGlossB}]
tesaṁ[that] pahānā[removal] ajjhattameva[internally] cittaṁ[mind] santiṭṭhati[steadied] sannisīdati[settled] ekodi[single] hoti[to be] samādhiyati.[concentrated]
\endgl
\nopagebreak
\linespread{0.5}
\begin{spacin}{0.2}
{\PaliGlossFT with their abandoning his mind becomes steadied internally, quieted, brought to singleness, and concentrated.}
\end{spacin}
\vskip 12pt
\end{samepage}
\begin{samepage}
\begingl[glneveryline={\PaliGlossA,\PaliGlossB}]
evampi,[that.to] bhikkhave,[-] bhikkhu[-] kāyagatāsatiṁ[relate to body] bhāveti.[develops]
\endgl
\nopagebreak
\linespread{0.5}
\begin{spacin}{0.2}
{\PaliGlossFT That too is how a bhikkhu develops mindfulness of the body.}
\end{spacin}
\vskip 12pt
\end{samepage}
\vskip 0.2in
\begin{samepage}
\begingl[glneveryline={\PaliGlossA,\PaliGlossB}]
“puna[again] caparaṁ,[then] bhikkhave,[-] bhikkhu[-] seyyathāpi[just as] passeyya[aside] aṭṭhikasaṅkhalikaṁ[bone.chain] nimmaṁsalohitamakkhitaṁ[without.flesh.blood.smear] nhārusambandhaṁ[sinew.connect]
\endgl
\nopagebreak
\linespread{0.5}
\begin{spacin}{0.2}
{\PaliGlossFT Again, as though he were to see a fleshless skeleton smeared with blood, held together with sinews,}
\end{spacin}
\vskip 12pt
\end{samepage}
\begin{samepage}
\begingl[glneveryline={\PaliGlossA,\PaliGlossB}]
so[he] imameva[this] kāyaṁ[the body] upasaṁharati;[compare]
\endgl
\nopagebreak
\linespread{0.5}
\begin{spacin}{0.2}
{\PaliGlossFT a bhikkhu compares this same body with it thus:}
\end{spacin}
\vskip 12pt
\end{samepage}
\begin{samepage}
\begingl[glneveryline={\PaliGlossA,\PaliGlossB}]
‘ayampi[this.also] kho[indeed] kāyo[body] evaṁdhammo[this.nature] evaṁbhāvī[this.become] evaṁanatīto’ti.[this.not.overcome]
\endgl
\nopagebreak
\linespread{0.5}
\begin{spacin}{0.2}
{\PaliGlossFT ‘This body too is of the same nature, it will be like that, it is not exempt from that fate.’}
\end{spacin}
\vskip 12pt
\end{samepage}
\begin{samepage}
\begingl[glneveryline={\PaliGlossA,\PaliGlossB}]
tassa[he] evaṁ[thus] appamattassa[careful] ātāpino[ardent] pahitattassa[able.truth] viharato[abides] ye[whatever] gehasitā[family life] sarasaṅkappā[memory.intention] te[his] pahīyanti.[abandoned]
\endgl
\nopagebreak
\linespread{0.5}
\begin{spacin}{0.2}
{\PaliGlossFT As he abides thus diligent, ardent, and resolute, his memories and intentions connected with the household life are abandoned;}
\end{spacin}
\vskip 12pt
\end{samepage}
\begin{samepage}
\begingl[glneveryline={\PaliGlossA,\PaliGlossB}]
tesaṁ[that] pahānā[removal] ajjhattameva[internally] cittaṁ[mind] santiṭṭhati[steadied] sannisīdati[settled] ekodi[single] hoti[to be] samādhiyati.[concentrated]
\endgl
\nopagebreak
\linespread{0.5}
\begin{spacin}{0.2}
{\PaliGlossFT with their abandoning his mind becomes steadied internally, quieted, brought to singleness, and concentrated.}
\end{spacin}
\vskip 12pt
\end{samepage}
\begin{samepage}
\begingl[glneveryline={\PaliGlossA,\PaliGlossB}]
evampi,[that.to] bhikkhave,[-] bhikkhu[-] kāyagatāsatiṁ[relate to body] bhāveti.[develops]
\endgl
\nopagebreak
\linespread{0.5}
\begin{spacin}{0.2}
{\PaliGlossFT That too is how a bhikkhu develops mindfulness of the body.}
\end{spacin}
\vskip 12pt
\end{samepage}
\vskip 0.2in
\begin{samepage}
\begingl[glneveryline={\PaliGlossA,\PaliGlossB}]
“puna[again] caparaṁ,[then] bhikkhave,[-] bhikkhu[-] seyyathāpi[just as] passeyya[aside] aṭṭhikasaṅkhalikaṁ[bone.chain] apagatamaṁsalohitaṁ[removed.flesh.blood] nhārusambandhaṁ.[sinew.connect]
\endgl
\nopagebreak
\linespread{0.5}
\begin{spacin}{0.2}
{\PaliGlossFT "Again, as though he were to see a skeleton without flesh and blood, held together with sinews,}
\end{spacin}
\vskip 12pt
\end{samepage}
\begin{samepage}
\begingl[glneveryline={\PaliGlossA,\PaliGlossB}]
so[he] imameva[this] kāyaṁ[the body] upasaṁharati;[compare]
\endgl
\nopagebreak
\linespread{0.5}
\begin{spacin}{0.2}
{\PaliGlossFT a bhikkhu compares this same body with it thus:}
\end{spacin}
\vskip 12pt
\end{samepage}
\begin{samepage}
\begingl[glneveryline={\PaliGlossA,\PaliGlossB}]
‘ayampi[this.also] kho[indeed] kāyo[body] evaṁdhammo[this.nature] evaṁbhāvī[this.become] evaṁanatīto’ti.[this.not.overcome]
\endgl
\nopagebreak
\linespread{0.5}
\begin{spacin}{0.2}
{\PaliGlossFT ‘This body too is of the same nature, it will be like that, it is not exempt from that fate.’}
\end{spacin}
\vskip 12pt
\end{samepage}
\begin{samepage}
\begingl[glneveryline={\PaliGlossA,\PaliGlossB}]
tassa[he] evaṁ[thus] appamattassa[careful] ātāpino[ardent] pahitattassa[able.truth] viharato[abides] ye[whatever] gehasitā[family life] sarasaṅkappā[memory.intention] te[his] pahīyanti.[abandoned]
\endgl
\nopagebreak
\linespread{0.5}
\begin{spacin}{0.2}
{\PaliGlossFT As he abides thus diligent, ardent, and resolute, his memories and intentions connected with the household life are abandoned;}
\end{spacin}
\vskip 12pt
\end{samepage}
\begin{samepage}
\begingl[glneveryline={\PaliGlossA,\PaliGlossB}]
tesaṁ[that] pahānā[removal] ajjhattameva[internally] cittaṁ[mind] santiṭṭhati[steadied] sannisīdati[settled] ekodi[single] hoti[to be] samādhiyati.[concentrated]
\endgl
\nopagebreak
\linespread{0.5}
\begin{spacin}{0.2}
{\PaliGlossFT with their abandoning his mind becomes steadied internally, quieted, brought to singleness, and concentrated.}
\end{spacin}
\vskip 12pt
\end{samepage}
\begin{samepage}
\begingl[glneveryline={\PaliGlossA,\PaliGlossB}]
evampi,[that.to] bhikkhave,[-] bhikkhu[-] kāyagatāsatiṁ[relate to body] bhāveti.[develops]
\endgl
\nopagebreak
\linespread{0.5}
\begin{spacin}{0.2}
{\PaliGlossFT That too is how a bhikkhu develops mindfulness of the body.}
\end{spacin}
\vskip 12pt
\end{samepage}
\vskip 0.2in
\begin{samepage}
\begingl[glneveryline={\PaliGlossA,\PaliGlossB}]
“puna[again] caparaṁ,[then] bhikkhave,[-] bhikkhu[-] seyyathāpi[just as] passeyya[aside] aṭṭhikāni[bone] apagatasambandhāni[removed.connection] disāvidisāvikkhittāni[directions.upset] aññena[other] hatthaṭṭhikaṁ[hand.bone] aññena[other] pādaṭṭhikaṁ[leg.bone] aññena[other] gopphakaṭṭhikaṁ[ankle.bone] aññena[other] jaṅghaṭṭhikaṁ[shine.bone] aññena[other] ūruṭṭhikaṁ[thigh.bone] aññena[other] kaṭiṭṭhikaṁ[hip.bone] aññena[other] phāsukaṭṭhikaṁ[rib.bone] aññena[other] piṭṭhiṭṭhikaṁ[back.bone] aññena[other] khandhaṭṭhikaṁ[breast.bone] aññena[other] gīvaṭṭhikaṁ[neck.bone] aññena[other] hanukaṭṭhikaṁ[jaw.bone] aññena[other] dantaṭṭhikaṁ[tooth.bone] aññena[other] sīsakaṭāhaṁ.[skull.bone]
\endgl
\nopagebreak
\linespread{0.5}
\begin{spacin}{0.2}
{\PaliGlossFT "Again, as though he were to see disconnected bones scattered in all directions—here a hand-bone, there a foot-bone, here a shin-bone, there a thigh-bone, here a hip-bone, there a back-bone, here a rib-bone, there a breast-bone, here an arm-bone, there a shoulder-bone, here a neck-bone, there a jaw-bone, here a tooth, there the skull,}
\end{spacin}
\vskip 12pt
\end{samepage}
\begin{samepage}
\begingl[glneveryline={\PaliGlossA,\PaliGlossB}]
so[he] imameva[this] kāyaṁ[the body] upasaṁharati;[compare]
\endgl
\nopagebreak
\linespread{0.5}
\begin{spacin}{0.2}
{\PaliGlossFT a bhikkhu compares this same body with it thus:}
\end{spacin}
\vskip 12pt
\end{samepage}
\begin{samepage}
\begingl[glneveryline={\PaliGlossA,\PaliGlossB}]
‘ayampi[this.also] kho[indeed] kāyo[body] evaṁdhammo[this.nature] evaṁbhāvī[this.become] evaṁanatīto’ti.[this.not.overcome]
\endgl
\nopagebreak
\linespread{0.5}
\begin{spacin}{0.2}
{\PaliGlossFT ‘This body too is of the same nature, it will be like that, it is not exempt from that fate.’}
\end{spacin}
\vskip 12pt
\end{samepage}
\begin{samepage}
\begingl[glneveryline={\PaliGlossA,\PaliGlossB}]
tassa[he] evaṁ[thus] appamattassa[careful] ātāpino[ardent] pahitattassa[able.truth] viharato[abides] ye[whatever] gehasitā[family life] sarasaṅkappā[memory.intention] te[his] pahīyanti.[abandoned]
\endgl
\nopagebreak
\linespread{0.5}
\begin{spacin}{0.2}
{\PaliGlossFT As he abides thus diligent, ardent, and resolute, his memories and intentions connected with the household life are abandoned;}
\end{spacin}
\vskip 12pt
\end{samepage}
\begin{samepage}
\begingl[glneveryline={\PaliGlossA,\PaliGlossB}]
tesaṁ[that] pahānā[removal] ajjhattameva[internally] cittaṁ[mind] santiṭṭhati[steadied] sannisīdati[settled] ekodi[single] hoti[to be] samādhiyati.[concentrated]
\endgl
\nopagebreak
\linespread{0.5}
\begin{spacin}{0.2}
{\PaliGlossFT with their abandoning his mind becomes steadied internally, quieted, brought to singleness, and concentrated.}
\end{spacin}
\vskip 12pt
\end{samepage}
\begin{samepage}
\begingl[glneveryline={\PaliGlossA,\PaliGlossB}]
evampi,[that.to] bhikkhave,[-] bhikkhu[-] kāyagatāsatiṁ[relate to body] bhāveti.[develops]
\endgl
\nopagebreak
\linespread{0.5}
\begin{spacin}{0.2}
{\PaliGlossFT That too is how a bhikkhu develops mindfulness of the body.}
\end{spacin}
\vskip 12pt
\end{samepage}
\vskip 0.2in
\begin{samepage}
\begingl[glneveryline={\PaliGlossA,\PaliGlossB}]
“puna[again] caparaṁ,[then] bhikkhave,[-] bhikkhu[-] seyyathāpi[just as] passeyya[aside] sarīraṁ[the body] sivathikāya[safe.body] chaḍḍitaṁ;[abandoned] aṭṭhikāni[bone] setāni[white] saṅkhavaṇṇapaṭibhāgāni.[conch.color.resemble]
\endgl
\nopagebreak
\linespread{0.5}
\begin{spacin}{0.2}
{\PaliGlossFT “Again, as though he were to see a corpse thrown aside in a charnel ground, bones bleached white, the colour of shells,}
\end{spacin}
\vskip 12pt
\end{samepage}
\begin{samepage}
\begingl[glneveryline={\PaliGlossA,\PaliGlossB}]
so[he] imameva[this] kāyaṁ[the body] upasaṁharati;[compare]
\endgl
\nopagebreak
\linespread{0.5}
\begin{spacin}{0.2}
{\PaliGlossFT a bhikkhu compares this same body with it thus:}
\end{spacin}
\vskip 12pt
\end{samepage}
\begin{samepage}
\begingl[glneveryline={\PaliGlossA,\PaliGlossB}]
‘ayampi[this.also] kho[indeed] kāyo[body] evaṁdhammo[this.nature] evaṁbhāvī[this.become] evaṁanatīto’ti.[this.not.overcome]
\endgl
\nopagebreak
\linespread{0.5}
\begin{spacin}{0.2}
{\PaliGlossFT ‘This body too is of the same nature, it will be like that, it is not exempt from that fate.’}
\end{spacin}
\vskip 12pt
\end{samepage}
\begin{samepage}
\begingl[glneveryline={\PaliGlossA,\PaliGlossB}]
tassa[he] evaṁ[thus] appamattassa[careful] ātāpino[ardent] pahitattassa[able.truth] viharato[abides] ye[whatever] gehasitā[family life] sarasaṅkappā[memory.intention] te[his] pahīyanti.[abandoned]
\endgl
\nopagebreak
\linespread{0.5}
\begin{spacin}{0.2}
{\PaliGlossFT As he abides thus diligent, ardent, and resolute, his memories and intentions connected with the household life are abandoned;}
\end{spacin}
\vskip 12pt
\end{samepage}
\begin{samepage}
\begingl[glneveryline={\PaliGlossA,\PaliGlossB}]
tesaṁ[that] pahānā[removal] ajjhattameva[internally] cittaṁ[mind] santiṭṭhati[steadied] sannisīdati[settled] ekodi[single] hoti[to be] samādhiyati.[concentrated]
\endgl
\nopagebreak
\linespread{0.5}
\begin{spacin}{0.2}
{\PaliGlossFT with their abandoning his mind becomes steadied internally, quieted, brought to singleness, and concentrated.}
\end{spacin}
\vskip 12pt
\end{samepage}
\begin{samepage}
\begingl[glneveryline={\PaliGlossA,\PaliGlossB}]
evampi,[that.to] bhikkhave,[-] bhikkhu[-] kāyagatāsatiṁ[relate to body] bhāveti.[develops]
\endgl
\nopagebreak
\linespread{0.5}
\begin{spacin}{0.2}
{\PaliGlossFT That too is how a bhikkhu develops mindfulness of the body.}
\end{spacin}
\vskip 12pt
\end{samepage}
\vskip 0.2in
\begin{samepage}
\begingl[glneveryline={\PaliGlossA,\PaliGlossB}]
“puna[again] caparaṁ,[then] bhikkhave,[-] bhikkhu[-] seyyathāpi[just as] passeyya[aside] aṭṭhikāni[bone] puñjakitāni[heap] terovassikāni[3.year]
\endgl
\nopagebreak
\linespread{0.5}
\begin{spacin}{0.2}
{\PaliGlossFT "Again, as though he were to see bones heaped up,}
\end{spacin}
\vskip 12pt
\end{samepage}
\begin{samepage}
\begingl[glneveryline={\PaliGlossA,\PaliGlossB}]
so[he] imameva[this] kāyaṁ[the body] upasaṁharati;[compare]
\endgl
\nopagebreak
\linespread{0.5}
\begin{spacin}{0.2}
{\PaliGlossFT a bhikkhu compares this same body with it thus:}
\end{spacin}
\vskip 12pt
\end{samepage}
\begin{samepage}
\begingl[glneveryline={\PaliGlossA,\PaliGlossB}]
‘ayampi[this.also] kho[indeed] kāyo[body] evaṁdhammo[this.nature] evaṁbhāvī[this.become] evaṁanatīto’ti.[this.not.overcome]
\endgl
\nopagebreak
\linespread{0.5}
\begin{spacin}{0.2}
{\PaliGlossFT ‘This body too is of the same nature, it will be like that, it is not exempt from that fate.’}
\end{spacin}
\vskip 12pt
\end{samepage}
\begin{samepage}
\begingl[glneveryline={\PaliGlossA,\PaliGlossB}]
tassa[he] evaṁ[thus] appamattassa[careful] ātāpino[ardent] pahitattassa[able.truth] viharato[abides] ye[whatever] gehasitā[family life] sarasaṅkappā[memory.intention] te[his] pahīyanti.[abandoned]
\endgl
\nopagebreak
\linespread{0.5}
\begin{spacin}{0.2}
{\PaliGlossFT As he abides thus diligent, ardent, and resolute, his memories and intentions connected with the household life are abandoned;}
\end{spacin}
\vskip 12pt
\end{samepage}
\begin{samepage}
\begingl[glneveryline={\PaliGlossA,\PaliGlossB}]
tesaṁ[that] pahānā[removal] ajjhattameva[internally] cittaṁ[mind] santiṭṭhati[steadied] sannisīdati[settled] ekodi[single] hoti[to be] samādhiyati.[concentrated]
\endgl
\nopagebreak
\linespread{0.5}
\begin{spacin}{0.2}
{\PaliGlossFT with their abandoning his mind becomes steadied internally, quieted, brought to singleness, and concentrated.}
\end{spacin}
\vskip 12pt
\end{samepage}
\begin{samepage}
\begingl[glneveryline={\PaliGlossA,\PaliGlossB}]
evampi,[that.to] bhikkhave,[-] bhikkhu[-] kāyagatāsatiṁ[relate to body] bhāveti.[develops]
\endgl
\nopagebreak
\linespread{0.5}
\begin{spacin}{0.2}
{\PaliGlossFT That too is how a bhikkhu develops mindfulness of the body.}
\end{spacin}
\vskip 12pt
\end{samepage}
\vskip 0.2in
\begin{samepage}
\begingl[glneveryline={\PaliGlossA,\PaliGlossB}]
“puna[again] caparaṁ,[then] bhikkhave,[-] bhikkhu[-] seyyathāpi[just as] passeyya[aside] aṭṭhikāni[bone] pūtīni[rotten] cuṇṇakajātāni.[powder.arisen]
\endgl
\nopagebreak
\linespread{0.5}
\begin{spacin}{0.2}
{\PaliGlossFT “Again, as though he were to see bones more than a year old, rotted and crumbled to dust,}
\end{spacin}
\vskip 12pt
\end{samepage}
\begin{samepage}
\begingl[glneveryline={\PaliGlossA,\PaliGlossB}]
so[he] imameva[this] kāyaṁ[the body] upasaṁharati;[compare]
\endgl
\nopagebreak
\linespread{0.5}
\begin{spacin}{0.2}
{\PaliGlossFT a bhikkhu compares this same body with it thus:}
\end{spacin}
\vskip 12pt
\end{samepage}
\begin{samepage}
\begingl[glneveryline={\PaliGlossA,\PaliGlossB}]
‘ayampi[this.also] kho[indeed] kāyo[body] evaṁdhammo[this.nature] evaṁbhāvī[this.become] evaṁanatīto’ti.[this.not.overcome]
\endgl
\nopagebreak
\linespread{0.5}
\begin{spacin}{0.2}
{\PaliGlossFT ‘This body too is of the same nature, it will be like that, it is not exempt from that fate.’}
\end{spacin}
\vskip 12pt
\end{samepage}
\begin{samepage}
\begingl[glneveryline={\PaliGlossA,\PaliGlossB}]
tassa[he] evaṁ[thus] appamattassa[careful] ātāpino[ardent] pahitattassa[able.truth] viharato[abides] ye[whatever] gehasitā[family life] sarasaṅkappā[memory.intention] te[his] pahīyanti.[abandoned]
\endgl
\nopagebreak
\linespread{0.5}
\begin{spacin}{0.2}
{\PaliGlossFT As he abides thus diligent, ardent, and resolute, his memories and intentions connected with the household life are abandoned;}
\end{spacin}
\vskip 12pt
\end{samepage}
\begin{samepage}
\begingl[glneveryline={\PaliGlossA,\PaliGlossB}]
tesaṁ[that] pahānā[removal] ajjhattameva[internally] cittaṁ[mind] santiṭṭhati[steadied] sannisīdati[settled] ekodi[single] hoti[to be] samādhiyati.[concentrated]
\endgl
\nopagebreak
\linespread{0.5}
\begin{spacin}{0.2}
{\PaliGlossFT with their abandoning his mind becomes steadied internally, quieted, brought to singleness, and concentrated.}
\end{spacin}
\vskip 12pt
\end{samepage}
\begin{samepage}
\begingl[glneveryline={\PaliGlossA,\PaliGlossB}]
evampi,[that.to] bhikkhave,[-] bhikkhu[-] kāyagatāsatiṁ[relate to body] bhāveti.[develops]
\endgl
\nopagebreak
\linespread{0.5}
\begin{spacin}{0.2}
{\PaliGlossFT That too is how a bhikkhu develops mindfulness of the body.}
\end{spacin}
\vskip 12pt
\end{samepage}
\vskip 0.2in
\begin{samepage}
\begingl[glneveryline={\PaliGlossA,\PaliGlossB}]
55.[-] “puna[again] caparaṁ,[then] bhikkhave,[-] bhikkhu[-] vivicceva[separate from] kāmehi,[pleasures] vivicca[separate from] akusalehi[unskilful] dhammehi[states] savitakkaṁ[with.applied] savicāraṁ[with.investigate] vivekajaṁ[secluded] pītisukhaṁ[rapture.joy] paṭhamaṁ[first] jhānaṁ[jhāna] upasampajja[have attained] viharati.[dwells]
\endgl
\nopagebreak
\linespread{0.5}
\begin{spacin}{0.2}
{\PaliGlossFT “Again, bhikkhus, quite secluded from sensual pleasures, secluded from unwholesome states, a bhikkhu enters upon and abides in the first jhāna, which is accompanied by applied and sustained thought, with rapture and pleasure born of seclusion.}
\end{spacin}
\vskip 12pt
\end{samepage}
\begin{samepage}
\begingl[glneveryline={\PaliGlossA,\PaliGlossB}]
so[he] imameva[this] kāyaṁ[the body] vivekajena[seclusion] pītisukhena[rapture.joy] abhisandeti[overflow] parisandeti[fill] paripūreti[become full] parippharati,[completely] nāssa[no.is] kiñci[that] sabbāvato[entire] kāyassa[body] vivekajena[seclusion] pītisukhena[rapture.joy] apphuṭaṁ[unpervaded] hoti.[to be]
\endgl
\nopagebreak
\linespread{0.5}
\begin{spacin}{0.2}
{\PaliGlossFT He makes the rapture and pleasure born of seclusion drench, steep, fill, and pervade this body, so that there is no part of his whole body unpervaded by the rapture and pleasure born of seclusion.}
\end{spacin}
\vskip 12pt
\end{samepage}
\begin{samepage}
\begingl[glneveryline={\PaliGlossA,\PaliGlossB}]
seyyathāpi,[just as] bhikkhave,[-] dakkho[skilled] nhāpako[bathman] vā[or] nhāpakantevāsī[bathman.pupil] vā[or] kaṁsathāle[metal dish] nhānīyacuṇṇāni[bath.powder] ākiritvā[pour] udakena[water] paripphosakaṁ[completely] paripphosakaṁ[completely] sanneyya,[mixes] sāyaṁ[with.it] nhānīyapiṇḍi[bath.ball] snehānugatā[moise.accompanied] snehaparetā[moise.overcome] santarabāhirā[pervaded] phuṭā[ permeate] snehena[moist] na[not] ca[and] pagghariṇī;[ooze]
\endgl
\nopagebreak
\linespread{0.5}
\begin{spacin}{0.2}
{\PaliGlossFT Just as a skilled bath man or a bath man’s apprentice  heaps bath powder in a metal basin and, sprinkling it gradually with water, kneads it till the moisture wets his ball of bath powder, soaks it and pervades it inside and out, yet the ball itself does not ooze;}
\end{spacin}
\vskip 12pt
\end{samepage}
\begin{samepage}
\begingl[glneveryline={\PaliGlossA,\PaliGlossB}]
evameva[only] kho,[indeed] bhikkhave,[-] bhikkhu[-] imameva[this] kāyaṁ[the body] vivekajena[seclusion] pītisukhena[rapture.joy] abhisandeti[overflow] parisandeti[fill] paripūreti[become full] parippharati;[completely] nāssa[no.is] kiñci[that] sabbāvato[entire] kāyassa[body] vivekajena[seclusion] pītisukhena[rapture.joy] apphuṭaṁ[unpervaded] hoti.[to be]
\endgl
\nopagebreak
\linespread{0.5}
\begin{spacin}{0.2}
{\PaliGlossFT so too, a bhikkhu makes the rapture and pleasure born of seclusion  drench, steep, fill, and pervade this body, so that there is no part of his whole body unpervaded by the rapture and pleasure born of seclusion.}
\end{spacin}
\vskip 12pt
\end{samepage}
\begin{samepage}
\begingl[glneveryline={\PaliGlossA,\PaliGlossB}]
tassa[he] evaṁ[thus] appamattassa[careful] ātāpino[ardent] pahitattassa[able.truth] viharato[abides] ye[whatever] gehasitā[family life] sarasaṅkappā[memory.intention] te[his] pahīyanti.[abandoned]
\endgl
\nopagebreak
\linespread{0.5}
\begin{spacin}{0.2}
{\PaliGlossFT As he abides thus diligent, ardent, and resolute, his memories and intentions connected with the household life are abandoned;}
\end{spacin}
\vskip 12pt
\end{samepage}
\begin{samepage}
\begingl[glneveryline={\PaliGlossA,\PaliGlossB}]
tesaṁ[that] pahānā[removal] ajjhattameva[internally] cittaṁ[mind] santiṭṭhati[steadied] sannisīdati[settled] ekodi[single] hoti[to be] samādhiyati.[concentrated]
\endgl
\nopagebreak
\linespread{0.5}
\begin{spacin}{0.2}
{\PaliGlossFT with their abandoning his mind becomes steadied internally, quieted, brought to singleness, and concentrated.}
\end{spacin}
\vskip 12pt
\end{samepage}
\begin{samepage}
\begingl[glneveryline={\PaliGlossA,\PaliGlossB}]
evampi,[that.to] bhikkhave,[-] bhikkhu[-] kāyagatāsatiṁ[relate to body] bhāveti.[develops]
\endgl
\nopagebreak
\linespread{0.5}
\begin{spacin}{0.2}
{\PaliGlossFT That too is how a bhikkhu develops mindfulness of the body.}
\end{spacin}
\vskip 12pt
\end{samepage}
\vskip 0.2in
\begin{samepage}
\begingl[glneveryline={\PaliGlossA,\PaliGlossB}]
“puna[again] caparaṁ,[then] bhikkhave,[-] bhikkhu[-] vitakkavicārānaṁ[apply.investigate] vūpasamā[allaying] ajjhattaṁ[self] sampasādanaṁ[tranquilizing] cetaso[mind] ekodibhāvaṁ[onepointedness] avitakkaṁ[not.applied] avicāraṁ[not.investigate] samādhijaṁ[from concentration] pītisukhaṁ[rapture.joy] dutiyaṁ[second] jhānaṁ[jhāna] upasampajja[have attained] viharati.[dwells]
\endgl
\nopagebreak
\linespread{0.5}
\begin{spacin}{0.2}
{\PaliGlossFT “Again, bhikkhus, with the stilling of applied and sustained thought, a bhikkhu enters upon and abides in the second jhāna, which has self-confidence and singleness of mind without applied and sustained thought, with rapture and pleasure born of concentration.}
\end{spacin}
\vskip 12pt
\end{samepage}
\begin{samepage}
\begingl[glneveryline={\PaliGlossA,\PaliGlossB}]
so[he] imameva[this] kāyaṁ[the body] samādhijena[from concentration] pītisukhena[rapture.joy] abhisandeti[overflow] parisandeti[fill] paripūreti[become full] parippharati;[completely] nāssa[no.is] kiñci[that] sabbāvato[entire] kāyassa[body] samādhijena[from concentration] pītisukhena[rapture.joy] apphuṭaṁ[unpervaded] hoti.[to be]
\endgl
\nopagebreak
\linespread{0.5}
\begin{spacin}{0.2}
{\PaliGlossFT He makes the rapture and pleasure born of concentration drench, steep, fill, and pervade this body, so that there is no part of his whole body unpervaded by the rapture and pleasure born of concentration.}
\end{spacin}
\vskip 12pt
\end{samepage}
\begin{samepage}
\begingl[glneveryline={\PaliGlossA,\PaliGlossB}]
seyyathāpi,[just as] bhikkhave,[-] udakarahado[water.lake] gambhīro[deep] ubbhidodako.[welling-up]
\endgl
\nopagebreak
\linespread{0.5}
\begin{spacin}{0.2}
{\PaliGlossFT Just as though there were a lake whose waters welled up from below;}
\end{spacin}
\vskip 12pt
\end{samepage}
\begin{samepage}
\begingl[glneveryline={\PaliGlossA,\PaliGlossB}]
tassa[he] nevassa[not.it] puratthimāya[eastern] disāya[direction] udakassa[water] āyamukhaṁ[inflow] na[not] pacchimāya[western] disāya[direction] udakassa[water] āyamukhaṁ[inflow] na[not] uttarāya[northern] disāya[direction] udakassa[water] āyamukhaṁ[inflow] na[not] dakkhiṇāya[southern] disāya[direction] udakassa[water] āyamukhaṁ;[inflow]
\endgl
\nopagebreak
\linespread{0.5}
\begin{spacin}{0.2}
{\PaliGlossFT and it had no inflow from east, west, north, or south;}
\end{spacin}
\vskip 12pt
\end{samepage}
\begin{samepage}
\begingl[glneveryline={\PaliGlossA,\PaliGlossB}]
devo[-] ca[and] na[not] kālena[time] kālaṁ[time] sammā[is appeased] dhāraṁ[showers] anuppaveccheyya;[supply] atha[then] kho[indeed] tamhāva[that] udakarahadā[water.lake] sītā[cool] vāridhārā[rain cloud] ubbhijjitvā[springs up] tameva[it.even] udakarahadaṁ[water.lake] sītena[cool] vārinā[water] abhisandeyya[overflow] parisandeyya[fill] paripūreyya[become full] paripphareyya,[completely] nāssa[no.is] kiñci[that] sabbāvato[entire] udakarahadassa[water.lake] sītena[cool] vārinā[water] apphuṭaṁ[unpervaded] assa;[to be]
\endgl
\nopagebreak
\linespread{0.5}
\begin{spacin}{0.2}
{\PaliGlossFT and would not be replenished from time to time by showers of rain, then the cool fount of water welling up in the lake would make the cool water drench, steep, fill, and pervade the lake, so that there would be no part of the whole lake unpervaded by cool water;}
\end{spacin}
\vskip 12pt
\end{samepage}
\begin{samepage}
\begingl[glneveryline={\PaliGlossA,\PaliGlossB}]
evameva[only] kho,[indeed] bhikkhave,[-] bhikkhu[-] imameva[this] kāyaṁ[the body] samādhijena[from concentration] pītisukhena[rapture.joy] abhisandeti[overflow] parisandeti[fill] paripūreti[become full] parippharati,[completely] nāssa[no.is] kiñci[that] sabbāvato[entire] kāyassa[body] samādhijena[from concentration] pītisukhena[rapture.joy] apphuṭaṁ[unpervaded] hoti.[to be]
\endgl
\nopagebreak
\linespread{0.5}
\begin{spacin}{0.2}
{\PaliGlossFT so too, a bhikkhu makes the rapture and pleasure born of concentration drench, steep, fill, and pervade this body, so that there is no part of his whole body unpervaded by the rapture and pleasure born of concentration.}
\end{spacin}
\vskip 12pt
\end{samepage}
\begin{samepage}
\begingl[glneveryline={\PaliGlossA,\PaliGlossB}]
tassa[he] evaṁ[thus] appamattassa[careful] ātāpino[ardent] pahitattassa[able.truth] viharato[abides] ye[whatever] gehasitā[family life] sarasaṅkappā[memory.intention] te[his] pahīyanti.[abandoned]
\endgl
\nopagebreak
\linespread{0.5}
\begin{spacin}{0.2}
{\PaliGlossFT As he abides thus diligent, ardent, and resolute, his memories and intentions connected with the household life are abandoned;}
\end{spacin}
\vskip 12pt
\end{samepage}
\begin{samepage}
\begingl[glneveryline={\PaliGlossA,\PaliGlossB}]
tesaṁ[that] pahānā[removal] ajjhattameva[internally] cittaṁ[mind] santiṭṭhati[steadied] sannisīdati[settled] ekodi[single] hoti[to be] samādhiyati.[concentrated]
\endgl
\nopagebreak
\linespread{0.5}
\begin{spacin}{0.2}
{\PaliGlossFT with their abandoning his mind becomes steadied internally, quieted, brought to singleness, and concentrated.}
\end{spacin}
\vskip 12pt
\end{samepage}
\begin{samepage}
\begingl[glneveryline={\PaliGlossA,\PaliGlossB}]
evampi,[that.to] bhikkhave,[-] bhikkhu[-] kāyagatāsatiṁ[relate to body] bhāveti.[develops]
\endgl
\nopagebreak
\linespread{0.5}
\begin{spacin}{0.2}
{\PaliGlossFT That too is how a bhikkhu develops mindfulness of the body.}
\end{spacin}
\vskip 12pt
\end{samepage}
\vskip 0.2in
\begin{samepage}
\begingl[glneveryline={\PaliGlossA,\PaliGlossB}]
“puna[again] caparaṁ,[then] bhikkhave,[-] bhikkhu[-] pītiyā[joy] ca[and] virāgā[dispassion] upekkhako[equanimity] ca[and] viharati[dwells] sato[mindful] ca[and] sampajāno,[full aware] sukhañca[happy.and] kāyena[body] paṭisaṁvedeti,[experience] yaṁ[which] taṁ[that] ariyā[noble] ācikkhanti:[relates] “upekkhako[equanimity] satimā[mindful] sukhavihārī”ti,[pleasant.abide] tatiyaṁ[third] jhānaṁ[jhāna] upasampajja[have attained] viharati.[dwells]
\endgl
\nopagebreak
\linespread{0.5}
\begin{spacin}{0.2}
{\PaliGlossFT “Again, bhikkhus, with the fading away as well of rapture, a bhikkhu abides in equanimity, and mindful and fully aware, still feeling pleasure with the body, he enters upon and abides in the third jhāna, on account of which noble ones announce: ‘He has a pleasant abiding who has equanimity and is mindful.’}
\end{spacin}
\vskip 12pt
\end{samepage}
\begin{samepage}
\begingl[glneveryline={\PaliGlossA,\PaliGlossB}]
so[he] imameva[this] kāyaṁ[the body] nippītikena[free.joy] sukhena[happiness] abhisandeti[overflow] parisandeti[fill] paripūreti[become full] parippharati,[completely] nāssa[no.is] kiñci[that] sabbāvato[entire] kāyassa[body] nippītikena[free.joy] sukhena[happiness] apphuṭaṁ[unpervaded] hoti.[to be]
\endgl
\nopagebreak
\linespread{0.5}
\begin{spacin}{0.2}
{\PaliGlossFT He makes the pleasure divested of rapture drench, steep, fill, and pervade this body, so that there is no part of his whole body unpervaded by the pleasure divested of rapture.}
\end{spacin}
\vskip 12pt
\end{samepage}
\begin{samepage}
\begingl[glneveryline={\PaliGlossA,\PaliGlossB}]
seyyathāpi,[just as] bhikkhave,[-] uppaliniyaṁ[lotuspond] vā[or] paduminiyaṁ[lotus] vā[or] puṇḍarīkiniyaṁ[white lotus] vā[or] appekaccāni[some] uppalāni[waterlily] vā[or] padumāni[lotus] vā[or] puṇḍarīkāni[white lotus] vā[or] udake[water] jātāni[arisen] udake[water] saṁvaḍḍhāni[grown up] udakānuggatāni[water.above] antonimuggaposīni,[within] tāni[that] yāva[which] caggā[and.tips] yāva[which] ca[and] mūlā[and.roots] sītena[cool] vārinā[water] abhisannāni[overflow] parisannāni[drench] paripūrāni[fill] paripphuṭāni,[completely] nāssa[no.is] kiñci[that] sabbāvataṁ[entire] uppalānaṁ[waterlily] vā[or] padumānaṁ[lotus] vā[or] puṇḍarīkānaṁ[white lotus] vā[or] sītena[cool] vārinā[water] apphuṭaṁ[unpervaded] assa;[to be]
\endgl
\nopagebreak
\linespread{0.5}
\begin{spacin}{0.2}
{\PaliGlossFT Just as in a pond of blue or white or red lotuses, some lotuses that are born and grow in the water thrive immersed in the water without rising out of it, and cool water drenches, steeps, fills, and pervades them to their tips and their roots, so that there is no part of all those lotuses unpervaded by cool water;}
\end{spacin}
\vskip 12pt
\end{samepage}
\begin{samepage}
\begingl[glneveryline={\PaliGlossA,\PaliGlossB}]
evameva[only] kho,[indeed] bhikkhave,[-] bhikkhu[-] imameva[this] kāyaṁ[the body] nippītikena[free.joy] sukhena[happiness] abhisandeti[overflow] parisandeti[fill] paripūreti[become full] parippharati,[completely] nāssa[no.is] kiñci[that] sabbāvato[entire] kāyassa[body] nippītikena[free.joy] sukhena[happiness] apphuṭaṁ[unpervaded] hoti.[to be]
\endgl
\nopagebreak
\linespread{0.5}
\begin{spacin}{0.2}
{\PaliGlossFT so too, a bhikkhu makes the pleasure divested of rapture drench, steep, fill, and pervade this body, so that there is no part of his whole body unpervaded by the pleasure divested of rapture.}
\end{spacin}
\vskip 12pt
\end{samepage}
\begin{samepage}
\begingl[glneveryline={\PaliGlossA,\PaliGlossB}]
tassa[he] evaṁ[thus] appamattassa[careful] ātāpino[ardent] pahitattassa[able.truth] viharato[abides] ye[whatever] gehasitā[family life] sarasaṅkappā[memory.intention] te[his] pahīyanti.[abandoned]
\endgl
\nopagebreak
\linespread{0.5}
\begin{spacin}{0.2}
{\PaliGlossFT As he abides thus diligent, ardent, and resolute, his memories and intentions connected with the household life are abandoned;}
\end{spacin}
\vskip 12pt
\end{samepage}
\begin{samepage}
\begingl[glneveryline={\PaliGlossA,\PaliGlossB}]
tesaṁ[that] pahānā[removal] ajjhattameva[internally] cittaṁ[mind] santiṭṭhati[steadied] sannisīdati[settled] ekodi[single] hoti[to be] samādhiyati.[concentrated]
\endgl
\nopagebreak
\linespread{0.5}
\begin{spacin}{0.2}
{\PaliGlossFT with their abandoning his mind becomes steadied internally, quieted, brought to singleness, and concentrated.}
\end{spacin}
\vskip 12pt
\end{samepage}
\begin{samepage}
\begingl[glneveryline={\PaliGlossA,\PaliGlossB}]
evampi,[that.to] bhikkhave,[-] bhikkhu[-] kāyagatāsatiṁ[relate to body] bhāveti.[develops]
\endgl
\nopagebreak
\linespread{0.5}
\begin{spacin}{0.2}
{\PaliGlossFT That too is how a bhikkhu develops mindfulness of the body.}
\end{spacin}
\vskip 12pt
\end{samepage}
\vskip 0.2in
\begin{samepage}
\begingl[glneveryline={\PaliGlossA,\PaliGlossB}]
“puna[again] caparaṁ,[then] bhikkhave,[-] bhikkhu[-] sukhassa[happiness] ca[and] pahānā[removal] pubbeva[prior] somanassadomanassānaṁ[joy.grief] atthaṅgamā[setting down] adukkhamasukhaṁ[pleasure.pain] upekkhāsatipārisuddhiṁ[equanimity.mindful.pure] catutthaṁ[fourth] jhānaṁ[jhāna] upasampajja[have attained] viharati.[dwells]
\endgl
\nopagebreak
\linespread{0.5}
\begin{spacin}{0.2}
{\PaliGlossFT “Again, bhikkhus, with the abandoning of pleasure and pain, and with the previous disappearance of joy and grief, a bhikkhu enters upon and abides in the fourth jhāna, which has neither-pain-nor-pleasure and purity of mindfulness due to equanimity.}
\end{spacin}
\vskip 12pt
\end{samepage}
\begin{samepage}
\begingl[glneveryline={\PaliGlossA,\PaliGlossB}]
so[he] imameva[this] kāyaṁ[the body] parisuddhena[purified] cetasā[mind] pariyodātena[pure.bright] pharitvā[pervaded] nisinno[sat] hoti;[to be] nāssa[no.is] kiñci[that] sabbāvato[entire] kāyassa[body] parisuddhena[purified] cetasā[mind] pariyodātena[pure.bright] apphuṭaṁ[unpervaded] hoti.[to be]
\endgl
\nopagebreak
\linespread{0.5}
\begin{spacin}{0.2}
{\PaliGlossFT He sits pervading this body with a pure bright mind, so that there is no part of his whole body unpervaded by the pure bright mind.}
\end{spacin}
\vskip 12pt
\end{samepage}
\begin{samepage}
\begingl[glneveryline={\PaliGlossA,\PaliGlossB}]
seyyathāpi,[just as] bhikkhave,[-] puriso[man] odātena[white] vatthena[cloth] sasīsaṁ[upto head] pārupitvā[cover] nisinno[sat] assa,[to be] nāssa[no.is] kiñci[that] sabbāvato[entire] kāyassa[body] odātena[white] vatthena[cloth] apphuṭaṁ[unpervaded] assa;[to be]
\endgl
\nopagebreak
\linespread{0.5}
\begin{spacin}{0.2}
{\PaliGlossFT Just as though a man were sitting covered from head down with a white cloth, so that there would be no part of his whole body not covered by the white cloth;}
\end{spacin}
\vskip 12pt
\end{samepage}
\begin{samepage}
\begingl[glneveryline={\PaliGlossA,\PaliGlossB}]
evameva[only] kho,[indeed] bhikkhave,[-] bhikkhu[-] imameva[this] kāyaṁ[the body] parisuddhena[purified] cetasā[mind] pariyodātena[pure.bright] pharitvā[pervaded] nisinno[sat] hoti,[to be] nāssa[no.is] kiñci[that] sabbāvato[entire] kāyassa[body] parisuddhena[purified] cetasā[mind] pariyodātena[pure.bright] apphuṭaṁ[unpervaded] hoti.[to be]
\endgl
\nopagebreak
\linespread{0.5}
\begin{spacin}{0.2}
{\PaliGlossFT so too, a bhikkhu sits pervading this body with a pure bright mind, so that there is no part of his whole body unpervaded by the pure bright mind.}
\end{spacin}
\vskip 12pt
\end{samepage}
\begin{samepage}
\begingl[glneveryline={\PaliGlossA,\PaliGlossB}]
tassa[he] evaṁ[thus] appamattassa[careful] ātāpino[ardent] pahitattassa[able.truth] viharato[abides] ye[whatever] gehasitā[family life] sarasaṅkappā[memory.intention] te[his] pahīyanti.[abandoned]
\endgl
\nopagebreak
\linespread{0.5}
\begin{spacin}{0.2}
{\PaliGlossFT As he abides thus diligent, ardent, and resolute, his memories and intentions based on the household life are abandoned;}
\end{spacin}
\vskip 12pt
\end{samepage}
\begin{samepage}
\begingl[glneveryline={\PaliGlossA,\PaliGlossB}]
tesaṁ[that] pahānā[removal] ajjhattameva[internally] cittaṁ[mind] santiṭṭhati[steadied] sannisīdati[settled] ekodi[single] hoti[to be] samādhiyati.[concentrated]
\endgl
\nopagebreak
\linespread{0.5}
\begin{spacin}{0.2}
{\PaliGlossFT with their abandoning his mind becomes steadied internally, quieted, brought to singleness, and concentrated.}
\end{spacin}
\vskip 12pt
\end{samepage}
\begin{samepage}
\begingl[glneveryline={\PaliGlossA,\PaliGlossB}]
evampi,[that.to] bhikkhave,[-] bhikkhu[-] kāyagatāsatiṁ[relate to body] bhāveti.[develops]
\endgl
\nopagebreak
\linespread{0.5}
\begin{spacin}{0.2}
{\PaliGlossFT That too is how a bhikkhu develops mindfulness of the body.}
\end{spacin}
\vskip 12pt
\end{samepage}
\vskip 0.2in
\begin{samepage}
\begingl[glneveryline={\PaliGlossA,\PaliGlossB}]
156.[-] “yassa[anyone] kassaci,[who] bhikkhave,[-] kāyagatāsati[body.direct.mindful] bhāvitā[developed] bahulīkatā,[increased] antogadhāvāssa[included] kusalā[good] dhammā[states] ye[whatever] keci[which] vijjābhāgiyā.[true knowledge]
\endgl
\nopagebreak
\linespread{0.5}
\begin{spacin}{0.2}
{\PaliGlossFT “Bhikkhus, anyone who has developed and cultivated mindfulness of the body has included within himself whatever wholesome states there are that partake of true knowledge.}
\end{spacin}
\vskip 12pt
\end{samepage}
\begin{samepage}
\begingl[glneveryline={\PaliGlossA,\PaliGlossB}]
seyyathāpi,[just as] bhikkhave,[-] yassa[anyone] kassaci[who] mahāsamuddo[great.ocean] cetasā[mind] phuṭo,[pervaded] antogadhāvāssa[included] kunnadiyo[stream] yā[whatever] kāci[which] samuddaṅgamā;[ocean.go]
\endgl
\nopagebreak
\linespread{0.5}
\begin{spacin}{0.2}
{\PaliGlossFT Just as anyone who has extended his mind over the great ocean has included within it whatever streams there are that flow into the ocean;}
\end{spacin}
\vskip 12pt
\end{samepage}
\begin{samepage}
\begingl[glneveryline={\PaliGlossA,\PaliGlossB}]
evameva[only] kho,[indeed] bhikkhave,[-] yassa[anyone] kassaci[who] kāyagatāsati[body.direct.mindful] bhāvitā[developed] bahulīkatā,[increased] antogadhāvāssa[included] kusalā[good] dhammā[states] ye[whatever] keci[which] vijjābhāgiyā.[true knowledge]
\endgl
\nopagebreak
\linespread{0.5}
\begin{spacin}{0.2}
{\PaliGlossFT so too, anyone who has developed and cultivated mindfulness of the body has included within himself whatever wholesome states there are that partake of true knowledge.}
\end{spacin}
\vskip 12pt
\end{samepage}
\begin{samepage}
\begingl[glneveryline={\PaliGlossA,\PaliGlossB}]
“yassa[anyone] kassaci,[who] bhikkhave,[-] kāyagatāsati[body.direct.mindful] abhāvitā[not developed] abahulīkatā,[not cultivate] labhati[attains] tassa[he] māro[Māra] otāraṁ,[access] labhati[attains] tassa[he] māro[Māra] ārammaṇaṁ.[foundation]
\endgl
\nopagebreak
\linespread{0.5}
\begin{spacin}{0.2}
{\PaliGlossFT “Bhikkhus, when anyone has not developed and cultivated mindfulness of the body, Māra finds an opportunity and a support in him.}
\end{spacin}
\vskip 12pt
\end{samepage}
\begin{samepage}
\begingl[glneveryline={\PaliGlossA,\PaliGlossB}]
seyyathāpi,[just as] bhikkhave,[-] puriso[man] garukaṁ[heavy] silāguḷaṁ[stone.ball] allamattikāpuñje[wet.clay.mound] pakkhipeyya.[throws into]
\endgl
\nopagebreak
\linespread{0.5}
\begin{spacin}{0.2}
{\PaliGlossFT Suppose a man were to throw a heavy stone ball upon a mound of wet clay.}
\end{spacin}
\vskip 12pt
\end{samepage}
\begin{samepage}
\begingl[glneveryline={\PaliGlossA,\PaliGlossB}]
taṁ[that] kiṁ[who] maññatha,[imagines] bhikkhave,[-]
\endgl
\nopagebreak
\linespread{0.5}
\begin{spacin}{0.2}
{\PaliGlossFT What do you think, bhikkhus?}
\end{spacin}
\vskip 12pt
\end{samepage}
\begin{samepage}
\begingl[glneveryline={\PaliGlossA,\PaliGlossB}]
api[and even] nu[(affirm)] taṁ[that] garukaṁ[heavy] silāguḷaṁ[stone.ball] allamattikāpuñje[wet.clay.mound] labhetha[attains] otāran”ti?[access]
\endgl
\nopagebreak
\linespread{0.5}
\begin{spacin}{0.2}
{\PaliGlossFT Would that heavy ball find entry into that mound of wet clay?”}
\end{spacin}
\vskip 12pt
\end{samepage}
\begin{samepage}
\begingl[glneveryline={\PaliGlossA,\PaliGlossB}]
“evaṁ,[thus] bhante”.[Sir]
\endgl
\nopagebreak
\linespread{0.5}
\begin{spacin}{0.2}
{\PaliGlossFT “Yes, venerable sir.”}
\end{spacin}
\vskip 12pt
\end{samepage}
\begin{samepage}
\begingl[glneveryline={\PaliGlossA,\PaliGlossB}]
“evameva[only] kho,[indeed] bhikkhave,[-] yassa[anyone] kassaci[who] kāyagatāsati[body.direct.mindful] abhāvitā[not developed] abahulīkatā,[not cultivate] labhati[attains] tassa[he] māro[Māra] otāraṁ,[access] labhati[attains] tassa[he] māro[Māra] ārammaṇaṁ.[foundation]
\endgl
\nopagebreak
\linespread{0.5}
\begin{spacin}{0.2}
{\PaliGlossFT “So too, bhikkhus, when anyone has not developed and cultivated mindfulness of the body, Māra finds an opportunity and a support in him.}
\end{spacin}
\vskip 12pt
\end{samepage}
\begin{samepage}
\begingl[glneveryline={\PaliGlossA,\PaliGlossB}]
seyyathāpi,[just as] bhikkhave,[-] sukkhaṁ[dry] kaṭṭhaṁ[piece wood] koḷāpaṁ;[sapless] atha[then] puriso[man] āgaccheyya[comes to] uttarāraṇiṁ[upper.firestick] ādāya[have taken]
\endgl
\nopagebreak
\linespread{0.5}
\begin{spacin}{0.2}
{\PaliGlossFT “Suppose there were a dry sapless piece of wood, and a man came with an upper fire-stick, thinking:}
\end{spacin}
\vskip 12pt
\end{samepage}
\begin{samepage}
\begingl[glneveryline={\PaliGlossA,\PaliGlossB}]
‘aggiṁ[fire] abhinibbattessāmi,[produce] tejo[heat] pātukarissāmī’ti.[manifest]
\endgl
\nopagebreak
\linespread{0.5}
\begin{spacin}{0.2}
{\PaliGlossFT ‘I shall light a fire, I shall produce heat.’}
\end{spacin}
\vskip 12pt
\end{samepage}
\begin{samepage}
\begingl[glneveryline={\PaliGlossA,\PaliGlossB}]
taṁ[that] kiṁ[who] maññatha,[imagines] bhikkhave,[-]
\endgl
\nopagebreak
\linespread{0.5}
\begin{spacin}{0.2}
{\PaliGlossFT What do you think, bhikkhus?}
\end{spacin}
\vskip 12pt
\end{samepage}
\begin{samepage}
\begingl[glneveryline={\PaliGlossA,\PaliGlossB}]
api[and even] nu[(affirm)] so[he] puriso[man] amuṁ[up to] sukkhaṁ[dry] kaṭṭhaṁ[piece wood] koḷāpaṁ[sapless] uttarāraṇiṁ[upper.firestick] ādāya[have taken] abhimanthento[agitate] aggiṁ[fire] abhinibbatteyya,[produced] tejo[heat] pātukareyyā”ti?[manifest]
\endgl
\nopagebreak
\linespread{0.5}
\begin{spacin}{0.2}
{\PaliGlossFT Could the man light a fire and produce heat by rubbing the dry sapless piece of wood with an upper fire-stick?”}
\end{spacin}
\vskip 12pt
\end{samepage}
\begin{samepage}
\begingl[glneveryline={\PaliGlossA,\PaliGlossB}]
“evaṁ,[thus] bhante”.[Sir]
\endgl
\nopagebreak
\linespread{0.5}
\begin{spacin}{0.2}
{\PaliGlossFT “Yes, venerable sir.”}
\end{spacin}
\vskip 12pt
\end{samepage}
\begin{samepage}
\begingl[glneveryline={\PaliGlossA,\PaliGlossB}]
“evameva[only] kho,[indeed] bhikkhave,[-] yassa[anyone] kassaci[who] kāyagatāsati[body.direct.mindful] abhāvitā[not developed] abahulīkatā,[not cultivate] labhati[attains] tassa[he] māro[Māra] otāraṁ,[access] labhati[attains] tassa[he] māro[Māra] ārammaṇaṁ.[foundation]
\endgl
\nopagebreak
\linespread{0.5}
\begin{spacin}{0.2}
{\PaliGlossFT “So too, bhikkhus, when anyone has not developed and cultivated mindfulness of the body, Māra finds an opportunity and a support in him.}
\end{spacin}
\vskip 12pt
\end{samepage}
\begin{samepage}
\begingl[glneveryline={\PaliGlossA,\PaliGlossB}]
seyyathāpi,[just as] bhikkhave,[-] udakamaṇiko[water.jar] ritto[empty] tuccho[devoid] ādhāre[stand] ṭhapito;[placed] atha[then] puriso[man] āgaccheyya[comes to] udakabhāraṁ[water.load] ādāya.[have taken]
\endgl
\nopagebreak
\linespread{0.5}
\begin{spacin}{0.2}
{\PaliGlossFT “Suppose there were a hollow empty water jug set out on a stand, and a man came with a supply of water.}
\end{spacin}
\vskip 12pt
\end{samepage}
\begin{samepage}
\begingl[glneveryline={\PaliGlossA,\PaliGlossB}]
taṁ[that] kiṁ[who] maññatha,[imagines] bhikkhave,[-]
\endgl
\nopagebreak
\linespread{0.5}
\begin{spacin}{0.2}
{\PaliGlossFT What do you think, bhikkhus?}
\end{spacin}
\vskip 12pt
\end{samepage}
\begin{samepage}
\begingl[glneveryline={\PaliGlossA,\PaliGlossB}]
api[and even] nu[(affirm)] so[he] puriso[man] labhetha[attains] udakassa[water] nikkhepanan”ti?[put down]
\endgl
\nopagebreak
\linespread{0.5}
\begin{spacin}{0.2}
{\PaliGlossFT Could the man pour the water into the jug?”}
\end{spacin}
\vskip 12pt
\end{samepage}
\begin{samepage}
\begingl[glneveryline={\PaliGlossA,\PaliGlossB}]
“evaṁ,[thus] bhante”.[Sir]
\endgl
\nopagebreak
\linespread{0.5}
\begin{spacin}{0.2}
{\PaliGlossFT “Yes, venerable sir.”}
\end{spacin}
\vskip 12pt
\end{samepage}
\begin{samepage}
\begingl[glneveryline={\PaliGlossA,\PaliGlossB}]
“evameva[only] kho,[indeed] bhikkhave,[-] yassa[anyone] kassaci[who] kāyagatāsati[body.direct.mindful] abhāvitā[not developed] abahulīkatā,[not cultivate] labhati[attains] tassa[he] māro[Māra] otāraṁ,[access] labhati[attains] tassa[he] māro[Māra] ārammaṇaṁ”.[foundation]
\endgl
\nopagebreak
\linespread{0.5}
\begin{spacin}{0.2}
{\PaliGlossFT “So too, bhikkhus, when anyone has not developed and cultivated mindfulness of the body, Māra finds an opportunity and a support in him.}
\end{spacin}
\vskip 12pt
\end{samepage}
\begin{samepage}
\begingl[glneveryline={\PaliGlossA,\PaliGlossB}]
157.[-] “yassa[anyone] kassaci,[who] bhikkhave,[-] kāyagatāsati[body.direct.mindful] bhāvitā[developed] bahulīkatā,[increased] na[not] tassa[he] labhati[attains] māro[Māra] otāraṁ,[access] na[not] tassa[he] labhati[attains] māro[Māra] ārammaṇaṁ.[foundation]
\endgl
\nopagebreak
\linespread{0.5}
\begin{spacin}{0.2}
{\PaliGlossFT “Bhikkhus, when anyone has developed and cultivated mindfulness of the body, Māra cannot find an opportunity or a support in him.}
\end{spacin}
\vskip 12pt
\end{samepage}
\begin{samepage}
\begingl[glneveryline={\PaliGlossA,\PaliGlossB}]
seyyathāpi,[just as] bhikkhave,[-] puriso[man] lahukaṁ[light] suttaguḷaṁ[string.ball] sabbasāramaye[all.pith.made] aggaḷaphalake[door panel] pakkhipeyya.[throws into]
\endgl
\nopagebreak
\linespread{0.5}
\begin{spacin}{0.2}
{\PaliGlossFT Suppose a man were to throw a light ball of string at a door-panel made entirely of heartwood.}
\end{spacin}
\vskip 12pt
\end{samepage}
\begin{samepage}
\begingl[glneveryline={\PaliGlossA,\PaliGlossB}]
taṁ[that] kiṁ[who] maññatha,[imagines] bhikkhave,[-]
\endgl
\nopagebreak
\linespread{0.5}
\begin{spacin}{0.2}
{\PaliGlossFT What do you think, bhikkhus?}
\end{spacin}
\vskip 12pt
\end{samepage}
\begin{samepage}
\begingl[glneveryline={\PaliGlossA,\PaliGlossB}]
api[and even] nu[(affirm)] so[he] puriso[man] taṁ[that] lahukaṁ[light] suttaguḷaṁ[string.ball] sabbasāramaye[all.pith.made] aggaḷaphalake[door panel] labhetha[attains] otāran”ti?[access]
\endgl
\nopagebreak
\linespread{0.5}
\begin{spacin}{0.2}
{\PaliGlossFT Would that light ball of string find entry through that door-panel made entirely of heartwood?”}
\end{spacin}
\vskip 12pt
\end{samepage}
\begin{samepage}
\begingl[glneveryline={\PaliGlossA,\PaliGlossB}]
“no[(neg)] hetaṁ,[indeed.this] bhante”.[Sir]
\endgl
\nopagebreak
\linespread{0.5}
\begin{spacin}{0.2}
{\PaliGlossFT “No, venerable sir.”}
\end{spacin}
\vskip 12pt
\end{samepage}
\begin{samepage}
\begingl[glneveryline={\PaliGlossA,\PaliGlossB}]
“evameva[only] kho,[indeed] bhikkhave,[-] yassa[anyone] kassaci[who] kāyagatāsati[body.direct.mindful] bhāvitā[developed] bahulīkatā,[increased] na[not] tassa[he] labhati[attains] māro[Māra] otāraṁ,[access] na[not] tassa[he] labhati[attains] māro[Māra] ārammaṇaṁ.[foundation]
\endgl
\nopagebreak
\linespread{0.5}
\begin{spacin}{0.2}
{\PaliGlossFT “So too, bhikkhus, when anyone has developed and cultivated mindfulness of the body, Māra cannot find an opportunity or a support in him.}
\end{spacin}
\vskip 12pt
\end{samepage}
\begin{samepage}
\begingl[glneveryline={\PaliGlossA,\PaliGlossB}]
seyyathāpi,[just as] bhikkhave,[-] allaṁ[wet] kaṭṭhaṁ[piece wood] sasnehaṁ;[with.oily] atha[then] puriso[man] āgaccheyya[comes to] uttarāraṇiṁ[upper.firestick] ādāya;[have taken]
\endgl
\nopagebreak
\linespread{0.5}
\begin{spacin}{0.2}
{\PaliGlossFT “Suppose there were a wet sappy piece of wood, and a man came with an upper fire-stick, thinking:}
\end{spacin}
\vskip 12pt
\end{samepage}
\begin{samepage}
\begingl[glneveryline={\PaliGlossA,\PaliGlossB}]
‘aggiṁ[fire] abhinibbattessāmi,[produce] tejo[heat] pātukarissāmī’ti.[manifest]
\endgl
\nopagebreak
\linespread{0.5}
\begin{spacin}{0.2}
{\PaliGlossFT ‘I shall light a fire, I shall produce heat.’}
\end{spacin}
\vskip 12pt
\end{samepage}
\begin{samepage}
\begingl[glneveryline={\PaliGlossA,\PaliGlossB}]
taṁ[that] kiṁ[who] maññatha,[imagines] bhikkhave,[-]
\endgl
\nopagebreak
\linespread{0.5}
\begin{spacin}{0.2}
{\PaliGlossFT What do you think, bhikkhus?}
\end{spacin}
\vskip 12pt
\end{samepage}
\begin{samepage}
\begingl[glneveryline={\PaliGlossA,\PaliGlossB}]
api[and even] nu[(affirm)] so[he] puriso[man] amuṁ[up to] allaṁ[wet] kaṭṭhaṁ[piece wood] sasnehaṁ[with.oily] uttarāraṇiṁ[upper.firestick] ādāya[have taken] abhimanthento[agitate] aggiṁ[fire] abhinibbatteyya,[produced] tejo[heat] pātukareyyā”ti?[manifest]
\endgl
\nopagebreak
\linespread{0.5}
\begin{spacin}{0.2}
{\PaliGlossFT Could the man light a fire and produce heat by taking the upper fire-stick and rubbing it against the wet sappy piece of wood?}
\end{spacin}
\vskip 12pt
\end{samepage}
\begin{samepage}
\begingl[glneveryline={\PaliGlossA,\PaliGlossB}]
“no[(neg)] hetaṁ,[indeed.this] bhante”.[Sir]
\endgl
\nopagebreak
\linespread{0.5}
\begin{spacin}{0.2}
{\PaliGlossFT —“No, venerable sir.”}
\end{spacin}
\vskip 12pt
\end{samepage}
\begin{samepage}
\begingl[glneveryline={\PaliGlossA,\PaliGlossB}]
“evameva[only] kho,[indeed] bhikkhave,[-] yassa[anyone] kassaci[who] kāyagatāsati[body.direct.mindful] bhāvitā[developed] bahulīkatā,[increased] na[not] tassa[he] labhati[attains] māro[Māra] otāraṁ,[access] na[not] tassa[he] labhati[attains] māro[Māra] ārammaṇaṁ.[foundation]
\endgl
\nopagebreak
\linespread{0.5}
\begin{spacin}{0.2}
{\PaliGlossFT “So too, bhikkhus, when anyone has developed and cultivated mindfulness of the body, Māra cannot find an opportunity or a support in him.}
\end{spacin}
\vskip 12pt
\end{samepage}
\begin{samepage}
\begingl[glneveryline={\PaliGlossA,\PaliGlossB}]
seyyathāpi,[just as] bhikkhave,[-] udakamaṇiko[water.jar] pūro[full] udakassa[water] samatittiko[brimful] kākapeyyo[crow.drinkable] ādhāre[stand] ṭhapito;[placed] atha[then] puriso[man] āgaccheyya[comes to] udakabhāraṁ[water.load] ādāya.[have taken]
\endgl
\nopagebreak
\linespread{0.5}
\begin{spacin}{0.2}
{\PaliGlossFT “Suppose, set out on a stand, there were a water jug full of water right up to the brim so that crows could drink from it, and a man came with a supply of water.}
\end{spacin}
\vskip 12pt
\end{samepage}
\begin{samepage}
\begingl[glneveryline={\PaliGlossA,\PaliGlossB}]
taṁ[that] kiṁ[who] maññatha,[imagines] bhikkhave,[-]
\endgl
\nopagebreak
\linespread{0.5}
\begin{spacin}{0.2}
{\PaliGlossFT What do you think, bhikkhus?}
\end{spacin}
\vskip 12pt
\end{samepage}
\begin{samepage}
\begingl[glneveryline={\PaliGlossA,\PaliGlossB}]
api[and even] nu[(affirm)] so[he] puriso[man] labhetha[attains] udakassa[water] nikkhepanan”ti?[put down]
\endgl
\nopagebreak
\linespread{0.5}
\begin{spacin}{0.2}
{\PaliGlossFT Could the man pour the water into the jug?”}
\end{spacin}
\vskip 12pt
\end{samepage}
\begin{samepage}
\begingl[glneveryline={\PaliGlossA,\PaliGlossB}]
“no[(neg)] hetaṁ,[indeed.this] bhante”.[Sir]
\endgl
\nopagebreak
\linespread{0.5}
\begin{spacin}{0.2}
{\PaliGlossFT “No, venerable sir.”}
\end{spacin}
\vskip 12pt
\end{samepage}
\begin{samepage}
\begingl[glneveryline={\PaliGlossA,\PaliGlossB}]
“evameva[only] kho,[indeed] bhikkhave,[-] yassa[anyone] kassaci[who] kāyagatāsati[body.direct.mindful] bhāvitā[developed] bahulīkatā,[increased] na[not] tassa[he] labhati[attains] māro[Māra] otāraṁ,[access] na[not] tassa[he] labhati[attains] māro[Māra] ārammaṇaṁ”.[foundation]
\endgl
\nopagebreak
\linespread{0.5}
\begin{spacin}{0.2}
{\PaliGlossFT “So too, bhikkhus, when anyone has developed and cultivated mindfulness of the body, Māra cannot find an opportunity or a support in him.}
\end{spacin}
\vskip 12pt
\end{samepage}
\begin{samepage}
\begingl[glneveryline={\PaliGlossA,\PaliGlossB}]
158.[-] “yassa[anyone] kassaci,[who] bhikkhave,[-] kāyagatāsati[body.direct.mindful] bhāvitā[developed] bahulīkatā,[increased] so[he] yassa[anyone] yassa[anyone] abhiññāsacchikaraṇīyassa[high-knowledge.fit to know] dhammassa[nature] cittaṁ[mind] abhininnāmeti[towards] abhiññāsacchikiriyāya,[high-knowledge.fit to know] ta[that] tatre[there] sakkhibhabbataṁ[witness.ability] pāpuṇāti[attains] sati[mindful] satiāyatane.[mindfl.sphere]
\endgl
\nopagebreak
\linespread{0.5}
\begin{spacin}{0.2}
{\PaliGlossFT “Bhikkhus, when anyone has developed and cultivated mindfulness of the body, then when he inclines his mind towards realising any state that may be realised by direct knowledge, he attains the ability to witness any aspect therein, there being a suitable basis.}
\end{spacin}
\vskip 12pt
\end{samepage}
\begin{samepage}
\begingl[glneveryline={\PaliGlossA,\PaliGlossB}]
seyyathāpi,[just as] bhikkhave,[-] udakamaṇiko[water.jar] pūro[full] udakassa[water] samatittiko[brimful] kākapeyyo[crow.drinkable] ādhāre[stand] ṭhapito.[placed]
\endgl
\nopagebreak
\linespread{0.5}
\begin{spacin}{0.2}
{\PaliGlossFT Suppose, set out on a stand, there were a water jug full of water right up to the brim so that crows could drink from it.}
\end{spacin}
\vskip 12pt
\end{samepage}
\begin{samepage}
\begingl[glneveryline={\PaliGlossA,\PaliGlossB}]
tamenaṁ[-] balavā[powerful] puriso[man] yato[since] yato[since] āviñcheyya,[turn] āgaccheyya[comes to] udakan”ti?[water]
\endgl
\nopagebreak
\linespread{0.5}
\begin{spacin}{0.2}
{\PaliGlossFT Whenever a strong man tips it, would water come out?”}
\end{spacin}
\vskip 12pt
\end{samepage}
\begin{samepage}
\begingl[glneveryline={\PaliGlossA,\PaliGlossB}]
“evaṁ,[thus] bhante”.[Sir]
\endgl
\nopagebreak
\linespread{0.5}
\begin{spacin}{0.2}
{\PaliGlossFT “Yes, venerable sir.”}
\end{spacin}
\vskip 12pt
\end{samepage}
\begin{samepage}
\begingl[glneveryline={\PaliGlossA,\PaliGlossB}]
“evameva[only] kho,[indeed] bhikkhave,[-] yassa[anyone] kassaci[who] kāyagatāsati[body.direct.mindful] bhāvitā[developed] bahulīkatā[increased] so,[he] yassa[anyone] yassa[anyone] abhiññāsacchikaraṇīyassa[high-knowledge.fit to know] dhammassa[nature] cittaṁ[mind] abhininnāmeti[towards] abhiññāsacchikiriyāya,[high-knowledge.fit to know] tatra[there] tatreva[therin] sakkhibhabbataṁ[witness.ability] pāpuṇāti[attains] sati[mindful] satiāyatane.[mindfl.sphere]
\endgl
\nopagebreak
\linespread{0.5}
\begin{spacin}{0.2}
{\PaliGlossFT “So too, bhikkhus, when anyone has developed and cultivated mindfulness of the body, then when he inclines his mind towards realising any state that may be realised by direct knowledge, he attains the ability to witness any aspect therein, there being a suitable basis.}
\end{spacin}
\vskip 12pt
\end{samepage}
\begin{samepage}
\begingl[glneveryline={\PaliGlossA,\PaliGlossB}]
seyyathāpi,[just as] bhikkhave,[-] same[level] bhūmibhāge[plot o'land] caturassā[square] pokkharaṇī[pond] assa[to be] āḷibandhā[embank.bound] pūrā[full] udakassa[water] samatittikā[brimful] kākapeyyā.[crow.drinkable]
\endgl
\nopagebreak
\linespread{0.5}
\begin{spacin}{0.2}
{\PaliGlossFT “Suppose there were a square pond on level ground, surrounded by an embankment, full of water right up to the brim so that crows could drink from it.}
\end{spacin}
\vskip 12pt
\end{samepage}
\begin{samepage}
\begingl[glneveryline={\PaliGlossA,\PaliGlossB}]
tamenaṁ[-] balavā[powerful] puriso[man] yato[since] yato[since] āḷiṁ[embankment] muñceyya[release] āgaccheyya[comes to] udakan”ti?[water]
\endgl
\nopagebreak
\linespread{0.5}
\begin{spacin}{0.2}
{\PaliGlossFT Whenever a strong man loosens the embankment, would water come out?}
\end{spacin}
\vskip 12pt
\end{samepage}
\begin{samepage}
\begingl[glneveryline={\PaliGlossA,\PaliGlossB}]
“evaṁ,[thus] bhante”.[Sir]
\endgl
\nopagebreak
\linespread{0.5}
\begin{spacin}{0.2}
{\PaliGlossFT “Yes, venerable sir.”}
\end{spacin}
\vskip 12pt
\end{samepage}
\begin{samepage}
\begingl[glneveryline={\PaliGlossA,\PaliGlossB}]
“evameva[only] kho,[indeed] bhikkhave,[-] yassa[anyone] kassaci[who] kāyagatāsati[body.direct.mindful] bhāvitā[developed] bahulīkatā,[increased] so[he] yassa[anyone] yassa[anyone] abhiññāsacchikaraṇīyassa[high-knowledge.fit to know] dhammassa[nature] cittaṁ[mind] abhininnāmeti[towards] abhiññāsacchikiriyāya,[high-knowledge.fit to know] tatra[there] tatreva[therin] sakkhibhabbataṁ[witness.ability] pāpuṇāti[attains] sati[mindful] satiāyatane.[mindfl.sphere]
\endgl
\nopagebreak
\linespread{0.5}
\begin{spacin}{0.2}
{\PaliGlossFT “So too, bhikkhus, when anyone has developed and cultivated mindfulness of the body, then when he inclines his mind towards realising any state that may be realised by direct knowledge, he attains the ability to witness any aspect therein, there being a suitable basis.}
\end{spacin}
\vskip 12pt
\end{samepage}
\begin{samepage}
\begingl[glneveryline={\PaliGlossA,\PaliGlossB}]
seyyathāpi,[just as] bhikkhave,[-] subhūmiyaṁ[well.ground] catumahāpathe[4.road] ājaññaratho[good breed.chariot] yutto[yoked] assa[to be] ṭhito[stand] odhastapatodo;[lying.goad] tamenaṁ[-] dakkho[skilled] yoggācariyo[suitable] assadammasārathi[horse.tamed.driver] abhiruhitvā[ascends] vāmena[left] hatthena[hand] rasmiyo[rein] gahetvā[have taken] dakkhiṇena[right] hatthena[hand] patodaṁ[goad] gahetvā[have taken] yenicchakaṁ[where.desire] yadicchakaṁ[if.desire] sāreyyāpi[move along] paccāsāreyyāpi;[make go]
\endgl
\nopagebreak
\linespread{0.5}
\begin{spacin}{0.2}
{\PaliGlossFT “Suppose there were a chariot on even ground at the crossroads, harnessed to thoroughbreds, waiting with goad lying ready, so that a skilled trainer, a charioteer of horses to be tamed, might mount it, and taking the reins in his left hand and the goad in his right hand, might drive out and back by any road whenever he likes.}
\end{spacin}
\vskip 12pt
\end{samepage}
\begin{samepage}
\begingl[glneveryline={\PaliGlossA,\PaliGlossB}]
evameva[only] kho,[indeed] bhikkhave,[-] yassa[anyone] kassaci[who] kāyagatāsati[body.direct.mindful] bhāvitā[developed] bahulīkatā,[increased] so[he] yassa[anyone] yassa[anyone] abhiññāsacchikaraṇīyassa[high-knowledge.fit to know] dhammassa[nature] cittaṁ[mind] abhininnāmeti[towards] abhiññāsacchikiriyāya,[high-knowledge.fit to know] tatra[there] tatreva[therin] sakkhibhabbataṁ[witness.ability] pāpuṇāti[attains] sati[mindful] satiāyatane”.[mindfl.sphere]
\endgl
\nopagebreak
\linespread{0.5}
\begin{spacin}{0.2}
{\PaliGlossFT So too, bhikkhus, when anyone has developed and cultivated mindfulness of the body, then when he inclines his mind towards realising any state that may be realised by direct knowledge, he attains the ability to witness any aspect therein, there being a suitable basis.}
\end{spacin}
\vskip 12pt
\end{samepage}
\begin{samepage}
\begingl[glneveryline={\PaliGlossA,\PaliGlossB}]
159.[-] “kāyagatāya,[body.directed] bhikkhave,[-] satiyā[mindful] āsevitāya[frequent.pract] bhāvitāya[develope] bahulīkatāya[cultivate] yānīkatāya[made a habit] vatthukatāya[made basis] anuṭṭhitāya[establish] paricitāya[accumulate] susamāraddhāya[well undertaken] dasānisaṁsā[ten.benefit] pāṭikaṅkhā.[expect]
\endgl
\nopagebreak
\linespread{0.5}
\begin{spacin}{0.2}
{\PaliGlossFT “Bhikkhus, when mindfulness of the body has been repeatedly practised, developed, cultivated, used as a vehicle, used as a basis, established, consolidated, and well undertaken, these ten benefits may be expected. What ten?}
\end{spacin}
\vskip 12pt
\end{samepage}
\begin{samepage}
\begingl[glneveryline={\PaliGlossA,\PaliGlossB}]
(i)[-] "aratiratisaho[dislike.like.endure] hoti,[to be] na[not] ca[and] taṁ[that] arati[dislike] sahati,[enduring] uppannaṁ[arisen] aratiṁ[dislike] abhibhuyya[conqueror] viharati.[dwells]
\endgl
\nopagebreak
\linespread{0.5}
\begin{spacin}{0.2}
{\PaliGlossFT (i) “One becomes a conqueror of discontent and delight, and discontent does not conquer oneself; one abides overcoming discontent whenever it arises.}
\end{spacin}
\vskip 12pt
\end{samepage}
\begin{samepage}
\begingl[glneveryline={\PaliGlossA,\PaliGlossB}]
(ii)[-] “bhayabheravasaho[fear.dread.endure] hoti,[to be] na[not] ca[and] taṁ[that] bhayabheravaṁ[fear.dread] sahati,[enduring] uppannaṁ[arisen] bhayabheravaṁ[fear.dread] abhibhuyya[conqueror] viharati.[dwells]
\endgl
\nopagebreak
\linespread{0.5}
\begin{spacin}{0.2}
{\PaliGlossFT (ii) “One becomes a conqueror of fear and dread, and fear and dread do not conquer oneself; one abides overcoming fear and dread whenever they arise.}
\end{spacin}
\vskip 12pt
\end{samepage}
\begin{samepage}
\begingl[glneveryline={\PaliGlossA,\PaliGlossB}]
(iii)[-] “khamo[bears] hoti[to be] sītassa[cold] uṇhassa[heat] jighacchāya[hungry] pipāsāya[thirst] ḍaṁsamakasavātātapasarīsapasamphassānaṁ[fly.mosquito.wind.heat.reptile.contact] duruttānaṁ[bad speech] durāgatānaṁ[off.color] vacanapathānaṁ,[way spoken] uppannānaṁ[arisen] sārīrikānaṁ[body connected] vedanānaṁ[sensation] dukkhānaṁ[painful] tibbānaṁ[piercing] kharānaṁ[rough] kaṭukānaṁ[severe] asātānaṁ[disagreeable] amanāpānaṁ[detesful] pāṇaharānaṁ[taking life] adhivāsakajātiko[endures] hoti.[to be]
\endgl
\nopagebreak
\linespread{0.5}
\begin{spacin}{0.2}
{\PaliGlossFT (iii) “One bears cold and heat, hunger and thirst, and contact with gadflies, mosquitoes, wind, the sun, and creeping things; one endures ill-spoken, unwelcome words and arisen bodily feelings that are painful, racking, sharp, piercing, disagreeable, distressing, and menacing to life.}
\end{spacin}
\vskip 12pt
\end{samepage}
\begin{samepage}
\begingl[glneveryline={\PaliGlossA,\PaliGlossB}]
(iv)[-] “catunnaṁ[four] jhānānaṁ[jhānā] ābhicetasikānaṁ[radiant.mind] diṭṭhadhammasukhavihārānaṁ[seen.dhamma.pleasant.abide] nikāmalābhī[desire.gain] hoti[to be] akicchalābhī[not.difficult.gain] akasiralābhī.[not.trouble.gain]
\endgl
\nopagebreak
\linespread{0.5}
\begin{spacin}{0.2}
{\PaliGlossFT (iv) “One obtains at will, without trouble or difficulty, the four jhānas that constitute the higher mind and  provide a pleasant abiding here and now.}
\end{spacin}
\vskip 12pt
\end{samepage}
\begin{samepage}
\begingl[glneveryline={\PaliGlossA,\PaliGlossB}]
(v)[-] “so[he] anekavihitaṁ[many.prepared] iddhividhaṁ[power.kind] paccānubhoti.[partake]
\endgl
\nopagebreak
\linespread{0.5}
\begin{spacin}{0.2}
{\PaliGlossFT (v) “One wields the various kinds of supernormal power:}
\end{spacin}
\vskip 12pt
\end{samepage}
\begin{samepage}
\begingl[glneveryline={\PaliGlossA,\PaliGlossB}]
ekopi[one] hutvā[having been] bahudhā[many] hoti,[to be] bahudhāpi[many] hutvā[having been] eko[one] hoti,[to be]
\endgl
\nopagebreak
\linespread{0.5}
\begin{spacin}{0.2}
{\PaliGlossFT having been one, he becomes many; having been many, he becomes one;}
\end{spacin}
\vskip 12pt
\end{samepage}
\begin{samepage}
\begingl[glneveryline={\PaliGlossA,\PaliGlossB}]
āvibhāvaṁ[before eye.become] tirobhāvaṁ;[disappear] tirokuṭṭaṁ[through.wall] tiropākāraṁ[through.enclosure] tiropabbataṁ[through.mountain] asajjamāno[not.touching] gacchati,[goes] seyyathāpi[just as] ākāse;[space]
\endgl
\nopagebreak
\linespread{0.5}
\begin{spacin}{0.2}
{\PaliGlossFT one appears and vanishes; one goes unhindered through a wall, through an enclosure, through a mountain as though through space;}
\end{spacin}
\vskip 12pt
\end{samepage}
\begin{samepage}
\begingl[glneveryline={\PaliGlossA,\PaliGlossB}]
pathaviyāpi[earth] ummujjanimujjaṁ[emerge.dive] karoti,[does] seyyathāpi[just as] udake;[water]
\endgl
\nopagebreak
\linespread{0.5}
\begin{spacin}{0.2}
{\PaliGlossFT one dives in and out of the earth as though it were water;}
\end{spacin}
\vskip 12pt
\end{samepage}
\begin{samepage}
\begingl[glneveryline={\PaliGlossA,\PaliGlossB}]
udakepi[water] abhijjamāne[not breaking] gacchati,[goes] seyyathāpi[just as] pathaviyaṁ;[earth]
\endgl
\nopagebreak
\linespread{0.5}
\begin{spacin}{0.2}
{\PaliGlossFT one walks on water without sinking as though it were earth;}
\end{spacin}
\vskip 12pt
\end{samepage}
\begin{samepage}
\begingl[glneveryline={\PaliGlossA,\PaliGlossB}]
ākāsepi[space] pallaṅkena[cross-legged] kamati,[goes] seyyathāpi[just as] pakkhī[winged-one] sakuṇo;[bird]
\endgl
\nopagebreak
\linespread{0.5}
\begin{spacin}{0.2}
{\PaliGlossFT seated cross-legged, one travels in space like a bird;}
\end{spacin}
\vskip 12pt
\end{samepage}
\begin{samepage}
\begingl[glneveryline={\PaliGlossA,\PaliGlossB}]
imepi[-] candimasūriye[moon.sun] evaṁmahiddhike[like.great.power] evaṁmahānubhāve[like.great.majesty] pāṇinā[hand] parimasati[completely] parimajjati,[strokes]
\endgl
\nopagebreak
\linespread{0.5}
\begin{spacin}{0.2}
{\PaliGlossFT with his hand one touches and strokes the moon and sun so powerful and mighty;}
\end{spacin}
\vskip 12pt
\end{samepage}
\begin{samepage}
\begingl[glneveryline={\PaliGlossA,\PaliGlossB}]
yāva[which] brahmalokāpi[brahma.world] kāyena[body] vasaṁ[control] vatteti.[exercise]
\endgl
\nopagebreak
\linespread{0.5}
\begin{spacin}{0.2}
{\PaliGlossFT one wields bodily mastery even as far as the Brahma-world.}
\end{spacin}
\vskip 12pt
\end{samepage}
\begin{samepage}
\begingl[glneveryline={\PaliGlossA,\PaliGlossB}]
(vi)[-] “dibbāya[divine] sotadhātuyā[ear.element] visuddhāya[purified] atikkantamānusikāya[surpassed.human.heap] ubho[both] sadde[sound] suṇāti[hears] dibbe[divine] ca[and] mānuse[human] ca,[and] ye[whatever] dūre[far] santike[near] ca.[and]
\endgl
\nopagebreak
\linespread{0.5}
\begin{spacin}{0.2}
{\PaliGlossFT (vi) “With the divine ear element, which is purified and surpasses the human, one hears both kinds of sounds, the divine and the human, those that are far as well as near.}
\end{spacin}
\vskip 12pt
\end{samepage}
\begin{samepage}
\begingl[glneveryline={\PaliGlossA,\PaliGlossB}]
(vii)[-] “parasattānaṁ[other.being] parapuggalānaṁ[other.person] cetasā[mind] ceto[mind] paricca[encompass] pajānāti.[know clearly]
\endgl
\nopagebreak
\linespread{0.5}
\begin{spacin}{0.2}
{\PaliGlossFT (vii) “One understands the minds of other beings, of other persons, having encompassed them with one’s own mind.}
\end{spacin}
\vskip 12pt
\end{samepage}
\begin{samepage}
\begingl[glneveryline={\PaliGlossA,\PaliGlossB}]
sarāgaṁ[with.lust] vā[or] cittaṁ[mind] ‘sarāgaṁ[with.lust] cittan’ti[mind] pajānāti,[know clearly] vītarāgaṁ[passionless] vā[or] cittaṁ[mind] ‘vītarāgaṁ[passionless] cittan’ti[mind] pajānāti,[know clearly]
\endgl
\nopagebreak
\linespread{0.5}
\begin{spacin}{0.2}
{\PaliGlossFT One understands a mind affected by lust as affected by lust and a mind unaffected by lust as unaffected by lust;}
\end{spacin}
\vskip 12pt
\end{samepage}
\begin{samepage}
\begingl[glneveryline={\PaliGlossA,\PaliGlossB}]
sadosaṁ[with.hate] vā[or] cittaṁ[mind] ‘sadosaṁ[with.hate] cittan’ti[mind] pajānāti,[know clearly] vītadosaṁ[without.hate] vā[or] cittaṁ[mind] ‘vītadosaṁ[without.hate] cittan’ti[mind] pajānāti,[know clearly]
\endgl
\nopagebreak
\linespread{0.5}
\begin{spacin}{0.2}
{\PaliGlossFT one understands a mind affected by hate as affected by hate and a mind unaffected by hate as unaffected by hate;}
\end{spacin}
\vskip 12pt
\end{samepage}
\begin{samepage}
\begingl[glneveryline={\PaliGlossA,\PaliGlossB}]
samohaṁ[with.delusion] vā[or] cittaṁ[mind] ‘samohaṁ[with.delusion] cittan’ti[mind] pajānāti,[know clearly] vītamohaṁ[without.delusion] vā[or] cittaṁ[mind] ‘vītamohaṁ[without.delusion] cittan’ti[mind] pajānāti,[know clearly]
\endgl
\nopagebreak
\linespread{0.5}
\begin{spacin}{0.2}
{\PaliGlossFT one understands a mind affected by delusion as affected by delusion and a mind unaffected by delusion as unaffected by delusion;}
\end{spacin}
\vskip 12pt
\end{samepage}
\begin{samepage}
\begingl[glneveryline={\PaliGlossA,\PaliGlossB}]
saṁkhittaṁ[contracted] vā[or] cittaṁ[mind] ‘saṁkhittaṁ[contracted] cittan’ti[mind] pajānāti,[know clearly] vikkhittaṁ[distracted] vā[or] cittaṁ[mind] ‘vikkhittaṁ[distracted] cittan’ti[mind] pajānāti,[know clearly]
\endgl
\nopagebreak
\linespread{0.5}
\begin{spacin}{0.2}
{\PaliGlossFT one understands a contracted mind as contracted and a distracted mind as distracted;}
\end{spacin}
\vskip 12pt
\end{samepage}
\begin{samepage}
\begingl[glneveryline={\PaliGlossA,\PaliGlossB}]
mahaggataṁ[exalted] vā[or] cittaṁ[mind] ‘mahaggataṁ[exalted] cittan’ti[mind] pajānāti,[know clearly] amahaggataṁ[unexalted] vā[or] cittaṁ[mind] ‘amahaggataṁ[unexalted] cittan’ti[mind] pajānāti,[know clearly]
\endgl
\nopagebreak
\linespread{0.5}
\begin{spacin}{0.2}
{\PaliGlossFT one understands an exalted mind as exalted and an unexalted mind as unexalted;}
\end{spacin}
\vskip 12pt
\end{samepage}
\begin{samepage}
\begingl[glneveryline={\PaliGlossA,\PaliGlossB}]
sauttaraṁ[surpassed] vā[or] cittaṁ[mind] ‘sauttaraṁ[surpassed] cittan’ti[mind] pajānāti,[know clearly] anuttaraṁ[unsurpassed] vā[or] cittaṁ[mind] ‘anuttaraṁ[unsurpassed] cittan’ti[mind] pajānāti,[know clearly]
\endgl
\nopagebreak
\linespread{0.5}
\begin{spacin}{0.2}
{\PaliGlossFT one understands a surpassed mind as surpassed and an unsurpassed mind as unsurpassed;}
\end{spacin}
\vskip 12pt
\end{samepage}
\begin{samepage}
\begingl[glneveryline={\PaliGlossA,\PaliGlossB}]
samāhitaṁ[collected] vā[or] cittaṁ[mind] ‘samāhitaṁ[collected] cittan’ti[mind] pajānāti,[know clearly] asamāhitaṁ[uncollected] vā[or] cittaṁ[mind] ‘asamāhitaṁ[uncollected] cittan’ti[mind] pajānāti,[know clearly]
\endgl
\nopagebreak
\linespread{0.5}
\begin{spacin}{0.2}
{\PaliGlossFT one understands a concentrated mind as concentrated and an unconcentrated mind as unconcentrated;}
\end{spacin}
\vskip 12pt
\end{samepage}
\begin{samepage}
\begingl[glneveryline={\PaliGlossA,\PaliGlossB}]
vimuttaṁ[released] vā[or] cittaṁ[mind] ‘vimuttaṁ[released] cittan’ti[mind] pajānāti,[know clearly] avimuttaṁ[unreleased] vā[or] cittaṁ[mind] ‘avimuttaṁ[unreleased] cittan’ti[mind] pajānāti.[know clearly]
\endgl
\nopagebreak
\linespread{0.5}
\begin{spacin}{0.2}
{\PaliGlossFT one understands a liberated mind as liberated and an unliberated mind as unliberated.}
\end{spacin}
\vskip 12pt
\end{samepage}
\begin{samepage}
\begingl[glneveryline={\PaliGlossA,\PaliGlossB}]
(viii)[-] “so[he] anekavihitaṁ[many.prepared] pubbenivāsaṁ[former.abode] anussarati,[remembers] seyyathidaṁ[such as] —[-] ekampi[one] jātiṁ[birth ] dvepi[two] jātiyo[birth] tissopi[three] jātiyo[birth] catassopi[four] jātiyo[birth] pañcapi[five] jātiyo[birth] dasapi[ten] jātiyo[birth] vīsampi[twenty] jātiyo[birth] tiṁsampi[thirty] jātiyo[birth] cattārīsampi[fourty] jātiyo[birth] paññāsampi[fifty] jātiyo[birth] jātisatampi[birth.hundred] jātisahassampi[birth.thousand] jātisatasahassampi[birth.hundred.thousand]
\endgl
\nopagebreak
\linespread{0.5}
\begin{spacin}{0.2}
{\PaliGlossFT (viii) “One recollects ones manifold past lives, that is, one birth, two births, three births, four births, five births, ten births, twenty births, thirty births, forty births, fifty births, a hundred births, a thousand births, a hundred thousand births,}
\end{spacin}
\vskip 12pt
\end{samepage}
\begin{samepage}
\begingl[glneveryline={\PaliGlossA,\PaliGlossB}]
anekepi[many] saṁvaṭṭakappe[contract.cycle.aeon] anekepi[many] vivaṭṭakappe[expand.cycle.aeon] anekepi[many] saṁvaṭṭavivaṭṭakappe;[worldcontraction.worldexpansion]
\endgl
\nopagebreak
\linespread{0.5}
\begin{spacin}{0.2}
{\PaliGlossFT many aeons of world-contraction, many aeons of world-expansion, many aeons of world-contraction and expansion:}
\end{spacin}
\vskip 12pt
\end{samepage}
\begin{samepage}
\begingl[glneveryline={\PaliGlossA,\PaliGlossB}]
‘amutrāsiṁ[such place.i was] evaṁnāmo[such.name] evaṁgotto[such.clan] evaṁvaṇṇo[such.appearance] evamāhāro[such.nutriment] evaṁsukhadukkhappaṭisaṁvedī[such.pleasure.pain.experience] evamāyupariyanto,[such.life.term]
\endgl
\nopagebreak
\linespread{0.5}
\begin{spacin}{0.2}
{\PaliGlossFT ‘There I was so named, of such a clan, with such an appearance, such was my nutriment, such my experience of pleasure and pain, such my life-term;}
\end{spacin}
\vskip 12pt
\end{samepage}
\begin{samepage}
\begingl[glneveryline={\PaliGlossA,\PaliGlossB}]
so[he] tato[from there] cuto[passing away] amutra[such place] udapādiṁ;[arose]
\endgl
\nopagebreak
\linespread{0.5}
\begin{spacin}{0.2}
{\PaliGlossFT and passing away from there, I reappeared elsewhere;}
\end{spacin}
\vskip 12pt
\end{samepage}
\begin{samepage}
\begingl[glneveryline={\PaliGlossA,\PaliGlossB}]
tatrāpāsiṁ[there.I was] evaṁnāmo[such.name] evaṁgotto[such.clan] evaṁvaṇṇo[such.appearance] evamāhāro[such.nutriment] evaṁsukhadukkhappaṭisaṁvedī[such.pleasure.pain.experience] evamāyupariyanto,[such.life.term]
\endgl
\nopagebreak
\linespread{0.5}
\begin{spacin}{0.2}
{\PaliGlossFT and there too I was so named, of such a clan, with such an appearance, such was my nutriment, such my experience of pleasure and pain, such my life-term;}
\end{spacin}
\vskip 12pt
\end{samepage}
\begin{samepage}
\begingl[glneveryline={\PaliGlossA,\PaliGlossB}]
so[he] tato[from there] cuto[passing away] idhūpapanno’ti.[here.reappear]
\endgl
\nopagebreak
\linespread{0.5}
\begin{spacin}{0.2}
{\PaliGlossFT and passing away from there, I reappeared here.’}
\end{spacin}
\vskip 12pt
\end{samepage}
\begin{samepage}
\begingl[glneveryline={\PaliGlossA,\PaliGlossB}]
iti[thus] sākāraṁ[characteristics] sauddesaṁ[explanation] anekavihitaṁ[many.prepared] pubbenivāsaṁ[former.abode] anussarati.[remembers]
\endgl
\nopagebreak
\linespread{0.5}
\begin{spacin}{0.2}
{\PaliGlossFT Thus with their aspects and particulars one recollects ones manifold past lives.}
\end{spacin}
\vskip 12pt
\end{samepage}
\begin{samepage}
\begingl[glneveryline={\PaliGlossA,\PaliGlossB}]
(ix)[-] “dibbena[divine] cakkhunā[eye] visuddhena[purified] atikkantamānusakena[beyond.human] satte[being] passati[sees] cavamāne[pass away] upapajjamāne[reappear] hīne[inferior] paṇīte[superior] suvaṇṇe[beautiful] dubbaṇṇe,[ugly] sugate[fortuanate] duggate[unfortuanate] yathākammūpage[accord to.action.going to] satte[being] pajānāti.[know clearly]
\endgl
\nopagebreak
\linespread{0.5}
\begin{spacin}{0.2}
{\PaliGlossFT (ix) “With the divine eye, which is purified and surpasses the human, one sees beings passing away and reappearing, inferior and superior, fair and ugly, fortunate and unfortunate, and one understands how beings pass on according to their actions.}
\end{spacin}
\vskip 12pt
\end{samepage}
\begin{samepage}
\begingl[glneveryline={\PaliGlossA,\PaliGlossB}]
(x)[-] “āsavānaṁ[taints] khayā[destruction] anāsavaṁ[free.taints] cetovimuttiṁ[mind.liberated] paññāvimuttiṁ[wisdom.liberated] diṭṭheva[vision] dhamme[the Norm] sayaṁ[by oneself] abhiññā[knowing] sacchikatvā[have realized] upasampajja[have attained] viharati.[dwells]
\endgl
\nopagebreak
\linespread{0.5}
\begin{spacin}{0.2}
{\PaliGlossFT (x) “By realising for oneself with direct knowledge, one here and now enters upon and abides in the deliverance of mind and deliverance by wisdom that are taintless with the destruction of the taints.}
\end{spacin}
\vskip 12pt
\end{samepage}
\begin{samepage}
\begingl[glneveryline={\PaliGlossA,\PaliGlossB}]
“kāyagatāya,[body.directed] bhikkhave,[-] satiyā[mindful] āsevitāya[frequent.pract] bhāvitāya[develope] bahulīkatāya[cultivate] yānīkatāya[made a habit] vatthukatāya[made basis] anuṭṭhitāya[establish] paricitāya[accumulate] susamāraddhāya[well undertaken] ime[this] dasānisaṁsā[ten.benefit] pāṭikaṅkhā”ti.[expect]
\endgl
\nopagebreak
\linespread{0.5}
\begin{spacin}{0.2}
{\PaliGlossFT “Bhikkhus, when mindfulness of the body has been repeatedly practised, developed, cultivated, used as a vehicle, used as a basis, established, consolidated, and well undertaken, these ten benefits may be expected.”}
\end{spacin}
\vskip 12pt
\end{samepage}
\begin{samepage}
\begingl[glneveryline={\PaliGlossA,\PaliGlossB}]
idamavoca[this.he said] bhagavā.[blessed] attamanā[delighted] te[his] bhikkhū[-] bhagavato[fortunate] bhāsitaṁ[said] abhinandunti.[rejoiced at]
\endgl
\nopagebreak
\linespread{0.5}
\begin{spacin}{0.2}
{\PaliGlossFT That is what the Blessed One said. The bhikkhus were satisfied and delighted in the Blessed One’s words.}
\end{spacin}
\vskip 12pt
\end{samepage}
\begin{samepage}
\begingl[glneveryline={\PaliGlossA,\PaliGlossB}]
kāyagatāsatisuttaṁ[body.related.mindful] niṭṭhitaṁ[finished] navamaṁ.[ninth]
\endgl
\nopagebreak
\linespread{0.5}
\begin{spacin}{0.2}
{\PaliGlossFT Contemplation of the body, concludes, 11(9)}
\end{spacin}
\vskip 12pt
\end{samepage}