\begin{samepage}
\ensurevspace{4\baselineskip}
\begin{leftcolumn*}
Collection of Middle Length Discourses, Shorter Chapter on the Pairs
The Lesser Series of Questions-and-Answers
\end{leftcolumn*}

\begin{rightcolumn}
Majjhima Nikāya, mūlapaṇṇāsapāḷi, 5. cūḷayamakavaggo,
4. cūḷavedallasuttaṃ (MN 44)
\end{rightcolumn}
\end{samepage}

\begin{samepage}
\ensurevspace{4\baselineskip}
\begin{leftcolumn*}
Thus I heard: At one time the Blessed One was living near Rājagaha at the Squirrel’s Feeding Place in Bamboo Wood.
\end{leftcolumn*}

\begin{rightcolumn}
evaṃ me sutaṃ — ekaṃ samayaṃ bhagavā rājagahe viharati veḷuvane kalandakanivāpe.
\end{rightcolumn}
\end{samepage}

\begin{samepage}
\ensurevspace{4\baselineskip}
\begin{leftcolumn*}
The devotee Visākha approached the nun Dhammadinnā, and after approaching and worshipping the nun Dhammadinnā, he sat on one side. While sitting on one side the devotee Visākha said this to the nun Dhammadinnā:
\end{leftcolumn*}

\begin{rightcolumn}
atha kho visākho upāsako yena dhammadinnā bhikkhunī tenupasaṅkami; upasaṅkamitvā dhammadinnaṃ bhikkhuniṃ abhivādetvā ekamantaṃ nisīdi. ekamantaṃ nisinno kho visākho upāsako dhammadinnaṃ bhikkhuniṃ etadavoca:
\end{rightcolumn}
\end{samepage}

\begin{samepage}
\ensurevspace{4\baselineskip}
\begin{leftcolumn*}
-
\end{leftcolumn*}

\begin{rightcolumn}
-
\end{rightcolumn}
\end{samepage}

\begin{samepage}
\ensurevspace{4\baselineskip}
\begin{leftcolumn*}
“ ‘Embodiment, embodiment,’ is said, Noble Lady. What, Noble Lady, is said to be embodiment by the Blessed One?”
\end{leftcolumn*}

\begin{rightcolumn}
“‘sakkāyo sakkāyo’ti, ayye, vuccati. katamo nu kho, ayye, sakkāyo vutto bhagavatā”ti?
\end{rightcolumn}
\end{samepage}

\begin{samepage}
\ensurevspace{4\baselineskip}
\begin{leftcolumn*}
“These five aggregates affected by clinging, friend Visākha, are said to be embodiment by the Blessed One, as follows:
\end{leftcolumn*}

\begin{rightcolumn}
“pañca kho ime, āvuso visākha, upādānakkhandhā sakkāyo vutto bhagavatā, seyyathidaṃ —
\end{rightcolumn}
\end{samepage}

\begin{samepage}
\ensurevspace{4\baselineskip}
\begin{leftcolumn*}
the form aggregate affected by clinging, the feelings aggregate affected by clinging, the perceptions aggregate affected by clinging, the (mental) formations aggregate affected by clinging, the consciousness aggregate affected by clinging.
\end{leftcolumn*}

\begin{rightcolumn}
rūpupādānakkhandho, vedanupādānakkhandho, saññupādānakkhandho, saṅkhārupādānakkhandho, viññāṇupādānakkhandho.
\end{rightcolumn}
\end{samepage}

\begin{samepage}
\ensurevspace{4\baselineskip}
\begin{leftcolumn*}
These five aggregates affected by clinging, friend Visākha, are said to be embodiment by the Blessed One.”
\end{leftcolumn*}

\begin{rightcolumn}
ime kho, āvuso visākha, pañcupādānakkhandhā sakkāyo vutto bhagavatā”ti.
\end{rightcolumn}
\end{samepage}

\begin{samepage}
\ensurevspace{4\baselineskip}
\begin{leftcolumn*}
-
\end{leftcolumn*}

\begin{rightcolumn}
-
\end{rightcolumn}
\end{samepage}

\begin{samepage}
\ensurevspace{4\baselineskip}
\begin{leftcolumn*}
“Well said, Noble Lady,” said the devotee Visākha, and after greatly rejoicing and gladly receiving this word of the nun Dhammadinnā, he asked a further question to the nun Dhammadinnā:
\end{leftcolumn*}

\begin{rightcolumn}
“sādhayye”ti kho visākho upāsako dhammadinnāya bhikkhuniyā bhāsitaṃ abhinanditvā anumoditvā dhammadinnaṃ bhikkhuniṃ uttariṃ pañhaṃ apucchi,
\end{rightcolumn}
\end{samepage}

\begin{samepage}
\ensurevspace{4\baselineskip}
\begin{leftcolumn*}
“ ‘The arising of embodiment, the arising of embodiment,’ is said, Noble Lady. What, Noble Lady, is said to be the arising of embodiment by the Blessed One?”
\end{leftcolumn*}

\begin{rightcolumn}
“‘sakkāyasamudayo sakkāyasamudayo’ti, ayye, vuccati. katamo nu kho, ayye, sakkāyasamudayo vutto bhagavatā”ti?
\end{rightcolumn}
\end{samepage}

\begin{samepage}
\ensurevspace{4\baselineskip}
\begin{leftcolumn*}
“It is that craving which leads to continuation in existence, friend Visākha, which is connected with enjoyment and passion, greatly enjoying this and that, as follows: craving for sense pleasures craving for continuation craving for discontinuation.
\end{leftcolumn*}

\begin{rightcolumn}
“yāyaṃ, āvuso visākha, taṇhā ponobbhavikā nandīrāgasahagatā tatratatrābhinandinī, seyyathidaṃ — kāmataṇhā bhavataṇhā vibhavataṇhā;
\end{rightcolumn}
\end{samepage}

\begin{samepage}
\ensurevspace{4\baselineskip}
\begin{leftcolumn*}
This, friend Visākha, is said to be the arising of embodiment by the Blessed One.
\end{leftcolumn*}

\begin{rightcolumn}
ayaṃ kho, āvuso visākha, sakkāyasamudayo vutto bhagavatā”ti.
\end{rightcolumn}
\end{samepage}

\begin{samepage}
\ensurevspace{4\baselineskip}
\begin{leftcolumn*}
-
\end{leftcolumn*}

\begin{rightcolumn}
-
\end{rightcolumn}
\end{samepage}

\begin{samepage}
\ensurevspace{4\baselineskip}
\begin{leftcolumn*}
“ ‘The cessation of embodiment, the cessation of embodiment,’ is said, Noble Lady. What, Noble Lady, is said to be the cessation of embodiment by the Blessed One?”
\end{leftcolumn*}

\begin{rightcolumn}
“‘sakkāyanirodho sakkāyanirodho’ti, ayye, vuccati. katamo nu kho, ayye, sakkāyanirodho vutto bhagavatā”ti?
\end{rightcolumn}
\end{samepage}

\begin{samepage}
\ensurevspace{4\baselineskip}
\begin{leftcolumn*}
“It is the complete fading away and cessation without remainder of that craving, friend Visākha, liberation, letting go, release and non-adherence.
\end{leftcolumn*}

\begin{rightcolumn}
“yo kho, āvuso visākha, tassāyeva taṇhāya asesavirāganirodho cāgo paṭinissaggo mutti anālayo;
\end{rightcolumn}
\end{samepage}

\begin{samepage}
\ensurevspace{4\baselineskip}
\begin{leftcolumn*}
This, friend Visākha, is said to be the cessation of embodiment by the Blessed One.”
\end{leftcolumn*}

\begin{rightcolumn}
ayaṃ kho, āvuso visākha, sakkāyanirodho vutto bhagavatā”ti.
\end{rightcolumn}
\end{samepage}

\begin{samepage}
\ensurevspace{4\baselineskip}
\begin{leftcolumn*}
-
\end{leftcolumn*}

\begin{rightcolumn}
-
\end{rightcolumn}
\end{samepage}

\begin{samepage}
\ensurevspace{4\baselineskip}
\begin{leftcolumn*}
“ ‘The path leading to the cessation of embodiment, the path leading to the cessation of embodiment,’ is said, Noble Lady. What, Noble Lady, is said to be the path leading to the cessation of embodiment by the Blessed One?”
\end{leftcolumn*}

\begin{rightcolumn}
“‘sakkāyanirodhagāminī paṭipadā sakkāyanirodhagāminī paṭipadā’ti, ayye, vuccati. katamā nu kho, ayye, sakkāyanirodhagāminī paṭipadā vuttā bhagavatā”ti?
\end{rightcolumn}
\end{samepage}

\begin{samepage}
\ensurevspace{4\baselineskip}
\begin{leftcolumn*}
“It is this noble path with eight factors, friend Visākha, as follows: right view, right thought, right speech, right action, right livelihood, right endeavour, right mindfulness, right concentration.”
\end{leftcolumn*}

\begin{rightcolumn}
“ayameva kho, āvuso visākha, ariyo aṭṭhaṅgiko maggo sakkāyanirodhagāminī paṭipadā vuttā bhagavatā, seyyathidaṃ — sammādiṭṭhi sammāsaṅkappo sammāvācā sammākammanto sammāājīvo sammāvāyāmo sammāsati sammāsamādhī”ti.
\end{rightcolumn}
\end{samepage}

\begin{samepage}
\ensurevspace{4\baselineskip}
\begin{leftcolumn*}
-
\end{leftcolumn*}

\begin{rightcolumn}
-
\end{rightcolumn}
\end{samepage}

\begin{samepage}
\ensurevspace{4\baselineskip}
\begin{leftcolumn*}
“Is this clinging, Noble Lady, (the same as) the five aggregates affected by clinging, or is clinging different from the five aggregates affected by clinging”
\end{leftcolumn*}

\begin{rightcolumn}
“taññeva nu kho, ayye, upādānaṃ te pañcupādānakkhandhā udāhu aññatra pañcahupādānakkhandhehi upādānan”ti?
\end{rightcolumn}
\end{samepage}

\begin{samepage}
\ensurevspace{4\baselineskip}
\begin{leftcolumn*}
“This clinging, friend Visākha, is not (the same as) the five aggregates affected by clinging, nor is clinging different from the five aggregates affected by clinging. But whatever desire and passion there is for the five aggregates affected by clinging, that is the clinging right there.”
\end{leftcolumn*}

\begin{rightcolumn}
“na kho, āvuso visākha, taññeva upādānaṃ te pañcupādānakkhandhā, nāpi aññatra pañcahupādānakkhandhehi upādānaṃ. yo kho, āvuso visākha, pañcasu upādānakkhandhesu chandarāgo taṃ tattha upādānan”ti.
\end{rightcolumn}
\end{samepage}

\begin{samepage}
\ensurevspace{4\baselineskip}
\begin{leftcolumn*}
-
\end{leftcolumn*}

\begin{rightcolumn}
-
\end{rightcolumn}
\end{samepage}

\begin{samepage}
\ensurevspace{4\baselineskip}
\begin{leftcolumn*}
“But what, Noble Lady, is embodiment view?”
\end{leftcolumn*}

\begin{rightcolumn}
“kathaṃ panāyye, sakkāyadiṭṭhi hotī”ti?
\end{rightcolumn}
\end{samepage}

\begin{samepage}
\ensurevspace{4\baselineskip}
\begin{leftcolumn*}
“Here, friend Visākha, an unlearned worldling, one who doesn’t meet the Noble Ones, who is unskilled in the Noble Dhamma, untrained in the Noble Dhamma, one who doesn’t meet Good People, who is unskilled in the Good People’s Dhamma, untrained in the Good People’s Dhamma,
\end{leftcolumn*}

\begin{rightcolumn}
“idhāvuso visākha, assutavā puthujjano, ariyānaṃ adassāvī ariyadhammassa akovido ariyadhamme avinīto, sappurisānaṃ adassāvī sappurisadhammassa akovido sappurisadhamme avinīto,
\end{rightcolumn}
\end{samepage}

\begin{samepage}
\ensurevspace{4\baselineskip}
\begin{leftcolumn*}
views bodily form as self, or self as endowed with bodily form, or bodily form as in self, or self as in bodily form.
\end{leftcolumn*}

\begin{rightcolumn}
rūpaṃ attato samanupassati, rūpavantaṃ vā attānaṃ, attani vā rūpaṃ, rūpasmiṃ vā attānaṃ.
\end{rightcolumn}
\end{samepage}

\begin{samepage}
\ensurevspace{4\baselineskip}
\begin{leftcolumn*}
Views feeling as self, or self as endowed with feeling, or feeling as in self, or self as in feeling.
\end{leftcolumn*}

\begin{rightcolumn}
vedanaṃ attato samanupassati, vedanavantaṃ vā attānaṃ, attani vā vedanaṃ, vedanasmiṃ vā attānaṃ.
\end{rightcolumn}
\end{samepage}

\begin{samepage}
\ensurevspace{4\baselineskip}
\begin{leftcolumn*}
Views perception as self, or self as endowed with perception, or perception as in self, or self as in perception.
\end{leftcolumn*}

\begin{rightcolumn}
saññaṃ attato samanupassati, saññavantaṃ vā attānaṃ, attani vā saññaṃ, saññasmiṃ vā attānaṃ.
\end{rightcolumn}
\end{samepage}

\begin{samepage}
\ensurevspace{4\baselineskip}
\begin{leftcolumn*}
Views (volitional) formations as self, or self as endowed with (volitional) formations, or (volitional) formations as in self, or self as in (volitional) formations.
\end{leftcolumn*}

\begin{rightcolumn}
saṅkhāre attato samanupassati, saṅkhāravantaṃ vā attānaṃ, attani vā saṅkhāre, saṅkhārasmiṃ vā attānaṃ.
\end{rightcolumn}
\end{samepage}

\begin{samepage}
\ensurevspace{4\baselineskip}
\begin{leftcolumn*}
Views consciousness as self, or self as endowed with consciousness, or consciousness as in self, or self as in consciousness.
\end{leftcolumn*}

\begin{rightcolumn}
viññāṇaṃ attato samanupassati, viññāṇavantaṃ vā attānaṃ, attani vā viññāṇaṃ, viññāṇasmiṃ vā attānaṃ.
\end{rightcolumn}
\end{samepage}

\begin{samepage}
\ensurevspace{4\baselineskip}
\begin{leftcolumn*}
This, friend Visākha, is embodiment view.”
\end{leftcolumn*}

\begin{rightcolumn}
evaṃ kho, āvuso visākha, sakkāyadiṭṭhi hotī”ti.
\end{rightcolumn}
\end{samepage}

\begin{samepage}
\ensurevspace{4\baselineskip}
\begin{leftcolumn*}
-
\end{leftcolumn*}

\begin{rightcolumn}
-
\end{rightcolumn}
\end{samepage}

\begin{samepage}
\ensurevspace{4\baselineskip}
\begin{leftcolumn*}
“But how, Noble Lady, is there no embodiment view?”
\end{leftcolumn*}

\begin{rightcolumn}
“kathaṃ panāyye, sakkāyadiṭṭhi na hotī”ti?
\end{rightcolumn}
\end{samepage}

\begin{samepage}
\ensurevspace{4\baselineskip}
\begin{leftcolumn*}
“Here, friend Visākha, a learned noble disciple, one who meets the Noble Ones, who is skilled in the Noble Dhamma, trained in the Noble Dhamma, one who meets Good People, who is skilled in the Good People’s Dhamma, trained in the Good People’s Dhamma,
\end{leftcolumn*}

\begin{rightcolumn}
“idhāvuso visākha, sutavā ariyasāvako, ariyānaṃ dassāvī ariyadhammassa kovido ariyadhamme suvinīto, sappurisānaṃ dassāvī sappurisadhammassa kovido sappurisadhamme suvinīto,
\end{rightcolumn}
\end{samepage}

\begin{samepage}
\ensurevspace{4\baselineskip}
\begin{leftcolumn*}
doesn’t view bodily form as self, or self as endowed with bodily form, or bodily form as in self, or self as in bodily form.
\end{leftcolumn*}

\begin{rightcolumn}
na rūpaṃ attato samanupassati, na rūpavantaṃ vā attānaṃ, na attani vā rūpaṃ, na rūpasmiṃ vā attānaṃ.
\end{rightcolumn}
\end{samepage}

\begin{samepage}
\ensurevspace{4\baselineskip}
\begin{leftcolumn*}
Doesn’t view feeling as self, or self as endowed with feeling, or feeling as in self, or self as in feeling.
\end{leftcolumn*}

\begin{rightcolumn}
na vedanaṃ attato samanupassati, na vedanavantaṃ vā attānaṃ, na attani vā vedanaṃ, na vedanasmiṃ vā attānaṃ.
\end{rightcolumn}
\end{samepage}

\begin{samepage}
\ensurevspace{4\baselineskip}
\begin{leftcolumn*}
Doesn’t view perception as self, or self as endowed with perception, or perception as in self, or self as in perception.
\end{leftcolumn*}

\begin{rightcolumn}
na saññaṃ attato samanupassati, na saññavantaṃ vā attānaṃ, na attani vā saññaṃ, na saññasmiṃ vā attānaṃ.
\end{rightcolumn}
\end{samepage}

\begin{samepage}
\ensurevspace{4\baselineskip}
\begin{leftcolumn*}
Doesn’t view (volitional) formations as self, or self as endowed with (volitional) formations, or (volitional) formations as in self, or self as in (volitional) formations.
\end{leftcolumn*}

\begin{rightcolumn}
na saṅkhāre attato samanupassati, na saṅkhāravantaṃ vā attānaṃ, na attani vā saṅkhāre, na saṅkhārasmiṃ vā attānaṃ.
\end{rightcolumn}
\end{samepage}

\begin{samepage}
\ensurevspace{4\baselineskip}
\begin{leftcolumn*}
Doesn’t view consciousness as self, or self as endowed with consciousness, or consciousness as in self, or self as in consciousness.
\end{leftcolumn*}

\begin{rightcolumn}
na viññāṇaṃ attato samanupassati, na viññāṇavantaṃ vā attānaṃ, na attani vā viññāṇaṃ, na viññāṇasmiṃ vā attānaṃ.
\end{rightcolumn}
\end{samepage}

\begin{samepage}
\ensurevspace{4\baselineskip}
\begin{leftcolumn*}
Thus, friend Visākha, there is no embodiment view.”
\end{leftcolumn*}

\begin{rightcolumn}
evaṃ kho, āvuso visākha, sakkāyadiṭṭhi na hotī”ti.
\end{rightcolumn}
\end{samepage}

\begin{samepage}
\ensurevspace{4\baselineskip}
\begin{leftcolumn*}
-
\end{leftcolumn*}

\begin{rightcolumn}
-
\end{rightcolumn}
\end{samepage}

\begin{samepage}
\ensurevspace{4\baselineskip}
\begin{leftcolumn*}
“But what, Noble Lady, is the Eightfold Noble Path?”
\end{leftcolumn*}

\begin{rightcolumn}
“katamo panāyye, ariyo aṭṭhaṅgiko maggo”ti?
\end{rightcolumn}
\end{samepage}

\begin{samepage}
\ensurevspace{4\baselineskip}
\begin{leftcolumn*}
“It is this noble path with eight factors, friend Visākha, as follows:
\end{leftcolumn*}

\begin{rightcolumn}
“ayameva kho, āvuso visākha, ariyo aṭṭhaṅgiko maggo, seyyathidaṃ —
\end{rightcolumn}
\end{samepage}

\begin{samepage}
\ensurevspace{4\baselineskip}
\begin{leftcolumn*}
right view, right thought, right speech, right action, right livelihood, right endeavour, right mindfulness, right concentration.”
\end{leftcolumn*}

\begin{rightcolumn}
sammādiṭṭhi sammāsaṅkappo sammāvācā sammākammanto sammāājīvo sammāvāyāmo sammāsati sammāsamādhī”ti.
\end{rightcolumn}
\end{samepage}

\begin{samepage}
\ensurevspace{4\baselineskip}
\begin{leftcolumn*}
-
\end{leftcolumn*}

\begin{rightcolumn}
-
\end{rightcolumn}
\end{samepage}

\begin{samepage}
\ensurevspace{4\baselineskip}
\begin{leftcolumn*}
“But is the eightfold Noble Path, Noble Lady, conditioned or unconditioned?”
\end{leftcolumn*}

\begin{rightcolumn}
“ariyo panāyye, aṭṭhaṅgiko maggo saṅkhato udāhu asaṅkhato”ti?
\end{rightcolumn}
\end{samepage}

\begin{samepage}
\ensurevspace{4\baselineskip}
\begin{leftcolumn*}
“The eightfold Noble Path, friend Visākha, is conditioned.”
\end{leftcolumn*}

\begin{rightcolumn}
“ariyo kho, āvuso visākha, aṭṭhaṅgiko maggo saṅkhato”ti.
\end{rightcolumn}
\end{samepage}

\begin{samepage}
\ensurevspace{4\baselineskip}
\begin{leftcolumn*}
-
\end{leftcolumn*}

\begin{rightcolumn}
-
\end{rightcolumn}
\end{samepage}

\begin{samepage}
\ensurevspace{4\baselineskip}
\begin{leftcolumn*}
“Are the three constituents comprised within the eightfold Noble Path, Noble Lady, or is the eightfold Noble Path comprised within the three constituents?”
\end{leftcolumn*}

\begin{rightcolumn}
“ariyena nu kho, ayye, aṭṭhaṅgikena maggena tayo khandhā saṅgahitā udāhu tīhi khandhehi ariyo aṭṭhaṅgiko maggo saṅgahito”ti?
\end{rightcolumn}
\end{samepage}

\begin{samepage}
\ensurevspace{4\baselineskip}
\begin{leftcolumn*}
“The three constituents are not comprised within the eightfold Noble Path, friend Visākha, but the eightfold Noble Path is comprised within the three constituents.
\end{leftcolumn*}

\begin{rightcolumn}
“na kho, āvuso visākha, ariyena aṭṭhaṅgikena maggena tayo khandhā saṅgahitā; tīhi ca kho, āvuso visākha, khandhehi ariyo aṭṭhaṅgiko maggo saṅgahito.
\end{rightcolumn}
\end{samepage}

\begin{samepage}
\ensurevspace{4\baselineskip}
\begin{leftcolumn*}
Whatever is right speech, friend Visākha, and whatever is right action, and whatever is right livelihood, these things are comprised within the virtue constituent.
\end{leftcolumn*}

\begin{rightcolumn}
yā cāvuso visākha, sammāvācā yo ca sammākammanto yo ca sammāājīvo ime dhammā sīlakkhandhe saṅgahitā.
\end{rightcolumn}
\end{samepage}

\begin{samepage}
\ensurevspace{4\baselineskip}
\begin{leftcolumn*}
Whatever is right endeavour, and whatever is right mindfulness, and whatever is right concentration, these things are comprised within the concentration constituent.
\end{leftcolumn*}

\begin{rightcolumn}
yo ca sammāvāyāmo yā ca sammāsati yo ca sammāsamādhi ime dhammā samādhikkhandhe saṅgahitā.
\end{rightcolumn}
\end{samepage}

\begin{samepage}
\ensurevspace{4\baselineskip}
\begin{leftcolumn*}
Whatever is right view, and whatever is right thought, these things are comprised within the wisdom constituent.
\end{leftcolumn*}

\begin{rightcolumn}
yā ca sammādiṭṭhi yo ca sammāsaṅkappo, ime dhammā paññākkhandhe saṅgahitā”ti.
\end{rightcolumn}
\end{samepage}

\begin{samepage}
\ensurevspace{4\baselineskip}
\begin{leftcolumn*}
-
\end{leftcolumn*}

\begin{rightcolumn}
-
\end{rightcolumn}
\end{samepage}

\begin{samepage}
\ensurevspace{4\baselineskip}
\begin{leftcolumn*}
“But what, Noble Lady, is concentration,
\end{leftcolumn*}

\begin{rightcolumn}
“katamo panāyye, samādhi,
\end{rightcolumn}
\end{samepage}

\begin{samepage}
\ensurevspace{4\baselineskip}
\begin{leftcolumn*}
what are the causes of concentration,
\end{leftcolumn*}

\begin{rightcolumn}
katame dhammā samādhinimittā,
\end{rightcolumn}
\end{samepage}

\begin{samepage}
\ensurevspace{4\baselineskip}
\begin{leftcolumn*}
what are the accessories to concentration,
\end{leftcolumn*}

\begin{rightcolumn}
katame dhammā samādhiparikkhārā,
\end{rightcolumn}
\end{samepage}

\begin{samepage}
\ensurevspace{4\baselineskip}
\begin{leftcolumn*}
what is the development of concentration?”
\end{leftcolumn*}

\begin{rightcolumn}
katamā samādhibhāvanā”ti?
\end{rightcolumn}
\end{samepage}

\begin{samepage}
\ensurevspace{4\baselineskip}
\begin{leftcolumn*}
“Whatever is one-pointedness of mind, friend Visākha, that is concentration,
\end{leftcolumn*}

\begin{rightcolumn}
“yā kho, āvuso visākha, cittassa ekaggatā ayaṃ samādhi;
\end{rightcolumn}
\end{samepage}

\begin{samepage}
\ensurevspace{4\baselineskip}
\begin{leftcolumn*}
the four ways of attending to mindfulness are the causes of concentration,
\end{leftcolumn*}

\begin{rightcolumn}
cattāro satipaṭṭhānā samādhinimittā;
\end{rightcolumn}
\end{samepage}

\begin{samepage}
\ensurevspace{4\baselineskip}
\begin{leftcolumn*}
the four right endeavours are the accessories to concentration,
\end{leftcolumn*}

\begin{rightcolumn}
cattāro sammappadhānā samādhiparikkhārā.
\end{rightcolumn}
\end{samepage}

\begin{samepage}
\ensurevspace{4\baselineskip}
\begin{leftcolumn*}
whatever repetition of these things there is, their development, being made much of, this is the development of concentration herein.”
\end{leftcolumn*}

\begin{rightcolumn}
yā tesaṃyeva dhammānaṃ āsevanā bhāvanā bahulīkammaṃ, ayaṃ ettha samādhibhāvanā”ti.
\end{rightcolumn}
\end{samepage}

\begin{samepage}
\ensurevspace{4\baselineskip}
\begin{leftcolumn*}
-
\end{leftcolumn*}

\begin{rightcolumn}
-
\end{rightcolumn}
\end{samepage}

\begin{samepage}
\ensurevspace{4\baselineskip}
\begin{leftcolumn*}
“But what, Noble Lady, are the formations?”
\end{leftcolumn*}

\begin{rightcolumn}
“kati panāyye, saṅkhārā”ti?
\end{rightcolumn}
\end{samepage}

\begin{samepage}
\ensurevspace{4\baselineskip}
\begin{leftcolumn*}
“There are these three formations, friend Visākha: the bodily formation, the speech formation, the mental formation.”
\end{leftcolumn*}

\begin{rightcolumn}
“tayome, āvuso visākha, saṅkhārā — kāyasaṅkhāro, vacīsaṅkhāro, cittasaṅkhāro”ti.
\end{rightcolumn}
\end{samepage}

\begin{samepage}
\ensurevspace{4\baselineskip}
\begin{leftcolumn*}
-
\end{leftcolumn*}

\begin{rightcolumn}
-
\end{rightcolumn}
\end{samepage}

\begin{samepage}
\ensurevspace{4\baselineskip}
\begin{leftcolumn*}
“But what, Noble Lady, is bodily formation, what is speech formation, what is mental formation?”
\end{leftcolumn*}

\begin{rightcolumn}
“katamo panāyye, kāyasaṅkhāro, katamo vacīsaṅkhāro, katamo cittasaṅkhāro”ti?
\end{rightcolumn}
\end{samepage}

\begin{samepage}
\ensurevspace{4\baselineskip}
\begin{leftcolumn*}
“In-breathing and out-breathing, friend Visākha, is bodily formation, thinking and reflection is speech formation, perception and feeling is mental formation.”
\end{leftcolumn*}

\begin{rightcolumn}
“assāsapassāsā kho, āvuso visākha, kāyasaṅkhāro, vitakkavicārā vacīsaṅkhāro, saññā ca vedanā ca cittasaṅkhāro”ti.
\end{rightcolumn}
\end{samepage}

\begin{samepage}
\ensurevspace{4\baselineskip}
\begin{leftcolumn*}
-
\end{leftcolumn*}

\begin{rightcolumn}
-
\end{rightcolumn}
\end{samepage}

\begin{samepage}
\ensurevspace{4\baselineskip}
\begin{leftcolumn*}
“But why is in-breathing and out-breathing, Noble Lady, bodily formation, why is thinking and reflection speech formation, why is perception and feeling mental formation?”
\end{leftcolumn*}

\begin{rightcolumn}
“kasmā panāyye, assāsapassāsā kāyasaṅkhāro, kasmā vitakkavicārā vacīsaṅkhāro, kasmā saññā ca vedanā ca cittasaṅkhāro”ti?
\end{rightcolumn}
\end{samepage}

\begin{samepage}
\ensurevspace{4\baselineskip}
\begin{leftcolumn*}
“In-breathing and out-breathing, friend Visākha, are bodily, these things are bound up with the body, therefore in-breathing and out-breathing is a bodily formation.
\end{leftcolumn*}

\begin{rightcolumn}
“assāsapassāsā kho, āvuso visākha, kāyikā ete dhammā kāyappaṭibaddhā, tasmā assāsapassāsā kāyasaṅkhāro.
\end{rightcolumn}
\end{samepage}

\begin{samepage}
\ensurevspace{4\baselineskip}
\begin{leftcolumn*}
Having thought and reflected beforehand, friend Visākha, he afterwards breaks forth with a word, therefore thinking and reflection is a speech formation.
\end{leftcolumn*}

\begin{rightcolumn}
pubbe kho, āvuso visākha, vitakketvā vicāretvā pacchā vācaṃ bhindati, tasmā vitakkavicārā vacīsaṅkhāro.
\end{rightcolumn}
\end{samepage}

\begin{samepage}
\ensurevspace{4\baselineskip}
\begin{leftcolumn*}
Perception and feeling are mental factors, these things are bound up with the mind, therefore perception and feeling are mental formations.”
\end{leftcolumn*}

\begin{rightcolumn}
saññā ca vedanā ca cetasikā ete dhammā cittappaṭibaddhā, tasmā saññā ca vedanā ca cittasaṅkhāro”ti.
\end{rightcolumn}
\end{samepage}

\begin{samepage}
\ensurevspace{4\baselineskip}
\begin{leftcolumn*}
-
\end{leftcolumn*}

\begin{rightcolumn}
-
\end{rightcolumn}
\end{samepage}

\begin{samepage}
\ensurevspace{4\baselineskip}
\begin{leftcolumn*}
“But how, Noble Lady, is the cessation of perception and feeling attained?”
\end{leftcolumn*}

\begin{rightcolumn}
“kathaṃ panāyye, saññāvedayitanirodhasamāpatti hotī”ti?
\end{rightcolumn}
\end{samepage}

\begin{samepage}
\ensurevspace{4\baselineskip}
\begin{leftcolumn*}
“A monastic who is attaining the cessation of perception and feeling, friend Visākha, does not think:
\end{leftcolumn*}

\begin{rightcolumn}
“na kho, āvuso visākha, saññāvedayitanirodhaṃ samāpajjantassa bhikkhuno evaṃ hoti —
\end{rightcolumn}
\end{samepage}

\begin{samepage}
\ensurevspace{4\baselineskip}
\begin{leftcolumn*}
‘I will attain the cessation of perception and feeling,’ or
\end{leftcolumn*}

\begin{rightcolumn}
‘ahaṃ saññāvedayitanirodhaṃ samāpajjissan’ti vā,
\end{rightcolumn}
\end{samepage}

\begin{samepage}
\ensurevspace{4\baselineskip}
\begin{leftcolumn*}
‘I am attaining the cessation of perception and feeling,’ or
\end{leftcolumn*}

\begin{rightcolumn}
‘ahaṃ saññāvedayitanirodhaṃ samāpajjāmī’ti vā,
\end{rightcolumn}
\end{samepage}

\begin{samepage}
\ensurevspace{4\baselineskip}
\begin{leftcolumn*}
‘I have attained the cessation of perception and feeling.’
\end{leftcolumn*}

\begin{rightcolumn}
‘ahaṃ saññāvedayitanirodhaṃ samāpanno’ti vā.
\end{rightcolumn}
\end{samepage}

\begin{samepage}
\ensurevspace{4\baselineskip}
\begin{leftcolumn*}
But previously his mind has been developed so that it leads to that state.”
\end{leftcolumn*}

\begin{rightcolumn}
atha khvāssa pubbeva tathā cittaṃ bhāvitaṃ hoti yaṃ taṃ tathattāya upanetī”ti.
\end{rightcolumn}
\end{samepage}

\begin{samepage}
\ensurevspace{4\baselineskip}
\begin{leftcolumn*}
-
\end{leftcolumn*}

\begin{rightcolumn}
-
\end{rightcolumn}
\end{samepage}

\begin{samepage}
\ensurevspace{4\baselineskip}
\begin{leftcolumn*}
“But for a monastic who has attained the cessation of perception and feeling, Noble Lady, which things cease first: bodily formation, or speech formation, or mental formation?”
\end{leftcolumn*}

\begin{rightcolumn}
“saññāvedayitanirodhaṃ samāpajjantass panāyye, bhikkhuno katame dhammā paṭhamaṃ nirujjhanti — yadi vā kāyasaṅkhāro, yadi vā vacīsaṅkhāro, yadi vā cittasaṅkhāro”ti?
\end{rightcolumn}
\end{samepage}

\begin{samepage}
\ensurevspace{4\baselineskip}
\begin{leftcolumn*}
“For a monastic who is attaining the cessation of perception and feeling, friend Visākha, first speech formation ceases, then bodily formation ceases, then mental formation ceases.”
\end{leftcolumn*}

\begin{rightcolumn}
“saññāvedayitanirodhaṃ samāpajjantassa kho, āvuso visākha, bhikkhuno paṭhamaṃ nirujjhati vacīsaṅkhāro, tato kāyasaṅkhāro, tato cittasaṅkhāro”ti.
\end{rightcolumn}
\end{samepage}

\begin{samepage}
\ensurevspace{4\baselineskip}
\begin{leftcolumn*}
-
\end{leftcolumn*}

\begin{rightcolumn}
-
\end{rightcolumn}
\end{samepage}

\begin{samepage}
\ensurevspace{4\baselineskip}
\begin{leftcolumn*}
“But what, Noble Lady, is the emergence from the cessation of perception and feeling?”
\end{leftcolumn*}

\begin{rightcolumn}
“kathaṃ panāyye, saññāvedayitanirodhasamāpattiyā vuṭṭhānaṃ hotī”ti?
\end{rightcolumn}
\end{samepage}

\begin{samepage}
\ensurevspace{4\baselineskip}
\begin{leftcolumn*}
“A monastic who is emerging from the cessation of perception and feeling, friend Visākha, does not think:
\end{leftcolumn*}

\begin{rightcolumn}
“na kho, āvuso visākha, saññāvedayitanirodhasamāpattiyā vuṭṭhahantassa bhikkhuno evaṃ hoti —
\end{rightcolumn}
\end{samepage}

\begin{samepage}
\ensurevspace{4\baselineskip}
\begin{leftcolumn*}
‘I will emerge from the cessation of perception and feeling,’ or,
\end{leftcolumn*}

\begin{rightcolumn}
‘ahaṃ saññāvedayitanirodhasamāpattiyā vuṭṭhahissan’ti vā,
\end{rightcolumn}
\end{samepage}

\begin{samepage}
\ensurevspace{4\baselineskip}
\begin{leftcolumn*}
‘I am emerging from the cessation of perception and feeling,’ or,
\end{leftcolumn*}

\begin{rightcolumn}
‘ahaṃ saññāvedayitanirodhasamāpattiyā vuṭṭhahāmī’ti vā,
\end{rightcolumn}
\end{samepage}

\begin{samepage}
\ensurevspace{4\baselineskip}
\begin{leftcolumn*}
‘I have emerged from the cessation of perception and feeling,’
\end{leftcolumn*}

\begin{rightcolumn}
‘ahaṃ saññāvedayitanirodhasamāpattiyā vuṭṭhito’ti vā.
\end{rightcolumn}
\end{samepage}

\begin{samepage}
\ensurevspace{4\baselineskip}
\begin{leftcolumn*}
But previously his mind has been developed so that it leads to that state.”
\end{leftcolumn*}

\begin{rightcolumn}
atha khvāssa pubbeva tathā cittaṃ bhāvitaṃ hoti yaṃ taṃ tathattāya upanetī”ti.
\end{rightcolumn}
\end{samepage}

\begin{samepage}
\ensurevspace{4\baselineskip}
\begin{leftcolumn*}
-
\end{leftcolumn*}

\begin{rightcolumn}
-
\end{rightcolumn}
\end{samepage}

\begin{samepage}
\ensurevspace{4\baselineskip}
\begin{leftcolumn*}
“But for a monastic who has emerged from the cessation of perception and feeling, Noble Lady, which things arise first: bodily formation, or speech formation, or mental formation?”
\end{leftcolumn*}

\begin{rightcolumn}
“saññāvedayitanirodhasamāpattiyā vuṭṭhahantassa panāyye, bhikkhuno katame dhammā paṭhamaṃ uppajjanti — yadi vā kāyasaṅkhāro, yadi vā vacīsaṅkhāro, yadi vā cittasaṅkhāro”ti?
\end{rightcolumn}
\end{samepage}

\begin{samepage}
\ensurevspace{4\baselineskip}
\begin{leftcolumn*}
“For a monastic who is emerging from the cessation of perception and feeling, friend Visākha, first mental formation arises, then bodily formation arises, then speech formation arises.”
\end{leftcolumn*}

\begin{rightcolumn}
“saññāvedayitanirodhasamāpattiyā vuṭṭhahantassa kho, āvuso visākha, bhikkhuno paṭhamaṃ uppajjati cittasaṅkhāro, tato kāyasaṅkhāro, tato vacīsaṅkhāro”ti.
\end{rightcolumn}
\end{samepage}

\begin{samepage}
\ensurevspace{4\baselineskip}
\begin{leftcolumn*}
-
\end{leftcolumn*}

\begin{rightcolumn}
-
\end{rightcolumn}
\end{samepage}

\begin{samepage}
\ensurevspace{4\baselineskip}
\begin{leftcolumn*}
“Having emerged from the cessation of perception and feeling, Noble Lady, how many contacts touch that monastic?”
\end{leftcolumn*}

\begin{rightcolumn}
“saññāvedayitanirodhasamāpattiyā vuṭṭhitaṃ panāyye, bhikkhuṃ kati phassā phusantī”ti?
\end{rightcolumn}
\end{samepage}

\begin{samepage}
\ensurevspace{4\baselineskip}
\begin{leftcolumn*}
“Having emerged from the cessation of perception and feeling, friend Visākha, three contacts touch that monastic: emptiness contact, signlessness contact, desirelessness contact.”
\end{leftcolumn*}

\begin{rightcolumn}
“saññāvedayitanirodhasamāpattiyā vuṭṭhitaṃ kho, āvuso visākha, bhikkhuṃ tayo phassā phusanti — suññato phasso, animitto phasso, appaṇihito phasso”ti.
\end{rightcolumn}
\end{samepage}

\begin{samepage}
\ensurevspace{4\baselineskip}
\begin{leftcolumn*}
-
\end{leftcolumn*}

\begin{rightcolumn}
-
\end{rightcolumn}
\end{samepage}

\begin{samepage}
\ensurevspace{4\baselineskip}
\begin{leftcolumn*}
“For a monastic who has emerged from the cessation of perception and feeling, Noble Lady, what does his mind incline towards, what does it slope towards, what does it slant towards?”
\end{leftcolumn*}

\begin{rightcolumn}
“saññāvedayitanirodhasamāpattiyā vuṭṭhitassa panāyye, bhikkhuno kiṃninnaṃ cittaṃ hoti kiṃpoṇaṃ kiṃpabbhāran”ti?
\end{rightcolumn}
\end{samepage}

\begin{samepage}
\ensurevspace{4\baselineskip}
\begin{leftcolumn*}
“For a monastic who has emerged from the cessation of perception and feeling, friend Visākha, his mind inclines towards seclusion, it slopes towards seclusion, it slants towards seclusion.”
\end{leftcolumn*}

\begin{rightcolumn}
“saññāvedayitanirodhasamāpattiyā vuṭṭhitassa kho, āvuso visākha, bhikkhuno vivekaninnaṃ cittaṃ hoti, vivekapoṇaṃ vivekapabbhāran”ti.
\end{rightcolumn}
\end{samepage}

\begin{samepage}
\ensurevspace{4\baselineskip}
\begin{leftcolumn*}
-
\end{leftcolumn*}

\begin{rightcolumn}
-
\end{rightcolumn}
\end{samepage}

\begin{samepage}
\ensurevspace{4\baselineskip}
\begin{leftcolumn*}
“But how many feelings are there, Noble Lady?”
\end{leftcolumn*}

\begin{rightcolumn}
“kati panāyye, vedanā”ti?
\end{rightcolumn}
\end{samepage}

\begin{samepage}
\ensurevspace{4\baselineskip}
\begin{leftcolumn*}
“There are three feelings, friend Visākha: pleasant feeling, unpleasant feeling, and neither-unpleasant-nor-pleasant feeling.”
\end{leftcolumn*}

\begin{rightcolumn}
“tisso kho imā, āvuso visākha, vedanā — sukhā vedanā, dukkhā vedanā, adukkhamasukhā vedanā”ti.
\end{rightcolumn}
\end{samepage}

\begin{samepage}
\ensurevspace{4\baselineskip}
\begin{leftcolumn*}
-
\end{leftcolumn*}

\begin{rightcolumn}
-
\end{rightcolumn}
\end{samepage}

\begin{samepage}
\ensurevspace{4\baselineskip}
\begin{leftcolumn*}
“But what, Noble Lady, is pleasant feeling, what is unpleasant feeling, what is neither-unpleasant-nor-pleasant feeling?”
\end{leftcolumn*}

\begin{rightcolumn}
“katamā panāyye, sukhā vedanā, katamā dukkhā vedanā, katamā adukkhamasukhā vedanā”ti?
\end{rightcolumn}
\end{samepage}

\begin{samepage}
\ensurevspace{4\baselineskip}
\begin{leftcolumn*}
“Whatever, friend Visākha, is bodily or mentally pleasant and agreeable feeling: that is pleasant feeling.
\end{leftcolumn*}

\begin{rightcolumn}
“yaṃ kho, āvuso visākha, kāyikaṃ vā cetasikaṃ vā sukhaṃ sātaṃ vedayitaṃ — ayaṃ sukhā vedanā.
\end{rightcolumn}
\end{samepage}

\begin{samepage}
\ensurevspace{4\baselineskip}
\begin{leftcolumn*}
Whatever, friend Visākha, is bodily or mentally unpleasant and disagreeable feeling: that is unpleasant feeling.
\end{leftcolumn*}

\begin{rightcolumn}
yaṃ kho, āvuso visākha, kāyikaṃ vā cetasikaṃ vā dukkhaṃ asātaṃ vedayitaṃ — ayaṃ dukkhā vedanā.
\end{rightcolumn}
\end{samepage}

\begin{samepage}
\ensurevspace{4\baselineskip}
\begin{leftcolumn*}
Whatever, friend Visākha, is bodily or mentally neither agreeable nor disagreeable feeling: that is neither-unpleasant-nor-pleasant feeling.”
\end{leftcolumn*}

\begin{rightcolumn}
yaṃ kho, āvuso visākha, kāyikaṃ vā cetasikaṃ vā neva sātaṃ nāsātaṃ vedayitaṃ — ayaṃ adukkhamasukhā vedanā”ti.
\end{rightcolumn}
\end{samepage}

\begin{samepage}
\ensurevspace{4\baselineskip}
\begin{leftcolumn*}
-
\end{leftcolumn*}

\begin{rightcolumn}
-
\end{rightcolumn}
\end{samepage}

\begin{samepage}
\ensurevspace{4\baselineskip}
\begin{leftcolumn*}
“But regarding pleasant feeling, Noble Lady: what is pleasant, what is unpleasant,
\end{leftcolumn*}

\begin{rightcolumn}
“sukhā panāyye, vedanā kiṃsukhā kiṃdukkhā,
\end{rightcolumn}
\end{samepage}

\begin{samepage}
\ensurevspace{4\baselineskip}
\begin{leftcolumn*}
regarding unpleasant feeling: what is pleasant, what is unpleasant,
\end{leftcolumn*}

\begin{rightcolumn}
dukkhā vedanā kiṃsukhā kiṃdukkhā,
\end{rightcolumn}
\end{samepage}

\begin{samepage}
\ensurevspace{4\baselineskip}
\begin{leftcolumn*}
regarding neither-unpleasant-nor-pleasant feeling: what is pleasant, what is unpleasant?”
\end{leftcolumn*}

\begin{rightcolumn}
adukkhamasukhā vedanā kiṃsukhā kiṃdukkhā”ti?
\end{rightcolumn}
\end{samepage}

\begin{samepage}
\ensurevspace{4\baselineskip}
\begin{leftcolumn*}
“Pleasant feeling, friend Visākha, is pleasant when it persists, unpleasant when it changes,
\end{leftcolumn*}

\begin{rightcolumn}
“sukhā kho, āvuso visākha, vedanā ṭhitisukhā vipariṇāmadukkhā;
\end{rightcolumn}
\end{samepage}

\begin{samepage}
\ensurevspace{4\baselineskip}
\begin{leftcolumn*}
unpleasant feeling is unpleasant when it persists, pleasant when it changes,
\end{leftcolumn*}

\begin{rightcolumn}
dukkhā vedanā ṭhitidukkhā vipariṇāmasukhā;
\end{rightcolumn}
\end{samepage}

\begin{samepage}
\ensurevspace{4\baselineskip}
\begin{leftcolumn*}
neither-unpleasant-nor-pleasant feeling is pleasant when known, and unpleasant when unknown.”
\end{leftcolumn*}

\begin{rightcolumn}
adukkhamasukhā vedanā ñāṇasukhā aññāṇadukkhā”ti.
\end{rightcolumn}
\end{samepage}

\begin{samepage}
\ensurevspace{4\baselineskip}
\begin{leftcolumn*}
-
\end{leftcolumn*}

\begin{rightcolumn}
-
\end{rightcolumn}
\end{samepage}

\begin{samepage}
\ensurevspace{4\baselineskip}
\begin{leftcolumn*}
“But for pleasant feeling, Noble Lady, what tendency underlies it,
\end{leftcolumn*}

\begin{rightcolumn}
“sukhāya panāyye, vedanāya kiṃ anusayo anuseti,
\end{rightcolumn}
\end{samepage}

\begin{samepage}
\ensurevspace{4\baselineskip}
\begin{leftcolumn*}
for unpleasant feeling what tendency underlies it,
\end{leftcolumn*}

\begin{rightcolumn}
dukkhāya vedanāya kiṃ anusayo anuseti,
\end{rightcolumn}
\end{samepage}

\begin{samepage}
\ensurevspace{4\baselineskip}
\begin{leftcolumn*}
for neither-unpleasant-nor-pleasant feeling what tendency underlies it?”
\end{leftcolumn*}

\begin{rightcolumn}
adukkhamasukhāya vedanāya kiṃ anusayo anusetī”ti?
\end{rightcolumn}
\end{samepage}

\begin{samepage}
\ensurevspace{4\baselineskip}
\begin{leftcolumn*}
“For pleasant feeling, friend Visākha, the tendency to passion underlies it,
\end{leftcolumn*}

\begin{rightcolumn}
“sukhāya kho, āvuso visākha, vedanāya rāgānusayo anuseti,
\end{rightcolumn}
\end{samepage}

\begin{samepage}
\ensurevspace{4\baselineskip}
\begin{leftcolumn*}
for unpleasant feeling the tendency to repulsion underlies it,
\end{leftcolumn*}

\begin{rightcolumn}
dukkhāya vedanāya paṭighānusayo anuseti,
\end{rightcolumn}
\end{samepage}

\begin{samepage}
\ensurevspace{4\baselineskip}
\begin{leftcolumn*}
for neither-unpleasant-nor-pleasant feeling ignorance underlies it.”
\end{leftcolumn*}

\begin{rightcolumn}
adukkhamasukhāya vedanāya avijjānusayo anusetī”ti.
\end{rightcolumn}
\end{samepage}

\begin{samepage}
\ensurevspace{4\baselineskip}
\begin{leftcolumn*}
-
\end{leftcolumn*}

\begin{rightcolumn}
-
\end{rightcolumn}
\end{samepage}

\begin{samepage}
\ensurevspace{4\baselineskip}
\begin{leftcolumn*}
“But for all pleasant feeling, Noble Lady, does the tendency to passion underlie it,
\end{leftcolumn*}

\begin{rightcolumn}
“sabbāya nu kho, ayye, sukhāya vedanāya rāgānusayo anuseti,
\end{rightcolumn}
\end{samepage}

\begin{samepage}
\ensurevspace{4\baselineskip}
\begin{leftcolumn*}
for all unpleasant feeling does the tendency to repulsion underlie it,
\end{leftcolumn*}

\begin{rightcolumn}
sabbāya dukkhāya vedanāya paṭighānusayo anuseti,
\end{rightcolumn}
\end{samepage}

\begin{samepage}
\ensurevspace{4\baselineskip}
\begin{leftcolumn*}
for all neither-unpleasant-nor-pleasant feeling does the tendency to ignorance underlie it?”
\end{leftcolumn*}

\begin{rightcolumn}
sabbāya adukkhamasukhāya vedanāya avijjānusayo anusetī”ti?
\end{rightcolumn}
\end{samepage}

\begin{samepage}
\ensurevspace{4\baselineskip}
\begin{leftcolumn*}
“Not for all pleasant feeling, friend Visākha, does the tendency to passion underlie it,
\end{leftcolumn*}

\begin{rightcolumn}
“na kho, āvuso visākha, sabbāya sukhāya vedanāya rāgānusayo anuseti,
\end{rightcolumn}
\end{samepage}

\begin{samepage}
\ensurevspace{4\baselineskip}
\begin{leftcolumn*}
not for all unpleasant feeling does the tendency to repulsion underlie it,
\end{leftcolumn*}

\begin{rightcolumn}
na sabbāya dukkhāya vedanāya paṭighānusayo anuseti,
\end{rightcolumn}
\end{samepage}

\begin{samepage}
\ensurevspace{4\baselineskip}
\begin{leftcolumn*}
not for all neither-unpleasant-nor-pleasant feeling does the tendency to ignorance underlie it.”
\end{leftcolumn*}

\begin{rightcolumn}
na sabbāya adukkhamasukhāya vedanāya avijjānusayo anusetī”ti.
\end{rightcolumn}
\end{samepage}

\begin{samepage}
\ensurevspace{4\baselineskip}
\begin{leftcolumn*}
-
\end{leftcolumn*}

\begin{rightcolumn}
-
\end{rightcolumn}
\end{samepage}

\begin{samepage}
\ensurevspace{4\baselineskip}
\begin{leftcolumn*}
“But for all pleasant feeling, Noble Lady, what should be abandoned,
\end{leftcolumn*}

\begin{rightcolumn}
“sukhāya panāyye, vedanāya kiṃ pahātabbaṃ,
\end{rightcolumn}
\end{samepage}

\begin{samepage}
\ensurevspace{4\baselineskip}
\begin{leftcolumn*}
for all unpleasant feeling what should be abandoned,
\end{leftcolumn*}

\begin{rightcolumn}
dukkhāya vedanāya kiṃ pahātabbaṃ,
\end{rightcolumn}
\end{samepage}

\begin{samepage}
\ensurevspace{4\baselineskip}
\begin{leftcolumn*}
for all neither-unpleasant-nor-pleasant feeling what should be abandoned?”
\end{leftcolumn*}

\begin{rightcolumn}
adukkhamasukhāya vedanāya kiṃ pahātabban”ti?
\end{rightcolumn}
\end{samepage}

\begin{samepage}
\ensurevspace{4\baselineskip}
\begin{leftcolumn*}
“For pleasant feeling, friend Visākha, the tendency to passion should be abandoned,
\end{leftcolumn*}

\begin{rightcolumn}
“sukhāya kho, āvuso visākha, vedanāya rāgānusayo pahātabbo,
\end{rightcolumn}
\end{samepage}

\begin{samepage}
\ensurevspace{4\baselineskip}
\begin{leftcolumn*}
for unpleasant feeling the tendency to repulsion should be abandoned,
\end{leftcolumn*}

\begin{rightcolumn}
dukkhāya vedanāya paṭighānusayo pahātabbo,
\end{rightcolumn}
\end{samepage}

\begin{samepage}
\ensurevspace{4\baselineskip}
\begin{leftcolumn*}
for neither-unpleasant-nor-pleasant feeling the tendency to ignorance should be abandoned.”
\end{leftcolumn*}

\begin{rightcolumn}
adukkhamasukhāya vedanāya avijjānusayo pahātabbo”ti.
\end{rightcolumn}
\end{samepage}

\begin{samepage}
\ensurevspace{4\baselineskip}
\begin{leftcolumn*}
-
\end{leftcolumn*}

\begin{rightcolumn}
-
\end{rightcolumn}
\end{samepage}

\begin{samepage}
\ensurevspace{4\baselineskip}
\begin{leftcolumn*}
“But for all pleasant feeling, Noble Lady, (is there) a tendency to passion that should be abandoned,
\end{leftcolumn*}

\begin{rightcolumn}
“sabbāya nu kho, ayye, sukhāya vedanāya rāgānusayo pahātabbo,
\end{rightcolumn}
\end{samepage}

\begin{samepage}
\ensurevspace{4\baselineskip}
\begin{leftcolumn*}
for all unpleasant feeling (is there) a tendency to repulsion that should be abandoned,
\end{leftcolumn*}

\begin{rightcolumn}
sabbāya dukkhāya vedanāya paṭighānusayo pahātabbo,
\end{rightcolumn}
\end{samepage}

\begin{samepage}
\ensurevspace{4\baselineskip}
\begin{leftcolumn*}
for all neither-unpleasant-nor-pleasant feeling (is there) a tendency to ignorance that should be abandoned?”
\end{leftcolumn*}

\begin{rightcolumn}
sabbāya adukkhamasukhāya vedanāya avijjānusayo pahātabbo”ti?
\end{rightcolumn}
\end{samepage}

\begin{samepage}
\ensurevspace{4\baselineskip}
\begin{leftcolumn*}
“Not for all pleasant feeling, friend Visākha, (is there) a tendency to passion that should be abandoned,
\end{leftcolumn*}

\begin{rightcolumn}
“na kho, āvuso visākha, sabbāya sukhāya vedanāya rāgānusayo pahātabbo,
\end{rightcolumn}
\end{samepage}

\begin{samepage}
\ensurevspace{4\baselineskip}
\begin{leftcolumn*}
not for all unpleasant feeling (is there) a tendency to repulsion that should be abandoned,
\end{leftcolumn*}

\begin{rightcolumn}
na sabbāya dukkhāya vedanāya paṭighānusayo pahātabbo,
\end{rightcolumn}
\end{samepage}

\begin{samepage}
\ensurevspace{4\baselineskip}
\begin{leftcolumn*}
not for all neither-unpleasant-nor-pleasant feeling (is there) a tendency to ignorance that should be abandoned.
\end{leftcolumn*}

\begin{rightcolumn}
na sabbāya adukkhamasukhāya vedanāya avijjānusayo pahātabbo.
\end{rightcolumn}
\end{samepage}

\begin{samepage}
\ensurevspace{4\baselineskip}
\begin{leftcolumn*}
Here, friend Visākha, a monastic, quite secluded from sense desires, secluded from unwholesome things, having thinking, reflection, and the happiness and rapture born of seclusion, dwells having attained the first absorption.
\end{leftcolumn*}

\begin{rightcolumn}
idhāvuso visākha, bhikkhu vivicceva kāmehi vivicca akusalehi dhammehi savitakkaṃ savicāraṃ vivekajaṃ pītisukhaṃ paṭhamaṃ jhānaṃ upasampajja viharati.
\end{rightcolumn}
\end{samepage}

\begin{samepage}
\ensurevspace{4\baselineskip}
\begin{leftcolumn*}
- On that basis passion is abandoned, and herein there is no more underlying tendency to passion.
\end{leftcolumn*}

\begin{rightcolumn}
- rāgaṃ tena pajahati, na tattha rāgānusayo anuseti.
\end{rightcolumn}
\end{samepage}

\begin{samepage}
\ensurevspace{4\baselineskip}
\begin{leftcolumn*}
Here, friend Visākha, a monastic considers thus: ‘When will I dwell having attained that sphere that the Noble Ones now dwell in having attained that sphere?’ Thus a longing to give attendance towards that unsurpassed freedom arises and with longing as condition sorrow (arises).
\end{leftcolumn*}

\begin{rightcolumn}
idhāvuso visākha, bhikkhu iti paṭisañcikkhati — ‘kudāssu nāmāhaṃ tadāyatanaṃ upasampajja viharissāmi yadariyā etarahi āyatanaṃ upasampajja viharantī’ti? iti anuttaresu vimokkhesu pihaṃ upaṭṭhāpayato uppajjati pihāppaccayā domanassaṃ.
\end{rightcolumn}
\end{samepage}

\begin{samepage}
\ensurevspace{4\baselineskip}
\begin{leftcolumn*}
- On that basis repulsion is abandoned, and herein there is no more underlying tendency to repulsion.
\end{leftcolumn*}

\begin{rightcolumn}
- paṭighaṃ tena pajahati, na tattha paṭighānusayo anuseti.
\end{rightcolumn}
\end{samepage}

\begin{samepage}
\ensurevspace{4\baselineskip}
\begin{leftcolumn*}
Here, friend Visākha, a monastic, having given up pleasure, given up pain, and with the previous disappearence of mental well-being and sorrow, without pain, without pleasure, and with complete purity of mindfulness owing to equanimity, dwells having attained the fourth absorption.
\end{leftcolumn*}

\begin{rightcolumn}
idhāvuso visākha, bhikkhu sukhassa ca pahānā, dukkhassa ca pahānā, pubbeva somanassadomanassānaṃ atthaṅgamā, adukkhamasukhaṃ upekkhāsatipārisuddhiṃ catutthaṃ jhānaṃ upasampajja viharati.
\end{rightcolumn}
\end{samepage}

\begin{samepage}
\ensurevspace{4\baselineskip}
\begin{leftcolumn*}
On that basis ignorance is abandoned, and herein there is no more underlying tendency to ignorance.”
\end{leftcolumn*}

\begin{rightcolumn}
- avijjaṃ tena pajahati, na tattha avijjānusayo anusetī”ti.
\end{rightcolumn}
\end{samepage}

\begin{samepage}
\ensurevspace{4\baselineskip}
\begin{leftcolumn*}
-
\end{leftcolumn*}

\begin{rightcolumn}
-
\end{rightcolumn}
\end{samepage}

\begin{samepage}
\ensurevspace{4\baselineskip}
\begin{leftcolumn*}
“But for pleasant feeling, Noble Lady, what is the complement?”
\end{leftcolumn*}

\begin{rightcolumn}
“sukhāya panāyye, vedanāya kiṃ paṭibhāgo”ti?
\end{rightcolumn}
\end{samepage}

\begin{samepage}
\ensurevspace{4\baselineskip}
\begin{leftcolumn*}
“For pleasant feeling, friend Visākha, the complement is unpleasant feeling.”
\end{leftcolumn*}

\begin{rightcolumn}
“sukhāya kho, āvuso visākha, vedanāya dukkhā vedanā paṭibhāgo”ti.
\end{rightcolumn}
\end{samepage}

\begin{samepage}
\ensurevspace{4\baselineskip}
\begin{leftcolumn*}
-
\end{leftcolumn*}

\begin{rightcolumn}
-
\end{rightcolumn}
\end{samepage}

\begin{samepage}
\ensurevspace{4\baselineskip}
\begin{leftcolumn*}
“But for unpleasant feeling, Noble Lady, what is the complement?”
\end{leftcolumn*}

\begin{rightcolumn}
“dukkhāya pannāyye, vedanāya kiṃ paṭibhāgo”ti?
\end{rightcolumn}
\end{samepage}

\begin{samepage}
\ensurevspace{4\baselineskip}
\begin{leftcolumn*}
“For unpleasant feeling, friend Visākha, the complement is pleasant feeling.”
\end{leftcolumn*}

\begin{rightcolumn}
“dukkhāya kho, āvuso visākha, vedanāya sukhā vedanā paṭibhāgo”ti.
\end{rightcolumn}
\end{samepage}

\begin{samepage}
\ensurevspace{4\baselineskip}
\begin{leftcolumn*}
-
\end{leftcolumn*}

\begin{rightcolumn}
-
\end{rightcolumn}
\end{samepage}

\begin{samepage}
\ensurevspace{4\baselineskip}
\begin{leftcolumn*}
“But for neither-unpleasant-nor-pleasant feeling, Noble Lady, what is the complement?”
\end{leftcolumn*}

\begin{rightcolumn}
“adukkhamasukhāya panāyye, vedanāya kiṃ paṭibhāgo”ti?
\end{rightcolumn}
\end{samepage}

\begin{samepage}
\ensurevspace{4\baselineskip}
\begin{leftcolumn*}
“For neither-unpleasant-nor-pleasant feeling, friend Visākha, the complement is ignorance.”
\end{leftcolumn*}

\begin{rightcolumn}
“adukkhamasukhāya kho, āvuso visākha, vedanāya avijjā paṭibhāgo”ti.
\end{rightcolumn}
\end{samepage}

\begin{samepage}
\ensurevspace{4\baselineskip}
\begin{leftcolumn*}
-
\end{leftcolumn*}

\begin{rightcolumn}
-
\end{rightcolumn}
\end{samepage}

\begin{samepage}
\ensurevspace{4\baselineskip}
\begin{leftcolumn*}
“But for ignorance, Noble Lady, what is the complement?”
\end{leftcolumn*}

\begin{rightcolumn}
“avijjāya panāyye, kiṃ paṭibhāgo”ti?
\end{rightcolumn}
\end{samepage}

\begin{samepage}
\ensurevspace{4\baselineskip}
\begin{leftcolumn*}
“For ignorance, friend Visākha, the complement is understanding.”
\end{leftcolumn*}

\begin{rightcolumn}
“avijjāya kho, āvuso visākha, vijjā paṭibhāgo”ti.
\end{rightcolumn}
\end{samepage}

\begin{samepage}
\ensurevspace{4\baselineskip}
\begin{leftcolumn*}
-
\end{leftcolumn*}

\begin{rightcolumn}
-
\end{rightcolumn}
\end{samepage}

\begin{samepage}
\ensurevspace{4\baselineskip}
\begin{leftcolumn*}
“But for understanding, Noble Lady, what is the complement?”
\end{leftcolumn*}

\begin{rightcolumn}
“vijjāya panāyye, kiṃ paṭibhāgo”ti?
\end{rightcolumn}
\end{samepage}

\begin{samepage}
\ensurevspace{4\baselineskip}
\begin{leftcolumn*}
“For understanding, friend Visākha, the complement is freedom.”
\end{leftcolumn*}

\begin{rightcolumn}
“vijjāya kho, āvuso visākha, vimutti paṭibhāgo”ti.
\end{rightcolumn}
\end{samepage}

\begin{samepage}
\ensurevspace{4\baselineskip}
\begin{leftcolumn*}
-
\end{leftcolumn*}

\begin{rightcolumn}
-
\end{rightcolumn}
\end{samepage}

\begin{samepage}
\ensurevspace{4\baselineskip}
\begin{leftcolumn*}
“But for freedom, Noble Lady, what is the complement?”
\end{leftcolumn*}

\begin{rightcolumn}
“vimuttiyā panāyye, kiṃ paṭibhāgo”ti?
\end{rightcolumn}
\end{samepage}

\begin{samepage}
\ensurevspace{4\baselineskip}
\begin{leftcolumn*}
“For freedom, friend Visākha, the complement is Nibbāna.”
\end{leftcolumn*}

\begin{rightcolumn}
“vimuttiyā kho, āvuso visākha, nibbānaṃ paṭibhāgo”ti.
\end{rightcolumn}
\end{samepage}

\begin{samepage}
\ensurevspace{4\baselineskip}
\begin{leftcolumn*}
-
\end{leftcolumn*}

\begin{rightcolumn}
-
\end{rightcolumn}
\end{samepage}

\begin{samepage}
\ensurevspace{4\baselineskip}
\begin{leftcolumn*}
“But for Nibbāna, Noble Lady, what is the complement?”
\end{leftcolumn*}

\begin{rightcolumn}
“nibbānassa panāyye, kiṃ paṭibhāgo”ti?
\end{rightcolumn}
\end{samepage}

\begin{samepage}
\ensurevspace{4\baselineskip}
\begin{leftcolumn*}
“You are not able to grasp, friend Visākha, answers to questions that are beyond your limits, like immersion in Nibbāna, friend Visākha, the spiritual life that ends in Nibbāna, that conclusion in Nibbāna.
\end{leftcolumn*}

\begin{rightcolumn}
“accayāsi, āvuso visākha, pañhaṃ, nāsakkhi pañhānaṃ pariyantaṃ gahetuṃ. nibbānogadhañhi, āvuso visākha, brahmacariyaṃ, nibbānaparāyanaṃ nibbānapariyosānaṃ.
\end{rightcolumn}
\end{samepage}

\begin{samepage}
\ensurevspace{4\baselineskip}
\begin{leftcolumn*}
Desiring this, friend Visākha, approach the Blessed One and you can ask him about this matter, and just as the Blessed One explains, so you should bear it in mind.”
\end{leftcolumn*}

\begin{rightcolumn}
ākaṅkhamāno ca tvaṃ, āvuso visākha, bhagavantaṃ upasaṅkamitvā etamatthaṃ puccheyyāsi, yathā ca te bhagavā byākaroti tathā naṃ dhāreyyāsī”ti.
\end{rightcolumn}
\end{samepage}

\begin{samepage}
\ensurevspace{4\baselineskip}
\begin{leftcolumn*}
-
\end{leftcolumn*}

\begin{rightcolumn}
-
\end{rightcolumn}
\end{samepage}

\begin{samepage}
\ensurevspace{4\baselineskip}
\begin{leftcolumn*}
Then the devotee Visākha, after greatly rejoicing and gladly receiving this word of the nun Dhammadinnā, having worshipped and circumambulated the nun Dhammadinnā, approached the Blessed One, and after approaching and worshipping the Blessed One, he sat down on one side. While sitting on one side the devotee Visākha related the whole conversation he had had with the nun Dhammadinnā to the Blessed One.
\end{leftcolumn*}

\begin{rightcolumn}
atha kho visākho upāsako dhammadinnāya bhikkhuniyā bhāsitaṃ abhinanditvā anumoditvā uṭṭhāyāsanā dhammadinnaṃ bhikkhuniṃ abhivādetvā padakkhiṇaṃ katvā yena bhagavā tenupasaṅkami; upasaṅkamitvā bhagavantaṃ abhivādetvā ekamantaṃ nisīdi. ekamantaṃ nisinno kho visākho upāsako yāvatako ahosi dhammadinnāya bhikkhuniyā saddhiṃ kathāsallāpo taṃ sabbaṃ bhagavato ārocesi.
\end{rightcolumn}
\end{samepage}

\begin{samepage}
\ensurevspace{4\baselineskip}
\begin{leftcolumn*}
That being said, the Blessed One said this to the devotee Visākha: “Wise, Visākha, is the nun Dhammadinnā, having great wisdom, Visākha, is the nun Dhammadinnā, if you were to ask me, Visākha, the same matter, I would answer it in the same way, in the way the nun Dhammadinnā has answered, for this is indeed the meaning, and so should you bear it in mind.”
\end{leftcolumn*}

\begin{rightcolumn}
evaṃ vutte, bhagavā visākhaṃ upāsakaṃ etadavoca — “paṇḍitā, visākha, dhammadinnā bhikkhunī, mahāpaññā, visākha, dhammadinnā bhikkhunī. maṃ cepi tvaṃ, visākha, etamatthaṃ puccheyyāsi, ahampi taṃ evamevaṃ byākareyyaṃ, yathā taṃ dhammadinnāya bhikkhuniyā byākataṃ. eso cevetassa attho. evañca naṃ dhārehī”ti.
\end{rightcolumn}
\end{samepage}

\begin{samepage}
\ensurevspace{4\baselineskip}
\begin{leftcolumn*}
The Blessed One said this, and the devotee Visākha was uplifted and greatly rejoiced in what was said by the Blessed One.
\end{leftcolumn*}

\begin{rightcolumn}
idamavoca bhagavā. attamano visākho upāsako bhagavato bhāsitaṃ abhinandīti.
\end{rightcolumn}
\end{samepage}

\begin{samepage}
\ensurevspace{4\baselineskip}
\begin{leftcolumn*}
The Lesser Series of Questions-and-Answers is Finished
\end{leftcolumn*}

\begin{rightcolumn}
cūḷavedallasuttaṃ niṭṭhitaṃ catutthaṃ.
\end{rightcolumn}
\end{samepage}

