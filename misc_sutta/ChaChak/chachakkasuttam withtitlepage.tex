\documentclass[9pt]{article}
\usepackage[margin=40pt,
		paperheight=612pt, 
		paperwidth=396pt]{geometry}
\usepackage{titlesec}
%\usepackage{polyglossia}
\usepackage{fontspec}
\usepackage[dvipsnames]{xcolor}
\setromanfont[BoldFont={Gentium Basic Bold},ItalicFont={Gentium Italic}]{Gentium}
\usepackage{expex}
\sloppy
%inhibit page breaks in paragraphs
\widowpenalties 1 10000
\raggedbottom
\raggedright

%Glossing Definitions
\lingset{glstyle=nlevel,%sets the compile method for ExPex to the nlevel style (alternates words with []) as opposed to the wrap style uses gla glb separted lines
	glhangindent=10pt,% sets the indentation to the entire gloss after the first set of lines
	glossbreaking=true, %Allows for Glosses to break across pages
	glwordalign=left, %sets allignment on pairs of words, center is other option
	glneveryline={,\footnotesize\it}, %sets the font attributes per line {gla,glb...}
	glnabovelineskip={,-4pt}, %sets the space above the glossing {gla,glb,...}
	extraglskip=-2pt} %sets additional space between each line of glossing

% Uncomment the following line if you do not want extra space before the free translation and the glosses, or other customisations to the examples.
\lingset{aboveglftskip=-.2ex,interpartskip=\baselineskip,everyglb=\footnotesize}
\newcommand*{\titlePaliSuttaTitlePage}{\begingroup % Create the command for including the title page in the document
\newlength{\drop} % Command for generating a specific amount of whitespace
\drop=0.1\textheight % Define the command as 10% of the total text height

\rule{\textwidth}{1pt}\par % Thick horizontal line
\vspace{2pt}\vspace{-\baselineskip} % Whitespace between lines
\rule{\textwidth}{0.4pt}\par % Thin horizontal line

\vspace{\drop} % Whitespace between the top lines and title
\centering % Center all text

{\Huge chachakkasuttaṃ}\\[1\baselineskip] % Title line 1
{\large (MN 148)}

\vspace{0.25\drop} % Whitespace between the title and short horizontal line
\rule{0.3\textwidth}{0.4pt}\par % Short horizontal line under the title
\vspace{\drop} % Whitespace between the thin horizontal line and the author name

{\large\textsc{Samantā cakkavālesu, atr'āgacchantu devatā;}}\par
{\large\textsc{Saddhammaṁ muni-rājassa suṇantu sagga-mokkha-daṁ}}\\[2\baselineskip]


{\large\textsc{Dhamma-ssavana-kālo  ayam-bhadantā}}\par
{\large\textsc{Dhamma-ssavana-kālo  ayam-bhadantā}}\par
{\large\textsc{Dhamma-ssavana-kālo  ayam-bhadantā}}\\[2\baselineskip]


{\large \textsc{Namo tassa Bhagavato arahato sammā-sambuddhassa}}\par %
{\large \textsc{Namo tassa Bhagavato arahato sammā-sambuddhassa}}\par %
{\large \textsc{Namo tassa Bhagavato arahato sammā-sambuddhassa}}\par %
\vspace*{\drop} % Whitespace under Namo tassa

\rule{\textwidth}{0.4pt}\par % Thin horizontal line
\vspace{2pt}\vspace{-\baselineskip} % Whitespace between lines
\rule{\textwidth}{1pt}\par % Thick horizontal line
\pagebreak
\endgroup}



\newcommand{\vsp}{\vspace{2mm}}

%Glossing Definitions
\lingset{glstyle=nlevel,%sets the compile method for ExPex to the nlevel style (alternates words with []) as opposed to the wrap style uses gla glb separted lines
	glhangindent=10pt,% sets the indentation to the entire gloss after the first set of lines
	glossbreaking=true, %Allows for Glosses to break across pages
	glwordalign=left, %sets allignment on pairs of words, center is other option
	glneveryline={,\footnotesize\it}, %sets the font attributes per line {gla,glb...}
	glnabovelineskip={,-4pt}, %sets the space above the glossing {gla,glb,...}
	extraglskip=-2pt} %sets additional space between each line of glossing

\begin{document}
\titlePaliSuttaTitlePage
 
420. evaṃ me sutaṃ — ekaṃ samayaṃ bhagavā sāvatthiyaṃ viharati jetavane anāthapiṇḍikassa ārāme. tatra kho bhagavā bhikkhū āmantesi —\\

“bhikkhavo”ti. “bhadante”ti te bhikkhū bhagavato paccassosuṃ. bhagavā etadavoca — “dhammaṃ vo, bhikkhave, desessāmi ādikalyāṇaṃ majjhekalyāṇaṃ pariyosānakalyāṇaṃ sātthaṃ sabyañjanaṃ, kevalaparipuṇṇaṃ parisuddhaṃ brahmacariyaṃ pakāsessāmi, yadidaṃ — cha chakkāni. taṃ suṇātha, sādhukaṃ manasi karotha; bhāsissāmī”ti. “evaṃ, bhante”ti kho te bhikkhū bhagavato paccassosuṃ. bhagavā etadavoca —\\\

“cha ajjhattikāni āyatanāni veditabbāni, cha bāhirāni āyatanāni veditabbāni, cha viññāṇakāyā veditabbā, cha phassakāyā veditabbā, cha vedanākāyā veditabbā, cha taṇhākāyā veditabbā.\\\

421.1 ‘cha ajjhattikāni āyatanāni veditabbānī’ti — iti kho panetaṃ vuttaṃ. kiñcetaṃ paṭicca vuttaṃ?\\\

cakkhāyatanaṃ, sotāyatanaṃ, ghānāyatanaṃ, jivhāyatanaṃ, kāyāyatanaṃ, manāyatanaṃ.\\\

‘cha ajjhattikāni āyatanāni veditabbānī’ti — iti yaṃ taṃ vuttaṃ, idametaṃ paṭicca vuttaṃ. idaṃ paṭhamaṃ chakkaṃ.\\\

2. ‘cha bāhirāni āyatanāni veditabbānī’ti — iti kho panetaṃ vuttaṃ. kiñcetaṃ paṭicca vuttaṃ?\\\

rūpāyatanaṃ, saddāyatanaṃ, gandhāyatanaṃ, rasāyatanaṃ, phoṭṭhabbāyatanaṃ, dhammāyatanaṃ.\\

‘cha bāhirāni āyatanāni veditabbānī’ti — iti yaṃ taṃ vuttaṃ, idametaṃ paṭicca vuttaṃ. idaṃ dutiyaṃ chakkaṃ.\\\

3. ‘cha viññāṇakāyā veditabbā’ti — iti kho panetaṃ vuttaṃ. kiñcetaṃ paṭicca vuttaṃ?\\\

cakkhuñca paṭicca rūpe ca uppajjati cakkhuviññāṇaṃ,\\

sotañca paṭicca sadde ca uppajjati sotaviññāṇaṃ,\\

ghānañca paṭicca gandhe ca uppajjati ghānaviññāṇaṃ,\\

jivhañca paṭicca rase ca uppajjati jivhāviññāṇaṃ,\\

kāyañca paṭicca phoṭṭhabbe ca uppajjati kāyaviññāṇaṃ,\\

manañca paṭicca dhamme ca uppajjati manoviññāṇaṃ.\\\

‘cha viññāṇakāyā veditabbā’ti — iti yaṃ taṃ vuttaṃ, idametaṃ paṭicca vuttaṃ. idaṃ tatiyaṃ chakkaṃ.\\\

4. ‘cha phassakāyā veditabbā’ti — iti kho panetaṃ vuttaṃ. kiñcetaṃ paṭicca vuttaṃ?\\\

cakkhuñca paṭicca rūpe ca uppajjati cakkhuviññāṇaṃ, tiṇṇaṃ saṅgati phasso;\

sotañca paṭicca sadde ca uppajjati sotaviññāṇaṃ, tiṇṇaṃ saṅgati phasso;\

ghānañca paṭicca gandhe ca uppajjati ghānaviññāṇaṃ, tiṇṇaṃ saṅgati phasso;\

jivhañca paṭicca rase ca uppajjati jivhāviññāṇaṃ, tiṇṇaṃ saṅgati phasso;\

kāyañca paṭicca phoṭṭhabbe ca uppajjati kāyaviññāṇaṃ, tiṇṇaṃ saṅgati phasso;\

manañca paṭicca dhamme ca uppajjati manoviññāṇaṃ, tiṇṇaṃ saṅgati phasso.\\\

‘cha phassakāyā veditabbā’ti — iti yaṃ taṃ vuttaṃ, idametaṃ paṭicca vuttaṃ. idaṃ catutthaṃ chakkaṃ.\\\

5. ‘cha vedanākāyā veditabbā’ti — iti kho panetaṃ vuttaṃ. kiñcetaṃ paṭicca vuttaṃ?\\\

cakkhuñca paṭicca rūpe ca uppajjati cakkhuviññāṇaṃ, tiṇṇaṃ saṅgati phasso, phassapaccayā vedanā;\

sotañca paṭicca sadde ca uppajjati sotaviññāṇaṃ, tiṇṇaṃ saṅgati phasso, phassapaccayā vedanā;\

ghānañca paṭicca gandhe ca uppajjati ghānaviññāṇaṃ, tiṇṇaṃ saṅgati phasso, phassapaccayā vedanā;\

jivhañca paṭicca rase ca uppajjati jivhāviññāṇaṃ, tiṇṇaṃ saṅgati phasso, phassapaccayā vedanā;\

kāyañca paṭicca phoṭṭhabbe ca uppajjati kāyaviññāṇaṃ, tiṇṇaṃ saṅgati phasso, phassapaccayā vedanā;\

manañca paṭicca dhamme ca uppajjati manoviññāṇaṃ, tiṇṇaṃ saṅgati phasso, phassapaccayā vedanā.\\\

‘cha vedanākāyā veditabbā’ti — iti yaṃ taṃ vuttaṃ, idametaṃ paṭicca vuttaṃ. idaṃ pañcamaṃ chakkaṃ.\\\
\pagebreak\\
6. ‘cha taṇhākāyā veditabbā’ti — iti kho panetaṃ vuttaṃ. kiñcetaṃ paṭicca vuttaṃ?\\\

cakkhuñca paṭicca rūpe ca uppajjati cakkhuviññāṇaṃ, tiṇṇaṃ saṅgati phasso, phassapaccayā vedanā, vedanāpaccayā taṇhā;\

sotañca paṭicca sadde ca uppajjati sotaviññāṇaṃ, tiṇṇaṃ saṅgati phasso, phassapaccayā vedanā, vedanāpaccayā taṇhā;\

ghānañca paṭicca gandhe ca uppajjati ghānaviññāṇaṃ, tiṇṇaṃ saṅgati phasso, phassapaccayā vedanā, vedanāpaccayā taṇhā;\

jivhañca paṭicca rase ca uppajjati jivhāviññāṇaṃ, tiṇṇaṃ saṅgati phasso, phassapaccayā vedanā, vedanāpaccayā taṇhā;\

kāyañca paṭicca phoṭṭhabbe ca uppajjati kāyaviññāṇaṃ, tiṇṇaṃ saṅgati phasso, phassapaccayā vedanā, vedanāpaccayā taṇhā;\

manañca paṭicca dhamme ca uppajjati manoviññāṇaṃ, tiṇṇaṃ saṅgati phasso, phassapaccayā vedanā, vedanāpaccayā taṇhā.\\\

‘cha taṇhākāyā veditabbā’ti — iti yaṃ taṃ vuttaṃ, idametaṃ paṭicca vuttaṃ. idaṃ chaṭṭhaṃ chakkaṃ.\\\

422.1. ‘cakkhu attā’ti yo vadeyya taṃ na upapajjati. cakkhussa uppādopi vayopi paññāyati. yassa kho pana uppādopi vayopi paññāyati, ‘attā me uppajjati ca veti cā’ti iccassa evamāgataṃ hoti. tasmā taṃ na upapajjati — ‘cakkhu attā’ti yo vadeyya. iti cakkhu anattā.\\\

‘rūpā attā’ti yo vadeyya taṃ na upapajjati. rūpānaṃ uppādopi vayopi paññāyati. yassa kho pana uppādopi vayopi paññāyati, ‘attā me uppajjati ca veti cā’ti iccassa evamāgataṃ hoti. tasmā taṃ na upapajjati — ‘rūpā attā’ti yo vadeyya. iti cakkhu anattā, rūpā anattā.\\\

‘cakkhuviññāṇaṃ attā’ti yo vadeyya taṃ na upapajjati. cakkhuviññāṇassa uppādopi vayopi paññāyati. yassa kho pana uppādopi vayopi paññāyati, ‘attā me uppajjati ca veti cā’ti iccassa evamāgataṃ hoti. tasmā taṃ na upapajjati — ‘cakkhuviññāṇaṃ attā’ti yo vadeyya. iti cakkhu anattā, rūpā anattā, cakkhuviññāṇaṃ anattā.\\\

‘cakkhusamphasso attā’ti yo vadeyya taṃ na upapajjati. cakkhusamphassassa uppādopi vayopi paññāyati. yassa kho pana uppādopi vayopi paññāyati, ‘attā me uppajjati ca veti cā’ti iccassa evamāgataṃ hoti. tasmā taṃ na upapajjati — ‘cakkhusamphasso attā’ti yo vadeyya. iti cakkhu anattā, rūpā anattā, cakkhuviññāṇaṃ anattā, cakkhusamphasso anattā.\\\

‘vedanā attā’ti yo vadeyya taṃ na upapajjati. vedanāya uppādopi vayopi paññāyati. yassa kho pana uppādopi vayopi paññāyati, ‘attā me uppajjati ca veti cā’ti iccassa evamāgataṃ hoti. tasmā taṃ na upapajjati — ‘vedanā attā’ti yo vadeyya. iti cakkhu anattā, rūpā anattā, cakkhuviññāṇaṃ anattā, cakkhusamphasso anattā, vedanā anattā.\\\

‘taṇhā attā’ti yo vadeyya taṃ na upapajjati. taṇhāya uppādopi vayopi paññāyati. yassa kho pana uppādopi vayopi paññāyati, ‘attā me uppajjati ca veti cā’ti iccassa evamāgataṃ hoti. tasmā taṃ na upapajjati — ‘taṇhā attā’ti yo vadeyya. iti cakkhu anattā, rūpā anattā, cakkhuviññāṇaṃ anattā, cakkhusamphasso anattā, vedanā anattā, taṇhā anattā.\\\

423.2 ‘sotaṃ attā’ti yo vadeyya taṃ na upapajjati. sotassa uppādopi vayopi paññāyati. yassa kho pana uppādopi vayopi paññāyati, ‘attā me uppajjati ca veti cā’ti iccassa evamāgataṃ hoti. tasmā taṃ na upapajjati — ‘sotaṁ attā’ti yo vadeyya. iti sotaṁ anattā.\\\

‘saddā anattā’ti yo vadeyya taṃ na upapajjati. saddānaṃ uppādopi vayopi paññāyati. yassa kho pana uppādopi vayopi paññāyati, ‘attā me uppajjati ca veti cā’ti iccassa evamāgataṃ hoti. tasmā taṃ na upapajjati — ‘saddā attā’ti yo vadeyya. iti sotaṁ anattā, saddā anattā.\\\

‘sotaviññāṇaṃ attā’ti yo vadeyya taṃ na upapajjati. sotaviññāṇassa uppādopi vayopi paññāyati. yassa kho pana uppādopi vayopi paññāyati, ‘attā me uppajjati ca veti cā’ti iccassa evamāgataṃ hoti. tasmā taṃ na upapajjati — ‘sotaviññāṇaṃ attā’ti yo vadeyya. iti sotaṁ anattā, saddā anattā, sotaviññāṇaṃ anattā.\\\

‘sotasamphasso attā’ti yo vadeyya taṃ na upapajjati. sotasamphassassa uppādopi vayopi paññāyati. yassa kho pana uppādopi vayopi paññāyati, ‘attā me uppajjati ca veti cā’ti iccassa evamāgataṃ hoti. tasmā taṃ na upapajjati — ‘sotasamphasso attā’ti yo vadeyya. iti sotaṁ anattā, saddā anattā, sotaviññāṇaṃ anattā, sotasamphasso anattā.\\\

‘vedanā attā’ti yo vadeyya taṃ na upapajjati. vedanāya uppādopi vayopi paññāyati. yassa kho pana uppādopi vayopi paññāyati, ‘attā me uppajjati ca veti cā’ti iccassa evamāgataṃ hoti. tasmā taṃ na upapajjati — ‘vedanā attā’ti yo vadeyya. iti sotaṁ anattā, saddā anattā, sotaviññāṇaṃ anattā, sotasamphasso anattā, vedanā anattā.\\\

‘taṇhā attā’ti yo vadeyya taṃ na upapajjati. taṇhāya uppādopi vayopi paññāyati. yassa kho pana uppādopi vayopi paññāyati, ‘attā me uppajjati ca veti cā’ti iccassa evamāgataṃ hoti. tasmā taṃ na upapajjati — ‘taṇhā attā’ti yo vadeyya. iti sotaṁ anattā, saddā anattā, sotaviññāṇaṃ anattā, sotasamphasso anattā, vedanā anattā, taṇhā anattā.\\\

3. ‘ghānaṃ attā’ti yo vadeyya taṃ na upapajjati. ghānassa uppādopi vayopi paññāyati. yassa kho pana uppādopi vayopi paññāyati, ‘attā me uppajjati ca veti cā’ti iccassa evamāgataṃ hoti. tasmā taṃ na upapajjati — ‘ghānaṃ attā’ti yo vadeyya. iti ghānaṃ anattā.\\\

‘gandhā attā’ti yo vadeyya taṃ na upapajjati. gandhānaṃ uppādopi vayopi paññāyati. yassa kho pana uppādopi vayopi paññāyati, ‘attā me uppajjati ca veti cā’ti iccassa evamāgataṃ hoti. tasmā taṃ na upapajjati — ‘gandhā attā’ti yo vadeyya. iti ghānaṃ anattā, gandhā anattā.\\\

‘ghānaviññāṇaṃ attā’ti yo vadeyya taṃ na upapajjati. ghānaviññāṇassa uppādopi vayopi paññāyati. yassa kho pana uppādopi vayopi paññāyati, ‘attā me uppajjati ca veti cā’ti iccassa evamāgataṃ hoti. tasmā taṃ na upapajjati — ‘ghānaviññāṇaṃ attā’ti yo vadeyya. iti ghānaṃ anattā, gandhā anattā, ghānaviññāṇaṃ anattā.\\\

‘ghānasamphasso attā’ti yo vadeyya taṃ na upapajjati. ghānasamphassassa uppādopi vayopi paññāyati. yassa kho pana uppādopi vayopi paññāyati, ‘attā me uppajjati ca veti cā’ti iccassa evamāgataṃ hoti. tasmā taṃ na upapajjati — ‘ghānasamphasso attā’ti yo vadeyya. iti ghānaṃ anattā, gandhā anattā, ghānaviññāṇaṃ anattā, ghānasamphasso anattā.\\\

‘vedanā attā’ti yo vadeyya taṃ na upapajjati. vedanāya uppādopi vayopi paññāyati. yassa kho pana uppādopi vayopi paññāyati, ‘attā me uppajjati ca veti cā’ti iccassa evamāgataṃ hoti. tasmā taṃ na upapajjati — ‘vedanā attā’ti yo vadeyya. iti ghānaṃ anattā, gandhā anattā, ghānaviññāṇaṃ anattā, ghānasamphasso anattā, vedanā anattā.\\\

‘taṇhā attā’ti yo vadeyya taṃ na upapajjati. taṇhāya uppādopi vayopi paññāyati. yassa kho pana uppādopi vayopi paññāyati, ‘attā me uppajjati ca veti cā’ti iccassa evamāgataṃ hoti. tasmā taṃ na upapajjati — ‘taṇhā attā’ti yo vadeyya. iti ghānaṃ anattā, gandhā anattā, ghānaviññāṇaṃ anattā, ghānasamphasso anattā, vedanā anattā, taṇhā anattā.\\\

4. ‘jivhā attā’ti yo vadeyya taṃ na upapajjati. jivhāya uppādopi vayopi paññāyati. yassa kho pana uppādopi vayopi paññāyati, ‘attā me uppajjati ca veti cā’ti iccassa evamāgataṃ hoti. tasmā taṃ na upapajjati — ‘jivhā attā’ti yo vadeyya. iti jivhā anattā.\\\

‘rasā attā’ti yo vadeyya taṃ na upapajjati. rasānaṃ uppādopi vayopi paññāyati. yassa kho pana uppādopi vayopi paññāyati, ‘attā me uppajjati ca veti cā’ti iccassa evamāgataṃ hoti. tasmā taṃ na upapajjati — ‘rasā attā’ti yo vadeyya. iti jivhā anattā, rasā anattā.\\\

‘jivhāviññāṇaṃ attā’ti yo vadeyya taṃ na upapajjati. jivhāviññāṇassa uppādopi vayopi paññāyati. yassa kho pana uppādopi vayopi paññāyati, ‘attā me uppajjati ca veti cā’ti iccassa evamāgataṃ hoti. tasmā taṃ na upapajjati — ‘jivhāviññāṇaṃ attā’ti yo vadeyya. iti jivhā anattā, rasā anattā, jivhāviññāṇaṃ anattā.\\\

‘jivhāsamphasso attā’ti yo vadeyya taṃ na upapajjati. jivhāsamphassassa uppādopi vayopi paññāyati. yassa kho pana uppādopi vayopi paññāyati, ‘attā me uppajjati ca veti cā’ti iccassa evamāgataṃ hoti. tasmā taṃ na upapajjati — ‘jivhāsamphasso attā’ti yo vadeyya. iti jivhā anattā, rasā anattā, jivhāviññāṇaṃ anattā, jivhāsamphasso anattā.\\\

‘vedanā attā’ti yo vadeyya taṃ na upapajjati. vedanāya uppādopi vayopi paññāyati. yassa kho pana uppādopi vayopi paññāyati, ‘attā me uppajjati ca veti cā’ti iccassa evamāgataṃ hoti. tasmā taṃ na upapajjati — ‘vedanā attā’ti yo vadeyya. iti jivhā anattā, rasā anattā, jivhāviññāṇaṃ anattā, jivhāsamphasso anattā, vedanā anattā.\\\

‘taṇhā attā’ti yo vadeyya taṃ na upapajjati. taṇhāya uppādopi vayopi paññāyati. yassa kho pana uppādopi vayopi paññāyati, ‘attā me uppajjati ca veti cā’ti iccassa evamāgataṃ hoti. tasmā taṃ na upapajjati — ‘taṇhā attā’ti yo vadeyya. iti jivhā anattā, rasā anattā, jivhāviññāṇaṃ anattā, jivhāsamphasso anattā, vedanā anattā, taṇhā anattā.\\\

5. ‘kāyo attā’ti yo vadeyya taṃ na upapajjati. kāyassa uppādopi vayopi paññāyati. yassa kho pana uppādopi vayopi paññāyati, ‘attā me uppajjati ca veti cā’ti iccassa evamāgataṃ hoti. tasmā taṃ na upapajjati — ‘kāyo attā’ti yo vadeyya. iti kāyo anattā.\\\

‘phoṭṭhabbā attā’ti yo vadeyya taṃ na upapajjati. phoṭṭhabbānaṃ uppādopi vayopi paññāyati. yassa kho pana uppādopi vayopi paññāyati, ‘attā me uppajjati ca veti cā’ti iccassa evamāgataṃ hoti. tasmā taṃ na upapajjati — ‘phoṭṭhabbā attā’ti yo vadeyya. iti kāyo anattā, phoṭṭhabbā anattā.\\\

‘kāyaviññāṇaṃ attā’ti yo vadeyya taṃ na upapajjati. kāyaviññāṇassa uppādopi vayopi paññāyati. yassa kho pana uppādopi vayopi paññāyati, ‘attā me uppajjati ca veti cā’ti iccassa evamāgataṃ hoti. tasmā taṃ na upapajjati — ‘kāyaviññāṇaṃ attā’ti yo vadeyya. iti kāyo anattā, phoṭṭhabbā anattā, kāyaviññāṇaṃ anattā.\\\

‘kāyasamphasso attā’ti yo vadeyya taṃ na upapajjati. kāyasamphassassa uppādopi vayopi paññāyati. yassa kho pana uppādopi vayopi paññāyati, ‘attā me uppajjati ca veti cā’ti iccassa evamāgataṃ hoti. tasmā taṃ na upapajjati — ‘kāyasamphasso attā’ti yo vadeyya. iti kāyo anattā, phoṭṭhabbā anattā, kāyaviññāṇaṃ anattā, kāyasamphasso anattā.\\\

‘vedanā attā’ti yo vadeyya taṃ na upapajjati. vedanāya uppādopi vayopi paññāyati. yassa kho pana uppādopi vayopi paññāyati, ‘attā me uppajjati ca veti cā’ti iccassa evamāgataṃ hoti. tasmā taṃ na upapajjati — ‘vedanā attā’ti yo vadeyya. iti kāyo anattā, phoṭṭhabbā anattā, kāyaviññāṇaṃ anattā, kāyasamphasso anattā, vedanā anattā.\\\

‘taṇhā attā’ti yo vadeyya taṃ na upapajjati. taṇhāya uppādopi vayopi paññāyati. yassa kho pana uppādopi vayopi paññāyati, ‘attā me uppajjati ca veti cā’ti iccassa evamāgataṃ hoti. tasmā taṃ na upapajjati — ‘taṇhā attā’ti yo vadeyya. iti kāyo anattā, phoṭṭhabbā anattā, kāyaviññāṇaṃ anattā, kāyasamphasso anattā, vedanā anattā, taṇhā anattā.\\\

6. ‘mano attā’ti yo vadeyya taṃ na upapajjati. manassa uppādopi vayopi paññāyati. yassa kho pana uppādopi vayopi paññāyati, ‘attā me uppajjati ca veti cā’ti iccassa evamāgataṃ hoti. tasmā taṃ na upapajjati — ‘mano attā’ti yo vadeyya. iti mano anattā.\\\

‘dhammā attā’ti yo vadeyya taṃ na upapajjati. dhammānaṃ uppādopi vayopi paññāyati. yassa kho pana uppādopi vayopi paññāyati, ‘attā me uppajjati ca veti cā’ti iccassa evamāgataṃ hoti. tasmā taṃ na upapajjati — ‘dhammā attā’ti yo vadeyya. iti mano anattā, dhammā anattā.\\\

‘manoviññāṇaṃ attā’ti yo vadeyya taṃ na upapajjati. manoviññāṇassa uppādopi vayopi paññāyati. yassa kho pana uppādopi vayopi paññāyati, ‘attā me uppajjati ca veti cā’ti iccassa evamāgataṃ hoti. tasmā taṃ na upapajjati — ‘manoviññāṇaṃ attā’ti yo vadeyya. iti mano anattā, dhammā anattā, manoviññāṇaṃ anattā.\\\

‘manosamphasso attā’ti yo vadeyya taṃ na upapajjati. manosamphassassa uppādopi vayopi paññāyati. yassa kho pana uppādopi vayopi paññāyati, ‘attā me uppajjati ca veti cā’ti iccassa evamāgataṃ hoti. tasmā taṃ na upapajjati — ‘manosamphasso attā’ti yo vadeyya. iti mano anattā, dhammā anattā, manoviññāṇaṃ anattā, manosamphasso anattā.\\\

‘vedanā attā’ti yo vadeyya taṃ na upapajjati. vedanāya uppādopi vayopi paññāyati. yassa kho pana uppādopi vayopi paññāyati, ‘attā me uppajjati ca veti cā’ti iccassa evamāgataṃ hoti. tasmā taṃ na upapajjati — ‘vedanā attā’ti yo vadeyya. iti mano anattā, dhammā anattā, manoviññāṇaṃ anattā, manosamphasso anattā, vedanā anattā.\\\

‘taṇhā attā’ti yo vadeyya taṃ na upapajjati. taṇhāya uppādopi vayopi paññāyati. yassa kho pana uppādopi vayopi paññāyati, ‘attā me uppajjati ca veti cā’ti iccassa evamāgataṃ hoti. tasmā taṃ na upapajjati — ‘taṇhā attā’ti yo vadeyya. iti mano anattā, dhammā anattā, manoviññāṇaṃ anattā, manosamphasso anattā, vedanā anattā, taṇhā anattā.\\\

424. “ayaṃ kho pana, bhikkhave, sakkāyasamudayagāminī paṭipadā —\\\

{\small{cakkhuṃ ‘etaṃ mama, esohamasmi, eso me attā’ti samanupassati;\

rūpe ‘etaṃ mama, esohamasmi, eso me attā’ti samanupassati;\

cakkhuviññāṇaṃ ‘etaṃ mama, esohamasmi, eso me attā’ti samanupassati;\

cakkhusamphassaṃ ‘etaṃ mama, esohamasmi, eso me attā’ti samanupassati;\

vedanaṃ ‘etaṃ mama, esohamasmi, eso me attā’ti samanupassati;\

taṇhaṃ ‘etaṃ mama, esohamasmi, eso me attā’ti samanupassati;\\\

sotaṃ ‘etaṃ mama, esohamasmi, eso me attā’ti samanupassati;\

saddā ‘etaṃ mama, esohamasmi, eso me attā’ti samanupassati;\

sotaviññāṇaṃ ‘etaṃ mama, esohamasmi, eso me attā’ti samanupassati;\

sotasamphassaṃ ‘etaṃ mama, esohamasmi, eso me attā’ti samanupassati;\

vedanaṃ ‘etaṃ mama, esohamasmi, eso me attā’ti samanupassati;\

taṇhaṃ ‘etaṃ mama, esohamasmi, eso me attā’ti samanupassati;\\\

ghānaṃ ‘etaṃ mama, esohamasmi, eso me attā’ti samanupassati;\

gandā ‘etaṃ mama, esohamasmi, eso me attā’ti samanupassati;\

ghānaviññāṇaṃ ‘etaṃ mama, esohamasmi, eso me attā’ti samanupassati;\

ghānasamphassaṃ ‘etaṃ mama, esohamasmi, eso me attā’ti samanupassati;\

vedanaṃ ‘etaṃ mama, esohamasmi, eso me attā’ti samanupassati;\

taṇhaṃ ‘etaṃ mama, esohamasmi, eso me attā’ti samanupassati;\\\

jivhaṃ ‘etaṃ mama, esohamasmi, eso me attā’ti samanupassati;\

rasā ‘etaṃ mama, esohamasmi, eso me attā’ti samanupassati;\

jivhāviññāṇaṃ ‘etaṃ mama, esohamasmi, eso me attā’ti samanupassati;\

jivhāsamphassaṃ ‘etaṃ mama, esohamasmi, eso me attā’ti samanupassati;\

vedanaṃ ‘etaṃ mama, esohamasmi, eso me attā’ti samanupassati;\

taṇhaṃ ‘etaṃ mama, esohamasmi, eso me attā’ti samanupassati;\\\

kāyaṃ ‘etaṃ mama, esohamasmi, eso me attā’ti samanupassati;\

phoṭṭhabbā ‘etaṃ mama, esohamasmi, eso me attā’ti samanupassati;\

kāyaviññāṇaṃ ‘etaṃ mama, esohamasmi, eso me attā’ti samanupassati;\

kāyasamphassaṃ ‘etaṃ mama, esohamasmi, eso me attā’ti samanupassati;\

vedanaṃ ‘etaṃ mama, esohamasmi, eso me attā’ti samanupassati;\

taṇhaṃ ‘etaṃ mama, esohamasmi, eso me attā’ti samanupassati;\\\

manaṃ ‘etaṃ mama, esohamasmi, eso me attā’ti samanupassati,\

dhamme ‘etaṃ mama, esohamasmi, eso me attā’ti samanupassati,\

manoviññāṇaṃ ‘etaṃ mama, esohamasmi, eso me attā’ti samanupassati,\

manosamphassaṃ ‘etaṃ mama, esohamasmi, eso me attā’ti samanupassati,\

vedanaṃ ‘etaṃ mama, esohamasmi, eso me attā’ti samanupassati,\

taṇhaṃ ‘etaṃ mama, esohamasmi, eso me attā’ti samanupassati.}}\\\

“ayaṃ kho pana, bhikkhave, sakkāyanirodhagāminī paṭipadā —\\\

{\small{cakkhuṃ ‘netaṃ mama, nesohamasmi, na meso attā’ti samanupassati.\

rūpe ‘netaṃ mama, nesohamasmi, na meso attā’ti samanupassati.\

cakkhuviññāṇaṃ ‘netaṃ mama, nesohamasmi, na meso attā’ti samanupassati.\

cakkhusamphassaṃ ‘netaṃ mama, nesohamasmi, na meso attā’ti samanupassati.\

vedanaṃ ‘netaṃ mama, nesohamasmi, na meso attā’ti samanupassati.\

taṇhaṃ ‘netaṃ mama, nesohamasmi, na meso attā’ti samanupassati.\\\

sotaṃ ‘netaṃ mama, nesohamasmi, na meso attā’ti samanupassati.\

saddā ‘netaṃ mama, nesohamasmi, na meso attā’ti samanupassati.\

sotaviññāṇaṃ ‘netaṃ mama, nesohamasmi, na meso attā’ti samanupassati.\

sotasamphassaṃ ‘netaṃ mama, nesohamasmi, na meso attā’ti samanupassati.\

vedanaṃ ‘netaṃ mama, nesohamasmi, na meso attā’ti samanupassati.\

taṇhaṃ ‘netaṃ mama, nesohamasmi, na meso attā’ti samanupassati.\\\

ghānaṃ ‘netaṃ mama, nesohamasmi, na meso attā’ti samanupassati.\

gandā ‘netaṃ mama, nesohamasmi, na meso attā’ti samanupassati.\

ghānaviññāṇaṃ ‘netaṃ mama, nesohamasmi, na meso attā’ti samanupassati.\

ghānasamphassaṃ ‘netaṃ mama, nesohamasmi, na meso attā’ti samanupassati.\

vedanaṃ ‘netaṃ mama, nesohamasmi, na meso attā’ti samanupassati.\

taṇhaṃ ‘netaṃ mama, nesohamasmi, na meso attā’ti samanupassati.\\\

jivhaṃ ‘netaṃ mama, nesohamasmi, na meso attā’ti samanupassati.\

rasā ‘netaṃ mama, nesohamasmi, na meso attā’ti samanupassati.\

jivhāviññāṇaṃ ‘netaṃ mama, nesohamasmi, na meso attā’ti samanupassati.\

jivhāsamphassaṃ ‘netaṃ mama, nesohamasmi, na meso attā’ti samanupassati.\

vedanaṃ ‘netaṃ mama, nesohamasmi, na meso attā’ti samanupassati.\

taṇhaṃ ‘netaṃ mama, nesohamasmi, na meso attā’ti samanupassati.\\\

kāyaṃ ‘netaṃ mama, nesohamasmi, na meso attā’ti samanupassati.\

phoṭṭhabbā ‘netaṃ mama, nesohamasmi, na meso attā’ti samanupassati.\

kāyaviññāṇaṃ ‘netaṃ mama, nesohamasmi, na meso attā’ti samanupassati.\

kāyasamphassaṃ ‘netaṃ mama, nesohamasmi, na meso attā’ti samanupassati.\

vedanaṃ ‘netaṃ mama, nesohamasmi, na meso attā’ti samanupassati.\

taṇhaṃ ‘netaṃ mama, nesohamasmi, na meso attā’ti samanupassati.\\\

manaṃ ‘netaṃ mama, nesohamasmi, na meso attā’ti samanupassati.\

dhamme ‘netaṃ mama, nesohamasmi, na meso attā’ti samanupassati.\

manoviññāṇaṃ ‘netaṃ mama, nesohamasmi, na meso attā’ti samanupassati.\

manosamphassaṃ ‘netaṃ mama, nesohamasmi, na meso attā’ti samanupassati.\

vedanaṃ ‘netaṃ mama, nesohamasmi, na meso attā’ti samanupassati.\

taṇhaṃ ‘netaṃ mama, nesohamasmi, na meso attā’ti samanupassati.}}\\\


425. “cakkhuñca, bhikkhave, paṭicca rūpe ca uppajjati cakkhuviññāṇaṃ,...\\

“sotañca, bhikkhave, paṭicca sadde ca uppajjati sotaviññāṇaṃ...\\

ghānañca, bhikkhave, paṭicca gandhe ca uppajjati ghānaviññāṇaṃ...\\

jivhañca, bhikkhave, paṭicca rase ca uppajjati jivhāviññāṇaṃ...\\

kāyañca, bhikkhave, paṭicca phoṭṭhabbe ca uppajjati kāyaviññāṇaṃ...\\

manañca, bhikkhave, paṭicca dhamme ca uppajjati manoviññāṇaṃ...\\\

\begingl
tiṇṇaṃ[three] saṅgati[meeting] phasso,[contact ] phassapaccayā[contact support] uppajjati[arises] vedayitaṃ[feeling] sukhaṃ[] vā[] dukkhaṃ[] vā[] adukkhamasukhaṃ[] vā.[] so[] sukhāya[] vedanāya[sensation] phuṭṭho[touched ] samāno[similar] abhinandati[rejoices at] abhivadati[declares] ajjhosāya[holden] tiṭṭhati.[remain] tassa[that to him] rāgānusayo[lust proclivity] anuseti.[lies dormant] dukkhāya[] vedanāya[sensation] phuṭṭho[touched ] samāno[similar] socati[mourn] kilamati[troubled] paridevati[laments] urattāḷiṃ[beat breast] kandati[cry] sammohaṃ[confusion] āpajjati.[undergo] tassa[that to him] paṭighānusayo[anger proclivity] anuseti.[lies dormant] adukkhamasukhāya[] vedanāya[sensation] phuṭṭho[touched ] samāno[similar] tassā[that to him] vedanāya[sensation] samudayañca[origin] atthaṅgamañca[disappearance ] assādañca[gratification] ādīnavañca[disadvantage] nissaraṇañca[escape ] yathābhūtaṃ[in reality] nappajānāti.[not clearly know] tassa[that to him] avijjānusayo[ignore proclivity] anuseti.[lies dormant] so[] vata,[] bhikkhave,[] sukhāya[] vedanāya[sensation] rāgānusayaṃ[lust proclivity] appahāya[not abandon] dukkhāya[] vedanāya[sensation] paṭighānusayaṃ[anger proclivity] appaṭivinodetvā[not dispelled] adukkhamasukhāya[] vedanāya[sensation] avijjānusayaṃ[ignore proclivity] asamūhanitvā[not remove] avijjaṃ[ignorance] appahāya[not abandon] vijjaṃ[higher knowledge] anuppādetvā[not give rise to] diṭṭheva[seen ] dhamme[] dukkhassantakaro[own's own suffering] bhavissatīti[becomes] —[] netaṃ[not this] ṭhānaṃ[reason] vijjati.[to be found] 
\endgl
\pagebreak\\
426. “cakkhuñca, bhikkhave, paṭicca rūpe ca uppajjati cakkhuviññāṇaṃ,...\\
“sotañca, bhikkhave, paṭicca sadde ca uppajjati sotaviññāṇaṃ...\\

“ghānañca, bhikkhave, paṭicca gandhe ca uppajjati ghānaviññāṇaṃ...\\

“jivhañca, bhikkhave, paṭicca rase ca uppajjati jivhāviññāṇaṃ...\\

“kāyañca, bhikkhave, paṭicca phoṭṭhabbe ca uppajjati kāyaviññāṇaṃ...\\

“manañca, bhikkhave, paṭicca dhamme ca uppajjati manoviññāṇaṃ...\\\

...
\begingl
tiṇṇaṃ[three] saṅgati[meeting] phasso,[contact ] phassapaccayā[contact support] uppajjati[arises] vedayitaṃ[feeling] sukhaṃ[] vā[] dukkhaṃ[] vā[] adukkhamasukhaṃ[] vā.[] so[] sukhāya[] vedanāya[sensation] phuṭṭho[touched ] samāno[similar] nābhinandati[not rejoices at] nābhivadati[not declare] nājjhosāya[not holden] tiṭṭhati.[remain] tassa[that to him] rāgānusayo[lust proclivity] nānuseti.[not dormant] dukkhāya[] vedanāya[sensation] phuṭṭho[touched ] samāno[similar] na[] socati[mourn] na[] kilamati[troubled] paridevati[laments] na[] urattāḷiṃ[beat breast] kandati[cry] na[] sammohaṃ[confusion] āpajjati.[undergo] tassa[that to him] paṭighānusayo[anger proclivity] nānuseti.[not dormant] adukkhamasukhāya[] vedanāya[sensation] phuṭṭho[touched ] samāno[similar] tassā[that to him] vedanāya[sensation] samudayañca[origin] atthaṅgamañca[disappearance ] assādañca[gratification] ādīnavañca[disadvantage] nissaraṇañca[escape ] yathābhūtaṃ[in reality] pajānāti.[knows clearly ] tassa[that to him] avijjānusayo[ignore proclivity] nānuseti.[not dormant] so[] vata,[] bhikkhave,[] sukhāya[] vedanāya[sensation] rāgānusayaṃ[lust proclivity] pahāya[having renounced] dukkhāya[] vedanāya[sensation] paṭighānusayaṃ[anger proclivity] paṭivinodetvā[having dispelled] adukkhamasukhāya[] vedanāya[sensation] avijjānusayaṃ[ignore proclivity] samūhanitvā[abolishes] avijjaṃ[ignorance] pahāya[having renounced] vijjaṃ[higher knowledge] uppādetvā[gives rise to ] diṭṭheva[seen ] dhamme[] dukkhassantakaro[own's own suffering] bhavissatīti[becomes] —[] ṭhānametaṃ[this reason] vijjati.[to be found]
\endgl
\pagebreak

427. “evaṃ passaṃ, bhikkhave, sutavā ariyasāvako...\

cakkhusmiṃ nibbindati, rūpesu nibbindati, cakkhuviññāṇe nibbindati, cakkhusamphasse nibbindati, vedanāya nibbindati, taṇhāya nibbindati.\

sotasmiṃ nibbindati, saddesu nibbindati, sotaviññāṇe nibbindati, sotasamphasse nibbindati, vedanāya nibbindati, taṇhāya nibbindati.\

ghānasmiṃ nibbindati, gandhesu nibbindati, ghānaviññāṇe nibbindati, ghānasamphasse nibbindati, vedanāya nibbindati, taṇhāya nibbindati.\

jivhāya nibbindati, rasesu nibbindati, jivhāviññāṇe nibbindati, jivhāsamphasse nibbindati, vedanāya nibbindati, taṇhāya nibbindati.\

kāyasmiṃ nibbindati, phoṭṭhabbesu nibbindati, kāyaviññāṇe nibbindati, kāyasamphasse nibbindati, vedanāya nibbindati, taṇhāya nibbindati.\

manasmiṃ nibbindati, dhammesu nibbindati, manoviññāṇe nibbindati, manosamphasse nibbindati, vedanāya nibbindati, taṇhāya nibbindati. \

nibbindaṃ virajjati, virāgā vimuccati. vimuttasmiṃ vimuttamiti ñāṇaṃ hoti. ‘khīṇā jāti, vusitaṃ brahmacariyaṃ, kataṃ karaṇīyaṃ, nāparaṃ itthattāyā’ti pajānātī”ti.\\\

idamavoca bhagavā. attamanā te bhikkhū bhagavato bhāsitaṃ abhinandunti. imasmiṃ kho pana veyyākaraṇasmiṃ bhaññamāne saṭṭhimattānaṃ bhikkhūnaṃ anupādāya āsavehi cittāni vimucciṃsūti.\\[.5cm]

\centering\textbf{chachakkasuttaṃ niṭṭhitaṃ chaṭṭhaṃ.}

\end{document}