\documentclass[11pt]{article}
\usepackage[margin=1in]{geometry}
\usepackage{titlesec}
\usepackage{polyglossia}
\defaultfontfeatures{Ligatures=TeX}
\setmainfont{Arial}
\usepackage{expex}
% Uncomment the following line if you do not want extra space before the free translation and the glosses, or other customisations to the examples.
\lingset{aboveglftskip=-.2ex,interpartskip=\baselineskip,everyglb=\footnotesize}

\title{Some \texttt{expex} Linguistic Examples}
\author{LianTze Lim}

\begin{document}
\textbf{Majjhima Nikāya, uparipaṇṇāsapāḷi, 3. suññatavaggo, 1. cūḷasuññatasuttaṃ (MN 121)}\\[.3cm]

Thus I heard: On one occasion the Blessed One was living at Savatthi in the Eastern Park, the Palace of Migara's Mother. Then when it was evening, the venerable Ananda rose from retreat, and he went to the Blessed One, and after paying homage to him, he sat down at one side. When he had done so, he said to the Blessed One 'Venerable sir, once the Blessed One was living in the Sakyan country. There is a town of the Sakyans called Nagaraka; there I heard and learnt this from the Blessed One's own lips: "Now I abide much in the voiding, Ananda." Venerable sir, was this well heard by me, well apprehended, well attended to and well remembered?' 'Certainly, Ananda, that was well heard by you, well apprehended, well attended to and well remembered. As formerly, so now too, I abide much in the void abiding.\\

 'Ananda, just as the Palace of Migara's Mother is void of elephants, cattle, horses and mares, void of gold and silver, void of the forgathering of women and men, and there is (present) only this non-voidness, that is to say, the single state (of non-voidness) dependent on (the presence of) the community of bhikkhus; so too, without giving attention to perception of village, without giving attention to perception of man, a bhikkhu gives attention to the single state (of non-voidness) dependent on (the presence of) perception of forest. His mind enters into that perception of forest and acquires confidence, steadiness and decision. He understands thus: "Disturbances that would be present dependent on perception of village are not present here, disturbances that would be present on perception of man are not present here, and only this measure of disturbance is present, that is to say, the single state (of non-voidness) dependent on (the presence of) perception of forest." He understands: "This field of perception is void of perception of village." He understands: "This field of perception is void of perception of man.", (and he understands): "There is (present) only this non-voidness, that is to say, the single state (of non-voidness) dependent on (the presence of) perception of forest." So he sees it as void of what is not there, but of what remains there he understands: "There is that still present there." Now this has been for him an alighting upon voidness that accords with what actually is, without perversion of meaning and is pure.\\

 'Again, Ananda, without giving attention to perception of man, without giving attention to perception of forest, a bhikkhu gives attention to the single state (of non-voidness) dependent on (the presence of) perception of earth. His mind enters into that perception of earth and acquires confidence, steadiness and decision. Just as though a bull's hide were freed from folds by stretching it with a hundred pegs, so too, without giving attention to all the ridges and hollows, the river ravines, the tracts of stumps and thorns, the rocky inequalities, on this earth, a bhikkhu gives attention to the single state (of non-voidness) dependent on (the presence of) perception of earth. His mind enters into the perception of earth and acquires confidence, steadiness and decision. He understands thus: "Disturbances that would have been present dependent on perception of man are not present here, disturbances that would be present dependent on perception of forest are not present here, and only this measure of disturbance is present, that is to say, the single state (of non-voidness) dependent on (the presence of) perception of earth." He understands: "This field of perception is void of perception of man." He understands: "This field of perception is void of perception of forest.", (and he understands): "There is (present) only this non-voidness, that is to say, the single state (of non-voidness) dependent on (the presence of) perception of earth." So he sees it as void of what is not there, but of what remains there he understands: "There is that still present there." Now this too has been for him an alighting upon voidness, that accords with what actually is, without perversion of meaning, and is pure.\\

 'Again, Ananda, without giving attention to perception of forest, without giving attention to perception of earth, a bhikkhu gives attention to the single state (on non-voidness) dependent on (the presence of) perception of the base consisting of infinite space. His mind enters into that perception of the base consisting of infinite space and acquires confidence, steadiness and decision. He understands thus: "Disturbances that would be present dependent on perception of forest are not present here, disturbances that would be present on perception of earth are not present here, and only this measure of disturbance is present, that is to say, the single state (of non-voidness) dependent on (the presence of) perception of the base consisting of infinite space." He understands: "This field of perception is void of perception of forest." He understands: "This field of perception is void of perception of earth." , (and he understands:) "There is (present) only this non-voidness, that is to say, the single state (of non-voidness) dependent on (the presence of) perception of the base consisting of infinite space." So he sees it as void of what is not there, but of what remains there he understands: "There is that still present there." Now this too has been for him an alighting upon voidness that accords with what actually is, without perversion of meaning and is pure.\\

 'Again, Ananda, without giving attention to perception of earth, without giving attention to perception of the base consisting of infinite space, a bhikkhu gives attention to the single state (of non-voidness) dependent on (the presence of) perception of the base consisting of infinite consciousness. His mind enters into the perception of the base consisting of infinite consciousness and acquires confidence, steadiness and decision. He understands thus: "Disturbances that would be present dependent on perception of earth are not present here, disturbances that would be present dependent on perception of the base consisting of infinite space are not present here, and only this measure of disturbance is present, that is to say, the single state (of non-voidness) dependent on (the presence of) perception of the base consisting of infinite consciousness." He understands: "This field of perception is void of perception of earth." He understands: "This field of perception is void of perception of the base consisting of infinite space", (and he understands): "There is (present) only this non-voidness, that is to say, the single state (of non-voidness) dependent on (the presence of) perception of the base consisting of infinite consciousness." So he sees it as void of what is not there, but of what remains there he understands: "There is that still present there." Now this too has been for him an alighting upon voidness that accords with what actually is, without perversion of meaning, and is pure.\\

 'Again, Ananda, without giving attention to perception of the base consisting of infinite space, without giving attention to perception of the base consisting of infinite consciousness, a bhikkhu gives attention to the single state (of non-voidness) dependent on (the presence of) perception of the base consisting of nothingness. His mind enters into the perception of the base consisting of nothingness and he acquires confidence, steadiness and decision. He understands thus: "Disturbances that would be present dependent on perception of the base of infinite space are not present here, disturbances that would be present dependent on perception of the base of infinite consciousness are not present here, and only this measure of disturbance is present, that is to say, the single state (of non-voidness) dependent on (the presence of) perception of the base consisting of nothingness." He understands: "This field of perception is void of perception of the base consisting of infinite space.", and he understands: "This field of perception is void of perception of the base consisting of infinite consciousness.", (and he understands): "There is (present) only this non-voidness, that is to say, the single state (of non-voidness) dependent on (the presence of) perception of the base consisting of nothingness." So he sees it as void of what is not there, but of what remains there he understands: "There is that still present there." Now this too has been for him an alighting upon voidness that accords with what is, without perversion of meaning and is pure.\\

 'Again, Ananda, without giving attention to perception of the base consisting of infinite consciousness, without giving attention to perception of the base consisting of nothingness, a bhikkhu gives attention to the single state (of non-voidness) dependent on (the presence of) perception of the base consisting of neither-perception-nor-non-perception. His mind enters into the perception of the base consisting of neither-perception-nor-non-perception and acquires confidence, steadiness and decision. He understands thus: "Disturbances that would be present dependent on the perception of the base consisting of infinite consciousness are not present here, disturbances that would present dependent on the perception of the base consisting of nothingness are not present here, and only this measure of disturbance is present, that is to say, the single state (of non-voidness) dependent on (the presence of) perception of the base consisting of neither-perception-nor-non-perception." He understands: "This field of perception is void of perception of the base consisting of infinite consciousness." He understands: "This field of perception is void of perception of the base consisting of nothingness.", (and he understands): "There is (present) only this non-voidness, that is to say, the single state (of non-voidness) dependent on (the presence of) perception of the base consisting of neither-perception-nor-non-perception." So he sees it as void of what is not there, but of what remains there he understands: "There is that still present there." Now this too has been for him an alighting upon voidness, that accords with what actually is, without perversion of meaning and is pure.\\

 'Again, Ananda, without giving attention to perception of the base consisting of nothingness, without giving attention to perception of the base consisting of neither-perception-nor-non-perception, a bhikkhu gives attention to the single state (of non-voidness) dependent on (the presence of) the signless concentration of mind. His mind enters into the signless concentration of mind and acquires confidence, steadiness and decision. He understands thus: "Disturbances that would be present dependent on the perception of the base consisting of nothingness are not present here, disturbances that would be present dependent on the perception of the base consisting of neither-perception-nor-non-perception are not present here, and only this measure of disturbance is present, that is to say, that (disturbance) which has life as its condition dependent on the presence of this body with its six bases." He understands: "This field of perception is void of perception of the base consisting of nothingness." He understands: "This field of perception is void of perception of the base consisting of neither-perception-nor-non-perception.", (and he understands): "There is (present) only this non-voidness, that is to say, that (non-voidness) with life as its condition dependent on this body with its six bases." So he sees it as void of what is not there, but of what remains there he understands: "There is that still present there." Now this too has been for him an alighting upon voidness, that accords with what actually is, without perversion of meaning and is pure.\\

 'Again, Ananda, without giving attention to perception of the base consisting of nothingness, without giving attention to perception of the base consisting of neither-perception-nor-non-perception, a bhikkhu gives attention to the single state (of non-voidness) dependent on (the presence of) the signless concentration of mind. His mind enters into the signless concentration of mind and acquires confidence, steadiness and decision. He understands thus: "This signless concentration of mind is conditioned and mentally produced." He understands: "Whatever is conditioned and mentally produced is impermanent and liable to cessation." When he knows and sees thus, his mind is liberated from the taint of sensual desire, from the taint of being, from the taint of ignorance. When liberated there comes the knowledge "It is liberated". He understands: "Birth is exhausted, the life divine has been lived out, what was to be done is done, there is no more of this to come."\\

He understands thus: "Disturbances that would be present dependent on the taint of sensual desire are not present here, disturbances that would be present dependent on the taint of being are not present here, disturbances that would be present dependent on the taint of ignorance are not present here, and only this measure of disturbance is present, that is to say, that (non-voidness) with life as its condition dependent on (the presence of) this body with its six bases." He understands: "This field of perception is void of the taint of sensual desire." He understands: "This field of perception is void of the taint of being." He understands: "This field of perception is void of the taint of ignorance.", (and he understands): "There is (present) only this non-voidness, that is to say, that (non-voidness) with life as its condition dependent on (the presence of) this body with its six bases." So he sees it as void of what is not there, but of what remains there he understands: "There is that still present there." Now this has been for him an alighting upon voidness that accords with what actually is, without perversion of meaning, is pure and is unsurpassed by any other.\\

 'Whatever monks or divines in the past have entered upon and abode in a voidness that was purified and unsurpassed by any other, they have all of them entered upon and abode in this voidness that is pure and unsurpassed by any other.
'Whatever monks and divines in the future will enter upon and abide in a voidness that will be purified and unsurpassed by any other, they will all of them enter upon and abide in this voidness that is pure and unsurpassed by any other.
'Whatever monks and divines in the present enter upon and abide in a voidness that is purified and unsurpassed by any other, they all of them will enter upon and abide in this voidness that is pure and unsurpassed by any other.
'Therefore, Ananda, you should train thus: "We will enter upon and abide in the voidness that is pure and unsurpassed by any other."'\\

That is what the Blessed One said. The venerable Ananda was satisfied, and he delighted in the Blessed One's words.
\end{document}