\documentclass[11pt]{article}
\usepackage[margin=45pt,ansibpaper, landscape]{geometry}
\usepackage{titlesec}
\usepackage{paracol}

\usepackage{polyglossia}
\sloppy

\usepackage{fontspec}
\setromanfont[BoldFont={Gentium Basic Bold},ItalicFont={Gentium Italic}]{Gentium}
\setcolumnwidth{310pt/60pt,310pt}




\begin{document}

\begin{paracol}{2}
	\begin{column}
Thus I heard: On one occasion the Blessed One was living at Savatthi in the Eastern Park, the Palace of Migara's Mother. Then when it was evening, the venerable Ananda rose from retreat, and he went to the Blessed One, and after paying homage to him, he sat down at one side.
\switchcolumn
 176. evaṃ me sutaṃ — ekaṃ samayaṃ bhagavā sāvatthiyaṃ viharati pubbārāme migāramātupāsāde. atha kho āyasmā ānando sāyanhasamayaṃ paṭisallānā vuṭṭhito yena bhagavā tenupasaṅkami; upasaṅkamitvā bhagavantaṃ abhivādetvā ekamantaṃ nisīdi.
 \switchcolumn*
When he had done so, he said to the Blessed One 'Venerable sir, once the Blessed One was living in the Sakyan country. There is a town of the Sakyans called Nagaraka; there I heard and learnt this from the Blessed One's own lips: "Now I abide much in the voiding, Ananda." Venerable sir, was this well heard by me, well apprehended, well attended to and well remembered?'
\switchcolumn
ekamantaṃ nisinno kho āyasmā ānando bhagavantaṃ etadavoca — “ekamidaṃ, bhante, samayaṃ bhagavā sakkesu viharati nagarakaṃ nāma sakyānaṃ nigamo. tattha me, bhante, bhagavato sammukhā sutaṃ, sammukhā paṭiggahitaṃ —‘suññatāvihārenāhaṃ, ānanda, etarahi bahulaṃ viharāmī’ti. kacci metaṃ, bhante, sussutaṃ suggahitaṃ sumanasikataṃ sūpadhāritan”ti?
\switchcolumn*
'Certainly, Ananda, that was well heard by you, well apprehended, well attended to and well remembered. As formerly, so now too, I abide much in the void abiding. 'Ananda, just as the Palace of Migara's Mother is void of elephants, cattle, horses and mares, void of gold and silver, void of the forgathering of women and men, and there is (present) only this non-voidness, that is to say, the single state (of non-voidness) dependent on (the presence of) the community of bhikkhus; so too, without giving attention to perception of village, without giving attention to perception of man, a bhikkhu gives attention to the single state (of non-voidness) dependent on (the presence of) perception of forest. His mind enters into that perception of forest and acquires confidence, steadiness and decision. He understands thus: "Disturbances that would be present dependent on perception of village are not present here, disturbances that would be present on perception of man are not present here, and only this measure of disturbance is present, that is to say, the single state (of non-voidness) dependent on (the presence of) perception of forest." He understands: "This field of perception is void of perception of village." He understands: "This field of perception is void of perception of man.", (and he understands): "There is (present) only this non-voidness, that is to say, the single state (of non-voidness) dependent on (the presence of) perception of forest." So he sees it as void of what is not there, but of what remains there he understands: "There is that still present there." Now this has been for him an alighting upon voidness that accords with what actually is, without perversion of meaning and is pure.
\switchcolumn
ekamantaṃ nisinno kho āyasmā ānando bhagavantaṃ etadavoca — “ekamidaṃ, bhante, samayaṃ bhagavā sakkesu viharati nagarakaṃ nāma sakyānaṃ nigamo. tattha me, bhante, bhagavato sammukhā sutaṃ, sammukhā paṭiggahitaṃ —‘suññatāvihārenāhaṃ, ānanda, etarahi bahulaṃ viharāmī’ti. kacci metaṃ, bhante, sussutaṃ suggahitaṃ sumanasikataṃ sūpadhāritan”ti? “taggha te etaṃ, ānanda, sussutaṃ suggahitaṃ sumanasikataṃ sūpadhāritaṃ. pubbepāhaṃ, ānanda, etarahipi suññatāvihārena bahulaṃ viharāmi. seyyathāpi, ānanda, ayaṃ migāramātupāsādo suñño hatthigavassavaḷavena, suñño jātarūparajatena, suñño itthipurisasannipātena atthi cevidaṃ asuññataṃ yadidaṃ — bhikkhusaṅghaṃ paṭicca ekattaṃ; evameva kho, ānanda, bhikkhu amanasikaritvā gāmasaññaṃ, amanasikaritvā manussasaññaṃ, araññasaññaṃ paṭicca manasi karoti ekattaṃ. tassa araññasaññāya cittaṃ pakkhandati pasīdati santiṭṭhati adhimuccati. so evaṃ pajānāti — ‘ye assu darathā gāmasaññaṃ paṭicca tedha na santi, ye assu darathā manussasaññaṃ paṭicca tedha na santi, atthi cevāyaṃ darathamattā yadidaṃ — araññasaññaṃ paṭicca ekattan’ti. so ‘suññamidaṃ saññāgataṃ gāmasaññāyā’ti pajānāti, ‘suññamidaṃ saññāgataṃ manussasaññāyā’ti pajānāti, ‘atthi cevidaṃ asuññataṃ yadidaṃ — araññasaññaṃ paṭicca ekattan’ti. iti yañhi kho tattha na hoti tena taṃ suññaṃ samanupassati, yaṃ pana tattha avasiṭṭhaṃ hoti taṃ ‘santamidaṃ atthī’”ti pajānāti. evampissa esā, ānanda, yathābhuccā avipallatthā parisuddhā suññatāvakkanti bhavati.
\switchcolumn*
 'Again, Ananda, without giving attention to perception of man, without giving attention to perception of forest, a bhikkhu gives attention to the single state (of non-voidness) dependent on (the presence of) perception of earth. His mind enters into that perception of earth and acquires confidence, steadiness and decision. Just as though a bull's hide were freed from folds by stretching it with a hundred pegs, so too, without giving attention to all the ridges and hollows, the river ravines, the tracts of stumps and thorns, the rocky inequalities, on this earth, a bhikkhu gives attention to the single state (of non-voidness) dependent on (the presence of) perception of earth. His mind enters into the perception of earth and acquires confidence, steadiness and decision. He understands thus: "Disturbances that would have been present dependent on perception of man are not present here, disturbances that would be present dependent on perception of forest are not present here, and only this measure of disturbance is present, that is to say, the single state (of non-voidness) dependent on (the presence of) perception of earth." He understands: "This field of perception is void of perception of man." He understands: "This field of perception is void of perception of forest.", (and he understands): "There is (present) only this non-voidness, that is to say, the single state (of non-voidness) dependent on (the presence of) perception of earth." So he sees it as void of what is not there, but of what remains there he understands: "There is that still present there." Now this too has been for him an alighting upon voidness, that accords with what actually is, without perversion of meaning, and is pure.
 \switchcolumn
 177. “puna caparaṃ, ānanda, bhikkhu amanasikaritvā manussasaññaṃ, amanasikaritvā araññasaññaṃ, pathavīsaññaṃ paṭicca manasi karoti ekattaṃ. tassa pathavīsaññāya cittaṃ pakkhandati pasīdati santiṭṭhati adhimuccati. seyyathāpi, ānanda, āsabhacammaṃ saṅkusatena suvihataṃ vigatavalikaṃ; evameva kho, ānanda, bhikkhu yaṃ imissā pathaviyā ukkūlavikkūlaṃ nadīviduggaṃ khāṇukaṇṭakaṭṭhānaṃ pabbatavisamaṃ taṃ sabbaṃ amanasikaritvā pathavīsaññaṃ paṭicca manasi karoti ekattaṃ. tassa pathavīsaññāya cittaṃ pakkhandati pasīdati santiṭṭhati adhimuccati. so evaṃ pajānāti — ‘ye assu darathā manussasaññaṃ paṭicca tedha na santi, ye assu darathā araññasaññaṃ paṭicca tedha na santi, atthi cevāyaṃ darathamattā yadidaṃ — pathavīsaññaṃ paṭicca ekattan’ti. so ‘suññamidaṃ saññāgataṃ manussasaññāyā’ti pajānāti, ‘suññamidaṃ saññāgataṃ araññasaññāyā’ti pajānāti, ‘atthi cevidaṃ asuññataṃ yadidaṃ — pathavīsaññaṃ paṭicca ekattan’ti. iti yañhi kho tattha na hoti tena taṃ suññaṃ samanupassati, yaṃ pana tattha avasiṭṭhaṃ hoti taṃ ‘santamidaṃ atthī’ti pajānāti. evampissa esā, ānanda, yathābhuccā avipallatthā parisuddhā suññatāvakkanti bhavati.
\switchcolumn*
 'Again, Ananda, without giving attention to perception of forest, without giving attention to perception of earth, a bhikkhu gives attention to the single state (on non-voidness) dependent on (the presence of) perception of the base consisting of infinite space. His mind enters into that perception of the base consisting of infinite space and acquires confidence, steadiness and decision. He understands thus: "Disturbances that would be present dependent on perception of forest are not present here, disturbances that would be present on perception of earth are not present here, and only this measure of disturbance is present, that is to say, the single state (of non-voidness) dependent on (the presence of) perception of the base consisting of infinite space." He understands: "This field of perception is void of perception of forest." He understands: "This field of perception is void of perception of earth." , (and he understands:) "There is (present) only this non-voidness, that is to say, the single state (of non-voidness) dependent on (the presence of) perception of the base consisting of infinite space." So he sees it as void of what is not there, but of what remains there he understands: "There is that still present there." Now this too has been for him an alighting upon voidness that accords with what actually is, without perversion of meaning and is pure.
\switchcolumn
 178. “puna caparaṃ, ānanda, bhikkhu amanasikaritvā araññasaññaṃ, amanasikaritvā pathavīsaññaṃ, ākāsānañcāyatanasaññaṃ paṭicca manasi karoti ekattaṃ. tassa ākāsānañcāyatanasaññāya cittaṃ pakkhandati pasīdati santiṭṭhati adhimuccati. so evaṃ pajānāti — ‘ye assu darathā araññasaññaṃ paṭicca tedha na santi, ye assu darathā pathavīsaññaṃ paṭicca tedha na santi, atthi cevāyaṃ darathamattā yadidaṃ — ākāsānañcāyatanasaññaṃ paṭicca ekattan’ti. so ‘suññamidaṃ saññāgataṃ araññasaññāyā’ti pajānāti, ‘suññamidaṃ saññāgataṃ pathavīsaññāyā’ti pajānāti, ‘atthi cevidaṃ asuññataṃ yadidaṃ — ākāsānañcāyatanasaññaṃ paṭicca ekattan’ti. iti yañhi kho tattha na hoti tena taṃ suññaṃ samanupassati, yaṃ pana tattha avasiṭṭhaṃ hoti taṃ ‘santamidaṃ atthī’ti pajānāti. evampissa esā, ānanda, yathābhuccā avipallatthā parisuddhā suññatāvakkanti bhavati.
\switchcolumn*
'Again, Ananda, without giving attention to perception of earth, without giving attention to perception of the base consisting of infinite space, a bhikkhu gives attention to the single state (of non-voidness) dependent on (the presence of) perception of the base consisting of infinite consciousness. His mind enters into the perception of the base consisting of infinite consciousness and acquires confidence, steadiness and decision. He understands thus: "Disturbances that would be present dependent on perception of earth are not present here, disturbances that would be present dependent on perception of the base consisting of infinite space are not present here, and only this measure of disturbance is present, that is to say, the single state (of non-voidness) dependent on (the presence of) perception of the base consisting of infinite consciousness." He understands: "This field of perception is void of perception of earth." He understands: "This field of perception is void of perception of the base consisting of infinite space", (and he understands): "There is (present) only this non-voidness, that is to say, the single state (of non-voidness) dependent on (the presence of) perception of the base consisting of infinite consciousness." So he sees it as void of what is not there, but of what remains there he understands: "There is that still present there." Now this too has been for him an alighting upon voidness that accords with what actually is, without perversion of meaning, and is pure.
\switchcolumn
179. “puna caparaṃ, ānanda, bhikkhu amanasikaritvā pathavīsaññaṃ, amanasikaritvā ākāsānañcāyatanasaññaṃ, viññāṇañcāyatanasaññaṃ paṭicca manasi karoti ekattaṃ. tassa viññāṇañcāyatanasaññāya cittaṃ pakkhandati pasīdati santiṭṭhati adhimuccati. so evaṃ pajānāti — ‘ye assu darathā pathavīsaññaṃ paṭicca tedha na santi, ye assu darathā ākāsānañcāyatanasaññaṃ paṭicca tedha na santi, atthi cevāyaṃ darathamattā yadidaṃ — viññāṇañcāyatanasaññaṃ paṭicca ekattan’ti. so ‘suññamidaṃ saññāgataṃ pathavīsaññāyā’ti pajānāti, ‘suññamidaṃ saññāgataṃ ākāsānañcāyatanasaññāyā’ti pajānāti, ‘atthi cevidaṃ asuññataṃ yadidaṃ — viññāṇañcāyatanasaññaṃ paṭicca ekattan’ti. iti yañhi kho tattha na hoti tena taṃ suññaṃ samanupassati, yaṃ pana tattha avasiṭṭhaṃ hoti taṃ ‘santamidaṃ atthī’ti pajānāti. evampissa esā, ānanda, yathābhuccā avipallatthā parisuddhā suññatāvakkanti bhavati.
\switchcolumn*
'Again, Ananda, without giving attention to perception of the base consisting of infinite space, without giving attention to perception of the base consisting of infinite consciousness, a bhikkhu gives attention to the single state (of non-voidness) dependent on (the presence of) perception of the base consisting of nothingness. His mind enters into the perception of the base consisting of nothingness and he acquires confidence, steadiness and decision. He understands thus: "Disturbances that would be present dependent on perception of the base of infinite space are not present here, disturbances that would be present dependent on perception of the base of infinite consciousness are not present here, and only this measure of disturbance is present, that is to say, the single state (of non-voidness) dependent on (the presence of) perception of the base consisting of nothingness." He understands: "This field of perception is void of perception of the base consisting of infinite space.", and he understands: "This field of perception is void of perception of the base consisting of infinite consciousness.", (and he understands): "There is (present) only this non-voidness, that is to say, the single state (of non-voidness) dependent on (the presence of) perception of the base consisting of nothingness." So he sees it as void of what is not there, but of what remains there he understands: "There is that still present there." Now this too has been for him an alighting upon voidness that accords with what is, without perversion of meaning and is pure.
\switchcolumn
180. “puna caparaṃ, ānanda, bhikkhu amanasikaritvā ākāsānañcāyatanasaññaṃ, amanasikaritvā viññāṇañcāyatanasaññaṃ, ākiñcaññāyatanasaññaṃ paṭicca manasi karoti ekattaṃ. tassa ākiñcaññāyatanasaññāya cittaṃ pakkhandati pasīdati santiṭṭhati adhimuccati. so evaṃ pajānāti — ‘ye assu darathā ākāsānañcāyatanasaññaṃ paṭicca tedha na santi, ye assu darathā viññāṇañcāyatanasaññaṃ paṭicca tedha na santi, atthi cevāyaṃ darathamattā yadidaṃ — ākiñcaññāyatanasaññaṃ paṭicca ekattan’ti. so ‘suññamidaṃ saññāgataṃ ākāsānañcāyatanasaññāyā’ti pajānāti, ‘suññamidaṃ saññāgataṃ viññāṇañcāyatanasaññāyā’ti pajānāti, ‘atthi cevidaṃ asuññataṃ yadidaṃ — ākiñcaññāyatanasaññaṃ paṭicca ekattan’ti. iti yañhi kho tattha na hoti tena taṃ suññaṃ samanupassati, yaṃ pana tattha avasiṭṭhaṃ hoti taṃ ‘santamidaṃ atthī’ti pajānāti. evampissa esā, ānanda, yathābhuccā avipallatthā parisuddhā suññatāvakkanti bhavati.
\switchcolumn*
'Again, Ananda, without giving attention to perception of the base consisting of infinite consciousness, without giving attention to perception of the base consisting of nothingness, a bhikkhu gives attention to the single state (of non-voidness) dependent on (the presence of) perception of the base consisting of neither-perception-nor-non-perception. His mind enters into the perception of the base consisting of neither-perception-nor-non-perception and acquires confidence, steadiness and decision. He understands thus: "Disturbances that would be present dependent on the perception of the base consisting of infinite consciousness are not present here, disturbances that would present dependent on the perception of the base consisting of nothingness are not present here, and only this measure of disturbance is present, that is to say, the single state (of non-voidness) dependent on (the presence of) perception of the base consisting of neither-perception-nor-non-perception." He understands: "This field of perception is void of perception of the base consisting of infinite consciousness." He understands: "This field of perception is void of perception of the base consisting of nothingness.", (and he understands): "There is (present) only this non-voidness, that is to say, the single state (of non-voidness) dependent on (the presence of) perception of the base consisting of neither-perception-nor-non-perception." So he sees it as void of what is not there, but of what remains there he understands: "There is that still present there." Now this too has been for him an alighting upon voidness, that accords with what actually is, without perversion of meaning and is pure.
\switchcolumn
181. “puna caparaṃ, ānanda bhikkhu amanasikaritvā viññāṇañcāyatanasaññaṃ, amanasikaritvā ākiñcaññāyatanasaññaṃ, nevasaññānāsaññāyatanasaññaṃ paṭicca manasi karoti ekattaṃ. tassa nevasaññānāsaññāyatanasaññāya cittaṃ pakkhandati pasīdati santiṭṭhati adhimuccati. so evaṃ pajānāti — ‘ye assu darathā viññāṇañcāyatanasaññaṃ paṭicca tedha na santi, ye assu darathā ākiñcaññāyatanasaññaṃ paṭicca tedha na santi, atthi cevāyaṃ darathamattā yadidaṃ — nevasaññānāsaññāyatanasaññaṃ paṭicca ekattan’ti. so ‘suññamidaṃ saññāgataṃ viññāṇañcāyatanasaññāyā’ti pajānāti, ‘suññamidaṃ saññāgataṃ ākiñcaññāyatanasaññāyā’ti pajānāti, ‘atthi cevidaṃ asuññataṃ yadidaṃ — nevasaññānāsaññāyatanasaññaṃ paṭicca ekattan’ti. iti yañhi kho tattha na hoti tena taṃ suññaṃ samanupassati, yaṃ pana tattha avasiṭṭhaṃ hoti taṃ ‘santamidaṃ atthī’ti pajānāti. evampissa esā, ānanda, yathābhuccā avipallatthā parisuddhā suññatāvakkanti bhavati.
\switchcolumn*
'Again, Ananda, without giving attention to perception of the base consisting of nothingness, without giving attention to perception of the base consisting of neither-perception-nor-non-perception, a bhikkhu gives attention to the single state (of non-voidness) dependent on (the presence of) the signless concentration of mind. His mind enters into the signless concentration of mind and acquires confidence, steadiness and decision. He understands thus: "Disturbances that would be present dependent on the perception of the base consisting of nothingness are not present here, disturbances that would be present dependent on the perception of the base consisting of neither-perception-nor-non-perception are not present here, and only this measure of disturbance is present, that is to say, that (disturbance) which has life as its condition dependent on the presence of this body with its six bases." He understands: "This field of perception is void of perception of the base consisting of nothingness." He understands: "This field of perception is void of perception of the base consisting of neither-perception-nor-non-perception.", (and he understands): "There is (present) only this non-voidness, that is to say, that (non-voidness) with life as its condition dependent on this body with its six bases." So he sees it as void of what is not there, but of what remains there he understands: "There is that still present there." Now this too has been for him an alighting upon voidness, that accords with what actually is, without perversion of meaning and is pure.
 \switchcolumn
 182. “puna caparaṃ, ānanda, bhikkhu amanasikaritvā ākiñcaññāyatanasaññaṃ, amanasikaritvā nevasaññānāsaññāyatanasaññaṃ, animittaṃ cetosamādhiṃ paṭicca manasi karoti ekattaṃ. tassa animitte cetosamādhimhi cittaṃ pakkhandati pasīdati santiṭṭhati adhimuccati. so evaṃ pajānāti — ‘ye assu darathā ākiñcaññāyatanasaññaṃ paṭicca tedha na santi, ye assu darathā nevasaññānāsaññāyatanasaññaṃ paṭicca tedha na santi, atthi cevāyaṃ darathamattā yadidaṃ — imameva kāyaṃ paṭicca saḷāyatanikaṃ jīvitapaccayā’ti. so ‘suññamidaṃ saññāgataṃ ākiñcaññāyatanasaññāyā’ti pajānāti, ‘suññamidaṃ saññāgataṃ nevasaññānāsaññāyatanasaññāyā’ti pajānāti, ‘atthi cevidaṃ asuññataṃ yadidaṃ — imameva kāyaṃ paṭicca saḷāyatanikaṃ jīvitapaccayā’ti. iti yañhi kho tattha na hoti tena taṃ suññaṃ samanupassati, yaṃ pana tattha avasiṭṭhaṃ hoti taṃ ‘santamidaṃ atthī’ti pajānāti. evampissa esā, ānanda, yathābhuccā avipallatthā parisuddhā suññatāvakkanti bhavati.
\switchcolumn*
'Again, Ananda, without giving attention to perception of the base consisting of nothingness, without giving attention to perception of the base consisting of neither-perception-nor-non-perception, a bhikkhu gives attention to the single state (of non-voidness) dependent on (the presence of) the signless concentration of mind. His mind enters into the signless concentration of mind and acquires confidence, steadiness and decision. He understands thus: "This signless concentration of mind is conditioned and mentally produced." He understands: "Whatever is conditioned and mentally produced is impermanent and liable to cessation." When he knows and sees thus, his mind is liberated from the taint of sensual desire, from the taint of being, from the taint of ignorance. When liberated there comes the knowledge "It is liberated". He understands: "Birth is exhausted, the life divine has been lived out, what was to be done is done, there is no more of this to come." He understands thus: "Disturbances that would be present dependent on the taint of sensual desire are not present here, disturbances that would be present dependent on the taint of being are not present here, disturbances that would be present dependent on the taint of ignorance are not present here, and only this measure of disturbance is present, that is to say, that (non-voidness) with life as its condition dependent on (the presence of) this body with its six bases." He understands: "This field of perception is void of the taint of sensual desire." He understands: "This field of perception is void of the taint of being." He understands: "This field of perception is void of the taint of ignorance.", (and he understands): "There is (present) only this non-voidness, that is to say, that (non-voidness) with life as its condition dependent on (the presence of) this body with its six bases." So he sees it as void of what is not there, but of what remains there he understands: "There is that still present there." Now this has been for him an alighting upon voidness that accords with what actually is, without perversion of meaning, is pure and is unsurpassed by any other.
\switchcolumn
183. “puna caparaṃ, ānanda, bhikkhu amanasikaritvā ākiñcaññāyatanasaññaṃ, amanasikaritvā nevasaññānāsaññāyatanasaññaṃ, animittaṃ cetosamādhiṃ paṭicca manasi karoti ekattaṃ. tassa animitte cetosamādhimhi cittaṃ pakkhandati pasīdati santiṭṭhati adhimuccati. so evaṃ pajānāti — ‘ayampi kho animitto cetosamādhi abhisaṅkhato abhisañcetayito’. ‘yaṃ kho pana kiñci abhisaṅkhataṃ abhisañcetayitaṃ tadaniccaṃ nirodhadhamman’ti pajānāti. tassa evaṃ jānato evaṃ passato kāmāsavāpi cittaṃ vimuccati, bhavāsavāpi cittaṃ vimuccati, avijjāsavāpi cittaṃ vimuccati. vimuttasmiṃ vimuttamiti ñāṇaṃ hoti. ‘khīṇā jāti, vusitaṃ brahmacariyaṃ, kataṃ karaṇīyaṃ, nāparaṃ itthattāyā’ti pajānāti. so evaṃ pajānāti — ‘ye assu darathā kāmāsavaṃ paṭicca tedha na santi, ye assu darathā bhavāsavaṃ paṭicca tedha na santi, ye assu darathā avijjāsavaṃ paṭicca tedha na santi, atthi cevāyaṃ darathamattā yadidaṃ — imameva kāyaṃ paṭicca saḷāyatanikaṃ jīvitapaccayā’ti. so ‘suññamidaṃ saññāgataṃ kāmāsavenā’ti pajānāti, ‘suññamidaṃ saññāgataṃ bhavāsavenā’ti pajānāti, ‘suññamidaṃ saññāgataṃ avijjāsavenā’ti pajānāti, ‘atthi cevidaṃ asuññataṃ yadidaṃ — imameva kāyaṃ paṭicca saḷāyatanikaṃ jīvitapaccayā’ti. iti yañhi kho tattha na hoti tena taṃ suññaṃ samanupassati, yaṃ pana tattha avasiṭṭhaṃ hoti taṃ ‘santamidaṃ atthī’ti pajānāti. evampissa esā, ānanda, yathābhuccā avipallatthā parisuddhā paramānuttarā suññatāvakkanti bhavati.\\
\switchcolumn*
 'Whatever monks or divines in the past have entered upon and abode in a voidness that was purified and unsurpassed by any other, they have all of them entered upon and abode in this voidness that is pure and unsurpassed by any other.
'Whatever monks and divines in the future will enter upon and abide in a voidness that will be purified and unsurpassed by any other, they will all of them enter upon and abide in this voidness that is pure and unsurpassed by any other.
'Whatever monks and divines in the present enter upon and abide in a voidness that is purified and unsurpassed by any other, they all of them will enter upon and abide in this voidness that is pure and unsurpassed by any other.
'Therefore, Ananda, you should train thus: "We will enter upon and abide in the voidness that is pure and unsurpassed by any other."'\\
\switchcolumn
184. “yepi hi keci, ānanda, atītamaddhānaṃ samaṇā vā brāhmaṇā vā parisuddhaṃ paramānuttaraṃ suññataṃ upasampajja vihariṃsu, sabbe te imaṃyeva parisuddhaṃ paramānuttaraṃ suññataṃ upasampajja vihariṃsu. yepi hi keci, ānanda, anāgatamaddhānaṃ samaṇā vā brāhmaṇā vā parisuddhaṃ paramānuttaraṃ suññataṃ upasampajja viharissanti, sabbe te imaṃyeva parisuddhaṃ paramānuttaraṃ suññataṃ upasampajja viharissanti. yepi hi keci, ānanda, etarahi samaṇā vā brāhmaṇā vā parisuddhaṃ paramānuttaraṃ suññataṃ upasampajja viharanti, sabbe te imaṃyeva parisuddhaṃ paramānuttaraṃ suññataṃ upasampajja viharanti. tasmātiha, ānanda, ‘parisuddhaṃ paramānuttaraṃ suññataṃ upasampajja viharissāmā’ti — evañhi vo, ānanda, sikkhitabban”ti.\\
\switchcolumn*
That is what the Blessed One said. The venerable Ananda was satisfied, and he delighted in the Blessed One's words.
\switchcolumn
idamavoca bhagavā. attamano āyasmā ānando bhagavato bhāsitaṃ abhinandīti.\\[.5cm]
\textbf{cūḷasuññatasuttaṃ niṭṭhitaṃ paṭhamaṃ }
\switchcolumn*

\end{column}
\end{paracol}

\end{document}


