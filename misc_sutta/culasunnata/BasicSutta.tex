\documentclass[11pt]{article}
\usepackage[margin=1in]{geometry}
\usepackage{titlesec}
\usepackage{polyglossia}
\defaultfontfeatures{Ligatures=TeX}
\setmainfont{Arial}
\usepackage{expex}
\sloppy

% Uncomment the following line if you do not want extra space before the free translation and the glosses, or other customisations to the examples.
\lingset{aboveglftskip=-.2ex,interpartskip=\baselineskip,everyglb=\footnotesize}

\title{Some \texttt{expex} Linguistic Examples}
\author{LianTze Lim}

\begin{document}
\textbf{Majjhima Nikāya, uparipaṇṇāsapāḷi, 3. suññatavaggo, 1. cūḷasuññatasuttaṃ (MN 121)}\\[.3cm]

176. evaṃ me sutaṃ — ekaṃ samayaṃ bhagavā sāvatthiyaṃ viharati pubbārāme migāramātupāsāde. atha kho āyasmā ānando sāyanhasamayaṃ paṭisallānā vuṭṭhito yena bhagavā tenupasaṅkami; upasaṅkamitvā bhagavantaṃ abhivādetvā ekamantaṃ nisīdi. ekamantaṃ nisinno kho āyasmā ānando bhagavantaṃ etadavoca — “ekamidaṃ, bhante, samayaṃ bhagavā sakkesu viharati nagarakaṃ nāma sakyānaṃ nigamo. tattha me, bhante, bhagavato sammukhā sutaṃ, sammukhā paṭiggahitaṃ — ‘suññatāvihārenāhaṃ, ānanda, etarahi bahulaṃ viharāmī’ti. kacci metaṃ, bhante, sussutaṃ suggahitaṃ sumanasikataṃ sūpadhāritan”ti? “taggha te etaṃ, ānanda, sussutaṃ suggahitaṃ sumanasikataṃ sūpadhāritaṃ. pubbepāhaṃ, ānanda, etarahipi suññatāvihārena bahulaṃ viharāmi. seyyathāpi, ānanda, ayaṃ migāramātupāsādo suñño hatthigavassavaḷavena, suñño jātarūparajatena, suñño itthipurisasannipātena atthi cevidaṃ asuññataṃ yadidaṃ — bhikkhusaṅghaṃ paṭicca ekattaṃ; evameva kho, ānanda, bhikkhu amanasikaritvā gāmasaññaṃ, amanasikaritvā manussasaññaṃ, araññasaññaṃ paṭicca manasi karoti ekattaṃ. tassa araññasaññāya cittaṃ pakkhandati pasīdati santiṭṭhati adhimuccati. so evaṃ pajānāti — ‘ye assu darathā gāmasaññaṃ paṭicca tedha na santi, ye assu darathā manussasaññaṃ paṭicca tedha na santi, atthi cevāyaṃ darathamattā yadidaṃ — araññasaññaṃ paṭicca ekattan’ti. so ‘suññamidaṃ saññāgataṃ gāmasaññāyā’ti pajānāti, ‘suññamidaṃ saññāgataṃ manussasaññāyā’ti pajānāti, ‘atthi cevidaṃ asuññataṃ yadidaṃ — araññasaññaṃ paṭicca ekattan’ti. iti yañhi kho tattha na hoti tena taṃ suññaṃ samanupassati, yaṃ pana tattha avasiṭṭhaṃ hoti taṃ ‘santamidaṃ atthī’”ti pajānāti. evampissa esā, ānanda, yathābhuccā avipallatthā parisuddhā suññatāvakkanti bhavati.\\

177. “puna caparaṃ, ānanda, bhikkhu amanasikaritvā manussasaññaṃ, amanasikaritvā araññasaññaṃ, pathavīsaññaṃ paṭicca manasi karoti ekattaṃ. tassa pathavīsaññāya cittaṃ pakkhandati pasīdati santiṭṭhati adhimuccati. seyyathāpi, ānanda, āsabhacammaṃ saṅkusatena suvihataṃ vigatavalikaṃ; evameva kho, ānanda, bhikkhu yaṃ imissā pathaviyā ukkūlavikkūlaṃ nadīviduggaṃ khāṇukaṇṭakaṭṭhānaṃ pabbatavisamaṃ taṃ sabbaṃ amanasikaritvā pathavīsaññaṃ paṭicca manasi karoti ekattaṃ. tassa pathavīsaññāya cittaṃ pakkhandati pasīdati santiṭṭhati adhimuccati. so evaṃ pajānāti — ‘ye assu darathā manussasaññaṃ paṭicca tedha na santi, ye assu darathā araññasaññaṃ paṭicca tedha na santi, atthi cevāyaṃ darathamattā yadidaṃ — pathavīsaññaṃ paṭicca ekattan’ti. so ‘suññamidaṃ saññāgataṃ manussasaññāyā’ti pajānāti, ‘suññamidaṃ saññāgataṃ araññasaññāyā’ti pajānāti, ‘atthi cevidaṃ asuññataṃ yadidaṃ — pathavīsaññaṃ paṭicca ekattan’ti. iti yañhi kho tattha na hoti tena taṃ suññaṃ samanupassati, yaṃ pana tattha avasiṭṭhaṃ hoti taṃ ‘santamidaṃ atthī’ti pajānāti. evampissa esā, ānanda, yathābhuccā avipallatthā parisuddhā suññatāvakkanti bhavati.\\

178. “puna caparaṃ, ānanda, bhikkhu amanasikaritvā araññasaññaṃ, amanasikaritvā pathavīsaññaṃ, ākāsānañcāyatanasaññaṃ paṭicca manasi karoti ekattaṃ. tassa ākāsānañcāyatanasaññāya cittaṃ pakkhandati pasīdati santiṭṭhati adhimuccati. so evaṃ pajānāti — ‘ye assu darathā araññasaññaṃ paṭicca tedha na santi, ye assu darathā pathavīsaññaṃ paṭicca tedha na santi, atthi cevāyaṃ darathamattā yadidaṃ — ākāsānañcāyatanasaññaṃ paṭicca ekattan’ti. so ‘suññamidaṃ saññāgataṃ araññasaññāyā’ti pajānāti, ‘suññamidaṃ saññāgataṃ pathavīsaññāyā’ti pajānāti, ‘atthi cevidaṃ asuññataṃ yadidaṃ — ākāsānañcāyatanasaññaṃ paṭicca ekattan’ti. iti yañhi kho tattha na hoti tena taṃ suññaṃ samanupassati, yaṃ pana tattha avasiṭṭhaṃ hoti taṃ ‘santamidaṃ atthī’ti pajānāti. evampissa esā, ānanda, yathābhuccā avipallatthā parisuddhā suññatāvakkanti bhavati.\\

179. “puna caparaṃ, ānanda, bhikkhu amanasikaritvā pathavīsaññaṃ, amanasikaritvā ākāsānañcāyatanasaññaṃ, viññāṇañcāyatanasaññaṃ paṭicca manasi karoti ekattaṃ. tassa viññāṇañcāyatanasaññāya cittaṃ pakkhandati pasīdati santiṭṭhati adhimuccati. so evaṃ pajānāti — ‘ye assu darathā pathavīsaññaṃ paṭicca tedha na santi, ye assu darathā ākāsānañcāyatanasaññaṃ paṭicca tedha na santi, atthi cevāyaṃ darathamattā yadidaṃ — viññāṇañcāyatanasaññaṃ paṭicca ekattan’ti. so ‘suññamidaṃ saññāgataṃ pathavīsaññāyā’ti pajānāti, ‘suññamidaṃ saññāgataṃ ākāsānañcāyatanasaññāyā’ti pajānāti, ‘atthi cevidaṃ asuññataṃ yadidaṃ — viññāṇañcāyatanasaññaṃ paṭicca ekattan’ti. iti yañhi kho tattha na hoti tena taṃ suññaṃ samanupassati, yaṃ pana tattha avasiṭṭhaṃ hoti taṃ ‘santamidaṃ atthī’ti pajānāti. evampissa esā, ānanda, yathābhuccā avipallatthā parisuddhā suññatāvakkanti bhavati.\\

180. “puna caparaṃ, ānanda, bhikkhu amanasikaritvā ākāsānañcāyatanasaññaṃ, amanasikaritvā viññāṇañcāyatanasaññaṃ, ākiñcaññāyatanasaññaṃ paṭicca manasi karoti ekattaṃ. tassa ākiñcaññāyatanasaññāya cittaṃ pakkhandati pasīdati santiṭṭhati adhimuccati. so evaṃ pajānāti — ‘ye assu darathā ākāsānañcāyatanasaññaṃ paṭicca tedha na santi, ye assu darathā viññāṇañcāyatanasaññaṃ paṭicca tedha na santi, atthi cevāyaṃ darathamattā yadidaṃ — ākiñcaññāyatanasaññaṃ paṭicca ekattan’ti. so ‘suññamidaṃ saññāgataṃ ākāsānañcāyatanasaññāyā’ti pajānāti, ‘suññamidaṃ saññāgataṃ viññāṇañcāyatanasaññāyā’ti pajānāti, ‘atthi cevidaṃ asuññataṃ yadidaṃ — ākiñcaññāyatanasaññaṃ paṭicca ekattan’ti. iti yañhi kho tattha na hoti tena taṃ suññaṃ samanupassati, yaṃ pana tattha avasiṭṭhaṃ hoti taṃ ‘santamidaṃ atthī’ti pajānāti. evampissa esā, ānanda, yathābhuccā avipallatthā parisuddhā suññatāvakkanti bhavati.\\

181. “puna caparaṃ, ānanda bhikkhu amanasikaritvā viññāṇañcāyatanasaññaṃ, amanasikaritvā ākiñcaññāyatanasaññaṃ, nevasaññānāsaññāyatanasaññaṃ paṭicca manasi karoti ekattaṃ. tassa nevasaññānāsaññāyatanasaññāya cittaṃ pakkhandati pasīdati santiṭṭhati adhimuccati. so evaṃ pajānāti — ‘ye assu darathā viññāṇañcāyatanasaññaṃ paṭicca tedha na santi, ye assu darathā ākiñcaññāyatanasaññaṃ paṭicca tedha na santi, atthi cevāyaṃ darathamattā yadidaṃ — nevasaññānāsaññāyatanasaññaṃ paṭicca ekattan’ti. so ‘suññamidaṃ saññāgataṃ viññāṇañcāyatanasaññāyā’ti pajānāti, ‘suññamidaṃ saññāgataṃ ākiñcaññāyatanasaññāyā’ti pajānāti, ‘atthi cevidaṃ asuññataṃ yadidaṃ — nevasaññānāsaññāyatanasaññaṃ paṭicca ekattan’ti. iti yañhi kho tattha na hoti tena taṃ suññaṃ samanupassati, yaṃ pana tattha avasiṭṭhaṃ hoti taṃ ‘santamidaṃ atthī’ti pajānāti. evampissa esā, ānanda, yathābhuccā avipallatthā parisuddhā suññatāvakkanti bhavati.\\

182. “puna caparaṃ, ānanda, bhikkhu amanasikaritvā ākiñcaññāyatanasaññaṃ, amanasikaritvā nevasaññānāsaññāyatanasaññaṃ, animittaṃ cetosamādhiṃ paṭicca manasi karoti ekattaṃ. tassa animitte cetosamādhimhi cittaṃ pakkhandati pasīdati santiṭṭhati adhimuccati. so evaṃ pajānāti — ‘ye assu darathā ākiñcaññāyatanasaññaṃ paṭicca tedha na santi, ye assu darathā nevasaññānāsaññāyatanasaññaṃ paṭicca tedha na santi, atthi cevāyaṃ darathamattā yadidaṃ — imameva kāyaṃ paṭicca saḷāyatanikaṃ jīvitapaccayā’ti. so ‘suññamidaṃ saññāgataṃ ākiñcaññāyatanasaññāyā’ti pajānāti, ‘suññamidaṃ saññāgataṃ nevasaññānāsaññāyatanasaññāyā’ti pajānāti, ‘atthi cevidaṃ asuññataṃ yadidaṃ — imameva kāyaṃ paṭicca saḷāyatanikaṃ jīvitapaccayā’ti. iti yañhi kho tattha na hoti tena taṃ suññaṃ samanupassati, yaṃ pana tattha avasiṭṭhaṃ hoti taṃ ‘santamidaṃ atthī’ti pajānāti. evampissa esā, ānanda, yathābhuccā avipallatthā parisuddhā suññatāvakkanti bhavati.\\

183. “puna caparaṃ, ānanda, bhikkhu amanasikaritvā ākiñcaññāyatanasaññaṃ, amanasikaritvā nevasaññānāsaññāyatanasaññaṃ, animittaṃ cetosamādhiṃ paṭicca manasi karoti ekattaṃ. tassa animitte cetosamādhimhi cittaṃ pakkhandati pasīdati santiṭṭhati adhimuccati. so evaṃ pajānāti — ‘ayampi kho animitto cetosamādhi abhisaṅkhato abhisañcetayito’. ‘yaṃ kho pana kiñci abhisaṅkhataṃ abhisañcetayitaṃ tadaniccaṃ nirodhadhamman’ti pajānāti. tassa evaṃ jānato evaṃ passato kāmāsavāpi cittaṃ vimuccati, bhavāsavāpi cittaṃ vimuccati, avijjāsavāpi cittaṃ vimuccati. vimuttasmiṃ vimuttamiti ñāṇaṃ hoti. ‘khīṇā jāti, vusitaṃ brahmacariyaṃ, kataṃ karaṇīyaṃ, nāparaṃ itthattāyā’ti pajānāti. so evaṃ pajānāti — ‘ye assu darathā kāmāsavaṃ paṭicca tedha na santi, ye assu darathā bhavāsavaṃ paṭicca tedha na santi, ye assu darathā avijjāsavaṃ paṭicca tedha na santi, atthi cevāyaṃ darathamattā yadidaṃ — imameva kāyaṃ paṭicca saḷāyatanikaṃ jīvitapaccayā’ti. so ‘suññamidaṃ saññāgataṃ kāmāsavenā’ti pajānāti, ‘suññamidaṃ saññāgataṃ bhavāsavenā’ti pajānāti, ‘suññamidaṃ saññāgataṃ avijjāsavenā’ti pajānāti, ‘atthi cevidaṃ asuññataṃ yadidaṃ — imameva kāyaṃ paṭicca saḷāyatanikaṃ jīvitapaccayā’ti. iti yañhi kho tattha na hoti tena taṃ suññaṃ samanupassati, yaṃ pana tattha avasiṭṭhaṃ hoti taṃ ‘santamidaṃ atthī’ti pajānāti. evampissa esā, ānanda, yathābhuccā avipallatthā parisuddhā paramānuttarā suññatāvakkanti bhavati.\\

184. “yepi hi keci, ānanda, atītamaddhānaṃ samaṇā vā brāhmaṇā vā parisuddhaṃ paramānuttaraṃ suññataṃ upasampajja vihariṃsu, sabbe te imaṃyeva parisuddhaṃ paramānuttaraṃ suññataṃ upasampajja vihariṃsu. yepi hi keci, ānanda, anāgatamaddhānaṃ samaṇā vā brāhmaṇā vā parisuddhaṃ paramānuttaraṃ suññataṃ upasampajja viharissanti, sabbe te imaṃyeva parisuddhaṃ paramānuttaraṃ suññataṃ upasampajja viharissanti. yepi hi keci, ānanda, etarahi samaṇā vā brāhmaṇā vā parisuddhaṃ paramānuttaraṃ suññataṃ upasampajja viharanti, sabbe te imaṃyeva parisuddhaṃ paramānuttaraṃ suññataṃ upasampajja viharanti. tasmātiha, ānanda, ‘parisuddhaṃ paramānuttaraṃ suññataṃ upasampajja viharissāmā’ti — evañhi vo, ānanda, sikkhitabban”ti.\\

idamavoca bhagavā. attamano āyasmā ānando bhagavato bhāsitaṃ abhinandīti.\\[.5cm]

\textbf{cūḷasuññatasuttaṃ niṭṭhitaṃ paṭhamaṃ}

\end{document}
