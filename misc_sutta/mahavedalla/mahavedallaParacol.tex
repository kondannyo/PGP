\begin{samepage}
\ensurevspace{4\baselineskip}
\begin{leftcolumn*}
MN 43
The Greater Series of Questions-and-Answers
\end{leftcolumn*}

\begin{rightcolumn}
Majjhima Nikāya, mūlapaṇṇāsapāḷi, 5. cūḷayamakavaggo,
3. mahāvedallasuttaṃ (MN 43)
\end{rightcolumn}
\end{samepage}

\begin{samepage}
\ensurevspace{4\baselineskip}
\begin{leftcolumn*}
I have heard that on one occasion the Blessed One was staying at Savatthi, in Jeta's Grove, Anathapindika's Park.
\end{leftcolumn*}

\begin{rightcolumn}
evaṃ me sutaṃ — ekaṃ samayaṃ bhagavā sāvatthiyaṃ viharati jetavane anāthapiṇḍikassa ārāme.
\end{rightcolumn}
\end{samepage}

\begin{samepage}
\ensurevspace{4\baselineskip}
\begin{leftcolumn*}
Then Ven. Maha Kotthita, arising from his seclusion in the late afternoon, went to Ven. Sariputta and, on arrival, exchanged courteous greetings with him.
\end{leftcolumn*}

\begin{rightcolumn}
atha kho āyasmā mahākoṭṭhiko sāyanhasamayaṃ paṭisallānā vuṭṭhito yenāyasmā sāriputto tenupasaṅkami; upasaṅkamitvā āyasmatā sāriputtena saddhiṃ sammodi.
\end{rightcolumn}
\end{samepage}

\begin{samepage}
\ensurevspace{4\baselineskip}
\begin{leftcolumn*}
After an exchange of friendly greetings and courtesies, he sat to one side.
\end{leftcolumn*}

\begin{rightcolumn}
sammodanīyaṃ kathaṃ sāraṇīyaṃ vītisāretvā ekamantaṃ nisīdi.
\end{rightcolumn}
\end{samepage}

\begin{samepage}
\ensurevspace{4\baselineskip}
\begin{leftcolumn*}
As he was sitting there, he said to Ven. Sariputta,
\end{leftcolumn*}

\begin{rightcolumn}
ekamantaṃ nisinno kho āyasmā mahākoṭṭhiko āyasmantaṃ sāriputtaṃ etadavoca,
\end{rightcolumn}
\end{samepage}

\begin{samepage}
\ensurevspace{4\baselineskip}
\begin{leftcolumn*}
-
\end{leftcolumn*}

\begin{rightcolumn}
-
\end{rightcolumn}
\end{samepage}

\begin{samepage}
\ensurevspace{4\baselineskip}
\begin{leftcolumn*}
"Friend, 'One of poor discernment, one of poor discernment': Thus is it said.
\end{leftcolumn*}

\begin{rightcolumn}
“‘duppañño duppañño’ti, āvuso, vuccati.
\end{rightcolumn}
\end{samepage}

\begin{samepage}
\ensurevspace{4\baselineskip}
\begin{leftcolumn*}
To what extent is one said to be 'one of poor discernment'?"
\end{leftcolumn*}

\begin{rightcolumn}
kittāvatā nu kho, āvuso, duppaññoti vuccatī”ti?
\end{rightcolumn}
\end{samepage}

\begin{samepage}
\ensurevspace{4\baselineskip}
\begin{leftcolumn*}
"'One doesn't discern, one doesn't discern': Thus, friend, one is said to be 'one of poor discernment.'
\end{leftcolumn*}

\begin{rightcolumn}
“‘nappajānāti nappajānātī’ti kho, āvuso, tasmā duppaññoti vuccati.
\end{rightcolumn}
\end{samepage}

\begin{samepage}
\ensurevspace{4\baselineskip}
\begin{leftcolumn*}
-
\end{leftcolumn*}

\begin{rightcolumn}
-
\end{rightcolumn}
\end{samepage}

\begin{samepage}
\ensurevspace{4\baselineskip}
\begin{leftcolumn*}
And what doesn't one discern?
\end{leftcolumn*}

\begin{rightcolumn}
“kiñca nappajānāti?
\end{rightcolumn}
\end{samepage}

\begin{samepage}
\ensurevspace{4\baselineskip}
\begin{leftcolumn*}
One doesn't discern, 'This is suffering.'
\end{leftcolumn*}

\begin{rightcolumn}
‘idaṃ dukkhan’ti nappajānāti,
\end{rightcolumn}
\end{samepage}

\begin{samepage}
\ensurevspace{4\baselineskip}
\begin{leftcolumn*}
One doesn't discern, 'This is the origination of suffering.'
\end{leftcolumn*}

\begin{rightcolumn}
‘ayaṃ dukkhasamudayo’ti nappajānāti,
\end{rightcolumn}
\end{samepage}

\begin{samepage}
\ensurevspace{4\baselineskip}
\begin{leftcolumn*}
One doesn't discern, 'This is the cessation of suffering.'
\end{leftcolumn*}

\begin{rightcolumn}
‘ayaṃ dukkhanirodho’ti nappajānāti,
\end{rightcolumn}
\end{samepage}

\begin{samepage}
\ensurevspace{4\baselineskip}
\begin{leftcolumn*}
One doesn't discern, 'This is the practice leading to the cessation of suffering.'
\end{leftcolumn*}

\begin{rightcolumn}
‘ayaṃ dukkhanirodhagāminī paṭipadā’ti nappajānāti.
\end{rightcolumn}
\end{samepage}

\begin{samepage}
\ensurevspace{4\baselineskip}
\begin{leftcolumn*}
'One doesn't discern, one doesn't discern': Thus one is said to be 'one of poor discernment.'"
\end{leftcolumn*}

\begin{rightcolumn}
‘nappajānāti nappajānātī’ti kho, āvuso, tasmā duppaññoti vuccatī”ti.
\end{rightcolumn}
\end{samepage}

\begin{samepage}
\ensurevspace{4\baselineskip}
\begin{leftcolumn*}
-
\end{leftcolumn*}

\begin{rightcolumn}
-
\end{rightcolumn}
\end{samepage}

\begin{samepage}
\ensurevspace{4\baselineskip}
\begin{leftcolumn*}
Saying, "Very good, friend," Ven. Maha Kotthita — delighting in and approving of Ven. Sariputta's statement — asked him a further question:
\end{leftcolumn*}

\begin{rightcolumn}
“‘sādhāvuso’ti kho āyasmā mahākoṭṭhiko āyasmato sāriputtassa bhāsitaṃ abhinanditvā anumoditvā āyasmantaṃ sāriputtaṃ uttariṃ pañhaṃ apucchi —
\end{rightcolumn}
\end{samepage}

\begin{samepage}
\ensurevspace{4\baselineskip}
\begin{leftcolumn*}
"Discerning, discerning': Thus is it said. To what extent, friend, is one said to be 'discerning'?"
\end{leftcolumn*}

\begin{rightcolumn}
“‘paññavā paññavā’ti, āvuso, vuccati. kittāvatā nu kho, āvuso, paññavāti vuccatī”ti?
\end{rightcolumn}
\end{samepage}

\begin{samepage}
\ensurevspace{4\baselineskip}
\begin{leftcolumn*}
"'One discerns, one discerns': Thus, friend, one is said to be 'discerning.'
\end{leftcolumn*}

\begin{rightcolumn}
“‘pajānāti pajānātī’ti kho, āvuso, tasmā paññavāti vuccati.
\end{rightcolumn}
\end{samepage}

\begin{samepage}
\ensurevspace{4\baselineskip}
\begin{leftcolumn*}
-
\end{leftcolumn*}

\begin{rightcolumn}
-
\end{rightcolumn}
\end{samepage}

\begin{samepage}
\ensurevspace{4\baselineskip}
\begin{leftcolumn*}
And what does one discern?
\end{leftcolumn*}

\begin{rightcolumn}
“kiñca pajānāti?
\end{rightcolumn}
\end{samepage}

\begin{samepage}
\ensurevspace{4\baselineskip}
\begin{leftcolumn*}
One discerns, 'This is suffering.'
\end{leftcolumn*}

\begin{rightcolumn}
‘idaṃ dukkhan’ti pajānāti,
\end{rightcolumn}
\end{samepage}

\begin{samepage}
\ensurevspace{4\baselineskip}
\begin{leftcolumn*}
One discerns, 'This is the origination of suffering.'
\end{leftcolumn*}

\begin{rightcolumn}
‘ayaṃ dukkhasamudayo’ti pajānāti,
\end{rightcolumn}
\end{samepage}

\begin{samepage}
\ensurevspace{4\baselineskip}
\begin{leftcolumn*}
One discerns, 'This is the cessation of suffering.'
\end{leftcolumn*}

\begin{rightcolumn}
‘ayaṃ dukkhanirodho’ti pajānāti,
\end{rightcolumn}
\end{samepage}

\begin{samepage}
\ensurevspace{4\baselineskip}
\begin{leftcolumn*}
One discerns, 'This is the practice leading to the cessation of suffering.'
\end{leftcolumn*}

\begin{rightcolumn}
‘ayaṃ dukkhanirodhagāminī paṭipadā’ti pajānāti.
\end{rightcolumn}
\end{samepage}

\begin{samepage}
\ensurevspace{4\baselineskip}
\begin{leftcolumn*}
'One discerns, one discerns': Thus one is said to be 'discerning.'"
\end{leftcolumn*}

\begin{rightcolumn}
‘pajānāti pajānātī’ti kho, āvuso, tasmā paññavāti vuccatī”ti.
\end{rightcolumn}
\end{samepage}

\begin{samepage}
\ensurevspace{4\baselineskip}
\begin{leftcolumn*}
-
\end{leftcolumn*}

\begin{rightcolumn}
-
\end{rightcolumn}
\end{samepage}

\begin{samepage}
\ensurevspace{4\baselineskip}
\begin{leftcolumn*}
"'Consciousness, consciousness': Thus is it said.
\end{leftcolumn*}

\begin{rightcolumn}
“‘viññāṇaṃ viññāṇan’ti, āvuso, vuccati.
\end{rightcolumn}
\end{samepage}

\begin{samepage}
\ensurevspace{4\baselineskip}
\begin{leftcolumn*}
To what extent, friend, is it said to be 'consciousness'?"
\end{leftcolumn*}

\begin{rightcolumn}
kittāvatā nu kho, āvuso, viññāṇanti vuccatī”ti?
\end{rightcolumn}
\end{samepage}

\begin{samepage}
\ensurevspace{4\baselineskip}
\begin{leftcolumn*}
"'It cognizes, it cognizes': Thus, friend, it is said to be 'consciousness.'
\end{leftcolumn*}

\begin{rightcolumn}
“‘vijānāti vijānātī’ti kho, āvuso, tasmā viññāṇanti vuccati.
\end{rightcolumn}
\end{samepage}

\begin{samepage}
\ensurevspace{4\baselineskip}
\begin{leftcolumn*}
-
\end{leftcolumn*}

\begin{rightcolumn}
-
\end{rightcolumn}
\end{samepage}

\begin{samepage}
\ensurevspace{4\baselineskip}
\begin{leftcolumn*}
And what does it cognize?
\end{leftcolumn*}

\begin{rightcolumn}
“kiñca vijānāti?
\end{rightcolumn}
\end{samepage}

\begin{samepage}
\ensurevspace{4\baselineskip}
\begin{leftcolumn*}
It cognizes 'pleasant.'
\end{leftcolumn*}

\begin{rightcolumn}
sukhantipi vijānāti,
\end{rightcolumn}
\end{samepage}

\begin{samepage}
\ensurevspace{4\baselineskip}
\begin{leftcolumn*}
It cognizes 'painful.'
\end{leftcolumn*}

\begin{rightcolumn}
dukkhantipi vijānāti,
\end{rightcolumn}
\end{samepage}

\begin{samepage}
\ensurevspace{4\baselineskip}
\begin{leftcolumn*}
It cognizes 'neither painful nor pleasant.'
\end{leftcolumn*}

\begin{rightcolumn}
adukkhamasukhantipi vijānāti.
\end{rightcolumn}
\end{samepage}

\begin{samepage}
\ensurevspace{4\baselineskip}
\begin{leftcolumn*}
'It cognizes, it cognizes': Thus it is said to be 'consciousness.'"
\end{leftcolumn*}

\begin{rightcolumn}
‘vijānāti vijānātī’ti kho, āvuso, tasmā viññāṇanti vuccatī”ti.
\end{rightcolumn}
\end{samepage}

\begin{samepage}
\ensurevspace{4\baselineskip}
\begin{leftcolumn*}
-
\end{leftcolumn*}

\begin{rightcolumn}
-
\end{rightcolumn}
\end{samepage}

\begin{samepage}
\ensurevspace{4\baselineskip}
\begin{leftcolumn*}
"Discernment and consciousness, friend: Are these qualities conjoined or disjoined?
\end{leftcolumn*}

\begin{rightcolumn}
“yā cāvuso, paññā yañca viññāṇaṃ — ime dhammā saṃsaṭṭhā udāhu visaṃsaṭṭhā?
\end{rightcolumn}
\end{samepage}

\begin{samepage}
\ensurevspace{4\baselineskip}
\begin{leftcolumn*}
Is it possible, having separated them one from the other, to delineate the difference between them?"
\end{leftcolumn*}

\begin{rightcolumn}
labbhā ca panimesaṃ dhammānaṃ vinibbhujitvā vinibbhujitvā nānākaraṇaṃ paññāpetun”ti?
\end{rightcolumn}
\end{samepage}

\begin{samepage}
\ensurevspace{4\baselineskip}
\begin{leftcolumn*}
"Discernment and consciousness are conjoined, friend, not disjoined.
\end{leftcolumn*}

\begin{rightcolumn}
“yā cāvuso, paññā yañca viññāṇaṃ — ime dhammā saṃsaṭṭhā, no visaṃsaṭṭhā.
\end{rightcolumn}
\end{samepage}

\begin{samepage}
\ensurevspace{4\baselineskip}
\begin{leftcolumn*}
It's not possible, having separated them one from the other, to delineate the difference between them.
\end{leftcolumn*}

\begin{rightcolumn}
na ca labbhā imesaṃ dhammānaṃ vinibbhujitvā vinibbhujitvā nānākaraṇaṃ paññāpetuṃ.
\end{rightcolumn}
\end{samepage}

\begin{samepage}
\ensurevspace{4\baselineskip}
\begin{leftcolumn*}
For what one discerns, that one cognizes. What one cognizes, that one discerns.
\end{leftcolumn*}

\begin{rightcolumn}
yaṃ hāvuso, pajānāti taṃ vijānāti, yaṃ vijānāti taṃ pajānāti.
\end{rightcolumn}
\end{samepage}

\begin{samepage}
\ensurevspace{4\baselineskip}
\begin{leftcolumn*}
Therefore these qualities are conjoined, not disjoined, and it is not possible, having separated them one from another, to delineate the difference between them."
\end{leftcolumn*}

\begin{rightcolumn}
tasmā ime dhammā saṃsaṭṭhā, no visaṃsaṭṭhā. na ca labbhā imesaṃ dhammānaṃ vinibbhujitvā vinibbhujitvā nānākaraṇaṃ paññāpetun”ti.
\end{rightcolumn}
\end{samepage}

\begin{samepage}
\ensurevspace{4\baselineskip}
\begin{leftcolumn*}
-
\end{leftcolumn*}

\begin{rightcolumn}
-
\end{rightcolumn}
\end{samepage}

\begin{samepage}
\ensurevspace{4\baselineskip}
\begin{leftcolumn*}
"Discernment and consciousness, friend: What is the difference between these qualities that are conjoined, not disjoined?"
\end{leftcolumn*}

\begin{rightcolumn}
“yā cāvuso, paññā yañca viññāṇaṃ — imesaṃ dhammānaṃ saṃsaṭṭhānaṃ no visaṃsaṭṭhānaṃ kiṃ nānākaraṇan”ti?
\end{rightcolumn}
\end{samepage}

\begin{samepage}
\ensurevspace{4\baselineskip}
\begin{leftcolumn*}
"Discernment and consciousness, friend: Of these qualities that are conjoined, not disjoined, discernment is to be developed, consciousness is to be fully comprehended."
\end{leftcolumn*}

\begin{rightcolumn}
“yā cāvuso, paññā yañca viññāṇaṃ — imesaṃ dhammānaṃ saṃsaṭṭhānaṃ no visaṃsaṭṭhānaṃ paññā bhāvetabbā, viññāṇaṃ pariññeyyaṃ. idaṃ nesaṃ nānākaraṇan”ti.
\end{rightcolumn}
\end{samepage}

\begin{samepage}
\ensurevspace{4\baselineskip}
\begin{leftcolumn*}
-
\end{leftcolumn*}

\begin{rightcolumn}
-
\end{rightcolumn}
\end{samepage}

\begin{samepage}
\ensurevspace{4\baselineskip}
\begin{leftcolumn*}
"'Feeling, feeling': Thus is it said.
\end{leftcolumn*}

\begin{rightcolumn}
“‘vedanā vedanā’ti, āvuso, vuccati.
\end{rightcolumn}
\end{samepage}

\begin{samepage}
\ensurevspace{4\baselineskip}
\begin{leftcolumn*}
To what extent, friend, is it said to be 'feeling'?"
\end{leftcolumn*}

\begin{rightcolumn}
kittāvatā nu kho, āvuso, vedanāti vuccatī”ti?
\end{rightcolumn}
\end{samepage}

\begin{samepage}
\ensurevspace{4\baselineskip}
\begin{leftcolumn*}
"'It feels, it feels': Thus, friend, it is said to be 'feeling.'
\end{leftcolumn*}

\begin{rightcolumn}
“‘vedeti vedetī’ti kho, āvuso, tasmā vedanāti vuccati.
\end{rightcolumn}
\end{samepage}

\begin{samepage}
\ensurevspace{4\baselineskip}
\begin{leftcolumn*}
-
\end{leftcolumn*}

\begin{rightcolumn}
-
\end{rightcolumn}
\end{samepage}

\begin{samepage}
\ensurevspace{4\baselineskip}
\begin{leftcolumn*}
And what does it feel?
\end{leftcolumn*}

\begin{rightcolumn}
“kiñca vedeti?
\end{rightcolumn}
\end{samepage}

\begin{samepage}
\ensurevspace{4\baselineskip}
\begin{leftcolumn*}
It feels pleasure.
\end{leftcolumn*}

\begin{rightcolumn}
sukhampi vedeti,
\end{rightcolumn}
\end{samepage}

\begin{samepage}
\ensurevspace{4\baselineskip}
\begin{leftcolumn*}
It feels pain.
\end{leftcolumn*}

\begin{rightcolumn}
dukkhampi vedeti,
\end{rightcolumn}
\end{samepage}

\begin{samepage}
\ensurevspace{4\baselineskip}
\begin{leftcolumn*}
It feels neither pleasure nor pain.
\end{leftcolumn*}

\begin{rightcolumn}
adukkhamasukhampi vedeti.
\end{rightcolumn}
\end{samepage}

\begin{samepage}
\ensurevspace{4\baselineskip}
\begin{leftcolumn*}
'It feels, it feels': Thus it is said to be 'feeling.'"
\end{leftcolumn*}

\begin{rightcolumn}
‘vedeti vedetī’ti kho, āvuso, tasmā vedanāti vuccatī”ti.
\end{rightcolumn}
\end{samepage}

\begin{samepage}
\ensurevspace{4\baselineskip}
\begin{leftcolumn*}
-
\end{leftcolumn*}

\begin{rightcolumn}
-
\end{rightcolumn}
\end{samepage}

\begin{samepage}
\ensurevspace{4\baselineskip}
\begin{leftcolumn*}
"'Perception, perception': Thus is it said.
\end{leftcolumn*}

\begin{rightcolumn}
“‘saññā saññā’ti, āvuso, vuccati.
\end{rightcolumn}
\end{samepage}

\begin{samepage}
\ensurevspace{4\baselineskip}
\begin{leftcolumn*}
To what extent, friend, is it said to be 'perception'?"
\end{leftcolumn*}

\begin{rightcolumn}
kittāvatā nu kho, āvuso, saññāti vuccatī”ti?
\end{rightcolumn}
\end{samepage}

\begin{samepage}
\ensurevspace{4\baselineskip}
\begin{leftcolumn*}
"'It perceives, it perceives': Thus, friend, it is said to be 'perception.'
\end{leftcolumn*}

\begin{rightcolumn}
“‘sañjānāti sañjānātī’ti kho, āvuso, tasmā saññāti vuccati.
\end{rightcolumn}
\end{samepage}

\begin{samepage}
\ensurevspace{4\baselineskip}
\begin{leftcolumn*}
-
\end{leftcolumn*}

\begin{rightcolumn}
-
\end{rightcolumn}
\end{samepage}

\begin{samepage}
\ensurevspace{4\baselineskip}
\begin{leftcolumn*}
And what does it perceive?
\end{leftcolumn*}

\begin{rightcolumn}
“kiñca sañjānāti?
\end{rightcolumn}
\end{samepage}

\begin{samepage}
\ensurevspace{4\baselineskip}
\begin{leftcolumn*}
It perceives blue.
\end{leftcolumn*}

\begin{rightcolumn}
nīlakampi sañjānāti,
\end{rightcolumn}
\end{samepage}

\begin{samepage}
\ensurevspace{4\baselineskip}
\begin{leftcolumn*}
It perceives yellow.
\end{leftcolumn*}

\begin{rightcolumn}
pītakampi sañjānāti,
\end{rightcolumn}
\end{samepage}

\begin{samepage}
\ensurevspace{4\baselineskip}
\begin{leftcolumn*}
It perceives red.
\end{leftcolumn*}

\begin{rightcolumn}
lohitakampi sañjānāti,
\end{rightcolumn}
\end{samepage}

\begin{samepage}
\ensurevspace{4\baselineskip}
\begin{leftcolumn*}
It perceives white.
\end{leftcolumn*}

\begin{rightcolumn}
odātampi sañjānāti.
\end{rightcolumn}
\end{samepage}

\begin{samepage}
\ensurevspace{4\baselineskip}
\begin{leftcolumn*}
'It perceives, it perceives': Thus it is said to be 'perception.'"
\end{leftcolumn*}

\begin{rightcolumn}
‘sañjānāti sañjānātī’ti kho, āvuso, tasmā saññāti vuccatī”ti.
\end{rightcolumn}
\end{samepage}

\begin{samepage}
\ensurevspace{4\baselineskip}
\begin{leftcolumn*}
-
\end{leftcolumn*}

\begin{rightcolumn}
-
\end{rightcolumn}
\end{samepage}

\begin{samepage}
\ensurevspace{4\baselineskip}
\begin{leftcolumn*}
"Feeling, perception, and consciousness, friend: Are these qualities conjoined or disjoined?
\end{leftcolumn*}

\begin{rightcolumn}
“yā cāvuso, vedanā yā ca saññā yañca viññāṇaṃ — ime dhammā saṃsaṭṭhā udāhu visaṃsaṭṭhā?
\end{rightcolumn}
\end{samepage}

\begin{samepage}
\ensurevspace{4\baselineskip}
\begin{leftcolumn*}
Is it possible, having separated them one from another, to delineate the difference among them?"
\end{leftcolumn*}

\begin{rightcolumn}
labbhā ca panimesaṃ dhammānaṃ vinibbhujitvā vinibbhujitvā nānākaraṇaṃ paññāpetun”ti?
\end{rightcolumn}
\end{samepage}

\begin{samepage}
\ensurevspace{4\baselineskip}
\begin{leftcolumn*}
"Feeling, perception, and consciousness are conjoined, friend, not disjoined.
\end{leftcolumn*}

\begin{rightcolumn}
“yā cāvuso, vedanā yā ca saññā yañca viññāṇaṃ — ime dhammā saṃsaṭṭhā, no visaṃsaṭṭhā.
\end{rightcolumn}
\end{samepage}

\begin{samepage}
\ensurevspace{4\baselineskip}
\begin{leftcolumn*}
It is not possible, having separated them one from another, to delineate the difference among them.
\end{leftcolumn*}

\begin{rightcolumn}
na ca labbhā imesaṃ dhammānaṃ vinibbhujitvā vinibbhujitvā nānākaraṇaṃ paññāpetuṃ.
\end{rightcolumn}
\end{samepage}

\begin{samepage}
\ensurevspace{4\baselineskip}
\begin{leftcolumn*}
For what one feels, that one perceives.
\end{leftcolumn*}

\begin{rightcolumn}
yaṃ hāvuso, vedeti taṃ sañjānāti,
\end{rightcolumn}
\end{samepage}

\begin{samepage}
\ensurevspace{4\baselineskip}
\begin{leftcolumn*}
What one perceives, that one cognizes.
\end{leftcolumn*}

\begin{rightcolumn}
yaṃ sañjānāti taṃ vijānāti.
\end{rightcolumn}
\end{samepage}

\begin{samepage}
\ensurevspace{4\baselineskip}
\begin{leftcolumn*}
Therefore these qualities are conjoined, not disjoined, and it is not possible, having separated them one from another, to delineate the difference among them."
\end{leftcolumn*}

\begin{rightcolumn}
tasmā ime dhammā saṃsaṭṭhā no visaṃsaṭṭhā. na ca labbhā imesaṃ dhammānaṃ vinibbhujitvā vinibbhujitvā nānākaraṇaṃ paññāpetun”ti.
\end{rightcolumn}
\end{samepage}

\begin{samepage}
\ensurevspace{4\baselineskip}
\begin{leftcolumn*}
-
\end{leftcolumn*}

\begin{rightcolumn}
-
\end{rightcolumn}
\end{samepage}

\begin{samepage}
\ensurevspace{4\baselineskip}
\begin{leftcolumn*}
"Friend, what can be known with the purified mind-consciousness divorced from the five [sense] faculties?"
\end{leftcolumn*}

\begin{rightcolumn}
“nissaṭṭhena hāvuso, pañcahi indriyehi parisuddhena manoviññāṇena kiṃ neyyan”ti?
\end{rightcolumn}
\end{samepage}

\begin{samepage}
\ensurevspace{4\baselineskip}
\begin{leftcolumn*}
"Friend, with the purified mind-consciousness divorced from the five faculties
\end{leftcolumn*}

\begin{rightcolumn}
“nissaṭṭhena āvuso, pañcahi indriyehi parisuddhena manoviññāṇena
\end{rightcolumn}
\end{samepage}

\begin{samepage}
\ensurevspace{4\baselineskip}
\begin{leftcolumn*}
the dimension of the boundless of space can be known [as] 'boundless space.'
\end{leftcolumn*}

\begin{rightcolumn}
‘ananto ākāso’ti ākāsānañcāyatanaṃ neyyaṃ,
\end{rightcolumn}
\end{samepage}

\begin{samepage}
\ensurevspace{4\baselineskip}
\begin{leftcolumn*}
The dimension of the boundless of consciousness can be known [as] 'boundless consciousness.'
\end{leftcolumn*}

\begin{rightcolumn}
‘anantaṃ viññāṇan’ti viññāṇañcāyatanaṃ neyyaṃ,
\end{rightcolumn}
\end{samepage}

\begin{samepage}
\ensurevspace{4\baselineskip}
\begin{leftcolumn*}
The dimension of nothingness can be known [as] 'There is nothing.'
\end{leftcolumn*}

\begin{rightcolumn}
‘natthi kiñcī’ti ākiñcaññāyatanaṃ neyyan”ti.
\end{rightcolumn}
\end{samepage}

\begin{samepage}
\ensurevspace{4\baselineskip}
\begin{leftcolumn*}
-
\end{leftcolumn*}

\begin{rightcolumn}
-
\end{rightcolumn}
\end{samepage}

\begin{samepage}
\ensurevspace{4\baselineskip}
\begin{leftcolumn*}
"With what does one know a quality that can be known?"
\end{leftcolumn*}

\begin{rightcolumn}
“neyyaṃ panāvuso, dhammaṃ kena pajānātī”ti?
\end{rightcolumn}
\end{samepage}

\begin{samepage}
\ensurevspace{4\baselineskip}
\begin{leftcolumn*}
"One knows a quality that can be known with the eye of discernment."
\end{leftcolumn*}

\begin{rightcolumn}
“neyyaṃ kho, āvuso, dhammaṃ paññācakkhunā pajānātī”ti.
\end{rightcolumn}
\end{samepage}

\begin{samepage}
\ensurevspace{4\baselineskip}
\begin{leftcolumn*}
-
\end{leftcolumn*}

\begin{rightcolumn}
-
\end{rightcolumn}
\end{samepage}

\begin{samepage}
\ensurevspace{4\baselineskip}
\begin{leftcolumn*}
"And what is the purpose of discernment?"
\end{leftcolumn*}

\begin{rightcolumn}
“paññā panāvuso, kimatthiyā”ti?
\end{rightcolumn}
\end{samepage}

\begin{samepage}
\ensurevspace{4\baselineskip}
\begin{leftcolumn*}
"The purpose of discernment is direct knowledge, its purpose is full comprehension, its purpose is abandoning."
\end{leftcolumn*}

\begin{rightcolumn}
“paññā kho, āvuso, abhiññatthā pariññatthā pahānatthā”ti.
\end{rightcolumn}
\end{samepage}

\begin{samepage}
\ensurevspace{4\baselineskip}
\begin{leftcolumn*}
-
\end{leftcolumn*}

\begin{rightcolumn}
-
\end{rightcolumn}
\end{samepage}

\begin{samepage}
\ensurevspace{4\baselineskip}
\begin{leftcolumn*}
"Friend, how many conditions are there for the arising of right view?"
\end{leftcolumn*}

\begin{rightcolumn}
“kati panāvuso, paccayā sammādiṭṭhiyā uppādāyā”ti?
\end{rightcolumn}
\end{samepage}

\begin{samepage}
\ensurevspace{4\baselineskip}
\begin{leftcolumn*}
"Friend, there are two conditions for the arising of right view: the voice of another and appropriate attention. These are the two conditions for the arising of right view."
\end{leftcolumn*}

\begin{rightcolumn}
“dve kho, āvuso, paccayā sammādiṭṭhiyā uppādāya — parato ca ghoso, yoniso ca manasikāro. ime kho, āvuso, dve paccayā sammādiṭṭhiyā uppādāyā”ti.
\end{rightcolumn}
\end{samepage}

\begin{samepage}
\ensurevspace{4\baselineskip}
\begin{leftcolumn*}
-
\end{leftcolumn*}

\begin{rightcolumn}
-
\end{rightcolumn}
\end{samepage}

\begin{samepage}
\ensurevspace{4\baselineskip}
\begin{leftcolumn*}
"And assisted by how many factors does right view have awareness-release as its fruit and reward, and discernment-release as its fruit and reward?"
\end{leftcolumn*}

\begin{rightcolumn}
“katihi panāvuso, aṅgehi anuggahitā sammādiṭṭhi cetovimuttiphalā ca hoti cetovimuttiphalānisaṃsā ca, paññāvimuttiphalā ca hoti paññāvimuttiphalānisaṃsā cā”ti?
\end{rightcolumn}
\end{samepage}

\begin{samepage}
\ensurevspace{4\baselineskip}
\begin{leftcolumn*}
"Assisted by five factors, right view has awareness-release as its fruit and reward, and discernment-release as its fruit and reward.
\end{leftcolumn*}

\begin{rightcolumn}
“pañcahi kho, āvuso, aṅgehi anuggahitā sammādiṭṭhi cetovimuttiphalā ca hoti cetovimuttiphalānisaṃsā ca, paññāvimuttiphalā ca hoti paññāvimuttiphalānisaṃsā ca.
\end{rightcolumn}
\end{samepage}

\begin{samepage}
\ensurevspace{4\baselineskip}
\begin{leftcolumn*}
There is the case where right view is...
\end{leftcolumn*}

\begin{rightcolumn}
idhāvuso, sammādiṭṭhi...
\end{rightcolumn}
\end{samepage}

\begin{samepage}
\ensurevspace{4\baselineskip}
\begin{leftcolumn*}
assisted by virtue,
\end{leftcolumn*}

\begin{rightcolumn}
sīlānuggahitā ca hoti,
\end{rightcolumn}
\end{samepage}

\begin{samepage}
\ensurevspace{4\baselineskip}
\begin{leftcolumn*}
assisted by learning,
\end{leftcolumn*}

\begin{rightcolumn}
sutānuggahitā ca hoti,
\end{rightcolumn}
\end{samepage}

\begin{samepage}
\ensurevspace{4\baselineskip}
\begin{leftcolumn*}
assisted by discussion,
\end{leftcolumn*}

\begin{rightcolumn}
sākacchānuggahitā ca hoti,
\end{rightcolumn}
\end{samepage}

\begin{samepage}
\ensurevspace{4\baselineskip}
\begin{leftcolumn*}
assisted by tranquility,
\end{leftcolumn*}

\begin{rightcolumn}
samathānuggahitā ca hoti,
\end{rightcolumn}
\end{samepage}

\begin{samepage}
\ensurevspace{4\baselineskip}
\begin{leftcolumn*}
assisted by insight.
\end{leftcolumn*}

\begin{rightcolumn}
vipassanānuggahitā ca hoti.
\end{rightcolumn}
\end{samepage}

\begin{samepage}
\ensurevspace{4\baselineskip}
\begin{leftcolumn*}
Assisted by these five factors, right view has awareness-release as its fruit and reward, and discernment-release as its fruit and reward."
\end{leftcolumn*}

\begin{rightcolumn}
imehi kho, āvuso, pañcahaṅgehi anuggahitā sammādiṭṭhi cetovimuttiphalā ca hoti cetovimuttiphalānisaṃsā ca, paññāvimuttiphalā ca hoti paññāvimuttiphalānisaṃsā cā”ti.
\end{rightcolumn}
\end{samepage}

\begin{samepage}
\ensurevspace{4\baselineskip}
\begin{leftcolumn*}
-
\end{leftcolumn*}

\begin{rightcolumn}
-
\end{rightcolumn}
\end{samepage}

\begin{samepage}
\ensurevspace{4\baselineskip}
\begin{leftcolumn*}
"Friend, how many kinds of becoming are there?"
\end{leftcolumn*}

\begin{rightcolumn}
“kati panāvuso, bhavā”ti?
\end{rightcolumn}
\end{samepage}

\begin{samepage}
\ensurevspace{4\baselineskip}
\begin{leftcolumn*}
"Friend, there are these three kinds of becoming: sensual becoming, form becoming, formless becoming."
\end{leftcolumn*}

\begin{rightcolumn}
“tayome, āvuso, bhavā — kāmabhavo, rūpabhavo, arūpabhavo”ti.
\end{rightcolumn}
\end{samepage}

\begin{samepage}
\ensurevspace{4\baselineskip}
\begin{leftcolumn*}
-
\end{leftcolumn*}

\begin{rightcolumn}
-
\end{rightcolumn}
\end{samepage}

\begin{samepage}
\ensurevspace{4\baselineskip}
\begin{leftcolumn*}
"And how is further becoming in the future brought about?"
\end{leftcolumn*}

\begin{rightcolumn}
“kathaṃ panāvuso, āyatiṃ punabbhavābhinibbatti hotī”ti?
\end{rightcolumn}
\end{samepage}

\begin{samepage}
\ensurevspace{4\baselineskip}
\begin{leftcolumn*}
"The delight, now here, now there, of beings hindered by ignorance and fettered by craving: That's how further becoming in the future is brought about."
\end{leftcolumn*}

\begin{rightcolumn}
“avijjānīvaraṇānaṃ kho, āvuso, sattānaṃ taṇhāsaṃyojanānaṃ tatratatrābhinandanā — evaṃ āyatiṃ punabbhavābhinibbatti hotī”ti.
\end{rightcolumn}
\end{samepage}

\begin{samepage}
\ensurevspace{4\baselineskip}
\begin{leftcolumn*}
-
\end{leftcolumn*}

\begin{rightcolumn}
-
\end{rightcolumn}
\end{samepage}

\begin{samepage}
\ensurevspace{4\baselineskip}
\begin{leftcolumn*}
"And how is further becoming in the future not brought about?"
\end{leftcolumn*}

\begin{rightcolumn}
“kathaṃ panāvuso, āyatiṃ punabbhavābhinibbatti na hotī”ti?
\end{rightcolumn}
\end{samepage}

\begin{samepage}
\ensurevspace{4\baselineskip}
\begin{leftcolumn*}
"Through the fading of ignorance, the arising of clear knowing, and the cessation of craving: That's how further becoming in the future is not brought about."
\end{leftcolumn*}

\begin{rightcolumn}
“avijjāvirāgā kho, āvuso, vijjuppādā taṇhānirodhā — evaṃ āyatiṃ punabbhavābhinibbatti na hotī”ti.
\end{rightcolumn}
\end{samepage}

\begin{samepage}
\ensurevspace{4\baselineskip}
\begin{leftcolumn*}
-
\end{leftcolumn*}

\begin{rightcolumn}
-
\end{rightcolumn}
\end{samepage}

\begin{samepage}
\ensurevspace{4\baselineskip}
\begin{leftcolumn*}
"What, friend, is the first jhana?"
\end{leftcolumn*}

\begin{rightcolumn}
“katamaṃ panāvuso, paṭhamaṃ jhānan”ti?
\end{rightcolumn}
\end{samepage}

\begin{samepage}
\ensurevspace{4\baselineskip}
\begin{leftcolumn*}
"There is the case, friend, where a monk — quite withdrawn from sensual pleasures, withdrawn from unskillful qualities — enters and remains in the first jhana: rapture and pleasure born from withdrawal, accompanied by directed thought and evaluation.
\end{leftcolumn*}

\begin{rightcolumn}
“idhāvuso, bhikkhu vivicceva kāmehi vivicca akusalehi dhammehi savitakkaṃ savicāraṃ vivekajaṃ pītisukhaṃ paṭhamaṃ jhānaṃ upasampajja viharati
\end{rightcolumn}
\end{samepage}

\begin{samepage}
\ensurevspace{4\baselineskip}
\begin{leftcolumn*}
This is called the first jhana."
\end{leftcolumn*}

\begin{rightcolumn}
idaṃ vuccati, āvuso, paṭhamaṃ jhānan”ti.
\end{rightcolumn}
\end{samepage}

\begin{samepage}
\ensurevspace{4\baselineskip}
\begin{leftcolumn*}
-
\end{leftcolumn*}

\begin{rightcolumn}
-
\end{rightcolumn}
\end{samepage}

\begin{samepage}
\ensurevspace{4\baselineskip}
\begin{leftcolumn*}
"And how many factors does the first jhana have?"
\end{leftcolumn*}

\begin{rightcolumn}
“paṭhamaṃ panāvuso, jhānaṃ katiaṅgikan”ti?
\end{rightcolumn}
\end{samepage}

\begin{samepage}
\ensurevspace{4\baselineskip}
\begin{leftcolumn*}
"The first jhana has five factors. There is the case where, in a monk who has attained the five-factored first jhana, there occurs directed thought, evaluation, rapture, pleasure, and singleness of mind.
\end{leftcolumn*}

\begin{rightcolumn}
“paṭhamaṃ kho, āvuso, jhānaṃ pañcaṅgikaṃ. idhāvuso, paṭhamaṃ jhānaṃ samāpannassa bhikkhuno vitakko ca vattati, vicāro ca pīti ca sukhañca cittekaggatā ca.
\end{rightcolumn}
\end{samepage}

\begin{samepage}
\ensurevspace{4\baselineskip}
\begin{leftcolumn*}
It's in this way that the first jhana has five factors."
\end{leftcolumn*}

\begin{rightcolumn}
paṭhamaṃ kho, āvuso, jhānaṃ evaṃ pañcaṅgikan”ti.
\end{rightcolumn}
\end{samepage}

\begin{samepage}
\ensurevspace{4\baselineskip}
\begin{leftcolumn*}
-
\end{leftcolumn*}

\begin{rightcolumn}
-
\end{rightcolumn}
\end{samepage}

\begin{samepage}
\ensurevspace{4\baselineskip}
\begin{leftcolumn*}
"And how many factors are abandoned in the first jhana, and with how many is it endowed?"
\end{leftcolumn*}

\begin{rightcolumn}
“paṭhamaṃ panāvuso, jhānaṃ kataṅgavippahīnaṃ kataṅgasamannāgatan”ti?
\end{rightcolumn}
\end{samepage}

\begin{samepage}
\ensurevspace{4\baselineskip}
\begin{leftcolumn*}
"Five factors are abandoned in the first jhana, and with five is it endowed.
\end{leftcolumn*}

\begin{rightcolumn}
“paṭhamaṃ kho, āvuso, jhānaṃ pañcaṅgavippahīnaṃ, pañcaṅgasamannāgataṃ.
\end{rightcolumn}
\end{samepage}

\begin{samepage}
\ensurevspace{4\baselineskip}
\begin{leftcolumn*}
There is the case where, in a monk who has attained the first jhana,
\end{leftcolumn*}

\begin{rightcolumn}
idhāvuso, paṭhamaṃ jhānaṃ samāpannassa bhikkhuno...
\end{rightcolumn}
\end{samepage}

\begin{samepage}
\ensurevspace{4\baselineskip}
\begin{leftcolumn*}
sensual desire is abandoned,
\end{leftcolumn*}

\begin{rightcolumn}
kāmacchando pahīno hoti,
\end{rightcolumn}
\end{samepage}

\begin{samepage}
\ensurevspace{4\baselineskip}
\begin{leftcolumn*}
ill will is abandoned,
\end{leftcolumn*}

\begin{rightcolumn}
byāpādo pahīno hoti,
\end{rightcolumn}
\end{samepage}

\begin{samepage}
\ensurevspace{4\baselineskip}
\begin{leftcolumn*}
sloth and torpor is abandoned,
\end{leftcolumn*}

\begin{rightcolumn}
thīnamiddhaṃ pahīnaṃ hoti,
\end{rightcolumn}
\end{samepage}

\begin{samepage}
\ensurevspace{4\baselineskip}
\begin{leftcolumn*}
restlessness and anxiety is abandoned,
\end{leftcolumn*}

\begin{rightcolumn}
uddhaccakukkuccaṃ pahīnaṃ hoti,
\end{rightcolumn}
\end{samepage}

\begin{samepage}
\ensurevspace{4\baselineskip}
\begin{leftcolumn*}
uncertainty is abandoned.
\end{leftcolumn*}

\begin{rightcolumn}
vicikicchā pahīnā hoti;
\end{rightcolumn}
\end{samepage}

\begin{samepage}
\ensurevspace{4\baselineskip}
\begin{leftcolumn*}
And there occur directed thought, evaluation, rapture, pleasure, and singleness of mind.
\end{leftcolumn*}

\begin{rightcolumn}
vitakko ca vattati, vicāro ca pīti ca sukhañca cittekaggatā ca.
\end{rightcolumn}
\end{samepage}

\begin{samepage}
\ensurevspace{4\baselineskip}
\begin{leftcolumn*}
It's in this way that five factors are abandoned in the first jhana, and with five it is endowed."
\end{leftcolumn*}

\begin{rightcolumn}
paṭhamaṃ kho, āvuso, jhānaṃ evaṃ pañcaṅgavippahīnaṃ pañcaṅgasamannāgatan”ti.
\end{rightcolumn}
\end{samepage}

\begin{samepage}
\ensurevspace{4\baselineskip}
\begin{leftcolumn*}
-
\end{leftcolumn*}

\begin{rightcolumn}
-
\end{rightcolumn}
\end{samepage}

\begin{samepage}
\ensurevspace{4\baselineskip}
\begin{leftcolumn*}
"Friend, there are these five faculties each with a separate field, a separate domain, and they do not experience one another's field and domain: the eye-faculty, the ear-faculty, the nose-faculty, the tongue-faculty, and the body-faculty.
\end{leftcolumn*}

\begin{rightcolumn}
“pañcimāni, āvuso, indriyāni nānāvisayāni nānāgocarāni, na aññamaññassa gocaravisayaṃ paccanubhonti, seyyathidaṃ — cakkhundriyaṃ, sotindriyaṃ, ghānindriyaṃ, jivhindriyaṃ, kāyindriyaṃ.
\end{rightcolumn}
\end{samepage}

\begin{samepage}
\ensurevspace{4\baselineskip}
\begin{leftcolumn*}
Now what do these five faculties — each with a separate field, a separate domain, not experiencing one another's field and domain,
\end{leftcolumn*}

\begin{rightcolumn}
imesaṃ kho, āvuso, pañcannaṃ indriyānaṃ nānāvisayānaṃ nānāgocarānaṃ, na aññamaññassa gocaravisayaṃ paccanubhontānaṃ,
\end{rightcolumn}
\end{samepage}

\begin{samepage}
\ensurevspace{4\baselineskip}
\begin{leftcolumn*}
What experiences [all] their fields and domains?"
\end{leftcolumn*}

\begin{rightcolumn}
kiṃ paṭisaraṇaṃ, ko ca nesaṃ gocaravisayaṃ paccanubhotī”ti?
\end{rightcolumn}
\end{samepage}

\begin{samepage}
\ensurevspace{4\baselineskip}
\begin{leftcolumn*}
"Friend, these five faculties — each with a separate field, a separate domain, not experiencing one another's field and domain: the eye-faculty, the ear-faculty, the nose-faculty, the tongue-faculty, and the body-faculty."
\end{leftcolumn*}

\begin{rightcolumn}
“pañcimāni, āvuso, indriyāni nānāvisayāni nānāgocarāni, na aññamaññassa gocaravisayaṃ paccanubhonti, seyyathidaṃ — cakkhundriyaṃ, sotindriyaṃ, ghānindriyaṃ, jivhindriyaṃ, kāyindriyaṃ.
\end{rightcolumn}
\end{samepage}

\begin{samepage}
\ensurevspace{4\baselineskip}
\begin{leftcolumn*}
"Friend, these five faculties — each with a separate field, a separate domain, not experiencing one another's field and domain: have the mind as their [common] arbitrator. The mind is what experiences [all] their fields and domains."
\end{leftcolumn*}

\begin{rightcolumn}
imesaṃ kho, āvuso, pañcannaṃ indriyānaṃ nānāvisayānaṃ nānāgocarānaṃ, na aññamaññassa gocaravisayaṃ paccanubhontānaṃ, mano paṭisaraṇaṃ, mano ca nesaṃ gocaravisayaṃ paccanubhotī”ti.
\end{rightcolumn}
\end{samepage}

\begin{samepage}
\ensurevspace{4\baselineskip}
\begin{leftcolumn*}
-
\end{leftcolumn*}

\begin{rightcolumn}
-
\end{rightcolumn}
\end{samepage}

\begin{samepage}
\ensurevspace{4\baselineskip}
\begin{leftcolumn*}
"Now, these five faculties — the eye-faculty, the ear-faculty, the nose-faculty, the tongue-faculty, and the body-faculty:
\end{leftcolumn*}

\begin{rightcolumn}
“pañcimāni, āvuso, indriyāni, seyyathidaṃ — cakkhundriyaṃ, sotindriyaṃ, ghānindriyaṃ, jivhindriyaṃ, kāyindriyaṃ.
\end{rightcolumn}
\end{samepage}

\begin{samepage}
\ensurevspace{4\baselineskip}
\begin{leftcolumn*}
In dependence on what do they remain standing?"
\end{leftcolumn*}

\begin{rightcolumn}
imāni kho, āvuso, pañcindriyāni kiṃ paṭicca tiṭṭhantī”ti?
\end{rightcolumn}
\end{samepage}

\begin{samepage}
\ensurevspace{4\baselineskip}
\begin{leftcolumn*}
"These five faculties — the eye-faculty, the ear-faculty, the nose-faculty, the tongue-faculty, and the body-faculty —
\end{leftcolumn*}

\begin{rightcolumn}
“pañcimāni, āvuso, indriyāni, seyyathidaṃ — cakkhundriyaṃ, sotindriyaṃ, ghānindriyaṃ, jivhindriyaṃ, kāyindriyaṃ.
\end{rightcolumn}
\end{samepage}

\begin{samepage}
\ensurevspace{4\baselineskip}
\begin{leftcolumn*}
remain standing in dependence on vitality."
\end{leftcolumn*}

\begin{rightcolumn}
imāni kho, āvuso, pañcindriyāni āyuṃ paṭicca tiṭṭhantī”ti.
\end{rightcolumn}
\end{samepage}

\begin{samepage}
\ensurevspace{4\baselineskip}
\begin{leftcolumn*}
-
\end{leftcolumn*}

\begin{rightcolumn}
-
\end{rightcolumn}
\end{samepage}

\begin{samepage}
\ensurevspace{4\baselineskip}
\begin{leftcolumn*}
"And vitality remains standing in dependence on what?"
\end{leftcolumn*}

\begin{rightcolumn}
“āyu panāvuso, kiṃ paṭicca tiṭṭhatī”ti?
\end{rightcolumn}
\end{samepage}

\begin{samepage}
\ensurevspace{4\baselineskip}
\begin{leftcolumn*}
"Vitality remains standing in dependence on heat."
\end{leftcolumn*}

\begin{rightcolumn}
“āyu usmaṃ paṭicca tiṭṭhatī”ti.
\end{rightcolumn}
\end{samepage}

\begin{samepage}
\ensurevspace{4\baselineskip}
\begin{leftcolumn*}
-
\end{leftcolumn*}

\begin{rightcolumn}
-
\end{rightcolumn}
\end{samepage}

\begin{samepage}
\ensurevspace{4\baselineskip}
\begin{leftcolumn*}
"And heat remains standing in dependence on what?"
\end{leftcolumn*}

\begin{rightcolumn}
“usmā panāvuso, kiṃ paṭicca tiṭṭhatī”ti?
\end{rightcolumn}
\end{samepage}

\begin{samepage}
\ensurevspace{4\baselineskip}
\begin{leftcolumn*}
"Heat remains standing in dependence on vitality."
\end{leftcolumn*}

\begin{rightcolumn}
“usmā āyuṃ paṭicca tiṭṭhatī”ti.
\end{rightcolumn}
\end{samepage}

\begin{samepage}
\ensurevspace{4\baselineskip}
\begin{leftcolumn*}
-
\end{leftcolumn*}

\begin{rightcolumn}
-
\end{rightcolumn}
\end{samepage}

\begin{samepage}
\ensurevspace{4\baselineskip}
\begin{leftcolumn*}
"Just now, friend Sariputta, we understood you to say,
\end{leftcolumn*}

\begin{rightcolumn}
“idāneva kho mayaṃ, āvuso, āyasmato sāriputtassa bhāsitaṃ evaṃ ājānāma —
\end{rightcolumn}
\end{samepage}

\begin{samepage}
\ensurevspace{4\baselineskip}
\begin{leftcolumn*}
'Vitality remains standing in dependence on heat.'
\end{leftcolumn*}

\begin{rightcolumn}
‘āyu usmaṃ paṭicca tiṭṭhatī’ti.
\end{rightcolumn}
\end{samepage}

\begin{samepage}
\ensurevspace{4\baselineskip}
\begin{leftcolumn*}
And just now we understood you to say,
\end{leftcolumn*}

\begin{rightcolumn}
idāneva pana mayaṃ, āvuso, āyasmato sāriputtassa bhāsitaṃ evaṃ ājānāma —
\end{rightcolumn}
\end{samepage}

\begin{samepage}
\ensurevspace{4\baselineskip}
\begin{leftcolumn*}
'Heat remains standing in dependence on vitality.'
\end{leftcolumn*}

\begin{rightcolumn}
‘usmā āyuṃ paṭicca tiṭṭhatī’ti.
\end{rightcolumn}
\end{samepage}

\begin{samepage}
\ensurevspace{4\baselineskip}
\begin{leftcolumn*}
Now how is the meaning of these statements to be seen?"
\end{leftcolumn*}

\begin{rightcolumn}
“yathā kathaṃ panāvuso, imassa bhāsitassa attho daṭṭhabbo”ti?
\end{rightcolumn}
\end{samepage}

\begin{samepage}
\ensurevspace{4\baselineskip}
\begin{leftcolumn*}
"In that case, friend, I will give you analogy, for there are cases where it is through an analogy that an intelligent person understands the meaning of a statement.
\end{leftcolumn*}

\begin{rightcolumn}
“tena hāvuso, upamaṃ te karissāmi; upamāyapidhekacce viññū purisā bhāsitassa atthaṃ ājānanti.
\end{rightcolumn}
\end{samepage}

\begin{samepage}
\ensurevspace{4\baselineskip}
\begin{leftcolumn*}
Suppose an oil lamp is burning. Its radiance is discerned in dependence on its flame, and its flame is discerned in dependence on its radiance.
\end{leftcolumn*}

\begin{rightcolumn}
seyyathāpi, āvuso, telappadīpassa jhāyato acciṃ paṭicca ābhā paññāyati, ābhaṃ paṭicca acci paññāyati;
\end{rightcolumn}
\end{samepage}

\begin{samepage}
\ensurevspace{4\baselineskip}
\begin{leftcolumn*}
In the same way, vitality remains standing in dependence on heat, and heat remains standing in dependence on vitality.
\end{leftcolumn*}

\begin{rightcolumn}
evameva kho, āvuso, āyu usmaṃ paṭicca tiṭṭhati, usmā āyuṃ paṭicca tiṭṭhatī”ti.
\end{rightcolumn}
\end{samepage}

\begin{samepage}
\ensurevspace{4\baselineskip}
\begin{leftcolumn*}
-
\end{leftcolumn*}

\begin{rightcolumn}
-
\end{rightcolumn}
\end{samepage}

\begin{samepage}
\ensurevspace{4\baselineskip}
\begin{leftcolumn*}
"Friend, are vitality-fabrications the same thing as feeling-states?
Or are vitality-fabrications one thing, and feeling-states another?"
\end{leftcolumn*}

\begin{rightcolumn}
“teva nu kho, āvuso, āyusaṅkhārā, te vedaniyā dhammā udāhu aññe āyusaṅkhārā aññe vedaniyā dhammā”ti?
“na kho, āvuso, teva āyusaṅkhārā te vedaniyā dhammā.
\end{rightcolumn}
\end{samepage}

\begin{samepage}
\ensurevspace{4\baselineskip}
\begin{leftcolumn*}
"Vitality-fabrications are not the same thing as feeling-states, friend. If vitality-fabrications were the same thing as feeling-states, the emergence of a monk from the attainment of the cessation of feeling and perception would not be discerned.
\end{leftcolumn*}

\begin{rightcolumn}
te ca hāvuso, āyusaṅkhārā abhaviṃsu te vedaniyā dhammā, na yidaṃ saññāvedayitanirodhaṃ samāpannassa bhikkhuno vuṭṭhānaṃ paññāyetha.
\end{rightcolumn}
\end{samepage}

\begin{samepage}
\ensurevspace{4\baselineskip}
\begin{leftcolumn*}
It's because vitality-fabrications are one thing and feeling-states another that the emergence of a monk from the attainment of the cessation of perception and feeling is discerned."
\end{leftcolumn*}

\begin{rightcolumn}
yasmā ca kho, āvuso, aññe āyusaṅkhārā aññe vedaniyā dhammā, tasmā saññāvedayitanirodhaṃ samāpannassa bhikkhuno vuṭṭhānaṃ paññāyatī”ti.
\end{rightcolumn}
\end{samepage}

\begin{samepage}
\ensurevspace{4\baselineskip}
\begin{leftcolumn*}
-
\end{leftcolumn*}

\begin{rightcolumn}
-
\end{rightcolumn}
\end{samepage}

\begin{samepage}
\ensurevspace{4\baselineskip}
\begin{leftcolumn*}
"When this body lacks how many qualities does it lie discarded and forsaken, like a senseless log?"
\end{leftcolumn*}

\begin{rightcolumn}
“yadā nu kho, āvuso, imaṃ kāyaṃ kati dhammā jahanti; athāyaṃ kāyo ujjhito avakkhitto seti, yathā kaṭṭhaṃ acetanan”ti?
\end{rightcolumn}
\end{samepage}

\begin{samepage}
\ensurevspace{4\baselineskip}
\begin{leftcolumn*}
"When this body lacks these three qualities — vitality, heat, and consciousness — it lies discarded and forsaken like a senseless log."
\end{leftcolumn*}

\begin{rightcolumn}
“yadā kho, āvuso, imaṃ kāyaṃ tayo dhammā jahanti — āyu usmā ca viññāṇaṃ; athāyaṃ kāyo ujjhito avakkhitto seti, yathā kaṭṭhaṃ acetanan”ti.
\end{rightcolumn}
\end{samepage}

\begin{samepage}
\ensurevspace{4\baselineskip}
\begin{leftcolumn*}
-
\end{leftcolumn*}

\begin{rightcolumn}
-
\end{rightcolumn}
\end{samepage}

\begin{samepage}
\ensurevspace{4\baselineskip}
\begin{leftcolumn*}
"What is the difference between one who is dead, who has completed his time, and a monk who has attained the cessation of perception and feeling?"
\end{leftcolumn*}

\begin{rightcolumn}
“yvāyaṃ, āvuso, mato kālaṅkato, yo cāyaṃ bhikkhu saññāvedayitanirodhaṃ samāpanno — imesaṃ kiṃ nānākaraṇan”ti?
\end{rightcolumn}
\end{samepage}

\begin{samepage}
\ensurevspace{4\baselineskip}
\begin{leftcolumn*}
"In the case of the one who is dead, who has completed his time,
\end{leftcolumn*}

\begin{rightcolumn}
“yvāyaṃ, āvuso, mato kālaṅkato tassa
\end{rightcolumn}
\end{samepage}

\begin{samepage}
\ensurevspace{4\baselineskip}
\begin{leftcolumn*}
his bodily fabrications have ceased and subsided,
\end{leftcolumn*}

\begin{rightcolumn}
kāyasaṅkhārā niruddhā paṭippassaddhā,
\end{rightcolumn}
\end{samepage}

\begin{samepage}
\ensurevspace{4\baselineskip}
\begin{leftcolumn*}
his verbal fabrications have ceased and subsided,
\end{leftcolumn*}

\begin{rightcolumn}
vacīsaṅkhārā niruddhā paṭippassaddhā,
\end{rightcolumn}
\end{samepage}

\begin{samepage}
\ensurevspace{4\baselineskip}
\begin{leftcolumn*}
his mental fabrications have ceased and subsided,
\end{leftcolumn*}

\begin{rightcolumn}
cittasaṅkhārā niruddhā paṭippassaddhā,
\end{rightcolumn}
\end{samepage}

\begin{samepage}
\ensurevspace{4\baselineskip}
\begin{leftcolumn*}
his vitality is exhausted,
\end{leftcolumn*}

\begin{rightcolumn}
āyu parikkhīṇo,
\end{rightcolumn}
\end{samepage}

\begin{samepage}
\ensurevspace{4\baselineskip}
\begin{leftcolumn*}
his heat subsided,
\end{leftcolumn*}

\begin{rightcolumn}
usmā vūpasantā,
\end{rightcolumn}
\end{samepage}

\begin{samepage}
\ensurevspace{4\baselineskip}
\begin{leftcolumn*}
and his faculties are scattered.
\end{leftcolumn*}

\begin{rightcolumn}
indriyāni paribhinnāni.
\end{rightcolumn}
\end{samepage}

\begin{samepage}
\ensurevspace{4\baselineskip}
\begin{leftcolumn*}
But in the case of a monk who has attained the cessation of perception and feeling,
\end{leftcolumn*}

\begin{rightcolumn}
yo cāyaṃ bhikkhu saññāvedayitanirodhaṃ samāpanno tassapi
\end{rightcolumn}
\end{samepage}

\begin{samepage}
\ensurevspace{4\baselineskip}
\begin{leftcolumn*}
his bodily fabrications have ceased and subsided,
\end{leftcolumn*}

\begin{rightcolumn}
kāyasaṅkhārā niruddhā paṭippassaddhā,
\end{rightcolumn}
\end{samepage}

\begin{samepage}
\ensurevspace{4\baselineskip}
\begin{leftcolumn*}
his verbal fabrications have ceased and subsided,
\end{leftcolumn*}

\begin{rightcolumn}
vacīsaṅkhārā niruddhā paṭippassaddhā,
\end{rightcolumn}
\end{samepage}

\begin{samepage}
\ensurevspace{4\baselineskip}
\begin{leftcolumn*}
his mental fabrications have ceased and subsided,
\end{leftcolumn*}

\begin{rightcolumn}
cittasaṅkhārā niruddhā paṭippassaddhā,
\end{rightcolumn}
\end{samepage}

\begin{samepage}
\ensurevspace{4\baselineskip}
\begin{leftcolumn*}
his vitality is not exhausted,
\end{leftcolumn*}

\begin{rightcolumn}
āyu na parikkhīṇo,
\end{rightcolumn}
\end{samepage}

\begin{samepage}
\ensurevspace{4\baselineskip}
\begin{leftcolumn*}
his heat has not subsided,
\end{leftcolumn*}

\begin{rightcolumn}
usmā avūpasantā,
\end{rightcolumn}
\end{samepage}

\begin{samepage}
\ensurevspace{4\baselineskip}
\begin{leftcolumn*}
and his faculties are exceptionally clear.
\end{leftcolumn*}

\begin{rightcolumn}
indriyāni vippasannāni.
\end{rightcolumn}
\end{samepage}

\begin{samepage}
\ensurevspace{4\baselineskip}
\begin{leftcolumn*}
This is the difference between one who is dead, who has completed his time, and a monk who has attained the cessation of perception and feeling."
\end{leftcolumn*}

\begin{rightcolumn}
yvāyaṃ, āvuso, mato kālaṅkato, yo cāyaṃ bhikkhu saññāvedayitanirodhaṃ samāpanno — idaṃ nesaṃ nānākaraṇan”ti.
\end{rightcolumn}
\end{samepage}

\begin{samepage}
\ensurevspace{4\baselineskip}
\begin{leftcolumn*}
-
\end{leftcolumn*}

\begin{rightcolumn}
-
\end{rightcolumn}
\end{samepage}

\begin{samepage}
\ensurevspace{4\baselineskip}
\begin{leftcolumn*}
"Friend, how many conditions are there for the attainment of the neither-pleasant-nor-painful awareness-release?"
\end{leftcolumn*}

\begin{rightcolumn}
“kati panāvuso, paccayā adukkhamasukhāya cetovimuttiyā samāpattiyā”ti?
\end{rightcolumn}
\end{samepage}

\begin{samepage}
\ensurevspace{4\baselineskip}
\begin{leftcolumn*}
"Friend, there are four conditions for the attainment of the neither-pleasant-nor-painful awareness-release.
\end{leftcolumn*}

\begin{rightcolumn}
“cattāro kho, āvuso, paccayā adukkhamasukhāya cetovimuttiyā samāpattiyā.
\end{rightcolumn}
\end{samepage}

\begin{samepage}
\ensurevspace{4\baselineskip}
\begin{leftcolumn*}
There is the case where a monk, with the abandoning of pleasure and pain — as with the earlier disappearance of joy and grief — enters and remains in the fourth jhana: purity of equanimity and mindfulness, neither-pleasure-nor-pain.
\end{leftcolumn*}

\begin{rightcolumn}
idhāvuso, bhikkhu sukhassa ca pahānā dukkhassa ca pahānā pubbeva somanassadomanassānaṃ atthaṅgamā adukkhamasukhaṃ upekkhāsatipārisuddhiṃ catutthaṃ jhānaṃ upasampajja viharati.
\end{rightcolumn}
\end{samepage}

\begin{samepage}
\ensurevspace{4\baselineskip}
\begin{leftcolumn*}
These are the four conditions for the attainment of the neither-pleasant-nor-painful awareness-release.
\end{leftcolumn*}

\begin{rightcolumn}
ime kho, āvuso, cattāro paccayā adukkhamasukhāya cetovimuttiyā samāpattiyā”ti.
\end{rightcolumn}
\end{samepage}

\begin{samepage}
\ensurevspace{4\baselineskip}
\begin{leftcolumn*}
-
\end{leftcolumn*}

\begin{rightcolumn}
-
\end{rightcolumn}
\end{samepage}

\begin{samepage}
\ensurevspace{4\baselineskip}
\begin{leftcolumn*}
"How many conditions are there for the attainment of the signless awareness-release?"
\end{leftcolumn*}

\begin{rightcolumn}
“kati panāvuso, paccayā animittāya cetovimuttiyā samāpattiyā”ti?
\end{rightcolumn}
\end{samepage}

\begin{samepage}
\ensurevspace{4\baselineskip}
\begin{leftcolumn*}
"There are two conditions for the attainment of the signless awareness-release: lack of attention to all signs and attention to the signless property.
\end{leftcolumn*}

\begin{rightcolumn}
“dve kho, āvuso, paccayā animittāya cetovimuttiyā samāpattiyā — sabbanimittānañca amanasikāro, animittāya ca dhātuyā manasikāro.
\end{rightcolumn}
\end{samepage}

\begin{samepage}
\ensurevspace{4\baselineskip}
\begin{leftcolumn*}
These are the two conditions for the attainment of the signless awareness-release."
\end{leftcolumn*}

\begin{rightcolumn}
ime kho, āvuso, dve paccayā animittāya cetovimuttiyā samāpattiyā”ti.
\end{rightcolumn}
\end{samepage}

\begin{samepage}
\ensurevspace{4\baselineskip}
\begin{leftcolumn*}
-
\end{leftcolumn*}

\begin{rightcolumn}
-
\end{rightcolumn}
\end{samepage}

\begin{samepage}
\ensurevspace{4\baselineskip}
\begin{leftcolumn*}
"And how many conditions are there for the persistence of the signless awareness-release?"
\end{leftcolumn*}

\begin{rightcolumn}
“kati panāvuso, paccayā animittāya cetovimuttiyā ṭhitiyā”ti?
\end{rightcolumn}
\end{samepage}

\begin{samepage}
\ensurevspace{4\baselineskip}
\begin{leftcolumn*}
"There are three conditions for the persistence of the signless awareness-release: lack of attention to all signs, attention to the signless property, and a prior act of will.
\end{leftcolumn*}

\begin{rightcolumn}
“tayo kho, āvuso, paccayā animittāya cetovimuttiyā ṭhitiyā — sabbanimittānañca amanasikāro, animittāya ca dhātuyā manasikāro, pubbe ca abhisaṅkhāro.
\end{rightcolumn}
\end{samepage}

\begin{samepage}
\ensurevspace{4\baselineskip}
\begin{leftcolumn*}
These are the three conditions for the persistence of the signless awareness-release."
\end{leftcolumn*}

\begin{rightcolumn}
ime kho, āvuso, tayo paccayā animittāya cetovimuttiyā ṭhitiyā”ti.
\end{rightcolumn}
\end{samepage}

\begin{samepage}
\ensurevspace{4\baselineskip}
\begin{leftcolumn*}
-
\end{leftcolumn*}

\begin{rightcolumn}
-
\end{rightcolumn}
\end{samepage}

\begin{samepage}
\ensurevspace{4\baselineskip}
\begin{leftcolumn*}
"And how many conditions are there for the emergence from the signless awareness-release?"
\end{leftcolumn*}

\begin{rightcolumn}
“kati panāvuso, paccayā animittāya cetovimuttiyā vuṭṭhānāyā”ti?
\end{rightcolumn}
\end{samepage}

\begin{samepage}
\ensurevspace{4\baselineskip}
\begin{leftcolumn*}
"There are two conditions for the emergence from the signless awareness-release: attention to all signs and lack of attention to the signless property.
\end{leftcolumn*}

\begin{rightcolumn}
“dve kho, āvuso, paccayā animittāya cetovimuttiyā vuṭṭhānāya — sabbanimittānañca manasikāro, animittāya ca dhātuyā amanasikāro.
\end{rightcolumn}
\end{samepage}

\begin{samepage}
\ensurevspace{4\baselineskip}
\begin{leftcolumn*}
These are the two conditions for the emergence from the signless awareness-release."
\end{leftcolumn*}

\begin{rightcolumn}
ime kho, āvuso, dve paccayā animittāya cetovimuttiyā vuṭṭhānāyā”ti.
\end{rightcolumn}
\end{samepage}

\begin{samepage}
\ensurevspace{4\baselineskip}
\begin{leftcolumn*}
-
\end{leftcolumn*}

\begin{rightcolumn}
-
\end{rightcolumn}
\end{samepage}

\begin{samepage}
\ensurevspace{4\baselineskip}
\begin{leftcolumn*}
"The immeasurable awareness-release,
\end{leftcolumn*}

\begin{rightcolumn}
“yā cāyaṃ, āvuso, appamāṇā cetovimutti,
\end{rightcolumn}
\end{samepage}

\begin{samepage}
\ensurevspace{4\baselineskip}
\begin{leftcolumn*}
the nothingness awareness-release,
\end{leftcolumn*}

\begin{rightcolumn}
yā ca ākiñcaññā cetovimutti,
\end{rightcolumn}
\end{samepage}

\begin{samepage}
\ensurevspace{4\baselineskip}
\begin{leftcolumn*}
the emptiness awareness-release,
\end{leftcolumn*}

\begin{rightcolumn}
yā ca suññatā cetovimutti,
\end{rightcolumn}
\end{samepage}

\begin{samepage}
\ensurevspace{4\baselineskip}
\begin{leftcolumn*}
the signless-awareness-release:
\end{leftcolumn*}

\begin{rightcolumn}
yā ca animittā cetovimutti —
\end{rightcolumn}
\end{samepage}

\begin{samepage}
\ensurevspace{4\baselineskip}
\begin{leftcolumn*}
Are these qualities different in meaning and different in name, or are they one in meaning and different only in name?"
\end{leftcolumn*}

\begin{rightcolumn}
ime dhammā nānātthā ceva nānābyañjanā ca udāhu ekatthā byañjanameva nānan”ti?
\end{rightcolumn}
\end{samepage}

\begin{samepage}
\ensurevspace{4\baselineskip}
\begin{leftcolumn*}
"The immeasurable awareness-release,
\end{leftcolumn*}

\begin{rightcolumn}
“yā cāyaṃ, āvuso, appamāṇā cetovimutti,
\end{rightcolumn}
\end{samepage}

\begin{samepage}
\ensurevspace{4\baselineskip}
\begin{leftcolumn*}
the nothingness awareness-release,
\end{leftcolumn*}

\begin{rightcolumn}
yā ca ākiñcaññā cetovimutti,
\end{rightcolumn}
\end{samepage}

\begin{samepage}
\ensurevspace{4\baselineskip}
\begin{leftcolumn*}
the emptiness awareness-release,
\end{leftcolumn*}

\begin{rightcolumn}
yā ca suññatā cetovimutti,
\end{rightcolumn}
\end{samepage}

\begin{samepage}
\ensurevspace{4\baselineskip}
\begin{leftcolumn*}
the signless-awareness-release:
\end{leftcolumn*}

\begin{rightcolumn}
yā ca animittā cetovimutti —
\end{rightcolumn}
\end{samepage}

\begin{samepage}
\ensurevspace{4\baselineskip}
\begin{leftcolumn*}
There is a way of explanation by which these qualities are different in meaning and different in name,
\end{leftcolumn*}

\begin{rightcolumn}
atthi kho, āvuso, pariyāyo yaṃ pariyāyaṃ āgamma ime dhammā nānātthā ceva nānābyañjanā ca;
\end{rightcolumn}
\end{samepage}

\begin{samepage}
\ensurevspace{4\baselineskip}
\begin{leftcolumn*}
and there is a way of explanation by which these qualities are one in meaning and different only in name.
\end{leftcolumn*}

\begin{rightcolumn}
atthi ca kho, āvuso, pariyāyo yaṃ pariyāyaṃ āgamma ime dhammā ekatthā, byañjanameva nānaṃ”.
\end{rightcolumn}
\end{samepage}

\begin{samepage}
\ensurevspace{4\baselineskip}
\begin{leftcolumn*}
-
\end{leftcolumn*}

\begin{rightcolumn}
-
\end{rightcolumn}
\end{samepage}

\begin{samepage}
\ensurevspace{4\baselineskip}
\begin{leftcolumn*}
"And what is the way of explanation by which these qualities are different in meaning and different in name?
\end{leftcolumn*}

\begin{rightcolumn}
“katamo cāvuso, pariyāyo yaṃ pariyāyaṃ āgamma ime dhammā nānātthā ceva nānābyañjanā ca”?
\end{rightcolumn}
\end{samepage}

\begin{samepage}
\ensurevspace{4\baselineskip}
\begin{leftcolumn*}
There is the case where a monk keeps pervading the first direction — as well as the second direction, the third, and the fourth — with an awareness imbued with good will.
\end{leftcolumn*}

\begin{rightcolumn}
“idhāvuso, bhikkhu mettāsahagatena cetasā ekaṃ disaṃ pharitvā viharati, tathā dutiyaṃ, tathā tatiyaṃ, tathā catutthaṃ.
\end{rightcolumn}
\end{samepage}

\begin{samepage}
\ensurevspace{4\baselineskip}
\begin{leftcolumn*}
Thus he keeps pervading above, below, and all around, everywhere and in every respect the all-encompassing world with an awareness imbued with good will: abundant, expansive, immeasurable, free from hostility, free from ill will.
\end{leftcolumn*}

\begin{rightcolumn}
iti uddhamadho tiriyaṃ sabbadhi sabbattatāya sabbāvantaṃ lokaṃ mettāsahagatena cetasā vipulena mahaggatena appamāṇena averena abyābajjhena pharitvā viharati.
\end{rightcolumn}
\end{samepage}

\begin{samepage}
\ensurevspace{4\baselineskip}
\begin{leftcolumn*}
"He keeps pervading the first direction — as well as the second direction, the third, and the fourth — with an awareness imbued with compassion.
\end{leftcolumn*}

\begin{rightcolumn}
karuṇāsahagatena cetasā ekaṃ disaṃ pharitvā viharati, tathā dutiyaṃ, tathā tatiyaṃ, tathā catutthaṃ.
\end{rightcolumn}
\end{samepage}

\begin{samepage}
\ensurevspace{4\baselineskip}
\begin{leftcolumn*}
Thus he keeps pervading above, below, and all around, everywhere and in every respect the all-encompassing world with an awareness imbued with compassion: abundant, expansive, immeasurable, free from hostility, free from ill will.
\end{leftcolumn*}

\begin{rightcolumn}
iti uddhamadho tiriyaṃ sabbadhi sabbattatāya sabbāvantaṃ lokaṃ karuṇāsahagatena cetasā vipulena mahaggatena appamāṇena averena abyābajjhena pharitvā viharati.
\end{rightcolumn}
\end{samepage}

\begin{samepage}
\ensurevspace{4\baselineskip}
\begin{leftcolumn*}
"He keeps pervading the first direction — as well as the second direction, the third, and the fourth — with an awareness imbued with appreciation.
\end{leftcolumn*}

\begin{rightcolumn}
muditāsahagatena cetasā ekaṃ disaṃ pharitvā viharati, tathā dutiyaṃ, tathā tatiyaṃ, tathā catutthaṃ.
\end{rightcolumn}
\end{samepage}

\begin{samepage}
\ensurevspace{4\baselineskip}
\begin{leftcolumn*}
Thus he keeps pervading above, below, and all around, everywhere and in every respect the all-encompassing world with an awareness imbued with equanimity: abundant, expansive, immeasurable, free from hostility, free from ill will.
\end{leftcolumn*}

\begin{rightcolumn}
iti uddhamadho tiriyaṃ sabbadhi sabbattatāya sabbāvantaṃ lokaṃ muditāsahagatena cetasā vipulena mahaggatena appamāṇena averena abyābajjhena pharitvā viharati.
\end{rightcolumn}
\end{samepage}

\begin{samepage}
\ensurevspace{4\baselineskip}
\begin{leftcolumn*}
"He keeps pervading the first direction — as well as the second direction, the third, and the fourth — with an awareness imbued with equanimity.
\end{leftcolumn*}

\begin{rightcolumn}
upekkhāsahagatena cetasā ekaṃ disaṃ pharitvā viharati, tathā dutiyaṃ, tathā tatiyaṃ, tathā catutthaṃ.
\end{rightcolumn}
\end{samepage}

\begin{samepage}
\ensurevspace{4\baselineskip}
\begin{leftcolumn*}
Thus he keeps pervading above, below, and all around, everywhere and in every respect the all-encompassing world with an awareness imbued with equanimity: abundant, expansive, immeasurable, free from hostility, free from ill will.
\end{leftcolumn*}

\begin{rightcolumn}
iti uddhamadho tiriyaṃ sabbadhi sabbattatāya sabbāvantaṃ lokaṃ upekkhāsahagatena cetasā vipulena mahaggatena appamāṇena averena abyābajjhena pharitvā viharati.
\end{rightcolumn}
\end{samepage}

\begin{samepage}
\ensurevspace{4\baselineskip}
\begin{leftcolumn*}
"This is called the immeasurable awareness-release.
\end{leftcolumn*}

\begin{rightcolumn}
ayaṃ vuccatāvuso, appamāṇā cetovimutti”.
\end{rightcolumn}
\end{samepage}

\begin{samepage}
\ensurevspace{4\baselineskip}
\begin{leftcolumn*}
-
\end{leftcolumn*}

\begin{rightcolumn}
-
\end{rightcolumn}
\end{samepage}

\begin{samepage}
\ensurevspace{4\baselineskip}
\begin{leftcolumn*}
"And what is the nothingness awareness-release?
\end{leftcolumn*}

\begin{rightcolumn}
“katamā cāvuso, ākiñcaññā cetovimutti”?
\end{rightcolumn}
\end{samepage}

\begin{samepage}
\ensurevspace{4\baselineskip}
\begin{leftcolumn*}
There is the case where a monk, with the complete transcending of the dimension of the boundless of consciousness, [perceiving,] 'There is nothing,' enters and remains in the dimension of nothingness.
\end{leftcolumn*}

\begin{rightcolumn}
“idhāvuso, bhikkhu sabbaso viññāṇañcāyatanaṃ samatikkamma natthi kiñcīti ākiñcaññāyatanaṃ upasampajja viharati.
\end{rightcolumn}
\end{samepage}

\begin{samepage}
\ensurevspace{4\baselineskip}
\begin{leftcolumn*}
This is called the nothingness awareness-release.
\end{leftcolumn*}

\begin{rightcolumn}
ayaṃ vuccatāvuso, ākiñcaññā cetovimutti”.
\end{rightcolumn}
\end{samepage}

\begin{samepage}
\ensurevspace{4\baselineskip}
\begin{leftcolumn*}
-
\end{leftcolumn*}

\begin{rightcolumn}
-
\end{rightcolumn}
\end{samepage}

\begin{samepage}
\ensurevspace{4\baselineskip}
\begin{leftcolumn*}
"And what is the emptiness awareness-release?
\end{leftcolumn*}

\begin{rightcolumn}
“katamā cāvuso, suññatā cetovimutti”?
\end{rightcolumn}
\end{samepage}

\begin{samepage}
\ensurevspace{4\baselineskip}
\begin{leftcolumn*}
There is the case where a monk, having gone into the wilderness, to the root of a tree, or into an empty dwelling, considers this: 'This is empty of self or of anything pertaining to self.'
\end{leftcolumn*}

\begin{rightcolumn}
“idhāvuso, bhikkhu araññagato vā rukkhamūlagato vā suññāgāragato vā iti paṭisañcikkhati — ‘suññamidaṃ attena vā attaniyena vā’ti.
\end{rightcolumn}
\end{samepage}

\begin{samepage}
\ensurevspace{4\baselineskip}
\begin{leftcolumn*}
This is called the emptiness awareness-release.
\end{leftcolumn*}

\begin{rightcolumn}
ayaṃ vuccatāvuso, suññatā cetovimutti”.
\end{rightcolumn}
\end{samepage}

\begin{samepage}
\ensurevspace{4\baselineskip}
\begin{leftcolumn*}
-
\end{leftcolumn*}

\begin{rightcolumn}
-
\end{rightcolumn}
\end{samepage}

\begin{samepage}
\ensurevspace{4\baselineskip}
\begin{leftcolumn*}
"And what is the signless awareness-release?
\end{leftcolumn*}

\begin{rightcolumn}
“katamā cāvuso, animittā cetovimutti”?
\end{rightcolumn}
\end{samepage}

\begin{samepage}
\ensurevspace{4\baselineskip}
\begin{leftcolumn*}
There is the case where a monk, through not attending to all signs, enters and remains in the signless concentration of awareness.
\end{leftcolumn*}

\begin{rightcolumn}
“idhāvuso, bhikkhu sabbanimittānaṃ amanasikārā animittaṃ cetosamādhiṃ upasampajja viharati.
\end{rightcolumn}
\end{samepage}

\begin{samepage}
\ensurevspace{4\baselineskip}
\begin{leftcolumn*}
This is called the signless awareness-release.
\end{leftcolumn*}

\begin{rightcolumn}
ayaṃ vuccatāvuso, animittā cetovimutti.
\end{rightcolumn}
\end{samepage}

\begin{samepage}
\ensurevspace{4\baselineskip}
\begin{leftcolumn*}
"This is the way of explaining by which these qualities are different in meaning and different in name.
\end{leftcolumn*}

\begin{rightcolumn}
ayaṃ kho, āvuso, pariyāyo yaṃ pariyāyaṃ āgamma ime dhammā nānātthā ceva nānābyañjanā ca”.
\end{rightcolumn}
\end{samepage}

\begin{samepage}
\ensurevspace{4\baselineskip}
\begin{leftcolumn*}
-
\end{leftcolumn*}

\begin{rightcolumn}
-
\end{rightcolumn}
\end{samepage}

\begin{samepage}
\ensurevspace{4\baselineskip}
\begin{leftcolumn*}
"And what is the way of explaining whereby these qualities are one in meaning and different only in name?
\end{leftcolumn*}

\begin{rightcolumn}
“katamo cāvuso, pariyāyo yaṃ pariyāyaṃ āgamma ime dhammā ekatthā byañjanameva nānaṃ”?
\end{rightcolumn}
\end{samepage}

\begin{samepage}
\ensurevspace{4\baselineskip}
\begin{leftcolumn*}
"Passion, friend, is a making of limits.
\end{leftcolumn*}

\begin{rightcolumn}
“rāgo kho, āvuso, pamāṇakaraṇo,
\end{rightcolumn}
\end{samepage}

\begin{samepage}
\ensurevspace{4\baselineskip}
\begin{leftcolumn*}
Aversion is a making of limits.
\end{leftcolumn*}

\begin{rightcolumn}
doso pamāṇakaraṇo,
\end{rightcolumn}
\end{samepage}

\begin{samepage}
\ensurevspace{4\baselineskip}
\begin{leftcolumn*}
Delusion is a making of limits.
\end{leftcolumn*}

\begin{rightcolumn}
moho pamāṇakaraṇo.
\end{rightcolumn}
\end{samepage}

\begin{samepage}
\ensurevspace{4\baselineskip}
\begin{leftcolumn*}
In a monk whose fermentations are ended, these have been abandoned, their root destroyed, made like a palmyra stump, deprived of the conditions of development, not destined for future arising.
\end{leftcolumn*}

\begin{rightcolumn}
te khīṇāsavassa bhikkhuno pahīnā ucchinnamūlā tālāvatthukatā anabhāvaṃkatā āyatiṃ anuppādadhammā.
\end{rightcolumn}
\end{samepage}

\begin{samepage}
\ensurevspace{4\baselineskip}
\begin{leftcolumn*}
Now, to the extent that there is immeasurable awareness-release, the unshakeable deliverance of mind is declared the foremost.
And this unshakeable deliverance of mind is empty of passion, empty of aversion, empty of delusion.
\end{leftcolumn*}

\begin{rightcolumn}
yāvatā kho, āvuso, appamāṇā cetovimuttiyo, akuppā tāsaṃ cetovimutti aggamakkhāyati.
sā kho panākuppā cetovimutti suññā rāgena, suññā dosena, suññā mohena.
\end{rightcolumn}
\end{samepage}

\begin{samepage}
\ensurevspace{4\baselineskip}
\begin{leftcolumn*}
"Passion is a something.
\end{leftcolumn*}

\begin{rightcolumn}
rāgo kho, āvuso, kiñcano,
\end{rightcolumn}
\end{samepage}

\begin{samepage}
\ensurevspace{4\baselineskip}
\begin{leftcolumn*}
Aversion is a something.
\end{leftcolumn*}

\begin{rightcolumn}
doso kiñcano,
\end{rightcolumn}
\end{samepage}

\begin{samepage}
\ensurevspace{4\baselineskip}
\begin{leftcolumn*}
Delusion is a something.
\end{leftcolumn*}

\begin{rightcolumn}
moho kiñcano.
\end{rightcolumn}
\end{samepage}

\begin{samepage}
\ensurevspace{4\baselineskip}
\begin{leftcolumn*}
In a monk whose fermentations are ended, these have been abandoned, their root destroyed, made like a palmyra stump, deprived of the conditions of development, not destined for future arising.
\end{leftcolumn*}

\begin{rightcolumn}
te khīṇāsavassa bhikkhuno pahīnā ucchinnamūlā tālāvatthukatā anabhāvaṃkatā āyatiṃ anuppādadhammā.
\end{rightcolumn}
\end{samepage}

\begin{samepage}
\ensurevspace{4\baselineskip}
\begin{leftcolumn*}
Now, to the extent that there is nothingness awareness-release, the unshakeable deliverance of mind is declared the foremost.
And this unshakeable deliverance of mind is empty of passion, empty of aversion, empty of delusion.
\end{leftcolumn*}

\begin{rightcolumn}
yāvatā kho, āvuso, ākiñcaññā cetovimuttiyo, akuppā tāsaṃ cetovimutti aggamakkhāyati.
sā kho panākuppā cetovimutti suññā rāgena, suññā dosena, suññā mohena.
\end{rightcolumn}
\end{samepage}

\begin{samepage}
\ensurevspace{4\baselineskip}
\begin{leftcolumn*}
"Passion is a making of signs.
\end{leftcolumn*}

\begin{rightcolumn}
rāgo kho, āvuso, nimittakaraṇo,
\end{rightcolumn}
\end{samepage}

\begin{samepage}
\ensurevspace{4\baselineskip}
\begin{leftcolumn*}
Aversion is a making of signs.
\end{leftcolumn*}

\begin{rightcolumn}
doso nimittakaraṇo,
\end{rightcolumn}
\end{samepage}

\begin{samepage}
\ensurevspace{4\baselineskip}
\begin{leftcolumn*}
Delusion is a making of signs.
\end{leftcolumn*}

\begin{rightcolumn}
moho nimittakaraṇo.
\end{rightcolumn}
\end{samepage}

\begin{samepage}
\ensurevspace{4\baselineskip}
\begin{leftcolumn*}
In a monk whose fermentations are ended, these have been abandoned, their root destroyed, made like a palmyra stump, deprived of the conditions of development, not destined for future arising.
\end{leftcolumn*}

\begin{rightcolumn}
te khīṇāsavassa bhikkhuno pahīnā ucchinnamūlā tālāvatthukatā anabhāvaṃkatā āyatiṃ anuppādadhammā.
\end{rightcolumn}
\end{samepage}

\begin{samepage}
\ensurevspace{4\baselineskip}
\begin{leftcolumn*}
Now, to the extent that there is signless awareness-release, the unshakeable deliverance of mind is declared the foremost.
And this unshakeable deliverance of mind is empty of passion, empty of aversion, empty of delusion.
\end{leftcolumn*}

\begin{rightcolumn}
yāvatā kho, āvuso, animittā cetovimuttiyo, akuppā tāsaṃ cetovimutti aggamakkhāyati.
sā kho panākuppā cetovimutti suññā rāgena, suññā dosena, suññā mohena.
\end{rightcolumn}
\end{samepage}

\begin{samepage}
\ensurevspace{4\baselineskip}
\begin{leftcolumn*}
"This, friend, is the way of explaining whereby these qualities are one in meaning and different only in name."
\end{leftcolumn*}

\begin{rightcolumn}
ayaṃ kho, āvuso, pariyāyo yaṃ pariyāyaṃ āgamma ime dhammā ekatthā byañjanameva nānan”ti.
\end{rightcolumn}
\end{samepage}

\begin{samepage}
\ensurevspace{4\baselineskip}
\begin{leftcolumn*}
-
\end{leftcolumn*}

\begin{rightcolumn}
-
\end{rightcolumn}
\end{samepage}

\begin{samepage}
\ensurevspace{4\baselineskip}
\begin{leftcolumn*}
That is what Ven. Sariputta said. Gratified, Ven. Maha Kotthita delighted in Ven. Sariputta's words.
\end{leftcolumn*}

\begin{rightcolumn}
idamavocāyasmā sāriputto. attamano āyasmā mahākoṭṭhiko āyasmato sāriputtassa bhāsitaṃ abhinandīti.
\end{rightcolumn}
\end{samepage}

\begin{samepage}
\ensurevspace{4\baselineskip}
\begin{leftcolumn*}
The Greater Series of Questions-and-Answers is complete
\end{leftcolumn*}

\begin{rightcolumn}
mahāvedallasuttaṃ niṭṭhitaṃ tatiyaṃ.
\end{rightcolumn}
\end{samepage}

