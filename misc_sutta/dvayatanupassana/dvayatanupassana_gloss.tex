
\begin{samepage}
\begingl[glneveryline={\PaliGlossA,\PaliGlossB}]
Khuddaka[-] Nikāya,[-] suttanipātapāḷi,[-] 3.[-] mahāvaggo,[-] 12.[-] dvayatānupassanāsuttaṃ[-] (KN[-] 5.38)[-]
\endgl
\nopagebreak
\linespread{0.5}
\begin{spacin}{0.2}
{\PaliGlossFT 12 CONTEMPLATION OF DYADS (DVAYATĀNUPASSANĀ SUTTA)}
\end{spacin}
\vskip 12pt
\end{samepage}
\vskip 0.2in
\begin{samepage}
\begingl[glneveryline={\PaliGlossA,\PaliGlossB}]
evaṁ[thus] me[to me] sutaṁ.[heard]
\endgl
\nopagebreak
\linespread{0.5}
\begin{spacin}{0.2}
{\PaliGlossFT Thus have I heard.}
\end{spacin}
\vskip 12pt
\end{samepage}
\begin{samepage}
\begingl[glneveryline={\PaliGlossA,\PaliGlossB}]
ekaṁ[one] samayaṁ[time] bhagavā[fortunate] sāvatthiyaṁ[in Kosala] viharati[dwell] pubbārāme[eastern.park] migāramātupāsāde.[-]
\endgl
\nopagebreak
\linespread{0.5}
\begin{spacin}{0.2}
{\PaliGlossFT On one occasion the Blessed One was dwelling in Sāvatthī in the Eastern Park in Migāramātā’s mansion.}
\end{spacin}
\vskip 12pt
\end{samepage}
\begin{samepage}
\begingl[glneveryline={\PaliGlossA,\PaliGlossB}]
tena[-] kho[-] pana[-] samayena[occasion] bhagavā[fortunate] tadahuposathe[-] pannarase[fifteenth] puṇṇāya[full] puṇṇamāya[full-moon day] rattiyā[night] bhikkhusaṅghaparivuto[congregation of monks] abbhokāse[open air] nisinno[sat down] hoti.[to be]
\endgl
\nopagebreak
\linespread{0.5}
\begin{spacin}{0.2}
{\PaliGlossFT Now on that occasion — the uposatha day of the fifteenth, the full-moon night — the Blessed One was seated in the open surrounded by the Sangha of bhikkhus.}
\end{spacin}
\vskip 12pt
\end{samepage}
\begin{samepage}
\begingl[glneveryline={\PaliGlossA,\PaliGlossB}]
atha[then] kho[-] bhagavā[fortunate] tuṇhībhūtaṁ[silent] tuṇhībhūtaṁ[silent] bhikkhusaṅghaṁ[congregation of monks] anuviloketvā[surveys] bhikkhū[-] āmantesi[address] —[-]
\endgl
\nopagebreak
\linespread{0.5}
\begin{spacin}{0.2}
{\PaliGlossFT [140] Then, having surveyed the completely silent Sangha of bhikkhus, he addressed them thus:}
\end{spacin}
\vskip 12pt
\end{samepage}
\begin{samepage}
\begingl[glneveryline={\PaliGlossA,\PaliGlossB}]
“‘ye[-] te,[-] bhikkhave,[-] kusalā[good action] dhammā[doctrine] ariyā[noble] niyyānikā[profitable] sambodhagāmino,[full.enlight.going] tesaṁ[-] vo,[-] bhikkhave,[-] kusalānaṁ[good action] dhammānaṁ[doctrine] ariyānaṁ[noble] niyyānikānaṁ[profitable] sambodhagāmīnaṁ[full.enlight.going] kā[-] upanisā[cause] savanāyā’ti,[listen]
\endgl
\nopagebreak
\linespread{0.5}
\begin{spacin}{0.2}
{\PaliGlossFT “Bhikkhus, if others ask you, ‘What is your aim in listening to those teachings that are wholesome, noble, emancipating, leading to enlightenment?’}
\end{spacin}
\vskip 12pt
\end{samepage}
\begin{samepage}
\begingl[glneveryline={\PaliGlossA,\PaliGlossB}]
iti[-] ce,[-] bhikkhave,[-] pucchitāro[asked] assu,[to be] te[-] evamassu[-] vacanīyā[utterance] —[-] ‘yāvadeva[-] dvayatānaṁ[dyad] dhammānaṁ[doctrine] yathābhūtaṁ[as.become] ñāṇāyā’ti.[wisdom]
\endgl
\nopagebreak
\linespread{0.5}
\begin{spacin}{0.2}
{\PaliGlossFT you should answer them thus: ‘For the accurate knowledge of things arranged in dyads.’}
\end{spacin}
\vskip 12pt
\end{samepage}
\begin{samepage}
\begingl[glneveryline={\PaliGlossA,\PaliGlossB}]
kiñca[-] dvayataṁ[pair] vadetha?[speak]
\endgl
\nopagebreak
\linespread{0.5}
\begin{spacin}{0.2}
{\PaliGlossFT And what would one call a dyad?}
\end{spacin}
\vskip 12pt
\end{samepage}
\vskip 0.2in
\begin{samepage}
\begingl[glneveryline={\PaliGlossA,\PaliGlossB}]
(1)[-]
\endgl
\nopagebreak
\linespread{0.5}
\begin{spacin}{0.2}
{\PaliGlossFT [1. The four noble truths]185}
\end{spacin}
\vskip 12pt
\end{samepage}
\begin{samepage}
\begingl[glneveryline={\PaliGlossA,\PaliGlossB}]
“idaṁ[-] dukkhaṁ,[suffering] ayaṁ[this] dukkhasamudayoti[suffer.origin] ayamekānupassanā.[this.one.contemplate]
\endgl
\nopagebreak
\linespread{0.5}
\begin{spacin}{0.2}
{\PaliGlossFT “’This is suffering, this is the origin of suffering’ — this is one contemplation.}
\end{spacin}
\vskip 12pt
\end{samepage}
\begin{samepage}
\begingl[glneveryline={\PaliGlossA,\PaliGlossB}]
ayaṁ[this] dukkhanirodho,[suffer.destruction] ayaṁ[this] dukkhanirodhagāminī[suffer.extinct.go] paṭipadāti,[practice] ayaṁ[this] dutiyānupassanā.[second.contemplate]
\endgl
\nopagebreak
\linespread{0.5}
\begin{spacin}{0.2}
{\PaliGlossFT ‘This is the cessation of suffering, this is the way leading to the cessation of suffering’ — this is a second contemplation.}
\end{spacin}
\vskip 12pt
\end{samepage}
\begin{samepage}
\begingl[glneveryline={\PaliGlossA,\PaliGlossB}]
evaṁ[thus] sammā[rightly] dvayatānupassino[pair.contemplate] kho,[-] bhikkhave,[-] bhikkhuno[-] appamattassa[vigilant] ātāpino[ardent] pahitattassa[resolute] viharato[abides] dvinnaṁ[pair] phalānaṁ[fruit] aññataraṁ[certain] phalaṁ[fruit] pāṭikaṅkhaṁ[expected] —[-] diṭṭheva[vision] dhamme[the Norm] aññā,[other] sati[state] vā[-] upādisese[some fuel left] anāgāmitā”ti.[not return]
\endgl
\nopagebreak
\linespread{0.5}
\begin{spacin}{0.2}
{\PaliGlossFT When a bhikkhu dwells thus correctly contemplating a dyad — heedful, ardent, and resolute — one of two fruits is to be expected of him: either final knowledge in this very life or, if there is a residue remaining, the state of non-returning.”}
\end{spacin}
\vskip 12pt
\end{samepage}
\begin{samepage}
\begingl[glneveryline={\PaliGlossA,\PaliGlossB}]
idamavoca[this said] bhagavā.[fortunate]
\endgl
\nopagebreak
\linespread{0.5}
\begin{spacin}{0.2}
{\PaliGlossFT This is what the Blessed One said.}
\end{spacin}
\vskip 12pt
\end{samepage}
\begin{samepage}
\begingl[glneveryline={\PaliGlossA,\PaliGlossB}]
idaṁ[-] vatvāna[having said] sugato[faring well] athāparaṁ[then also] etadavoca[he said] satthā[taught] —[-]
\endgl
\nopagebreak
\linespread{0.5}
\begin{spacin}{0.2}
{\PaliGlossFT Having said this, the Fortunate One, the Teacher, further said this:}
\end{spacin}
\vskip 12pt
\end{samepage}
\vskip 0.2in
\begin{samepage}
\begingl[glneveryline={\PaliGlossA,\PaliGlossB}]
729.[-] “ye[-] dukkhaṁ[suffering] nappajānanti,[not.understand] atho[-] dukkhassa[suffering] sambhavaṁ.[origin]
\endgl
\nopagebreak
\linespread{0.5}
\begin{spacin}{0.2}
{\PaliGlossFT 724. “Those who do not understand suffering,  or the origin of suffering;}
\end{spacin}
\vskip 12pt
\end{samepage}
\begin{samepage}
\begingl[glneveryline={\PaliGlossA,\PaliGlossB}]
yattha[-] ca[-] sabbaso[altogether] dukkhaṁ,[suffering] asesaṁ[entire] uparujjhati.[ceases]
\endgl
\nopagebreak
\linespread{0.5}
\begin{spacin}{0.2}
{\PaliGlossFT who do not know where suffering completely ceases without remainder;}
\end{spacin}
\vskip 12pt
\end{samepage}
\begin{samepage}
\begingl[glneveryline={\PaliGlossA,\PaliGlossB}]
tañca[-] maggaṁ[path] na[not] jānanti,[knowing] dukkhūpasamagāminaṁ.[suffering.alleviation]
\endgl
\nopagebreak
\linespread{0.5}
\begin{spacin}{0.2}
{\PaliGlossFT and who do not know the path that leads to the allaying of suffering:}
\end{spacin}
\vskip 12pt
\end{samepage}
\begin{samepage}
\begingl[glneveryline={\PaliGlossA,\PaliGlossB}]
730.[-] “cetovimuttihīnā[mind.emancipate] te,[-] atho[-] paññāvimuttiyā.[insight.emancipate]
\endgl
\nopagebreak
\linespread{0.5}
\begin{spacin}{0.2}
{\PaliGlossFT 725. “they are destitute of liberation of mind  and also of liberation by wisdom.}
\end{spacin}
\vskip 12pt
\end{samepage}
\begin{samepage}
\begingl[glneveryline={\PaliGlossA,\PaliGlossB}]
abhabbā[unable] te[-] antakiriyāya,[end.make] te[-] ve[indeed] jātijarūpagā.[born.old age]
\endgl
\nopagebreak
\linespread{0.5}
\begin{spacin}{0.2}
{\PaliGlossFT Incapable of making an end, they fare on to birth and old age.}
\end{spacin}
\vskip 12pt
\end{samepage}
\begin{samepage}
\begingl[glneveryline={\PaliGlossA,\PaliGlossB}]
731.[-] “ye[-] ca[-] dukkhaṁ[suffering] pajānanti,[knowledge] atho[-] dukkhassa[suffering] sambhavaṁ.[origin]
\endgl
\nopagebreak
\linespread{0.5}
\begin{spacin}{0.2}
{\PaliGlossFT 726. “But those who understand suffering,  and the origin of suffering;}
\end{spacin}
\vskip 12pt
\end{samepage}
\begin{samepage}
\begingl[glneveryline={\PaliGlossA,\PaliGlossB}]
yattha[-] ca[-] sabbaso[altogether] dukkhaṁ,[suffering] asesaṁ[entire] uparujjhati.[ceases]
\endgl
\nopagebreak
\linespread{0.5}
\begin{spacin}{0.2}
{\PaliGlossFT [who know as well] where suffering}
\end{spacin}
\vskip 12pt
\end{samepage}
\begin{samepage}
\begingl[glneveryline={\PaliGlossA,\PaliGlossB}]
tañca[-] maggaṁ[path] pajānanti,[knowledge] dukkhūpasamagāminaṁ.[suffering.alleviation]
\endgl
\nopagebreak
\linespread{0.5}
\begin{spacin}{0.2}
{\PaliGlossFT completely ceases without remainder;  and who understand the path  that leads to the allaying of suffering:}
\end{spacin}
\vskip 12pt
\end{samepage}
\begin{samepage}
\begingl[glneveryline={\PaliGlossA,\PaliGlossB}]
732.[-] “cetovimuttisampannā,[mind.emancipate.possess] atho[-] paññāvimuttiyā.[insight.emancipate]
\endgl
\nopagebreak
\linespread{0.5}
\begin{spacin}{0.2}
{\PaliGlossFT 727. “they are possessed of mind’s liberation  and also liberation by wisdom.}
\end{spacin}
\vskip 12pt
\end{samepage}
\begin{samepage}
\begingl[glneveryline={\PaliGlossA,\PaliGlossB}]
bhabbā[capable] te[-] antakiriyāya,[end.make] na[not] te[-] jātijarūpagā”ti.[born.old age]
\endgl
\nopagebreak
\linespread{0.5}
\begin{spacin}{0.2}
{\PaliGlossFT Capable of making an end, they fare no more to birth and old age.}
\end{spacin}
\vskip 12pt
\end{samepage}
\vskip 0.2in
\begin{samepage}
\begingl[glneveryline={\PaliGlossA,\PaliGlossB}]
(2)[-]
\endgl
\nopagebreak
\linespread{0.5}
\begin{spacin}{0.2}
{\PaliGlossFT [2. Acquisition]}
\end{spacin}
\vskip 12pt
\end{samepage}
\begin{samepage}
\begingl[glneveryline={\PaliGlossA,\PaliGlossB}]
“‘siyā[could.be] aññenapi[another] pariyāyena[method] sammā[rightly] dvayatānupassanā’ti,[pair.contemplate] iti[-] ce,[-] bhikkhave,[-] pucchitāro[asked] assu;[to be]
\endgl
\nopagebreak
\linespread{0.5}
\begin{spacin}{0.2}
{\PaliGlossFT “If, bhikkhus, there are those who ask, ‘Could there be correct contemplation of dyads in some other way?’}
\end{spacin}
\vskip 12pt
\end{samepage}
\begin{samepage}
\begingl[glneveryline={\PaliGlossA,\PaliGlossB}]
‘siyā’tissu[could.be] vacanīyā.[utterance]
\endgl
\nopagebreak
\linespread{0.5}
\begin{spacin}{0.2}
{\PaliGlossFT you should answer them thus: ‘There could be.’}
\end{spacin}
\vskip 12pt
\end{samepage}
\begin{samepage}
\begingl[glneveryline={\PaliGlossA,\PaliGlossB}]
kathañca[how] siyā?[could.be]
\endgl
\nopagebreak
\linespread{0.5}
\begin{spacin}{0.2}
{\PaliGlossFT And how could there be?}
\end{spacin}
\vskip 12pt
\end{samepage}
\begin{samepage}
\begingl[glneveryline={\PaliGlossA,\PaliGlossB}]
yaṁ[-] kiñci[-] dukkhaṁ[suffering] sambhoti[arises] sabbaṁ[all] upadhipaccayāti,[attachment.cause] ayamekānupassanā.[this.one.contemplate]
\endgl
\nopagebreak
\linespread{0.5}
\begin{spacin}{0.2}
{\PaliGlossFT ‘Whatever suffering originates is all conditioned by acquisition’ — ­this is one contemplation.}
\end{spacin}
\vskip 12pt
\end{samepage}
\begin{samepage}
\begingl[glneveryline={\PaliGlossA,\PaliGlossB}]
upadhīnaṁ[attachment] tveva[that] asesavirāganirodhā[entire.dispation.cease] natthi[not.is] dukkhassa[suffering] sambhavoti,[origin] ayaṁ[this] dutiyānupassanā.[second.contemplate]
\endgl
\nopagebreak
\linespread{0.5}
\begin{spacin}{0.2}
{\PaliGlossFT ‘With the remainderless fading away and cessation of acquisitions, there is no origination of suffering’ — ­this is a second contemplation.}
\end{spacin}
\vskip 12pt
\end{samepage}
\begin{samepage}
\begingl[glneveryline={\PaliGlossA,\PaliGlossB}]
evaṁ[thus] sammā[rightly] dvayatānupassino[pair.contemplate] kho,[-] bhikkhave,[-] bhikkhuno[-] appamattassa[vigilant] ātāpino[ardent] pahitattassa[resolute] viharato[abides] dvinnaṁ[pair] phalānaṁ[fruit] aññataraṁ[certain] phalaṁ[fruit] pāṭikaṅkhaṁ[expected] —[-] diṭṭheva[vision] dhamme[the Norm] aññā,[other] sati[state] vā[-] upādisese[some fuel left] anāgāmitā”ti.[not return]
\endgl
\nopagebreak
\linespread{0.5}
\begin{spacin}{0.2}
{\PaliGlossFT When a bhikkhu dwells thus correctly contemplating a dyad — heedful, ardent, and resolute — one of two fruits is to be expected of him: either final knowledge in this very life or, if there is a residue remaining, the state of non-returning.”}
\end{spacin}
\vskip 12pt
\end{samepage}
\begin{samepage}
\begingl[glneveryline={\PaliGlossA,\PaliGlossB}]
idamavoca[this said] bhagavā.[fortunate]
\endgl
\nopagebreak
\linespread{0.5}
\begin{spacin}{0.2}
{\PaliGlossFT This is what the Blessed One said.}
\end{spacin}
\vskip 12pt
\end{samepage}
\begin{samepage}
\begingl[glneveryline={\PaliGlossA,\PaliGlossB}]
athāparaṁ[then also] etadavoca[he said] satthā[taught] —[-]
\endgl
\nopagebreak
\linespread{0.5}
\begin{spacin}{0.2}
{\PaliGlossFT the Teacher further said this:}
\end{spacin}
\vskip 12pt
\end{samepage}
\vskip 0.2in
\begin{samepage}
\begingl[glneveryline={\PaliGlossA,\PaliGlossB}]
733.[-] “upadhinidānā[attachment.cause] pabhavanti[originate] dukkhā,[suffering] ye[-] keci[-] lokasmimanekarūpā.[world.various.form]
\endgl
\nopagebreak
\linespread{0.5}
\begin{spacin}{0.2}
{\PaliGlossFT 728. “Sufferings in their many forms in the world originate based on acquisition.}
\end{spacin}
\vskip 12pt
\end{samepage}
\begin{samepage}
\begingl[glneveryline={\PaliGlossA,\PaliGlossB}]
yo[whoever] ve[indeed] avidvā[find] upadhiṁ[attachment] karoti,[make] punappunaṁ[again.again] dukkhamupeti[suffer.obtain] mando.[dull]
\endgl
\nopagebreak
\linespread{0.5}
\begin{spacin}{0.2}
{\PaliGlossFT The ignorant dullard who creates acquisition encounters suffering again and again.}
\end{spacin}
\vskip 12pt
\end{samepage}
\begin{samepage}
\begingl[glneveryline={\PaliGlossA,\PaliGlossB}]
tasmā[therefore] pajānaṁ[knows clearly] upadhiṁ[attachment] na[not] kayirā,[create] dukkhassa[suffering] jātippabhavānupassī”ti.[birth.origin.know]
\endgl
\nopagebreak
\linespread{0.5}
\begin{spacin}{0.2}
{\PaliGlossFT Therefore, understanding, one should not create acquisition, contemplating it as the genesis and origin of suffering.}
\end{spacin}
\vskip 12pt
\end{samepage}
\vskip 0.2in
\begin{samepage}
\begingl[glneveryline={\PaliGlossA,\PaliGlossB}]
(3)[-]
\endgl
\nopagebreak
\linespread{0.5}
\begin{spacin}{0.2}
{\PaliGlossFT [3. Ignorance]}
\end{spacin}
\vskip 12pt
\end{samepage}
\begin{samepage}
\begingl[glneveryline={\PaliGlossA,\PaliGlossB}]
“‘siyā[could.be] aññenapi[another] pariyāyena[method] sammā[rightly] dvayatānupassanā’ti,[pair.contemplate] iti[-] ce,[-] bhikkhave,[-] pucchitāro[asked] assu;[to be]
\endgl
\nopagebreak
\linespread{0.5}
\begin{spacin}{0.2}
{\PaliGlossFT “If, bhikkhus, there are those who ask, ‘Could there be correct contemplation of dyads in some other way?’}
\end{spacin}
\vskip 12pt
\end{samepage}
\begin{samepage}
\begingl[glneveryline={\PaliGlossA,\PaliGlossB}]
‘siyā’tissu[could.be] vacanīyā.[utterance]
\endgl
\nopagebreak
\linespread{0.5}
\begin{spacin}{0.2}
{\PaliGlossFT you should answer them thus: ‘There could be.’}
\end{spacin}
\vskip 12pt
\end{samepage}
\begin{samepage}
\begingl[glneveryline={\PaliGlossA,\PaliGlossB}]
kathañca[how] siyā?[could.be]
\endgl
\nopagebreak
\linespread{0.5}
\begin{spacin}{0.2}
{\PaliGlossFT And how could there be?}
\end{spacin}
\vskip 12pt
\end{samepage}
\begin{samepage}
\begingl[glneveryline={\PaliGlossA,\PaliGlossB}]
yaṁ[-] kiñci[-] dukkhaṁ[suffering] sambhoti[arises] sabbaṁ[all] avijjāpaccayāti,[ignorance.cause] ayamekānupassanā.[this.one.contemplate]
\endgl
\nopagebreak
\linespread{0.5}
\begin{spacin}{0.2}
{\PaliGlossFT ‘Whatever suffering originates is all conditioned by ignorance’ — ­this is one contemplation.}
\end{spacin}
\vskip 12pt
\end{samepage}
\begin{samepage}
\begingl[glneveryline={\PaliGlossA,\PaliGlossB}]
avijjāya[ignorance] tveva[that] asesavirāganirodhā[entire.dispation.cease] natthi[not.is] dukkhassa[suffering] sambhavoti,[origin] ayaṁ[this] dutiyānupassanā.[second.contemplate]
\endgl
\nopagebreak
\linespread{0.5}
\begin{spacin}{0.2}
{\PaliGlossFT ‘With the remainderless fading away and cessation of ignorance, there is no origination of suffering’ — this is a second contemplation.}
\end{spacin}
\vskip 12pt
\end{samepage}
\begin{samepage}
\begingl[glneveryline={\PaliGlossA,\PaliGlossB}]
evaṁ[thus] sammā[rightly] dvayatānupassino[pair.contemplate] kho,[-] bhikkhave,[-] bhikkhuno[-] appamattassa[vigilant] ātāpino[ardent] pahitattassa[resolute] viharato[abides] dvinnaṁ[pair] phalānaṁ[fruit] aññataraṁ[certain] phalaṁ[fruit] pāṭikaṅkhaṁ[expected] —[-] diṭṭheva[vision] dhamme[the Norm] aññā,[other] sati[state] vā[-] upādisese[some fuel left] anāgāmitā”ti.[not return]
\endgl
\nopagebreak
\linespread{0.5}
\begin{spacin}{0.2}
{\PaliGlossFT When a bhikkhu dwells thus correctly contemplating a dyad — heedful, ardent, and resolute — one of two fruits is to be expected of him: either final knowledge in this very life or, if there is a residue remaining, the state of non-returning.”}
\end{spacin}
\vskip 12pt
\end{samepage}
\begin{samepage}
\begingl[glneveryline={\PaliGlossA,\PaliGlossB}]
idamavoca[this said] bhagavā.[fortunate]
\endgl
\nopagebreak
\linespread{0.5}
\begin{spacin}{0.2}
{\PaliGlossFT This is what the Blessed One said.}
\end{spacin}
\vskip 12pt
\end{samepage}
\begin{samepage}
\begingl[glneveryline={\PaliGlossA,\PaliGlossB}]
athāparaṁ[then also] etadavoca[he said] satthā[taught] —[-]
\endgl
\nopagebreak
\linespread{0.5}
\begin{spacin}{0.2}
{\PaliGlossFT the Teacher further said this:}
\end{spacin}
\vskip 12pt
\end{samepage}
\vskip 0.2in
\begin{samepage}
\begingl[glneveryline={\PaliGlossA,\PaliGlossB}]
734.[-] “jātimaraṇasaṁsāraṁ,[birth.death.saṁsāra] ye[-] vajanti[proceed] punappunaṁ.[again.again]
\endgl
\nopagebreak
\linespread{0.5}
\begin{spacin}{0.2}
{\PaliGlossFT 729. “Those who travel again and again in the saṃsāra of birth and death,}
\end{spacin}
\vskip 12pt
\end{samepage}
\begin{samepage}
\begingl[glneveryline={\PaliGlossA,\PaliGlossB}]
itthabhāvaññathābhāvaṁ,[thus.become.not.thus.become] avijjāyeva[ignorance] sā[-] gati.[going]
\endgl
\nopagebreak
\linespread{0.5}
\begin{spacin}{0.2}
{\PaliGlossFT with its becoming thus, becoming otherwise:  that journey is due to ignorance.}
\end{spacin}
\vskip 12pt
\end{samepage}
\begin{samepage}
\begingl[glneveryline={\PaliGlossA,\PaliGlossB}]
735.[-] “avijjā[ignorance] hāyaṁ[!this] mahāmoho,[great.delusion] yenidaṁ[proceeds] saṁsitaṁ[expected] ciraṁ.[lasting long]
\endgl
\nopagebreak
\linespread{0.5}
\begin{spacin}{0.2}
{\PaliGlossFT 730. “It is because of ignorance, this great delusion,  that one has wandered on for so long.}
\end{spacin}
\vskip 12pt
\end{samepage}
\begin{samepage}
\begingl[glneveryline={\PaliGlossA,\PaliGlossB}]
vijjāgatā[knowledge.gone] ca[-] ye[-] sattā,[being] na[not] te[-] gacchanti[go] punabbhavan”ti.[new existence]
\endgl
\nopagebreak
\linespread{0.5}
\begin{spacin}{0.2}
{\PaliGlossFT But those beings who have gained clear knowledge do not come back to renewed existence.}
\end{spacin}
\vskip 12pt
\end{samepage}
\vskip 0.2in
\begin{samepage}
\begingl[glneveryline={\PaliGlossA,\PaliGlossB}]
(4)[-]
\endgl
\nopagebreak
\linespread{0.5}
\begin{spacin}{0.2}
{\PaliGlossFT [4. Volitional activities]}
\end{spacin}
\vskip 12pt
\end{samepage}
\begin{samepage}
\begingl[glneveryline={\PaliGlossA,\PaliGlossB}]
“‘siyā[could.be] aññenapi[another] pariyāyena[method] sammā[rightly] dvayatānupassanā’ti,[pair.contemplate] iti[-] ce,[-] bhikkhave,[-] pucchitāro[asked] assu;[to be]
\endgl
\nopagebreak
\linespread{0.5}
\begin{spacin}{0.2}
{\PaliGlossFT “If, bhikkhus, there are those who ask, ‘Could there be correct contemplation of dyads in some other way?’}
\end{spacin}
\vskip 12pt
\end{samepage}
\begin{samepage}
\begingl[glneveryline={\PaliGlossA,\PaliGlossB}]
‘siyā’tissu[could.be] vacanīyā.[utterance]
\endgl
\nopagebreak
\linespread{0.5}
\begin{spacin}{0.2}
{\PaliGlossFT you should answer them thus: ‘There could be.’}
\end{spacin}
\vskip 12pt
\end{samepage}
\begin{samepage}
\begingl[glneveryline={\PaliGlossA,\PaliGlossB}]
kathañca[how] siyā?[could.be]
\endgl
\nopagebreak
\linespread{0.5}
\begin{spacin}{0.2}
{\PaliGlossFT And how could there be?}
\end{spacin}
\vskip 12pt
\end{samepage}
\begin{samepage}
\begingl[glneveryline={\PaliGlossA,\PaliGlossB}]
yaṁ[-] kiñci[-] dukkhaṁ[suffering] sambhoti[arises] sabbaṁ[all] saṅkhārapaccayāti,[formation.cause] ayamekānupassanā.[this.one.contemplate]
\endgl
\nopagebreak
\linespread{0.5}
\begin{spacin}{0.2}
{\PaliGlossFT ‘Whatever suffering originates is all conditioned by volitional activities’ — this is one contemplation.}
\end{spacin}
\vskip 12pt
\end{samepage}
\begin{samepage}
\begingl[glneveryline={\PaliGlossA,\PaliGlossB}]
saṅkhārānaṁ[formation] tveva[that] asesavirāganirodhā[entire.dispation.cease] natthi[not.is] dukkhassa[suffering] sambhavoti,[origin] ayaṁ[this] dutiyānupassanā.[second.contemplate]
\endgl
\nopagebreak
\linespread{0.5}
\begin{spacin}{0.2}
{\PaliGlossFT ‘With the remainderless fading away and cessation of volitional activities, there is no origination of suffering’ — this is a second contemplation.}
\end{spacin}
\vskip 12pt
\end{samepage}
\begin{samepage}
\begingl[glneveryline={\PaliGlossA,\PaliGlossB}]
evaṁ[thus] sammā[rightly] dvayatānupassino[pair.contemplate] kho,[-] bhikkhave,[-] bhikkhuno[-] appamattassa[vigilant] ātāpino[ardent] pahitattassa[resolute] viharato[abides] dvinnaṁ[pair] phalānaṁ[fruit] aññataraṁ[certain] phalaṁ[fruit] pāṭikaṅkhaṁ[expected] —[-] diṭṭheva[vision] dhamme[the Norm] aññā,[other] sati[state] vā[-] upādisese[some fuel left] anāgāmitā”ti.[not return]
\endgl
\nopagebreak
\linespread{0.5}
\begin{spacin}{0.2}
{\PaliGlossFT When a bhikkhu dwells thus correctly contemplating a dyad — heedful, ardent, and resolute — one of two fruits is to be expected of him: either final knowledge in this very life or, if there is a residue remaining, the state of non-returning.”}
\end{spacin}
\vskip 12pt
\end{samepage}
\begin{samepage}
\begingl[glneveryline={\PaliGlossA,\PaliGlossB}]
idamavoca[this said] bhagavā.[fortunate]
\endgl
\nopagebreak
\linespread{0.5}
\begin{spacin}{0.2}
{\PaliGlossFT This is what the Blessed One said.}
\end{spacin}
\vskip 12pt
\end{samepage}
\begin{samepage}
\begingl[glneveryline={\PaliGlossA,\PaliGlossB}]
athāparaṁ[then also] etadavoca[he said] satthā[taught] —[-]
\endgl
\nopagebreak
\linespread{0.5}
\begin{spacin}{0.2}
{\PaliGlossFT the Teacher further said this:}
\end{spacin}
\vskip 12pt
\end{samepage}
\vskip 0.2in
\begin{samepage}
\begingl[glneveryline={\PaliGlossA,\PaliGlossB}]
736.[-] “yaṁ[-] kiñci[-] dukkhaṁ[suffering] sambhoti,[arises] sabbaṁ[all] saṅkhārapaccayā.[formation.cause]
\endgl
\nopagebreak
\linespread{0.5}
\begin{spacin}{0.2}
{\PaliGlossFT 731. “Whatever suffering originates  is all conditioned by volitional activities.}
\end{spacin}
\vskip 12pt
\end{samepage}
\begin{samepage}
\begingl[glneveryline={\PaliGlossA,\PaliGlossB}]
saṅkhārānaṁ[formation] nirodhena,[cessation] natthi[not.is] dukkhassa[suffering] sambhavo.[origin]
\endgl
\nopagebreak
\linespread{0.5}
\begin{spacin}{0.2}
{\PaliGlossFT With the cessation of volitional activities, there is no origination of suffering.}
\end{spacin}
\vskip 12pt
\end{samepage}
\begin{samepage}
\begingl[glneveryline={\PaliGlossA,\PaliGlossB}]
737.[-] “etamādīnavaṁ[this.disadvantage] ñatvā,[having known] dukkhaṁ[suffering] saṅkhārapaccayā.[formation.cause]
\endgl
\nopagebreak
\linespread{0.5}
\begin{spacin}{0.2}
{\PaliGlossFT 732. “When one has known this danger, ‘Suffering is conditioned by volitional activities,’}
\end{spacin}
\vskip 12pt
\end{samepage}
\begin{samepage}
\begingl[glneveryline={\PaliGlossA,\PaliGlossB}]
sabbasaṅkhārasamathā,[all.formation.stilling] saññānaṁ[perception] uparodhanā.[destruction]
\endgl
\nopagebreak
\linespread{0.5}
\begin{spacin}{0.2}
{\PaliGlossFT by the stilling of all volitional activities,  by the stopping of perceptions,}
\end{spacin}
\vskip 12pt
\end{samepage}
\begin{samepage}
\begingl[glneveryline={\PaliGlossA,\PaliGlossB}]
evaṁ[thus] dukkhakkhayo[suffer.extinction] hoti,[to be] etaṁ[that] ñatvā[having known] yathātathaṁ.[as it.truth]
\endgl
\nopagebreak
\linespread{0.5}
\begin{spacin}{0.2}
{\PaliGlossFT the destruction of suffering occurs when one has known this as it really is.}
\end{spacin}
\vskip 12pt
\end{samepage}
\begin{samepage}
\begingl[glneveryline={\PaliGlossA,\PaliGlossB}]
738.[-] “sammaddasā[right.seeing] vedaguno,[highest knowledge] sammadaññāya[understood perfectly] paṇḍitā.[wise]
\endgl
\nopagebreak
\linespread{0.5}
\begin{spacin}{0.2}
{\PaliGlossFT 733. “Seeing rightly, the masters of knowledge, the wise ones, having correctly known this,}
\end{spacin}
\vskip 12pt
\end{samepage}
\begin{samepage}
\begingl[glneveryline={\PaliGlossA,\PaliGlossB}]
abhibhuyya[conqueror] mārasaṁyogaṁ,[mara.bond] na[not] gacchanti[go] punabbhavan”ti.[new existence]
\endgl
\nopagebreak
\linespread{0.5}
\begin{spacin}{0.2}
{\PaliGlossFT overcome the yoke of Māra  and do not come back to renewed existence.}
\end{spacin}
\vskip 12pt
\end{samepage}
\vskip 0.2in
\begin{samepage}
\begingl[glneveryline={\PaliGlossA,\PaliGlossB}]
(5)[-]
\endgl
\nopagebreak
\linespread{0.5}
\begin{spacin}{0.2}
{\PaliGlossFT [5. Consciousness]}
\end{spacin}
\vskip 12pt
\end{samepage}
\begin{samepage}
\begingl[glneveryline={\PaliGlossA,\PaliGlossB}]
“‘siyā[could.be] aññenapi[another] pariyāyena[method] sammā[rightly] dvayatānupassanā’ti,[pair.contemplate] iti[-] ce,[-] bhikkhave,[-] pucchitāro[asked] assu;[to be]
\endgl
\nopagebreak
\linespread{0.5}
\begin{spacin}{0.2}
{\PaliGlossFT “If, bhikkhus, there are those who ask, ‘Could there be correct contemplation of dyads in some other way?’}
\end{spacin}
\vskip 12pt
\end{samepage}
\begin{samepage}
\begingl[glneveryline={\PaliGlossA,\PaliGlossB}]
‘siyā’tissu[could.be] vacanīyā.[utterance]
\endgl
\nopagebreak
\linespread{0.5}
\begin{spacin}{0.2}
{\PaliGlossFT you should answer them thus: ‘There could be.’}
\end{spacin}
\vskip 12pt
\end{samepage}
\begin{samepage}
\begingl[glneveryline={\PaliGlossA,\PaliGlossB}]
kathañca[how] siyā?[could.be]
\endgl
\nopagebreak
\linespread{0.5}
\begin{spacin}{0.2}
{\PaliGlossFT And how could there be?}
\end{spacin}
\vskip 12pt
\end{samepage}
\begin{samepage}
\begingl[glneveryline={\PaliGlossA,\PaliGlossB}]
yaṁ[-] kiñci[-] dukkhaṁ[suffering] sambhoti[arises] sabbaṁ[all] viññāṇapaccayāti,[consciousness.cause] ayamekānupassanā.[this.one.contemplate]
\endgl
\nopagebreak
\linespread{0.5}
\begin{spacin}{0.2}
{\PaliGlossFT ‘Whatever suffering originates is all conditioned by consciousness’ — ­this is one contemplation.}
\end{spacin}
\vskip 12pt
\end{samepage}
\begin{samepage}
\begingl[glneveryline={\PaliGlossA,\PaliGlossB}]
viññāṇassa[consciousness] tveva[that] asesavirāganirodhā[entire.dispation.cease] natthi[not.is] dukkhassa[suffering] sambhavoti,[origin] ayaṁ[this] dutiyānupassanā.[second.contemplate]
\endgl
\nopagebreak
\linespread{0.5}
\begin{spacin}{0.2}
{\PaliGlossFT ‘With the remainderless fading away and cessation of consciousness, there is no origination of suffering’ — this is a second contemplation.}
\end{spacin}
\vskip 12pt
\end{samepage}
\begin{samepage}
\begingl[glneveryline={\PaliGlossA,\PaliGlossB}]
evaṁ[thus] sammā[rightly] dvayatānupassino[pair.contemplate] kho,[-] bhikkhave,[-] bhikkhuno[-] appamattassa[vigilant] ātāpino[ardent] pahitattassa[resolute] viharato[abides] dvinnaṁ[pair] phalānaṁ[fruit] aññataraṁ[certain] phalaṁ[fruit] pāṭikaṅkhaṁ[expected] —[-] diṭṭheva[vision] dhamme[the Norm] aññā,[other] sati[state] vā[-] upādisese[some fuel left] anāgāmitā”ti.[not return]
\endgl
\nopagebreak
\linespread{0.5}
\begin{spacin}{0.2}
{\PaliGlossFT When a bhikkhu dwells thus correctly contemplating a dyad — heedful, ardent, and resolute — one of two fruits is to be expected of him: either final knowledge in this very life or, if there is a residue remaining, the state of non-returning.”}
\end{spacin}
\vskip 12pt
\end{samepage}
\begin{samepage}
\begingl[glneveryline={\PaliGlossA,\PaliGlossB}]
idamavoca[this said] bhagavā.[fortunate]
\endgl
\nopagebreak
\linespread{0.5}
\begin{spacin}{0.2}
{\PaliGlossFT This is what the Blessed One said.}
\end{spacin}
\vskip 12pt
\end{samepage}
\begin{samepage}
\begingl[glneveryline={\PaliGlossA,\PaliGlossB}]
athāparaṁ[then also] etadavoca[he said] satthā[taught] —[-]
\endgl
\nopagebreak
\linespread{0.5}
\begin{spacin}{0.2}
{\PaliGlossFT the Teacher further said this:}
\end{spacin}
\vskip 12pt
\end{samepage}
\vskip 0.2in
\begin{samepage}
\begingl[glneveryline={\PaliGlossA,\PaliGlossB}]
739.[-] “yaṁ[-] kiñci[-] dukkhaṁ[suffering] sambhoti,[arises] sabbaṁ[all] viññāṇapaccayā.[consciousness.cause]
\endgl
\nopagebreak
\linespread{0.5}
\begin{spacin}{0.2}
{\PaliGlossFT 734. “Whatever suffering originates  is all conditioned by consciousness.}
\end{spacin}
\vskip 12pt
\end{samepage}
\begin{samepage}
\begingl[glneveryline={\PaliGlossA,\PaliGlossB}]
viññāṇassa[consciousness] nirodhena,[cessation] natthi[not.is] dukkhassa[suffering] sambhavo.[origin]
\endgl
\nopagebreak
\linespread{0.5}
\begin{spacin}{0.2}
{\PaliGlossFT With the cessation of consciousness, there is no origination of suffering.}
\end{spacin}
\vskip 12pt
\end{samepage}
\begin{samepage}
\begingl[glneveryline={\PaliGlossA,\PaliGlossB}]
740.[-] “etamādīnavaṁ[this.disadvantage] ñatvā,[having known] dukkhaṁ[suffering] viññāṇapaccayā.[consciousness.cause]
\endgl
\nopagebreak
\linespread{0.5}
\begin{spacin}{0.2}
{\PaliGlossFT 735. “Having understood this danger, ‘Suffering is conditioned by consciousness,’}
\end{spacin}
\vskip 12pt
\end{samepage}
\begin{samepage}
\begingl[glneveryline={\PaliGlossA,\PaliGlossB}]
viññāṇūpasamā[consciousness.cause] bhikkhu,[-] nicchāto[no hunger] parinibbuto”ti.[final.peace]
\endgl
\nopagebreak
\linespread{0.5}
\begin{spacin}{0.2}
{\PaliGlossFT by the stilling of consciousness, a bhikkhu, hungerless, has attained nibbāna.}
\end{spacin}
\vskip 12pt
\end{samepage}
\vskip 0.2in
\begin{samepage}
\begingl[glneveryline={\PaliGlossA,\PaliGlossB}]
(6)[-]
\endgl
\nopagebreak
\linespread{0.5}
\begin{spacin}{0.2}
{\PaliGlossFT [6. Contact]}
\end{spacin}
\vskip 12pt
\end{samepage}
\begin{samepage}
\begingl[glneveryline={\PaliGlossA,\PaliGlossB}]
“‘siyā[could.be] aññenapi[another] pariyāyena[method] sammā[rightly] dvayatānupassanā’ti,[pair.contemplate] iti[-] ce,[-] bhikkhave,[-] pucchitāro[asked] assu;[to be]
\endgl
\nopagebreak
\linespread{0.5}
\begin{spacin}{0.2}
{\PaliGlossFT “If, bhikkhus, there are those who ask, ‘Could there be correct contemplation of dyads in some other way?’}
\end{spacin}
\vskip 12pt
\end{samepage}
\begin{samepage}
\begingl[glneveryline={\PaliGlossA,\PaliGlossB}]
‘siyā’tissu[could.be] vacanīyā.[utterance]
\endgl
\nopagebreak
\linespread{0.5}
\begin{spacin}{0.2}
{\PaliGlossFT you should answer them thus: ‘There could be.’}
\end{spacin}
\vskip 12pt
\end{samepage}
\begin{samepage}
\begingl[glneveryline={\PaliGlossA,\PaliGlossB}]
kathañca[how] siyā?[could.be]
\endgl
\nopagebreak
\linespread{0.5}
\begin{spacin}{0.2}
{\PaliGlossFT And how could there be?}
\end{spacin}
\vskip 12pt
\end{samepage}
\begin{samepage}
\begingl[glneveryline={\PaliGlossA,\PaliGlossB}]
yaṁ[-] kiñci[-] dukkhaṁ[suffering] sambhoti[arises] sabbaṁ[all] phassapaccayāti,[contact.cause] ayamekānupassanā.[this.one.contemplate]
\endgl
\nopagebreak
\linespread{0.5}
\begin{spacin}{0.2}
{\PaliGlossFT ‘Whatever suffering originates is all conditioned by contact’ — this is one contemplation.}
\end{spacin}
\vskip 12pt
\end{samepage}
\begin{samepage}
\begingl[glneveryline={\PaliGlossA,\PaliGlossB}]
phassassa[contact] tveva[that] asesavirāganirodhā[entire.dispation.cease] natthi[not.is] dukkhassa[suffering] sambhavoti,[origin] ayaṁ[this] dutiyānupassanā.[second.contemplate]
\endgl
\nopagebreak
\linespread{0.5}
\begin{spacin}{0.2}
{\PaliGlossFT ‘With the remainderless fading away and cessation of contact, there is no origination of suffering’ — this is a second contemplation.}
\end{spacin}
\vskip 12pt
\end{samepage}
\begin{samepage}
\begingl[glneveryline={\PaliGlossA,\PaliGlossB}]
evaṁ[thus] sammā[rightly] dvayatānupassino[pair.contemplate] kho,[-] bhikkhave,[-] bhikkhuno[-] appamattassa[vigilant] ātāpino[ardent] pahitattassa[resolute] viharato[abides] dvinnaṁ[pair] phalānaṁ[fruit] aññataraṁ[certain] phalaṁ[fruit] pāṭikaṅkhaṁ[expected] —[-] diṭṭheva[vision] dhamme[the Norm] aññā,[other] sati[state] vā[-] upādisese[some fuel left] anāgāmitā”ti.[not return]
\endgl
\nopagebreak
\linespread{0.5}
\begin{spacin}{0.2}
{\PaliGlossFT When a bhikkhu dwells thus correctly contemplating a dyad — heedful, ardent, and resolute — one of two fruits is to be expected of him: either final knowledge in this very life or, if there is a residue remaining, the state of non-returning.”}
\end{spacin}
\vskip 12pt
\end{samepage}
\begin{samepage}
\begingl[glneveryline={\PaliGlossA,\PaliGlossB}]
idamavoca[this said] bhagavā.[fortunate]
\endgl
\nopagebreak
\linespread{0.5}
\begin{spacin}{0.2}
{\PaliGlossFT This is what the Blessed One said.}
\end{spacin}
\vskip 12pt
\end{samepage}
\begin{samepage}
\begingl[glneveryline={\PaliGlossA,\PaliGlossB}]
athāparaṁ[then also] etadavoca[he said] satthā[taught] —[-]
\endgl
\nopagebreak
\linespread{0.5}
\begin{spacin}{0.2}
{\PaliGlossFT the Teacher further said this:}
\end{spacin}
\vskip 12pt
\end{samepage}
\vskip 0.2in
\begin{samepage}
\begingl[glneveryline={\PaliGlossA,\PaliGlossB}]
741.[-] “tesaṁ[-] phassaparetānaṁ,[contact.overcome] bhavasotānusārinaṁ.[existence.stream.flowing]
\endgl
\nopagebreak
\linespread{0.5}
\begin{spacin}{0.2}
{\PaliGlossFT 736. “Those afflicted by contact, flowing along with the stream of existence,}
\end{spacin}
\vskip 12pt
\end{samepage}
\begin{samepage}
\begingl[glneveryline={\PaliGlossA,\PaliGlossB}]
kummaggapaṭipannānaṁ,[wrong path.followed] ārā[far from] saṁyojanakkhayo.[fetter.destruct]
\endgl
\nopagebreak
\linespread{0.5}
\begin{spacin}{0.2}
{\PaliGlossFT have entered upon a deviant path: the destruction of the fetters is far from them.}
\end{spacin}
\vskip 12pt
\end{samepage}
\begin{samepage}
\begingl[glneveryline={\PaliGlossA,\PaliGlossB}]
742.[-] “ye[-] ca[-] phassaṁ[contact] pariññāya,[having known] aññāyupasame[other.peace] ratā.[delight]
\endgl
\nopagebreak
\linespread{0.5}
\begin{spacin}{0.2}
{\PaliGlossFT 737. “But those who have fully understood contact, who, having known it, delight in peace,}
\end{spacin}
\vskip 12pt
\end{samepage}
\begin{samepage}
\begingl[glneveryline={\PaliGlossA,\PaliGlossB}]
te[-] ve[indeed] phassābhisamayā,[contact.realize] nicchātā[no hunger] parinibbutā”ti.[final.peace]
\endgl
\nopagebreak
\linespread{0.5}
\begin{spacin}{0.2}
{\PaliGlossFT by breaking through contact, hungerless, are fully quenched.}
\end{spacin}
\vskip 12pt
\end{samepage}
\vskip 0.2in
\begin{samepage}
\begingl[glneveryline={\PaliGlossA,\PaliGlossB}]
(7)[-]
\endgl
\nopagebreak
\linespread{0.5}
\begin{spacin}{0.2}
{\PaliGlossFT [7. Feeling]}
\end{spacin}
\vskip 12pt
\end{samepage}
\begin{samepage}
\begingl[glneveryline={\PaliGlossA,\PaliGlossB}]
“‘siyā[could.be] aññenapi[another] pariyāyena[method] sammā[rightly] dvayatānupassanā’ti,[pair.contemplate] iti[-] ce,[-] bhikkhave,[-] pucchitāro[asked] assu;[to be]
\endgl
\nopagebreak
\linespread{0.5}
\begin{spacin}{0.2}
{\PaliGlossFT “If, bhikkhus, there are those who ask, ‘Could there be correct contemplation of dyads in some other way?’}
\end{spacin}
\vskip 12pt
\end{samepage}
\begin{samepage}
\begingl[glneveryline={\PaliGlossA,\PaliGlossB}]
‘siyā’tissu[could.be] vacanīyā.[utterance]
\endgl
\nopagebreak
\linespread{0.5}
\begin{spacin}{0.2}
{\PaliGlossFT you should answer them thus: ‘There could be.’}
\end{spacin}
\vskip 12pt
\end{samepage}
\begin{samepage}
\begingl[glneveryline={\PaliGlossA,\PaliGlossB}]
kathañca[how] siyā?[could.be]
\endgl
\nopagebreak
\linespread{0.5}
\begin{spacin}{0.2}
{\PaliGlossFT And how could there be?}
\end{spacin}
\vskip 12pt
\end{samepage}
\begin{samepage}
\begingl[glneveryline={\PaliGlossA,\PaliGlossB}]
yaṁ[-] kiñci[-] dukkhaṁ[suffering] sambhoti[arises] sabbaṁ[all] vedanāpaccayāti,[feeling.cause] ayamekānupassanā.[this.one.contemplate]
\endgl
\nopagebreak
\linespread{0.5}
\begin{spacin}{0.2}
{\PaliGlossFT ‘Whatever suffering originates is all conditioned by feeling’ — this is one contemplation.}
\end{spacin}
\vskip 12pt
\end{samepage}
\begin{samepage}
\begingl[glneveryline={\PaliGlossA,\PaliGlossB}]
vedanānaṁ[feeling] tveva[that] asesavirāganirodhā[entire.dispation.cease] natthi[not.is] dukkhassa[suffering] sambhavoti,[origin] ayaṁ[this] dutiyānupassanā.[second.contemplate]
\endgl
\nopagebreak
\linespread{0.5}
\begin{spacin}{0.2}
{\PaliGlossFT ‘With the remainderless fading away and cessation of feeling, there is no origination of suffering’ — this is a second contemplation.}
\end{spacin}
\vskip 12pt
\end{samepage}
\begin{samepage}
\begingl[glneveryline={\PaliGlossA,\PaliGlossB}]
evaṁ[thus] sammā[rightly] dvayatānupassino[pair.contemplate] kho,[-] bhikkhave,[-] bhikkhuno[-] appamattassa[vigilant] ātāpino[ardent] pahitattassa[resolute] viharato[abides] dvinnaṁ[pair] phalānaṁ[fruit] aññataraṁ[certain] phalaṁ[fruit] pāṭikaṅkhaṁ[expected] —[-] diṭṭheva[vision] dhamme[the Norm] aññā,[other] sati[state] vā[-] upādisese[some fuel left] anāgāmitā”ti.[not return]
\endgl
\nopagebreak
\linespread{0.5}
\begin{spacin}{0.2}
{\PaliGlossFT When a bhikkhu dwells thus correctly contemplating a dyad — heedful, ardent, and resolute — one of two fruits is to be expected of him: either final knowledge in this very life or, if there is a residue remaining, the state of non-returning.”}
\end{spacin}
\vskip 12pt
\end{samepage}
\begin{samepage}
\begingl[glneveryline={\PaliGlossA,\PaliGlossB}]
idamavoca[this said] bhagavā.[fortunate]
\endgl
\nopagebreak
\linespread{0.5}
\begin{spacin}{0.2}
{\PaliGlossFT This is what the Blessed One said.}
\end{spacin}
\vskip 12pt
\end{samepage}
\begin{samepage}
\begingl[glneveryline={\PaliGlossA,\PaliGlossB}]
athāparaṁ[then also] etadavoca[he said] satthā[taught] —[-]
\endgl
\nopagebreak
\linespread{0.5}
\begin{spacin}{0.2}
{\PaliGlossFT the Teacher further said this:}
\end{spacin}
\vskip 12pt
\end{samepage}
\vskip 0.2in
\begin{samepage}
\begingl[glneveryline={\PaliGlossA,\PaliGlossB}]
743.[-] “sukhaṁ[pleasant] vā[-] yadi[if] vā[-] dukkhaṁ,[suffering] adukkhamasukhaṁ[neither.pain.nor.pleasant] saha.[endure]
\endgl
\nopagebreak
\linespread{0.5}
\begin{spacin}{0.2}
{\PaliGlossFT 738. “Whether it is pleasant or painful  or neither-painful-nor-pleasant,}
\end{spacin}
\vskip 12pt
\end{samepage}
\begin{samepage}
\begingl[glneveryline={\PaliGlossA,\PaliGlossB}]
ajjhattañca[internal] bahiddhā[external] ca,[-] yaṁ[-] kiñci[-] atthi[to be] veditaṁ.[felt]
\endgl
\nopagebreak
\linespread{0.5}
\begin{spacin}{0.2}
{\PaliGlossFT whatever there is that is felt internally and externally,}
\end{spacin}
\vskip 12pt
\end{samepage}
\begin{samepage}
\begingl[glneveryline={\PaliGlossA,\PaliGlossB}]
744.[-] “etaṁ[that] dukkhanti[suffering] ñatvāna,[having known] mosadhammaṁ[false.nature] palokinaṁ.[certain decay]
\endgl
\nopagebreak
\linespread{0.5}
\begin{spacin}{0.2}
{\PaliGlossFT 739. “having known, ‘This is suffering,  of a false nature, disintegrating,’}
\end{spacin}
\vskip 12pt
\end{samepage}
\begin{samepage}
\begingl[glneveryline={\PaliGlossA,\PaliGlossB}]
phussa[touch] phussa[touch] vayaṁ[loss] passaṁ,[see] evaṁ[thus] tattha[there] vijānati.[knows]
\endgl
\nopagebreak
\linespread{0.5}
\begin{spacin}{0.2}
{\PaliGlossFT having touched and touched them, seeing their vanishing, one understands them thus.}
\end{spacin}
\vskip 12pt
\end{samepage}
\begin{samepage}
\begingl[glneveryline={\PaliGlossA,\PaliGlossB}]
vedanānaṁ[feeling] khayā[destruction] bhikkhu,[-] nicchāto[no hunger] parinibbuto”ti.[final.peace]
\endgl
\nopagebreak
\linespread{0.5}
\begin{spacin}{0.2}
{\PaliGlossFT Through the destruction of feelings, a bhikkhu, hungerless, is fully quenched.}
\end{spacin}
\vskip 12pt
\end{samepage}
\vskip 0.2in
\begin{samepage}
\begingl[glneveryline={\PaliGlossA,\PaliGlossB}]
(8)[-]
\endgl
\nopagebreak
\linespread{0.5}
\begin{spacin}{0.2}
{\PaliGlossFT [8. Craving]}
\end{spacin}
\vskip 12pt
\end{samepage}
\begin{samepage}
\begingl[glneveryline={\PaliGlossA,\PaliGlossB}]
“‘siyā[could.be] aññenapi[another] pariyāyena[method] sammā[rightly] dvayatānupassanā’ti,[pair.contemplate] iti[-] ce,[-] bhikkhave,[-] pucchitāro[asked] assu;[to be]
\endgl
\nopagebreak
\linespread{0.5}
\begin{spacin}{0.2}
{\PaliGlossFT “If, bhikkhus, there are those who ask, ‘Could there be correct contemplation of dyads in some other way?’}
\end{spacin}
\vskip 12pt
\end{samepage}
\begin{samepage}
\begingl[glneveryline={\PaliGlossA,\PaliGlossB}]
‘siyā’tissu[could.be] vacanīyā.[utterance]
\endgl
\nopagebreak
\linespread{0.5}
\begin{spacin}{0.2}
{\PaliGlossFT you should answer them thus: ‘There could be.’}
\end{spacin}
\vskip 12pt
\end{samepage}
\begin{samepage}
\begingl[glneveryline={\PaliGlossA,\PaliGlossB}]
kathañca[how] siyā?[could.be]
\endgl
\nopagebreak
\linespread{0.5}
\begin{spacin}{0.2}
{\PaliGlossFT And how could there be?}
\end{spacin}
\vskip 12pt
\end{samepage}
\begin{samepage}
\begingl[glneveryline={\PaliGlossA,\PaliGlossB}]
yaṁ[-] kiñci[-] dukkhaṁ[suffering] sambhoti[arises] sabbaṁ[all] taṇhāpaccayāti,[-] ayamekānupassanā.[this.one.contemplate]
\endgl
\nopagebreak
\linespread{0.5}
\begin{spacin}{0.2}
{\PaliGlossFT ‘Whatever suffering originates is all conditioned by craving’ — this is one contemplation.}
\end{spacin}
\vskip 12pt
\end{samepage}
\begin{samepage}
\begingl[glneveryline={\PaliGlossA,\PaliGlossB}]
taṇhāya[craving] tveva[that] asesavirāganirodhā[entire.dispation.cease] natthi[not.is] dukkhassa[suffering] sambhavoti,[origin] ayaṁ[this] dutiyānupassanā.[second.contemplate]
\endgl
\nopagebreak
\linespread{0.5}
\begin{spacin}{0.2}
{\PaliGlossFT ‘With the remainderless fading away and cessation of craving, there is no origination of suffering’ — this is a second contemplation.}
\end{spacin}
\vskip 12pt
\end{samepage}
\begin{samepage}
\begingl[glneveryline={\PaliGlossA,\PaliGlossB}]
evaṁ[thus] sammā[rightly] dvayatānupassino[pair.contemplate] kho,[-] bhikkhave,[-] bhikkhuno[-] appamattassa[vigilant] ātāpino[ardent] pahitattassa[resolute] viharato[abides] dvinnaṁ[pair] phalānaṁ[fruit] aññataraṁ[certain] phalaṁ[fruit] pāṭikaṅkhaṁ[expected] —[-] diṭṭheva[vision] dhamme[the Norm] aññā,[other] sati[state] vā[-] upādisese[some fuel left] anāgāmitā”ti.[not return]
\endgl
\nopagebreak
\linespread{0.5}
\begin{spacin}{0.2}
{\PaliGlossFT When a bhikkhu dwells thus correctly contemplating a dyad — heedful, ardent, and resolute — one of two fruits is to be expected of him: either final knowledge in this very life or, if there is a residue remaining, the state of non-returning.”}
\end{spacin}
\vskip 12pt
\end{samepage}
\begin{samepage}
\begingl[glneveryline={\PaliGlossA,\PaliGlossB}]
idamavoca[this said] bhagavā.[fortunate]
\endgl
\nopagebreak
\linespread{0.5}
\begin{spacin}{0.2}
{\PaliGlossFT This is what the Blessed One said.}
\end{spacin}
\vskip 12pt
\end{samepage}
\begin{samepage}
\begingl[glneveryline={\PaliGlossA,\PaliGlossB}]
athāparaṁ[then also] etadavoca[he said] satthā[taught] —[-]
\endgl
\nopagebreak
\linespread{0.5}
\begin{spacin}{0.2}
{\PaliGlossFT the Teacher further said this:}
\end{spacin}
\vskip 12pt
\end{samepage}
\vskip 0.2in
\begin{samepage}
\begingl[glneveryline={\PaliGlossA,\PaliGlossB}]
745.[-] “taṇhādutiyo[craving.companion] puriso,[man] dīghamaddhāna[long.journey] saṁsaraṁ.[wondering on]
\endgl
\nopagebreak
\linespread{0.5}
\begin{spacin}{0.2}
{\PaliGlossFT 740. “With craving as partner, a person, wandering on this long journey,}
\end{spacin}
\vskip 12pt
\end{samepage}
\begin{samepage}
\begingl[glneveryline={\PaliGlossA,\PaliGlossB}]
itthabhāvaññathābhāvaṁ,[thus.become.not.thus.become] saṁsāraṁ[wondering on] nātivattati.[not.go beyond]
\endgl
\nopagebreak
\linespread{0.5}
\begin{spacin}{0.2}
{\PaliGlossFT does not transcend saṃsāra,  with its becoming thus, becoming otherwise.}
\end{spacin}
\vskip 12pt
\end{samepage}
\begin{samepage}
\begingl[glneveryline={\PaliGlossA,\PaliGlossB}]
746.[-] “etamādīnavaṁ[this.disadvantage] ñatvā,[having known] taṇhaṁ[craving] dukkhassa[suffering] sambhavaṁ.[origin]
\endgl
\nopagebreak
\linespread{0.5}
\begin{spacin}{0.2}
{\PaliGlossFT 741. “Having known this danger, ‘Craving is the origin of suffering,’}
\end{spacin}
\vskip 12pt
\end{samepage}
\begin{samepage}
\begingl[glneveryline={\PaliGlossA,\PaliGlossB}]
vītataṇho[without.craving] anādāno,[without.taking] sato[mindful] bhikkhu[-] paribbaje”ti.[wonder about]
\endgl
\nopagebreak
\linespread{0.5}
\begin{spacin}{0.2}
{\PaliGlossFT a bhikkhu should wander mindfully, free of craving, without grasping.}
\end{spacin}
\vskip 12pt
\end{samepage}
\vskip 0.2in
\begin{samepage}
\begingl[glneveryline={\PaliGlossA,\PaliGlossB}]
(9)[-]
\endgl
\nopagebreak
\linespread{0.5}
\begin{spacin}{0.2}
{\PaliGlossFT [9. Clinging]}
\end{spacin}
\vskip 12pt
\end{samepage}
\begin{samepage}
\begingl[glneveryline={\PaliGlossA,\PaliGlossB}]
“‘siyā[could.be] aññenapi[another] pariyāyena[method] sammā[rightly] dvayatānupassanā’ti,[pair.contemplate] iti[-] ce,[-] bhikkhave,[-] pucchitāro[asked] assu;[to be]
\endgl
\nopagebreak
\linespread{0.5}
\begin{spacin}{0.2}
{\PaliGlossFT “If, bhikkhus, there are those who ask, ‘Could there be correct contemplation of dyads in some other way?’}
\end{spacin}
\vskip 12pt
\end{samepage}
\begin{samepage}
\begingl[glneveryline={\PaliGlossA,\PaliGlossB}]
‘siyā’tissu[could.be] vacanīyā.[utterance]
\endgl
\nopagebreak
\linespread{0.5}
\begin{spacin}{0.2}
{\PaliGlossFT you should answer them thus: ‘There could be.’}
\end{spacin}
\vskip 12pt
\end{samepage}
\begin{samepage}
\begingl[glneveryline={\PaliGlossA,\PaliGlossB}]
kathañca[how] siyā?[could.be]
\endgl
\nopagebreak
\linespread{0.5}
\begin{spacin}{0.2}
{\PaliGlossFT And how could there be?}
\end{spacin}
\vskip 12pt
\end{samepage}
\begin{samepage}
\begingl[glneveryline={\PaliGlossA,\PaliGlossB}]
yaṁ[-] kiñci[-] dukkhaṁ[suffering] sambhoti[arises] sabbaṁ[all] upādānapaccayāti,[grasping.cause] ayamekānupassanā.[this.one.contemplate]
\endgl
\nopagebreak
\linespread{0.5}
\begin{spacin}{0.2}
{\PaliGlossFT ‘Whatever suffering originates is all conditioned by clinging’ — this is one contemplation.}
\end{spacin}
\vskip 12pt
\end{samepage}
\begin{samepage}
\begingl[glneveryline={\PaliGlossA,\PaliGlossB}]
upādānānaṁ[grasping] tveva[that] asesavirāganirodhā[entire.dispation.cease] natthi[not.is] dukkhassa[suffering] sambhavoti,[origin] ayaṁ[this] dutiyānupassanā.[second.contemplate]
\endgl
\nopagebreak
\linespread{0.5}
\begin{spacin}{0.2}
{\PaliGlossFT ‘With the remainderless fading away and cessation of clinging, there is no origination of suffering’ — this is a second contemplation.}
\end{spacin}
\vskip 12pt
\end{samepage}
\begin{samepage}
\begingl[glneveryline={\PaliGlossA,\PaliGlossB}]
evaṁ[thus] sammā[rightly] dvayatānupassino[pair.contemplate] kho,[-] bhikkhave,[-] bhikkhuno[-] appamattassa[vigilant] ātāpino[ardent] pahitattassa[resolute] viharato[abides] dvinnaṁ[pair] phalānaṁ[fruit] aññataraṁ[certain] phalaṁ[fruit] pāṭikaṅkhaṁ[expected] —[-] diṭṭheva[vision] dhamme[the Norm] aññā,[other] sati[state] vā[-] upādisese[some fuel left] anāgāmitā”ti.[not return]
\endgl
\nopagebreak
\linespread{0.5}
\begin{spacin}{0.2}
{\PaliGlossFT When a bhikkhu dwells thus correctly contemplating a dyad — heedful, ardent, and resolute — one of two fruits is to be expected of him: either final knowledge in this very life or, if there is a residue remaining, the state of non-returning.”}
\end{spacin}
\vskip 12pt
\end{samepage}
\begin{samepage}
\begingl[glneveryline={\PaliGlossA,\PaliGlossB}]
idamavoca[this said] bhagavā.[fortunate]
\endgl
\nopagebreak
\linespread{0.5}
\begin{spacin}{0.2}
{\PaliGlossFT This is what the Blessed One said.}
\end{spacin}
\vskip 12pt
\end{samepage}
\begin{samepage}
\begingl[glneveryline={\PaliGlossA,\PaliGlossB}]
athāparaṁ[then also] etadavoca[he said] satthā[taught] —[-]
\endgl
\nopagebreak
\linespread{0.5}
\begin{spacin}{0.2}
{\PaliGlossFT the Teacher further said this:}
\end{spacin}
\vskip 12pt
\end{samepage}
\vskip 0.2in
\begin{samepage}
\begingl[glneveryline={\PaliGlossA,\PaliGlossB}]
747.[-] “upādānapaccayā[grasping.cause] bhavo,[existance] bhūto[exist] dukkhaṁ[suffering] nigacchati.[undergoes]
\endgl
\nopagebreak
\linespread{0.5}
\begin{spacin}{0.2}
{\PaliGlossFT 742. “Existence is conditioned by clinging; an existent being undergoes suffering.}
\end{spacin}
\vskip 12pt
\end{samepage}
\begin{samepage}
\begingl[glneveryline={\PaliGlossA,\PaliGlossB}]
jātassa[born] maraṇaṁ[death] hoti,[to be] eso[seek] dukkhassa[suffering] sambhavo.[origin]
\endgl
\nopagebreak
\linespread{0.5}
\begin{spacin}{0.2}
{\PaliGlossFT For one who is born there is death; this is the origin of suffering.}
\end{spacin}
\vskip 12pt
\end{samepage}
\begin{samepage}
\begingl[glneveryline={\PaliGlossA,\PaliGlossB}]
748.[-] “tasmā[therefore] upādānakkhayā,[grasp.extinction] sammadaññāya[understood perfectly] paṇḍitā.[wise]
\endgl
\nopagebreak
\linespread{0.5}
\begin{spacin}{0.2}
{\PaliGlossFT 743. “Therefore, having correctly understood, having directly known the destruction of birth,}
\end{spacin}
\vskip 12pt
\end{samepage}
\begin{samepage}
\begingl[glneveryline={\PaliGlossA,\PaliGlossB}]
jātikkhayaṁ[birth.destruction] abhiññāya,[well understood ] na[not] gacchanti[go] punabbhavan”ti.[new existence]
\endgl
\nopagebreak
\linespread{0.5}
\begin{spacin}{0.2}
{\PaliGlossFT through the destruction of clinging the wise do not come back to renewed existence.}
\end{spacin}
\vskip 12pt
\end{samepage}
\vskip 0.2in
\begin{samepage}
\begingl[glneveryline={\PaliGlossA,\PaliGlossB}]
(10)[-]
\endgl
\nopagebreak
\linespread{0.5}
\begin{spacin}{0.2}
{\PaliGlossFT [10. Instigation]}
\end{spacin}
\vskip 12pt
\end{samepage}
\begin{samepage}
\begingl[glneveryline={\PaliGlossA,\PaliGlossB}]
“‘siyā[could.be] aññenapi[another] pariyāyena[method] sammā[rightly] dvayatānupassanā’ti,[pair.contemplate] iti[-] ce,[-] bhikkhave,[-] pucchitāro[asked] assu;[to be]
\endgl
\nopagebreak
\linespread{0.5}
\begin{spacin}{0.2}
{\PaliGlossFT “If, bhikkhus, there are those who ask, ‘Could there be correct contemplation of dyads in some other way?’}
\end{spacin}
\vskip 12pt
\end{samepage}
\begin{samepage}
\begingl[glneveryline={\PaliGlossA,\PaliGlossB}]
‘siyā’tissu[could.be] vacanīyā.[utterance]
\endgl
\nopagebreak
\linespread{0.5}
\begin{spacin}{0.2}
{\PaliGlossFT you should answer them thus: ‘There could be.’}
\end{spacin}
\vskip 12pt
\end{samepage}
\begin{samepage}
\begingl[glneveryline={\PaliGlossA,\PaliGlossB}]
kathañca[how] siyā?[could.be]
\endgl
\nopagebreak
\linespread{0.5}
\begin{spacin}{0.2}
{\PaliGlossFT And how could there be?}
\end{spacin}
\vskip 12pt
\end{samepage}
\begin{samepage}
\begingl[glneveryline={\PaliGlossA,\PaliGlossB}]
yaṁ[-] kiñci[-] dukkhaṁ[suffering] sambhoti[arises] sabbaṁ[all] ārambhapaccayāti,[instigate.cause] ayamekānupassanā.[this.one.contemplate]
\endgl
\nopagebreak
\linespread{0.5}
\begin{spacin}{0.2}
{\PaliGlossFT ‘Whatever suffering originates is all conditioned by instigation’ — ­this is one contemplation.}
\end{spacin}
\vskip 12pt
\end{samepage}
\begin{samepage}
\begingl[glneveryline={\PaliGlossA,\PaliGlossB}]
ārambhānaṁ[-] tveva[that] asesavirāganirodhā[entire.dispation.cease] natthi[not.is] dukkhassa[suffering] sambhavoti,[origin] ayaṁ[this] dutiyānupassanā.[second.contemplate]
\endgl
\nopagebreak
\linespread{0.5}
\begin{spacin}{0.2}
{\PaliGlossFT ‘With the remainderless fading away and cessation of instigation, there is no origination of suffering’ — this is a second contemplation.}
\end{spacin}
\vskip 12pt
\end{samepage}
\begin{samepage}
\begingl[glneveryline={\PaliGlossA,\PaliGlossB}]
evaṁ[thus] sammā[rightly] dvayatānupassino[pair.contemplate] kho,[-] bhikkhave,[-] bhikkhuno[-] appamattassa[vigilant] ātāpino[ardent] pahitattassa[resolute] viharato[abides] dvinnaṁ[pair] phalānaṁ[fruit] aññataraṁ[certain] phalaṁ[fruit] pāṭikaṅkhaṁ[expected] —[-] diṭṭheva[vision] dhamme[the Norm] aññā,[other] sati[state] vā[-] upādisese[some fuel left] anāgāmitā”ti.[not return]
\endgl
\nopagebreak
\linespread{0.5}
\begin{spacin}{0.2}
{\PaliGlossFT When a bhikkhu dwells thus correctly contemplating a dyad — heedful, ardent, and resolute — one of two fruits is to be expected of him: either final knowledge in this very life or, if there is a residue remaining, the state of non-returning.”}
\end{spacin}
\vskip 12pt
\end{samepage}
\begin{samepage}
\begingl[glneveryline={\PaliGlossA,\PaliGlossB}]
idamavoca[this said] bhagavā.[fortunate]
\endgl
\nopagebreak
\linespread{0.5}
\begin{spacin}{0.2}
{\PaliGlossFT This is what the Blessed One said.}
\end{spacin}
\vskip 12pt
\end{samepage}
\begin{samepage}
\begingl[glneveryline={\PaliGlossA,\PaliGlossB}]
athāparaṁ[then also] etadavoca[he said] satthā[taught] —[-]
\endgl
\nopagebreak
\linespread{0.5}
\begin{spacin}{0.2}
{\PaliGlossFT the Teacher further said this:}
\end{spacin}
\vskip 12pt
\end{samepage}
\vskip 0.2in
\begin{samepage}
\begingl[glneveryline={\PaliGlossA,\PaliGlossB}]
749.[-] “yaṁ[-] kiñci[-] dukkhaṁ[suffering] sambhoti,[arises] sabbaṁ[all] ārambhapaccayā.[instigate.cause]
\endgl
\nopagebreak
\linespread{0.5}
\begin{spacin}{0.2}
{\PaliGlossFT 744. “Whatever suffering originates  is all conditioned by instigation.}
\end{spacin}
\vskip 12pt
\end{samepage}
\begin{samepage}
\begingl[glneveryline={\PaliGlossA,\PaliGlossB}]
ārambhānaṁ[-] nirodhena,[cessation] natthi[not.is] dukkhassa[suffering] sambhavo.[origin]
\endgl
\nopagebreak
\linespread{0.5}
\begin{spacin}{0.2}
{\PaliGlossFT With the cessation of instigation, there is no origination of suffering.}
\end{spacin}
\vskip 12pt
\end{samepage}
\begin{samepage}
\begingl[glneveryline={\PaliGlossA,\PaliGlossB}]
750.[-] “etamādīnavaṁ[this.disadvantage] ñatvā,[having known] dukkhaṁ[suffering] ārambhapaccayā.[instigate.cause]
\endgl
\nopagebreak
\linespread{0.5}
\begin{spacin}{0.2}
{\PaliGlossFT 745. “Having known this danger, ‘Suffering is conditioned by instigation,’}
\end{spacin}
\vskip 12pt
\end{samepage}
\begin{samepage}
\begingl[glneveryline={\PaliGlossA,\PaliGlossB}]
sabbārambhaṁ[all.instigation] paṭinissajja,[forsakes] anārambhe[non.instigation] vimuttino.[released]
\endgl
\nopagebreak
\linespread{0.5}
\begin{spacin}{0.2}
{\PaliGlossFT having relinquished all instigation, one is liberated in non-instigation.}
\end{spacin}
\vskip 12pt
\end{samepage}
\begin{samepage}
\begingl[glneveryline={\PaliGlossA,\PaliGlossB}]
751.[-] “ucchinnabhavataṇhassa,[destroyed.existence.crave] santacittassa[peace.mind] bhikkhuno.[-]
\endgl
\nopagebreak
\linespread{0.5}
\begin{spacin}{0.2}
{\PaliGlossFT 746. “A bhikkhu with a peaceful mind, who has cut off the craving for existence,}
\end{spacin}
\vskip 12pt
\end{samepage}
\begin{samepage}
\begingl[glneveryline={\PaliGlossA,\PaliGlossB}]
vikkhīṇo[totally destroyed] jātisaṁsāro,[birth.wandering] natthi[not.is] tassa[that] punabbhavo”ti.[re-becoming]
\endgl
\nopagebreak
\linespread{0.5}
\begin{spacin}{0.2}
{\PaliGlossFT has finished with the wandering on in births; for him there is no renewed existence.}
\end{spacin}
\vskip 12pt
\end{samepage}
\vskip 0.2in
\begin{samepage}
\begingl[glneveryline={\PaliGlossA,\PaliGlossB}]
(11)[-]
\endgl
\nopagebreak
\linespread{0.5}
\begin{spacin}{0.2}
{\PaliGlossFT [11. Nutriment]}
\end{spacin}
\vskip 12pt
\end{samepage}
\begin{samepage}
\begingl[glneveryline={\PaliGlossA,\PaliGlossB}]
“‘siyā[could.be] aññenapi[another] pariyāyena[method] sammā[rightly] dvayatānupassanā’ti,[pair.contemplate] iti[-] ce,[-] bhikkhave,[-] pucchitāro[asked] assu;[to be]
\endgl
\nopagebreak
\linespread{0.5}
\begin{spacin}{0.2}
{\PaliGlossFT “If, bhikkhus, there are those who ask, ‘Could there be correct contemplation of dyads in some other way?’}
\end{spacin}
\vskip 12pt
\end{samepage}
\begin{samepage}
\begingl[glneveryline={\PaliGlossA,\PaliGlossB}]
‘siyā’tissu[could.be] vacanīyā.[utterance]
\endgl
\nopagebreak
\linespread{0.5}
\begin{spacin}{0.2}
{\PaliGlossFT you should answer them thus: ‘There could be.’}
\end{spacin}
\vskip 12pt
\end{samepage}
\begin{samepage}
\begingl[glneveryline={\PaliGlossA,\PaliGlossB}]
kathañca[how] siyā?[could.be]
\endgl
\nopagebreak
\linespread{0.5}
\begin{spacin}{0.2}
{\PaliGlossFT And how could there be?}
\end{spacin}
\vskip 12pt
\end{samepage}
\begin{samepage}
\begingl[glneveryline={\PaliGlossA,\PaliGlossB}]
yaṁ[-] kiñci[-] dukkhaṁ[suffering] sambhoti[arises] sabbaṁ[all] āhārapaccayāti,[nutriment.cause] ayamekānupassanā.[this.one.contemplate]
\endgl
\nopagebreak
\linespread{0.5}
\begin{spacin}{0.2}
{\PaliGlossFT ‘Whatever suffering originates is all conditioned by nutriment’ — this is one contemplation.}
\end{spacin}
\vskip 12pt
\end{samepage}
\begin{samepage}
\begingl[glneveryline={\PaliGlossA,\PaliGlossB}]
āhārānaṁ[nutriment] tveva[that] asesavirāganirodhā[entire.dispation.cease] natthi[not.is] dukkhassa[suffering] sambhavoti,[origin] ayaṁ[this] dutiyānupassanā.[second.contemplate]
\endgl
\nopagebreak
\linespread{0.5}
\begin{spacin}{0.2}
{\PaliGlossFT ‘With the remainderless fading away and cessation of nutriment, there is no origination of suffering’ — this is a second contemplation.}
\end{spacin}
\vskip 12pt
\end{samepage}
\begin{samepage}
\begingl[glneveryline={\PaliGlossA,\PaliGlossB}]
evaṁ[thus] sammā[rightly] dvayatānupassino[pair.contemplate] kho,[-] bhikkhave,[-] bhikkhuno[-] appamattassa[vigilant] ātāpino[ardent] pahitattassa[resolute] viharato[abides] dvinnaṁ[pair] phalānaṁ[fruit] aññataraṁ[certain] phalaṁ[fruit] pāṭikaṅkhaṁ[expected] —[-] diṭṭheva[vision] dhamme[the Norm] aññā,[other] sati[state] vā[-] upādisese[some fuel left] anāgāmitā”ti.[not return]
\endgl
\nopagebreak
\linespread{0.5}
\begin{spacin}{0.2}
{\PaliGlossFT When a bhikkhu dwells thus correctly contemplating a dyad — heedful, ardent, and resolute — one of two fruits is to be expected of him: either final knowledge in this very life or, if there is a residue remaining, the state of non-returning.”}
\end{spacin}
\vskip 12pt
\end{samepage}
\begin{samepage}
\begingl[glneveryline={\PaliGlossA,\PaliGlossB}]
idamavoca[this said] bhagavā.[fortunate]
\endgl
\nopagebreak
\linespread{0.5}
\begin{spacin}{0.2}
{\PaliGlossFT This is what the Blessed One said.}
\end{spacin}
\vskip 12pt
\end{samepage}
\begin{samepage}
\begingl[glneveryline={\PaliGlossA,\PaliGlossB}]
athāparaṁ[then also] etadavoca[he said] satthā[taught] —[-]
\endgl
\nopagebreak
\linespread{0.5}
\begin{spacin}{0.2}
{\PaliGlossFT the Teacher further said this:}
\end{spacin}
\vskip 12pt
\end{samepage}
\vskip 0.2in
\begin{samepage}
\begingl[glneveryline={\PaliGlossA,\PaliGlossB}]
752.[-] “yaṁ[-] kiñci[-] dukkhaṁ[suffering] sambhoti,[arises] sabbaṁ[all] āhārapaccayā.[nutriment.cause]
\endgl
\nopagebreak
\linespread{0.5}
\begin{spacin}{0.2}
{\PaliGlossFT 747. “Whatever suffering originates  is all conditioned by nutriment.}
\end{spacin}
\vskip 12pt
\end{samepage}
\begin{samepage}
\begingl[glneveryline={\PaliGlossA,\PaliGlossB}]
āhārānaṁ[nutriment] nirodhena,[cessation] natthi[not.is] dukkhassa[suffering] sambhavo.[origin]
\endgl
\nopagebreak
\linespread{0.5}
\begin{spacin}{0.2}
{\PaliGlossFT With the cessation of nutriment, there is no origination of suffering.}
\end{spacin}
\vskip 12pt
\end{samepage}
\begin{samepage}
\begingl[glneveryline={\PaliGlossA,\PaliGlossB}]
753.[-] “etamādīnavaṁ[this.disadvantage] ñatvā,[having known] dukkhaṁ[suffering] āhārapaccayā.[nutriment.cause]
\endgl
\nopagebreak
\linespread{0.5}
\begin{spacin}{0.2}
{\PaliGlossFT 748. “Having known this danger, ‘Suffering is conditioned by nutriment,’}
\end{spacin}
\vskip 12pt
\end{samepage}
\begin{samepage}
\begingl[glneveryline={\PaliGlossA,\PaliGlossB}]
sabbāhāraṁ[all.nutriment] pariññāya,[having known] sabbāhāramanissito.[all.nutriment.unattached]
\endgl
\nopagebreak
\linespread{0.5}
\begin{spacin}{0.2}
{\PaliGlossFT having fully understood all nutriment, one is not attached to any nutriment.}
\end{spacin}
\vskip 12pt
\end{samepage}
\begin{samepage}
\begingl[glneveryline={\PaliGlossA,\PaliGlossB}]
754.[-] “ārogyaṁ[health] sammadaññāya,[understood perfectly] āsavānaṁ[effluent] parikkhayā.[exhaustion]
\endgl
\nopagebreak
\linespread{0.5}
\begin{spacin}{0.2}
{\PaliGlossFT 749. “Having correctly understood the state of health  through the utter destruction of the influxes,}
\end{spacin}
\vskip 12pt
\end{samepage}
\begin{samepage}
\begingl[glneveryline={\PaliGlossA,\PaliGlossB}]
saṅkhāya[have considered] sevī[practised] dhammaṭṭho,[righteous] saṅkhyaṁ[defined] nopeti[not] vedagū”ti.[attained highest knowledge]
\endgl
\nopagebreak
\linespread{0.5}
\begin{spacin}{0.2}
{\PaliGlossFT using with reflection, firm in the Dhamma, a master of knowledge cannot be designated.}
\end{spacin}
\vskip 12pt
\end{samepage}
\vskip 0.2in
\begin{samepage}
\begingl[glneveryline={\PaliGlossA,\PaliGlossB}]
(12)[-]
\endgl
\nopagebreak
\linespread{0.5}
\begin{spacin}{0.2}
{\PaliGlossFT [12. Agitation]}
\end{spacin}
\vskip 12pt
\end{samepage}
\begin{samepage}
\begingl[glneveryline={\PaliGlossA,\PaliGlossB}]
“‘siyā[could.be] aññenapi[another] pariyāyena[method] sammā[rightly] dvayatānupassanā’ti,[pair.contemplate] iti[-] ce,[-] bhikkhave,[-] pucchitāro[asked] assu;[to be]
\endgl
\nopagebreak
\linespread{0.5}
\begin{spacin}{0.2}
{\PaliGlossFT “If, bhikkhus, there are those who ask, ‘Could there be correct contemplation of dyads in some other way?’}
\end{spacin}
\vskip 12pt
\end{samepage}
\begin{samepage}
\begingl[glneveryline={\PaliGlossA,\PaliGlossB}]
‘siyā’tissu[could.be] vacanīyā.[utterance]
\endgl
\nopagebreak
\linespread{0.5}
\begin{spacin}{0.2}
{\PaliGlossFT you should answer them thus: ‘There could be.’}
\end{spacin}
\vskip 12pt
\end{samepage}
\begin{samepage}
\begingl[glneveryline={\PaliGlossA,\PaliGlossB}]
kathañca[how] siyā?[could.be]
\endgl
\nopagebreak
\linespread{0.5}
\begin{spacin}{0.2}
{\PaliGlossFT And how could there be?}
\end{spacin}
\vskip 12pt
\end{samepage}
\begin{samepage}
\begingl[glneveryline={\PaliGlossA,\PaliGlossB}]
yaṁ[-] kiñci[-] dukkhaṁ[suffering] sambhoti[arises] sabbaṁ[all] iñjitapaccayāti,[agitation.cause] ayamekānupassanā.[this.one.contemplate]
\endgl
\nopagebreak
\linespread{0.5}
\begin{spacin}{0.2}
{\PaliGlossFT ‘Whatever suffering originates is all conditioned by agitation’ — this is one contemplation.}
\end{spacin}
\vskip 12pt
\end{samepage}
\begin{samepage}
\begingl[glneveryline={\PaliGlossA,\PaliGlossB}]
iñjitānaṁ[shaken] tveva[that] asesavirāganirodhā[entire.dispation.cease] natthi[not.is] dukkhassa[suffering] sambhavoti,[origin] ayaṁ[this] dutiyānupassanā.[second.contemplate]
\endgl
\nopagebreak
\linespread{0.5}
\begin{spacin}{0.2}
{\PaliGlossFT ‘With the remainderless fading away and cessation of agitation, there is no origination of suffering’ — this is a second contemplation.}
\end{spacin}
\vskip 12pt
\end{samepage}
\begin{samepage}
\begingl[glneveryline={\PaliGlossA,\PaliGlossB}]
evaṁ[thus] sammā[rightly] dvayatānupassino[pair.contemplate] kho,[-] bhikkhave,[-] bhikkhuno[-] appamattassa[vigilant] ātāpino[ardent] pahitattassa[resolute] viharato[abides] dvinnaṁ[pair] phalānaṁ[fruit] aññataraṁ[certain] phalaṁ[fruit] pāṭikaṅkhaṁ[expected] —[-] diṭṭheva[vision] dhamme[the Norm] aññā,[other] sati[state] vā[-] upādisese[some fuel left] anāgāmitā”ti.[not return]
\endgl
\nopagebreak
\linespread{0.5}
\begin{spacin}{0.2}
{\PaliGlossFT When a bhikkhu dwells thus correctly contemplating a dyad — heedful, ardent, and resolute — one of two fruits is to be expected of him: either final knowledge in this very life or, if there is a residue remaining, the state of non-returning.”}
\end{spacin}
\vskip 12pt
\end{samepage}
\begin{samepage}
\begingl[glneveryline={\PaliGlossA,\PaliGlossB}]
idamavoca[this said] bhagavā.[fortunate]
\endgl
\nopagebreak
\linespread{0.5}
\begin{spacin}{0.2}
{\PaliGlossFT This is what the Blessed One said.}
\end{spacin}
\vskip 12pt
\end{samepage}
\begin{samepage}
\begingl[glneveryline={\PaliGlossA,\PaliGlossB}]
athāparaṁ[then also] etadavoca[he said] satthā[taught] —[-]
\endgl
\nopagebreak
\linespread{0.5}
\begin{spacin}{0.2}
{\PaliGlossFT the Teacher further said this:}
\end{spacin}
\vskip 12pt
\end{samepage}
\vskip 0.2in
\begin{samepage}
\begingl[glneveryline={\PaliGlossA,\PaliGlossB}]
755.[-] “yaṁ[-] kiñci[-] dukkhaṁ[suffering] sambhoti,[arises] sabbaṁ[all] iñjitapaccayā.[agitation.cause]
\endgl
\nopagebreak
\linespread{0.5}
\begin{spacin}{0.2}
{\PaliGlossFT 750. “Whatever suffering originates  is all conditioned by agitation.}
\end{spacin}
\vskip 12pt
\end{samepage}
\begin{samepage}
\begingl[glneveryline={\PaliGlossA,\PaliGlossB}]
iñjitānaṁ[shaken] nirodhena,[cessation] natthi[not.is] dukkhassa[suffering] sambhavo.[origin]
\endgl
\nopagebreak
\linespread{0.5}
\begin{spacin}{0.2}
{\PaliGlossFT With the cessation of agitation, there is no origination of suffering.}
\end{spacin}
\vskip 12pt
\end{samepage}
\begin{samepage}
\begingl[glneveryline={\PaliGlossA,\PaliGlossB}]
756.[-] “etamādīnavaṁ[this.disadvantage] ñatvā,[having known] dukkhaṁ[suffering] iñjitapaccayā.[agitation.cause]
\endgl
\nopagebreak
\linespread{0.5}
\begin{spacin}{0.2}
{\PaliGlossFT 751. “Having known this danger, ‘Suffering is conditioned by agitation,’}
\end{spacin}
\vskip 12pt
\end{samepage}
\begin{samepage}
\begingl[glneveryline={\PaliGlossA,\PaliGlossB}]
tasmā[therefore] hi[because] ejaṁ[agitation] vossajja,[given up] saṅkhāre[formations] uparundhiya.[kept in check]
\endgl
\nopagebreak
\linespread{0.5}
\begin{spacin}{0.2}
{\PaliGlossFT therefore having given up impulse, having put a stop to volitional activities,}
\end{spacin}
\vskip 12pt
\end{samepage}
\begin{samepage}
\begingl[glneveryline={\PaliGlossA,\PaliGlossB}]
anejo[free from lust] anupādāno,[unattached] sato[mindful] bhikkhu[-] paribbaje”ti.[wonder about]
\endgl
\nopagebreak
\linespread{0.5}
\begin{spacin}{0.2}
{\PaliGlossFT without impulse, without clinging, a bhikkhu should wander mindfully.}
\end{spacin}
\vskip 12pt
\end{samepage}
\vskip 0.2in
\begin{samepage}
\begingl[glneveryline={\PaliGlossA,\PaliGlossB}]
(13)[-]
\endgl
\nopagebreak
\linespread{0.5}
\begin{spacin}{0.2}
{\PaliGlossFT [13. Dependency]}
\end{spacin}
\vskip 12pt
\end{samepage}
\begin{samepage}
\begingl[glneveryline={\PaliGlossA,\PaliGlossB}]
“‘siyā[could.be] aññenapi[another] pariyāyena[method] sammā[rightly] dvayatānupassanā’ti,[pair.contemplate] iti[-] ce,[-] bhikkhave,[-] pucchitāro[asked] assu;[to be]
\endgl
\nopagebreak
\linespread{0.5}
\begin{spacin}{0.2}
{\PaliGlossFT “If, bhikkhus, there are those who ask, ‘Could there be correct contemplation of dyads in some other way?’}
\end{spacin}
\vskip 12pt
\end{samepage}
\begin{samepage}
\begingl[glneveryline={\PaliGlossA,\PaliGlossB}]
‘siyā’tissu[could.be] vacanīyā.[utterance]
\endgl
\nopagebreak
\linespread{0.5}
\begin{spacin}{0.2}
{\PaliGlossFT you should answer them thus: ‘There could be.’}
\end{spacin}
\vskip 12pt
\end{samepage}
\begin{samepage}
\begingl[glneveryline={\PaliGlossA,\PaliGlossB}]
kathañca[how] siyā?[could.be]
\endgl
\nopagebreak
\linespread{0.5}
\begin{spacin}{0.2}
{\PaliGlossFT And how could there be?}
\end{spacin}
\vskip 12pt
\end{samepage}
\begin{samepage}
\begingl[glneveryline={\PaliGlossA,\PaliGlossB}]
nissitassa[dependent on] calitaṁ[agitated] hotīti,[exists] ayamekānupassanā.[this.one.contemplate]
\endgl
\nopagebreak
\linespread{0.5}
\begin{spacin}{0.2}
{\PaliGlossFT ‘For one who is dependent there is quaking’ — this is one contemplation.}
\end{spacin}
\vskip 12pt
\end{samepage}
\begin{samepage}
\begingl[glneveryline={\PaliGlossA,\PaliGlossB}]
anissito[independent] na[not] calatīti,[agitated] ayaṁ[this] dutiyānupassanā.[second.contemplate]
\endgl
\nopagebreak
\linespread{0.5}
\begin{spacin}{0.2}
{\PaliGlossFT ‘One who is independent does not quake’ — this is a second contemplation.}
\end{spacin}
\vskip 12pt
\end{samepage}
\begin{samepage}
\begingl[glneveryline={\PaliGlossA,\PaliGlossB}]
evaṁ[thus] sammā[rightly] dvayatānupassino[pair.contemplate] kho,[-] bhikkhave,[-] bhikkhuno[-] appamattassa[vigilant] ātāpino[ardent] pahitattassa[resolute] viharato[abides] dvinnaṁ[pair] phalānaṁ[fruit] aññataraṁ[certain] phalaṁ[fruit] pāṭikaṅkhaṁ[expected] —[-] diṭṭheva[vision] dhamme[the Norm] aññā,[other] sati[state] vā[-] upādisese[some fuel left] anāgāmitā”ti.[not return]
\endgl
\nopagebreak
\linespread{0.5}
\begin{spacin}{0.2}
{\PaliGlossFT When a bhikkhu dwells thus correctly contemplating a dyad — heedful, ardent, and resolute — one of two fruits is to be expected of him: either final knowledge in this very life or, if there is a residue remaining, the state of non-returning.”}
\end{spacin}
\vskip 12pt
\end{samepage}
\begin{samepage}
\begingl[glneveryline={\PaliGlossA,\PaliGlossB}]
idamavoca[this said] bhagavā.[fortunate]
\endgl
\nopagebreak
\linespread{0.5}
\begin{spacin}{0.2}
{\PaliGlossFT This is what the Blessed One said.}
\end{spacin}
\vskip 12pt
\end{samepage}
\begin{samepage}
\begingl[glneveryline={\PaliGlossA,\PaliGlossB}]
athāparaṁ[then also] etadavoca[he said] satthā[taught] —[-]
\endgl
\nopagebreak
\linespread{0.5}
\begin{spacin}{0.2}
{\PaliGlossFT the Teacher further said this:}
\end{spacin}
\vskip 12pt
\end{samepage}
\vskip 0.2in
\begin{samepage}
\begingl[glneveryline={\PaliGlossA,\PaliGlossB}]
757.[-] “anissito[independent] na[not] calati,[agitated] nissito[dependent] ca[-] upādiyaṁ.[grasping]
\endgl
\nopagebreak
\linespread{0.5}
\begin{spacin}{0.2}
{\PaliGlossFT 752. “One who is independent does not quake, but one who is dependent, clinging [to things],}
\end{spacin}
\vskip 12pt
\end{samepage}
\begin{samepage}
\begingl[glneveryline={\PaliGlossA,\PaliGlossB}]
itthabhāvaññathābhāvaṁ,[thus.become.not.thus.become] saṁsāraṁ[wondering on] nātivattati.[not.go beyond]
\endgl
\nopagebreak
\linespread{0.5}
\begin{spacin}{0.2}
{\PaliGlossFT does not transcend saṃsāra,  with its becoming thus, becoming otherwise.}
\end{spacin}
\vskip 12pt
\end{samepage}
\begin{samepage}
\begingl[glneveryline={\PaliGlossA,\PaliGlossB}]
758.[-] “etamādīnavaṁ[this.disadvantage] ñatvā,[having known] nissayesu[support] mahabbhayaṁ.[great fear]
\endgl
\nopagebreak
\linespread{0.5}
\begin{spacin}{0.2}
{\PaliGlossFT 753. “Having known this danger, ‘There is great peril in dependencies,’}
\end{spacin}
\vskip 12pt
\end{samepage}
\begin{samepage}
\begingl[glneveryline={\PaliGlossA,\PaliGlossB}]
anissito[independent] anupādāno,[unattached] sato[mindful] bhikkhu[-] paribbaje”ti.[wonder about]
\endgl
\nopagebreak
\linespread{0.5}
\begin{spacin}{0.2}
{\PaliGlossFT independent, without clinging, a bhikkhu should wander mindfully.}
\end{spacin}
\vskip 12pt
\end{samepage}
\vskip 0.2in
\begin{samepage}
\begingl[glneveryline={\PaliGlossA,\PaliGlossB}]
(14)[-]
\endgl
\nopagebreak
\linespread{0.5}
\begin{spacin}{0.2}
{\PaliGlossFT [14. Form and formless states]}
\end{spacin}
\vskip 12pt
\end{samepage}
\begin{samepage}
\begingl[glneveryline={\PaliGlossA,\PaliGlossB}]
“‘siyā[could.be] aññenapi[another] pariyāyena[method] sammā[rightly] dvayatānupassanā’ti,[pair.contemplate] iti[-] ce,[-] bhikkhave,[-] pucchitāro[asked] assu;[to be]
\endgl
\nopagebreak
\linespread{0.5}
\begin{spacin}{0.2}
{\PaliGlossFT “If, bhikkhus, there are those who ask, ‘Could there be correct contemplation of dyads in some other way?’}
\end{spacin}
\vskip 12pt
\end{samepage}
\begin{samepage}
\begingl[glneveryline={\PaliGlossA,\PaliGlossB}]
‘siyā’tissu[could.be] vacanīyā.[utterance]
\endgl
\nopagebreak
\linespread{0.5}
\begin{spacin}{0.2}
{\PaliGlossFT you should answer them thus: ‘There could be.’}
\end{spacin}
\vskip 12pt
\end{samepage}
\begin{samepage}
\begingl[glneveryline={\PaliGlossA,\PaliGlossB}]
kathañca[how] siyā?[could.be]
\endgl
\nopagebreak
\linespread{0.5}
\begin{spacin}{0.2}
{\PaliGlossFT And how could there be?}
\end{spacin}
\vskip 12pt
\end{samepage}
\begin{samepage}
\begingl[glneveryline={\PaliGlossA,\PaliGlossB}]
rūpehi,[form] bhikkhave,[-] arūpā[formless] santatarāti,[more peaceful] ayamekānupassanā.[this.one.contemplate]
\endgl
\nopagebreak
\linespread{0.5}
\begin{spacin}{0.2}
{\PaliGlossFT ‘Formless states are more peaceful than states of form’ — this is one contemplation.}
\end{spacin}
\vskip 12pt
\end{samepage}
\begin{samepage}
\begingl[glneveryline={\PaliGlossA,\PaliGlossB}]
arūpehi[formless] nirodho[cessation] santataroti,[more peaceful] ayaṁ[this] dutiyānupassanā.[second.contemplate]
\endgl
\nopagebreak
\linespread{0.5}
\begin{spacin}{0.2}
{\PaliGlossFT [147] ‘Cessation is more peaceful than formless states’ — this is a second contemplation.}
\end{spacin}
\vskip 12pt
\end{samepage}
\begin{samepage}
\begingl[glneveryline={\PaliGlossA,\PaliGlossB}]
evaṁ[thus] sammā[rightly] dvayatānupassino[pair.contemplate] kho,[-] bhikkhave,[-] bhikkhuno[-] appamattassa[vigilant] ātāpino[ardent] pahitattassa[resolute] viharato[abides] dvinnaṁ[pair] phalānaṁ[fruit] aññataraṁ[certain] phalaṁ[fruit] pāṭikaṅkhaṁ[expected] —[-] diṭṭheva[vision] dhamme[the Norm] aññā,[other] sati[state] vā[-] upādisese[some fuel left] anāgāmitā”ti.[not return]
\endgl
\nopagebreak
\linespread{0.5}
\begin{spacin}{0.2}
{\PaliGlossFT When a bhikkhu dwells thus correctly contemplating a dyad — heedful, ardent, and resolute — one of two fruits is to be expected of him: either final knowledge in this very life or, if there is a residue remaining, the state of non-returning.”}
\end{spacin}
\vskip 12pt
\end{samepage}
\begin{samepage}
\begingl[glneveryline={\PaliGlossA,\PaliGlossB}]
idamavoca[this said] bhagavā.[fortunate]
\endgl
\nopagebreak
\linespread{0.5}
\begin{spacin}{0.2}
{\PaliGlossFT This is what the Blessed One said.}
\end{spacin}
\vskip 12pt
\end{samepage}
\begin{samepage}
\begingl[glneveryline={\PaliGlossA,\PaliGlossB}]
athāparaṁ[then also] etadavoca[he said] satthā[taught] —[-]
\endgl
\nopagebreak
\linespread{0.5}
\begin{spacin}{0.2}
{\PaliGlossFT the Teacher further said this:}
\end{spacin}
\vskip 12pt
\end{samepage}
\vskip 0.2in
\begin{samepage}
\begingl[glneveryline={\PaliGlossA,\PaliGlossB}]
759.[-] “ye[-] ca[-] rūpūpagā[form.come into] sattā,[being] ye[-] ca[-] arūpaṭṭhāyino.[formless.state]
\endgl
\nopagebreak
\linespread{0.5}
\begin{spacin}{0.2}
{\PaliGlossFT 754. “Those beings who fare on to form  and those who dwell in the formless,}
\end{spacin}
\vskip 12pt
\end{samepage}
\begin{samepage}
\begingl[glneveryline={\PaliGlossA,\PaliGlossB}]
nirodhaṁ[cessation] appajānantā,[not.understand] āgantāro[one coming] punabbhavaṁ.[new existence]
\endgl
\nopagebreak
\linespread{0.5}
\begin{spacin}{0.2}
{\PaliGlossFT not understanding cessation, come back to renewed existence.}
\end{spacin}
\vskip 12pt
\end{samepage}
\begin{samepage}
\begingl[glneveryline={\PaliGlossA,\PaliGlossB}]
760.[-] “ye[-] ca[-] rūpe[form] pariññāya,[having known] arūpesu[formless] asaṇṭhitā.[unsettled]
\endgl
\nopagebreak
\linespread{0.5}
\begin{spacin}{0.2}
{\PaliGlossFT 755. “But those who have fully understood forms, without settling down in formless states,}
\end{spacin}
\vskip 12pt
\end{samepage}
\begin{samepage}
\begingl[glneveryline={\PaliGlossA,\PaliGlossB}]
nirodhe[cessation] ye[-] vimuccanti,[release] te[-] janā[people] maccuhāyino”ti.[victorious over death]
\endgl
\nopagebreak
\linespread{0.5}
\begin{spacin}{0.2}
{\PaliGlossFT who are liberated in cessation: those people have abandoned death.}
\end{spacin}
\vskip 12pt
\end{samepage}
\vskip 0.2in
\begin{samepage}
\begingl[glneveryline={\PaliGlossA,\PaliGlossB}]
(15)[-]
\endgl
\nopagebreak
\linespread{0.5}
\begin{spacin}{0.2}
{\PaliGlossFT [15. Truth and falsity]}
\end{spacin}
\vskip 12pt
\end{samepage}
\begin{samepage}
\begingl[glneveryline={\PaliGlossA,\PaliGlossB}]
“‘siyā[could.be] aññenapi[another] pariyāyena[method] sammā[rightly] dvayatānupassanā’ti,[pair.contemplate] iti[-] ce,[-] bhikkhave,[-] pucchitāro[asked] assu;[to be]
\endgl
\nopagebreak
\linespread{0.5}
\begin{spacin}{0.2}
{\PaliGlossFT “If, bhikkhus, there are those who ask, ‘Could there be correct contemplation of dyads in some other way?’}
\end{spacin}
\vskip 12pt
\end{samepage}
\begin{samepage}
\begingl[glneveryline={\PaliGlossA,\PaliGlossB}]
‘siyā’tissu[could.be] vacanīyā.[utterance]
\endgl
\nopagebreak
\linespread{0.5}
\begin{spacin}{0.2}
{\PaliGlossFT you should answer them thus: ‘There could be.’}
\end{spacin}
\vskip 12pt
\end{samepage}
\begin{samepage}
\begingl[glneveryline={\PaliGlossA,\PaliGlossB}]
kathañca[how] siyā?[could.be]
\endgl
\nopagebreak
\linespread{0.5}
\begin{spacin}{0.2}
{\PaliGlossFT And how could there be?}
\end{spacin}
\vskip 12pt
\end{samepage}
\begin{samepage}
\begingl[glneveryline={\PaliGlossA,\PaliGlossB}]
yaṁ,[-] bhikkhave,[-] sadevakassa[with devas] lokassa[the world] samārakassa[with Māra] sabrahmakassa[with Brahma] sassamaṇabrāhmaṇiyā[with samana and brahman] pajāya[produced] sadevamanussāya[with.god.human] idaṁ[-] saccanti[truth] upanijjhāyitaṁ[considered] tadamariyānaṁ[-] etaṁ[that] musāti[false] yathābhūtaṁ[as.become] sammappaññāya[properly] sudiṭṭhaṁ,[well seen] ayamekānupassanā.[this.one.contemplate]
\endgl
\nopagebreak
\linespread{0.5}
\begin{spacin}{0.2}
{\PaliGlossFT ‘In this world, bhikkhus, with its devas, Māra, and Brahmā, among this population with its ascetics and brahmins, its devas and humans, that which is regarded as “This is true,” the noble ones have seen it well with correct wisdom thus: “This is false” ’ — this is one contemplation.}
\end{spacin}
\vskip 12pt
\end{samepage}
\begin{samepage}
\begingl[glneveryline={\PaliGlossA,\PaliGlossB}]
yaṁ,[-] bhikkhave,[-] sadevakassa[with devas] lokassa[the world] samārakassa[with Māra] sabrahmakassa[with Brahma] sassamaṇabrāhmaṇiyā[with samana and brahman] pajāya[produced] sadevamanussāya[with.god.human] idaṁ[-] musāti[false] upanijjhāyitaṁ,[considered] tadamariyānaṁ[-] etaṁ[that] saccanti[truth] yathābhūtaṁ[as.become] sammappaññāya[properly] sudiṭṭhaṁ,[well seen] ayaṁ[this] dutiyānupassanā.[second.contemplate]
\endgl
\nopagebreak
\linespread{0.5}
\begin{spacin}{0.2}
{\PaliGlossFT In this world, bhikkhus, with its devas, Māra, and Brahmā, among this population with its ascetics and brahmins, its devas and humans, that which is regarded as “This is false,” the noble ones have seen it well with correct wisdom thus: “This is true”’ — this is a second contemplation.}
\end{spacin}
\vskip 12pt
\end{samepage}
\begin{samepage}
\begingl[glneveryline={\PaliGlossA,\PaliGlossB}]
evaṁ[thus] sammā[rightly] dvayatānupassino[pair.contemplate] kho,[-] bhikkhave,[-] bhikkhuno[-] appamattassa[vigilant] ātāpino[ardent] pahitattassa[resolute] viharato[abides] dvinnaṁ[pair] phalānaṁ[fruit] aññataraṁ[certain] phalaṁ[fruit] pāṭikaṅkhaṁ[expected] —[-] diṭṭheva[vision] dhamme[the Norm] aññā,[other] sati[state] vā[-] upādisese[some fuel left] anāgāmitā”ti.[not return]
\endgl
\nopagebreak
\linespread{0.5}
\begin{spacin}{0.2}
{\PaliGlossFT When a bhikkhu dwells thus correctly contemplating a dyad — heedful, ardent, and resolute — one of two fruits is to be expected of him: either final knowledge in this very life or, if there is a residue remaining, the state of non-returning.”}
\end{spacin}
\vskip 12pt
\end{samepage}
\begin{samepage}
\begingl[glneveryline={\PaliGlossA,\PaliGlossB}]
idamavoca[this said] bhagavā.[fortunate]
\endgl
\nopagebreak
\linespread{0.5}
\begin{spacin}{0.2}
{\PaliGlossFT This is what the Blessed One said.}
\end{spacin}
\vskip 12pt
\end{samepage}
\begin{samepage}
\begingl[glneveryline={\PaliGlossA,\PaliGlossB}]
athāparaṁ[then also] etadavoca[he said] satthā[taught] —[-]
\endgl
\nopagebreak
\linespread{0.5}
\begin{spacin}{0.2}
{\PaliGlossFT the Teacher further said this:}
\end{spacin}
\vskip 12pt
\end{samepage}
\vskip 0.2in
\begin{samepage}
\begingl[glneveryline={\PaliGlossA,\PaliGlossB}]
761.[-] “anattani[not.self] attamāniṁ,[self.conceive] passa[sees] lokaṁ[the world] sadevakaṁ.[with.devas]
\endgl
\nopagebreak
\linespread{0.5}
\begin{spacin}{0.2}
{\PaliGlossFT 756. “Behold the world together with its devas conceiving a self in what is non-self.}
\end{spacin}
\vskip 12pt
\end{samepage}
\begin{samepage}
\begingl[glneveryline={\PaliGlossA,\PaliGlossB}]
niviṭṭhaṁ[established in] nāmarūpasmiṁ,[name and form] idaṁ[-] saccanti[truth] maññati.[imagines]
\endgl
\nopagebreak
\linespread{0.5}
\begin{spacin}{0.2}
{\PaliGlossFT Settled upon name-and-form, they conceive: ‘This is true.’}
\end{spacin}
\vskip 12pt
\end{samepage}
\begin{samepage}
\begingl[glneveryline={\PaliGlossA,\PaliGlossB}]
762.[-] “yena[because of] yena[because of] hi[because] maññanti,[imagine] tato[from there] taṁ[that] hoti[to be] aññathā.[otherwise]
\endgl
\nopagebreak
\linespread{0.5}
\begin{spacin}{0.2}
{\PaliGlossFT 757. “In whatever way they conceive it, it turns out otherwise.}
\end{spacin}
\vskip 12pt
\end{samepage}
\begin{samepage}
\begingl[glneveryline={\PaliGlossA,\PaliGlossB}]
tañhi[that] tassa[that] musā[false] hoti,[to be] mosadhammañhi[false.nature] ittaraṁ.[short-lived]
\endgl
\nopagebreak
\linespread{0.5}
\begin{spacin}{0.2}
{\PaliGlossFT That indeed is its falsity, for the transient is of a false nature.}
\end{spacin}
\vskip 12pt
\end{samepage}
\begin{samepage}
\begingl[glneveryline={\PaliGlossA,\PaliGlossB}]
763.[-] “amosadhammaṁ[not.false.nature] nibbānaṁ,[nibbana] tadariyā[that nobel one] saccato[truth] vidū.[wise]
\endgl
\nopagebreak
\linespread{0.5}
\begin{spacin}{0.2}
{\PaliGlossFT 758. “Nibbāna is of a non-false nature:  that the noble ones know as truth.}
\end{spacin}
\vskip 12pt
\end{samepage}
\begin{samepage}
\begingl[glneveryline={\PaliGlossA,\PaliGlossB}]
te[-] ve[indeed] saccābhisamayā,[comprehension of the reality] nicchātā[no hunger] parinibbutā”ti.[final.peace]
\endgl
\nopagebreak
\linespread{0.5}
\begin{spacin}{0.2}
{\PaliGlossFT Through the breakthrough to truth, hungerless, they are fully quenched.}
\end{spacin}
\vskip 12pt
\end{samepage}
\vskip 0.2in
\begin{samepage}
\begingl[glneveryline={\PaliGlossA,\PaliGlossB}]
(16)[-]
\endgl
\nopagebreak
\linespread{0.5}
\begin{spacin}{0.2}
{\PaliGlossFT [16. Happiness and suffering]}
\end{spacin}
\vskip 12pt
\end{samepage}
\begin{samepage}
\begingl[glneveryline={\PaliGlossA,\PaliGlossB}]
“‘siyā[could.be] aññenapi[another] pariyāyena[method] sammā[rightly] dvayatānupassanā’ti,[pair.contemplate] iti[-] ce,[-] bhikkhave,[-] pucchitāro[asked] assu;[to be]
\endgl
\nopagebreak
\linespread{0.5}
\begin{spacin}{0.2}
{\PaliGlossFT “If, bhikkhus, there are those who ask, ‘Could there be correct contemplation of dyads in some other way?’}
\end{spacin}
\vskip 12pt
\end{samepage}
\begin{samepage}
\begingl[glneveryline={\PaliGlossA,\PaliGlossB}]
‘siyā’tissu[could.be] vacanīyā.[utterance]
\endgl
\nopagebreak
\linespread{0.5}
\begin{spacin}{0.2}
{\PaliGlossFT you should answer them thus: ‘There could be.’}
\end{spacin}
\vskip 12pt
\end{samepage}
\begin{samepage}
\begingl[glneveryline={\PaliGlossA,\PaliGlossB}]
kathañca[how] siyā?[could.be]
\endgl
\nopagebreak
\linespread{0.5}
\begin{spacin}{0.2}
{\PaliGlossFT And how could there be?}
\end{spacin}
\vskip 12pt
\end{samepage}
\begin{samepage}
\begingl[glneveryline={\PaliGlossA,\PaliGlossB}]
yaṁ,[-] bhikkhave,[-] sadevakassa[with devas] lokassa[the world] samārakassa[with Māra] sabrahmakassa[with Brahma] sassamaṇabrāhmaṇiyā[with samana and brahman] pajāya[produced] sadevamanussāya[with.god.human] idaṁ[-] sukhanti[happiness] upanijjhāyitaṁ,[considered] tadamariyānaṁ[-] etaṁ[that] dukkhanti[suffering] yathābhūtaṁ[as.become] sammappaññāya[properly] sudiṭṭhaṁ,[well seen] ayamekānupassanā.[this.one.contemplate]
\endgl
\nopagebreak
\linespread{0.5}
\begin{spacin}{0.2}
{\PaliGlossFT ‘In this world, bhikkhus, with its devas, Māra, and Brahmā, among this population with its ascetics and brahmins, its devas and humans, that which is regarded as “This is happiness,” the noble ones have seen well with correct wisdom thus: “This is suffering” ’ — this is one contemplation.}
\end{spacin}
\vskip 12pt
\end{samepage}
\begin{samepage}
\begingl[glneveryline={\PaliGlossA,\PaliGlossB}]
yaṁ,[-] bhikkhave,[-] sadevakassa[with devas] lokassa[the world] samārakassa[with Māra] sabrahmakassa[with Brahma] sassamaṇabrāhmaṇiyā[with samana and brahman] pajāya[produced] sadevamanussāya[with.god.human] idaṁ[-] dukkhanti[suffering] upanijjhāyitaṁ[considered] tadamariyānaṁ[-] etaṁ[that] sukhanti[happiness] yathābhūtaṁ[as.become] sammappaññāya[properly] sudiṭṭhaṁ,[well seen] ayaṁ[this] dutiyānupassanā.[second.contemplate]
\endgl
\nopagebreak
\linespread{0.5}
\begin{spacin}{0.2}
{\PaliGlossFT In this world, bhikkhus, with its devas, Māra, and Brahmā, among this population with its ascetics and brahmins, its devas and humans, that which is regarded as “This is suffering,” the noble ones have seen well with correct wisdom thus, “This is happiness” ’ — this is a second contemplation.}
\end{spacin}
\vskip 12pt
\end{samepage}
\begin{samepage}
\begingl[glneveryline={\PaliGlossA,\PaliGlossB}]
evaṁ[thus] sammā[rightly] dvayatānupassino[pair.contemplate] kho,[-] bhikkhave,[-] bhikkhuno[-] appamattassa[vigilant] ātāpino[ardent] pahitattassa[resolute] viharato[abides] dvinnaṁ[pair] phalānaṁ[fruit] aññataraṁ[certain] phalaṁ[fruit] pāṭikaṅkhaṁ[expected] —[-] diṭṭheva[vision] dhamme[the Norm] aññā,[other] sati[state] vā[-] upādisese[some fuel left] anāgāmitāti.[one not return]
\endgl
\nopagebreak
\linespread{0.5}
\begin{spacin}{0.2}
{\PaliGlossFT When a bhikkhu dwells thus correctly contemplating a dyad — heedful, ardent, and resolute — one of two fruits is to be expected of him: either final knowledge in this very life, or, if there is a residue of clinging, the state of non-returning.”}
\end{spacin}
\vskip 12pt
\end{samepage}
\begin{samepage}
\begingl[glneveryline={\PaliGlossA,\PaliGlossB}]
idamavoca[this said] bhagavā.[fortunate]
\endgl
\nopagebreak
\linespread{0.5}
\begin{spacin}{0.2}
{\PaliGlossFT This is what the Blessed One said.}
\end{spacin}
\vskip 12pt
\end{samepage}
\begin{samepage}
\begingl[glneveryline={\PaliGlossA,\PaliGlossB}]
idaṁ[-] vatvāna[having said] sugato[faring well] athāparaṁ[then also] etadavoca[he said] satthā[taught] —[-]
\endgl
\nopagebreak
\linespread{0.5}
\begin{spacin}{0.2}
{\PaliGlossFT Having said this, the Fortunate One, the Teacher, further said this:}
\end{spacin}
\vskip 12pt
\end{samepage}
\vskip 0.2in
\begin{samepage}
\begingl[glneveryline={\PaliGlossA,\PaliGlossB}]
764.[-] “rūpā[form] saddā[sound] rasā[taste] gandhā,[odour] phassā[touch] dhammā[doctrine] ca[-] kevalā.[entire]
\endgl
\nopagebreak
\linespread{0.5}
\begin{spacin}{0.2}
{\PaliGlossFT 759. “Forms, sounds, tastes, odors, textures, and objects of mind —}
\end{spacin}
\vskip 12pt
\end{samepage}
\begin{samepage}
\begingl[glneveryline={\PaliGlossA,\PaliGlossB}]
iṭṭhā[agreeable] kantā[desireable?] manāpā[pleasing] ca,[-] yāvatatthīti[as far up to] vuccati.[called]
\endgl
\nopagebreak
\linespread{0.5}
\begin{spacin}{0.2}
{\PaliGlossFT all are desirable, lovely, agreeable, so long as it is said: ‘They are.’}
\end{spacin}
\vskip 12pt
\end{samepage}
\begin{samepage}
\begingl[glneveryline={\PaliGlossA,\PaliGlossB}]
765.[-] “sadevakassa[with devas] lokassa,[the world] ete[-] vo[-] sukhasammatā.[deemed pleasure]
\endgl
\nopagebreak
\linespread{0.5}
\begin{spacin}{0.2}
{\PaliGlossFT 760. “These are considered as happiness in the world with its devas;}
\end{spacin}
\vskip 12pt
\end{samepage}
\begin{samepage}
\begingl[glneveryline={\PaliGlossA,\PaliGlossB}]
yattha[-] cete[-] nirujjhanti,[cease] taṁ[that] nesaṁ[leads] dukkhasammataṁ.[suffering]
\endgl
\nopagebreak
\linespread{0.5}
\begin{spacin}{0.2}
{\PaliGlossFT but where these cease,  that they consider suffering.}
\end{spacin}
\vskip 12pt
\end{samepage}
\begin{samepage}
\begingl[glneveryline={\PaliGlossA,\PaliGlossB}]
766.[-] “sukhanti[happiness] diṭṭhamariyehi,[seen.deathless] sakkāyassuparodhanaṁ.[existing body.breakup]
\endgl
\nopagebreak
\linespread{0.5}
\begin{spacin}{0.2}
{\PaliGlossFT 761. “The noble ones have seen as happiness the ceasing of the personal entity.}
\end{spacin}
\vskip 12pt
\end{samepage}
\begin{samepage}
\begingl[glneveryline={\PaliGlossA,\PaliGlossB}]
paccanīkamidaṁ[undergo.counter] hoti,[to be] sabbalokena[whole.world] passataṁ.[see]
\endgl
\nopagebreak
\linespread{0.5}
\begin{spacin}{0.2}
{\PaliGlossFT Running counter to the entire world  is this [insight] of those who see.}
\end{spacin}
\vskip 12pt
\end{samepage}
\begin{samepage}
\begingl[glneveryline={\PaliGlossA,\PaliGlossB}]
767.[-] “yaṁ[-] pare[-] sukhato[happiness] āhu,[speak] tadariyā[that nobel one] āhu[speak] dukkhato.[suffering]
\endgl
\nopagebreak
\linespread{0.5}
\begin{spacin}{0.2}
{\PaliGlossFT 762. “What others speak of as happiness,  that the noble ones speak of as suffering.}
\end{spacin}
\vskip 12pt
\end{samepage}
\begin{samepage}
\begingl[glneveryline={\PaliGlossA,\PaliGlossB}]
yaṁ[-] pare[-] dukkhato[suffering] āhu,[speak] tadariyā[that nobel one] sukhato[happiness] vidū.[wise]
\endgl
\nopagebreak
\linespread{0.5}
\begin{spacin}{0.2}
{\PaliGlossFT What others speak of as suffering,  that the noble ones have known as happiness.}
\end{spacin}
\vskip 12pt
\end{samepage}
\begin{samepage}
\begingl[glneveryline={\PaliGlossA,\PaliGlossB}]
768.[-] “passa[sees] dhammaṁ[doctrine] durājānaṁ,[difficult.know] sampamūḷhetthaviddasu.[confound.fool]
\endgl
\nopagebreak
\linespread{0.5}
\begin{spacin}{0.2}
{\PaliGlossFT Behold this Dhamma hard to comprehend: here the foolish are bewildered.}
\end{spacin}
\vskip 12pt
\end{samepage}
\begin{samepage}
\begingl[glneveryline={\PaliGlossA,\PaliGlossB}]
nivutānaṁ[surrounded] tamo[complete] hoti,[to be] andhakāro[darkness] apassataṁ.[not.see]
\endgl
\nopagebreak
\linespread{0.5}
\begin{spacin}{0.2}
{\PaliGlossFT 763. “There is gloom for those who are blocked, darkness for those who do not see,}
\end{spacin}
\vskip 12pt
\end{samepage}
\begin{samepage}
\begingl[glneveryline={\PaliGlossA,\PaliGlossB}]
769.[-] “satañca[mindful] vivaṭaṁ[opened] hoti,[to be] āloko[light] passatāmiva.[sees]
\endgl
\nopagebreak
\linespread{0.5}
\begin{spacin}{0.2}
{\PaliGlossFT but for the good it is opened up like light for those who see.}
\end{spacin}
\vskip 12pt
\end{samepage}
\begin{samepage}
\begingl[glneveryline={\PaliGlossA,\PaliGlossB}]
santike[near] na[not] vijānanti,[knowledge] maggā[way] dhammassa[the Norm] kovidā.[clever]
\endgl
\nopagebreak
\linespread{0.5}
\begin{spacin}{0.2}
{\PaliGlossFT The brutes unskilled in the Dhamma do not understand it even when close.}
\end{spacin}
\vskip 12pt
\end{samepage}
\begin{samepage}
\begingl[glneveryline={\PaliGlossA,\PaliGlossB}]
770.[-] “bhavarāgaparetehi,[existence.lust.afflicted] bhavasotānusāribhi.[existence.stream.striving]
\endgl
\nopagebreak
\linespread{0.5}
\begin{spacin}{0.2}
{\PaliGlossFT 764. “This Dhamma is not easily understood by those afflicted by lust for existence,}
\end{spacin}
\vskip 12pt
\end{samepage}
\begin{samepage}
\begingl[glneveryline={\PaliGlossA,\PaliGlossB}]
māradheyyānupannehi,[mara realm.gone into] nāyaṁ[carried away] dhammo[doctrine] susambudho.[-]
\endgl
\nopagebreak
\linespread{0.5}
\begin{spacin}{0.2}
{\PaliGlossFT by those flowing in the stream of existence, deeply mired in Māra’s realm.}
\end{spacin}
\vskip 12pt
\end{samepage}
\begin{samepage}
\begingl[glneveryline={\PaliGlossA,\PaliGlossB}]
771.[-] “ko[-] nu[-] aññatramariyehi,[other.deathless] padaṁ[-] sambuddhumarahati.[fully englighted one]
\endgl
\nopagebreak
\linespread{0.5}
\begin{spacin}{0.2}
{\PaliGlossFT 765. “Who else apart from the noble ones are able to understand this state?}
\end{spacin}
\vskip 12pt
\end{samepage}
\begin{samepage}
\begingl[glneveryline={\PaliGlossA,\PaliGlossB}]
yaṁ[-] padaṁ[-] sammadaññāya,[understood perfectly] parinibbanti[final nibbana] anāsavā”ti.[free from influx]
\endgl
\nopagebreak
\linespread{0.5}
\begin{spacin}{0.2}
{\PaliGlossFT When they have correctly known that state, those without influxes attain nibbāna.”}
\end{spacin}
\vskip 12pt
\end{samepage}
\vskip 0.2in
\begin{samepage}
\begingl[glneveryline={\PaliGlossA,\PaliGlossB}]
idamavoca[this said] bhagavā.[fortunate]
\endgl
\nopagebreak
\linespread{0.5}
\begin{spacin}{0.2}
{\PaliGlossFT This is what the Blessed One said.}
\end{spacin}
\vskip 12pt
\end{samepage}
\begin{samepage}
\begingl[glneveryline={\PaliGlossA,\PaliGlossB}]
attamanā[delighted] te[-] bhikkhū[-] bhagavato[the blessed] bhāsitaṁ[said] abhinandunti.[rejoices at]
\endgl
\nopagebreak
\linespread{0.5}
\begin{spacin}{0.2}
{\PaliGlossFT Elated, those bhikkhus delighted in the Blessed One’s statement.}
\end{spacin}
\vskip 12pt
\end{samepage}
\begin{samepage}
\begingl[glneveryline={\PaliGlossA,\PaliGlossB}]
imasmiṁ[-] ca[-] pana[-] veyyākaraṇasmiṁ[discourse] bhaññamāne[spoken] saṭṭhimattānaṁ[sixty] bhikkhūnaṁ[-] anupādāya[without grasping] āsavehi[influx] cittāni[mind] vimucciṁsūti.[released]
\endgl
\nopagebreak
\linespread{0.5}
\begin{spacin}{0.2}
{\PaliGlossFT And while this discourse was being spoken, the minds of sixty bhikkhus were liberated from the influxes by non-clinging.}
\end{spacin}
\vskip 12pt
\end{samepage}
\vskip 0.2in
\begin{samepage}
\begingl[glneveryline={\PaliGlossA,\PaliGlossB}]
dvayatānupassanāsuttaṁ[-] dvādasamaṁ[-] niṭṭhitaṁ.[-]
\endgl
\nopagebreak
\linespread{0.5}
\begin{spacin}{0.2}
{\PaliGlossFT Contemplation of pairs Twelve Complete}
\end{spacin}
\vskip 12pt
\end{samepage}