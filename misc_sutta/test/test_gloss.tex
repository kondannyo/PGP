
\begin{samepage}
\begin{leftcolumn*}
\EnglishColumn{3. puttamaṃsūpamasuttaṃ (SN 12.63)}
\hspace{0pt}\end{leftcolumn*}

\begin{rightcolumn}\PaliColumn63 (3) Son’s Flesh}
\hspace{0pt}
\end{rightcolumn}
\end{samepage}
\vskip 0.2in
\begin{samepage}
\begin{leftcolumn*}
\EnglishColumn{63. sāvatthiyaṃ ...  pe ...}
\hspace{0pt}\end{leftcolumn*}

\begin{rightcolumn}\PaliColumnAt Sāvatthı̄.}
\hspace{0pt}
\end{rightcolumn}
\end{samepage}
\begin{samepage}
\begin{leftcolumn*}
\EnglishColumn{“cattārome, bhikkhave, āhārā bhūtānaṃ vā sattānaṃ ṭhitiyā sambhavesīnaṃ vā anuggahāya.}
\hspace{0pt}\end{leftcolumn*}

\begin{rightcolumn}\PaliColumn“Bhikkhus, there are these four kinds of nutriment for the maintenance of beings that have already come to be and for the assistance of those about to come to be.}
\hspace{0pt}
\end{rightcolumn}
\end{samepage}
\begin{samepage}
\begin{leftcolumn*}
\EnglishColumn{katame cattāro?}
\hspace{0pt}\end{leftcolumn*}

\begin{rightcolumn}\PaliColumnWhat four?}
\hspace{0pt}
\end{rightcolumn}
\end{samepage}
\begin{samepage}
\begin{leftcolumn*}
\EnglishColumn{kabaḷīkāro āhāro oḷāriko vā sukhumo vā, phasso dutiyo, manosañcetanā tatiyā, viññāṇaṃ catutthaṃ.}
\hspace{0pt}\end{leftcolumn*}

\begin{rightcolumn}\PaliColumnThe nutriment edible food, gross or subtle; second, contact; third, mental volition; fourth, consciousness.}
\hspace{0pt}
\end{rightcolumn}
\end{samepage}
\begin{samepage}
\begin{leftcolumn*}
\EnglishColumn{ime kho, bhikkhave, cattāro āhārā bhūtānaṃ vā sattānaṃ ṭhitiyā sambhavesīnaṃ vā anuggahāya”.}
\hspace{0pt}\end{leftcolumn*}

\begin{rightcolumn}\PaliColumnThese are the four kinds of nutriment for the maintenance of beings that have already come to be and for the assistance of those about to come to be.}
\hspace{0pt}
\end{rightcolumn}
\end{samepage}
\vskip 0.2in
\begin{samepage}
\begin{leftcolumn*}
\EnglishColumn{“kathañca, bhikkhave, kabaḷīkāro āhāro daṭṭhabbo?}
\hspace{0pt}\end{leftcolumn*}

\begin{rightcolumn}\PaliColumn“And how, bhikkhus, should the nutriment edible food be seen?}
\hspace{0pt}
\end{rightcolumn}
\end{samepage}
\begin{samepage}
\begin{leftcolumn*}
\EnglishColumn{seyyathāpi, bhikkhave, dve jāyampatikā parittaṃ sambalaṃ ādāya kantāramaggaṃ paṭipajjeyyuṃ.}
\hspace{0pt}\end{leftcolumn*}

\begin{rightcolumn}\PaliColumnSuppose a couple, husband and wife, had taken limited provisions and were travelling through a desert.}
\hspace{0pt}
\end{rightcolumn}
\end{samepage}
\begin{samepage}
\begin{leftcolumn*}
\EnglishColumn{tesamassa ekaputtako piyo manāpo.}
\hspace{0pt}\end{leftcolumn*}

\begin{rightcolumn}\PaliColumnThey have with them their only son, dear and beloved.}
\hspace{0pt}
\end{rightcolumn}
\end{samepage}
\begin{samepage}
\begin{leftcolumn*}
\EnglishColumn{atha kho tesaṃ, bhikkhave, dvinnaṃ jāyampatikānaṃ kantāragatānaṃ yā parittā sambalamattā, sā parikkhayaṃ pariyādānaṃ gaccheyya.  siyā ca nesaṃ kantārāvaseso anatiṇṇo.}
\hspace{0pt}\end{leftcolumn*}

\begin{rightcolumn}\PaliColumnThen, in the middle of the desert, their limited provisions would be used up and exhausted, while the rest of the desert remains to be crossed.}
\hspace{0pt}
\end{rightcolumn}
\end{samepage}
\begin{samepage}
\begin{leftcolumn*}
\EnglishColumn{atha kho tesaṃ, bhikkhave, dvinnaṃ jāyampatikānaṃ evamassa — ‘amhākaṃ kho yā parittā sambalamattā sā parikkhīṇā pariyādiṇṇā.  atthi cāyaṃ kantārāvaseso anittiṇṇo.}
\hspace{0pt}\end{leftcolumn*}

\begin{rightcolumn}\PaliColumnThe husband and wife would think: ‘Our limited provisions have been used up and exhausted, while the rest of this desert remains to be crossed.}
\hspace{0pt}
\end{rightcolumn}
\end{samepage}
\begin{samepage}
\begin{leftcolumn*}
\EnglishColumn{yaṃnūna mayaṃ imaṃ ekaputtakaṃ piyaṃ manāpaṃ vadhitvā vallūrañca soṇḍikañca karitvā puttamaṃsāni khādantā evaṃ taṃ kantārāvasesaṃ nitthareyyāma, mā sabbeva tayo vinassimhā’ti.}
\hspace{0pt}\end{leftcolumn*}

\begin{rightcolumn}\PaliColumnLet us kill our only son, dear and beloved, and prepare dried and spiced meat.  By eating our son’s flesh we can cross the rest of this desert.  Let not all three of us perish!’}
\hspace{0pt}
\end{rightcolumn}
\end{samepage}
\begin{samepage}
\begin{leftcolumn*}
\EnglishColumn{atha kho te, bhikkhave, dve jāyampatikā taṃ ekaputtakaṃ piyaṃ manāpaṃ vadhitvā vallūrañca soṇḍikañca karitvā puttamaṃsāni khādantā evaṃ taṃ kantārāvasesaṃ nitthareyyuṃ.}
\hspace{0pt}\end{leftcolumn*}

\begin{rightcolumn}\PaliColumn“Then, bhikkhus, the husband and wife would kill their only son, dear and beloved, prepare dried and spiced meat, and by eating their son’s flesh they would cross the rest of the desert.}
\hspace{0pt}
\end{rightcolumn}
\end{samepage}
\begin{samepage}
\begin{leftcolumn*}
\EnglishColumn{te puttamaṃsāni ceva khādeyyuṃ, ure ca paṭipiseyyuṃ — ‘kahaṃ, ekaputtaka, kahaṃ, ekaputtakā’ti.}
\hspace{0pt}\end{leftcolumn*}

\begin{rightcolumn}\PaliColumnWhile they are eating their son’s flesh, they would beat their breasts and cry: ‘Where are you, our only son?  Where are you, our only son?’}
\hspace{0pt}
\end{rightcolumn}
\end{samepage}
\vskip 0.2in
\begin{samepage}
\begin{leftcolumn*}
\EnglishColumn{viññāṇe ce, bhikkhave, āhāre natthi rāgo natthi nandī natthi taṇhā, appatiṭṭhitaṃ tattha viññāṇaṃ avirūḷhaṃ.}
\hspace{0pt}\end{leftcolumn*}

\begin{rightcolumn}\PaliColumn“If, bhikkhus, there is no lust for the nutriment consciousness, if there is no delight, if there is no craving, consciousness does not become established there and come to growth.}
\hspace{0pt}
\end{rightcolumn}
\end{samepage}
\begin{samepage}
\begin{leftcolumn*}
\EnglishColumn{yattha appatiṭṭhitaṃ viññāṇaṃ avirūḷhaṃ, natthi tattha nāmarūpassa avakkanti.}
\hspace{0pt}\end{leftcolumn*}

\begin{rightcolumn}\PaliColumnWhere consciousness does not become established and come to growth, there is no descent of name-and-form.}
\hspace{0pt}
\end{rightcolumn}
\end{samepage}
\begin{samepage}
\begin{leftcolumn*}
\EnglishColumn{yattha natthi nāmarūpassa avakkanti, natthi tattha saṅkhārānaṃ vuddhi.}
\hspace{0pt}\end{leftcolumn*}

\begin{rightcolumn}\PaliColumnWhere there is no descent of name-and-form, there is no growth of volitional formations.}
\hspace{0pt}
\end{rightcolumn}
\end{samepage}
\begin{samepage}
\begin{leftcolumn*}
\EnglishColumn{yattha natthi saṅkhārānaṃ vuddhi, natthi tattha āyatiṃ punabbhavābhinibbatti.}
\hspace{0pt}\end{leftcolumn*}

\begin{rightcolumn}\PaliColumnWhere there is no growth of volitional formations, there is no production of future renewed existence.}
\hspace{0pt}
\end{rightcolumn}
\end{samepage}
\begin{samepage}
\begin{leftcolumn*}
\EnglishColumn{yattha natthi āyatiṃ punabbhavābhinibbatti, natthi tattha āyatiṃ jātijarāmaraṇaṃ.}
\hspace{0pt}\end{leftcolumn*}

\begin{rightcolumn}\PaliColumnWhere there is no production of future renewed existence, there is no future birth, aging, and death.}
\hspace{0pt}
\end{rightcolumn}
\end{samepage}
\begin{samepage}
\begin{leftcolumn*}
\EnglishColumn{yattha natthi āyatiṃ jātijarāmaraṇaṃ, asokaṃ taṃ, bhikkhave, adaraṃ anupāyāsanti vadāmi.”ti.}
\hspace{0pt}\end{leftcolumn*}

\begin{rightcolumn}\PaliColumnWhere there is no future birth, aging, and death, I say that is without sorrow, anguish, and despair.}
\hspace{0pt}
\end{rightcolumn}
\end{samepage}
\vskip 0.2in
\begin{samepage}
\begin{leftcolumn*}
\EnglishColumn{catutthaṃ.}
\hspace{0pt}\end{leftcolumn*}

\begin{rightcolumn}\PaliColumnFourth (64)}
\hspace{0pt}
\end{rightcolumn}
\end{samepage}