\documentclass[11pt]{article}
\usepackage[margin=1in]{geometry}
\usepackage{titlesec}
\usepackage{polyglossia}
\defaultfontfeatures{Ligatures=TeX}
\setmainfont{Arial}
\usepackage{expex}
\raggedright
\sloppy

% Uncomment the following line if you do not want extra space before the free translation and the glosses, or other customisations to the examples.
\lingset{aboveglftskip=-.2ex,interpartskip=\baselineskip,everyglb=\footnotesize}

\title{Some \texttt{expex} Linguistic Examples}
\author{LianTze Lim}

\begin{document}
\textbf{Majjhima Nikāya 149
The Great Sixfold Base}\\[.3cm]

Thus have I heard. On one occasion the Blessed One was living at Sāvatthī in Jeta’s Grove, Anāthapiṇḍika’s Park. There he addressed the bhikkhus thus: “Bhikkhus.”—“Venerable sir,” they replied. The Blessed One said this:\\

“Bhikkhus, I shall teach you a discourse on the great sixfold base. Listen and attend closely to what I shall say.”—“Yes, venerable sir,” the bhikkhus replied. The Blessed One said this:\\

“Bhikkhus, when one does not know and see the eye as it actually is, when one does not know and see forms as they actually are, when one does not know and see eye-consciousness as it actually is, when one does not know and see eye-contact as it actually is, when one does not know and see as it actually is the feeling felt as pleasant or painful or neither-painful-nor-pleasant that arises with eye-contact as condition, then one is inflamed by lust for the eye, for forms, for eye-consciousness, for eye-contact, for the feeling felt as pleasant or painful or neither-painful-nor-pleasant that arises with eye-contact as condition.\\

“When one abides inflamed by lust, fettered, infatuated, contemplating gratification, then the five aggregates affected by clinging are built up for oneself in the future; and one’s craving—which brings renewal of being, is accompanied by delight and lust, and delights in this and that—increases. One’s bodily and mental troubles increase, one’s bodily and mental torments increase, one’s bodily and mental fevers increase, and one experiences bodily and mental suffering.\\

“When one does not know and see the ear as it actually is…When one does not know and see the nose as it actually is…When one does not know and see the tongue as it actually is…When one does not know and see the body as it actually is… When one does not know and see the mind as it actually is…one experiences bodily and mental suffering.\\

“Bhikkhus, when one knows and sees the eye as it actually is, when one knows and sees forms as they actually are, when one knows and sees eye-consciousness as it actually is, when one knows and sees eye-contact as it actually is, when one knows and sees as it actually is the feeling felt as pleasant or painful or neither-painful-nor-pleasant that arises with eye-contact as condition, then one is not inflamed by lust for the eye, for forms, for eye-consciousness, for eye-contact, for the feeling felt as pleasant or painful or neither-painful-nor-pleasant that arises with eye-contact as condition.\\

“When one abides uninflamed by lust, unfettered, uninfatuated, contemplating danger, then the five aggregates affected by clinging are diminished for oneself in the future; and one’s craving—which brings renewal of being, is accompanied by delight and lust, and delights in this or that—is abandoned. One’s bodily and mental troubles are abandoned, one’s bodily and mental torments are abandoned, one’s bodily and mental fevers are abandoned, and one experiences bodily and mental pleasure.\\

“The view of a person such as this is right view. His intention is right intention, his effort is right effort, his mindfulness is right mindfulness, his concentration is right concentration. But his bodily action, his verbal action, and his livelihood have already been well purified earlier. Thus this Noble Eightfold Path comes to fulfilment in him by development. When he develops this Noble Eightfold Path, the four foundations of mindfulness also come to fulfilment in him by development; the four right kinds of striving also come to fulfilment in him by development; the four bases for spiritual power also come to fulfilment in him by development; the five faculties also come to fulfilment in him by development; the five powers also come to fulfilment in him by development; the seven enlightenment factors also come to fulfilment in him by development. These two things—serenity and insight—occur in him yoked evenly together. He fully understands by direct knowledge those things that should be fully understood by direct knowledge. He abandons by direct knowledge those things that should be abandoned by direct knowledge. He develops by direct knowledge those things that should be developed by direct knowledge. He realises by direct knowledge those things that should be realised by direct knowledge.\\

“And what things should be fully understood by direct knowledge? The answer to that is: the five aggregates affected by clinging, that is, the material form aggregate affected by clinging, the feeling aggregate affected by clinging, the perception aggregate affected by clinging, the formations aggregate affected by clinging, the consciousness aggregate affected by clinging. These are the things that should be fully understood by direct knowledge.\\

“And what things should be abandoned by direct knowledge? Ignorance and craving for being. These are the things that should be abandoned by direct knowledge.\\

“And what things should be developed by direct knowledge? Serenity and insight. These are the things that should be developed by direct knowledge.\

“And what things should be realised by direct knowledge? True knowledge and deliverance. These are the things that should be realised by direct knowledge.\\
“When one knows and sees the ear as it actually is… These are the things that should be realised by direct knowledge.\\

“When one knows and sees the nose as it actually is… These are the things that should be realised by direct knowledge.\\

“When one knows and sees the tongue as it actually is...These are the things that should be realised by direct knowledge.\\

“When one knows and sees the body as it actually is… These are the things that should be realised by direct knowledge.\\

“When one knows and sees the mind as it actually is… These are the things that should be realised by direct knowledge.”\\

That is what the Blessed One said. The bhikkhus were satisfied and delighted in the Blessed One’s words.\\


\end{document}