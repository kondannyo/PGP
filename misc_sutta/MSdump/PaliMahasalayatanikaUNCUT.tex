@ pali-ms-01.tex
Majjhima Nikāya, uparipaṇṇāsapāḷi, 5. saḷāyatanavaggo n n, 7.
mahāsaḷāyatanikasuttaṃ

@ pali-ms-02.tex
♦ 428. evaṃ me sutaṃ — ekaṃ samayaṃ bhagavā sāvatthiyaṃ viharati jetavane anāthapiṇḍikassa ārāme. tatra kho bhagavā bhikkhū āmantesi — “bhikkhavo”ti. “bhadante”ti te bhikkhū bhagavato paccassosuṃ. bhagavā etadavoca —

@ pali-ms-03.tex
“mahāsaḷāyatanikaṃ vo, bhikkhave, desessāmi. taṃ suṇātha, sādhukaṃ manasi karotha; bhāsissāmī”ti. “evaṃ, bhante”ti kho te bhikkhū bhagavato paccassosuṃ. bhagavā etadavoca —

@ pali-ms-04.tex
♦ 429. “cakkhuṃ, bhikkhave, ajānaṃ apassaṃ yathābhūtaṃ,

@ pali-ms-04a.tex
rūpe ajānaṃ apassaṃ yathābhūtaṃ,

@ pali-ms-04b.tex
cakkhuviññāṇaṃ ajānaṃ apassaṃ yathābhūtaṃ,

@ pali-ms-04c.tex
cakkhusamphassaṃ ajānaṃ apassaṃ yathābhūtaṃ,

@ pali-ms-04d.tex
yamidaṃ cakkhusamphassapaccayā uppajjati vedayitaṃ sukhaṃ vā dukkhaṃ vā adukkhamasukhaṃ vā tampi ajānaṃ apassaṃ yathābhūtaṃ,

@ pali-ms-04e.tex
cakkhusmiṃ sārajjati, rūpesu sārajjati, cakkhuviññāṇe sārajjati, cakkhusamphasse sārajjati, yamidaṃ cakkhusamphassapaccayā uppajjati vedayitaṃ sukhaṃ vā dukkhaṃ vā adukkhamasukhaṃ vā tasmimpi sārajjati.

@ pali-ms-05.tex
♦ “tassa sārattassa saṃyuttassa sammūḷhassa assādānupassino viharato āyatiṃ pañcupādānakkhandhā upacayaṃ gacchanti. taṇhā cassa ponobbhavikā nandīrāgasahagatā tatratatrābhinandinī, sā cassa pavaḍḍhati. tassa kāyikāpi darathā pavaḍḍhanti, cetasikāpi darathā pavaḍḍhanti; kāyikāpi santāpā pavaḍḍhanti, cetasikāpi santāpā pavaḍḍhanti; kāyikāpi pariḷāhā pavaḍḍhanti, cetasikāpi pariḷāhā pavaḍḍhanti. so kāyadukkhampi cetodukkhampi paṭisaṃvedeti.

@ pali-ms-06.tex
♦ “sotaṃ, bhikkhave, ajānaṃ apassaṃ yathābhūtaṃ, sadde ajānaṃ apassaṃ yathābhūtaṃ, sotaviññāṇaṃ ajānaṃ apassaṃ yathābhūtaṃ, sotasamphassaṃ ajānaṃ apassaṃ yathābhūtaṃ, yamidaṃ sotasamphassapaccayā uppajjati vedayitaṃ sukhaṃ vā dukkhaṃ vā adukkhamasukhaṃ vā tampi ajānaṃ apassaṃ yathābhūtaṃ, sotasmiṃ sārajjati, saddesu sārajjati, sotaviññāṇe sārajjati, sotasamphasse sārajjati, yamidaṃ sotasamphassapaccayā uppajjati vedayitaṃ sukhaṃ vā dukkhaṃ vā adukkhamasukhaṃ vā tasmimpi sārajjati.

@ pali-ms-07.tex
♦ “tassa sārattassa saṃyuttassa sammūḷhassa assādānupassino viharato āyatiṃ pañcupādānakkhandhā upacayaṃ gacchanti. taṇhā cassa ponobbhavikā nandīrāgasahagatā tatratatrābhinandinī, sā cassa pavaḍḍhati. tassa kāyikāpi darathā pavaḍḍhanti, cetasikāpi darathā pavaḍḍhanti; kāyikāpi santāpā pavaḍḍhanti, cetasikāpi santāpā pavaḍḍhanti; kāyikāpi pariḷāhā pavaḍḍhanti, cetasikāpi pariḷāhā pavaḍḍhanti. so kāyadukkhampi cetodukkhampi paṭisaṃvedeti.

@ pali-ms-08.tex
ghānaṃ, bhikkhave, ajānaṃ apassaṃ yathābhūtaṃ, gandhe ajānaṃ apassaṃ yathābhūtaṃ, ghānaviññāṇaṃ ajānaṃ apassaṃ yathābhūtaṃ, ghānasamphassaṃ ajānaṃ apassaṃ yathābhūtaṃ, yamidaṃ ghānasamphassapaccayā uppajjati vedayitaṃ sukhaṃ vā dukkhaṃ vā adukkhamasukhaṃ vā tampi ajānaṃ apassaṃ yathābhūtaṃ, ghānasmiṃ sārajjati, gandhesu sārajjati, ghānaviññāṇe sārajjati, ghānasamphasse sārajjati, yamidaṃ ghānasamphassapaccayā uppajjati vedayitaṃ sukhaṃ vā dukkhaṃ vā adukkhamasukhaṃ vā tasmimpi sārajjati.

@ pali-ms-09.tex
♦ “tassa sārattassa saṃyuttassa sammūḷhassa assādānupassino viharato āyatiṃ pañcupādānakkhandhā upacayaṃ gacchanti. taṇhā cassa ponobbhavikā nandīrāgasahagatā tatratatrābhinandinī, sā cassa pavaḍḍhati. tassa kāyikāpi darathā pavaḍḍhanti, cetasikāpi darathā pavaḍḍhanti; kāyikāpi santāpā pavaḍḍhanti, cetasikāpi santāpā pavaḍḍhanti; kāyikāpi pariḷāhā pavaḍḍhanti, cetasikāpi pariḷāhā pavaḍḍhanti. so kāyadukkhampi cetodukkhampi paṭisaṃvedeti.

@ pali-ms-10.tex
jivhaṃ, bhikkhave, ajānaṃ apassaṃ yathābhūtaṃ, rase ajānaṃ apassaṃ yathābhūtaṃ, jivhaviññāṇaṃ ajānaṃ apassaṃ yathābhūtaṃ, jivhasamphassaṃ ajānaṃ apassaṃ yathābhūtaṃ, yamidaṃ jivhasamphassapaccayā uppajjati vedayitaṃ sukhaṃ vā dukkhaṃ vā adukkhamasukhaṃ vā tampi ajānaṃ apassaṃ yathābhūtaṃ, jivhasmiṃ sārajjati, rasesu sārajjati, jivhaviññāṇe sārajjati, jivhasamphasse sārajjati, yamidaṃ jivhasamphassapaccayā uppajjati vedayitaṃ sukhaṃ vā dukkhaṃ vā adukkhamasukhaṃ vā tasmimpi sārajjati.

@ pali-ms-11.tex
♦ “tassa sārattassa saṃyuttassa sammūḷhassa assādānupassino viharato āyatiṃ pañcupādānakkhandhā upacayaṃ gacchanti. taṇhā cassa ponobbhavikā nandīrāgasahagatā tatratatrābhinandinī, sā cassa pavaḍḍhati. tassa kāyikāpi darathā pavaḍḍhanti, cetasikāpi darathā pavaḍḍhanti; kāyikāpi santāpā pavaḍḍhanti, cetasikāpi santāpā pavaḍḍhanti; kāyikāpi pariḷāhā pavaḍḍhanti, cetasikāpi pariḷāhā pavaḍḍhanti. so kāyadukkhampi cetodukkhampi paṭisaṃvedeti.

@ pali-ms-12.tex
kāyaṃ, bhikkhave, ajānaṃ apassaṃ yathābhūtaṃ, phoṭṭhabbe ajānaṃ apassaṃ yathābhūtaṃ, kāyaviññāṇaṃ ajānaṃ apassaṃ yathābhūtaṃ, kāyasamphassaṃ ajānaṃ apassaṃ yathābhūtaṃ, yamidaṃ kāyasamphassapaccayā uppajjati vedayitaṃ sukhaṃ vā dukkhaṃ vā adukkhamasukhaṃ vā tampi ajānaṃ apassaṃ yathābhūtaṃ, kāyasmiṃ sārajjati, phoṭṭhabbesu sārajjati, kāyaviññāṇe sārajjati, kāyasamphasse sārajjati, yamidaṃ kāyasamphassapaccayā uppajjati vedayitaṃ sukhaṃ vā dukkhaṃ vā adukkhamasukhaṃ vā tasmimpi sārajjati.

@ pali-ms-13.tex
♦ “tassa sārattassa saṃyuttassa sammūḷhassa assādānupassino viharato āyatiṃ pañcupādānakkhandhā upacayaṃ gacchanti. taṇhā cassa ponobbhavikā nandīrāgasahagatā tatratatrābhinandinī, sā cassa pavaḍḍhati. tassa kāyikāpi darathā pavaḍḍhanti, cetasikāpi darathā pavaḍḍhanti; kāyikāpi santāpā pavaḍḍhanti, cetasikāpi santāpā pavaḍḍhanti; kāyikāpi pariḷāhā pavaḍḍhanti, cetasikāpi pariḷāhā pavaḍḍhanti. so kāyadukkhampi cetodukkhampi paṭisaṃvedeti.

@ pali-ms-14.tex
manaṃ, bhikkhave, ajānaṃ apassaṃ yathābhūtaṃ, dhamme, bhikkhave, ajānaṃ apassaṃ yathābhūtaṃ, manoviññāṇaṃ, bhikkhave, ajānaṃ apassaṃ yathābhūtaṃ, manosamphassaṃ, bhikkhave, ajānaṃ apassaṃ yathābhūtaṃ, yamidaṃ manosamphassapaccayā uppajjati vedayitaṃ sukhaṃ vā dukkhaṃ vā adukkhamasukhaṃ vā tampi ajānaṃ apassaṃ yathābhūtaṃ, manasmiṃ sārajjati, dhammesu sārajjati, manoviññāṇe sārajjati, manosamphasse sārajjati, yamidaṃ manosamphassapaccayā uppajjati vedayitaṃ sukhaṃ vā dukkhaṃ vā adukkhamasukhaṃ vā tasmimpi sārajjati.

@ pali-ms-15.tex
♦ “tassa sārattassa saṃyuttassa sammūḷhassa assādānupassino viharato āyatiṃ pañcupādānakkhandhā upacayaṃ gacchanti. taṇhā cassa ponobbhavikā nandīrāgasahagatā tatratatrābhinandinī, sā cassa pavaḍḍhati. tassa kāyikāpi darathā pavaḍḍhanti, cetasikāpi darathā pavaḍḍhanti; kāyikāpi santāpā pavaḍḍhanti, cetasikāpi santāpā pavaḍḍhanti; kāyikāpi pariḷāhā pavaḍḍhanti, cetasikāpi pariḷāhā pavaḍḍhanti. so kāyadukkhampi cetodukkhampi paṭisaṃvedeti.

@ pali-ms-16.tex
♦ 430. “cakkhuñca kho, bhikkhave, jānaṃ passaṃ yathābhūtaṃ, rūpe jānaṃ passaṃ yathābhūtaṃ, cakkhuviññāṇaṃ jānaṃ passaṃ yathābhūtaṃ, cakkhusamphassaṃ jānaṃ passaṃ yathābhūtaṃ, yamidaṃ cakkhusamphassapaccayā uppajjati vedayitaṃ sukhaṃ vā dukkhaṃ vā adukkhamasukhaṃ vā tampi jānaṃ passaṃ yathābhūtaṃ, cakkhusmiṃ na sārajjati, rūpesu na sārajjati, cakkhuviññāṇe na sārajjati, cakkhusamphasse na sārajjati, yamidaṃ cakkhusamphassapaccayā uppajjati vedayitaṃ sukhaṃ vā dukkhaṃ vā adukkhamasukhaṃ vā tasmimpi na sārajjati...

@ pali-ms-17.tex
	♦ 432. “sotaṃ, bhikkhave, jānaṃ passaṃ yathābhūtaṃ, sadde jānaṃ passaṃ yathābhūtaṃ, sotaviññāṇaṃ jānaṃ passaṃ yathābhūtaṃ, sotasamphassaṃ jānaṃ passaṃ yathābhūtaṃ, yamidaṃ sotasamphassapaccayā uppajjati vedayitaṃ sukhaṃ vā dukkhaṃ vā adukkhamasukhaṃ vā tampi jānaṃ passaṃ yathābhūtaṃ, sotasmiṃ na sārajjati, saddesu na sārajjati, sotaviññāṇe na sārajjati, sotasamphasse na sārajjati, yamidaṃ sotasamphassapaccayā uppajjati vedayitaṃ sukhaṃ vā dukkhaṃ vā adukkhamasukhaṃ vā tasmimpi na sārajjati...

@ pali-ms-18.tex
	ghānaṃ bhikkhave, jānaṃ passaṃ yathābhūtaṃ, gandhe jānaṃ passaṃ yathābhūtaṃ, ghānaviññāṇaṃ jānaṃ passaṃ yathābhūtaṃ, ghānasamphassaṃ jānaṃ passaṃ yathābhūtaṃ, yamidaṃ ghānasamphassapaccayā uppajjati vedayitaṃ sukhaṃ vā dukkhaṃ vā adukkhamasukhaṃ vā tampi jānaṃ passaṃ yathābhūtaṃ, ghānasmiṃ na sārajjati, gandhesu na sārajjati, ghānaviññāṇe na sārajjati, ghānasamphasse na sārajjati, yamidaṃ ghānasamphassapaccayā uppajjati vedayitaṃ sukhaṃ vā dukkhaṃ vā adukkhamasukhaṃ vā tasmimpi na sārajjati...

@ pali-ms-19.tex
	jivhaṃ, bhikkhave, jānaṃ passaṃ yathābhūtaṃ, rase jānaṃ passaṃ yathābhūtaṃ, jivhaviññāṇaṃ jānaṃ passaṃ yathābhūtaṃ, jivhasamphassaṃ jānaṃ passaṃ yathābhūtaṃ, yamidaṃ jivhasamphassapaccayā uppajjati vedayitaṃ sukhaṃ vā dukkhaṃ vā adukkhamasukhaṃ vā tampi jānaṃ passaṃ yathābhūtaṃ, jivhasmiṃ na sārajjati, rasesu na sārajjati, jivhaviññāṇe na sārajjati, jivhasamphasse na sārajjati, yamidaṃ jivhasamphassapaccayā uppajjati vedayitaṃ sukhaṃ vā dukkhaṃ vā adukkhamasukhaṃ vā tasmimpi na sārajjati...

@ pali-ms-20.tex
	kāyaṃ, bhikkhave, jānaṃ passaṃ yathābhūtaṃ, phoṭṭhabbe jānaṃ passaṃ yathābhūtaṃ, kāyaviññāṇaṃ jānaṃ passaṃ yathābhūtaṃ, kāyasamphassaṃ jānaṃ passaṃ yathābhūtaṃ, yamidaṃ kāyasamphassapaccayā uppajjati vedayitaṃ sukhaṃ vā dukkhaṃ vā adukkhamasukhaṃ vā tampi jānaṃ passaṃ yathābhūtaṃ, kāyasmiṃ na sārajjati, phoṭṭhabbesu na sārajjati, kāyaviññāṇe na sārajjati, kāyasamphasse na sārajjati, yamidaṃ kāyasamphassapaccayā uppajjati vedayitaṃ sukhaṃ vā dukkhaṃ vā adukkhamasukhaṃ vā tasmimpi na sārajjati...

@ pali-ms-21.tex
	manaṃ, bhikkhave, jānaṃ passaṃ yathābhūtaṃ, dhamme jānaṃ passaṃ yathābhūtaṃ, manoviññāṇaṃ jānaṃ passaṃ yathābhūtaṃ, manosamphassaṃ jānaṃ passaṃ yathābhūtaṃ, yamidaṃ manosamphassapaccayā uppajjati vedayitaṃ sukhaṃ vā dukkhaṃ vā adukkhamasukhaṃ vā tampi jānaṃ passaṃ yathābhūtaṃ, manasmiṃ na sārajjati, dhammesu na sārajjati, manoviññāṇe na sārajjati, manosamphasse na sārajjati, yamidaṃ manosamphassapaccayā uppajjati vedayitaṃ sukhaṃ vā dukkhaṃ vā adukkhamasukhaṃ vā tasmimpi na sārajjati.

@ pali-ms-22.tex
♦ “tassa asārattassa asaṃyuttassa asammūḷhassa ādīnavānupassino viharato āyatiṃ pañcupādānakkhandhā apacayaṃ gacchanti. taṇhā cassa ponobbhavikā nandīrāgasahagatā tatratatrābhinandinī, sā cassa pahīyati. tassa kāyikāpi darathā pahīyanti, cetasikāpi darathā pahīyanti; kāyikāpi santāpā pahīyanti, cetasikāpi santāpā pahīyanti; kāyikāpi pariḷāhā pahīyanti, cetasikāpi pariḷāhā pahīyanti. so kāyasukhampi cetosukhampi paṭisaṃvedeti.

@ pali-ms-23.tex
♦ 433. “yā tathābhūtassa diṭṭhi sāssa hoti sammādiṭṭhi; yo tathābhūtassa saṅkappo svāssa hoti sammāsaṅkappo; yo tathābhūtassa vāyāmo svāssa hoti sammāvāyāmo; yā tathābhūtassa sati sāssa hoti sammāsati; yo tathābhūtassa samādhi svāssa hoti sammāsamādhi. pubbeva kho panassa kāyakammaṃ vacīkammaṃ ājīvo suparisuddho hoti. evamassāyaṃ ariyo aṭṭhaṅgiko maggo bhāvanāpāripūriṃ gacchati.

@ pali-ms-24.tex
♦ “tassa evaṃ imaṃ ariyaṃ aṭṭhaṅgikaṃ maggaṃ bhāvayato cattāropi satipaṭṭhānā bhāvanāpāripūriṃ gacchanti, cattāropi sammappadhānā bhāvanāpāripūriṃ gacchanti, cattāropi iddhipādā bhāvanāpāripūriṃ gacchanti, pañcapi indriyāni bhāvanāpāripūriṃ gacchanti, pañcapi balāni bhāvanāpāripūriṃ gacchanti, sattapi bojjhaṅgā bhāvanāpāripūriṃ gacchanti.

@ pali-ms-25.tex
♦ “tassime dve dhammā yuganandhā vattanti — samatho ca vipassanā ca. so ye dhammā abhiññā pariññeyyā te dhamme abhiññā parijānāti. ye dhammā abhiññā pahātabbā te dhamme abhiññā pajahati. ye dhammā abhiññā bhāvetabbā te dhamme abhiññā bhāveti. ye dhammā abhiññā sacchikātabbā te dhamme abhiññā sacchikaroti.

@ pali-ms-26.tex
♦ “katame ca, bhikkhave, dhammā abhiññā pariññeyyā? ‘pañcupādānakkhandhā’ tissa vacanīyaṃ, seyyathidaṃ — rūpupādānakkhandho, vedanupādānakkhandho, saññupādānakkhandho, saṅkhārupādānakkhandho, viññāṇupādānakkhandho. ime dhammā abhiññā pariññeyyā.

@ pali-ms-27.tex
♦ “katame ca, bhikkhave, dhammā abhiññā pahātabbā? avijjā ca bhavataṇhā ca — ime dhammā abhiññā pahātabbā.

@ pali-ms-28.tex
♦ “katame ca, bhikkhave, dhammā abhiññā bhāvetabbā? samatho ca vipassanā ca — ime dhammā abhiññā bhāvetabbā.

@ pali-ms-29.tex
♦ “katame ca, bhikkhave, dhammā abhiññā sacchikātabbā? vijjā ca vimutti ca — ime dhammā abhiññā sacchikātabbā”ti.

@ pali-ms-30.tex
♦ idamavoca bhagavā. attamanā te bhikkhū bhagavato bhāsitaṃ abhinandunti.

@ pali-ms-31.tex
♦ mahāsaḷāyatanikasuttaṃ niṭṭhitaṃ sattamaṃ.
