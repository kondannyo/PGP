
\begin{samepage}
\begin{leftcolumn*}
\EnglishColumn{Majjhima Nikāya, mūlapaṇṇāsapāḷi, 4. mahāyamakavaggo n, 8. mahātaṇhāsaṅkhayasuttaṃ n (MN 38)}
\hspace{0pt}\end{leftcolumn*}

\begin{rightcolumn}\PaliColumn{The Greater Discourse on the Destruction of Craving}
\hspace{0pt}
\end{rightcolumn}
\end{samepage}
\vskip 0.2in
\begin{samepage}
\begin{leftcolumn*}
\EnglishColumn{evaṃ me sutaṃ.}
\hspace{0pt}\end{leftcolumn*}

\begin{rightcolumn}\PaliColumn{Thus have I heard.}
\hspace{0pt}
\end{rightcolumn}
\end{samepage}
\begin{samepage}
\begin{leftcolumn*}
\EnglishColumn{ekaṃ samayaṃ bhagavā sāvatthiyaṃ viharati jetavane anāthapiṇḍikassa ārāme.}
\hspace{0pt}\end{leftcolumn*}

\begin{rightcolumn}\PaliColumn{On one occasion the Blessed One was living at Sāvatthī in Jeta’s Grove, Anāthapiṇḍika’s Park.}
\hspace{0pt}
\end{rightcolumn}
\end{samepage}
\begin{samepage}
\begin{leftcolumn*}
\EnglishColumn{tena kho pana samayena sātissa nāma bhikkhuno kevaṭṭaputtassa evarūpaṃ pāpakaṃ diṭṭhigataṃ uppannaṃ hoti:}
\hspace{0pt}\end{leftcolumn*}

\begin{rightcolumn}\PaliColumn{Now on that occasion a pernicious view had arisen in a bhikkhu named Sāti, son of a fisherman, thus:}
\hspace{0pt}
\end{rightcolumn}
\end{samepage}
\begin{samepage}
\begin{leftcolumn*}
\EnglishColumn{“tathāhaṃ bhagavatā dhammaṃ desitaṃ ājānāmi yathā tadevidaṃ viññāṇaṃ sandhāvati saṃsarati anaññan”ti.}
\hspace{0pt}\end{leftcolumn*}

\begin{rightcolumn}\PaliColumn{“As I understand the Dhamma taught by the Blessed One, it is this same consciousness that runs and wanders through the round of rebirths, not another.”}
\hspace{0pt}
\end{rightcolumn}
\end{samepage}
\vskip 0.2in
\begin{samepage}
\begin{leftcolumn*}
\EnglishColumn{assosuṃ kho sambahulā bhikkhū:}
\hspace{0pt}\end{leftcolumn*}

\begin{rightcolumn}\PaliColumn{Several bhikkhus, having heard about this,}
\hspace{0pt}
\end{rightcolumn}
\end{samepage}
\begin{samepage}
\begin{leftcolumn*}
\EnglishColumn{“sātissa kira nāma bhikkhuno kevaṭṭaputtassa evarūpaṃ pāpakaṃ diṭṭhigataṃ uppannaṃ - ‘tathāhaṃ bhagavatā dhammaṃ desitaṃ ājānāmi yathā tadevidaṃ viññāṇaṃ sandhāvati saṃsarati, anaññan’”ti.}
\hspace{0pt}\end{leftcolumn*}

\begin{rightcolumn}\PaliColumn{}
\hspace{0pt}
\end{rightcolumn}
\end{samepage}
\begin{samepage}
\begin{leftcolumn*}
\EnglishColumn{atha kho te bhikkhū yena sāti bhikkhu kevaṭṭaputto tenupasaṅkamiṃsu; upasaṅkamitvā sātiṃ bhikkhuṃ kevaṭṭaputtaṃ etadavocuṃ:}
\hspace{0pt}\end{leftcolumn*}

\begin{rightcolumn}\PaliColumn{went to the bhikkhu Sāti and asked him:}
\hspace{0pt}
\end{rightcolumn}
\end{samepage}
\begin{samepage}
\begin{leftcolumn*}
\EnglishColumn{“saccaṃ kira te, āvuso sāti, evarūpaṃ pāpakaṃ diṭṭhigataṃ uppannaṃ}
\hspace{0pt}\end{leftcolumn*}

\begin{rightcolumn}\PaliColumn{“Friend Sāti, is it true that such a pernicious view has arisen in you?”}
\hspace{0pt}
\end{rightcolumn}
\end{samepage}
\begin{samepage}
\begin{leftcolumn*}
\EnglishColumn{‘tathāhaṃ bhagavatā dhammaṃ desitaṃ ājānāmi yathā tadevidaṃ viññāṇaṃ sandhāvati saṃsarati, anaññan’”ti?}
\hspace{0pt}\end{leftcolumn*}

\begin{rightcolumn}\PaliColumn{“As I understand the Dhamma taught by the Blessed One, it is this same consciousness that runs and wanders through the round of rebirths, not another.”}
\hspace{0pt}
\end{rightcolumn}
\end{samepage}
\begin{samepage}
\begin{leftcolumn*}
\EnglishColumn{“evaṃ byā kho ahaṃ, āvuso, bhagavatā dhammaṃ desitaṃ ājānāmi yathā tadevidaṃ viññāṇaṃ sandhāvati saṃsarati, anaññan”ti.}
\hspace{0pt}\end{leftcolumn*}

\begin{rightcolumn}\PaliColumn{“Exactly so, friends.  As I understand the Dhamma taught by the Blessed One, it is this same consciousness that runs and wanders through the round of rebirths, not another.”}
\hspace{0pt}
\end{rightcolumn}
\end{samepage}
\begin{samepage}
\begin{leftcolumn*}
\EnglishColumn{atha kho te bhikkhū sātiṃ bhikkhuṃ kevaṭṭaputtaṃ etasmā pāpakā diṭṭhigatā vivecetukāmā samanuyuñjanti samanugāhanti samanubhāsanti:}
\hspace{0pt}\end{leftcolumn*}

\begin{rightcolumn}\PaliColumn{Then those bhikkhus, desiring to detach him from that pernicious view, pressed and questioned and cross-questioned him thus:}
\hspace{0pt}
\end{rightcolumn}
\end{samepage}
\begin{samepage}
\begin{leftcolumn*}
\EnglishColumn{“mā evaṃ, āvuso sāti, avaca, mā bhagavantaṃ abbhācikkhi, na hi sādhu bhagavato abbhakkhānaṃ, na hi bhagavā evaṃ vadeyya.}
\hspace{0pt}\end{leftcolumn*}

\begin{rightcolumn}\PaliColumn{“Friend Sāti, do not say so. Do not misrepresent the Blessed One; it is not good to misrepresent the Blessed One.  The Blessed One would not speak thus.}
\hspace{0pt}
\end{rightcolumn}
\end{samepage}
\begin{samepage}
\begin{leftcolumn*}
\EnglishColumn{anekapariyāyenāvuso sāti, paṭiccasamuppannaṃ viññāṇaṃ vuttaṃ bhagavatā, aññatra paccayā natthi viññāṇassa sambhavo”ti. evampi kho sāti bhikkhu kevaṭṭaputto tehi bhikkhūhi samanuyuñjiyamāno samanugāhiyamāno samanubhāsiyamāno tadeva pāpakaṃ diṭṭhigataṃ thāmasā parāmāsā abhinivissa voharati:}
\hspace{0pt}\end{leftcolumn*}

\begin{rightcolumn}\PaliColumn{For in many ways the Blessed One has stated consciousness to be dependently arisen, since without a condition there is no origination of consciousness.”}
\hspace{0pt}
\end{rightcolumn}
\end{samepage}
\begin{samepage}
\begin{leftcolumn*}
\EnglishColumn{“evaṃ byā kho ahaṃ, āvuso, bhagavatā dhammaṃ desitaṃ ājānāmi yathā tadevidaṃ viññāṇaṃ sandhāvati saṃsarati anaññan”ti.}
\hspace{0pt}\end{leftcolumn*}

\begin{rightcolumn}\PaliColumn{Yet although pressed and questioned and cross-questioned by those bhikkhus in this way, the bhikkhu Sāti, son of a fisherman, still obstinately adhered to that pernicious view and continued to insist upon it.}
\hspace{0pt}
\end{rightcolumn}
\end{samepage}
\vskip 0.2in
\begin{samepage}
\begin{leftcolumn*}
\EnglishColumn{yato kho te bhikkhū nāsakkhiṃsu sātiṃ bhikkhuṃ kevaṭṭaputtaṃ etasmā pāpakā diṭṭhigatā vivecetuṃ, atha kho te bhikkhū yena bhagavā tenupasaṅkamiṃsu; upasaṅkamitvā bhagavantaṃ abhivādetvā ekamantaṃ nisīdiṃsu.}
\hspace{0pt}\end{leftcolumn*}

\begin{rightcolumn}\PaliColumn{Since the bhikkhus were unable to detach him from that pernicious view, they went to the Blessed One, and after paying homage to him, they sat down at one side and told him all that had occurred, adding:}
\hspace{0pt}
\end{rightcolumn}
\end{samepage}
\begin{samepage}
\begin{leftcolumn*}
\EnglishColumn{ekamantaṃ nisinnā kho te bhikkhū bhagavantaṃ etadavocuṃ:}
\hspace{0pt}\end{leftcolumn*}

\begin{rightcolumn}\PaliColumn{“Venerable sir, since we could not detach the bhikkhu Sāti, son of a fisherman, from this pernicious view, we have reported this matter to the Blessed One.”}
\hspace{0pt}
\end{rightcolumn}
\end{samepage}
\begin{samepage}
\begin{leftcolumn*}
\EnglishColumn{“sātissa nāma, bhante, bhikkhuno kevaṭṭaputtassa evarūpaṃ pāpakaṃ diṭṭhigataṃ uppannaṃ - ‘tathāhaṃ bhagavatā dhammaṃ desitaṃ ājānāmi yathā tadevidaṃ viññāṇaṃ sandhāvati saṃsarati, anaññan’ti. assumha kho mayaṃ, bhante, sātissa kira nāma bhikkhuno kevaṭṭaputtassa evarūpaṃ pāpakaṃ diṭṭhigataṃ uppannaṃ - ‘tathāhaṃ bhagavatā dhammaṃ desitaṃ ājānāmi yathā tadevidaṃ viññāṇaṃ sandhāvati saṃsarati, anaññan’ti. atha kho mayaṃ, bhante, yena sāti bhikkhu kevaṭṭaputto tenupasaṅkamimha; upasaṅkamitvā sātiṃ bhikkhuṃ kevaṭṭaputtaṃ etadavocumha - ‘saccaṃ kira te, āvuso sāti, evarūpaṃ pāpakaṃ diṭṭhigataṃ uppannaṃ:}
\hspace{0pt}\end{leftcolumn*}

\begin{rightcolumn}\PaliColumn{}
\hspace{0pt}
\end{rightcolumn}
\end{samepage}
\begin{samepage}
\begin{leftcolumn*}
\EnglishColumn{“tathāhaṃ bhagavatā dhammaṃ desitaṃ ājānāmi yathā tadevidaṃ viññāṇaṃ sandhāvati saṃsarati, anaññan”ti?}
\hspace{0pt}\end{leftcolumn*}

\begin{rightcolumn}\PaliColumn{}
\hspace{0pt}
\end{rightcolumn}
\end{samepage}
\begin{samepage}
\begin{leftcolumn*}
\EnglishColumn{evaṃ vutte, bhante, sāti bhikkhu kevaṭṭaputto amhe etadavoca - ‘evaṃ byā kho ahaṃ, āvuso, bhagavatā dhammaṃ desitaṃ ājānāmi yathā tadevidaṃ viññāṇaṃ sandhāvati saṃsarati, anaññan’ti. atha kho mayaṃ, bhante, sātiṃ bhikkhuṃ kevaṭṭaputtaṃ etasmā pāpakā diṭṭhigatā vivecetukāmā samanuyuñjimha samanugāhimha samanubhāsimha - ‘mā evaṃ, āvuso sāti, avaca, mā bhagavantaṃ abbhācikkhi, na hi sādhu bhagavato abbhakkhānaṃ, na hi bhagavā evaṃ vadeyya.}
\hspace{0pt}\end{leftcolumn*}

\begin{rightcolumn}\PaliColumn{}
\hspace{0pt}
\end{rightcolumn}
\end{samepage}
\begin{samepage}
\begin{leftcolumn*}
\EnglishColumn{anekapariyāyenāvuso sāti, paṭiccasamuppannaṃ viññāṇaṃ vuttaṃ bhagavatā, aññatra paccayā natthi viññāṇassa sambhavo’ti. evampi kho, bhante, sāti bhikkhu kevaṭṭaputto amhehi samanuyuñjiyamāno samanugāhiyamāno samanubhāsiyamāno tadeva pāpakaṃ diṭṭhigataṃ thāmasā parāmasā abhinivissa voharati - ‘evaṃ byā kho ahaṃ, āvuso, bhagavatā dhammaṃ desitaṃ ājānāmi yathā tadevidaṃ viññāṇaṃ sandhāvati saṃsarati, anaññan’ti. yato kho mayaṃ, bhante, nāsakkhimha sātiṃ bhikkhuṃ kevaṭṭaputtaṃ etasmā pāpakā diṭṭhigatā vivecetuṃ, atha mayaṃ etamatthaṃ bhagavato ārocemā”ti.}
\hspace{0pt}\end{leftcolumn*}

\begin{rightcolumn}\PaliColumn{}
\hspace{0pt}
\end{rightcolumn}
\end{samepage}
\vskip 0.2in
\begin{samepage}
\begin{leftcolumn*}
\EnglishColumn{atha kho bhagavā aññataraṃ bhikkhuṃ āmantesi:}
\hspace{0pt}\end{leftcolumn*}

\begin{rightcolumn}\PaliColumn{Then the Blessed One addressed a certain bhikkhu thus:}
\hspace{0pt}
\end{rightcolumn}
\end{samepage}
\begin{samepage}
\begin{leftcolumn*}
\EnglishColumn{“ehi tvaṃ bhikkhu, mama vacanena sātiṃ bhikkhuṃ kevaṭṭaputtaṃ āmantehi - ‘satthā taṃ, āvuso sāti, āmantetī’”ti.}
\hspace{0pt}\end{leftcolumn*}

\begin{rightcolumn}\PaliColumn{“Come, bhikkhu, tell the bhikkhu Sāti, son of a fisherman, in my name that the Teacher calls him.”}
\hspace{0pt}
\end{rightcolumn}
\end{samepage}
\begin{samepage}
\begin{leftcolumn*}
\EnglishColumn{“evaṃ, bhante”ti kho so bhikkhu bhagavato paṭissutvā yena sāti bhikkhu kevaṭṭaputto tenupasaṅkami; upasaṅkamitvā sātiṃ bhikkhuṃ kevaṭṭaputtaṃ etadavoca:}
\hspace{0pt}\end{leftcolumn*}

\begin{rightcolumn}\PaliColumn{“Yes, venerable sir,” he replied, and he went to the bhikkhu Sāti and told him:}
\hspace{0pt}
\end{rightcolumn}
\end{samepage}
\begin{samepage}
\begin{leftcolumn*}
\EnglishColumn{“satthā taṃ, āvuso sāti, āmantetī”ti.}
\hspace{0pt}\end{leftcolumn*}

\begin{rightcolumn}\PaliColumn{“The Teacher calls you, friend Sāti.”}
\hspace{0pt}
\end{rightcolumn}
\end{samepage}
\begin{samepage}
\begin{leftcolumn*}
\EnglishColumn{“evamāvuso”ti kho sāti bhikkhu kevaṭṭaputto tassa bhikkhuno paṭissutvā yena bhagavā tenupasaṅkami; upasaṅkamitvā bhagavantaṃ abhivādetvā ekamantaṃ nisīdi.}
\hspace{0pt}\end{leftcolumn*}

\begin{rightcolumn}\PaliColumn{“Yes, friend,” he replied, and he went to the Blessed One, and after paying homage to him, sat down at one side.}
\hspace{0pt}
\end{rightcolumn}
\end{samepage}
\begin{samepage}
\begin{leftcolumn*}
\EnglishColumn{ekamantaṃ nisinnaṃ kho sātiṃ bhikkhuṃ kevaṭṭaputtaṃ bhagavā etadavoca:}
\hspace{0pt}\end{leftcolumn*}

\begin{rightcolumn}\PaliColumn{The Blessed One then asked him:}
\hspace{0pt}
\end{rightcolumn}
\end{samepage}
\begin{samepage}
\begin{leftcolumn*}
\EnglishColumn{“saccaṃ kira, te, sāti, evarūpaṃ pāpakaṃ diṭṭhigataṃ uppannaṃ - ‘tathāhaṃ bhagavatā dhammaṃ desitaṃ ājānāmi yathā tadevidaṃ viññāṇaṃ sandhāvati saṃsarati, anaññan’”ti?}
\hspace{0pt}\end{leftcolumn*}

\begin{rightcolumn}\PaliColumn{“Sāti, is it true that the following pernicious view has arisen in you: ‘As I understand the Dhamma taught by the Blessed One, it is this same consciousness that runs and wanders through the round of rebirths, not another’?”}
\hspace{0pt}
\end{rightcolumn}
\end{samepage}
\begin{samepage}
\begin{leftcolumn*}
\EnglishColumn{“evaṃ byā kho ahaṃ, bhante, bhagavatā dhammaṃ desitaṃ ājānāmi yathā tadevidaṃ viññāṇaṃ sandhāvati saṃsarati, anaññan”ti.}
\hspace{0pt}\end{leftcolumn*}

\begin{rightcolumn}\PaliColumn{“Exactly so, venerable sir.  As I understand the Dhamma taught by the Blessed One, it is this same consciousness that runs and wanders through the round of rebirths, not another.”}
\hspace{0pt}
\end{rightcolumn}
\end{samepage}
\begin{samepage}
\begin{leftcolumn*}
\EnglishColumn{“katamaṃ taṃ, sāti, viññāṇan”ti?}
\hspace{0pt}\end{leftcolumn*}

\begin{rightcolumn}\PaliColumn{“What is that consciousness, Sāti?”}
\hspace{0pt}
\end{rightcolumn}
\end{samepage}
\begin{samepage}
\begin{leftcolumn*}
\EnglishColumn{“yvāyaṃ, bhante, vado vedeyyo tatra tatra kalyāṇapāpakānaṃ kammānaṃ vipākaṃ paṭisaṃvedetī”ti.}
\hspace{0pt}\end{leftcolumn*}

\begin{rightcolumn}\PaliColumn{“Venerable sir, it is that which speaks and feels and experiences here and there the result of good and bad actions.”}
\hspace{0pt}
\end{rightcolumn}
\end{samepage}
\begin{samepage}
\begin{leftcolumn*}
\EnglishColumn{“kassa nu kho nāma tvaṃ, moghapurisa, mayā evaṃ dhammaṃ desitaṃ ājānāsi?}
\hspace{0pt}\end{leftcolumn*}

\begin{rightcolumn}\PaliColumn{“Misguided man, to whom have you ever known me to teach the Dhamma in that way?}
\hspace{0pt}
\end{rightcolumn}
\end{samepage}
\begin{samepage}
\begin{leftcolumn*}
\EnglishColumn{nanu mayā, moghapurisa, anekapariyāyena paṭiccasamuppannaṃ viññāṇaṃ vuttaṃ, aññatra paccayā natthi viññāṇassa sambhavoti?}
\hspace{0pt}\end{leftcolumn*}

\begin{rightcolumn}\PaliColumn{Misguided man, have I not stated in many ways consciousness to be dependently arisen, since without a condition there is no origination of consciousness?}
\hspace{0pt}
\end{rightcolumn}
\end{samepage}
\begin{samepage}
\begin{leftcolumn*}
\EnglishColumn{atha ca pana tvaṃ, moghapurisa, attanā duggahitena amhe ceva abbhācikkhasi, attānañca khaṇasi, bahuñca apuññaṃ pasavasi.  tañhi te, moghapurisa, bhavissati dīgharattaṃ ahitāya dukkhāyā”ti.}
\hspace{0pt}\end{leftcolumn*}

\begin{rightcolumn}\PaliColumn{But you, misguided man, have misrepresented us by your wrong grasp and injured yourself and stored up much demerit; for this will lead to your harm and suffering for a long time.”}
\hspace{0pt}
\end{rightcolumn}
\end{samepage}
\vskip 0.2in
\begin{samepage}
\begin{leftcolumn*}
\EnglishColumn{atha kho bhagavā bhikkhū āmantesi:}
\hspace{0pt}\end{leftcolumn*}

\begin{rightcolumn}\PaliColumn{Then the Blessed One addressed the bhikkhus thus:}
\hspace{0pt}
\end{rightcolumn}
\end{samepage}
\begin{samepage}
\begin{leftcolumn*}
\EnglishColumn{“taṃ kiṃ maññatha, bhikkhave, api nāyaṃ sāti bhikkhu kevaṭṭaputto usmīkatopi imasmiṃ dhammavinaye”ti?}
\hspace{0pt}\end{leftcolumn*}

\begin{rightcolumn}\PaliColumn{“Bhikkhus, what do you think?  Has this bhikkhu Sāti, son of a fisherman, kindled even a spark of wisdom in this Dhamma and Discipline?”}
\hspace{0pt}
\end{rightcolumn}
\end{samepage}
\begin{samepage}
\begin{leftcolumn*}
\EnglishColumn{“kiñhi siyā bhante?}
\hspace{0pt}\end{leftcolumn*}

\begin{rightcolumn}\PaliColumn{“How could he, venerable sir?}
\hspace{0pt}
\end{rightcolumn}
\end{samepage}
\begin{samepage}
\begin{leftcolumn*}
\EnglishColumn{no hetaṃ, bhante”ti. evaṃ vutte, sāti bhikkhu kevaṭṭaputto tuṇhībhūto maṅkubhūto pattakkhandho adhomukho pajjhāyanto appaṭibhāno nisīdi.}
\hspace{0pt}\end{leftcolumn*}

\begin{rightcolumn}\PaliColumn{No, venerable sir.”}
\hspace{0pt}
\end{rightcolumn}
\end{samepage}
\begin{samepage}
\begin{leftcolumn*}
\EnglishColumn{atha kho bhagavā sātiṃ bhikkhuṃ kevaṭṭaputtaṃ tuṇhībhūtaṃ maṅkubhūtaṃ pattakkhandhaṃ adhomukhaṃ pajjhāyantaṃ appaṭibhānaṃ viditvā sātiṃ bhikkhuṃ kevaṭṭaputtaṃ etadavoca:}
\hspace{0pt}\end{leftcolumn*}

\begin{rightcolumn}\PaliColumn{When this was said, the bhikkhu Sāti, son of a fisherman, sat silent, dismayed, with shoulders drooping and head down, glum, and without response. Then, knowing this, the Blessed One told him:}
\hspace{0pt}
\end{rightcolumn}
\end{samepage}
\begin{samepage}
\begin{leftcolumn*}
\EnglishColumn{“paññāyissasi kho tvaṃ, moghapurisa, etena sakena pāpakena diṭṭhigatena.}
\hspace{0pt}\end{leftcolumn*}

\begin{rightcolumn}\PaliColumn{“Misguided man, you will be recognised by your own pernicious view.}
\hspace{0pt}
\end{rightcolumn}
\end{samepage}
\begin{samepage}
\begin{leftcolumn*}
\EnglishColumn{idhāhaṃ bhikkhū paṭipucchissāmī”ti.}
\hspace{0pt}\end{leftcolumn*}

\begin{rightcolumn}\PaliColumn{I shall question the bhikkhus on this matter.”}
\hspace{0pt}
\end{rightcolumn}
\end{samepage}
\vskip 0.2in
\begin{samepage}
\begin{leftcolumn*}
\EnglishColumn{atha kho bhagavā bhikkhū āmantesi:}
\hspace{0pt}\end{leftcolumn*}

\begin{rightcolumn}\PaliColumn{Then the Blessed One addressed the bhikkhus thus:}
\hspace{0pt}
\end{rightcolumn}
\end{samepage}
\begin{samepage}
\begin{leftcolumn*}
\EnglishColumn{“tumhepi me, bhikkhave, evaṃ dhammaṃ desitaṃ ājānātha yathāyaṃ sāti bhikkhu kevaṭṭaputto attanā duggahitena amhe ceva abbhācikkhati, attānañca khaṇati, bahuñca apuññaṃ pasavatī”ti?}
\hspace{0pt}\end{leftcolumn*}

\begin{rightcolumn}\PaliColumn{“Bhikkhus, do you understand the Dhamma taught by me as this bhikkhu Sāti, son of a fisherman, does when he misrepresents us by his wrong grasp and injures himself and stores up much demerit?”}
\hspace{0pt}
\end{rightcolumn}
\end{samepage}
\begin{samepage}
\begin{leftcolumn*}
\EnglishColumn{“no hetaṃ, bhante!}
\hspace{0pt}\end{leftcolumn*}

\begin{rightcolumn}\PaliColumn{“No, venerable sir.}
\hspace{0pt}
\end{rightcolumn}
\end{samepage}
\begin{samepage}
\begin{leftcolumn*}
\EnglishColumn{anekapariyāyena hi no, bhante, paṭiccasamuppannaṃ viññāṇaṃ vuttaṃ bhagavatā, aññatra paccayā natthi viññāṇassa sambhavo”ti.}
\hspace{0pt}\end{leftcolumn*}

\begin{rightcolumn}\PaliColumn{For in many discourses the Blessed One has stated consciousness to be dependently arisen, since without a condition there is no origination of consciousness.”}
\hspace{0pt}
\end{rightcolumn}
\end{samepage}
\begin{samepage}
\begin{leftcolumn*}
\EnglishColumn{“sādhu sādhu, bhikkhave!}
\hspace{0pt}\end{leftcolumn*}

\begin{rightcolumn}\PaliColumn{“Good, bhikkhus.}
\hspace{0pt}
\end{rightcolumn}
\end{samepage}
\begin{samepage}
\begin{leftcolumn*}
\EnglishColumn{sādhu kho me tumhe, bhikkhave, evaṃ dhammaṃ desitaṃ ājānātha.}
\hspace{0pt}\end{leftcolumn*}

\begin{rightcolumn}\PaliColumn{It is good that you understand the Dhamma taught by me thus.}
\hspace{0pt}
\end{rightcolumn}
\end{samepage}
\begin{samepage}
\begin{leftcolumn*}
\EnglishColumn{anekapariyāyena hi vo, bhikkhave, paṭiccasamuppannaṃ viññāṇaṃ vuttaṃ mayā, aññatra paccayā natthi viññāṇassa sambhavoti.}
\hspace{0pt}\end{leftcolumn*}

\begin{rightcolumn}\PaliColumn{For in many ways I have stated consciousness to be dependently arisen, since without a condition there is no origination of consciousness.}
\hspace{0pt}
\end{rightcolumn}
\end{samepage}
\begin{samepage}
\begin{leftcolumn*}
\EnglishColumn{atha ca panāyaṃ sāti bhikkhu kevaṭṭaputto attanā duggahitena amhe ceva abbhācikkhati, attānañca khaṇati, bahuñca apuññaṃ pasavati pasavati.  tañhi tassa moghapurisassa bhavissati dīgharattaṃ ahitāya dukkhāya.}
\hspace{0pt}\end{leftcolumn*}

\begin{rightcolumn}\PaliColumn{But this bhikkhu Sāti, son of a fisherman, misrepresents us by his wrong grasp and injures himself and stores up much demerit; for this will lead to the harm and suffering of this misguided man for a long time.}
\hspace{0pt}
\end{rightcolumn}
\end{samepage}
\vskip 0.2in
\begin{samepage}
\begin{leftcolumn*}
\EnglishColumn{“yaṃ yadeva, bhikkhave, paccayaṃ paṭicca uppajjati viññāṇaṃ, tena teneva viññāṇaṃtveva saṅkhyaṃ gacchati.}
\hspace{0pt}\end{leftcolumn*}

\begin{rightcolumn}\PaliColumn{“Bhikkhus, consciousness is reckoned by the particular condition dependent upon which it arises.}
\hspace{0pt}
\end{rightcolumn}
\end{samepage}
\begin{samepage}
\begin{leftcolumn*}
\EnglishColumn{cakkhuñca paṭicca rūpe ca uppajjati viññāṇaṃ, cakkhuviññāṇaṃtveva saṅkhyaṃ gacchati;}
\hspace{0pt}\end{leftcolumn*}

\begin{rightcolumn}\PaliColumn{When consciousness arises dependent on the eye and forms, it is reckoned as eye-consciousness;}
\hspace{0pt}
\end{rightcolumn}
\end{samepage}
\begin{samepage}
\begin{leftcolumn*}
\EnglishColumn{sotañca paṭicca sadde ca uppajjati viññāṇaṃ, sotaviññāṇaṃtveva saṅkhyaṃ gacchati;}
\hspace{0pt}\end{leftcolumn*}

\begin{rightcolumn}\PaliColumn{when consciousness arises dependent on the ear and sounds, it is reckoned as ear-consciousness;}
\hspace{0pt}
\end{rightcolumn}
\end{samepage}
\begin{samepage}
\begin{leftcolumn*}
\EnglishColumn{ghānañca paṭicca gandhe ca uppajjati viññāṇaṃ, ghānaviññāṇaṃtveva saṅkhyaṃ gacchati;}
\hspace{0pt}\end{leftcolumn*}

\begin{rightcolumn}\PaliColumn{when consciousness arises dependent on the nose and odours, it is reckoned as nose-consciousness;}
\hspace{0pt}
\end{rightcolumn}
\end{samepage}
\begin{samepage}
\begin{leftcolumn*}
\EnglishColumn{jivhañca paṭicca rase ca uppajjati viññāṇaṃ, jivhāviññāṇaṃtveva saṅkhyaṃ gacchati;}
\hspace{0pt}\end{leftcolumn*}

\begin{rightcolumn}\PaliColumn{when consciousness arises dependent on the tongue and flavours, it is reckoned as tongue-consciousness;}
\hspace{0pt}
\end{rightcolumn}
\end{samepage}
\begin{samepage}
\begin{leftcolumn*}
\EnglishColumn{kāyañca paṭicca phoṭṭhabbe ca uppajjati viññāṇaṃ, kāyaviññāṇaṃtveva saṅkhyaṃ gacchati;}
\hspace{0pt}\end{leftcolumn*}

\begin{rightcolumn}\PaliColumn{when consciousness arises dependent on the body and tangibles, it is reckoned as body-consciousness;}
\hspace{0pt}
\end{rightcolumn}
\end{samepage}
\begin{samepage}
\begin{leftcolumn*}
\EnglishColumn{manañca paṭicca dhamme ca uppajjati viññāṇaṃ, manoviññāṇaṃtveva saṅkhyaṃ gacchati.}
\hspace{0pt}\end{leftcolumn*}

\begin{rightcolumn}\PaliColumn{when consciousness arises dependent on the mind and mind-objects, it is reckoned as mind-consciousness.}
\hspace{0pt}
\end{rightcolumn}
\end{samepage}
\vskip 0.2in
\begin{samepage}
\begin{leftcolumn*}
\EnglishColumn{“seyyathāpi, bhikkhave, yaṃ yadeva paccayaṃ paṭicca aggi jalati tena teneva saṅkhyaṃ gacchati.}
\hspace{0pt}\end{leftcolumn*}

\begin{rightcolumn}\PaliColumn{Just as fire is reckoned by the particular condition dependent on which it burns—when fire burns dependent on logs, it is reckoned as a log fire;}
\hspace{0pt}
\end{rightcolumn}
\end{samepage}
\begin{samepage}
\begin{leftcolumn*}
\EnglishColumn{kaṭṭhañca paṭicca aggi jalati, kaṭṭhaggitveva saṅkhyaṃ gacchati;}
\hspace{0pt}\end{leftcolumn*}

\begin{rightcolumn}\PaliColumn{when fire burns dependent on faggots, it is reckoned as a faggot fire;}
\hspace{0pt}
\end{rightcolumn}
\end{samepage}
\begin{samepage}
\begin{leftcolumn*}
\EnglishColumn{sakalikañca paṭicca aggi jalati, sakalikaggitveva saṅkhyaṃ gacchati;}
\hspace{0pt}\end{leftcolumn*}

\begin{rightcolumn}\PaliColumn{when fire burns dependent on grass, it is reckoned as a grass fire;}
\hspace{0pt}
\end{rightcolumn}
\end{samepage}
\begin{samepage}
\begin{leftcolumn*}
\EnglishColumn{tiṇañca paṭicca aggi jalati, tiṇaggitveva saṅkhyaṃ gacchati; gomayañca paṭicca aggi jalati, gomayaggitveva saṅkhyaṃ gacchati;}
\hspace{0pt}\end{leftcolumn*}

\begin{rightcolumn}\PaliColumn{when fire burns dependent on cowdung, it is reckoned as a cowdung fire;}
\hspace{0pt}
\end{rightcolumn}
\end{samepage}
\begin{samepage}
\begin{leftcolumn*}
\EnglishColumn{thusañca paṭicca aggi jalati, thusaggitveva saṅkhyaṃ gacchati;}
\hspace{0pt}\end{leftcolumn*}

\begin{rightcolumn}\PaliColumn{when fire burns dependent on chaff, it is reckoned as a chaff fire;}
\hspace{0pt}
\end{rightcolumn}
\end{samepage}
\begin{samepage}
\begin{leftcolumn*}
\EnglishColumn{saṅkārañca paṭicca aggi jalati, saṅkāraggitveva saṅkhyaṃ gacchati.}
\hspace{0pt}\end{leftcolumn*}

\begin{rightcolumn}\PaliColumn{when fire burns dependent on rubbish, it is reckoned as a rubbish fire;}
\hspace{0pt}
\end{rightcolumn}
\end{samepage}
\begin{samepage}
\begin{leftcolumn*}
\EnglishColumn{evameva kho, bhikkhave, yaṃ yadeva paccayaṃ paṭicca uppajjati viññāṇaṃ, tena teneva saṅkhyaṃ gacchati.}
\hspace{0pt}\end{leftcolumn*}

\begin{rightcolumn}\PaliColumn{so too, consciousness is reckoned by the particular condition dependent on which it arises.}
\hspace{0pt}
\end{rightcolumn}
\end{samepage}
\vskip 0.2in
\begin{samepage}
\begin{leftcolumn*}
\EnglishColumn{cakkhuñca paṭicca rūpe ca uppajjati viññāṇaṃ, cakkhuviññāṇaṃtveva saṅkhyaṃ gacchati;}
\hspace{0pt}\end{leftcolumn*}

\begin{rightcolumn}\PaliColumn{When consciousness arises dependent on the eye and forms, it is reckoned as eye-consciousness;}
\hspace{0pt}
\end{rightcolumn}
\end{samepage}
\begin{samepage}
\begin{leftcolumn*}
\EnglishColumn{sotañca paṭicca sadde ca uppajjati viññāṇaṃ, sotaviññāṇaṃtveva saṅkhyaṃ gacchati;}
\hspace{0pt}\end{leftcolumn*}

\begin{rightcolumn}\PaliColumn{when consciousness arises dependent on the ear and sounds, it is reckoned as ear-consciousness;}
\hspace{0pt}
\end{rightcolumn}
\end{samepage}
\begin{samepage}
\begin{leftcolumn*}
\EnglishColumn{ghānañca paṭicca gandhe ca uppajjati viññāṇaṃ, ghāṇaviññāṇaṃtveva saṅkhyaṃ gacchati;}
\hspace{0pt}\end{leftcolumn*}

\begin{rightcolumn}\PaliColumn{when consciousness arises dependent on the nose and odours, it is reckoned as nose-consciousness;}
\hspace{0pt}
\end{rightcolumn}
\end{samepage}
\begin{samepage}
\begin{leftcolumn*}
\EnglishColumn{jivhañca paṭicca rase ca uppajjati viññāṇaṃ, jivhāviññāṇaṃtveva saṅkhyaṃ gacchati;}
\hspace{0pt}\end{leftcolumn*}

\begin{rightcolumn}\PaliColumn{when consciousness arises dependent on the tongue and flavours, it is reckoned as tongue-consciousness;}
\hspace{0pt}
\end{rightcolumn}
\end{samepage}
\begin{samepage}
\begin{leftcolumn*}
\EnglishColumn{kāyañca paṭicca phoṭṭhabbe ca uppajjati viññāṇaṃ, kāyaviññāṇaṃtveva saṅkhyaṃ gacchati;}
\hspace{0pt}\end{leftcolumn*}

\begin{rightcolumn}\PaliColumn{when consciousness arises dependent on the body and tangibles, it is reckoned as body-consciousness;}
\hspace{0pt}
\end{rightcolumn}
\end{samepage}
\begin{samepage}
\begin{leftcolumn*}
\EnglishColumn{manañca paṭicca dhamme ca uppajjati viññāṇaṃ, manoviññāṇaṃtveva saṅkhyaṃ gacchati.}
\hspace{0pt}\end{leftcolumn*}

\begin{rightcolumn}\PaliColumn{when consciousness arises dependent on the mind and mind-objects, it is reckoned as mind-consciousness.}
\hspace{0pt}
\end{rightcolumn}
\end{samepage}
\vskip 0.2in
\begin{samepage}
\begin{leftcolumn*}
\EnglishColumn{“bhūtamidanti, bhikkhave, passathā”ti?}
\hspace{0pt}\end{leftcolumn*}

\begin{rightcolumn}\PaliColumn{“Bhikkhus, do you see: ‘This has come to be’?”}
\hspace{0pt}
\end{rightcolumn}
\end{samepage}
\begin{samepage}
\begin{leftcolumn*}
\EnglishColumn{“evaṃ, bhante”.}
\hspace{0pt}\end{leftcolumn*}

\begin{rightcolumn}\PaliColumn{“Yes, venerable sir.”}
\hspace{0pt}
\end{rightcolumn}
\end{samepage}
\begin{samepage}
\begin{leftcolumn*}
\EnglishColumn{“tadāhārasambhavanti, bhikkhave, passathā”ti?}
\hspace{0pt}\end{leftcolumn*}

\begin{rightcolumn}\PaliColumn{“Bhikkhus, do you see: ‘Its origination occurs with that as nutriment’?”}
\hspace{0pt}
\end{rightcolumn}
\end{samepage}
\begin{samepage}
\begin{leftcolumn*}
\EnglishColumn{“evaṃ, bhante”.}
\hspace{0pt}\end{leftcolumn*}

\begin{rightcolumn}\PaliColumn{“Yes, venerable sir.”}
\hspace{0pt}
\end{rightcolumn}
\end{samepage}
\begin{samepage}
\begin{leftcolumn*}
\EnglishColumn{“tadāhāranirodhā yaṃ bhūtaṃ, taṃ nirodhadhammanti, bhikkhave, passathā”ti?}
\hspace{0pt}\end{leftcolumn*}

\begin{rightcolumn}\PaliColumn{“Bhikkhus, do you see: ‘With the cessation of that nutriment, what has come to be is subject to cessation’?”}
\hspace{0pt}
\end{rightcolumn}
\end{samepage}
\begin{samepage}
\begin{leftcolumn*}
\EnglishColumn{“evaṃ, bhante”.}
\hspace{0pt}\end{leftcolumn*}

\begin{rightcolumn}\PaliColumn{“Yes, venerable sir.”}
\hspace{0pt}
\end{rightcolumn}
\end{samepage}
\vskip 0.2in
\begin{samepage}
\begin{leftcolumn*}
\EnglishColumn{“bhūtamidaṃ nossūti, bhikkhave, kaṅkhato uppajjati vicikicchā”ti?}
\hspace{0pt}\end{leftcolumn*}

\begin{rightcolumn}\PaliColumn{“Bhikkhus, does doubt arise when one is uncertain thus: ‘Has this come to be’?”}
\hspace{0pt}
\end{rightcolumn}
\end{samepage}
\begin{samepage}
\begin{leftcolumn*}
\EnglishColumn{“evaṃ, bhante”.}
\hspace{0pt}\end{leftcolumn*}

\begin{rightcolumn}\PaliColumn{“Yes, venerable sir.”}
\hspace{0pt}
\end{rightcolumn}
\end{samepage}
\begin{samepage}
\begin{leftcolumn*}
\EnglishColumn{“tadāhārasambhavaṃ nossūti, bhikkhave, kaṅkhato uppajjati vicikicchā”ti?}
\hspace{0pt}\end{leftcolumn*}

\begin{rightcolumn}\PaliColumn{“Bhikkhus, does doubt arise when one is uncertain thus: ‘Does its origination occur with that as nutriment’?”}
\hspace{0pt}
\end{rightcolumn}
\end{samepage}
\begin{samepage}
\begin{leftcolumn*}
\EnglishColumn{“evaṃ, bhante”.}
\hspace{0pt}\end{leftcolumn*}

\begin{rightcolumn}\PaliColumn{“Yes, venerable sir.”}
\hspace{0pt}
\end{rightcolumn}
\end{samepage}
\begin{samepage}
\begin{leftcolumn*}
\EnglishColumn{“tadāhāranirodhā yaṃ bhūtaṃ, taṃ nirodhadhammaṃ nossūti, bhikkhave, kaṅkhato uppajjati vicikicchā”ti?}
\hspace{0pt}\end{leftcolumn*}

\begin{rightcolumn}\PaliColumn{“Bhikkhus, does doubt arise when one is uncertain thus: ‘With the cessation of that nutriment, is what has come to be subject to cessation’?”}
\hspace{0pt}
\end{rightcolumn}
\end{samepage}
\begin{samepage}
\begin{leftcolumn*}
\EnglishColumn{“evaṃ, bhante”.}
\hspace{0pt}\end{leftcolumn*}

\begin{rightcolumn}\PaliColumn{“Yes, venerable sir.”}
\hspace{0pt}
\end{rightcolumn}
\end{samepage}
\vskip 0.2in
\begin{samepage}
\begin{leftcolumn*}
\EnglishColumn{“bhūtamidanti, bhikkhave, yathābhūtaṃ sammappaññāya passato yā vicikicchā sā pahīyatī”ti?}
\hspace{0pt}\end{leftcolumn*}

\begin{rightcolumn}\PaliColumn{“Bhikkhus, is doubt abandoned in one who sees as it actually is with proper wisdom thus: ‘This has come to be’?”}
\hspace{0pt}
\end{rightcolumn}
\end{samepage}
\begin{samepage}
\begin{leftcolumn*}
\EnglishColumn{“evaṃ, bhante”.}
\hspace{0pt}\end{leftcolumn*}

\begin{rightcolumn}\PaliColumn{“Yes, venerable sir.”}
\hspace{0pt}
\end{rightcolumn}
\end{samepage}
\begin{samepage}
\begin{leftcolumn*}
\EnglishColumn{“tadāhārasambhavanti, bhikkhave, yathābhūtaṃ sammappaññāya passatāe yā vicikicchā sā pahīyatī”ti?}
\hspace{0pt}\end{leftcolumn*}

\begin{rightcolumn}\PaliColumn{“Bhikkhus, is doubt abandoned in one who sees as it actually is with proper wisdom thus: ‘Its origination occurs with that as nutriment’?”}
\hspace{0pt}
\end{rightcolumn}
\end{samepage}
\begin{samepage}
\begin{leftcolumn*}
\EnglishColumn{“evaṃ, bhante”.}
\hspace{0pt}\end{leftcolumn*}

\begin{rightcolumn}\PaliColumn{“Yes, venerable sir.”}
\hspace{0pt}
\end{rightcolumn}
\end{samepage}
\begin{samepage}
\begin{leftcolumn*}
\EnglishColumn{“tadāhāranirodhā yaṃ bhūtaṃ, taṃ nirodhadhammanti, bhikkhave, yathābhūtaṃ sammappaññāya passatāe yā vicikicchā sā pahīyatī”ti?}
\hspace{0pt}\end{leftcolumn*}

\begin{rightcolumn}\PaliColumn{“Bhikkhus, is doubt abandoned in one who sees as it actually is with proper wisdom thus: ‘With the cessation of that nutriment, what has come to be is subject to cessation’?”}
\hspace{0pt}
\end{rightcolumn}
\end{samepage}
\begin{samepage}
\begin{leftcolumn*}
\EnglishColumn{“evaṃ, bhante”.}
\hspace{0pt}\end{leftcolumn*}

\begin{rightcolumn}\PaliColumn{“Yes, venerable sir.”}
\hspace{0pt}
\end{rightcolumn}
\end{samepage}
\vskip 0.2in
\begin{samepage}
\begin{leftcolumn*}
\EnglishColumn{“bhūtamidanti, bhikkhave, itipi vo ettha nibbicikicchā”ti?}
\hspace{0pt}\end{leftcolumn*}

\begin{rightcolumn}\PaliColumn{“Bhikkhus, are you thus free from doubt here: ‘This has come to be’?”}
\hspace{0pt}
\end{rightcolumn}
\end{samepage}
\begin{samepage}
\begin{leftcolumn*}
\EnglishColumn{“evaṃ, bhante”.}
\hspace{0pt}\end{leftcolumn*}

\begin{rightcolumn}\PaliColumn{“Yes, venerable sir.”}
\hspace{0pt}
\end{rightcolumn}
\end{samepage}
\begin{samepage}
\begin{leftcolumn*}
\EnglishColumn{“tadāhārasambhavanti, bhikkhave, itipi vo ettha nibbicikicchā”ti?}
\hspace{0pt}\end{leftcolumn*}

\begin{rightcolumn}\PaliColumn{“Bhikkhus, are you thus free from doubt here: ‘Its origination occurs with that as nutriment’?”}
\hspace{0pt}
\end{rightcolumn}
\end{samepage}
\begin{samepage}
\begin{leftcolumn*}
\EnglishColumn{“evaṃ, bhante”.}
\hspace{0pt}\end{leftcolumn*}

\begin{rightcolumn}\PaliColumn{“Yes, venerable sir.”}
\hspace{0pt}
\end{rightcolumn}
\end{samepage}
\begin{samepage}
\begin{leftcolumn*}
\EnglishColumn{“tadāhāranirodhā yaṃ bhūtaṃ taṃ nirodhadhammanti, bhikkhave, itipi vo ettha nibbicikicchā”ti?}
\hspace{0pt}\end{leftcolumn*}

\begin{rightcolumn}\PaliColumn{“Bhikkhus, are you thus free from doubt here: ‘With the cessation of that nutriment, what has come to be is subject to cessation’?”}
\hspace{0pt}
\end{rightcolumn}
\end{samepage}
\begin{samepage}
\begin{leftcolumn*}
\EnglishColumn{“evaṃ, bhante”.}
\hspace{0pt}\end{leftcolumn*}

\begin{rightcolumn}\PaliColumn{“Yes, venerable sir.”}
\hspace{0pt}
\end{rightcolumn}
\end{samepage}
\vskip 0.2in
\begin{samepage}
\begin{leftcolumn*}
\EnglishColumn{“bhūtamidanti, bhikkhave, yathābhūtaṃ sammappaññāya sudiṭṭhan”ti?}
\hspace{0pt}\end{leftcolumn*}

\begin{rightcolumn}\PaliColumn{“Bhikkhus, has it been seen well by you as it actually is with proper wisdom thus: ‘This has come to be’?”}
\hspace{0pt}
\end{rightcolumn}
\end{samepage}
\begin{samepage}
\begin{leftcolumn*}
\EnglishColumn{“evaṃ, bhante”.}
\hspace{0pt}\end{leftcolumn*}

\begin{rightcolumn}\PaliColumn{“Yes, venerable sir.”}
\hspace{0pt}
\end{rightcolumn}
\end{samepage}
\begin{samepage}
\begin{leftcolumn*}
\EnglishColumn{“tadāhārasambhavanti, bhikkhave, yathābhūtaṃ sammappaññāya sudiṭṭhan”ti?}
\hspace{0pt}\end{leftcolumn*}

\begin{rightcolumn}\PaliColumn{“Bhikkhus, has it been seen well by you as it actually is with proper wisdom thus: ‘Its origination occurs with that as nutriment’?”}
\hspace{0pt}
\end{rightcolumn}
\end{samepage}
\begin{samepage}
\begin{leftcolumn*}
\EnglishColumn{“evaṃ, bhante”.}
\hspace{0pt}\end{leftcolumn*}

\begin{rightcolumn}\PaliColumn{“Yes, venerable sir.”}
\hspace{0pt}
\end{rightcolumn}
\end{samepage}
\begin{samepage}
\begin{leftcolumn*}
\EnglishColumn{“tadāhāranirodhā yaṃ bhūtaṃ taṃ nirodhadhammanti, bhikkhave, yathābhūtaṃ sammappaññāya sudiṭṭhan”ti?}
\hspace{0pt}\end{leftcolumn*}

\begin{rightcolumn}\PaliColumn{“Bhikkhus, has it been seen well by you as it actually is with proper wisdom thus: ‘With the cessation of that nutriment, what has come to be is subject to cessation’?”}
\hspace{0pt}
\end{rightcolumn}
\end{samepage}
\begin{samepage}
\begin{leftcolumn*}
\EnglishColumn{“evaṃ, bhante”.}
\hspace{0pt}\end{leftcolumn*}

\begin{rightcolumn}\PaliColumn{“Yes, venerable sir.”}
\hspace{0pt}
\end{rightcolumn}
\end{samepage}
\vskip 0.2in
\begin{samepage}
\begin{leftcolumn*}
\EnglishColumn{“imaṃ ce tumhe, bhikkhave, diṭṭhiṃ evaṃ parisuddhaṃ evaṃ pariyodātaṃ allīyetha kelāyetha dhanāyetha mamāyetha, api nu me tumhe, bhikkhave, kullūpamaṃ dhammaṃ desitaṃ ājāneyyātha nittharaṇatthāya no gahaṇatthāyā”ti?}
\hspace{0pt}\end{leftcolumn*}

\begin{rightcolumn}\PaliColumn{“Bhikkhus, purified and bright as this view is, if you adhere to it, cherish it, treasure it, and treat it as a possession, would you then understand that the Dhamma has been taught as similar to a raft, being for the purpose of crossing over, not for the purpose of grasping?”}
\hspace{0pt}
\end{rightcolumn}
\end{samepage}
\begin{samepage}
\begin{leftcolumn*}
\EnglishColumn{“no hetaṃ, bhante”.}
\hspace{0pt}\end{leftcolumn*}

\begin{rightcolumn}\PaliColumn{“No, venerable sir.”}
\hspace{0pt}
\end{rightcolumn}
\end{samepage}
\vskip 0.2in
\begin{samepage}
\begin{leftcolumn*}
\EnglishColumn{“imaṃ ce tumhe, bhikkhave, diṭṭhiṃ evaṃ parisuddhaṃ evaṃ pariyodātaṃ na allīyetha na kelāyetha na dhanāyetha na mamāyetha, api nu me tumhe, bhikkhave, kullūpamaṃ dhammaṃ desitaṃ ājāneyyātha nittharaṇatthāya no gahaṇatthāyā”ti?}
\hspace{0pt}\end{leftcolumn*}

\begin{rightcolumn}\PaliColumn{“Bhikkhus, purified and bright as this view is, if you do not adhere to it, cherish it, treasure it, and treat it as a possession, would you then understand that the Dhamma has been taught as similar to a raft, being for the purpose of crossing over, not for the purpose of grasping?”}
\hspace{0pt}
\end{rightcolumn}
\end{samepage}
\begin{samepage}
\begin{leftcolumn*}
\EnglishColumn{“evaṃ, bhante”.}
\hspace{0pt}\end{leftcolumn*}

\begin{rightcolumn}\PaliColumn{“Yes, venerable sir.”}
\hspace{0pt}
\end{rightcolumn}
\end{samepage}
\vskip 0.2in
\begin{samepage}
\begin{leftcolumn*}
\EnglishColumn{“cattārome, bhikkhave, āhārā bhūtānaṃ vā sattānaṃ ṭhitiyā, sambhavesīnaṃ vā anuggahāya.}
\hspace{0pt}\end{leftcolumn*}

\begin{rightcolumn}\PaliColumn{“Bhikkhus, there are these four kinds of nutriment for the maintenance of beings that already have come to be and for the support of those about to come to be.}
\hspace{0pt}
\end{rightcolumn}
\end{samepage}
\begin{samepage}
\begin{leftcolumn*}
\EnglishColumn{katame cattāro?}
\hspace{0pt}\end{leftcolumn*}

\begin{rightcolumn}\PaliColumn{What four?}
\hspace{0pt}
\end{rightcolumn}
\end{samepage}
\begin{samepage}
\begin{leftcolumn*}
\EnglishColumn{kabaḷīkāro āhāro oḷāriko vā sukhumo vā, phasso dutiyo, manosañcetanā tatiyā, viññāṇaṃ catutthaṃ.}
\hspace{0pt}\end{leftcolumn*}

\begin{rightcolumn}\PaliColumn{They are: physical food as nutriment, gross or subtle; contact as the second; mental volition as the third; and consciousness as the fourth.}
\hspace{0pt}
\end{rightcolumn}
\end{samepage}
\vskip 0.2in
\begin{samepage}
\begin{leftcolumn*}
\EnglishColumn{“ime ca, bhikkhave, cattāro āhārā kiṃnidānā kiṃsamudayā kiṃjātikā kiṃpabhavā?}
\hspace{0pt}\end{leftcolumn*}

\begin{rightcolumn}\PaliColumn{“Now, bhikkhus, these four kinds of nutriment have what as their source, what as their origin, from what are they born and produced?}
\hspace{0pt}
\end{rightcolumn}
\end{samepage}
\begin{samepage}
\begin{leftcolumn*}
\EnglishColumn{“ime cattāro āhārā taṇhānidānā taṇhāsamudayā taṇhājātikā taṇhāpabhavā.}
\hspace{0pt}\end{leftcolumn*}

\begin{rightcolumn}\PaliColumn{These four kinds of nutriment have craving as their source, craving as their origin; they are born and produced from craving.}
\hspace{0pt}
\end{rightcolumn}
\end{samepage}
\vskip 0.2in
\begin{samepage}
\begin{leftcolumn*}
\EnglishColumn{“taṇhā cāyaṃ, bhikkhave, kiṃnidānā kiṃsamudayā kiṃjātikā kiṃpabhavā?}
\hspace{0pt}\end{leftcolumn*}

\begin{rightcolumn}\PaliColumn{And this craving has what as its source, what as its origin, from what is it born and produced?}
\hspace{0pt}
\end{rightcolumn}
\end{samepage}
\begin{samepage}
\begin{leftcolumn*}
\EnglishColumn{“taṇhā vedanānidānā vedanāsamudayā vedanājātikā vedanāpabhavā.}
\hspace{0pt}\end{leftcolumn*}

\begin{rightcolumn}\PaliColumn{Craving has feeling as its source, feeling as its origin; it is born and produced from feeling.}
\hspace{0pt}
\end{rightcolumn}
\end{samepage}
\vskip 0.2in
\begin{samepage}
\begin{leftcolumn*}
\EnglishColumn{“vedanā cāyaṃ, bhikkhave, kiṃnidānā kiṃsamudayā kiṃjātikā kiṃpabhavā?}
\hspace{0pt}\end{leftcolumn*}

\begin{rightcolumn}\PaliColumn{And this feeling has what as its source, what as its origin, from what is it born and produced?}
\hspace{0pt}
\end{rightcolumn}
\end{samepage}
\begin{samepage}
\begin{leftcolumn*}
\EnglishColumn{“vedanā phassanidānā phassasamudayā phassajātikā phassapabhavā.}
\hspace{0pt}\end{leftcolumn*}

\begin{rightcolumn}\PaliColumn{Feeling has contact as its source, contact as its origin; it is born and produced from contact.}
\hspace{0pt}
\end{rightcolumn}
\end{samepage}
\vskip 0.2in
\begin{samepage}
\begin{leftcolumn*}
\EnglishColumn{“phasso cāyaṃ, bhikkhave, kiṃnidāno kiṃsamudayo kiṃjātiko kiṃpabhavo?}
\hspace{0pt}\end{leftcolumn*}

\begin{rightcolumn}\PaliColumn{And this contact has what as its source, what as its origin, from what is it born and produced?}
\hspace{0pt}
\end{rightcolumn}
\end{samepage}
\begin{samepage}
\begin{leftcolumn*}
\EnglishColumn{“phasso saḷāyatananidāno saḷāyatanasamudayo saḷāyatanajātiko saḷāyatanapabhavo.}
\hspace{0pt}\end{leftcolumn*}

\begin{rightcolumn}\PaliColumn{Contact has the sixfold base as its source, the sixfold base as its origin; it is born and produced from the sixfold base.}
\hspace{0pt}
\end{rightcolumn}
\end{samepage}
\vskip 0.2in
\begin{samepage}
\begin{leftcolumn*}
\EnglishColumn{“saḷāyatanaṃ cidaṃ, bhikkhave, kiṃnidānaṃ kiṃsamudayaṃ kiṃjātikaṃ kiṃpabhavaṃ?}
\hspace{0pt}\end{leftcolumn*}

\begin{rightcolumn}\PaliColumn{And this sixfold base has what as its source, what as its origin, from what is it born and produced?}
\hspace{0pt}
\end{rightcolumn}
\end{samepage}
\begin{samepage}
\begin{leftcolumn*}
\EnglishColumn{“saḷāyatanaṃ nāmarūpanidānaṃ nāmarūpasamudayaṃ nāmarūpajātikaṃ nāmarūpapabhavaṃ.}
\hspace{0pt}\end{leftcolumn*}

\begin{rightcolumn}\PaliColumn{The sixfold base has name and form as its source, name and form as its origin; it is born and produced from name and form.}
\hspace{0pt}
\end{rightcolumn}
\end{samepage}
\vskip 0.2in
\begin{samepage}
\begin{leftcolumn*}
\EnglishColumn{“nāmarūpaṃ cidaṃ, bhikkhave, kiṃnidānaṃ kiṃsamudayaṃ kiṃjātikaṃ kiṃpabhavaṃ?}
\hspace{0pt}\end{leftcolumn*}

\begin{rightcolumn}\PaliColumn{And this name and form has what as its source, what as its origin, from what is it born and produced?}
\hspace{0pt}
\end{rightcolumn}
\end{samepage}
\begin{samepage}
\begin{leftcolumn*}
\EnglishColumn{“nāmarūpaṃ viññāṇanidānaṃ viññāṇasamudayaṃ viññāṇajātikaṃ viññāṇapabhavaṃ.}
\hspace{0pt}\end{leftcolumn*}

\begin{rightcolumn}\PaliColumn{Name and form has consciousness as its source, consciousness as its origin; it is born and produced from consciousness.}
\hspace{0pt}
\end{rightcolumn}
\end{samepage}
\vskip 0.2in
\begin{samepage}
\begin{leftcolumn*}
\EnglishColumn{“viññāṇaṃ cidaṃ, bhikkhave, kiṃnidānaṃ kiṃsamudayaṃ kiṃjātikaṃ kiṃpabhavaṃ?}
\hspace{0pt}\end{leftcolumn*}

\begin{rightcolumn}\PaliColumn{And this consciousness has what as its source, what as its origin, from what is it born and produced?}
\hspace{0pt}
\end{rightcolumn}
\end{samepage}
\begin{samepage}
\begin{leftcolumn*}
\EnglishColumn{“viññāṇaṃ saṅkhāranidānaṃ saṅkhārasamudayaṃ saṅkhārajātikaṃ saṅkhārapabhavaṃ.}
\hspace{0pt}\end{leftcolumn*}

\begin{rightcolumn}\PaliColumn{Consciousness has formations as its source, formations as its origin; it is born and produced from formations.}
\hspace{0pt}
\end{rightcolumn}
\end{samepage}
\vskip 0.2in
\begin{samepage}
\begin{leftcolumn*}
\EnglishColumn{“saṅkhārā cime, bhikkhave, kiṃnidānā kiṃsamudayā kiṃjātikā kiṃpabhavā?}
\hspace{0pt}\end{leftcolumn*}

\begin{rightcolumn}\PaliColumn{And these formations have what as their source, what as their origin, from what are they born and produced?}
\hspace{0pt}
\end{rightcolumn}
\end{samepage}
\begin{samepage}
\begin{leftcolumn*}
\EnglishColumn{“saṅkhārā avijjānidānā avijjāsamudayā avijjājātikā avijjāpabhavā.}
\hspace{0pt}\end{leftcolumn*}

\begin{rightcolumn}\PaliColumn{Formations have ignorance as their source, ignorance as their origin; they are born and produced from ignorance.}
\hspace{0pt}
\end{rightcolumn}
\end{samepage}
\vskip 0.2in
\begin{samepage}
\begin{leftcolumn*}
\EnglishColumn{“iti kho, bhikkhave, avijjāpaccayā saṅkhārā,}
\hspace{0pt}\end{leftcolumn*}

\begin{rightcolumn}\PaliColumn{“So, bhikkhus, with ignorance as condition, formations [come to be];}
\hspace{0pt}
\end{rightcolumn}
\end{samepage}
\begin{samepage}
\begin{leftcolumn*}
\EnglishColumn{saṅkhārapaccayā viññāṇaṃ,}
\hspace{0pt}\end{leftcolumn*}

\begin{rightcolumn}\PaliColumn{with formations as condition, consciousness;}
\hspace{0pt}
\end{rightcolumn}
\end{samepage}
\begin{samepage}
\begin{leftcolumn*}
\EnglishColumn{viññāṇapaccayā nāmarūpaṃ,}
\hspace{0pt}\end{leftcolumn*}

\begin{rightcolumn}\PaliColumn{with consciousness as condition, name and form;}
\hspace{0pt}
\end{rightcolumn}
\end{samepage}
\begin{samepage}
\begin{leftcolumn*}
\EnglishColumn{nāmarūpapaccayā saḷāyatanaṃ,}
\hspace{0pt}\end{leftcolumn*}

\begin{rightcolumn}\PaliColumn{with name and form as condition, the sixfold base;}
\hspace{0pt}
\end{rightcolumn}
\end{samepage}
\begin{samepage}
\begin{leftcolumn*}
\EnglishColumn{saḷāyatanapaccayā phasso,}
\hspace{0pt}\end{leftcolumn*}

\begin{rightcolumn}\PaliColumn{with the sixfold base as condition, contact;}
\hspace{0pt}
\end{rightcolumn}
\end{samepage}
\begin{samepage}
\begin{leftcolumn*}
\EnglishColumn{phassapaccayā vedanā,}
\hspace{0pt}\end{leftcolumn*}

\begin{rightcolumn}\PaliColumn{with contact as condition, feeling;}
\hspace{0pt}
\end{rightcolumn}
\end{samepage}
\begin{samepage}
\begin{leftcolumn*}
\EnglishColumn{vedanāpaccayā taṇhā,}
\hspace{0pt}\end{leftcolumn*}

\begin{rightcolumn}\PaliColumn{with feeling as condition, craving;}
\hspace{0pt}
\end{rightcolumn}
\end{samepage}
\begin{samepage}
\begin{leftcolumn*}
\EnglishColumn{taṇhāpaccayā upādānaṃ,}
\hspace{0pt}\end{leftcolumn*}

\begin{rightcolumn}\PaliColumn{with craving as condition, clinging;}
\hspace{0pt}
\end{rightcolumn}
\end{samepage}
\begin{samepage}
\begin{leftcolumn*}
\EnglishColumn{upādānapaccayā bhavo,}
\hspace{0pt}\end{leftcolumn*}

\begin{rightcolumn}\PaliColumn{with clinging as condition, being;}
\hspace{0pt}
\end{rightcolumn}
\end{samepage}
\begin{samepage}
\begin{leftcolumn*}
\EnglishColumn{bhavapaccayā jāti,}
\hspace{0pt}\end{leftcolumn*}

\begin{rightcolumn}\PaliColumn{with being as condition, birth;}
\hspace{0pt}
\end{rightcolumn}
\end{samepage}
\begin{samepage}
\begin{leftcolumn*}
\EnglishColumn{jātipaccayā jarāmaraṇaṃ sokaparidevadukkhadomanassupāyāsā sambhavanti.}
\hspace{0pt}\end{leftcolumn*}

\begin{rightcolumn}\PaliColumn{with birth as condition, ageing and death, sorrow, lamentation, pain, grief, and despair come to be.}
\hspace{0pt}
\end{rightcolumn}
\end{samepage}
\begin{samepage}
\begin{leftcolumn*}
\EnglishColumn{evametassa kevalassa dukkhakkhandhassa samudayo hoti.’”}
\hspace{0pt}\end{leftcolumn*}

\begin{rightcolumn}\PaliColumn{Such is the origin of this whole mass of suffering.}
\hspace{0pt}
\end{rightcolumn}
\end{samepage}
\vskip 0.2in
\begin{samepage}
\begin{leftcolumn*}
\EnglishColumn{“jātipaccayā jarāmaraṇanti iti kho panetaṃ vuttaṃ;}
\hspace{0pt}\end{leftcolumn*}

\begin{rightcolumn}\PaliColumn{“‘With birth as condition, ageing and death’: so it was said.}
\hspace{0pt}
\end{rightcolumn}
\end{samepage}
\begin{samepage}
\begin{leftcolumn*}
\EnglishColumn{jātipaccayā nu kho, bhikkhave, jarāmaraṇaṃ, no vā, kathaṃ vā ettha hotī”ti?}
\hspace{0pt}\end{leftcolumn*}

\begin{rightcolumn}\PaliColumn{Now, bhikkhus, do ageing and death have birth as condition or not, or how do you take it in this case?”}
\hspace{0pt}
\end{rightcolumn}
\end{samepage}
\begin{samepage}
\begin{leftcolumn*}
\EnglishColumn{“jātipaccayā, bhante, jarāmaraṇaṃ;}
\hspace{0pt}\end{leftcolumn*}

\begin{rightcolumn}\PaliColumn{“Ageing and death have birth as condition, venerable sir.}
\hspace{0pt}
\end{rightcolumn}
\end{samepage}
\begin{samepage}
\begin{leftcolumn*}
\EnglishColumn{evaṃ no ettha hoti - jātipaccayā jarāmaraṇan”ti.}
\hspace{0pt}\end{leftcolumn*}

\begin{rightcolumn}\PaliColumn{Thus we take it in this case: ‘With birth as condition, ageing and death.’”}
\hspace{0pt}
\end{rightcolumn}
\end{samepage}
\vskip 0.2in
\begin{samepage}
\begin{leftcolumn*}
\EnglishColumn{“bhavapaccayā jātīti iti kho panetaṃ vuttaṃ;}
\hspace{0pt}\end{leftcolumn*}

\begin{rightcolumn}\PaliColumn{“‘With being as condition, birth’: so it was said.}
\hspace{0pt}
\end{rightcolumn}
\end{samepage}
\begin{samepage}
\begin{leftcolumn*}
\EnglishColumn{bhavapaccayā nu kho, bhikkhave, jāti, no vā, kathaṃ vā ettha hotī”ti?}
\hspace{0pt}\end{leftcolumn*}

\begin{rightcolumn}\PaliColumn{Now, bhikkhus, does birth have being as condition or not, or how do you take it in this case?”}
\hspace{0pt}
\end{rightcolumn}
\end{samepage}
\begin{samepage}
\begin{leftcolumn*}
\EnglishColumn{“bhavapaccayā, bhante, jāti;}
\hspace{0pt}\end{leftcolumn*}

\begin{rightcolumn}\PaliColumn{“Birth has being as condition, venerable sir.}
\hspace{0pt}
\end{rightcolumn}
\end{samepage}
\begin{samepage}
\begin{leftcolumn*}
\EnglishColumn{evaṃ no ettha hoti - bhavapaccayā jātī”ti.}
\hspace{0pt}\end{leftcolumn*}

\begin{rightcolumn}\PaliColumn{Thus we take it in this case: ‘With being as condition, birth.’”}
\hspace{0pt}
\end{rightcolumn}
\end{samepage}
\vskip 0.2in
\begin{samepage}
\begin{leftcolumn*}
\EnglishColumn{“upādānapaccayā bhavoti iti kho panetaṃ vuttaṃ;}
\hspace{0pt}\end{leftcolumn*}

\begin{rightcolumn}\PaliColumn{“‘With clinging as condition, being’: so it was said.}
\hspace{0pt}
\end{rightcolumn}
\end{samepage}
\begin{samepage}
\begin{leftcolumn*}
\EnglishColumn{upādānapaccayā nu kho, bhikkhave, bhavo, no vā, kathaṃ vā ettha hotī”ti?}
\hspace{0pt}\end{leftcolumn*}

\begin{rightcolumn}\PaliColumn{Now, bhikkhus, does being have clinging as condition or not, or how do you take it in this case?”}
\hspace{0pt}
\end{rightcolumn}
\end{samepage}
\begin{samepage}
\begin{leftcolumn*}
\EnglishColumn{“upādānapaccayā, bhante, bhavo;}
\hspace{0pt}\end{leftcolumn*}

\begin{rightcolumn}\PaliColumn{“Being has clinging as condition, venerable sir.}
\hspace{0pt}
\end{rightcolumn}
\end{samepage}
\begin{samepage}
\begin{leftcolumn*}
\EnglishColumn{evaṃ no ettha hoti - upādānapaccayā bhavo”ti.}
\hspace{0pt}\end{leftcolumn*}

\begin{rightcolumn}\PaliColumn{Thus we take it in this case: ‘With clinging as condition, being.’”}
\hspace{0pt}
\end{rightcolumn}
\end{samepage}
\vskip 0.2in
\begin{samepage}
\begin{leftcolumn*}
\EnglishColumn{“taṇhāpaccayā upādānanti iti kho panetaṃ vuttaṃ;}
\hspace{0pt}\end{leftcolumn*}

\begin{rightcolumn}\PaliColumn{“‘With craving as condition, clinging’: so it was said.}
\hspace{0pt}
\end{rightcolumn}
\end{samepage}
\begin{samepage}
\begin{leftcolumn*}
\EnglishColumn{taṇhāpaccayā nu kho, bhikkhave, upādānaṃ, no vā, kathaṃ vā ettha hotī”ti?}
\hspace{0pt}\end{leftcolumn*}

\begin{rightcolumn}\PaliColumn{Now, bhikkhus, does clinging have craving as condition or not, or how do you take it in this case?”}
\hspace{0pt}
\end{rightcolumn}
\end{samepage}
\begin{samepage}
\begin{leftcolumn*}
\EnglishColumn{“taṇhāpaccayā, bhante, upādānaṃ;}
\hspace{0pt}\end{leftcolumn*}

\begin{rightcolumn}\PaliColumn{“Clinging has craving as condition, venerable sir.}
\hspace{0pt}
\end{rightcolumn}
\end{samepage}
\begin{samepage}
\begin{leftcolumn*}
\EnglishColumn{evaṃ no ettha hoti - taṇhāpaccayā upādānan”ti.}
\hspace{0pt}\end{leftcolumn*}

\begin{rightcolumn}\PaliColumn{Thus we take it in this case: ‘With craving as condition, clinging.’”}
\hspace{0pt}
\end{rightcolumn}
\end{samepage}
\vskip 0.2in
\begin{samepage}
\begin{leftcolumn*}
\EnglishColumn{“vedanāpaccayā taṇhāti iti kho panetaṃ vuttaṃ;}
\hspace{0pt}\end{leftcolumn*}

\begin{rightcolumn}\PaliColumn{“‘With feeling as condition, craving’: so it was said.}
\hspace{0pt}
\end{rightcolumn}
\end{samepage}
\begin{samepage}
\begin{leftcolumn*}
\EnglishColumn{vedanāpaccayā nu kho, bhikkhave, taṇhā, no vā, kathaṃ vā ettha hotī”ti?}
\hspace{0pt}\end{leftcolumn*}

\begin{rightcolumn}\PaliColumn{Now, bhikkhus, does craving have feeling as condition or not, or how do you take it in this case?”}
\hspace{0pt}
\end{rightcolumn}
\end{samepage}
\begin{samepage}
\begin{leftcolumn*}
\EnglishColumn{“vedanāpaccayā, bhante, taṇhā;}
\hspace{0pt}\end{leftcolumn*}

\begin{rightcolumn}\PaliColumn{“Craving has feeling as condition, venerable sir.}
\hspace{0pt}
\end{rightcolumn}
\end{samepage}
\begin{samepage}
\begin{leftcolumn*}
\EnglishColumn{evaṃ no ettha hoti - vedanāpaccayā taṇhā”ti.}
\hspace{0pt}\end{leftcolumn*}

\begin{rightcolumn}\PaliColumn{Thus we take it in this case: ‘With feeling as condition, craving.’”}
\hspace{0pt}
\end{rightcolumn}
\end{samepage}
\vskip 0.2in
\begin{samepage}
\begin{leftcolumn*}
\EnglishColumn{“phassapaccayā vedanāti iti kho panetaṃ vuttaṃ;}
\hspace{0pt}\end{leftcolumn*}

\begin{rightcolumn}\PaliColumn{“‘With contact as condition, feeling’: so it was said.}
\hspace{0pt}
\end{rightcolumn}
\end{samepage}
\begin{samepage}
\begin{leftcolumn*}
\EnglishColumn{phassapaccayā nu kho, bhikkhave, vedanā, no vā, kathaṃ vā ettha hotī”ti?}
\hspace{0pt}\end{leftcolumn*}

\begin{rightcolumn}\PaliColumn{Now, bhikkhus, does feeling have contact as condition or not, or how do you take it in this case?”}
\hspace{0pt}
\end{rightcolumn}
\end{samepage}
\begin{samepage}
\begin{leftcolumn*}
\EnglishColumn{“phassapaccayā, bhante, vedanā;}
\hspace{0pt}\end{leftcolumn*}

\begin{rightcolumn}\PaliColumn{“Feeling has contact as condition, venerable sir.}
\hspace{0pt}
\end{rightcolumn}
\end{samepage}
\begin{samepage}
\begin{leftcolumn*}
\EnglishColumn{evaṃ no ettha hoti - phassapaccayā vedanā”ti.}
\hspace{0pt}\end{leftcolumn*}

\begin{rightcolumn}\PaliColumn{Thus we take it in this case: ‘With contact as condition, feeling.’”}
\hspace{0pt}
\end{rightcolumn}
\end{samepage}
\vskip 0.2in
\begin{samepage}
\begin{leftcolumn*}
\EnglishColumn{“saḷāyatanapaccayā phassoti iti kho panetaṃ vuttaṃ;}
\hspace{0pt}\end{leftcolumn*}

\begin{rightcolumn}\PaliColumn{“‘With the sixfold base as condition, contact’: so it was said.}
\hspace{0pt}
\end{rightcolumn}
\end{samepage}
\begin{samepage}
\begin{leftcolumn*}
\EnglishColumn{saḷāyatanapaccayā nu kho, bhikkhave, phasso, no vā, kathaṃ vā ettha hotī”ti?}
\hspace{0pt}\end{leftcolumn*}

\begin{rightcolumn}\PaliColumn{Now, bhikkhus, does contact have the sixfold base as condition or not, or how do you take it in this case?”}
\hspace{0pt}
\end{rightcolumn}
\end{samepage}
\begin{samepage}
\begin{leftcolumn*}
\EnglishColumn{“saḷāyatanapaccayā, bhante, phasso;}
\hspace{0pt}\end{leftcolumn*}

\begin{rightcolumn}\PaliColumn{“Contact has the sixfold base as condition, venerable sir.}
\hspace{0pt}
\end{rightcolumn}
\end{samepage}
\begin{samepage}
\begin{leftcolumn*}
\EnglishColumn{evaṃ no ettha hoti - saḷāyatanapaccayā phasso”ti.}
\hspace{0pt}\end{leftcolumn*}

\begin{rightcolumn}\PaliColumn{Thus we take it in this case: ‘With the sixfold base as condition, contact.’”}
\hspace{0pt}
\end{rightcolumn}
\end{samepage}
\vskip 0.2in
\begin{samepage}
\begin{leftcolumn*}
\EnglishColumn{“nāmarūpapaccayā saḷāyatananti iti kho panetaṃ vuttaṃ;}
\hspace{0pt}\end{leftcolumn*}

\begin{rightcolumn}\PaliColumn{“‘With name and form as condition, the sixfold base’: so it was said.}
\hspace{0pt}
\end{rightcolumn}
\end{samepage}
\begin{samepage}
\begin{leftcolumn*}
\EnglishColumn{nāmarūpapaccayā nu kho, bhikkhave, saḷāyatanaṃ, no vā, kathaṃ vā ettha hotī”ti?}
\hspace{0pt}\end{leftcolumn*}

\begin{rightcolumn}\PaliColumn{Now, bhikkhus, does the sixfold base have name and form as condition or not, or how do you take it in this case?”}
\hspace{0pt}
\end{rightcolumn}
\end{samepage}
\begin{samepage}
\begin{leftcolumn*}
\EnglishColumn{“nāmarūpapaccayā, bhante, saḷāyatanaṃ;}
\hspace{0pt}\end{leftcolumn*}

\begin{rightcolumn}\PaliColumn{“The sixfold base has name and form as condition, venerable sir.}
\hspace{0pt}
\end{rightcolumn}
\end{samepage}
\begin{samepage}
\begin{leftcolumn*}
\EnglishColumn{evaṃ no ettha hoti - nāmarūpapaccayā saḷāyatanan”ti.}
\hspace{0pt}\end{leftcolumn*}

\begin{rightcolumn}\PaliColumn{Thus we take it in this case: ‘With name and form as condition, the sixfold base.’”}
\hspace{0pt}
\end{rightcolumn}
\end{samepage}
\vskip 0.2in
\begin{samepage}
\begin{leftcolumn*}
\EnglishColumn{“viññāṇapaccayā nāmarūpanti iti kho panetaṃ vuttaṃ;}
\hspace{0pt}\end{leftcolumn*}

\begin{rightcolumn}\PaliColumn{“‘With consciousness as condition, name and form’: so it was said.}
\hspace{0pt}
\end{rightcolumn}
\end{samepage}
\begin{samepage}
\begin{leftcolumn*}
\EnglishColumn{viññāṇapaccayā nu kho, bhikkhave, nāmarūpaṃ, no vā, kathaṃ vā ettha hotī”ti?}
\hspace{0pt}\end{leftcolumn*}

\begin{rightcolumn}\PaliColumn{Now, bhikkhus, does name and form have consciousness as condition or not, or how do you take it in this case?”}
\hspace{0pt}
\end{rightcolumn}
\end{samepage}
\begin{samepage}
\begin{leftcolumn*}
\EnglishColumn{“viññāṇapaccayā, bhante, nāmarūpaṃ;}
\hspace{0pt}\end{leftcolumn*}

\begin{rightcolumn}\PaliColumn{“Name and form has consciousness as condition, venerable sir.}
\hspace{0pt}
\end{rightcolumn}
\end{samepage}
\begin{samepage}
\begin{leftcolumn*}
\EnglishColumn{evaṃ no ettha hoti - viññāṇapaccayā nāmarūpan”ti.}
\hspace{0pt}\end{leftcolumn*}

\begin{rightcolumn}\PaliColumn{Thus we take it in this case: ‘With consciousness as condition, name and form.’”}
\hspace{0pt}
\end{rightcolumn}
\end{samepage}
\vskip 0.2in
\begin{samepage}
\begin{leftcolumn*}
\EnglishColumn{“saṅkhārapaccayā viññāṇanti iti kho panetaṃ vuttaṃ;}
\hspace{0pt}\end{leftcolumn*}

\begin{rightcolumn}\PaliColumn{“‘With formations as condition, consciousness’: so it was said.}
\hspace{0pt}
\end{rightcolumn}
\end{samepage}
\begin{samepage}
\begin{leftcolumn*}
\EnglishColumn{saṅkhārapaccayā nu kho, bhikkhave, viññāṇaṃ, no vā, kathaṃ vā ettha hotī”ti?}
\hspace{0pt}\end{leftcolumn*}

\begin{rightcolumn}\PaliColumn{Now, bhikkhus, does consciousness have formations as condition or not, or how do you take it in this case?”}
\hspace{0pt}
\end{rightcolumn}
\end{samepage}
\begin{samepage}
\begin{leftcolumn*}
\EnglishColumn{“saṅkhārapaccayā, bhante, viññāṇaṃ;}
\hspace{0pt}\end{leftcolumn*}

\begin{rightcolumn}\PaliColumn{“Consciousness has formations as condition, venerable sir.}
\hspace{0pt}
\end{rightcolumn}
\end{samepage}
\begin{samepage}
\begin{leftcolumn*}
\EnglishColumn{evaṃ no ettha hoti - saṅkhārapaccayā viññāṇan”ti.}
\hspace{0pt}\end{leftcolumn*}

\begin{rightcolumn}\PaliColumn{Thus we take it in this case: ‘With formations as condition, consciousness.’”}
\hspace{0pt}
\end{rightcolumn}
\end{samepage}
\vskip 0.2in
\begin{samepage}
\begin{leftcolumn*}
\EnglishColumn{“avijjāpaccayā saṅkhārāti iti kho panetaṃ vuttaṃ;}
\hspace{0pt}\end{leftcolumn*}

\begin{rightcolumn}\PaliColumn{“‘With ignorance as condition, formations’: so it was said.}
\hspace{0pt}
\end{rightcolumn}
\end{samepage}
\begin{samepage}
\begin{leftcolumn*}
\EnglishColumn{avijjāpaccayā nu kho, bhikkhave, saṅkhārā, no vā, kathaṃ vā ettha hotī”ti?}
\hspace{0pt}\end{leftcolumn*}

\begin{rightcolumn}\PaliColumn{Now, bhikkhus, do formations have ignorance as condition or not, or how do you take it in this case?”}
\hspace{0pt}
\end{rightcolumn}
\end{samepage}
\begin{samepage}
\begin{leftcolumn*}
\EnglishColumn{“avijjāpaccayā, bhante, saṅkhārā;}
\hspace{0pt}\end{leftcolumn*}

\begin{rightcolumn}\PaliColumn{“Formations have ignorance as condition, venerable sir.}
\hspace{0pt}
\end{rightcolumn}
\end{samepage}
\begin{samepage}
\begin{leftcolumn*}
\EnglishColumn{evaṃ no ettha hoti - avijjāpaccayā saṅkhārā”ti.}
\hspace{0pt}\end{leftcolumn*}

\begin{rightcolumn}\PaliColumn{Thus we take it in this case: ‘With ignorance as condition, formations.’”}
\hspace{0pt}
\end{rightcolumn}
\end{samepage}
\vskip 0.2in
\begin{samepage}
\begin{leftcolumn*}
\EnglishColumn{“sādhu, bhikkhave.}
\hspace{0pt}\end{leftcolumn*}

\begin{rightcolumn}\PaliColumn{“Good, bhikkhus.}
\hspace{0pt}
\end{rightcolumn}
\end{samepage}
\begin{samepage}
\begin{leftcolumn*}
\EnglishColumn{iti kho, bhikkhave, tumhepi evaṃ vadetha, ahampi evaṃ vadāmi -}
\hspace{0pt}\end{leftcolumn*}

\begin{rightcolumn}\PaliColumn{So you say thus, and I also say thus:}
\hspace{0pt}
\end{rightcolumn}
\end{samepage}
\begin{samepage}
\begin{leftcolumn*}
\EnglishColumn{imasmiṃ sati idaṃ hoti, imassuppādā idaṃ uppajjati,}
\hspace{0pt}\end{leftcolumn*}

\begin{rightcolumn}\PaliColumn{‘When this exists, that comes to be; with the arising of this, that arises.’}
\hspace{0pt}
\end{rightcolumn}
\end{samepage}
\begin{samepage}
\begin{leftcolumn*}
\EnglishColumn{yadidaṃ - avijjāpaccayā saṅkhārā,}
\hspace{0pt}\end{leftcolumn*}

\begin{rightcolumn}\PaliColumn{That is, with ignorance as condition, formations [come to be];}
\hspace{0pt}
\end{rightcolumn}
\end{samepage}
\begin{samepage}
\begin{leftcolumn*}
\EnglishColumn{saṅkhārapaccayā viññāṇaṃ,}
\hspace{0pt}\end{leftcolumn*}

\begin{rightcolumn}\PaliColumn{with formations as condition, consciousness;}
\hspace{0pt}
\end{rightcolumn}
\end{samepage}
\begin{samepage}
\begin{leftcolumn*}
\EnglishColumn{viññāṇapaccayā nāmarūpaṃ,}
\hspace{0pt}\end{leftcolumn*}

\begin{rightcolumn}\PaliColumn{with consciousness as condition, name and form;}
\hspace{0pt}
\end{rightcolumn}
\end{samepage}
\begin{samepage}
\begin{leftcolumn*}
\EnglishColumn{nāmarūpapaccayā saḷāyatanaṃ,}
\hspace{0pt}\end{leftcolumn*}

\begin{rightcolumn}\PaliColumn{with name and form as condition, the sixfold base;}
\hspace{0pt}
\end{rightcolumn}
\end{samepage}
\begin{samepage}
\begin{leftcolumn*}
\EnglishColumn{saḷāyatanapaccayā phasso,}
\hspace{0pt}\end{leftcolumn*}

\begin{rightcolumn}\PaliColumn{with the sixfold base as condition, contact;}
\hspace{0pt}
\end{rightcolumn}
\end{samepage}
\begin{samepage}
\begin{leftcolumn*}
\EnglishColumn{phassapaccayā vedanā,}
\hspace{0pt}\end{leftcolumn*}

\begin{rightcolumn}\PaliColumn{with contact as condition, feeling;}
\hspace{0pt}
\end{rightcolumn}
\end{samepage}
\begin{samepage}
\begin{leftcolumn*}
\EnglishColumn{vedanāpaccayā taṇhā,}
\hspace{0pt}\end{leftcolumn*}

\begin{rightcolumn}\PaliColumn{with feeling as condition, craving;}
\hspace{0pt}
\end{rightcolumn}
\end{samepage}
\begin{samepage}
\begin{leftcolumn*}
\EnglishColumn{taṇhāpaccayā upādānaṃ,}
\hspace{0pt}\end{leftcolumn*}

\begin{rightcolumn}\PaliColumn{with craving as condition, clinging;}
\hspace{0pt}
\end{rightcolumn}
\end{samepage}
\begin{samepage}
\begin{leftcolumn*}
\EnglishColumn{upādānapaccayā bhavo,}
\hspace{0pt}\end{leftcolumn*}

\begin{rightcolumn}\PaliColumn{with clinging as condition, being;}
\hspace{0pt}
\end{rightcolumn}
\end{samepage}
\begin{samepage}
\begin{leftcolumn*}
\EnglishColumn{bhavapaccayā jāti,}
\hspace{0pt}\end{leftcolumn*}

\begin{rightcolumn}\PaliColumn{with being as condition, birth;}
\hspace{0pt}
\end{rightcolumn}
\end{samepage}
\begin{samepage}
\begin{leftcolumn*}
\EnglishColumn{jātipaccayā jarāmaraṇaṃ sokaparidevadukkhadomanassupāyāsā sambhavanti.}
\hspace{0pt}\end{leftcolumn*}

\begin{rightcolumn}\PaliColumn{with birth as condition, ageing and death, sorrow, lamentation, pain, grief, and despair come to be.}
\hspace{0pt}
\end{rightcolumn}
\end{samepage}
\begin{samepage}
\begin{leftcolumn*}
\EnglishColumn{evametassa kevalassa dukkhakkhandhassa samudayo hoti.}
\hspace{0pt}\end{leftcolumn*}

\begin{rightcolumn}\PaliColumn{Such is the origin of this whole mass of suffering.}
\hspace{0pt}
\end{rightcolumn}
\end{samepage}
\vskip 0.2in
\begin{samepage}
\begin{leftcolumn*}
\EnglishColumn{“avijjāyatveva asesavirāganirodhā saṅkhāranirodho,}
\hspace{0pt}\end{leftcolumn*}

\begin{rightcolumn}\PaliColumn{“But with the remainderless fading away and cessation of ignorance comes cessation of formations;}
\hspace{0pt}
\end{rightcolumn}
\end{samepage}
\begin{samepage}
\begin{leftcolumn*}
\EnglishColumn{saṅkhāranirodhā viññāṇanirodho,}
\hspace{0pt}\end{leftcolumn*}

\begin{rightcolumn}\PaliColumn{with the cessation of formations, cessation of consciousness;}
\hspace{0pt}
\end{rightcolumn}
\end{samepage}
\begin{samepage}
\begin{leftcolumn*}
\EnglishColumn{viññāṇanirodhā nāmarūpanirodho,}
\hspace{0pt}\end{leftcolumn*}

\begin{rightcolumn}\PaliColumn{with the cessation of consciousness, cessation of name and form;}
\hspace{0pt}
\end{rightcolumn}
\end{samepage}
\begin{samepage}
\begin{leftcolumn*}
\EnglishColumn{nāmarūpanirodhā saḷāyatananirodho,}
\hspace{0pt}\end{leftcolumn*}

\begin{rightcolumn}\PaliColumn{with the cessation of name and form, cessation of the sixfold base;}
\hspace{0pt}
\end{rightcolumn}
\end{samepage}
\begin{samepage}
\begin{leftcolumn*}
\EnglishColumn{saḷāyatananirodhā phassanirodho,}
\hspace{0pt}\end{leftcolumn*}

\begin{rightcolumn}\PaliColumn{with the cessation of the sixfold base, cessation of contact;}
\hspace{0pt}
\end{rightcolumn}
\end{samepage}
\begin{samepage}
\begin{leftcolumn*}
\EnglishColumn{phassanirodhā vedanānirodho,}
\hspace{0pt}\end{leftcolumn*}

\begin{rightcolumn}\PaliColumn{with the cessation of contact, cessation of feeling;}
\hspace{0pt}
\end{rightcolumn}
\end{samepage}
\begin{samepage}
\begin{leftcolumn*}
\EnglishColumn{vedanānirodhā taṇhānirodho,}
\hspace{0pt}\end{leftcolumn*}

\begin{rightcolumn}\PaliColumn{with the cessation of feeling, cessation of craving;}
\hspace{0pt}
\end{rightcolumn}
\end{samepage}
\begin{samepage}
\begin{leftcolumn*}
\EnglishColumn{taṇhānirodhā upādānanirodho,}
\hspace{0pt}\end{leftcolumn*}

\begin{rightcolumn}\PaliColumn{with the cessation of craving, cessation of clinging;}
\hspace{0pt}
\end{rightcolumn}
\end{samepage}
\begin{samepage}
\begin{leftcolumn*}
\EnglishColumn{upādānanirodhā bhavanirodho,}
\hspace{0pt}\end{leftcolumn*}

\begin{rightcolumn}\PaliColumn{with the cessation of clinging, cessation of being;}
\hspace{0pt}
\end{rightcolumn}
\end{samepage}
\begin{samepage}
\begin{leftcolumn*}
\EnglishColumn{bhavanirodhā jātinirodho,}
\hspace{0pt}\end{leftcolumn*}

\begin{rightcolumn}\PaliColumn{with the cessation of being, cessation of birth;}
\hspace{0pt}
\end{rightcolumn}
\end{samepage}
\begin{samepage}
\begin{leftcolumn*}
\EnglishColumn{jātinirodhā jarāmaraṇaṃ sokaparidevadukkhadomanassupāyāsā nirujjhanti.}
\hspace{0pt}\end{leftcolumn*}

\begin{rightcolumn}\PaliColumn{with the cessation of birth, ageing and death, sorrow, lamentation, pain, grief, and despair cease.}
\hspace{0pt}
\end{rightcolumn}
\end{samepage}
\begin{samepage}
\begin{leftcolumn*}
\EnglishColumn{evametassa kevalassa dukkhakkhandhassa nirodho hoti.}
\hspace{0pt}\end{leftcolumn*}

\begin{rightcolumn}\PaliColumn{Such is the cessation of this whole mass of suffering.}
\hspace{0pt}
\end{rightcolumn}
\end{samepage}
\vskip 0.2in
\begin{samepage}
\begin{leftcolumn*}
\EnglishColumn{“jātinirodhā jarāmaraṇanirodhoti iti kho panetaṃ vuttaṃ;}
\hspace{0pt}\end{leftcolumn*}

\begin{rightcolumn}\PaliColumn{“‘With the cessation of birth, cessation of ageing and death’: so it was said.}
\hspace{0pt}
\end{rightcolumn}
\end{samepage}
\begin{samepage}
\begin{leftcolumn*}
\EnglishColumn{jātinirodhā nu kho, bhikkhave, jarāmaraṇanirodho, no vā, kathaṃ vā ettha hotī”ti?}
\hspace{0pt}\end{leftcolumn*}

\begin{rightcolumn}\PaliColumn{Now, bhikkhus, do ageing and death cease with the cessation of birth or not, or how do you take it in this case?”}
\hspace{0pt}
\end{rightcolumn}
\end{samepage}
\begin{samepage}
\begin{leftcolumn*}
\EnglishColumn{“jātinirodhā, bhante, jarāmaraṇanirodho;}
\hspace{0pt}\end{leftcolumn*}

\begin{rightcolumn}\PaliColumn{“Ageing and death cease with the cessation of birth, venerable sir.}
\hspace{0pt}
\end{rightcolumn}
\end{samepage}
\begin{samepage}
\begin{leftcolumn*}
\EnglishColumn{evaṃ no ettha hoti - jātinirodhā jarāmaraṇanirodho”ti.}
\hspace{0pt}\end{leftcolumn*}

\begin{rightcolumn}\PaliColumn{Thus we take it in this case: ‘With the cessation of birth, cessation of ageing and death.’”}
\hspace{0pt}
\end{rightcolumn}
\end{samepage}
\vskip 0.2in
\begin{samepage}
\begin{leftcolumn*}
\EnglishColumn{“bhavanirodhā jātinirodhoti iti kho panetaṃ vuttaṃ;}
\hspace{0pt}\end{leftcolumn*}

\begin{rightcolumn}\PaliColumn{“‘With the cessation of being, cessation of birth’: so it was said.}
\hspace{0pt}
\end{rightcolumn}
\end{samepage}
\begin{samepage}
\begin{leftcolumn*}
\EnglishColumn{bhavanirodhā nu kho, bhikkhave, jātinirodho, no vā, kathaṃ vā ettha hotī”ti?}
\hspace{0pt}\end{leftcolumn*}

\begin{rightcolumn}\PaliColumn{Now, bhikkhus, does birth cease with the cessation of being or not, or how do you take it in this case?”}
\hspace{0pt}
\end{rightcolumn}
\end{samepage}
\begin{samepage}
\begin{leftcolumn*}
\EnglishColumn{“bhavanirodhā, bhante, jātinirodho;}
\hspace{0pt}\end{leftcolumn*}

\begin{rightcolumn}\PaliColumn{“Birth ceases with the cessation of being, venerable sir.}
\hspace{0pt}
\end{rightcolumn}
\end{samepage}
\begin{samepage}
\begin{leftcolumn*}
\EnglishColumn{evaṃ no ettha hoti - bhavanirodhā jātinirodho”ti.}
\hspace{0pt}\end{leftcolumn*}

\begin{rightcolumn}\PaliColumn{Thus we take it in this case: ‘With the cessation of being, cessation of birth.’”}
\hspace{0pt}
\end{rightcolumn}
\end{samepage}
\vskip 0.2in
\begin{samepage}
\begin{leftcolumn*}
\EnglishColumn{“upādānanirodhā bhavanirodhoti iti kho panetaṃ vuttaṃ;}
\hspace{0pt}\end{leftcolumn*}

\begin{rightcolumn}\PaliColumn{‘With the cessation of clinging, cessation of being’: so it was said.}
\hspace{0pt}
\end{rightcolumn}
\end{samepage}
\begin{samepage}
\begin{leftcolumn*}
\EnglishColumn{upādānanirodhā nu kho, bhikkhave, bhavanirodho, no vā, kathaṃ vā ettha hotī”ti?}
\hspace{0pt}\end{leftcolumn*}

\begin{rightcolumn}\PaliColumn{Now, bhikkhus, does being cease with the cessation of clinging or not, or how do you take it in this case?”}
\hspace{0pt}
\end{rightcolumn}
\end{samepage}
\begin{samepage}
\begin{leftcolumn*}
\EnglishColumn{“upādānanirodhā, bhante, bhavanirodho;}
\hspace{0pt}\end{leftcolumn*}

\begin{rightcolumn}\PaliColumn{“Being ceases with the cessation of clinging, venerable sir.}
\hspace{0pt}
\end{rightcolumn}
\end{samepage}
\begin{samepage}
\begin{leftcolumn*}
\EnglishColumn{evaṃ no ettha hoti - upādānanirodhā bhavanirodho”ti.}
\hspace{0pt}\end{leftcolumn*}

\begin{rightcolumn}\PaliColumn{Thus we take it in this case: ‘With the cessation of clinging, cessation of being.’”}
\hspace{0pt}
\end{rightcolumn}
\end{samepage}
\vskip 0.2in
\begin{samepage}
\begin{leftcolumn*}
\EnglishColumn{“taṇhānirodhā upādānanirodhoti iti kho panetaṃ vuttaṃ;}
\hspace{0pt}\end{leftcolumn*}

\begin{rightcolumn}\PaliColumn{‘With the cessation of craving, cessation of clinging’: so it was said.}
\hspace{0pt}
\end{rightcolumn}
\end{samepage}
\begin{samepage}
\begin{leftcolumn*}
\EnglishColumn{taṇhānirodhā nu kho, bhikkhave, upādānanirodho, no vā, kathaṃ vā ettha hotī”ti?}
\hspace{0pt}\end{leftcolumn*}

\begin{rightcolumn}\PaliColumn{Now, bhikkhus, does clinging cease with the cessation of craving or not, or how do you take it in this case?”}
\hspace{0pt}
\end{rightcolumn}
\end{samepage}
\begin{samepage}
\begin{leftcolumn*}
\EnglishColumn{“taṇhānirodhā, bhante, upādānanirodho;}
\hspace{0pt}\end{leftcolumn*}

\begin{rightcolumn}\PaliColumn{“Clinging ceases with the cessation of craving, venerable sir.}
\hspace{0pt}
\end{rightcolumn}
\end{samepage}
\begin{samepage}
\begin{leftcolumn*}
\EnglishColumn{evaṃ no ettha hoti - taṇhānirodhā upādānanirodho”ti.}
\hspace{0pt}\end{leftcolumn*}

\begin{rightcolumn}\PaliColumn{Thus we take it in this case: ‘With the cessation of craving, cessation of clinging.’”}
\hspace{0pt}
\end{rightcolumn}
\end{samepage}
\vskip 0.2in
\begin{samepage}
\begin{leftcolumn*}
\EnglishColumn{“vedanānirodhā taṇhānirodhoti iti kho panetaṃ vuttaṃ;}
\hspace{0pt}\end{leftcolumn*}

\begin{rightcolumn}\PaliColumn{‘With the cessation of feeling, cessation of craving’: so it was said.}
\hspace{0pt}
\end{rightcolumn}
\end{samepage}
\begin{samepage}
\begin{leftcolumn*}
\EnglishColumn{vedanānirodhā nu kho, bhikkhave, taṇhānirodho, no vā, kathaṃ vā ettha hotī”ti?}
\hspace{0pt}\end{leftcolumn*}

\begin{rightcolumn}\PaliColumn{Now, bhikkhus, does craving cease with the cessation of feeling or not, or how do you take it in this case?”}
\hspace{0pt}
\end{rightcolumn}
\end{samepage}
\begin{samepage}
\begin{leftcolumn*}
\EnglishColumn{“vedanānirodhā, bhante, taṇhānirodho;}
\hspace{0pt}\end{leftcolumn*}

\begin{rightcolumn}\PaliColumn{“Craving ceases with the cessation of feeling, venerable sir.}
\hspace{0pt}
\end{rightcolumn}
\end{samepage}
\begin{samepage}
\begin{leftcolumn*}
\EnglishColumn{evaṃ no ettha hoti - vedanānirodhā taṇhānirodho”ti.}
\hspace{0pt}\end{leftcolumn*}

\begin{rightcolumn}\PaliColumn{Thus we take it in this case: ‘With the cessation of feeling, cessation of craving.’”}
\hspace{0pt}
\end{rightcolumn}
\end{samepage}
\vskip 0.2in
\begin{samepage}
\begin{leftcolumn*}
\EnglishColumn{“phassanirodhā vedanānirodhoti iti kho panetaṃ vuttaṃ;}
\hspace{0pt}\end{leftcolumn*}

\begin{rightcolumn}\PaliColumn{‘With the cessation of contact, cessation of feeling’: so it was said.}
\hspace{0pt}
\end{rightcolumn}
\end{samepage}
\begin{samepage}
\begin{leftcolumn*}
\EnglishColumn{phassanirodhā nu kho, bhikkhave, vedanānirodho, no vā, kathaṃ vā ettha hotī”ti?}
\hspace{0pt}\end{leftcolumn*}

\begin{rightcolumn}\PaliColumn{Now, bhikkhus, does feeling cease with the cessation of contact or not, or how do you take it in this case?”}
\hspace{0pt}
\end{rightcolumn}
\end{samepage}
\begin{samepage}
\begin{leftcolumn*}
\EnglishColumn{“phassanirodhā, bhante, vedanānirodho;}
\hspace{0pt}\end{leftcolumn*}

\begin{rightcolumn}\PaliColumn{“Feeling ceases with the cessation of contact, venerable sir.}
\hspace{0pt}
\end{rightcolumn}
\end{samepage}
\begin{samepage}
\begin{leftcolumn*}
\EnglishColumn{evaṃ no ettha hoti - phassanirodhā vedanānirodho”ti.}
\hspace{0pt}\end{leftcolumn*}

\begin{rightcolumn}\PaliColumn{Thus we take it in this case: ‘With the cessation of contact, cessation of feeling.’”}
\hspace{0pt}
\end{rightcolumn}
\end{samepage}
\vskip 0.2in
\begin{samepage}
\begin{leftcolumn*}
\EnglishColumn{“saḷāyatananirodhā phassanirodhoti iti kho panetaṃ vuttaṃ;}
\hspace{0pt}\end{leftcolumn*}

\begin{rightcolumn}\PaliColumn{’With the cessation of the sixfold base, cessation of contact’: so it was said.}
\hspace{0pt}
\end{rightcolumn}
\end{samepage}
\begin{samepage}
\begin{leftcolumn*}
\EnglishColumn{saḷāyatananirodhā nu kho, bhikkhave, phassanirodho, no vā, kathaṃ vā ettha hotīti?}
\hspace{0pt}\end{leftcolumn*}

\begin{rightcolumn}\PaliColumn{Now, bhikkhus, does the sixfold base cease with the cessation of contact or not, or how do you take it in this case?”}
\hspace{0pt}
\end{rightcolumn}
\end{samepage}
\begin{samepage}
\begin{leftcolumn*}
\EnglishColumn{saḷāyatananirodhā, bhante, phassanirodho;}
\hspace{0pt}\end{leftcolumn*}

\begin{rightcolumn}\PaliColumn{“Contact ceases with the cessation of the sixfold base, venerable sir.}
\hspace{0pt}
\end{rightcolumn}
\end{samepage}
\begin{samepage}
\begin{leftcolumn*}
\EnglishColumn{evaṃ no ettha hoti - saḷāyatananirodhā phassanirodho”ti.}
\hspace{0pt}\end{leftcolumn*}

\begin{rightcolumn}\PaliColumn{Thus we take it in this case: ‘With the cessation of the sixfold base, cessation of contact.’”}
\hspace{0pt}
\end{rightcolumn}
\end{samepage}
\vskip 0.2in
\begin{samepage}
\begin{leftcolumn*}
\EnglishColumn{“nāmarūpanirodhā saḷāyatananirodhoti iti kho panetaṃ vuttaṃ;}
\hspace{0pt}\end{leftcolumn*}

\begin{rightcolumn}\PaliColumn{‘With the cessation of name and form, cessation of the sixfold base’: so it was said.}
\hspace{0pt}
\end{rightcolumn}
\end{samepage}
\begin{samepage}
\begin{leftcolumn*}
\EnglishColumn{nāmarūpanirodhā nu kho, bhikkhave, saḷāyatananirodho, no vā, kathaṃ vā ettha hotī”ti?}
\hspace{0pt}\end{leftcolumn*}

\begin{rightcolumn}\PaliColumn{Now, bhikkhus, does the sixfold base cease with the cessation of name and form or not, or how do you take it in this case?”}
\hspace{0pt}
\end{rightcolumn}
\end{samepage}
\begin{samepage}
\begin{leftcolumn*}
\EnglishColumn{“nāmarūpanirodhā, bhante, saḷāyatananirodho;}
\hspace{0pt}\end{leftcolumn*}

\begin{rightcolumn}\PaliColumn{“The sixfold base ceases with the cessation of name and form, venerable sir.}
\hspace{0pt}
\end{rightcolumn}
\end{samepage}
\begin{samepage}
\begin{leftcolumn*}
\EnglishColumn{evaṃ no ettha hoti - nāmarūpanirodhā saḷāyatananirodho”ti.}
\hspace{0pt}\end{leftcolumn*}

\begin{rightcolumn}\PaliColumn{Thus we take it in this case: ‘With the cessation of name and form, cessation of the sixfold base.’”}
\hspace{0pt}
\end{rightcolumn}
\end{samepage}
\vskip 0.2in
\begin{samepage}
\begin{leftcolumn*}
\EnglishColumn{“viññāṇanirodhā nāmarūpanirodhoti iti kho panetaṃ vuttaṃ;}
\hspace{0pt}\end{leftcolumn*}

\begin{rightcolumn}\PaliColumn{‘With the cessation of consciousness, cessation of name and form’: so it was said.}
\hspace{0pt}
\end{rightcolumn}
\end{samepage}
\begin{samepage}
\begin{leftcolumn*}
\EnglishColumn{viññāṇanirodhā nu kho, bhikkhave, nāmarūpanirodho, no vā, kathaṃ vā ettha hotī”ti?}
\hspace{0pt}\end{leftcolumn*}

\begin{rightcolumn}\PaliColumn{Now, bhikkhus, does name and form cease with the cessation of consciousness or not, or how do you take it in this case?”}
\hspace{0pt}
\end{rightcolumn}
\end{samepage}
\begin{samepage}
\begin{leftcolumn*}
\EnglishColumn{“viññāṇanirodhā, bhante, nāmarūpanirodho;}
\hspace{0pt}\end{leftcolumn*}

\begin{rightcolumn}\PaliColumn{“Name and form ceases with the cessation of consciousness, venerable sir.}
\hspace{0pt}
\end{rightcolumn}
\end{samepage}
\begin{samepage}
\begin{leftcolumn*}
\EnglishColumn{evaṃ no ettha hoti - viññāṇanirodhā nāmarūpanirodho”ti.}
\hspace{0pt}\end{leftcolumn*}

\begin{rightcolumn}\PaliColumn{Thus we take it in this case: ‘With the cessation of consciousness, cessation of name and form.’”}
\hspace{0pt}
\end{rightcolumn}
\end{samepage}
\vskip 0.2in
\begin{samepage}
\begin{leftcolumn*}
\EnglishColumn{“saṅkhāranirodhā viññāṇanirodhoti iti kho panetaṃ vuttaṃ;}
\hspace{0pt}\end{leftcolumn*}

\begin{rightcolumn}\PaliColumn{‘With the cessation of formations, cessation of consciousness’: so it was said.}
\hspace{0pt}
\end{rightcolumn}
\end{samepage}
\begin{samepage}
\begin{leftcolumn*}
\EnglishColumn{saṅkhāranirodhā nu kho, bhikkhave, viññāṇanirodho, no vā, kathaṃ vā ettha hotī”ti?}
\hspace{0pt}\end{leftcolumn*}

\begin{rightcolumn}\PaliColumn{Now, bhikkhus, does consciousness cease with the cessation of formations or not, or how do you take it in this case?”}
\hspace{0pt}
\end{rightcolumn}
\end{samepage}
\begin{samepage}
\begin{leftcolumn*}
\EnglishColumn{“saṅkhāranirodhā, bhante, viññāṇanirodho;}
\hspace{0pt}\end{leftcolumn*}

\begin{rightcolumn}\PaliColumn{“Consciousness ceases with the cessation of formations, venerable sir.}
\hspace{0pt}
\end{rightcolumn}
\end{samepage}
\begin{samepage}
\begin{leftcolumn*}
\EnglishColumn{evaṃ no ettha hoti - saṅkhāranirodhā viññāṇanirodho”ti.}
\hspace{0pt}\end{leftcolumn*}

\begin{rightcolumn}\PaliColumn{Thus we take it in this case: ‘With the cessation of formations, cessation of consciousness.’”}
\hspace{0pt}
\end{rightcolumn}
\end{samepage}
\vskip 0.2in
\begin{samepage}
\begin{leftcolumn*}
\EnglishColumn{“avijjānirodhā saṅkhāranirodhoti iti kho panetaṃ vuttaṃ;}
\hspace{0pt}\end{leftcolumn*}

\begin{rightcolumn}\PaliColumn{‘With the cessation of ignorance, cessation of formations’: so it was said.}
\hspace{0pt}
\end{rightcolumn}
\end{samepage}
\begin{samepage}
\begin{leftcolumn*}
\EnglishColumn{avijjānirodhā nu kho, bhikkhave, saṅkhāranirodho, no vā, kathaṃ vā ettha hotī”ti?}
\hspace{0pt}\end{leftcolumn*}

\begin{rightcolumn}\PaliColumn{Now, bhikkhus, do formations cease with the cessation of ignorance or not, or how do you take it in this case?”}
\hspace{0pt}
\end{rightcolumn}
\end{samepage}
\begin{samepage}
\begin{leftcolumn*}
\EnglishColumn{“avijjānirodhā, bhante, saṅkhāranirodho;}
\hspace{0pt}\end{leftcolumn*}

\begin{rightcolumn}\PaliColumn{“Formations cease with the cessation of ignorance, venerable sir.}
\hspace{0pt}
\end{rightcolumn}
\end{samepage}
\begin{samepage}
\begin{leftcolumn*}
\EnglishColumn{evaṃ no ettha hoti - avijjānirodhā saṅkhāranirodho”ti.}
\hspace{0pt}\end{leftcolumn*}

\begin{rightcolumn}\PaliColumn{Thus we take it in this case: ‘With the cessation of ignorance, cessation of formations.’”}
\hspace{0pt}
\end{rightcolumn}
\end{samepage}
\vskip 0.2in
\begin{samepage}
\begin{leftcolumn*}
\EnglishColumn{“sādhu, bhikkhave.}
\hspace{0pt}\end{leftcolumn*}

\begin{rightcolumn}\PaliColumn{“Good, bhikkhus.}
\hspace{0pt}
\end{rightcolumn}
\end{samepage}
\begin{samepage}
\begin{leftcolumn*}
\EnglishColumn{iti kho, bhikkhave, tumhepi evaṃ vadetha, ahampi evaṃ vadāmi -}
\hspace{0pt}\end{leftcolumn*}

\begin{rightcolumn}\PaliColumn{So you say thus, and I also say thus:}
\hspace{0pt}
\end{rightcolumn}
\end{samepage}
\begin{samepage}
\begin{leftcolumn*}
\EnglishColumn{imasmiṃ asati idaṃ na hoti, imassa nirodhā idaṃ nirujjhati,}
\hspace{0pt}\end{leftcolumn*}

\begin{rightcolumn}\PaliColumn{‘When this does not exist, that does not come to be; with the cessation of this, that ceases.’}
\hspace{0pt}
\end{rightcolumn}
\end{samepage}
\begin{samepage}
\begin{leftcolumn*}
\EnglishColumn{yadidaṃ - avijjānirodhā saṅkhāranirodho,}
\hspace{0pt}\end{leftcolumn*}

\begin{rightcolumn}\PaliColumn{That is, with the cessation of ignorance comes cessation of formations;}
\hspace{0pt}
\end{rightcolumn}
\end{samepage}
\begin{samepage}
\begin{leftcolumn*}
\EnglishColumn{saṅkhāranirodhā viññāṇanirodho,}
\hspace{0pt}\end{leftcolumn*}

\begin{rightcolumn}\PaliColumn{with the cessation of formations, cessation of consciousness;}
\hspace{0pt}
\end{rightcolumn}
\end{samepage}
\begin{samepage}
\begin{leftcolumn*}
\EnglishColumn{viññāṇanirodhā nāmarūpanirodho,}
\hspace{0pt}\end{leftcolumn*}

\begin{rightcolumn}\PaliColumn{with the cessation of consciousness, cessation of name and form;}
\hspace{0pt}
\end{rightcolumn}
\end{samepage}
\begin{samepage}
\begin{leftcolumn*}
\EnglishColumn{nāmarūpanirodhā saḷāyatananirodho,}
\hspace{0pt}\end{leftcolumn*}

\begin{rightcolumn}\PaliColumn{with the cessation of name and form, cessation of the sixfold base;}
\hspace{0pt}
\end{rightcolumn}
\end{samepage}
\begin{samepage}
\begin{leftcolumn*}
\EnglishColumn{saḷāyatananirodhā phassanirodho,}
\hspace{0pt}\end{leftcolumn*}

\begin{rightcolumn}\PaliColumn{with the cessation of the sixfold base, cessation of contact;}
\hspace{0pt}
\end{rightcolumn}
\end{samepage}
\begin{samepage}
\begin{leftcolumn*}
\EnglishColumn{phassanirodhā vedanānirodho,}
\hspace{0pt}\end{leftcolumn*}

\begin{rightcolumn}\PaliColumn{with the cessation of contact, cessation of feeling;}
\hspace{0pt}
\end{rightcolumn}
\end{samepage}
\begin{samepage}
\begin{leftcolumn*}
\EnglishColumn{vedanānirodhā taṇhānirodho,}
\hspace{0pt}\end{leftcolumn*}

\begin{rightcolumn}\PaliColumn{with the cessation of feeling, cessation of craving;}
\hspace{0pt}
\end{rightcolumn}
\end{samepage}
\begin{samepage}
\begin{leftcolumn*}
\EnglishColumn{taṇhānirodhā upādānanirodho,}
\hspace{0pt}\end{leftcolumn*}

\begin{rightcolumn}\PaliColumn{with the cessation of craving, cessation of clinging;}
\hspace{0pt}
\end{rightcolumn}
\end{samepage}
\begin{samepage}
\begin{leftcolumn*}
\EnglishColumn{upādānanirodhā bhavanirodho,}
\hspace{0pt}\end{leftcolumn*}

\begin{rightcolumn}\PaliColumn{with the cessation of clinging, cessation of being;}
\hspace{0pt}
\end{rightcolumn}
\end{samepage}
\begin{samepage}
\begin{leftcolumn*}
\EnglishColumn{bhavanirodhā jātinirodho,}
\hspace{0pt}\end{leftcolumn*}

\begin{rightcolumn}\PaliColumn{with the cessation of being, cessation of birth;}
\hspace{0pt}
\end{rightcolumn}
\end{samepage}
\begin{samepage}
\begin{leftcolumn*}
\EnglishColumn{jātinirodhā jarāmaraṇaṃ sokaparidevadukkhadomanassupāyāsā nirujjhanti.}
\hspace{0pt}\end{leftcolumn*}

\begin{rightcolumn}\PaliColumn{with the cessation of birth, ageing and death, sorrow, lamentation, pain, grief, and despair cease.}
\hspace{0pt}
\end{rightcolumn}
\end{samepage}
\begin{samepage}
\begin{leftcolumn*}
\EnglishColumn{evametassa kevalassa dukkhakkhandhassa nirodho hoti.}
\hspace{0pt}\end{leftcolumn*}

\begin{rightcolumn}\PaliColumn{Such is the cessation of this whole mass of suffering.}
\hspace{0pt}
\end{rightcolumn}
\end{samepage}
\vskip 0.2in
\begin{samepage}
\begin{leftcolumn*}
\EnglishColumn{“api nu tumhe, bhikkhave, evaṃ jānantā evaṃ passantā pubbantaṃ vā paṭidhāveyyātha -}
\hspace{0pt}\end{leftcolumn*}

\begin{rightcolumn}\PaliColumn{“Bhikkhus, knowing and seeing in this way, would you run back to the past thus:}
\hspace{0pt}
\end{rightcolumn}
\end{samepage}
\begin{samepage}
\begin{leftcolumn*}
\EnglishColumn{‘ahesumha nu kho mayaṃ atītamaddhānaṃ,}
\hspace{0pt}\end{leftcolumn*}

\begin{rightcolumn}\PaliColumn{‘Were we in the past?}
\hspace{0pt}
\end{rightcolumn}
\end{samepage}
\begin{samepage}
\begin{leftcolumn*}
\EnglishColumn{nanu kho ahesumha atītamaddhānaṃ,}
\hspace{0pt}\end{leftcolumn*}

\begin{rightcolumn}\PaliColumn{Were we not in the past?}
\hspace{0pt}
\end{rightcolumn}
\end{samepage}
\begin{samepage}
\begin{leftcolumn*}
\EnglishColumn{kiṃ nu kho ahesumha atītamaddhānaṃ,}
\hspace{0pt}\end{leftcolumn*}

\begin{rightcolumn}\PaliColumn{What were we in the past?}
\hspace{0pt}
\end{rightcolumn}
\end{samepage}
\begin{samepage}
\begin{leftcolumn*}
\EnglishColumn{kathaṃ nu kho ahesumha atītamaddhānaṃ,}
\hspace{0pt}\end{leftcolumn*}

\begin{rightcolumn}\PaliColumn{How were we in the past?}
\hspace{0pt}
\end{rightcolumn}
\end{samepage}
\begin{samepage}
\begin{leftcolumn*}
\EnglishColumn{kiṃ hutvā kiṃ ahesumha nu kho mayaṃ atītamaddhānan’”ti?}
\hspace{0pt}\end{leftcolumn*}

\begin{rightcolumn}\PaliColumn{Having been what, what did we become in the past?’?”}
\hspace{0pt}
\end{rightcolumn}
\end{samepage}
\vskip 0.2in
\begin{samepage}
\begin{leftcolumn*}
\EnglishColumn{“no hetaṃ, bhante”.}
\hspace{0pt}\end{leftcolumn*}

\begin{rightcolumn}\PaliColumn{“No, venerable sir.”}
\hspace{0pt}
\end{rightcolumn}
\end{samepage}
\vskip 0.2in
\begin{samepage}
\begin{leftcolumn*}
\EnglishColumn{“api nu tumhe, bhikkhave, evaṃ jānantā evaṃ passantā aparantaṃ vā paṭidhāveyyātha -}
\hspace{0pt}\end{leftcolumn*}

\begin{rightcolumn}\PaliColumn{“Knowing and seeing in this way, would you run forward to the future thus:}
\hspace{0pt}
\end{rightcolumn}
\end{samepage}
\begin{samepage}
\begin{leftcolumn*}
\EnglishColumn{bhavissāma nu kho mayaṃ anāgatamaddhānaṃ,}
\hspace{0pt}\end{leftcolumn*}

\begin{rightcolumn}\PaliColumn{‘Shall we be in the future?}
\hspace{0pt}
\end{rightcolumn}
\end{samepage}
\begin{samepage}
\begin{leftcolumn*}
\EnglishColumn{nanu kho bhavissāma anāgatamaddhānaṃ,}
\hspace{0pt}\end{leftcolumn*}

\begin{rightcolumn}\PaliColumn{Shall we not be in the future?}
\hspace{0pt}
\end{rightcolumn}
\end{samepage}
\begin{samepage}
\begin{leftcolumn*}
\EnglishColumn{kiṃ nu kho bhavissāma anāgatamaddhānaṃ,}
\hspace{0pt}\end{leftcolumn*}

\begin{rightcolumn}\PaliColumn{What shall we be in the future?}
\hspace{0pt}
\end{rightcolumn}
\end{samepage}
\begin{samepage}
\begin{leftcolumn*}
\EnglishColumn{kathaṃ nu kho bhavissāma anāgatamaddhānaṃ,}
\hspace{0pt}\end{leftcolumn*}

\begin{rightcolumn}\PaliColumn{How shall we be in the future?}
\hspace{0pt}
\end{rightcolumn}
\end{samepage}
\begin{samepage}
\begin{leftcolumn*}
\EnglishColumn{kiṃ hutvā kiṃ bhavissāma nu kho mayaṃ anāgatamaddhānan”ti?}
\hspace{0pt}\end{leftcolumn*}

\begin{rightcolumn}\PaliColumn{Having been what, what shall we become in the future?’?”}
\hspace{0pt}
\end{rightcolumn}
\end{samepage}
\vskip 0.2in
\begin{samepage}
\begin{leftcolumn*}
\EnglishColumn{“no hetaṃ, bhante”.}
\hspace{0pt}\end{leftcolumn*}

\begin{rightcolumn}\PaliColumn{“No, venerable sir.”}
\hspace{0pt}
\end{rightcolumn}
\end{samepage}
\vskip 0.2in
\begin{samepage}
\begin{leftcolumn*}
\EnglishColumn{“api nu tumhe, bhikkhave, evaṃ jānantā evaṃ passantā etarahi vā paccuppannamaddhānaṃ ajjhattaṃ kathaṃkathī assatha -}
\hspace{0pt}\end{leftcolumn*}

\begin{rightcolumn}\PaliColumn{“Knowing and seeing in this way, would you now be inwardly perplexed about the present thus:}
\hspace{0pt}
\end{rightcolumn}
\end{samepage}
\begin{samepage}
\begin{leftcolumn*}
\EnglishColumn{ahaṃ nu khosmi,}
\hspace{0pt}\end{leftcolumn*}

\begin{rightcolumn}\PaliColumn{‘Am I?}
\hspace{0pt}
\end{rightcolumn}
\end{samepage}
\begin{samepage}
\begin{leftcolumn*}
\EnglishColumn{no nu khosmi,}
\hspace{0pt}\end{leftcolumn*}

\begin{rightcolumn}\PaliColumn{Am I not?}
\hspace{0pt}
\end{rightcolumn}
\end{samepage}
\begin{samepage}
\begin{leftcolumn*}
\EnglishColumn{kiṃ nu khosmi,}
\hspace{0pt}\end{leftcolumn*}

\begin{rightcolumn}\PaliColumn{What am I?}
\hspace{0pt}
\end{rightcolumn}
\end{samepage}
\begin{samepage}
\begin{leftcolumn*}
\EnglishColumn{kathaṃ nu khosmi,}
\hspace{0pt}\end{leftcolumn*}

\begin{rightcolumn}\PaliColumn{How am I?}
\hspace{0pt}
\end{rightcolumn}
\end{samepage}
\begin{samepage}
\begin{leftcolumn*}
\EnglishColumn{ayaṃ nu kho satto kuto āgato,}
\hspace{0pt}\end{leftcolumn*}

\begin{rightcolumn}\PaliColumn{Where has this being come from?}
\hspace{0pt}
\end{rightcolumn}
\end{samepage}
\begin{samepage}
\begin{leftcolumn*}
\EnglishColumn{so kuhiṃgāmī bhavissatī”ti?}
\hspace{0pt}\end{leftcolumn*}

\begin{rightcolumn}\PaliColumn{Where will it go?’?”}
\hspace{0pt}
\end{rightcolumn}
\end{samepage}
\vskip 0.2in
\begin{samepage}
\begin{leftcolumn*}
\EnglishColumn{“no hetaṃ, bhante”.}
\hspace{0pt}\end{leftcolumn*}

\begin{rightcolumn}\PaliColumn{“No, venerable sir.”}
\hspace{0pt}
\end{rightcolumn}
\end{samepage}
\vskip 0.2in
\begin{samepage}
\begin{leftcolumn*}
\EnglishColumn{“api nu tumhe, ikkhave, evaṃ jānantā evaṃ passantā evaṃ vadeyyātha -}
\hspace{0pt}\end{leftcolumn*}

\begin{rightcolumn}\PaliColumn{“Bhikkhus, knowing and seeing in this way, would you speak thus:}
\hspace{0pt}
\end{rightcolumn}
\end{samepage}
\begin{samepage}
\begin{leftcolumn*}
\EnglishColumn{satthā no garu, satthugāravena ca mayaṃ evaṃ vademā”ti?}
\hspace{0pt}\end{leftcolumn*}

\begin{rightcolumn}\PaliColumn{‘The Teacher is respected by us.  We speak as we do out of respect for the Teacher’?”}
\hspace{0pt}
\end{rightcolumn}
\end{samepage}
\vskip 0.2in
\begin{samepage}
\begin{leftcolumn*}
\EnglishColumn{“no hetaṃ, bhante”.}
\hspace{0pt}\end{leftcolumn*}

\begin{rightcolumn}\PaliColumn{“No, venerable sir.”}
\hspace{0pt}
\end{rightcolumn}
\end{samepage}
\vskip 0.2in
\begin{samepage}
\begin{leftcolumn*}
\EnglishColumn{“api nu tumhe, bhikkhave, evaṃ jānantā evaṃ passantā evaṃ vadeyyātha -}
\hspace{0pt}\end{leftcolumn*}

\begin{rightcolumn}\PaliColumn{“Knowing and seeing in this way, would you speak thus:}
\hspace{0pt}
\end{rightcolumn}
\end{samepage}
\begin{samepage}
\begin{leftcolumn*}
\EnglishColumn{samaṇo evamāha, samaṇā ca nāma mayaṃ evaṃ vademā”ti?}
\hspace{0pt}\end{leftcolumn*}

\begin{rightcolumn}\PaliColumn{‘The Recluse says this, and we speak thus at the bidding of the Recluse’?”}
\hspace{0pt}
\end{rightcolumn}
\end{samepage}
\vskip 0.2in
\begin{samepage}
\begin{leftcolumn*}
\EnglishColumn{“no hetaṃ, bhante”.}
\hspace{0pt}\end{leftcolumn*}

\begin{rightcolumn}\PaliColumn{“No, venerable sir.”}
\hspace{0pt}
\end{rightcolumn}
\end{samepage}
\vskip 0.2in
\begin{samepage}
\begin{leftcolumn*}
\EnglishColumn{“api nu tumhe, bhikkhave, evaṃ jānantā evaṃ passantā aññaṃ satthāraṃ uddiseyyāthā”ti?}
\hspace{0pt}\end{leftcolumn*}

\begin{rightcolumn}\PaliColumn{“Knowing and seeing in this way, would you acknowledge another teacher?”}
\hspace{0pt}
\end{rightcolumn}
\end{samepage}
\vskip 0.2in
\begin{samepage}
\begin{leftcolumn*}
\EnglishColumn{“no hetaṃ, bhante”.}
\hspace{0pt}\end{leftcolumn*}

\begin{rightcolumn}\PaliColumn{“No, venerable sir.”}
\hspace{0pt}
\end{rightcolumn}
\end{samepage}
\vskip 0.2in
\begin{samepage}
\begin{leftcolumn*}
\EnglishColumn{“api nu tumhe, bhikkhave, evaṃ jānantā evaṃ passantā yāni tāni puthusamaṇabrāhmaṇānaṃ vata kotūhalamaṅgalāni tāni sārato paccāgaccheyyāthā”ti?}
\hspace{0pt}\end{leftcolumn*}

\begin{rightcolumn}\PaliColumn{“Knowing and seeing in this way, would you return to the observances, tumultuous debates, and auspicious signs of ordinary recluses and brahmins, taking them as the core [of the holy life]?”}
\hspace{0pt}
\end{rightcolumn}
\end{samepage}
\vskip 0.2in
\begin{samepage}
\begin{leftcolumn*}
\EnglishColumn{“no hetaṃ, bhante”.}
\hspace{0pt}\end{leftcolumn*}

\begin{rightcolumn}\PaliColumn{“No, venerable sir.”}
\hspace{0pt}
\end{rightcolumn}
\end{samepage}
\vskip 0.2in
\begin{samepage}
\begin{leftcolumn*}
\EnglishColumn{“nanu, bhikkhave, yadeva tumhākaṃ sāmaṃ ñātaṃ sāmaṃ diṭṭhaṃ sāmaṃ viditaṃ, tadeva tumhe vadethā”ti.}
\hspace{0pt}\end{leftcolumn*}

\begin{rightcolumn}\PaliColumn{“Do you speak only of what you have known, seen, and understood for yourselves?”}
\hspace{0pt}
\end{rightcolumn}
\end{samepage}
\vskip 0.2in
\begin{samepage}
\begin{leftcolumn*}
\EnglishColumn{“evaṃ, bhante”.}
\hspace{0pt}\end{leftcolumn*}

\begin{rightcolumn}\PaliColumn{“Yes, venerable sir.”}
\hspace{0pt}
\end{rightcolumn}
\end{samepage}
\vskip 0.2in
\begin{samepage}
\begin{leftcolumn*}
\EnglishColumn{“sādhu, bhikkhave,}
\hspace{0pt}\end{leftcolumn*}

\begin{rightcolumn}\PaliColumn{“Good, bhikkhus.}
\hspace{0pt}
\end{rightcolumn}
\end{samepage}
\begin{samepage}
\begin{leftcolumn*}
\EnglishColumn{upanītā kho me tumhe, bhikkhave, iminā sandiṭṭhikena dhammena akālikena ehipassikena opaneyyikena paccattaṃ veditabbena viññūhi.}
\hspace{0pt}\end{leftcolumn*}

\begin{rightcolumn}\PaliColumn{So you have been guided by me with this Dhamma, which is visible here and now, immediately effective, inviting inspection, onward leading, to be experienced by the wise for themselves.}
\hspace{0pt}
\end{rightcolumn}
\end{samepage}
\begin{samepage}
\begin{leftcolumn*}
\EnglishColumn{sandiṭṭhiko ayaṃ, bhikkhave, dhammo akāliko ehipassiko opaneyyiko paccattaṃ veditabbo viññūhi - iti yantaṃ vuttaṃ, idametaṃ paṭicca vuttan”ti.}
\hspace{0pt}\end{leftcolumn*}

\begin{rightcolumn}\PaliColumn{For it was with reference to this that it has been said: ‘Bhikkhus, this Dhamma is visible here and now, immediately effective, inviting inspection, onward leading, to be experienced by the wise for themselves.’}
\hspace{0pt}
\end{rightcolumn}
\end{samepage}
\vskip 0.2in
\begin{samepage}
\begin{leftcolumn*}
\EnglishColumn{“tiṇṇaṃ kho pana, bhikkhave, sannipātā gabbhassāvakkanti hoti.}
\hspace{0pt}\end{leftcolumn*}

\begin{rightcolumn}\PaliColumn{“Bhikkhus, the descent of the embryo takes place through the union of three things.}
\hspace{0pt}
\end{rightcolumn}
\end{samepage}
\begin{samepage}
\begin{leftcolumn*}
\EnglishColumn{idha mātāpitaro ca sannipatitā honti, mātā ca na utunī hoti, gandhabbo ca na paccupaṭṭhito hoti, neva tāva gabbhassāvakkanti hoti.}
\hspace{0pt}\end{leftcolumn*}

\begin{rightcolumn}\PaliColumn{Here, there is the union of the mother and father, but the mother is not in season, and the gandhabba is not present—in this case no descent of an embryo takes place.}
\hspace{0pt}
\end{rightcolumn}
\end{samepage}
\begin{samepage}
\begin{leftcolumn*}
\EnglishColumn{idha mātāpitaro ca sannipatitā honti, mātā ca utunī hoti, gandhabbo ca na paccupaṭṭhito hoti, neva tāva gabbhassāvakkanti hoti.}
\hspace{0pt}\end{leftcolumn*}

\begin{rightcolumn}\PaliColumn{Here, there is the union of the mother and father, and the mother is in season, but the gandhabba is not present—in this case too no descent of the embryo takes place.}
\hspace{0pt}
\end{rightcolumn}
\end{samepage}
\begin{samepage}
\begin{leftcolumn*}
\EnglishColumn{yato ca kho, bhikkhave, mātāpitaro ca sannipatitā honti, mātā ca utunī hoti, gandhabbo ca paccupaṭṭhito hoti - evaṃ tiṇṇaṃ sannipātā gabbhassāvakkanti hoti.}
\hspace{0pt}\end{leftcolumn*}

\begin{rightcolumn}\PaliColumn{But when there is the union of the mother and father, and the mother is in season, and the gandhabba is present, through the union of these three things the descent of the embryo takes place.}
\hspace{0pt}
\end{rightcolumn}
\end{samepage}
\vskip 0.2in
\begin{samepage}
\begin{leftcolumn*}
\EnglishColumn{tamenaṃ, bhikkhave, mātā nava vā dasa vā māse gabbhaṃ kucchinā pariharati mahatā saṃsayena garubhāraṃ.}
\hspace{0pt}\end{leftcolumn*}

\begin{rightcolumn}\PaliColumn{“The mother then carries the embryo in her womb for nine or ten months with much anxiety, as a heavy burden.}
\hspace{0pt}
\end{rightcolumn}
\end{samepage}
\begin{samepage}
\begin{leftcolumn*}
\EnglishColumn{tamenaṃ, bhikkhave, mātā navannaṃ vā dasannaṃ vā māsānaṃ accayena vijāyati mahatā saṃsayena garubhāraṃ.}
\hspace{0pt}\end{leftcolumn*}

\begin{rightcolumn}\PaliColumn{Then, at the end of nine or ten months, the mother gives birth with much anxiety, as a heavy burden.}
\hspace{0pt}
\end{rightcolumn}
\end{samepage}
\begin{samepage}
\begin{leftcolumn*}
\EnglishColumn{tamenaṃ jātaṃ samānaṃ sakena lohitena poseti.}
\hspace{0pt}\end{leftcolumn*}

\begin{rightcolumn}\PaliColumn{Then, when the child is born, she nourishes it with her own blood;}
\hspace{0pt}
\end{rightcolumn}
\end{samepage}
\begin{samepage}
\begin{leftcolumn*}
\EnglishColumn{lohitañhetaṃ, bhikkhave, ariyassa vinaye yadidaṃ mātuthaññaṃ.}
\hspace{0pt}\end{leftcolumn*}

\begin{rightcolumn}\PaliColumn{for the mother’s breast-milk is called blood in the Noble One’s Discipline.}
\hspace{0pt}
\end{rightcolumn}
\end{samepage}
\vskip 0.2in
\begin{samepage}
\begin{leftcolumn*}
\EnglishColumn{sa kho so, bhikkhave, kumāro vuddhimanvāya indriyānaṃ paripākamanvāya yāni tāni kumārakānaṃ kīḷāpanakāni tehi kīḷati, seyyathidaṃ - vaṅkakaṃ ghaṭikaṃ mokkhacikaṃ ciṅgulakaṃ pattāḷhakaṃ rathakaṃ dhanukaṃ.}
\hspace{0pt}\end{leftcolumn*}

\begin{rightcolumn}\PaliColumn{“When he grows up and his faculties mature, the child plays at such games as toy ploughs, tipcat, somersaults, toy windmills, toy measures, toy cars, and a toy bow and arrow.}
\hspace{0pt}
\end{rightcolumn}
\end{samepage}
\begin{samepage}
\begin{leftcolumn*}
\EnglishColumn{sa kho so, bhikkhave, kumāro vuddhimanvāya indriyānaṃ paripākamanvāya pañcahi kāmaguṇehi samappito samaṅgībhūto paricāreti -}
\hspace{0pt}\end{leftcolumn*}

\begin{rightcolumn}\PaliColumn{“When he grows up and his faculties mature [still further], the youth enjoys himself provided and endowed with the five cords of sensual pleasure.}
\hspace{0pt}
\end{rightcolumn}
\end{samepage}
\begin{samepage}
\begin{leftcolumn*}
\EnglishColumn{cakkhuviññeyyehi rūpehi iṭṭhehi kantehi manāpehi piyarūpehi kāmūpasaṃhitehi rajanīyehi,}
\hspace{0pt}\end{leftcolumn*}

\begin{rightcolumn}\PaliColumn{With forms cognizable by the eye that are wished for, desired, agreeable and likeable, connected with sensual desire, and provocative of lust.}
\hspace{0pt}
\end{rightcolumn}
\end{samepage}
\begin{samepage}
\begin{leftcolumn*}
\EnglishColumn{sotaviññeyyehi saddehi iṭṭhehi kantehi manāpehi piyarūpehi kāmūpasaṃhitehi rajanīyehi,}
\hspace{0pt}\end{leftcolumn*}

\begin{rightcolumn}\PaliColumn{Sounds cognizable by the ear that are wished for, desired, agreeable and likeable, connected with sensual desire, and provocative of lust.}
\hspace{0pt}
\end{rightcolumn}
\end{samepage}
\begin{samepage}
\begin{leftcolumn*}
\EnglishColumn{ghānaviññeyyehi gandhehi iṭṭhehi kantehi manāpehi piyarūpehi kāmūpasaṃhitehi rajanīyehi,}
\hspace{0pt}\end{leftcolumn*}

\begin{rightcolumn}\PaliColumn{Odours cognizable by the nose that are wished for, desired, agreeable and likeable, connected with sensual desire, and provocative of lust.}
\hspace{0pt}
\end{rightcolumn}
\end{samepage}
\begin{samepage}
\begin{leftcolumn*}
\EnglishColumn{jivhāviññeyyehi rasehi iṭṭhehi kantehi manāpehi piyarūpehi kāmūpasaṃhitehi rajanīyehi,}
\hspace{0pt}\end{leftcolumn*}

\begin{rightcolumn}\PaliColumn{Flavours cognizable by the tongue that are wished for, desired, agreeable and likeable, connected with sensual desire, and provocative of lust.}
\hspace{0pt}
\end{rightcolumn}
\end{samepage}
\begin{samepage}
\begin{leftcolumn*}
\EnglishColumn{kāyaviññeyyehi phoṭṭhabbehi iṭṭhehi kantehi manāpehi piyarūpehi kāmūpasaṃhitehi rajanīyehi.}
\hspace{0pt}\end{leftcolumn*}

\begin{rightcolumn}\PaliColumn{Tangibles cognizable by the body that are wished for, desired, agreeable and likeable, connected with sensual desire, and provocative of lust.}
\hspace{0pt}
\end{rightcolumn}
\end{samepage}
\vskip 0.2in
\begin{samepage}
\begin{leftcolumn*}
\EnglishColumn{“so cakkhunā rūpaṃ disvā piyarūpe rūpe sārajjati,}
\hspace{0pt}\end{leftcolumn*}

\begin{rightcolumn}\PaliColumn{“On seeing a form with the eye, he lusts after it if it is pleasing;}
\hspace{0pt}
\end{rightcolumn}
\end{samepage}
\begin{samepage}
\begin{leftcolumn*}
\EnglishColumn{appiyarūpe rūpe byāpajjati,}
\hspace{0pt}\end{leftcolumn*}

\begin{rightcolumn}\PaliColumn{he dislikes it if it is unpleasing.}
\hspace{0pt}
\end{rightcolumn}
\end{samepage}
\begin{samepage}
\begin{leftcolumn*}
\EnglishColumn{anupaṭṭhitakāyasati ca viharati parittacetaso.}
\hspace{0pt}\end{leftcolumn*}

\begin{rightcolumn}\PaliColumn{He abides with mindfulness of the body unestablished, with a limited mind,}
\hspace{0pt}
\end{rightcolumn}
\end{samepage}
\begin{samepage}
\begin{leftcolumn*}
\EnglishColumn{tañca cetovimuttiṃ paññāvimuttiṃ yathābhūtaṃ nappajānāti - yatthassa te pāpakā akusalā dhammā aparisesā nirujjhanti.}
\hspace{0pt}\end{leftcolumn*}

\begin{rightcolumn}\PaliColumn{and he does not understand as it actually is the deliverance of mind and deliverance by wisdom wherein those evil unwholesome states cease without remainder.}
\hspace{0pt}
\end{rightcolumn}
\end{samepage}
\begin{samepage}
\begin{leftcolumn*}
\EnglishColumn{so evaṃ anurodhavirodhaṃ samāpanno yaṃ kiñci vedanaṃ vedeti sukhaṃ vā dukkhaṃ vā adukkhamasukhaṃ vā, so taṃ vedanaṃ abhinandati abhivadati ajjhosāya tiṭṭhati.}
\hspace{0pt}\end{leftcolumn*}

\begin{rightcolumn}\PaliColumn{Engaged as he is in favouring and opposing, whatever feeling he feels—whether pleasant or painful or neither-painful-nor-pleasant—he delights in that feeling, welcomes it, and remains holding to it.}
\hspace{0pt}
\end{rightcolumn}
\end{samepage}
\begin{samepage}
\begin{leftcolumn*}
\EnglishColumn{tassa taṃ vedanaṃ abhinandato abhivadato ajjhosāya tiṭṭhato uppajjati nandī.}
\hspace{0pt}\end{leftcolumn*}

\begin{rightcolumn}\PaliColumn{As he does so, delight arises in him.}
\hspace{0pt}
\end{rightcolumn}
\end{samepage}
\begin{samepage}
\begin{leftcolumn*}
\EnglishColumn{yā vedanāsu nandī tadupādānaṃ,}
\hspace{0pt}\end{leftcolumn*}

\begin{rightcolumn}\PaliColumn{Now delight in feelings is clinging.}
\hspace{0pt}
\end{rightcolumn}
\end{samepage}
\begin{samepage}
\begin{leftcolumn*}
\EnglishColumn{tassupādānapaccayā bhavo,}
\hspace{0pt}\end{leftcolumn*}

\begin{rightcolumn}\PaliColumn{With his clinging as condition, being [comes to be];}
\hspace{0pt}
\end{rightcolumn}
\end{samepage}
\begin{samepage}
\begin{leftcolumn*}
\EnglishColumn{bhavapaccayā jāti,}
\hspace{0pt}\end{leftcolumn*}

\begin{rightcolumn}\PaliColumn{with being as condition, birth;}
\hspace{0pt}
\end{rightcolumn}
\end{samepage}
\begin{samepage}
\begin{leftcolumn*}
\EnglishColumn{jātipaccayā jarāmaraṇaṃ sokaparidevadukkhadomanassupāyāsā sambhavanti.}
\hspace{0pt}\end{leftcolumn*}

\begin{rightcolumn}\PaliColumn{with birth as condition, ageing and death, sorrow, lamentation, pain, grief, and despair come to be.}
\hspace{0pt}
\end{rightcolumn}
\end{samepage}
\begin{samepage}
\begin{leftcolumn*}
\EnglishColumn{evametassa kevalassa dukkhakkhandhassa samudayo hoti.}
\hspace{0pt}\end{leftcolumn*}

\begin{rightcolumn}\PaliColumn{Such is the origin of this whole mass of suffering.}
\hspace{0pt}
\end{rightcolumn}
\end{samepage}
\vskip 0.2in
\begin{samepage}
\begin{leftcolumn*}
\EnglishColumn{sotena saddaṃ sutvā disvā piyarūpe sadde sārajjati,}
\hspace{0pt}\end{leftcolumn*}

\begin{rightcolumn}\PaliColumn{“On hearing a sound with the ear, he lusts after it if it is pleasing;}
\hspace{0pt}
\end{rightcolumn}
\end{samepage}
\begin{samepage}
\begin{leftcolumn*}
\EnglishColumn{appiyarūpe sadde byāpajjati,}
\hspace{0pt}\end{leftcolumn*}

\begin{rightcolumn}\PaliColumn{he dislikes it if it is unpleasing.}
\hspace{0pt}
\end{rightcolumn}
\end{samepage}
\begin{samepage}
\begin{leftcolumn*}
\EnglishColumn{anupaṭṭhitakāyasati ca viharati parittacetaso.}
\hspace{0pt}\end{leftcolumn*}

\begin{rightcolumn}\PaliColumn{He abides with mindfulness of the body unestablished, with a limited mind,}
\hspace{0pt}
\end{rightcolumn}
\end{samepage}
\begin{samepage}
\begin{leftcolumn*}
\EnglishColumn{tañca cetovimuttiṃ paññāvimuttiṃ yathābhūtaṃ nappajānāti - yatthassa te pāpakā akusalā dhammā aparisesā nirujjhanti.}
\hspace{0pt}\end{leftcolumn*}

\begin{rightcolumn}\PaliColumn{and he does not understand as it actually is the deliverance of mind and deliverance by wisdom wherein those evil unwholesome states cease without remainder.}
\hspace{0pt}
\end{rightcolumn}
\end{samepage}
\begin{samepage}
\begin{leftcolumn*}
\EnglishColumn{so evaṃ anurodhavirodhaṃ samāpanno yaṃ kiñci vedanaṃ vedeti sukhaṃ vā dukkhaṃ vā adukkhamasukhaṃ vā, so taṃ vedanaṃ abhinandati abhivadati ajjhosāya tiṭṭhati.}
\hspace{0pt}\end{leftcolumn*}

\begin{rightcolumn}\PaliColumn{Engaged as he is in favouring and opposing, whatever feeling he feels—whether pleasant or painful or neither-painful-nor-pleasant—he delights in that feeling, welcomes it, and remains holding to it.}
\hspace{0pt}
\end{rightcolumn}
\end{samepage}
\begin{samepage}
\begin{leftcolumn*}
\EnglishColumn{tassa taṃ vedanaṃ abhinandato abhivadato ajjhosāya tiṭṭhato uppajjati nandī.}
\hspace{0pt}\end{leftcolumn*}

\begin{rightcolumn}\PaliColumn{As he does so, delight arises in him.}
\hspace{0pt}
\end{rightcolumn}
\end{samepage}
\begin{samepage}
\begin{leftcolumn*}
\EnglishColumn{yā vedanāsu nandī tadupādānaṃ,}
\hspace{0pt}\end{leftcolumn*}

\begin{rightcolumn}\PaliColumn{Now delight in feelings is clinging.}
\hspace{0pt}
\end{rightcolumn}
\end{samepage}
\begin{samepage}
\begin{leftcolumn*}
\EnglishColumn{tassupādānapaccayā bhavo,}
\hspace{0pt}\end{leftcolumn*}

\begin{rightcolumn}\PaliColumn{With his clinging as condition, being [comes to be];}
\hspace{0pt}
\end{rightcolumn}
\end{samepage}
\begin{samepage}
\begin{leftcolumn*}
\EnglishColumn{bhavapaccayā jāti,}
\hspace{0pt}\end{leftcolumn*}

\begin{rightcolumn}\PaliColumn{with being as condition, birth;}
\hspace{0pt}
\end{rightcolumn}
\end{samepage}
\begin{samepage}
\begin{leftcolumn*}
\EnglishColumn{jātipaccayā jarāmaraṇaṃ sokaparidevadukkhadomanassupāyāsā sambhavanti.}
\hspace{0pt}\end{leftcolumn*}

\begin{rightcolumn}\PaliColumn{with birth as condition, ageing and death, sorrow, lamentation, pain, grief, and despair come to be.}
\hspace{0pt}
\end{rightcolumn}
\end{samepage}
\begin{samepage}
\begin{leftcolumn*}
\EnglishColumn{evametassa kevalassa dukkhakkhandhassa samudayo hoti.}
\hspace{0pt}\end{leftcolumn*}

\begin{rightcolumn}\PaliColumn{Such is the origin of this whole mass of suffering.}
\hspace{0pt}
\end{rightcolumn}
\end{samepage}
\vskip 0.2in
\begin{samepage}
\begin{leftcolumn*}
\EnglishColumn{ghānena gandhaṃ ghāyitvā disvā piyarūpe gandhe sārajjati,}
\hspace{0pt}\end{leftcolumn*}

\begin{rightcolumn}\PaliColumn{On smelling an odour with the nose, he lusts after it if it is pleasing;}
\hspace{0pt}
\end{rightcolumn}
\end{samepage}
\begin{samepage}
\begin{leftcolumn*}
\EnglishColumn{appiyarūpe gandhe byāpajjati,}
\hspace{0pt}\end{leftcolumn*}

\begin{rightcolumn}\PaliColumn{he dislikes it if it is unpleasing.}
\hspace{0pt}
\end{rightcolumn}
\end{samepage}
\begin{samepage}
\begin{leftcolumn*}
\EnglishColumn{anupaṭṭhitakāyasati ca viharati parittacetaso.}
\hspace{0pt}\end{leftcolumn*}

\begin{rightcolumn}\PaliColumn{He abides with mindfulness of the body unestablished, with a limited mind,}
\hspace{0pt}
\end{rightcolumn}
\end{samepage}
\begin{samepage}
\begin{leftcolumn*}
\EnglishColumn{tañca cetovimuttiṃ paññāvimuttiṃ yathābhūtaṃ nappajānāti - yatthassa te pāpakā akusalā dhammā aparisesā nirujjhanti.}
\hspace{0pt}\end{leftcolumn*}

\begin{rightcolumn}\PaliColumn{and he does not understand as it actually is the deliverance of mind and deliverance by wisdom wherein those evil unwholesome states cease without remainder.}
\hspace{0pt}
\end{rightcolumn}
\end{samepage}
\begin{samepage}
\begin{leftcolumn*}
\EnglishColumn{so evaṃ anurodhavirodhaṃ samāpanno yaṃ kiñci vedanaṃ vedeti sukhaṃ vā dukkhaṃ vā adukkhamasukhaṃ vā, so taṃ vedanaṃ abhinandati abhivadati ajjhosāya tiṭṭhati.}
\hspace{0pt}\end{leftcolumn*}

\begin{rightcolumn}\PaliColumn{Engaged as he is in favouring and opposing, whatever feeling he feels—whether pleasant or painful or neither-painful-nor-pleasant—he delights in that feeling, welcomes it, and remains holding to it.}
\hspace{0pt}
\end{rightcolumn}
\end{samepage}
\begin{samepage}
\begin{leftcolumn*}
\EnglishColumn{tassa taṃ vedanaṃ abhinandato abhivadato ajjhosāya tiṭṭhato uppajjati nandī.}
\hspace{0pt}\end{leftcolumn*}

\begin{rightcolumn}\PaliColumn{As he does so, delight arises in him.}
\hspace{0pt}
\end{rightcolumn}
\end{samepage}
\begin{samepage}
\begin{leftcolumn*}
\EnglishColumn{yā vedanāsu nandī tadupādānaṃ,}
\hspace{0pt}\end{leftcolumn*}

\begin{rightcolumn}\PaliColumn{Now delight in feelings is clinging.}
\hspace{0pt}
\end{rightcolumn}
\end{samepage}
\begin{samepage}
\begin{leftcolumn*}
\EnglishColumn{tassupādānapaccayā bhavo,}
\hspace{0pt}\end{leftcolumn*}

\begin{rightcolumn}\PaliColumn{With his clinging as condition, being [comes to be];}
\hspace{0pt}
\end{rightcolumn}
\end{samepage}
\begin{samepage}
\begin{leftcolumn*}
\EnglishColumn{bhavapaccayā jāti,}
\hspace{0pt}\end{leftcolumn*}

\begin{rightcolumn}\PaliColumn{with being as condition, birth;}
\hspace{0pt}
\end{rightcolumn}
\end{samepage}
\begin{samepage}
\begin{leftcolumn*}
\EnglishColumn{jātipaccayā jarāmaraṇaṃ sokaparidevadukkhadomanassupāyāsā sambhavanti.}
\hspace{0pt}\end{leftcolumn*}

\begin{rightcolumn}\PaliColumn{with birth as condition, ageing and death, sorrow, lamentation, pain, grief, and despair come to be.}
\hspace{0pt}
\end{rightcolumn}
\end{samepage}
\begin{samepage}
\begin{leftcolumn*}
\EnglishColumn{evametassa kevalassa dukkhakkhandhassa samudayo hoti.}
\hspace{0pt}\end{leftcolumn*}

\begin{rightcolumn}\PaliColumn{Such is the origin of this whole mass of suffering.}
\hspace{0pt}
\end{rightcolumn}
\end{samepage}
\vskip 0.2in
\begin{samepage}
\begin{leftcolumn*}
\EnglishColumn{jivhāya rasaṃ sāyitvā disvā piyarūpe rase sārajjati,}
\hspace{0pt}\end{leftcolumn*}

\begin{rightcolumn}\PaliColumn{On tasting a flavour with the tongue, he lusts after it if it is pleasing;}
\hspace{0pt}
\end{rightcolumn}
\end{samepage}
\begin{samepage}
\begin{leftcolumn*}
\EnglishColumn{appiyarūpe rase byāpajjati,}
\hspace{0pt}\end{leftcolumn*}

\begin{rightcolumn}\PaliColumn{he dislikes it if it is unpleasing.}
\hspace{0pt}
\end{rightcolumn}
\end{samepage}
\begin{samepage}
\begin{leftcolumn*}
\EnglishColumn{anupaṭṭhitakāyasati ca viharati parittacetaso.}
\hspace{0pt}\end{leftcolumn*}

\begin{rightcolumn}\PaliColumn{He abides with mindfulness of the body unestablished, with a limited mind,}
\hspace{0pt}
\end{rightcolumn}
\end{samepage}
\begin{samepage}
\begin{leftcolumn*}
\EnglishColumn{tañca cetovimuttiṃ paññāvimuttiṃ yathābhūtaṃ nappajānāti - yatthassa te pāpakā akusalā dhammā aparisesā nirujjhanti.}
\hspace{0pt}\end{leftcolumn*}

\begin{rightcolumn}\PaliColumn{and he does not understand as it actually is the deliverance of mind and deliverance by wisdom wherein those evil unwholesome states cease without remainder.}
\hspace{0pt}
\end{rightcolumn}
\end{samepage}
\begin{samepage}
\begin{leftcolumn*}
\EnglishColumn{so evaṃ anurodhavirodhaṃ samāpanno yaṃ kiñci vedanaṃ vedeti sukhaṃ vā dukkhaṃ vā adukkhamasukhaṃ vā, so taṃ vedanaṃ abhinandati abhivadati ajjhosāya tiṭṭhati.}
\hspace{0pt}\end{leftcolumn*}

\begin{rightcolumn}\PaliColumn{Engaged as he is in favouring and opposing, whatever feeling he feels—whether pleasant or painful or neither-painful-nor-pleasant—he delights in that feeling, welcomes it, and remains holding to it.}
\hspace{0pt}
\end{rightcolumn}
\end{samepage}
\begin{samepage}
\begin{leftcolumn*}
\EnglishColumn{tassa taṃ vedanaṃ abhinandato abhivadato ajjhosāya tiṭṭhato uppajjati nandī.}
\hspace{0pt}\end{leftcolumn*}

\begin{rightcolumn}\PaliColumn{As he does so, delight arises in him.}
\hspace{0pt}
\end{rightcolumn}
\end{samepage}
\begin{samepage}
\begin{leftcolumn*}
\EnglishColumn{yā vedanāsu nandī tadupādānaṃ,}
\hspace{0pt}\end{leftcolumn*}

\begin{rightcolumn}\PaliColumn{Now delight in feelings is clinging.}
\hspace{0pt}
\end{rightcolumn}
\end{samepage}
\begin{samepage}
\begin{leftcolumn*}
\EnglishColumn{tassupādānapaccayā bhavo,}
\hspace{0pt}\end{leftcolumn*}

\begin{rightcolumn}\PaliColumn{With his clinging as condition, being [comes to be];}
\hspace{0pt}
\end{rightcolumn}
\end{samepage}
\begin{samepage}
\begin{leftcolumn*}
\EnglishColumn{bhavapaccayā jāti,}
\hspace{0pt}\end{leftcolumn*}

\begin{rightcolumn}\PaliColumn{with being as condition, birth;}
\hspace{0pt}
\end{rightcolumn}
\end{samepage}
\begin{samepage}
\begin{leftcolumn*}
\EnglishColumn{jātipaccayā jarāmaraṇaṃ sokaparidevadukkhadomanassupāyāsā sambhavanti.}
\hspace{0pt}\end{leftcolumn*}

\begin{rightcolumn}\PaliColumn{with birth as condition, ageing and death, sorrow, lamentation, pain, grief, and despair come to be.}
\hspace{0pt}
\end{rightcolumn}
\end{samepage}
\begin{samepage}
\begin{leftcolumn*}
\EnglishColumn{evametassa kevalassa dukkhakkhandhassa samudayo hoti.}
\hspace{0pt}\end{leftcolumn*}

\begin{rightcolumn}\PaliColumn{Such is the origin of this whole mass of suffering.}
\hspace{0pt}
\end{rightcolumn}
\end{samepage}
\vskip 0.2in
\begin{samepage}
\begin{leftcolumn*}
\EnglishColumn{kāyena phoṭṭhabbaṃ phusitvā disvā piyarūpe phoṭṭhabbe sārajjati,}
\hspace{0pt}\end{leftcolumn*}

\begin{rightcolumn}\PaliColumn{On touching a tangible with the body, he lusts after it if it is pleasing;}
\hspace{0pt}
\end{rightcolumn}
\end{samepage}
\begin{samepage}
\begin{leftcolumn*}
\EnglishColumn{appiyarūpe phoṭṭhabbe byāpajjati,}
\hspace{0pt}\end{leftcolumn*}

\begin{rightcolumn}\PaliColumn{he dislikes it if it is unpleasing.}
\hspace{0pt}
\end{rightcolumn}
\end{samepage}
\begin{samepage}
\begin{leftcolumn*}
\EnglishColumn{anupaṭṭhitakāyasati ca viharati parittacetaso.}
\hspace{0pt}\end{leftcolumn*}

\begin{rightcolumn}\PaliColumn{He abides with mindfulness of the body unestablished, with a limited mind,}
\hspace{0pt}
\end{rightcolumn}
\end{samepage}
\begin{samepage}
\begin{leftcolumn*}
\EnglishColumn{tañca cetovimuttiṃ paññāvimuttiṃ yathābhūtaṃ nappajānāti - yatthassa te pāpakā akusalā dhammā aparisesā nirujjhanti.}
\hspace{0pt}\end{leftcolumn*}

\begin{rightcolumn}\PaliColumn{and he does not understand as it actually is the deliverance of mind and deliverance by wisdom wherein those evil unwholesome states cease without remainder.}
\hspace{0pt}
\end{rightcolumn}
\end{samepage}
\begin{samepage}
\begin{leftcolumn*}
\EnglishColumn{so evaṃ anurodhavirodhaṃ samāpanno yaṃ kiñci vedanaṃ vedeti sukhaṃ vā dukkhaṃ vā adukkhamasukhaṃ vā, so taṃ vedanaṃ abhinandati abhivadati ajjhosāya tiṭṭhati.}
\hspace{0pt}\end{leftcolumn*}

\begin{rightcolumn}\PaliColumn{Engaged as he is in favouring and opposing, whatever feeling he feels—whether pleasant or painful or neither-painful-nor-pleasant—he delights in that feeling, welcomes it, and remains holding to it.}
\hspace{0pt}
\end{rightcolumn}
\end{samepage}
\begin{samepage}
\begin{leftcolumn*}
\EnglishColumn{tassa taṃ vedanaṃ abhinandato abhivadato ajjhosāya tiṭṭhato uppajjati nandī.}
\hspace{0pt}\end{leftcolumn*}

\begin{rightcolumn}\PaliColumn{As he does so, delight arises in him.}
\hspace{0pt}
\end{rightcolumn}
\end{samepage}
\begin{samepage}
\begin{leftcolumn*}
\EnglishColumn{yā vedanāsu nandī tadupādānaṃ,}
\hspace{0pt}\end{leftcolumn*}

\begin{rightcolumn}\PaliColumn{Now delight in feelings is clinging.}
\hspace{0pt}
\end{rightcolumn}
\end{samepage}
\begin{samepage}
\begin{leftcolumn*}
\EnglishColumn{tassupādānapaccayā bhavo,}
\hspace{0pt}\end{leftcolumn*}

\begin{rightcolumn}\PaliColumn{With his clinging as condition, being [comes to be];}
\hspace{0pt}
\end{rightcolumn}
\end{samepage}
\begin{samepage}
\begin{leftcolumn*}
\EnglishColumn{bhavapaccayā jāti,}
\hspace{0pt}\end{leftcolumn*}

\begin{rightcolumn}\PaliColumn{with being as condition, birth;}
\hspace{0pt}
\end{rightcolumn}
\end{samepage}
\begin{samepage}
\begin{leftcolumn*}
\EnglishColumn{jātipaccayā jarāmaraṇaṃ sokaparidevadukkhadomanassupāyāsā sambhavanti.}
\hspace{0pt}\end{leftcolumn*}

\begin{rightcolumn}\PaliColumn{with birth as condition, ageing and death, sorrow, lamentation, pain, grief, and despair come to be.}
\hspace{0pt}
\end{rightcolumn}
\end{samepage}
\begin{samepage}
\begin{leftcolumn*}
\EnglishColumn{evametassa kevalassa dukkhakkhandhassa samudayo hoti.}
\hspace{0pt}\end{leftcolumn*}

\begin{rightcolumn}\PaliColumn{Such is the origin of this whole mass of suffering.}
\hspace{0pt}
\end{rightcolumn}
\end{samepage}
\vskip 0.2in
\begin{samepage}
\begin{leftcolumn*}
\EnglishColumn{manasā dhammaṃ viññāya piyarūpe dhamme sārajjati,}
\hspace{0pt}\end{leftcolumn*}

\begin{rightcolumn}\PaliColumn{On cognizing a mind-object with the mind, he lusts after it if it is pleasing;}
\hspace{0pt}
\end{rightcolumn}
\end{samepage}
\begin{samepage}
\begin{leftcolumn*}
\EnglishColumn{appiyarūpe dhamme byāpajjati,}
\hspace{0pt}\end{leftcolumn*}

\begin{rightcolumn}\PaliColumn{he dislikes it if it is unpleasing.}
\hspace{0pt}
\end{rightcolumn}
\end{samepage}
\begin{samepage}
\begin{leftcolumn*}
\EnglishColumn{anupaṭṭhitakāyasati ca viharati parittacetaso.}
\hspace{0pt}\end{leftcolumn*}

\begin{rightcolumn}\PaliColumn{He abides with mindfulness of the body unestablished, with a limited mind,}
\hspace{0pt}
\end{rightcolumn}
\end{samepage}
\begin{samepage}
\begin{leftcolumn*}
\EnglishColumn{tañca cetovimuttiṃ paññāvimuttiṃ yathābhūtaṃ nappajānāti - yatthassa te pāpakā akusalā dhammā aparisesā nirujjhanti.}
\hspace{0pt}\end{leftcolumn*}

\begin{rightcolumn}\PaliColumn{and he does not understand as it actually is the deliverance of mind and deliverance by wisdom wherein those evil unwholesome states cease without remainder.}
\hspace{0pt}
\end{rightcolumn}
\end{samepage}
\begin{samepage}
\begin{leftcolumn*}
\EnglishColumn{so evaṃ anurodhavirodhaṃ samāpanno yaṃ kiñci vedanaṃ vedeti sukhaṃ vā dukkhaṃ vā adukkhamasukhaṃ vā, so taṃ vedanaṃ abhinandati abhivadati ajjhosāya tiṭṭhati.}
\hspace{0pt}\end{leftcolumn*}

\begin{rightcolumn}\PaliColumn{Engaged as he is in favouring and opposing, whatever feeling he feels—whether pleasant or painful or neither-painful-nor-pleasant—he delights in that feeling, welcomes it, and remains holding to it.}
\hspace{0pt}
\end{rightcolumn}
\end{samepage}
\begin{samepage}
\begin{leftcolumn*}
\EnglishColumn{tassa taṃ vedanaṃ abhinandato abhivadato ajjhosāya tiṭṭhato uppajjati nandī.}
\hspace{0pt}\end{leftcolumn*}

\begin{rightcolumn}\PaliColumn{As he does so, delight arises in him.}
\hspace{0pt}
\end{rightcolumn}
\end{samepage}
\begin{samepage}
\begin{leftcolumn*}
\EnglishColumn{yā vedanāsu nandī tadupādānaṃ,}
\hspace{0pt}\end{leftcolumn*}

\begin{rightcolumn}\PaliColumn{Now delight in feelings is clinging.}
\hspace{0pt}
\end{rightcolumn}
\end{samepage}
\begin{samepage}
\begin{leftcolumn*}
\EnglishColumn{tassupādānapaccayā bhavo,}
\hspace{0pt}\end{leftcolumn*}

\begin{rightcolumn}\PaliColumn{With his clinging as condition, being [comes to be];}
\hspace{0pt}
\end{rightcolumn}
\end{samepage}
\begin{samepage}
\begin{leftcolumn*}
\EnglishColumn{bhavapaccayā jāti,}
\hspace{0pt}\end{leftcolumn*}

\begin{rightcolumn}\PaliColumn{with being as condition, birth;}
\hspace{0pt}
\end{rightcolumn}
\end{samepage}
\begin{samepage}
\begin{leftcolumn*}
\EnglishColumn{jātipaccayā jarāmaraṇaṃ sokaparidevadukkhadomanassupāyāsā sambhavanti.}
\hspace{0pt}\end{leftcolumn*}

\begin{rightcolumn}\PaliColumn{with birth as condition, ageing and death, sorrow, lamentation, pain, grief, and despair come to be.}
\hspace{0pt}
\end{rightcolumn}
\end{samepage}
\begin{samepage}
\begin{leftcolumn*}
\EnglishColumn{evametassa kevalassa dukkhakkhandhassa samudayo hoti.}
\hspace{0pt}\end{leftcolumn*}

\begin{rightcolumn}\PaliColumn{Such is the origin of this whole mass of suffering.}
\hspace{0pt}
\end{rightcolumn}
\end{samepage}
\vskip 0.2in
\begin{samepage}
\begin{leftcolumn*}
\EnglishColumn{“idha, bhikkhave, tathāgato loke uppajjati arahaṃ sammāsambuddho vijjācaraṇasampanno sugato lokavidū anuttaro purisadammasārathi satthā devamanussānaṃ buddho bhagavā.}
\hspace{0pt}\end{leftcolumn*}

\begin{rightcolumn}\PaliColumn{“Here, bhikkhus, a Tathāgata appears in the world, accomplished, fully enlightened, perfect in true knowledge and conduct, sublime, knower of worlds, incomparable leader of persons to be tamed, teacher of gods and humans, enlightened, blessed.}
\hspace{0pt}
\end{rightcolumn}
\end{samepage}
\begin{samepage}
\begin{leftcolumn*}
\EnglishColumn{so imaṃ lokaṃ sadevakaṃ samārakaṃ sabrahmakaṃ sassamaṇabrāhmaṇiṃ pajaṃ sadevamanussaṃ sayaṃ abhiññā sacchikatvā pavedeti.}
\hspace{0pt}\end{leftcolumn*}

\begin{rightcolumn}\PaliColumn{He declares this world with its gods, its Māras, and its Brahmās, this generation with its recluses and brahmins, its princes and its people, which he has himself realised with direct knowledge.}
\hspace{0pt}
\end{rightcolumn}
\end{samepage}
\begin{samepage}
\begin{leftcolumn*}
\EnglishColumn{so dhammaṃ deseti ādikalyāṇaṃ majjhekalyāṇaṃ pariyosānakalyāṇaṃ sātthaṃ sabyañjanaṃ; kevalaparipuṇṇaṃ parisuddhaṃ brahmacariyaṃ pakāseti.}
\hspace{0pt}\end{leftcolumn*}

\begin{rightcolumn}\PaliColumn{He teaches the Dhamma good in the beginning, good in the middle, and good in the end, with the right meaning and phrasing, and he reveals a holy life that is utterly perfect and pure.}
\hspace{0pt}
\end{rightcolumn}
\end{samepage}
\vskip 0.2in
\begin{samepage}
\begin{leftcolumn*}
\EnglishColumn{taṃ dhammaṃ suṇāti gahapati vā gahapatiputto vā aññatarasmiṃ vā kule paccājāto.}
\hspace{0pt}\end{leftcolumn*}

\begin{rightcolumn}\PaliColumn{“A householder or householder’s son or one born in some other clan hears that Dhamma.}
\hspace{0pt}
\end{rightcolumn}
\end{samepage}
\begin{samepage}
\begin{leftcolumn*}
\EnglishColumn{so taṃ dhammaṃ sutvā tathāgate saddhaṃ paṭilabhati.}
\hspace{0pt}\end{leftcolumn*}

\begin{rightcolumn}\PaliColumn{On hearing the Dhamma he acquires faith in the Tathāgata.}
\hspace{0pt}
\end{rightcolumn}
\end{samepage}
\begin{samepage}
\begin{leftcolumn*}
\EnglishColumn{so tena saddhāpaṭilābhena samannāgato iti paṭisañcikkhati -}
\hspace{0pt}\end{leftcolumn*}

\begin{rightcolumn}\PaliColumn{Possessing that faith, he considers thus:}
\hspace{0pt}
\end{rightcolumn}
\end{samepage}
\begin{samepage}
\begin{leftcolumn*}
\EnglishColumn{‘sambādho gharāvāso rajāpatho, abbhokāso pabbajjā.}
\hspace{0pt}\end{leftcolumn*}

\begin{rightcolumn}\PaliColumn{‘Household life is crowded and dusty; life gone forth is wide open.}
\hspace{0pt}
\end{rightcolumn}
\end{samepage}
\begin{samepage}
\begin{leftcolumn*}
\EnglishColumn{nayidaṃ sukaraṃ agāraṃ ajjhāvasatā ekantaparipuṇṇaṃ ekantaparisuddhaṃ saṅkhalikhitaṃ brahmacariyaṃ carituṃ.}
\hspace{0pt}\end{leftcolumn*}

\begin{rightcolumn}\PaliColumn{It is not easy, while living in a home, to lead the holy life utterly perfect and pure as a polished shell.}
\hspace{0pt}
\end{rightcolumn}
\end{samepage}
\begin{samepage}
\begin{leftcolumn*}
\EnglishColumn{yaṃnūnāhaṃ kesamassuṃ ohāretvā, kāsāyāni vatthāni acchādetvā, agārasmā anagāriyaṃ pabbajeyyan’”ti.}
\hspace{0pt}\end{leftcolumn*}

\begin{rightcolumn}\PaliColumn{Suppose I shave off my hair and beard, put on the yellow robe, and go forth from the home life into homelessness.’}
\hspace{0pt}
\end{rightcolumn}
\end{samepage}
\begin{samepage}
\begin{leftcolumn*}
\EnglishColumn{so aparena samayena appaṃ vā bhogakkhandhaṃ pahāya, mahantaṃ vā bhogakkhandhaṃ pahāya, appaṃ vā ñātiparivaṭṭaṃ pahāya, mahantaṃ vā ñātiparivaṭṭaṃ pahāya, kesamassuṃ ohāretvā, kāsāyāni vatthāni acchādetvā, agārasmā anagāriyaṃ pabbajati.}
\hspace{0pt}\end{leftcolumn*}

\begin{rightcolumn}\PaliColumn{On a later occasion, abandoning a small or a large fortune, abandoning a small or a large circle of relatives, he shaves off his hair and beard, puts on the yellow robe, and goes forth from the home life into homelessness.}
\hspace{0pt}
\end{rightcolumn}
\end{samepage}
\vskip 0.2in
\begin{samepage}
\begin{leftcolumn*}
\EnglishColumn{“so evaṃ pabbajito samāno bhikkhūnaṃ sikkhāsājīvasamāpanno pāṇātipātaṃ pahāya pāṇātipātā paṭivirato hoti, nihitadaṇḍo nihitasattho lajjī dayāpanno sabbapāṇabhūtahitānukampī viharati.}
\hspace{0pt}\end{leftcolumn*}

\begin{rightcolumn}\PaliColumn{“Having thus gone forth and possessing the bhikkhu’s training and way of life, abandoning the killing of living beings, he abstains from killing living beings; with rod and weapon laid aside, conscientious, merciful, he abides compassionate to all living beings.}
\hspace{0pt}
\end{rightcolumn}
\end{samepage}
\vskip 0.2in
\begin{samepage}
\begin{leftcolumn*}
\EnglishColumn{“adinnādānaṃ pahāya adinnādānā paṭivirato hoti, dinnādāyī dinnapāṭikaṅkhī athenena sucibhūtena attanā viharati.}
\hspace{0pt}\end{leftcolumn*}

\begin{rightcolumn}\PaliColumn{Abandoning the taking of what is not given, he abstains from taking what is not given; taking only what is given, expecting only what is given, by not stealing he abides in purity.}
\hspace{0pt}
\end{rightcolumn}
\end{samepage}
\vskip 0.2in
\begin{samepage}
\begin{leftcolumn*}
\EnglishColumn{“abrahmacariyaṃ pahāya brahmacārī hoti, ārācārī virato methunā gāmadhammā.}
\hspace{0pt}\end{leftcolumn*}

\begin{rightcolumn}\PaliColumn{Abandoning incelibacy, he observes celibacy, living apart, abstaining from the vulgar practice of sexual intercourse.}
\hspace{0pt}
\end{rightcolumn}
\end{samepage}
\vskip 0.2in
\begin{samepage}
\begin{leftcolumn*}
\EnglishColumn{“musāvādaṃ pahāya musāvādā paṭivirato hoti, saccavādī saccasandho theto paccayiko avisaṃvādako lokassa.}
\hspace{0pt}\end{leftcolumn*}

\begin{rightcolumn}\PaliColumn{“Abandoning false speech, he abstains from false speech; he speaks truth, adheres to truth, is trustworthy and reliable, one who is no deceiver of the world.}
\hspace{0pt}
\end{rightcolumn}
\end{samepage}
\begin{samepage}
\begin{leftcolumn*}
\EnglishColumn{“pisuṇaṃ vācaṃ pahāya pisuṇāya vācāya paṭivirato hoti - ito sutvā na amutra akkhātā imesaṃ bhedāya, amutra vā sutvā na imesaṃ akkhātā amūsaṃ bhedāya. iti bhinnānaṃ vā sandhātā, sahitānaṃ vā anuppadātā samaggārāmo samaggarato samagganandī, samaggakaraṇiṃ vācaṃ bhāsitā hoti.}
\hspace{0pt}\end{leftcolumn*}

\begin{rightcolumn}\PaliColumn{Abandoning malicious speech, he abstains from malicious speech; he does not repeat elsewhere what he has heard here in order to divide [those people] from these, nor does he repeat to these people what he has heard elsewhere in order to divide [these people] from those; thus he is one who reunites those who are divided, a promoter of friendships, who enjoys concord, rejoices in concord, delights in concord, a speaker of words that promote concord.}
\hspace{0pt}
\end{rightcolumn}
\end{samepage}
\begin{samepage}
\begin{leftcolumn*}
\EnglishColumn{“pharusaṃ vācaṃ pahāya pharusāya vācāya paṭivirato hoti - yā sā vācā nelā kaṇṇasukhā pemanīyā hadayaṅgamā porī bahujanakantā bahujanamanāpā tathārūpiṃ vācaṃ bhāsitā hoti.}
\hspace{0pt}\end{leftcolumn*}

\begin{rightcolumn}\PaliColumn{Abandoning harsh speech, he abstains from harsh speech; he speaks such words as are gentle, pleasing to the ear, and loveable, as go to the heart, are courteous, desired by many and agreeable to many.}
\hspace{0pt}
\end{rightcolumn}
\end{samepage}
\begin{samepage}
\begin{leftcolumn*}
\EnglishColumn{“samphappalāpaṃ pahāya samphappalāpā paṭivirato hoti, kālavādī bhūtavādī atthavādī dhammavādī vinayavādī, nidhānavatiṃ vācaṃ bhāsitā kālena, sāpadesaṃ pariyantavatiṃ atthasaṃhitaṃ.}
\hspace{0pt}\end{leftcolumn*}

\begin{rightcolumn}\PaliColumn{Abandoning gossip, he abstains from gossip; he speaks at the right time, speaks what is fact, speaks on what is good, speaks on the Dhamma and the Discipline; at the right time he speaks such words as are worth recording, reasonable, moderate, and beneficial.}
\hspace{0pt}
\end{rightcolumn}
\end{samepage}
\vskip 0.2in
\begin{samepage}
\begin{leftcolumn*}
\EnglishColumn{“so bījagāmabhūtagāmasamārambhā paṭivirato hoti,}
\hspace{0pt}\end{leftcolumn*}

\begin{rightcolumn}\PaliColumn{“He abstains from injuring seeds and plants.}
\hspace{0pt}
\end{rightcolumn}
\end{samepage}
\begin{samepage}
\begin{leftcolumn*}
\EnglishColumn{ekabhattiko hoti rattūparato, virato vikālabhojanā.}
\hspace{0pt}\end{leftcolumn*}

\begin{rightcolumn}\PaliColumn{He practises eating only one meal a day, abstaining from eating at night and outside the proper time.}
\hspace{0pt}
\end{rightcolumn}
\end{samepage}
\begin{samepage}
\begin{leftcolumn*}
\EnglishColumn{naccagītavāditavisūkadassanā paṭivirato hoti,}
\hspace{0pt}\end{leftcolumn*}

\begin{rightcolumn}\PaliColumn{He abstains from dancing, singing, music, and theatrical shows.}
\hspace{0pt}
\end{rightcolumn}
\end{samepage}
\begin{samepage}
\begin{leftcolumn*}
\EnglishColumn{mālāgandhavilepanadhāraṇamaṇḍanavibhūsanaṭṭhānā paṭivirato hoti,}
\hspace{0pt}\end{leftcolumn*}

\begin{rightcolumn}\PaliColumn{He abstains from wearing garlands, smartening himself with scent, and embellishing himself with unguents.}
\hspace{0pt}
\end{rightcolumn}
\end{samepage}
\begin{samepage}
\begin{leftcolumn*}
\EnglishColumn{uccāsayanamahāsayanā paṭivirato hoti,}
\hspace{0pt}\end{leftcolumn*}

\begin{rightcolumn}\PaliColumn{He abstains from high and large couches.}
\hspace{0pt}
\end{rightcolumn}
\end{samepage}
\begin{samepage}
\begin{leftcolumn*}
\EnglishColumn{jātarūparajatapaṭiggahaṇā paṭivirato hoti,}
\hspace{0pt}\end{leftcolumn*}

\begin{rightcolumn}\PaliColumn{He abstains from accepting gold and silver.}
\hspace{0pt}
\end{rightcolumn}
\end{samepage}
\begin{samepage}
\begin{leftcolumn*}
\EnglishColumn{āmakadhaññapaṭiggahaṇā paṭivirato hoti,}
\hspace{0pt}\end{leftcolumn*}

\begin{rightcolumn}\PaliColumn{He abstains from accepting raw grain.}
\hspace{0pt}
\end{rightcolumn}
\end{samepage}
\begin{samepage}
\begin{leftcolumn*}
\EnglishColumn{āmakamaṃsapaṭiggahaṇā paṭivirato hoti,}
\hspace{0pt}\end{leftcolumn*}

\begin{rightcolumn}\PaliColumn{He abstains from accepting raw meat.}
\hspace{0pt}
\end{rightcolumn}
\end{samepage}
\begin{samepage}
\begin{leftcolumn*}
\EnglishColumn{itthikumārikapaṭiggahaṇā paṭivirato hoti,}
\hspace{0pt}\end{leftcolumn*}

\begin{rightcolumn}\PaliColumn{He abstains from accepting women and girls.}
\hspace{0pt}
\end{rightcolumn}
\end{samepage}
\begin{samepage}
\begin{leftcolumn*}
\EnglishColumn{dāsidāsapaṭiggahaṇā paṭivirato hoti,}
\hspace{0pt}\end{leftcolumn*}

\begin{rightcolumn}\PaliColumn{He abstains from accepting men and women slaves.}
\hspace{0pt}
\end{rightcolumn}
\end{samepage}
\begin{samepage}
\begin{leftcolumn*}
\EnglishColumn{ajeḷakapaṭiggahaṇā paṭivirato hoti,}
\hspace{0pt}\end{leftcolumn*}

\begin{rightcolumn}\PaliColumn{He abstains from accepting goats and sheep.}
\hspace{0pt}
\end{rightcolumn}
\end{samepage}
\begin{samepage}
\begin{leftcolumn*}
\EnglishColumn{kukkuṭasūkarapaṭiggahaṇā paṭivirato hoti,}
\hspace{0pt}\end{leftcolumn*}

\begin{rightcolumn}\PaliColumn{He abstains from accepting fowl and pigs.}
\hspace{0pt}
\end{rightcolumn}
\end{samepage}
\begin{samepage}
\begin{leftcolumn*}
\EnglishColumn{hatthigavāssavaḷavapaṭiggahaṇā paṭivirato hoti,}
\hspace{0pt}\end{leftcolumn*}

\begin{rightcolumn}\PaliColumn{He abstains from accepting elephants, cattle, horses, and mares.}
\hspace{0pt}
\end{rightcolumn}
\end{samepage}
\begin{samepage}
\begin{leftcolumn*}
\EnglishColumn{khettavatthupaṭiggahaṇā paṭivirato hoti,}
\hspace{0pt}\end{leftcolumn*}

\begin{rightcolumn}\PaliColumn{He abstains from accepting fields and land.}
\hspace{0pt}
\end{rightcolumn}
\end{samepage}
\begin{samepage}
\begin{leftcolumn*}
\EnglishColumn{dūteyyapahiṇagamanānuyogā paṭivirato hoti,}
\hspace{0pt}\end{leftcolumn*}

\begin{rightcolumn}\PaliColumn{He abstains from going on errands and running messages.}
\hspace{0pt}
\end{rightcolumn}
\end{samepage}
\begin{samepage}
\begin{leftcolumn*}
\EnglishColumn{kayavikkayā paṭivirato hoti,}
\hspace{0pt}\end{leftcolumn*}

\begin{rightcolumn}\PaliColumn{He abstains from buying and selling.}
\hspace{0pt}
\end{rightcolumn}
\end{samepage}
\begin{samepage}
\begin{leftcolumn*}
\EnglishColumn{tulākūṭakaṃsakūṭamānakūṭā paṭivirato hoti,}
\hspace{0pt}\end{leftcolumn*}

\begin{rightcolumn}\PaliColumn{He abstains from false weights, false metals, and false measures.}
\hspace{0pt}
\end{rightcolumn}
\end{samepage}
\begin{samepage}
\begin{leftcolumn*}
\EnglishColumn{ukkoṭanavañcana-nikati-sāciyogā paṭivirato hoti,}
\hspace{0pt}\end{leftcolumn*}

\begin{rightcolumn}\PaliColumn{He abstains from accepting bribes, deceiving, defrauding, and trickery.}
\hspace{0pt}
\end{rightcolumn}
\end{samepage}
\begin{samepage}
\begin{leftcolumn*}
\EnglishColumn{chedana-vadhabandhanaviparāmosa-ālopa-sahasākārā paṭivirato hoti.}
\hspace{0pt}\end{leftcolumn*}

\begin{rightcolumn}\PaliColumn{He abstains from wounding, murdering, binding, brigandage, plunder, and violence.}
\hspace{0pt}
\end{rightcolumn}
\end{samepage}
\vskip 0.2in
\begin{samepage}
\begin{leftcolumn*}
\EnglishColumn{“so santuṭṭho hoti kāyaparihārikena cīvarena kucchiparihārikena piṇḍapātena.}
\hspace{0pt}\end{leftcolumn*}

\begin{rightcolumn}\PaliColumn{“He becomes content with robes to protect his body and with almsfood to maintain his stomach,}
\hspace{0pt}
\end{rightcolumn}
\end{samepage}
\begin{samepage}
\begin{leftcolumn*}
\EnglishColumn{so yena yeneva pakkamati samādāyeva pakkamati.}
\hspace{0pt}\end{leftcolumn*}

\begin{rightcolumn}\PaliColumn{and wherever he goes, he sets out taking only these with him.}
\hspace{0pt}
\end{rightcolumn}
\end{samepage}
\begin{samepage}
\begin{leftcolumn*}
\EnglishColumn{seyyathāpi nāma pakkhī sakuṇo yena yeneva ḍeti sapattabhārova ḍeti,}
\hspace{0pt}\end{leftcolumn*}

\begin{rightcolumn}\PaliColumn{Just as a bird, wherever it goes, flies with its wings as its only burden,}
\hspace{0pt}
\end{rightcolumn}
\end{samepage}
\begin{samepage}
\begin{leftcolumn*}
\EnglishColumn{evameva bhikkhu santuṭṭho hoti kāyaparihārikena cīvarena, kucchiparihārikena piṇḍapātena.}
\hspace{0pt}\end{leftcolumn*}

\begin{rightcolumn}\PaliColumn{so too the bhikkhu becomes content with robes to protect his body and with almsfood to maintain his stomach,}
\hspace{0pt}
\end{rightcolumn}
\end{samepage}
\begin{samepage}
\begin{leftcolumn*}
\EnglishColumn{so yena yeneva pakkamati samādāyeva pakkamati.}
\hspace{0pt}\end{leftcolumn*}

\begin{rightcolumn}\PaliColumn{and wherever he goes, he sets out taking only these with him.}
\hspace{0pt}
\end{rightcolumn}
\end{samepage}
\begin{samepage}
\begin{leftcolumn*}
\EnglishColumn{so iminā ariyena sīlakkhandhena samannāgato ajjhattaṃ anavajjasukhaṃ paṭisaṃvedeti.}
\hspace{0pt}\end{leftcolumn*}

\begin{rightcolumn}\PaliColumn{Possessing this aggregate of noble virtue, he experiences within himself a bliss that is blameless.}
\hspace{0pt}
\end{rightcolumn}
\end{samepage}
\vskip 0.2in
\begin{samepage}
\begin{leftcolumn*}
\EnglishColumn{“so cakkhunā rūpaṃ disvā na nimittaggāhī hoti nānubyañjanaggāhī.}
\hspace{0pt}\end{leftcolumn*}

\begin{rightcolumn}\PaliColumn{“On seeing a form with the eye, he does not grasp at its signs and features.}
\hspace{0pt}
\end{rightcolumn}
\end{samepage}
\begin{samepage}
\begin{leftcolumn*}
\EnglishColumn{yatvādhikaraṇamenaṃ cakkhundriyaṃ asaṃvutaṃ viharantaṃ abhijjhādomanassā pāpakā akusalā dhammā anvāssaveyyuṃ tassa saṃvarāya paṭipajjati, rakkhati cakkhundriyaṃ, cakkhundriye saṃvaraṃ āpajjati.}
\hspace{0pt}\end{leftcolumn*}

\begin{rightcolumn}\PaliColumn{Since, if he left the eye faculty unguarded, evil unwholesome states of covetousness and grief might invade him, he practises the way of its restraint, he guards the eye faculty, he undertakes the restraint of the eye faculty.}
\hspace{0pt}
\end{rightcolumn}
\end{samepage}
\vskip 0.2in
\begin{samepage}
\begin{leftcolumn*}
\EnglishColumn{sotena saddaṃ sutvā na nimittaggāhī hoti nānubyañjanaggāhī.}
\hspace{0pt}\end{leftcolumn*}

\begin{rightcolumn}\PaliColumn{On hearing a sound with the ear, he does not grasp at its signs and features.}
\hspace{0pt}
\end{rightcolumn}
\end{samepage}
\begin{samepage}
\begin{leftcolumn*}
\EnglishColumn{yatvādhikaraṇamenaṃ cakkhundriyaṃ asaṃvutaṃ viharantaṃ abhijjhādomanassā pāpakā akusalā dhammā anvāssaveyyuṃ tassa saṃvarāya paṭipajjati, rakkhati cakkhundriyaṃ, cakkhundriye saṃvaraṃ āpajjati.}
\hspace{0pt}\end{leftcolumn*}

\begin{rightcolumn}\PaliColumn{Since, if he left the ear faculty unguarded, evil unwholesome states of covetousness and grief might invade him, he practises the way of its restraint, he guards the ear faculty, he undertakes the restraint of the ear faculty.}
\hspace{0pt}
\end{rightcolumn}
\end{samepage}
\vskip 0.2in
\begin{samepage}
\begin{leftcolumn*}
\EnglishColumn{ghānena gandhaṃ ghāyitvā na nimittaggāhī hoti nānubyañjanaggāhī.}
\hspace{0pt}\end{leftcolumn*}

\begin{rightcolumn}\PaliColumn{On smelling an odour with the nose, he does not grasp at its signs and features.}
\hspace{0pt}
\end{rightcolumn}
\end{samepage}
\begin{samepage}
\begin{leftcolumn*}
\EnglishColumn{yatvādhikaraṇamenaṃ gandhindriyaṃ asaṃvutaṃ viharantaṃ abhijjhādomanassā pāpakā akusalā dhammā anvāssaveyyuṃ tassa saṃvarāya paṭipajjati, rakkhati gandhindriyaṃ, gandhindriye saṃvaraṃ āpajjati.}
\hspace{0pt}\end{leftcolumn*}

\begin{rightcolumn}\PaliColumn{Since, if he left the nose faculty unguarded, evil unwholesome states of covetousness and grief might invade him, he practises the way of its restraint, he guards the nose faculty, he undertakes the restraint of the nose faculty.}
\hspace{0pt}
\end{rightcolumn}
\end{samepage}
\vskip 0.2in
\begin{samepage}
\begin{leftcolumn*}
\EnglishColumn{jivhāya rasaṃ sāyitvā na nimittaggāhī hoti nānubyañjanaggāhī.}
\hspace{0pt}\end{leftcolumn*}

\begin{rightcolumn}\PaliColumn{On tasting a flavour with the tongue, he does not grasp at its signs and features.}
\hspace{0pt}
\end{rightcolumn}
\end{samepage}
\begin{samepage}
\begin{leftcolumn*}
\EnglishColumn{yatvādhikaraṇamenaṃ jivhindriye asaṃvutaṃ viharantaṃ abhijjhādomanassā pāpakā akusalā dhammā anvāssaveyyuṃ tassa saṃvarāya paṭipajjati, rakkhati jivhindriyaṃ, jivhindriye saṃvaraṃ āpajjati.}
\hspace{0pt}\end{leftcolumn*}

\begin{rightcolumn}\PaliColumn{Since, if he left the tongue faculty unguarded, evil unwholesome states of covetousness and grief might invade him, he practises the way of its restraint, he guards the tongue faculty, he undertakes the restraint of the tongue faculty.}
\hspace{0pt}
\end{rightcolumn}
\end{samepage}
\vskip 0.2in
\begin{samepage}
\begin{leftcolumn*}
\EnglishColumn{kāyena phoṭṭhabbaṃ phusitvā disvā na nimittaggāhī hoti nānubyañjanaggāhī.}
\hspace{0pt}\end{leftcolumn*}

\begin{rightcolumn}\PaliColumn{On touching a tangible with the body, he does not grasp at its signs and features.}
\hspace{0pt}
\end{rightcolumn}
\end{samepage}
\begin{samepage}
\begin{leftcolumn*}
\EnglishColumn{yatvādhikaraṇamenaṃ phoṭṭhabbindriyaṃ asaṃvutaṃ viharantaṃ abhijjhādomanassā pāpakā akusalā dhammā anvāssaveyyuṃ tassa saṃvarāya paṭipajjati, rakkhati phoṭṭhabbindriyaṃ, phoṭṭhabbindriye saṃvaraṃ āpajjati.}
\hspace{0pt}\end{leftcolumn*}

\begin{rightcolumn}\PaliColumn{Since, if he left the body faculty unguarded, evil unwholesome states of covetousness and grief might invade him, he practises the way of its restraint, he guards the body faculty, he undertakes the restraint of the body faculty.}
\hspace{0pt}
\end{rightcolumn}
\end{samepage}
\vskip 0.2in
\begin{samepage}
\begin{leftcolumn*}
\EnglishColumn{manasā dhammaṃ viññāya na nimittaggāhī hoti nānubyañjanaggāhī.}
\hspace{0pt}\end{leftcolumn*}

\begin{rightcolumn}\PaliColumn{On cognizing a mind-object with the mind, he does not grasp at its signs and features.}
\hspace{0pt}
\end{rightcolumn}
\end{samepage}
\begin{samepage}
\begin{leftcolumn*}
\EnglishColumn{yatvādhikaraṇamenaṃ manindriyaṃ asaṃvutaṃ viharantaṃ abhijjhādomanassā pāpakā akusalā dhammā anvāssaveyyuṃ tassa saṃvarāya paṭipajjati, rakkhati manindriyaṃ manindriye saṃvaraṃ āpajjati.}
\hspace{0pt}\end{leftcolumn*}

\begin{rightcolumn}\PaliColumn{Since, if he left the mind faculty unguarded, evil unwholesome states of covetousness and grief might invade him, he practises the way of its restraint, he guards the mind faculty, he undertakes the restraint of the mind faculty.}
\hspace{0pt}
\end{rightcolumn}
\end{samepage}
\begin{samepage}
\begin{leftcolumn*}
\EnglishColumn{so iminā ariyena indriyasaṃvarena samannāgato ajjhattaṃ abyāsekasukhaṃ paṭisaṃvedeti.}
\hspace{0pt}\end{leftcolumn*}

\begin{rightcolumn}\PaliColumn{Possessing this noble restraint of the faculties, he experiences within himself a bliss that is unsullied.}
\hspace{0pt}
\end{rightcolumn}
\end{samepage}
\vskip 0.2in
\begin{samepage}
\begin{leftcolumn*}
\EnglishColumn{“so abhikkante paṭikkante sampajānakārī hoti,}
\hspace{0pt}\end{leftcolumn*}

\begin{rightcolumn}\PaliColumn{“He becomes one who acts in full awareness when going forward and returning;}
\hspace{0pt}
\end{rightcolumn}
\end{samepage}
\begin{samepage}
\begin{leftcolumn*}
\EnglishColumn{ālokite vilokite sampajānakārī hoti,}
\hspace{0pt}\end{leftcolumn*}

\begin{rightcolumn}\PaliColumn{who acts in full awareness when looking ahead and looking away;}
\hspace{0pt}
\end{rightcolumn}
\end{samepage}
\begin{samepage}
\begin{leftcolumn*}
\EnglishColumn{samiñjite pasārite sampajānakārī hoti,}
\hspace{0pt}\end{leftcolumn*}

\begin{rightcolumn}\PaliColumn{who acts in full awareness when flexing and extending his limbs;}
\hspace{0pt}
\end{rightcolumn}
\end{samepage}
\begin{samepage}
\begin{leftcolumn*}
\EnglishColumn{saṅghāṭipattacīvaradhāraṇe sampajānakārī hoti,}
\hspace{0pt}\end{leftcolumn*}

\begin{rightcolumn}\PaliColumn{who acts in full awareness when wearing his robes and carrying his outer robe and bowl;}
\hspace{0pt}
\end{rightcolumn}
\end{samepage}
\begin{samepage}
\begin{leftcolumn*}
\EnglishColumn{asite pīte khāyite sāyite sampajānakārī hoti,}
\hspace{0pt}\end{leftcolumn*}

\begin{rightcolumn}\PaliColumn{who acts in full awareness when eating, drinking, consuming food, and tasting;}
\hspace{0pt}
\end{rightcolumn}
\end{samepage}
\begin{samepage}
\begin{leftcolumn*}
\EnglishColumn{uccārapassāvakamme sampajānakārī hoti,}
\hspace{0pt}\end{leftcolumn*}

\begin{rightcolumn}\PaliColumn{who acts in full awareness when defecating and urinating;}
\hspace{0pt}
\end{rightcolumn}
\end{samepage}
\begin{samepage}
\begin{leftcolumn*}
\EnglishColumn{gate ṭhite nisinne sutte jāgarite bhāsite tuṇhībhāve sampajānakārī hoti.}
\hspace{0pt}\end{leftcolumn*}

\begin{rightcolumn}\PaliColumn{who acts in full awareness when walking, standing, sitting, falling asleep, waking up, talking, and keeping silent.}
\hspace{0pt}
\end{rightcolumn}
\end{samepage}
\vskip 0.2in
\begin{samepage}
\begin{leftcolumn*}
\EnglishColumn{“so iminā ca ariyena sīlakkhandhena samannāgato, (imāya ca ariyāya santuṭṭhiyā samannāgato), iminā ca ariyena indriyasaṃvarena samannāgato, iminā ca ariyena satisampajaññena samannāgato, vivittaṃ senāsanaṃ bhajati -}
\hspace{0pt}\end{leftcolumn*}

\begin{rightcolumn}\PaliColumn{“Possessing this aggregate of noble virtue, and this noble restraint of the faculties, and possessing this noble mindfulness and full awareness, he resorts to a secluded resting place:}
\hspace{0pt}
\end{rightcolumn}
\end{samepage}
\begin{samepage}
\begin{leftcolumn*}
\EnglishColumn{araññaṃ rukkhamūlaṃ pabbataṃ kandaraṃ giriguhaṃ susānaṃ vanapatthaṃ abbhokāsaṃ palālapuñjaṃ.}
\hspace{0pt}\end{leftcolumn*}

\begin{rightcolumn}\PaliColumn{the forest, the root of a tree, a mountain, a ravine, a hillside cave, a charnel ground, a jungle thicket, an open space, a heap of straw.}
\hspace{0pt}
\end{rightcolumn}
\end{samepage}
\vskip 0.2in
\begin{samepage}
\begin{leftcolumn*}
\EnglishColumn{jo pacchābhattaṃ piṇḍapātapaṭikkanto nisīdati pallaṅkaṃ ābhujitvā, ujuṃ kāyaṃ paṇidhāya, parimukhaṃ satiṃ upaṭṭhapetvā.}
\hspace{0pt}\end{leftcolumn*}

\begin{rightcolumn}\PaliColumn{“On returning from his almsround, after his meal he sits down, folding his legs crosswise, setting his body erect, and establishing mindfulness before him.}
\hspace{0pt}
\end{rightcolumn}
\end{samepage}
\begin{samepage}
\begin{leftcolumn*}
\EnglishColumn{so abhijjhaṃ loke pahāya vigatābhijjhena cetasā viharati,}
\hspace{0pt}\end{leftcolumn*}

\begin{rightcolumn}\PaliColumn{Abandoning covetousness for the world, he abides with a mind free from covetousness;}
\hspace{0pt}
\end{rightcolumn}
\end{samepage}
\begin{samepage}
\begin{leftcolumn*}
\EnglishColumn{abhijjhāya cittaṃ parisodheti;}
\hspace{0pt}\end{leftcolumn*}

\begin{rightcolumn}\PaliColumn{he purifies his mind from covetousness.}
\hspace{0pt}
\end{rightcolumn}
\end{samepage}
\begin{samepage}
\begin{leftcolumn*}
\EnglishColumn{byāpādapadosaṃ pahāya abyāpannacitto viharati, sabbapāṇabhūtahitānukampī,}
\hspace{0pt}\end{leftcolumn*}

\begin{rightcolumn}\PaliColumn{Abandoning ill will and hatred, he abides with a mind free from ill will, compassionate for the welfare of all living beings;}
\hspace{0pt}
\end{rightcolumn}
\end{samepage}
\begin{samepage}
\begin{leftcolumn*}
\EnglishColumn{byāpādapadosā cittaṃ parisodheti;}
\hspace{0pt}\end{leftcolumn*}

\begin{rightcolumn}\PaliColumn{he purifies his mind from ill will and hatred.}
\hspace{0pt}
\end{rightcolumn}
\end{samepage}
\begin{samepage}
\begin{leftcolumn*}
\EnglishColumn{thīnamiddhaṃ pahāya vigatathīnamiddho viharati ālokasaññī, sato sampajāno,}
\hspace{0pt}\end{leftcolumn*}

\begin{rightcolumn}\PaliColumn{Abandoning sloth and torpor, he abides free from sloth and torpor, percipient of light, mindful and fully aware;}
\hspace{0pt}
\end{rightcolumn}
\end{samepage}
\begin{samepage}
\begin{leftcolumn*}
\EnglishColumn{thīnamiddhā cittaṃ parisodheti;}
\hspace{0pt}\end{leftcolumn*}

\begin{rightcolumn}\PaliColumn{he purifies his mind from sloth and torpor.}
\hspace{0pt}
\end{rightcolumn}
\end{samepage}
\begin{samepage}
\begin{leftcolumn*}
\EnglishColumn{uddhaccakukkuccaṃ pahāya anuddhato viharati ajjhattaṃ vūpasantacitto,}
\hspace{0pt}\end{leftcolumn*}

\begin{rightcolumn}\PaliColumn{Abandoning restlessness and remorse, he abides unagitated with a mind inwardly peaceful;}
\hspace{0pt}
\end{rightcolumn}
\end{samepage}
\begin{samepage}
\begin{leftcolumn*}
\EnglishColumn{uddhaccakukkuccā cittaṃ parisodheti;}
\hspace{0pt}\end{leftcolumn*}

\begin{rightcolumn}\PaliColumn{he purifies his mind from restlessness and remorse.}
\hspace{0pt}
\end{rightcolumn}
\end{samepage}
\begin{samepage}
\begin{leftcolumn*}
\EnglishColumn{vicikicchaṃ pahāya tiṇṇavicikiccho viharati akathaṃkathī kusalesu dhammesu,}
\hspace{0pt}\end{leftcolumn*}

\begin{rightcolumn}\PaliColumn{Abandoning doubt, he abides having gone beyond doubt, unperplexed about wholesome states;}
\hspace{0pt}
\end{rightcolumn}
\end{samepage}
\begin{samepage}
\begin{leftcolumn*}
\EnglishColumn{vicikicchāya cittaṃ parisodheti.}
\hspace{0pt}\end{leftcolumn*}

\begin{rightcolumn}\PaliColumn{he purifies his mind from doubt.}
\hspace{0pt}
\end{rightcolumn}
\end{samepage}
\vskip 0.2in
\begin{samepage}
\begin{leftcolumn*}
\EnglishColumn{“so ime pañca nīvaraṇe pahāya cetaso upakkilese paññāya dubbalīkaraṇe, vivicceva kāmehi vivicca akusalehi dhammehi savitakkaṃ savicāraṃ vivekajaṃ pītisukhaṃ paṭhamaṃ jhānaṃ upasampajja viharati.}
\hspace{0pt}\end{leftcolumn*}

\begin{rightcolumn}\PaliColumn{“Having thus abandoned these five hindrances, imperfections of the mind that weaken wisdom, quite secluded from sensual pleasures, secluded from unwholesome states, he enters upon and abides in the first jhāna, which is accompanied by applied and sustained thought, with rapture and pleasure born of seclusion.}
\hspace{0pt}
\end{rightcolumn}
\end{samepage}
\vskip 0.2in
\begin{samepage}
\begin{leftcolumn*}
\EnglishColumn{puna caparaṁ, bhikkhave, bhikkhu vitakkavicārānaṁ vūpasamā ajjhattaṁ sampasādanaṁ cetaso ekodibhāvaṁ avitakkaṁ avicāraṁ samādhijaṁ pītisukhaṁ dutiyaṁ jhānaṁ upasampajja viharati.}
\hspace{0pt}\end{leftcolumn*}

\begin{rightcolumn}\PaliColumn{With the stilling of applied and sustained thought, he enters upon and abides in the second jhāna, which has self-confidence and singleness of mind without applied and sustained thought, with rapture and pleasure born of concentration.}
\hspace{0pt}
\end{rightcolumn}
\end{samepage}
\vskip 0.2in
\begin{samepage}
\begin{leftcolumn*}
\EnglishColumn{puna caparaṁ, bhikkhave, bhikkhu pītiyā ca virāgā upekkhako ca viharati sato ca sampajāno, sukhañca kāyena paṭisaṁvedeti, yaṁ taṁ ariyā ācikkhanti: “upekkhako satimā sukhavihārī”ti, tatiyaṁ jhānaṁ upasampajja viharati.}
\hspace{0pt}\end{leftcolumn*}

\begin{rightcolumn}\PaliColumn{With the fading away as well of rapture, a bhikkhu abides in equanimity, and mindful and fully aware, still feeling pleasure with the body, he enters upon and abides in the third jhāna, on account of which noble ones announce: ‘He has a pleasant abiding who has equanimity and is mindful.’}
\hspace{0pt}
\end{rightcolumn}
\end{samepage}
\vskip 0.2in
\begin{samepage}
\begin{leftcolumn*}
\EnglishColumn{puna caparaṁ, bhikkhave, bhikkhu sukhassa ca pahānā pubbeva somanassadomanassānaṁ atthaṅgamā adukkhamasukhaṁ upekkhāsatipārisuddhiṁ catutthaṁ jhānaṁ upasampajja viharati.}
\hspace{0pt}\end{leftcolumn*}

\begin{rightcolumn}\PaliColumn{With the abandoning of pleasure and pain, and with the previous disappearance of joy and grief, a bhikkhu enters upon and abides in the fourth jhāna, which has neither-pain-nor-pleasure and purity of mindfulness due to equanimity.}
\hspace{0pt}
\end{rightcolumn}
\end{samepage}
\vskip 0.2in
\begin{samepage}
\begin{leftcolumn*}
\EnglishColumn{“so cakkhunā rūpaṃ disvā piyarūpe rūpe na sārajjati,}
\hspace{0pt}\end{leftcolumn*}

\begin{rightcolumn}\PaliColumn{“On seeing a form with the eye, he does not lust after it if it is pleasing;}
\hspace{0pt}
\end{rightcolumn}
\end{samepage}
\begin{samepage}
\begin{leftcolumn*}
\EnglishColumn{appiyarūpe rūpe na byāpajjati,}
\hspace{0pt}\end{leftcolumn*}

\begin{rightcolumn}\PaliColumn{he does not dislike it if it is unpleasing.}
\hspace{0pt}
\end{rightcolumn}
\end{samepage}
\begin{samepage}
\begin{leftcolumn*}
\EnglishColumn{upaṭṭhitakāyasati ca viharati appamāṇacetaso.}
\hspace{0pt}\end{leftcolumn*}

\begin{rightcolumn}\PaliColumn{He abides with mindfulness of the body established, with an immeasurable mind,}
\hspace{0pt}
\end{rightcolumn}
\end{samepage}
\begin{samepage}
\begin{leftcolumn*}
\EnglishColumn{tañca cetovimuttiṃ paññāvimuttiṃ yathābhūtaṃ pajānāti - yatthassa te pāpakā akusalā dhammā aparisesā nirujjhanti.}
\hspace{0pt}\end{leftcolumn*}

\begin{rightcolumn}\PaliColumn{and he understands as it actually is the deliverance of mind and deliverance by wisdom wherein those evil unwholesome states cease without remainder.}
\hspace{0pt}
\end{rightcolumn}
\end{samepage}
\begin{samepage}
\begin{leftcolumn*}
\EnglishColumn{so evaṃ anurodhavirodhavippahīno yaṃ kiñci vedanaṃ vedeti, sukhaṃ vā dukkhaṃ vā adukkhamasukhaṃ vā, so taṃ vedanaṃ nābhinandati nābhivadati nājjhosāya tiṭṭhati.}
\hspace{0pt}\end{leftcolumn*}

\begin{rightcolumn}\PaliColumn{Having thus abandoned favouring and opposing, whatever feeling he feels, whether pleasant or painful or neither-painful-nor-pleasant, he does not delight in that feeling, welcome it, or remain holding to it.}
\hspace{0pt}
\end{rightcolumn}
\end{samepage}
\begin{samepage}
\begin{leftcolumn*}
\EnglishColumn{tassa taṃ vedanaṃ anabhinandato anabhivadato anajjhosāya tiṭṭhato yā vedanāsu nandī sā nirujjhati.}
\hspace{0pt}\end{leftcolumn*}

\begin{rightcolumn}\PaliColumn{As he does not do so, delight in feelings ceases in him.}
\hspace{0pt}
\end{rightcolumn}
\end{samepage}
\begin{samepage}
\begin{leftcolumn*}
\EnglishColumn{tassa nandīnirodhā upādānanirodho,}
\hspace{0pt}\end{leftcolumn*}

\begin{rightcolumn}\PaliColumn{With the cessation of his delight comes cessation of clinging;}
\hspace{0pt}
\end{rightcolumn}
\end{samepage}
\begin{samepage}
\begin{leftcolumn*}
\EnglishColumn{upādānanirodhā bhavanirodho,}
\hspace{0pt}\end{leftcolumn*}

\begin{rightcolumn}\PaliColumn{with the cessation of clinging, cessation of being;}
\hspace{0pt}
\end{rightcolumn}
\end{samepage}
\begin{samepage}
\begin{leftcolumn*}
\EnglishColumn{bhavanirodhā jātinirodho,}
\hspace{0pt}\end{leftcolumn*}

\begin{rightcolumn}\PaliColumn{with the cessation of being, cessation of birth;}
\hspace{0pt}
\end{rightcolumn}
\end{samepage}
\begin{samepage}
\begin{leftcolumn*}
\EnglishColumn{jātinirodhā jarāmaraṇaṃ sokaparidevadukkhadomanassupāyāsā nirujjhanti.}
\hspace{0pt}\end{leftcolumn*}

\begin{rightcolumn}\PaliColumn{with the cessation of birth, ageing and death, sorrow, lamentation, pain, grief, and despair cease.}
\hspace{0pt}
\end{rightcolumn}
\end{samepage}
\begin{samepage}
\begin{leftcolumn*}
\EnglishColumn{evametassa kevalassa dukkhakkhandhassa nirodho hoti.}
\hspace{0pt}\end{leftcolumn*}

\begin{rightcolumn}\PaliColumn{Such is the cessation of this whole mass of suffering.}
\hspace{0pt}
\end{rightcolumn}
\end{samepage}
\vskip 0.2in
\begin{samepage}
\begin{leftcolumn*}
\EnglishColumn{sotena saddaṃ sutvā piyarūpe sadde na sārajjati,}
\hspace{0pt}\end{leftcolumn*}

\begin{rightcolumn}\PaliColumn{“On hearing a sound with the ear, he does not lust after it if it is pleasing;}
\hspace{0pt}
\end{rightcolumn}
\end{samepage}
\begin{samepage}
\begin{leftcolumn*}
\EnglishColumn{appiyarūpe sadde na byāpajjati,}
\hspace{0pt}\end{leftcolumn*}

\begin{rightcolumn}\PaliColumn{he does not dislike it if it is unpleasing.}
\hspace{0pt}
\end{rightcolumn}
\end{samepage}
\begin{samepage}
\begin{leftcolumn*}
\EnglishColumn{upaṭṭhitakāyasati ca viharati appamāṇacetaso.}
\hspace{0pt}\end{leftcolumn*}

\begin{rightcolumn}\PaliColumn{He abides with mindfulness of the body established, with an immeasurable mind,}
\hspace{0pt}
\end{rightcolumn}
\end{samepage}
\begin{samepage}
\begin{leftcolumn*}
\EnglishColumn{tañca cetovimuttiṃ paññāvimuttiṃ yathābhūtaṃ pajānāti - yatthassa te pāpakā akusalā dhammā aparisesā nirujjhanti.}
\hspace{0pt}\end{leftcolumn*}

\begin{rightcolumn}\PaliColumn{and he understands as it actually is the deliverance of mind and deliverance by wisdom wherein those evil unwholesome states cease without remainder.}
\hspace{0pt}
\end{rightcolumn}
\end{samepage}
\begin{samepage}
\begin{leftcolumn*}
\EnglishColumn{so evaṃ anurodhavirodhavippahīno yaṃ kiñci vedanaṃ vedeti, sukhaṃ vā dukkhaṃ vā adukkhamasukhaṃ vā, so taṃ vedanaṃ nābhinandati nābhivadati nājjhosāya tiṭṭhati.}
\hspace{0pt}\end{leftcolumn*}

\begin{rightcolumn}\PaliColumn{Having thus abandoned favouring and opposing, whatever feeling he feels, whether pleasant or painful or neither-painful-nor-pleasant, he does not delight in that feeling, welcome it, or remain holding to it.}
\hspace{0pt}
\end{rightcolumn}
\end{samepage}
\begin{samepage}
\begin{leftcolumn*}
\EnglishColumn{tassa taṃ vedanaṃ anabhinandato anabhivadato anajjhosāya tiṭṭhato yā vedanāsu nandī sā nirujjhati.}
\hspace{0pt}\end{leftcolumn*}

\begin{rightcolumn}\PaliColumn{As he does not do so, delight in feelings ceases in him.}
\hspace{0pt}
\end{rightcolumn}
\end{samepage}
\begin{samepage}
\begin{leftcolumn*}
\EnglishColumn{tassa nandīnirodhā upādānanirodho,}
\hspace{0pt}\end{leftcolumn*}

\begin{rightcolumn}\PaliColumn{With the cessation of his delight comes cessation of clinging;}
\hspace{0pt}
\end{rightcolumn}
\end{samepage}
\begin{samepage}
\begin{leftcolumn*}
\EnglishColumn{upādānanirodhā bhavanirodho,}
\hspace{0pt}\end{leftcolumn*}

\begin{rightcolumn}\PaliColumn{with the cessation of clinging, cessation of being;}
\hspace{0pt}
\end{rightcolumn}
\end{samepage}
\begin{samepage}
\begin{leftcolumn*}
\EnglishColumn{bhavanirodhā jātinirodho,}
\hspace{0pt}\end{leftcolumn*}

\begin{rightcolumn}\PaliColumn{with the cessation of being, cessation of birth;}
\hspace{0pt}
\end{rightcolumn}
\end{samepage}
\begin{samepage}
\begin{leftcolumn*}
\EnglishColumn{jātinirodhā jarāmaraṇaṃ sokaparidevadukkhadomanassupāyāsā nirujjhanti.}
\hspace{0pt}\end{leftcolumn*}

\begin{rightcolumn}\PaliColumn{with the cessation of birth, ageing and death, sorrow, lamentation, pain, grief, and despair cease.}
\hspace{0pt}
\end{rightcolumn}
\end{samepage}
\begin{samepage}
\begin{leftcolumn*}
\EnglishColumn{evametassa kevalassa dukkhakkhandhassa nirodho hoti.}
\hspace{0pt}\end{leftcolumn*}

\begin{rightcolumn}\PaliColumn{Such is the cessation of this whole mass of suffering.}
\hspace{0pt}
\end{rightcolumn}
\end{samepage}
\vskip 0.2in
\begin{samepage}
\begin{leftcolumn*}
\EnglishColumn{ghānena gandhaṃ ghāyitvā piyarūpe gandhe na sārajjati,}
\hspace{0pt}\end{leftcolumn*}

\begin{rightcolumn}\PaliColumn{On smelling an odour with the nose, he does not lust after it if it is pleasing;}
\hspace{0pt}
\end{rightcolumn}
\end{samepage}
\begin{samepage}
\begin{leftcolumn*}
\EnglishColumn{appiyarūpe gandhe na byāpajjati,}
\hspace{0pt}\end{leftcolumn*}

\begin{rightcolumn}\PaliColumn{he does not dislike it if it is unpleasing.}
\hspace{0pt}
\end{rightcolumn}
\end{samepage}
\begin{samepage}
\begin{leftcolumn*}
\EnglishColumn{upaṭṭhitakāyasati ca viharati appamāṇacetaso.}
\hspace{0pt}\end{leftcolumn*}

\begin{rightcolumn}\PaliColumn{He abides with mindfulness of the body established, with an immeasurable mind,}
\hspace{0pt}
\end{rightcolumn}
\end{samepage}
\begin{samepage}
\begin{leftcolumn*}
\EnglishColumn{tañca cetovimuttiṃ paññāvimuttiṃ yathābhūtaṃ pajānāti - yatthassa te pāpakā akusalā dhammā aparisesā nirujjhanti.}
\hspace{0pt}\end{leftcolumn*}

\begin{rightcolumn}\PaliColumn{and he understands as it actually is the deliverance of mind and deliverance by wisdom wherein those evil unwholesome states cease without remainder.}
\hspace{0pt}
\end{rightcolumn}
\end{samepage}
\begin{samepage}
\begin{leftcolumn*}
\EnglishColumn{so evaṃ anurodhavirodhavippahīno yaṃ kiñci vedanaṃ vedeti, sukhaṃ vā dukkhaṃ vā adukkhamasukhaṃ vā, so taṃ vedanaṃ nābhinandati nābhivadati nājjhosāya tiṭṭhati.}
\hspace{0pt}\end{leftcolumn*}

\begin{rightcolumn}\PaliColumn{Having thus abandoned favouring and opposing, whatever feeling he feels, whether pleasant or painful or neither-painful-nor-pleasant, he does not delight in that feeling, welcome it, or remain holding to it.}
\hspace{0pt}
\end{rightcolumn}
\end{samepage}
\begin{samepage}
\begin{leftcolumn*}
\EnglishColumn{tassa taṃ vedanaṃ anabhinandato anabhivadato anajjhosāya tiṭṭhato yā vedanāsu nandī sā nirujjhati.}
\hspace{0pt}\end{leftcolumn*}

\begin{rightcolumn}\PaliColumn{As he does not do so, delight in feelings ceases in him.}
\hspace{0pt}
\end{rightcolumn}
\end{samepage}
\begin{samepage}
\begin{leftcolumn*}
\EnglishColumn{tassa nandīnirodhā upādānanirodho,}
\hspace{0pt}\end{leftcolumn*}

\begin{rightcolumn}\PaliColumn{With the cessation of his delight comes cessation of clinging;}
\hspace{0pt}
\end{rightcolumn}
\end{samepage}
\begin{samepage}
\begin{leftcolumn*}
\EnglishColumn{upādānanirodhā bhavanirodho,}
\hspace{0pt}\end{leftcolumn*}

\begin{rightcolumn}\PaliColumn{with the cessation of clinging, cessation of being;}
\hspace{0pt}
\end{rightcolumn}
\end{samepage}
\begin{samepage}
\begin{leftcolumn*}
\EnglishColumn{bhavanirodhā jātinirodho,}
\hspace{0pt}\end{leftcolumn*}

\begin{rightcolumn}\PaliColumn{with the cessation of being, cessation of birth;}
\hspace{0pt}
\end{rightcolumn}
\end{samepage}
\begin{samepage}
\begin{leftcolumn*}
\EnglishColumn{jātinirodhā jarāmaraṇaṃ sokaparidevadukkhadomanassupāyāsā nirujjhanti.}
\hspace{0pt}\end{leftcolumn*}

\begin{rightcolumn}\PaliColumn{with the cessation of birth, ageing and death, sorrow, lamentation, pain, grief, and despair cease.}
\hspace{0pt}
\end{rightcolumn}
\end{samepage}
\begin{samepage}
\begin{leftcolumn*}
\EnglishColumn{evametassa kevalassa dukkhakkhandhassa nirodho hoti.}
\hspace{0pt}\end{leftcolumn*}

\begin{rightcolumn}\PaliColumn{Such is the cessation of this whole mass of suffering.}
\hspace{0pt}
\end{rightcolumn}
\end{samepage}
\vskip 0.2in
\begin{samepage}
\begin{leftcolumn*}
\EnglishColumn{jivhāya rasaṃ sāyitvā piyarūpe rase na sārajjati,}
\hspace{0pt}\end{leftcolumn*}

\begin{rightcolumn}\PaliColumn{On tasting a flavour with the tongue, he does not lust after it if it is pleasing;}
\hspace{0pt}
\end{rightcolumn}
\end{samepage}
\begin{samepage}
\begin{leftcolumn*}
\EnglishColumn{appiyarūpe rase na byāpajjati,}
\hspace{0pt}\end{leftcolumn*}

\begin{rightcolumn}\PaliColumn{he does not dislike it if it is unpleasing.}
\hspace{0pt}
\end{rightcolumn}
\end{samepage}
\begin{samepage}
\begin{leftcolumn*}
\EnglishColumn{upaṭṭhitakāyasati ca viharati appamāṇacetaso.}
\hspace{0pt}\end{leftcolumn*}

\begin{rightcolumn}\PaliColumn{He abides with mindfulness of the body established, with an immeasurable mind,}
\hspace{0pt}
\end{rightcolumn}
\end{samepage}
\begin{samepage}
\begin{leftcolumn*}
\EnglishColumn{tañca cetovimuttiṃ paññāvimuttiṃ yathābhūtaṃ pajānāti - yatthassa te pāpakā akusalā dhammā aparisesā nirujjhanti.}
\hspace{0pt}\end{leftcolumn*}

\begin{rightcolumn}\PaliColumn{and he understands as it actually is the deliverance of mind and deliverance by wisdom wherein those evil unwholesome states cease without remainder.}
\hspace{0pt}
\end{rightcolumn}
\end{samepage}
\begin{samepage}
\begin{leftcolumn*}
\EnglishColumn{so evaṃ anurodhavirodhavippahīno yaṃ kiñci vedanaṃ vedeti, sukhaṃ vā dukkhaṃ vā adukkhamasukhaṃ vā, so taṃ vedanaṃ nābhinandati nābhivadati nājjhosāya tiṭṭhati.}
\hspace{0pt}\end{leftcolumn*}

\begin{rightcolumn}\PaliColumn{Having thus abandoned favouring and opposing, whatever feeling he feels, whether pleasant or painful or neither-painful-nor-pleasant, he does not delight in that feeling, welcome it, or remain holding to it.}
\hspace{0pt}
\end{rightcolumn}
\end{samepage}
\begin{samepage}
\begin{leftcolumn*}
\EnglishColumn{tassa taṃ vedanaṃ anabhinandato anabhivadato anajjhosāya tiṭṭhato yā vedanāsu nandī sā nirujjhati.}
\hspace{0pt}\end{leftcolumn*}

\begin{rightcolumn}\PaliColumn{As he does not do so, delight in feelings ceases in him.}
\hspace{0pt}
\end{rightcolumn}
\end{samepage}
\begin{samepage}
\begin{leftcolumn*}
\EnglishColumn{tassa nandīnirodhā upādānanirodho,}
\hspace{0pt}\end{leftcolumn*}

\begin{rightcolumn}\PaliColumn{With the cessation of his delight comes cessation of clinging;}
\hspace{0pt}
\end{rightcolumn}
\end{samepage}
\begin{samepage}
\begin{leftcolumn*}
\EnglishColumn{upādānanirodhā bhavanirodho,}
\hspace{0pt}\end{leftcolumn*}

\begin{rightcolumn}\PaliColumn{with the cessation of clinging, cessation of being;}
\hspace{0pt}
\end{rightcolumn}
\end{samepage}
\begin{samepage}
\begin{leftcolumn*}
\EnglishColumn{bhavanirodhā jātinirodho,}
\hspace{0pt}\end{leftcolumn*}

\begin{rightcolumn}\PaliColumn{with the cessation of being, cessation of birth;}
\hspace{0pt}
\end{rightcolumn}
\end{samepage}
\begin{samepage}
\begin{leftcolumn*}
\EnglishColumn{jātinirodhā jarāmaraṇaṃ sokaparidevadukkhadomanassupāyāsā nirujjhanti.}
\hspace{0pt}\end{leftcolumn*}

\begin{rightcolumn}\PaliColumn{with the cessation of birth, ageing and death, sorrow, lamentation, pain, grief, and despair cease.}
\hspace{0pt}
\end{rightcolumn}
\end{samepage}
\begin{samepage}
\begin{leftcolumn*}
\EnglishColumn{evametassa kevalassa dukkhakkhandhassa nirodho hoti.}
\hspace{0pt}\end{leftcolumn*}

\begin{rightcolumn}\PaliColumn{Such is the cessation of this whole mass of suffering.}
\hspace{0pt}
\end{rightcolumn}
\end{samepage}
\vskip 0.2in
\begin{samepage}
\begin{leftcolumn*}
\EnglishColumn{kāyena phoṭṭhabbaṃ phusitvā piyarūpe phoṭṭhabbe na sārajjati,}
\hspace{0pt}\end{leftcolumn*}

\begin{rightcolumn}\PaliColumn{On touching a tangible with the body, he does not lust after it if it is pleasing;}
\hspace{0pt}
\end{rightcolumn}
\end{samepage}
\begin{samepage}
\begin{leftcolumn*}
\EnglishColumn{appiyarūpe phoṭṭhabbe na byāpajjati,}
\hspace{0pt}\end{leftcolumn*}

\begin{rightcolumn}\PaliColumn{he does not dislike it if it is unpleasing.}
\hspace{0pt}
\end{rightcolumn}
\end{samepage}
\begin{samepage}
\begin{leftcolumn*}
\EnglishColumn{upaṭṭhitakāyasati ca viharati appamāṇacetaso.}
\hspace{0pt}\end{leftcolumn*}

\begin{rightcolumn}\PaliColumn{He abides with mindfulness of the body established, with an immeasurable mind,}
\hspace{0pt}
\end{rightcolumn}
\end{samepage}
\begin{samepage}
\begin{leftcolumn*}
\EnglishColumn{tañca cetovimuttiṃ paññāvimuttiṃ yathābhūtaṃ pajānāti - yatthassa te pāpakā akusalā dhammā aparisesā nirujjhanti.}
\hspace{0pt}\end{leftcolumn*}

\begin{rightcolumn}\PaliColumn{and he understands as it actually is the deliverance of mind and deliverance by wisdom wherein those evil unwholesome states cease without remainder.}
\hspace{0pt}
\end{rightcolumn}
\end{samepage}
\begin{samepage}
\begin{leftcolumn*}
\EnglishColumn{so evaṃ anurodhavirodhavippahīno yaṃ kiñci vedanaṃ vedeti, sukhaṃ vā dukkhaṃ vā adukkhamasukhaṃ vā, so taṃ vedanaṃ nābhinandati nābhivadati nājjhosāya tiṭṭhati.}
\hspace{0pt}\end{leftcolumn*}

\begin{rightcolumn}\PaliColumn{Having thus abandoned favouring and opposing, whatever feeling he feels, whether pleasant or painful or neither-painful-nor-pleasant, he does not delight in that feeling, welcome it, or remain holding to it.}
\hspace{0pt}
\end{rightcolumn}
\end{samepage}
\begin{samepage}
\begin{leftcolumn*}
\EnglishColumn{tassa taṃ vedanaṃ anabhinandato anabhivadato anajjhosāya tiṭṭhato yā vedanāsu nandī sā nirujjhati.}
\hspace{0pt}\end{leftcolumn*}

\begin{rightcolumn}\PaliColumn{As he does not do so, delight in feelings ceases in him.}
\hspace{0pt}
\end{rightcolumn}
\end{samepage}
\begin{samepage}
\begin{leftcolumn*}
\EnglishColumn{tassa nandīnirodhā upādānanirodho,}
\hspace{0pt}\end{leftcolumn*}

\begin{rightcolumn}\PaliColumn{With the cessation of his delight comes cessation of clinging;}
\hspace{0pt}
\end{rightcolumn}
\end{samepage}
\begin{samepage}
\begin{leftcolumn*}
\EnglishColumn{upādānanirodhā bhavanirodho,}
\hspace{0pt}\end{leftcolumn*}

\begin{rightcolumn}\PaliColumn{with the cessation of clinging, cessation of being;}
\hspace{0pt}
\end{rightcolumn}
\end{samepage}
\begin{samepage}
\begin{leftcolumn*}
\EnglishColumn{bhavanirodhā jātinirodho,}
\hspace{0pt}\end{leftcolumn*}

\begin{rightcolumn}\PaliColumn{with the cessation of being, cessation of birth;}
\hspace{0pt}
\end{rightcolumn}
\end{samepage}
\begin{samepage}
\begin{leftcolumn*}
\EnglishColumn{jātinirodhā jarāmaraṇaṃ sokaparidevadukkhadomanassupāyāsā nirujjhanti.}
\hspace{0pt}\end{leftcolumn*}

\begin{rightcolumn}\PaliColumn{with the cessation of birth, ageing and death, sorrow, lamentation, pain, grief, and despair cease.}
\hspace{0pt}
\end{rightcolumn}
\end{samepage}
\begin{samepage}
\begin{leftcolumn*}
\EnglishColumn{evametassa kevalassa dukkhakkhandhassa nirodho hoti.}
\hspace{0pt}\end{leftcolumn*}

\begin{rightcolumn}\PaliColumn{Such is the cessation of this whole mass of suffering.}
\hspace{0pt}
\end{rightcolumn}
\end{samepage}
\vskip 0.2in
\begin{samepage}
\begin{leftcolumn*}
\EnglishColumn{manasā dhammaṃ viññāya piyarūpe dhamme na sārajjati,}
\hspace{0pt}\end{leftcolumn*}

\begin{rightcolumn}\PaliColumn{On cognizing a mind-object with the mind, he does not lust after it if it is pleasing;}
\hspace{0pt}
\end{rightcolumn}
\end{samepage}
\begin{samepage}
\begin{leftcolumn*}
\EnglishColumn{appiyarūpe dhamme na byāpajjati,}
\hspace{0pt}\end{leftcolumn*}

\begin{rightcolumn}\PaliColumn{he does not dislike it if it is unpleasing.}
\hspace{0pt}
\end{rightcolumn}
\end{samepage}
\begin{samepage}
\begin{leftcolumn*}
\EnglishColumn{upaṭṭhitakāyasati ca viharati appamāṇacetaso.}
\hspace{0pt}\end{leftcolumn*}

\begin{rightcolumn}\PaliColumn{He abides with mindfulness of the body established, with an immeasurable mind,}
\hspace{0pt}
\end{rightcolumn}
\end{samepage}
\begin{samepage}
\begin{leftcolumn*}
\EnglishColumn{tañca cetovimuttiṃ paññāvimuttiṃ yathābhūtaṃ pajānāti - yatthassa te pāpakā akusalā dhammā aparisesā nirujjhanti.}
\hspace{0pt}\end{leftcolumn*}

\begin{rightcolumn}\PaliColumn{and he understands as it actually is the deliverance of mind and deliverance by wisdom wherein those evil unwholesome states cease without remainder.}
\hspace{0pt}
\end{rightcolumn}
\end{samepage}
\begin{samepage}
\begin{leftcolumn*}
\EnglishColumn{so evaṃ anurodhavirodhavippahīno yaṃ kiñci vedanaṃ vedeti, sukhaṃ vā dukkhaṃ vā adukkhamasukhaṃ vā, so taṃ vedanaṃ nābhinandati nābhivadati nājjhosāya tiṭṭhati.}
\hspace{0pt}\end{leftcolumn*}

\begin{rightcolumn}\PaliColumn{Having thus abandoned favouring and opposing, whatever feeling he feels, whether pleasant or painful or neither-painful-nor-pleasant, he does not delight in that feeling, welcome it, or remain holding to it.}
\hspace{0pt}
\end{rightcolumn}
\end{samepage}
\begin{samepage}
\begin{leftcolumn*}
\EnglishColumn{tassa taṃ vedanaṃ anabhinandato anabhivadato anajjhosāya tiṭṭhato yā vedanāsu nandī sā nirujjhati.}
\hspace{0pt}\end{leftcolumn*}

\begin{rightcolumn}\PaliColumn{As he does not do so, delight in feelings ceases in him.}
\hspace{0pt}
\end{rightcolumn}
\end{samepage}
\begin{samepage}
\begin{leftcolumn*}
\EnglishColumn{tassa nandīnirodhā upādānanirodho,}
\hspace{0pt}\end{leftcolumn*}

\begin{rightcolumn}\PaliColumn{With the cessation of his delight comes cessation of clinging;}
\hspace{0pt}
\end{rightcolumn}
\end{samepage}
\begin{samepage}
\begin{leftcolumn*}
\EnglishColumn{upādānanirodhā bhavanirodho,}
\hspace{0pt}\end{leftcolumn*}

\begin{rightcolumn}\PaliColumn{with the cessation of clinging, cessation of being;}
\hspace{0pt}
\end{rightcolumn}
\end{samepage}
\begin{samepage}
\begin{leftcolumn*}
\EnglishColumn{bhavanirodhā jātinirodho,}
\hspace{0pt}\end{leftcolumn*}

\begin{rightcolumn}\PaliColumn{with the cessation of being, cessation of birth;}
\hspace{0pt}
\end{rightcolumn}
\end{samepage}
\begin{samepage}
\begin{leftcolumn*}
\EnglishColumn{jātinirodhā jarāmaraṇaṃ sokaparidevadukkhadomanassupāyāsā nirujjhanti.}
\hspace{0pt}\end{leftcolumn*}

\begin{rightcolumn}\PaliColumn{with the cessation of birth, ageing and death, sorrow, lamentation, pain, grief, and despair cease.}
\hspace{0pt}
\end{rightcolumn}
\end{samepage}
\begin{samepage}
\begin{leftcolumn*}
\EnglishColumn{evametassa kevalassa dukkhakkhandhassa nirodho hoti.}
\hspace{0pt}\end{leftcolumn*}

\begin{rightcolumn}\PaliColumn{Such is the cessation of this whole mass of suffering.}
\hspace{0pt}
\end{rightcolumn}
\end{samepage}
\vskip 0.2in
\begin{samepage}
\begin{leftcolumn*}
\EnglishColumn{imaṃ kho me tumhe, bhikkhave, saṃkhittena taṇhāsaṅkhayavimuttiṃ dhāretha, sātiṃ pana bhikkhuṃ kevaṭṭaputtaṃ mahātaṇhājālataṇhāsaṅghāṭappaṭimukkan”ti.}
\hspace{0pt}\end{leftcolumn*}

\begin{rightcolumn}\PaliColumn{“Bhikkhus, remember this [discourse] of mine briefly as deliverance in the destruction of craving; but [remember] the bhikkhu Sāti, son of a fisherman, as caught up in a vast net of craving, in the trammel of craving.”}
\hspace{0pt}
\end{rightcolumn}
\end{samepage}
\vskip 0.2in
\begin{samepage}
\begin{leftcolumn*}
\EnglishColumn{idamavoca bhagavā.}
\hspace{0pt}\end{leftcolumn*}

\begin{rightcolumn}\PaliColumn{That is what the Blessed One said.}
\hspace{0pt}
\end{rightcolumn}
\end{samepage}
\begin{samepage}
\begin{leftcolumn*}
\EnglishColumn{attamanā te bhikkhū bhagavato bhāsitaṃ abhinandunti.}
\hspace{0pt}\end{leftcolumn*}

\begin{rightcolumn}\PaliColumn{The bhikkhus were satisfied and delighted in the Blessed One’s words.}
\hspace{0pt}
\end{rightcolumn}
\end{samepage}