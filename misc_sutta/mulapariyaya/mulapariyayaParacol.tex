\begin{samepage}
\ensurevspace{4\baselineskip}
\begin{leftcolumn*}
\EnglishColumn{Mūlapariyāya Sutta
The Root of All Things
}
\hspace{0pt} \\ 
\end{leftcolumn*}

\begin{rightcolumn}
\PaliColumn{Majjhima Nikāya, mūlapaṇṇāsapāḷi, 1. mūlapariyāyavaggo, 1. mūlapariyāyasuttaṃ (MN 1)
}
\hspace{0pt} \\ 
\end{rightcolumn}
\end{samepage}

\begin{samepage}
\ensurevspace{4\baselineskip}
\begin{leftcolumn*}
\EnglishColumn{Thus have I heard.
On one occasion the Blessed One was living in Ukkaṭṭhā in the Subhaga Grove at the root of a royal sāla tree.
}
\hspace{0pt} \\ 
\end{leftcolumn*}

\begin{rightcolumn}
\PaliColumn{evaṁ me sutaṁ.
ekaṁ samayaṁ bhagavā ukkaṭṭhāyaṁ viharati subhagavane sālarājamūle.
}
\hspace{0pt} \\ 
\end{rightcolumn}
\end{samepage}

\begin{samepage}
\ensurevspace{4\baselineskip}
\begin{leftcolumn*}
\EnglishColumn{There he addressed the bhikkhus thus:
“Bhikkhus.” — “Venerable sir,” they replied.
}
\hspace{0pt} \\ 
\end{leftcolumn*}

\begin{rightcolumn}
\PaliColumn{tatra kho bhagavā bhikkhū āmantesi,
“bhikkhavo”ti.
“bhadante”ti te bhikkhū bhagavato paccassosuṁ.
}
\hspace{0pt} \\ 
\end{rightcolumn}
\end{samepage}

\begin{samepage}
\ensurevspace{4\baselineskip}
\begin{leftcolumn*}
\EnglishColumn{The Blessed One said this:
“Bhikkhus, I shall teach you a discourse on the root of all things.
Listen and attend closely to what I shall say.”
}
\hspace{0pt} \\ 
\end{leftcolumn*}

\begin{rightcolumn}
\PaliColumn{bhagavā etadavoca,
“sabbadhammamūlapariyāyaṁ vo, bhikkhave, desessāmi.
taṁ suṇātha, sādhukaṁ manasi karotha, bhāsissāmī”ti.
}
\hspace{0pt} \\ 
\end{rightcolumn}
\end{samepage}

\begin{samepage}
\ensurevspace{4\baselineskip}
\begin{leftcolumn*}
\EnglishColumn{“Yes, venerable sir,” the bhikkhus replied.
}
\hspace{0pt} \\ 
\end{leftcolumn*}

\begin{rightcolumn}
\PaliColumn{“evaṁ, bhante”ti kho te bhikkhū bhagavato paccassosuṁ.
}
\hspace{0pt} \\ 
\end{rightcolumn}
\end{samepage}

\begin{samepage}
\ensurevspace{4\baselineskip}
\begin{leftcolumn*}
\EnglishColumn{The Blessed One said this:
“Here, bhikkhus, an untaught ordinary person,
who has no regard for noble ones and is unskilled and undisciplined in their Dhamma,
who has no regard for true men and is unskilled and undisciplined in their Dhamma,
}
\hspace{0pt} \\ 
\end{leftcolumn*}

\begin{rightcolumn}
\PaliColumn{bhagavā etadavoca —
“idha, bhikkhave, assutavā puthujjano ariyānaṁ adassāvī ariyadhammassa akovido ariyadhamme avinīto,
sappurisānaṁ adassāvī sappurisadhammassa akovido sappurisadhamme avinīto
}
\hspace{0pt} \\ 
\end{rightcolumn}
\end{samepage}

\begin{samepage}
\ensurevspace{4\baselineskip}
\begin{leftcolumn*}
\EnglishColumn{"He perceives earth as earth.\\
Having perceived earth as earth,
he conceives [himself as] earth,\\
he conceives [himself] in earth,
he conceives [himself apart] from earth,\\
he conceives earth to be ‘mine,’
he delights in earth.\\
Why is that?
Because he has not fully understood it, I say.\\
}
\hspace{0pt} \\ 
\end{leftcolumn*}

\begin{rightcolumn}
\PaliColumn{pathaviṁ pathavito sañjānāti;\\
pathaviṁ pathavito saññatvā pathaviṁ maññati,\\
pathaviyā maññati,
pathavito maññati,\\
pathaviṁ meti maññati,
pathaviṁ abhinandati.\\
taṁ kissa hetu?
‘apariññātaṁ tassā’ti vadāmi.\\
}
\hspace{0pt} \\ 
\end{rightcolumn}
\end{samepage}

\begin{samepage}
\ensurevspace{4\baselineskip}
\begin{leftcolumn*}
\EnglishColumn{“He perceives water as water.\\
Having perceived water as water,
he conceives [himself as] water,\\
he conceives [himself] in water,
he conceives [himself apart] from water,\\
he conceives water to be ‘mine,’
he delights in water.\\
Why is that?
Because he has not fully understood it, I say.\\
}
\hspace{0pt} \\ 
\end{leftcolumn*}

\begin{rightcolumn}
\PaliColumn{“āpaṁ āpato sañjānāti;\\
āpaṁ āpato saññatvā āpaṁ maññati,\\
āpasmiṁ maññati,
āpato maññati,\\
āpaṁ meti maññati,
āpaṁ abhinandati.\\
taṁ kissa hetu?
‘apariññātaṁ tassā’ti vadāmi.\\
}
\hspace{0pt} \\ 
\end{rightcolumn}
\end{samepage}

\begin{samepage}
\ensurevspace{4\baselineskip}
\begin{leftcolumn*}
\EnglishColumn{“He perceives fire as fire.\\
Having perceived fire as fire,
he conceives [himself as] fire,\\
he conceives [himself] in fire,
he conceives [himself apart] from fire,\\
he conceives fire to be ‘mine,’
he delights in fire.\\
Why is that?
Because he has not fully understood it, I say.\\
}
\hspace{0pt} \\ 
\end{leftcolumn*}

\begin{rightcolumn}
\PaliColumn{“tejaṁ tejato sañjānāti;\\
tejaṁ tejato saññatvā tejaṁ maññati,\\
tejasmiṁ maññati,
tejato maññati,\\
tejaṁ meti maññati,
tejaṁ abhinandati.\\
taṁ kissa hetu?
‘apariññātaṁ tassā’ti vadāmi.\\
}
\hspace{0pt} \\ 
\end{rightcolumn}
\end{samepage}

\begin{samepage}
\ensurevspace{4\baselineskip}
\begin{leftcolumn*}
\EnglishColumn{He perceives air as air.\\
Having perceived air as air,
he conceives [himself as] air,\\
he conceives [himself] in air,
he conceives [himself apart] from air,\\
he conceives air to be ‘mine,’
he delights in air.\\
Why is that?
Because he has not fully understood it, I say.\\
}
\hspace{0pt} \\ 
\end{leftcolumn*}

\begin{rightcolumn}
\PaliColumn{“vāyaṁ vāyato sañjānāti;\\
vāyaṁ vāyato saññatvā vāyaṁ maññati,\\
vāyasmiṁ maññati,
vāyato maññati,\\
vāyaṁ meti maññati,
vāyaṁ abhinandati.\\
taṁ kissa hetu?
‘apariññātaṁ tassā’ti vadāmi.\\
}
\hspace{0pt} \\ 
\end{rightcolumn}
\end{samepage}

\begin{samepage}
\ensurevspace{4\baselineskip}
\begin{leftcolumn*}
\EnglishColumn{He perceives beings as beings.\\
Having perceived beings as beings,
he conceives beings,\\
he conceives [himself] in beings,
he conceives [himself apart] from beings,\\
he conceives beings to be ‘mine,’
he delights in beings.\\
Why is that?
Because he has not fully understood it, I say.\\
}
\hspace{0pt} \\ 
\end{leftcolumn*}

\begin{rightcolumn}
\PaliColumn{“bhūte bhūtato sañjānāti;\\
bhūte bhūtato saññatvā bhūte maññati,\\
bhūtesu maññati,
bhūtato maññati,\\
bhūte meti maññati,
bhūte abhinandati.\\
taṁ kissa hetu?
‘apariññātaṁ tassā’ti vadāmi.\\
}
\hspace{0pt} \\ 
\end{rightcolumn}
\end{samepage}

\begin{samepage}
\ensurevspace{4\baselineskip}
\begin{leftcolumn*}
\EnglishColumn{“He perceives gods as gods.\\
Having perceived gods as gods,
he conceives gods,\\
he conceives [himself] in gods,
he conceives [himself apart] from gods,\\
he conceives gods to be ‘mine,’
he delights in gods.\\
Why is that?
Because he has not fully understood it, I say.\\
}
\hspace{0pt} \\ 
\end{leftcolumn*}

\begin{rightcolumn}
\PaliColumn{“deve devato sañjānāti;\\
deve devato saññatvā deve maññati,\\
devesu maññati,
devato maññati,\\
deve meti maññati,
deve abhinandati.\\
taṁ kissa hetu?
‘apariññātaṁ tassā’ti vadāmi.\\
}
\hspace{0pt} \\ 
\end{rightcolumn}
\end{samepage}

\begin{samepage}
\ensurevspace{4\baselineskip}
\begin{leftcolumn*}
\EnglishColumn{“He perceives Pajāpati as Pajāpati.\\
Having perceived Pajāpati as Pajāpati,
he conceives Pajāpati,\\
he conceives [himself] in Pajāpati,
he conceives [himself apart] from Pajāpati,\\
he conceives Pajāpati to be ‘mine,’
he delights in Pajāpati.\\
Why is that?
Because he has not fully understood it, I say.\\
}
\hspace{0pt} \\ 
\end{leftcolumn*}

\begin{rightcolumn}
\PaliColumn{“pajāpatiṁ pajāpatito sañjānāti;\\
pajāpatiṁ pajāpatito saññatvā pajāpatiṁ maññati,\\
pajāpatismiṁ maññati,
pajāpatito maññati,\\
pajāpatiṁ meti maññati,
pajāpatiṁ abhinandati.\\
taṁ kissa hetu?
‘apariññātaṁ tassā’ti vadāmi.\\
}
\hspace{0pt} \\ 
\end{rightcolumn}
\end{samepage}

\begin{samepage}
\ensurevspace{4\baselineskip}
\begin{leftcolumn*}
\EnglishColumn{“He perceives Brahmā as Brahmā.\\
Having perceived Brahmā as Brahmā,
he conceives Brahmā,\\
he conceives [himself] in Brahmā,
he conceives [himself apart] from Brahmā,\\
he conceives Brahmā to be ‘mine,’
he delights in Brahmā.\\
Why is that?
Because he has not fully understood it, I say.\\
}
\hspace{0pt} \\ 
\end{leftcolumn*}

\begin{rightcolumn}
\PaliColumn{“brahmaṁ brahmato sañjānāti;\\
brahmaṁ brahmato saññatvā brahmaṁ maññati,\\
brahmasmiṁ maññati,
brahmato maññati,\\
brahmaṁ meti maññati,
brahmaṁ abhinandati.\\
taṁ kissa hetu?
‘apariññātaṁ tassā’ti vadāmi.\\
}
\hspace{0pt} \\ 
\end{rightcolumn}
\end{samepage}

\begin{samepage}
\ensurevspace{4\baselineskip}
\begin{leftcolumn*}
\EnglishColumn{(SR - Streaming Radiance)\\
“He perceives the Gods of SR as the Gods of SR.\\
Having perceived the Gods of SR as the Gods of SR,
he conceives the Gods of SR,\\
he conceives [himself] in the Gods of SR,
he conceives [himself apart] from the Gods of SR,\\
he conceives the Gods of SR to be ‘mine,’
he delights in the Gods of SR.\\
Why is that?
Because he has not fully understood it, I say.\\
}
\hspace{0pt} \\ 
\end{leftcolumn*}

\begin{rightcolumn}
\PaliColumn{“ābhassare ābhassarato sañjānāti;\\
ābhassare ābhassarato saññatvā ābhassare maññati,\\
ābhassaresu maññati,
ābhassarato maññati,\\
ābhassare meti maññati,
ābhassare abhinandati.\\
taṁ kissa hetu?
‘apariññātaṁ tassā’ti vadāmi.\\
}
\hspace{0pt} \\ 
\end{rightcolumn}
\end{samepage}

\begin{samepage}
\ensurevspace{4\baselineskip}
\begin{leftcolumn*}
\EnglishColumn{(RG - Refulgent Glory)\\
"He perceives the Gods of RG as the Gods of RG.\\
Having perceived the Gods of RG as the Gods of RG,
he conceives the Gods of RG,\\
he conceives [himself] in the Gods of RG,
he conceives [himself apart] from the Gods of RG,\\
he conceives the Gods of RG to be ‘mine,’
he delights in the Gods of RG.\\
Why is that?
Because he has not fully understood it, I say.\\
}
\hspace{0pt} \\ 
\end{leftcolumn*}

\begin{rightcolumn}
\PaliColumn{“subhakiṇhe subhakiṇhato sañjānāti;\\
subhakiṇhe subhakiṇhato saññatvā subhakiṇhe maññati,\\
subhakiṇhesu maññati,
subhakiṇhato maññati,\\
subhakiṇhe meti maññati,
subhakiṇhe abhinandati.\\
taṁ kissa hetu?
‘apariññātaṁ tassā’ti vadāmi.\\
}
\hspace{0pt} \\ 
\end{rightcolumn}
\end{samepage}

\begin{samepage}
\ensurevspace{4\baselineskip}
\begin{leftcolumn*}
\EnglishColumn{(GF - Great Fruit)\\
"He perceives the Gods of GF as the Gods of GF.\\
Having perceived the Gods of GF as the Gods of GF,
he conceives the Gods of GF,\\
he conceives [himself] in the Gods of GF,
he conceives [himself apart] from the Gods of GF,\\
he conceives the Gods of GF to be ‘mine,’
he delights in the Gods of GF.\\
Why is that?
Because he has not fully understood it, I say.\\
}
\hspace{0pt} \\ 
\end{leftcolumn*}

\begin{rightcolumn}
\PaliColumn{“vehapphale vehapphalato sañjānāti;\\
vehapphale vehapphalato saññatvā vehapphale maññati,\\
vehapphalesu maññati,
vehapphalato maññati,\\
vehapphale meti maññati,
vehapphale abhinandati.\\
taṁ kissa hetu?
‘apariññātaṁ tassā’ti vadāmi.\\
}
\hspace{0pt} \\ 
\end{rightcolumn}
\end{samepage}

\begin{samepage}
\ensurevspace{4\baselineskip}
\begin{leftcolumn*}
\EnglishColumn{“He perceives the Overlord as the Overlord.\\
Having perceived the Overlord as the Overlord,
he conceives the Overlord,\\
he conceives [himself] in the Overlord,
he conceives [himself apart] from the Overlord,\\
he conceives the Overlord to be ‘mine,’
he delights in the Overlord.\\
Why is that?
Because he has not fully understood it, I say.\\
}
\hspace{0pt} \\ 
\end{leftcolumn*}

\begin{rightcolumn}
\PaliColumn{“abhibhuṁ abhibhūto sañjānāti;\\
abhibhuṁ abhibhūto saññatvā abhibhuṁ maññati,\\
abhibhusmiṁ maññati,
abhibhūto maññati,\\
abhibhuṁ meti maññati,
abhibhuṁ abhinandati.\\
taṁ kissa hetu?
‘apariññātaṁ tassā’ti vadāmi.\\
}
\hspace{0pt} \\ 
\end{rightcolumn}
\end{samepage}

\begin{samepage}
\ensurevspace{4\baselineskip}
\begin{leftcolumn*}
\EnglishColumn{(US - unbound space)\\
“He perceives the base of US as the base of US.\\
Having perceived the base of US as the base of US,
he conceives [himself as] the base of US,\\
he conceives [himself] in the base of US,
he conceives [himself apart] from the base of US,\\
he conceives the base of US to be ‘mine,’
he delights in the base of US.\\
Why is that?
Because he has not fully understood it, I say.\\
}
\hspace{0pt} \\ 
\end{leftcolumn*}

\begin{rightcolumn}
\PaliColumn{“ākāsānañcāyatanaṁ ākāsānañcāyatanato sañjānāti;\\
ākāsānañcāyatanaṁ ākāsānañcāyatanato saññatvā ākāsānañcāyatanaṁ maññati,\\
ākāsānañcāyatanasmiṁ maññati,
ākāsānañcāyatanato maññati,\\
ākāsānañcāyatanaṁ meti maññati,
ākāsānañcāyatanaṁ abhinandati.\\
taṁ kissa hetu?
‘apariññātaṁ tassā’ti vadāmi.\\
}
\hspace{0pt} \\ 
\end{rightcolumn}
\end{samepage}

\begin{samepage}
\ensurevspace{4\baselineskip}
\begin{leftcolumn*}
\EnglishColumn{(UC - unbound consciousness)\\
“He perceives the base of UC as the base of UC.\\
Having perceived the base of UC as the base of UC,
he conceives [himself as] the base of UC,\\
he conceives [himself] in the base of UC,
he conceives [himself apart] from the base of UC,\\
he conceives the base of UC to be ‘mine,’
he delights in the base of UC.\\
Why is that?
Because he has not fully understood it, I say.\\
}
\hspace{0pt} \\ 
\end{leftcolumn*}

\begin{rightcolumn}
\PaliColumn{“viññāṇañcāyatanaṁ viññāṇañcāyatanato sañjānāti;\\
viññāṇañcāyatanaṁ viññāṇañcāyatanato saññatvā viññāṇañcāyatanaṁ maññati,\\
viññāṇañcāyatanasmiṁ maññati,
viññāṇañcāyatanato maññati,\\
viññāṇañcāyatanaṁ meti maññati,
viññāṇañcāyatanaṁ abhinandati.\\
taṁ kissa hetu?
‘apariññātaṁ tassā’ti vadāmi.\\
}
\hspace{0pt} \\ 
\end{rightcolumn}
\end{samepage}

\begin{samepage}
\ensurevspace{4\baselineskip}
\begin{leftcolumn*}
\EnglishColumn{(NT - no-thingness)\\
“He perceives the base of NT as the base of NT.\\
Having perceived the base of NT as the base of NT,
he conceives [himself as] the base of NT,\\
he conceives [himself] in the base of NT,
he conceives [himself apart] from the base of NT,\\
he conceives the base of NT to be ‘mine,’
he delights in the base of NT.\\
Why is that?
Because he has not fully understood it, I say.\\
}
\hspace{0pt} \\ 
\end{leftcolumn*}

\begin{rightcolumn}
\PaliColumn{“ākiñcaññāyatanaṁ ākiñcaññāyatanato sañjānāti;\\
ākiñcaññāyatanaṁ ākiñcaññāyatanato saññatvā ākiñcaññāyatanaṁ maññati,\\
ākiñcaññāyatanasmiṁ maññati,
ākiñcaññāyatanato maññati,\\
ākiñcaññāyatanaṁ meti maññati,
ākiñcaññāyatanaṁ abhinandati.\\
taṁ kissa hetu?
‘apariññātaṁ tassā’ti vadāmi.\\
}
\hspace{0pt} \\ 
\end{rightcolumn}
\end{samepage}

\begin{samepage}
\ensurevspace{4\baselineskip}
\begin{leftcolumn*}
\EnglishColumn{(NPnNP - neither-perception-nor-non-perception)\\
“He perceives the base of NPnNP as the base of NPnNP.\\
Having perceived the base of NPnNP as the base of NPnNP,
he conceives [himself as] the base of NPnNP,\\
he conceives [himself] in the base of NPnNP,
he conceives [himself apart] from the base of NPnNP,\\
he conceives the base of NPnNP to be ‘mine,’
he delights in the base of NPnNP.\\
Why is that?
Because he has not fully understood it, I say.\\
}
\hspace{0pt} \\ 
\end{leftcolumn*}

\begin{rightcolumn}
\PaliColumn{“nevasaññānāsaññāyatanaṁ nevasaññānāsaññāyatanato sañjānāti;\\
nevasaññānāsaññāyatanaṁ nevasaññānāsaññāyatanato saññatvā nevasaññānāsaññāyatanaṁ maññati,\\
nevasaññānāsaññāyatanasmiṁ maññati,
nevasaññānāsaññāyatanato maññati,\\
nevasaññānāsaññāyatanaṁ meti maññati,
nevasaññānāsaññāyatanaṁ abhinandati.\\
taṁ kissa hetu?
‘apariññātaṁ tassā’ti vadāmi.\\
}
\hspace{0pt} \\ 
\end{rightcolumn}
\end{samepage}

\begin{samepage}
\ensurevspace{4\baselineskip}
\begin{leftcolumn*}
\EnglishColumn{“He perceives the seen as the seen.\\
Having perceived the seen as the seen,
he conceives [himself as] the seen,\\
he conceives [himself] in the seen,
he conceives [himself apart] from the seen,\\
he conceives the seen to be ‘mine,’
he delights in the seen.\\
Why is that?
Because he has not fully understood it, I say.\\
}
\hspace{0pt} \\ 
\end{leftcolumn*}

\begin{rightcolumn}
\PaliColumn{“diṭṭhaṁ diṭṭhato sañjānāti;\\
diṭṭhaṁ diṭṭhato saññatvā diṭṭhaṁ maññati,\\
diṭṭhasmiṁ maññati,
diṭṭhato maññati,\\
diṭṭhaṁ meti maññati,
diṭṭhaṁ abhinandati.\\
taṁ kissa hetu?
‘apariññātaṁ tassā’ti vadāmi.\\
}
\hspace{0pt} \\ 
\end{rightcolumn}
\end{samepage}

\begin{samepage}
\ensurevspace{4\baselineskip}
\begin{leftcolumn*}
\EnglishColumn{“He perceives the heard as the heard.\\
Having perceived the heard as the heard,
he conceives [himself as] the heard,\\
he conceives [himself] in the heard,
he conceives [himself apart] from the heard,\\
he conceives the heard to be ‘mine,’
he delights in the heard.\\
Why is that?
Because he has not fully understood it, I say.\\
}
\hspace{0pt} \\ 
\end{leftcolumn*}

\begin{rightcolumn}
\PaliColumn{“sutaṁ sutato sañjānāti;\\
sutaṁ sutato saññatvā sutaṁ maññati,\\
sutasmiṁ maññati,
sutato maññati,\\
sutaṁ meti maññati,
sutaṁ abhinandati.\\
taṁ kissa hetu?
‘apariññātaṁ tassā’ti vadāmi.\\
}
\hspace{0pt} \\ 
\end{rightcolumn}
\end{samepage}

\begin{samepage}
\ensurevspace{4\baselineskip}
\begin{leftcolumn*}
\EnglishColumn{“He perceives the sensed as the sensed.\\
Having perceived the sensed as the sensed,
he conceives [himself as] the sensed,\\
he conceives [himself] in the sensed,
he conceives [himself apart] from the sensed,\\
he conceives the sensed to be ‘mine,’
he delights in the sensed.\\
Why is that?
Because he has not fully understood it, I say.\\
}
\hspace{0pt} \\ 
\end{leftcolumn*}

\begin{rightcolumn}
\PaliColumn{“mutaṁ mutato sañjānāti;\\
mutaṁ mutato saññatvā mutaṁ maññati,\\
mutasmiṁ maññati,
mutato maññati,\\
mutaṁ meti maññati,
mutaṁ abhinandati.\\
taṁ kissa hetu?
‘apariññātaṁ tassā’ti vadāmi.\\
}
\hspace{0pt} \\ 
\end{rightcolumn}
\end{samepage}

\begin{samepage}
\ensurevspace{4\baselineskip}
\begin{leftcolumn*}
\EnglishColumn{“He perceives the cognized as the cognized.\\
Having perceived the cognized as the cognized,
he conceives [himself as] the cognized,\\
he conceives [himself] in the cognized,
he conceives [himself apart] from the cognized,\\
he conceives the cognized to be ‘mine,’
he delights in the cognized.\\
Why is that?
Because he has not fully understood it, I say.\\
}
\hspace{0pt} \\ 
\end{leftcolumn*}

\begin{rightcolumn}
\PaliColumn{“viññātaṁ viññātato sañjānāti;\\
viññātaṁ viññātato saññatvā viññātaṁ maññati,\\
viññātasmiṁ maññati,
viññātato maññati,\\
viññātaṁ meti maññati,
viññātaṁ abhinandati.\\
taṁ kissa hetu?
‘apariññātaṁ tassā’ti vadāmi.\\
}
\hspace{0pt} \\ 
\end{rightcolumn}
\end{samepage}

\begin{samepage}
\ensurevspace{4\baselineskip}
\begin{leftcolumn*}
\EnglishColumn{“He perceives unity as unity.\\
Having perceived unity as unity,
he conceives [himsel as] unity,\\
he conceives [himself] in unity,
he conceives [himself apart] from unity,\\
he conceives unity to be ‘mine,’
he delights in unity.\\
Why is that?
Because he has not fully understood it, I say.\\
}
\hspace{0pt} \\ 
\end{leftcolumn*}

\begin{rightcolumn}
\PaliColumn{“ekattaṁ ekattato sañjānāti;\\
ekattaṁ ekattato saññatvā ekattaṁ maññati,\\
ekattasmiṁ maññati,
ekattato maññati,\\
ekattaṁ meti maññati,
ekattaṁ abhinandati.\\
taṁ kissa hetu?
‘apariññātaṁ tassā’ti vadāmi.\\
}
\hspace{0pt} \\ 
\end{rightcolumn}
\end{samepage}

\begin{samepage}
\ensurevspace{4\baselineskip}
\begin{leftcolumn*}
\EnglishColumn{“He perceives diversity as diversity.\\
having perceived diversity as diversity,
he conceives [himself as] diversity,\\
he conceives [himself] in diversity,
he conceives [himself apart] from diversity,\\
he conceives diversity to be ‘mine,’
he delights in diversity.\\
why is that?
because he has not fully understood it, i say.\\
}
\hspace{0pt} \\ 
\end{leftcolumn*}

\begin{rightcolumn}
\PaliColumn{“nānattaṁ nānattato sañjānāti;\\
nānattaṁ nānattato saññatvā nānattaṁ maññati,\\
nānattasmiṁ maññati,
nānattato maññati,\\
nānattaṁ meti maññati,
nānattaṁ abhinandati.\\
taṁ kissa hetu?
‘apariññātaṁ tassā’ti vadāmi.\\
}
\hspace{0pt} \\ 
\end{rightcolumn}
\end{samepage}

\begin{samepage}
\ensurevspace{4\baselineskip}
\begin{leftcolumn*}
\EnglishColumn{“He perceives all as all.\\
having perceived all as all,
he conceives [himself as] all,\\
he conceives [himself] in all,
he conceives [himself apart] from all,\\
he conceives all to be ‘mine,’
he delights in all.\\
why is that?
because he has not fully understood it, I say.\\
}
\hspace{0pt} \\ 
\end{leftcolumn*}

\begin{rightcolumn}
\PaliColumn{“sabbaṁ sabbato sañjānāti;\\
sabbaṁ sabbato saññatvā sabbaṁ maññati,\\
sabbasmiṁ maññati,
sabbato maññati,\\
sabbaṁ meti maññati,
sabbaṁ abhinandati.\\
taṁ kissa hetu?
‘apariññātaṁ tassā’ti vadāmi.\\
}
\hspace{0pt} \\ 
\end{rightcolumn}
\end{samepage}

\begin{samepage}
\ensurevspace{4\baselineskip}
\begin{leftcolumn*}
\EnglishColumn{“He perceives Nibbāna as Nibbāna.\\
Having perceived Nibbāna as Nibbāna,
he conceives [himself as] Nibbāna,\\
he conceives [himself] in Nibbāna,
he conceives [himself apart] from Nibbāna,\\
he conceives Nibbāna to be ‘mine,’
he delights in Nibbāna.\\
Why is that?
Because he has not fully understood it, I say.\\
}
\hspace{0pt} \\ 
\end{leftcolumn*}

\begin{rightcolumn}
\PaliColumn{“nibbānaṁ nibbānato sañjānāti;\\
nibbānaṁ nibbānato saññatvā nibbānaṁ maññati,\\
nibbānasmiṁ maññati,
nibbānato maññati,\\
nibbānaṁ meti maññati,
nibbānaṁ abhinandati.\\
taṁ kissa hetu?
‘apariññātaṁ tassā’ti vadāmi.\\
}
\hspace{0pt} \\ 
\end{rightcolumn}
\end{samepage}

\begin{samepage}
\ensurevspace{4\baselineskip}
\begin{leftcolumn*}
\EnglishColumn{“Bhikkhus, a bhikkhu who is in higher training,
whose mind has not yet reached the goal,
and who is still aspiring to the supreme security from bondage,
}
\hspace{0pt} \\ 
\end{leftcolumn*}

\begin{rightcolumn}
\PaliColumn{“yopi so, bhikkhave, bhikkhu sekkho appattamānaso anuttaraṁ yogakkhemaṁ patthayamāno viharati,
}
\hspace{0pt} \\ 
\end{rightcolumn}
\end{samepage}

\begin{samepage}
\ensurevspace{4\baselineskip}
\begin{leftcolumn*}
\EnglishColumn{directly knows earth as earth.\\
Having directly known earth as earth,
he should not conceive [himself as] earth,\\
he should not conceive [himself] in earth,
he should not conceive [himself apart] from earth,\\
he should not conceive earth to be ‘mine,’
he should not delight in earth.\\
Why is that?
Because he must fully understand it, I say.\\
}
\hspace{0pt} \\ 
\end{leftcolumn*}

\begin{rightcolumn}
\PaliColumn{sopi pathaviṁ pathavito abhijānāti;\\
pathaviṁ pathavito abhiññāya pathaviṁ mā maññi,\\
pathaviyā mā maññi,
pathavito mā maññi,\\
pathaviṁ meti mā maññi,
pathaviṁ mābhinandi.\\
taṁ kissa hetu?
‘pariññeyyaṁ tassā’ti vadāmi.\\
}
\hspace{0pt} \\ 
\end{rightcolumn}
\end{samepage}

\begin{samepage}
\ensurevspace{4\baselineskip}
\begin{leftcolumn*}
\EnglishColumn{“He directly knows water as water.\\
Having directly known water as water,
he should not conceive [himself as] water,\\
he should not conceive [himself] in water,
he should not conceive [himself apart] from water,\\
he should not conceive water to be ‘mine,’
he should not delight in water.\\
Why is that?
Because he must fully understand it, I say.\\
}
\hspace{0pt} \\ 
\end{leftcolumn*}

\begin{rightcolumn}
\PaliColumn{"āpaṁ āpato abhijānāti;\\
āpaṁ āpato abhiññāya āpaṁ mā maññi,\\
āpasmiṁ mā maññi,
āpato mā maññi,\\
āpaṁ meti mā maññi,
āpaṁ mābhinandi.\\
taṁ kissa hetu?
‘pariññeyyaṁ tassā’ti vadāmi.\\
}
\hspace{0pt} \\ 
\end{rightcolumn}
\end{samepage}

\begin{samepage}
\ensurevspace{4\baselineskip}
\begin{leftcolumn*}
\EnglishColumn{“He directly knows fire as fire.\\
Having directly known fire as fire,
he should not conceive [himself as] fire,\\
he should not conceive [himself] in fire,
he should not conceive [himself apart] from fire,\\
he should not conceive fire to be ‘mine,’
he should not delight in fire.\\
Why is that?
Because he must fully understand it, I say.\\
}
\hspace{0pt} \\ 
\end{leftcolumn*}

\begin{rightcolumn}
\PaliColumn{"tejaṁ tejato abhijānāti;\\
tejaṁ tejato abhiññāya tejaṁ mā maññi,\\
tejasmiṁ mā maññi,
tejato mā maññi,\\
tejaṁ meti mā maññi,
tejaṁ mābhinandi.\\
taṁ kissa hetu?
‘pariññeyyaṁ tassā’ti vadāmi.\\
}
\hspace{0pt} \\ 
\end{rightcolumn}
\end{samepage}

\begin{samepage}
\ensurevspace{4\baselineskip}
\begin{leftcolumn*}
\EnglishColumn{“He directly knows air as air.\\
Having directly known air as air,
he should not conceive [himself as] air,\\
he should not conceive [himself] in air,
he should not conceive [himself apart] from air,\\
he should not conceive air to be ‘mine,’
he should not delight in air.\\
Why is that?
Because he must fully understand it, I say.\\
}
\hspace{0pt} \\ 
\end{leftcolumn*}

\begin{rightcolumn}
\PaliColumn{"vāyaṁ vāyato abhijānāti;\\
vāyaṁ vāyato abhiññāya vāyaṁ mā maññi,\\
vāyasmiṁ mā maññi,
vāyato mā maññi,\\
vāyaṁ meti mā maññi,
vāyaṁ mābhinandi.\\
taṁ kissa hetu?
‘pariññeyyaṁ tassā’ti vadāmi.\\
}
\hspace{0pt} \\ 
\end{rightcolumn}
\end{samepage}

\begin{samepage}
\ensurevspace{4\baselineskip}
\begin{leftcolumn*}
\EnglishColumn{“He directly knows beings as beings.\\
Having directly known beings as beings,
he should not conceive [himself as] beings,\\
he should not conceive [himself] in beings,
he should not conceive [himself apart] from beings,\\
he should not conceive beings to be ‘mine,’
he should not delight in beings.\\
Why is that?
Because he must fully understand it, I say.\\
}
\hspace{0pt} \\ 
\end{leftcolumn*}

\begin{rightcolumn}
\PaliColumn{"bhūte bhūtato abhijānāti;\\
bhūte bhūtato abhiññāya bhūte mā maññi,\\
bhūtesu mā maññi,
bhūtato mā maññi,\\
bhūte meti mā maññi,
bhūte mābhinandi.\\
taṁ kissa hetu?
‘pariññeyyaṁ tassā’ti vadāmi.\\
}
\hspace{0pt} \\ 
\end{rightcolumn}
\end{samepage}

\begin{samepage}
\ensurevspace{4\baselineskip}
\begin{leftcolumn*}
\EnglishColumn{“He directly knows gods as gods.\\
Having directly known gods as gods,
he should not conceive [himself as] gods,\\
he should not conceive [himself] in gods,
he should not conceive [himself apart] from gods,\\
he should not conceive gods to be ‘mine,’
he should not delight in gods.\\
Why is that?
Because he must fully understand it, I say.\\
}
\hspace{0pt} \\ 
\end{leftcolumn*}

\begin{rightcolumn}
\PaliColumn{"deve devato abhijānāti;\\
deve devato abhiññāya deve mā maññi,\\
devesu mā maññi,
devato mā maññi,\\
deve meti mā maññi,
deve mābhinandi.\\
taṁ kissa hetu?
‘pariññeyyaṁ tassā’ti vadāmi.\\
}
\hspace{0pt} \\ 
\end{rightcolumn}
\end{samepage}

\begin{samepage}
\ensurevspace{4\baselineskip}
\begin{leftcolumn*}
\EnglishColumn{“He directly knows Pajāpati as Pajāpati.\\
Having directly known Pajāpati as Pajāpati,
he should not conceive [himself as] Pajāpati,\\
he should not conceive [himself] in Pajāpati,
he should not conceive [himself apart] from Pajāpati,\\
he should not conceive Pajāpati to be ‘mine,’
he should not delight in Pajāpati.\\
Why is that?
Because he must fully understand it, I say.\\
}
\hspace{0pt} \\ 
\end{leftcolumn*}

\begin{rightcolumn}
\PaliColumn{"pajāpatiṁ pajāpatito abhijānāti;\\
pajāpatiṁ pajāpatito abhiññāya pajāpatiṁ mā maññi,\\
pajāpatismiṁ mā maññi,
pajāpatito mā maññi,\\
pajāpatiṁ meti mā maññi,
pajāpatiṁ mābhinandi.\\
taṁ kissa hetu?
‘pariññeyyaṁ tassā’ti vadāmi.\\
}
\hspace{0pt} \\ 
\end{rightcolumn}
\end{samepage}

\begin{samepage}
\ensurevspace{4\baselineskip}
\begin{leftcolumn*}
\EnglishColumn{“He directly knows Brahmā as Brahmā.\\
Having directly known Brahmā as Brahmā,
he should not conceive [himself as] Brahmā,\\
he should not conceive [himself] in Brahmā,
he should not conceive [himself apart] from Brahmā,\\
he should not conceive Brahmā to be ‘mine,’
he should not delight in Brahmā.\\
Why is that?
Because he must fully understand it, I say.\\
}
\hspace{0pt} \\ 
\end{leftcolumn*}

\begin{rightcolumn}
\PaliColumn{"brahmaṁ brahmato abhijānāti;\\
brahmaṁ brahmato abhiññāya brahmaṁ mā maññi,\\
brahmasmiṁ mā maññi,
brahmato mā maññi,\\
brahmaṁ meti mā maññi,
brahmaṁ mābhinandi.\\
taṁ kissa hetu?
‘pariññeyyaṁ tassā’ti vadāmi.\\
}
\hspace{0pt} \\ 
\end{rightcolumn}
\end{samepage}

\begin{samepage}
\ensurevspace{4\baselineskip}
\begin{leftcolumn*}
\EnglishColumn{(SR - Streaming Radiance)\\
“He directly knows the Gods of SR as the Gods of SR.\\
Having directly known the Gods of SR as the Gods of SR,
he should not conceive [himself as] the Gods of SR,\\
he should not conceive [himself] in the Gods of SR,
he should not conceive [himself apart] from the Gods of SR,\\
he should not conceive the Gods of SR to be ‘mine,’
he should not delight in the Gods of SR.\\
Why is that?
Because he must fully understand it, I say.\\
}
\hspace{0pt} \\ 
\end{leftcolumn*}

\begin{rightcolumn}
\PaliColumn{"ābhassare ābhassarato abhijānāti;\\
ābhassare ābhassarato abhiññāya ābhassare mā maññi,\\
ābhassaresu mā maññi,
ābhassarato mā maññi,\\
ābhassare meti mā maññi,
ābhassare mābhinandi.\\
taṁ kissa hetu?
‘pariññeyyaṁ tassā’ti vadāmi.\\
}
\hspace{0pt} \\ 
\end{rightcolumn}
\end{samepage}

\begin{samepage}
\ensurevspace{4\baselineskip}
\begin{leftcolumn*}
\EnglishColumn{(RG - Refulgent Glory)\\
“He directly knows the Gods of RG as the Gods of RG.\\
Having directly known the Gods of RG as the Gods of RG,
he should not conceive [himself as] the Gods of RG,\\
he should not conceive [himself] in the Gods of RG,
he should not conceive [himself apart] from the Gods of RG,\\
he should not conceive the Gods of RG to be ‘mine,’
he should not delight in the Gods of RG.\\
Why is that?
Because he must fully understand it, I say.\\
}
\hspace{0pt} \\ 
\end{leftcolumn*}

\begin{rightcolumn}
\PaliColumn{"subhakiṇhe subhakiṇhato abhijānāti;\\
subhakiṇhe subhakiṇhato abhiññāya subhakiṇhe mā maññi,\\
subhakiṇhesu mā maññi,
subhakiṇhato mā maññi,\\
subhakiṇhe meti mā maññi,
subhakiṇhe mābhinandi.\\
taṁ kissa hetu?
‘pariññeyyaṁ tassā’ti vadāmi.\\
}
\hspace{0pt} \\ 
\end{rightcolumn}
\end{samepage}

\begin{samepage}
\ensurevspace{4\baselineskip}
\begin{leftcolumn*}
\EnglishColumn{(GF - Great Fruit)\\
“He directly knows the Gods of GF as the Gods of GF.\\
Having directly known the Gods of GF as the Gods of GF,
he should not conceive [himself as] the Gods of GF,\\
he should not conceive [himself] in the Gods of GF,
he should not conceive [himself apart] from the Gods of GF,\\
he should not conceive the Gods of GF to be ‘mine,’
he should not delight in the Gods of GF.\\
Why is that?
Because he must fully understand it, I say.\\
}
\hspace{0pt} \\ 
\end{leftcolumn*}

\begin{rightcolumn}
\PaliColumn{"vehapphale vehapphalato abhijānāti;\\
vehapphale vehapphalato abhiññāya vehapphale mā maññi,\\
vehapphalesu mā maññi,
vehapphalato mā maññi,\\
vehapphale meti mā maññi,
vehapphale mābhinandi.\\
taṁ kissa hetu?
‘pariññeyyaṁ tassā’ti vadāmi.\\
}
\hspace{0pt} \\ 
\end{rightcolumn}
\end{samepage}

\begin{samepage}
\ensurevspace{4\baselineskip}
\begin{leftcolumn*}
\EnglishColumn{“He directly knows the Overlord as the Overlord.\\
Having directly known the Overlord as the Overlord,
he should not conceive [himself as] the Overlord,\\
he should not conceive [himself] in the Overlord,
he should not conceive [himself apart] from the Overlord,\\
he should not conceive the Overlord to be ‘mine,’
he should not delight in the Overlord.\\
Why is that?
Because he must fully understand it, I say.\\
}
\hspace{0pt} \\ 
\end{leftcolumn*}

\begin{rightcolumn}
\PaliColumn{"abhibhuṁ abhibhūto abhijānāti;\\
abhibhuṁ abhibhūto abhiññāya abhibhuṁ mā maññi,\\
abhibhusmiṁ mā maññi,
abhibhūto mā maññi,\\
abhibhuṁ meti mā maññi,
abhibhuṁ mābhinandi.\\
taṁ kissa hetu?
‘pariññeyyaṁ tassā’ti vadāmi.\\
}
\hspace{0pt} \\ 
\end{rightcolumn}
\end{samepage}

\begin{samepage}
\ensurevspace{4\baselineskip}
\begin{leftcolumn*}
\EnglishColumn{(US - unbound space)\\
“He directly knows the base of US as the base of US.\\
Having directly known the base of US as the base of US,
he should not conceive [himself as] the base of US,\\
he should not conceive [himself] in the base of US,
he should not conceive [himself apart] from the base of US,\\
he should not conceive the base of US to be ‘mine,’
he should not delight in the base of US.\\
Why is that?
Because he must fully understand it, I say.\\
}
\hspace{0pt} \\ 
\end{leftcolumn*}

\begin{rightcolumn}
\PaliColumn{"ākāsānañcāyatanaṁ ākāsānañcāyatanato abhijānāti;\\
ākāsānañcāyatanaṁ ākāsānañcāyatanato abhiññāya ākāsānañcāyatanaṁ mā maññi,\\
ākāsānañcāyatanasmiṁ mā maññi,
ākāsānañcāyatanato mā maññi,\\
ākāsānañcāyatanaṁ meti mā maññi,
ākāsānañcāyatanaṁ mābhinandi.\\
taṁ kissa hetu?
‘pariññeyyaṁ tassā’ti vadāmi.\\
}
\hspace{0pt} \\ 
\end{rightcolumn}
\end{samepage}

\begin{samepage}
\ensurevspace{4\baselineskip}
\begin{leftcolumn*}
\EnglishColumn{(UC - unbound consciousness)\\
“He directly knows the base of UC as the base of UC.\\
Having directly known the base of UC as the base of UC,
he should not conceive [himself as] the base of UC,\\
he should not conceive [himself] in the base of UC,
he should not conceive [himself apart] from the base of UC,\\
he should not conceive the base of UC to be ‘mine,’
he should not delight in the base of UC.\\
Why is that?
Because he must fully understand it, I say.\\
}
\hspace{0pt} \\ 
\end{leftcolumn*}

\begin{rightcolumn}
\PaliColumn{"viññāṇañcāyatanaṁ viññāṇañcāyatanato abhijānāti;\\
viññāṇañcāyatanaṁ viññāṇañcāyatanato abhiññāya viññāṇañcāyatanaṁ mā maññi,\\
viññāṇañcāyatanasmiṁ mā maññi,
viññāṇañcāyatanato mā maññi,\\
viññāṇañcāyatanaṁ meti mā maññi,
viññāṇañcāyatanaṁ mābhinandi.\\
taṁ kissa hetu?
‘pariññeyyaṁ tassā’ti vadāmi.\\
}
\hspace{0pt} \\ 
\end{rightcolumn}
\end{samepage}

\begin{samepage}
\ensurevspace{4\baselineskip}
\begin{leftcolumn*}
\EnglishColumn{(NT - no-thingness)\\
“He directly knows the base of NT as the base of NT.\\
Having directly known the base of NT as the base of NT,
he should not conceive [himself as] the base of NT,\\
he should not conceive [himself] in the base of NT,
he should not conceive [himself apart] from the base of NT,\\
he should not conceive the base of NT to be ‘mine,’
he should not delight in the base of NT.\\
Why is that?
Because he must fully understand it, I say.\\
}
\hspace{0pt} \\ 
\end{leftcolumn*}

\begin{rightcolumn}
\PaliColumn{"ākiñcaññāyatanaṁ ākiñcaññāyatanato abhijānāti;\\
ākiñcaññāyatanaṁ ākiñcaññāyatanato abhiññāya ākiñcaññāyatanaṁ mā maññi,\\
ākiñcaññāyatanasmiṁ mā maññi,
ākiñcaññāyatanato mā maññi,\\
ākiñcaññāyatanaṁ meti mā maññi,
ākiñcaññāyatanaṁ mābhinandi.\\
taṁ kissa hetu?
‘pariññeyyaṁ tassā’ti vadāmi.\\
}
\hspace{0pt} \\ 
\end{rightcolumn}
\end{samepage}

\begin{samepage}
\ensurevspace{4\baselineskip}
\begin{leftcolumn*}
\EnglishColumn{(NPnNP - neither-perception-nor-non-perception)\\
“He directly knows the base of NPnNP as the base of NPnNP.\\
Having directly known the base of NPnNP as the base of NPnNP,
he should not conceive [himself as] the base of NPnNP,\\
he should not conceive [himself] in the base of NPnNP,
he should not conceive [himself apart] from the base of NPnNP,\\
he should not conceive the base of NPnNP to be ‘mine,’
he should not delight in the base of NPnNP.\\
Why is that?
Because he must fully understand it, I say.\\
}
\hspace{0pt} \\ 
\end{leftcolumn*}

\begin{rightcolumn}
\PaliColumn{"nevasaññānāsaññāyatanaṁ nevasaññānāsaññāyatanato abhijānāti;\\
nevasaññānāsaññāyatanaṁ nevasaññānāsaññāyatanato abhiññāya nevasaññānāsaññāyatanaṁ mā maññi,\\
nevasaññānāsaññāyatanasmiṁ mā maññi,
nevasaññānāsaññāyatanato mā maññi,\\
nevasaññānāsaññāyatanaṁ meti mā maññi,
nevasaññānāsaññāyatanaṁ mābhinandi.\\
taṁ kissa hetu?
‘pariññeyyaṁ tassā’ti vadāmi.\\
}
\hspace{0pt} \\ 
\end{rightcolumn}
\end{samepage}

\begin{samepage}
\ensurevspace{4\baselineskip}
\begin{leftcolumn*}
\EnglishColumn{“He directly knows the seen as the seen.\\
Having directly known the seen as the seen,
he should not conceive [himself as] the seen,\\
he should not conceive [himself] in the seen,
he should not conceive [himself apart] from the seen,\\
he should not conceive the seen to be ‘mine,’
he should not delight in the seen.\\
Why is that?
Because he must fully understand it, I say.\\
}
\hspace{0pt} \\ 
\end{leftcolumn*}

\begin{rightcolumn}
\PaliColumn{"diṭṭhaṁ diṭṭhato abhijānāti;\\
diṭṭhaṁ diṭṭhato abhiññāya diṭṭhaṁ mā maññi,\\
diṭṭhasmiṁ mā maññi,
diṭṭhato mā maññi,\\
diṭṭhaṁ meti mā maññi,
diṭṭhaṁ mābhinandi.\\
taṁ kissa hetu?
‘pariññeyyaṁ tassā’ti vadāmi.\\
}
\hspace{0pt} \\ 
\end{rightcolumn}
\end{samepage}

\begin{samepage}
\ensurevspace{4\baselineskip}
\begin{leftcolumn*}
\EnglishColumn{“He directly knows the heard as the heard.\\
Having directly known the heard as the heard,
he should not conceive [himself as] the heard,\\
he should not conceive [himself] in the heard,
he should not conceive [himself apart] from the heard,\\
he should not conceive the heard to be ‘mine,’
he should not delight in the heard.\\
Why is that?
Because he must fully understand it, I say.\\
}
\hspace{0pt} \\ 
\end{leftcolumn*}

\begin{rightcolumn}
\PaliColumn{"sutaṁ sutato abhijānāti;\\
sutaṁ sutato abhiññāya sutaṁ mā maññi,\\
sutasmiṁ mā maññi,
sutato mā maññi,\\
sutaṁ meti mā maññi,
sutaṁ mābhinandi.\\
taṁ kissa hetu?
‘pariññeyyaṁ tassā’ti vadāmi.\\
}
\hspace{0pt} \\ 
\end{rightcolumn}
\end{samepage}

\begin{samepage}
\ensurevspace{4\baselineskip}
\begin{leftcolumn*}
\EnglishColumn{“He directly knows the sensed as the sensed.\\
Having directly known the sensed as the sensed,
he should not conceive [himself as] the sensed,\\
he should not conceive [himself] in the sensed,
he should not conceive [himself apart] from the sensed,\\
he should not conceive the sensed to be ‘mine,’
he should not delight in the sensed.\\
Why is that?
Because he must fully understand it, I say.\\
}
\hspace{0pt} \\ 
\end{leftcolumn*}

\begin{rightcolumn}
\PaliColumn{"mutaṁ mutato abhijānāti;\\
mutaṁ mutato abhiññāya mutaṁ mā maññi,\\
mutasmiṁ mā maññi,
mutato mā maññi,\\
mutaṁ meti mā maññi,
mutaṁ mābhinandi.\\
taṁ kissa hetu?
‘pariññeyyaṁ tassā’ti vadāmi.\\
}
\hspace{0pt} \\ 
\end{rightcolumn}
\end{samepage}

\begin{samepage}
\ensurevspace{4\baselineskip}
\begin{leftcolumn*}
\EnglishColumn{“He directly knows the cognized as the cognized.\\
Having directly known the cognized as the cognized,
he should not conceive [himself as] the cognized,\\
he should not conceive [himself] in the cognized,
he should not conceive [himself apart] from the cognized,\\
he should not conceive the cognized to be ‘mine,’
he should not delight in the cognized.\\
Why is that?
Because he must fully understand it, I say.\\
}
\hspace{0pt} \\ 
\end{leftcolumn*}

\begin{rightcolumn}
\PaliColumn{"viññātaṁ viññātato abhijānāti;\\
viññātaṁ viññātato abhiññāya viññātaṁ mā maññi,\\
viññātasmiṁ mā maññi,
viññātato mā maññi,\\
viññātaṁ meti mā maññi,
viññātaṁ mābhinandi.\\
taṁ kissa hetu?
‘pariññeyyaṁ tassā’ti vadāmi.\\
}
\hspace{0pt} \\ 
\end{rightcolumn}
\end{samepage}

\begin{samepage}
\ensurevspace{4\baselineskip}
\begin{leftcolumn*}
\EnglishColumn{“He directly knows unity as unity.\\
Having directly known unity as unity,
he should not conceive [himself as] unity,\\
he should not conceive [himself] in unity,
he should not conceive [himself apart] from unity,\\
he should not conceive unity to be ‘mine,’
he should not delight in unity.\\
Why is that?
Because he must fully understand it, I say.\\
}
\hspace{0pt} \\ 
\end{leftcolumn*}

\begin{rightcolumn}
\PaliColumn{"ekattaṁ ekattato abhijānāti;\\
ekattaṁ ekattato abhiññāya ekattaṁ mā maññi,\\
ekattasmiṁ mā maññi,
ekattato mā maññi,\\
ekattaṁ meti mā maññi,
ekattaṁ mābhinandi.\\
taṁ kissa hetu?
‘pariññeyyaṁ tassā’ti vadāmi.\\
}
\hspace{0pt} \\ 
\end{rightcolumn}
\end{samepage}

\begin{samepage}
\ensurevspace{4\baselineskip}
\begin{leftcolumn*}
\EnglishColumn{“He directly knows diversity as diversity.\\
Having directly known diversity as diversity,
he should not conceive [himself as] diversity,\\
he should not conceive [himself] in diversity,
he should not conceive [himself apart] from diversity,\\
he should not conceive diversity to be ‘mine,’
he should not delight in diversity.\\
Why is that?
Because he must fully understand it, I say.\\
}
\hspace{0pt} \\ 
\end{leftcolumn*}

\begin{rightcolumn}
\PaliColumn{"nānattaṁ nānattato abhijānāti;\\
nānattaṁ nānattato abhiññāya nānattaṁ mā maññi,\\
nānattasmiṁ mā maññi,
nānattato mā maññi,\\
nānattaṁ meti mā maññi,
nānattaṁ mābhinandi.\\
taṁ kissa hetu?
‘pariññeyyaṁ tassā’ti vadāmi.\\
}
\hspace{0pt} \\ 
\end{rightcolumn}
\end{samepage}

\begin{samepage}
\ensurevspace{4\baselineskip}
\begin{leftcolumn*}
\EnglishColumn{“He directly knows all as all.\\
Having directly known all as all,
he should not conceive [himself as] all,\\
he should not conceive [himself] in all,
he should not conceive [himself apart] from all,\\
he should not conceive all to be ‘mine,’
he should not delight in all.\\
Why is that?
Because he must fully understand it, I say.\\
}
\hspace{0pt} \\ 
\end{leftcolumn*}

\begin{rightcolumn}
\PaliColumn{"sabbaṁ sabbato abhijānāti;\\
sabbaṁ sabbato abhiññāya sabbaṁ mā maññi,\\
sabbasmiṁ mā maññi,
sabbato mā maññi,\\
sabbaṁ meti mā maññi,
sabbaṁ mābhinandi.\\
taṁ kissa hetu?
‘pariññeyyaṁ tassā’ti vadāmi.\\
}
\hspace{0pt} \\ 
\end{rightcolumn}
\end{samepage}

\begin{samepage}
\ensurevspace{4\baselineskip}
\begin{leftcolumn*}
\EnglishColumn{“He directly knows Nibbāna as Nibbāna.\\
Having directly known Nibbāna as Nibbāna,
he should not conceive [himself as] Nibbāna,\\
he should not conceive [himself] in Nibbāna,
he should not conceive [himself apart] from Nibbāna,\\
he should not conceive Nibbāna to be ‘mine,’
he should not delight in Nibbāna.\\
Why is that?
Because he must fully understand it, I say.\\
}
\hspace{0pt} \\ 
\end{leftcolumn*}

\begin{rightcolumn}
\PaliColumn{"nibbānaṁ nibbānato abhijānāti;\\
nibbānaṁ nibbānato abhiññāya nibbānaṁ mā maññi,\\
nibbānasmiṁ mā maññi,
nibbānato mā maññi,\\
nibbānaṁ meti mā maññi,
nibbānaṁ mābhinandi.\\
taṁ kissa hetu?
‘pariññeyyaṁ tassā’ti vadāmi.\\
}
\hspace{0pt} \\ 
\end{rightcolumn}
\end{samepage}

\begin{samepage}
\ensurevspace{4\baselineskip}
\begin{leftcolumn*}
\EnglishColumn{(THE ARAHANT — 1 to 4)
}
\hspace{0pt} \\ 
\end{leftcolumn*}

\begin{rightcolumn}
\PaliColumn{(ARAHAṀ - 1 to 4)
}
\hspace{0pt} \\ 
\end{rightcolumn}
\end{samepage}

\begin{samepage}
\ensurevspace{4\baselineskip}
\begin{leftcolumn*}
\EnglishColumn{“Bhikkhus, a bhikkhu who is an arahant with taints destroyed,
who has lived the holy life,
done what had to be done, 
laid down the burden,
reached his own goal,
destroyed the fetters of being,
and is completely liberated through final knowledge,
}
\hspace{0pt} \\ 
\end{leftcolumn*}

\begin{rightcolumn}
\PaliColumn{“yopi so, bhikkhave, bhikkhu arahaṁ khīṇāsavo vusitavā katakaraṇīyo ohitabhāro anuppattasadattho parikkhīṇabhavasaṁyojano sammadaññā vimutto,
}
\hspace{0pt} \\ 
\end{rightcolumn}
\end{samepage}

\begin{samepage}
\ensurevspace{4\baselineskip}
\begin{leftcolumn*}
\EnglishColumn{He too directly knows earth as earth. \\
having directly known earth as earth,
he does not conceive [himself as] earth,\\
he does not conceive [himself] in earth,
he does not conceive [himself apart] from earth,\\
he does not conceive earth to be ‘mine,’
he does not delight in earth.\\
Why is that?\\
(1) Because he has fully understood it, I say.\\
(2) Because he is free from lust through the destruction of lust.\\
(3) Because he is free from hate through the destruction of hate.\\
(4) Because he is free from delusion through the destruction of delusion.\\
}
\hspace{0pt} \\ 
\end{leftcolumn*}

\begin{rightcolumn}
\PaliColumn{sopi pathaviṁ pathavito abhijānāti;\\
pathaviṁ pathavito abhiññāya pathaviṁ na maññati,\\
pathaviyā na maññati,
pathavito na maññati,\\
pathaviṁ meti na maññati,
pathaviṁ nābhinandati.\\
taṁ kissa hetu?\\
(1) ‘pariññātaṁ tassā’ti vadāmi.\\
(2) khayā rāgassa, vītarāgattā.\\
(3) khayā dosassa, vītadosattā.\\
(4) khayā mohassa, vītamohattā.\\
}
\hspace{0pt} \\ 
\end{rightcolumn}
\end{samepage}

\begin{samepage}
\ensurevspace{4\baselineskip}
\begin{leftcolumn*}
\EnglishColumn{"He too directly knows water as water. \\
having directly known water as water,
he does not conceive [himself as] water,\\
he does not conceive [himself] in water,
he does not conceive [himself apart] from water,\\
he does not conceive water to be ‘mine,’
he does not delight in water.\\
Why is that?\\
(1) Because he has fully understood it, I say.\\
(2) Because he is free from lust through the destruction of lust.\\
(3) Because he is free from hate through the destruction of hate.\\
(4) Because he is free from delusion through the destruction of delusion.\\
}
\hspace{0pt} \\ 
\end{leftcolumn*}

\begin{rightcolumn}
\PaliColumn{"āpaṁ āpato abhijānāti;\\
āpaṁ āpato abhiññāya āpaṁ na maññati,\\
āpasmiṁ na maññati,
āpato na maññati,\\
āpaṁ meti na maññati,
āpaṁ nābhinandati.\\
taṁ kissa hetu?\\
(1) ‘pariññātaṁ tassā’ti vadāmi.\\
(2) khayā rāgassa, vītarāgattā.\\
(3) khayā dosassa, vītadosattā.\\
(4) khayā mohassa, vītamohattā.\\
}
\hspace{0pt} \\ 
\end{rightcolumn}
\end{samepage}

\begin{samepage}
\ensurevspace{4\baselineskip}
\begin{leftcolumn*}
\EnglishColumn{"He too directly knows fire as fire. \\
having directly known fire as fire,
he does not conceive [himself as] fire,\\
he does not conceive [himself] in fire,
he does not conceive [himself apart] from fire,\\
he does not conceive fire to be ‘mine,’
he does not delight in fire.\\
Why is that?\\
(1) Because he has fully understood it, I say.\\
(2) Because he is free from lust through the destruction of lust.\\
(3) Because he is free from hate through the destruction of hate.\\
(4) Because he is free from delusion through the destruction of delusion.\\
}
\hspace{0pt} \\ 
\end{leftcolumn*}

\begin{rightcolumn}
\PaliColumn{"tejaṁ tejato abhijānāti;\\
tejaṁ tejato abhiññāya tejaṁ na maññati,\\
tejasmiṁ na maññati,
tejato na maññati,\\
tejaṁ meti na maññati,
tejaṁ nābhinandati.\\
taṁ kissa hetu?\\
(1) ‘pariññātaṁ tassā’ti vadāmi.\\
(2) khayā rāgassa, vītarāgattā.\\
(3) khayā dosassa, vītadosattā.\\
(4) khayā mohassa, vītamohattā.\\
}
\hspace{0pt} \\ 
\end{rightcolumn}
\end{samepage}

\begin{samepage}
\ensurevspace{4\baselineskip}
\begin{leftcolumn*}
\EnglishColumn{"He too directly knows air as air. \\
having directly known air as air,
he does not conceive [himself as] air,\\
he does not conceive [himself] in air,
he does not conceive [himself apart] from air,\\
he does not conceive air to be ‘mine,’
he does not delight in air.\\
Why is that?\\
(1) Because he has fully understood it, I say.\\
(2) Because he is free from lust through the destruction of lust.\\
(3) Because he is free from hate through the destruction of hate.\\
(4) Because he is free from delusion through the destruction of delusion.\\
}
\hspace{0pt} \\ 
\end{leftcolumn*}

\begin{rightcolumn}
\PaliColumn{"vāyaṁ vāyato abhijānāti;\\
vāyaṁ vāyato abhiññāya vāyaṁ na maññati,\\
vāyasmiṁ na maññati,
vāyato na maññati,\\
vāyaṁ meti na maññati,
vāyaṁ nābhinandati.\\
taṁ kissa hetu?\\
(1) ‘pariññātaṁ tassā’ti vadāmi.\\
(2) khayā rāgassa, vītarāgattā.\\
(3) khayā dosassa, vītadosattā.\\
(4) khayā mohassa, vītamohattā.\\
}
\hspace{0pt} \\ 
\end{rightcolumn}
\end{samepage}

\begin{samepage}
\ensurevspace{4\baselineskip}
\begin{leftcolumn*}
\EnglishColumn{"He too directly knows beings as beings. \\
having directly known beings as beings,
he does not conceive [himself as] beings,\\
he does not conceive [himself] in beings,
he does not conceive [himself apart] from beings,\\
he does not conceive beings to be ‘mine,’
he does not delight in beings.\\
Why is that?\\
(1) Because he has fully understood it, I say.\\
(2) Because he is free from lust through the destruction of lust.\\
(3) Because he is free from hate through the destruction of hate.\\
(4) Because he is free from delusion through the destruction of delusion.\\
}
\hspace{0pt} \\ 
\end{leftcolumn*}

\begin{rightcolumn}
\PaliColumn{"bhūte bhūtato abhijānāti;\\
bhūte bhūtato abhiññāya bhūte na maññati,\\
bhūtesu na maññati,
bhūtato na maññati,\\
bhūte meti na maññati,
bhūte nābhinandati.\\
taṁ kissa hetu?\\
(1) ‘pariññātaṁ tassā’ti vadāmi.\\
(2) khayā rāgassa, vītarāgattā.\\
(3) khayā dosassa, vītadosattā.\\
(4) khayā mohassa, vītamohattā.\\
}
\hspace{0pt} \\ 
\end{rightcolumn}
\end{samepage}

\begin{samepage}
\ensurevspace{4\baselineskip}
\begin{leftcolumn*}
\EnglishColumn{"He too directly knows gods as gods. \\
having directly known gods as gods,
he does not conceive [himself as] gods,\\
he does not conceive [himself] in gods,
he does not conceive [himself apart] from gods,\\
he does not conceive gods to be ‘mine,’
he does not delight in gods.\\
Why is that?\\
(1) Because he has fully understood it, I say.\\
(2) Because he is free from lust through the destruction of lust.\\
(3) Because he is free from hate through the destruction of hate.\\
(4) Because he is free from delusion through the destruction of delusion.\\
}
\hspace{0pt} \\ 
\end{leftcolumn*}

\begin{rightcolumn}
\PaliColumn{"deve devato abhijānāti;\\
deve devato abhiññāya deve na maññati,\\
devesu na maññati,
devato na maññati,\\
deve meti na maññati,
deve nābhinandati.\\
taṁ kissa hetu?\\
(1) ‘pariññātaṁ tassā’ti vadāmi.\\
(2) khayā rāgassa, vītarāgattā.\\
(3) khayā dosassa, vītadosattā.\\
(4) khayā mohassa, vītamohattā.\\
}
\hspace{0pt} \\ 
\end{rightcolumn}
\end{samepage}

\begin{samepage}
\ensurevspace{4\baselineskip}
\begin{leftcolumn*}
\EnglishColumn{"He too directly knows Pajāpati as Pajāpati. \\
having directly known Pajāpati as Pajāpati,
he does not conceive [himself as] Pajāpati,\\
he does not conceive [himself] in Pajāpati,
he does not conceive [himself apart] from Pajāpati,\\
he does not conceive Pajāpati to be ‘mine,’
he does not delight in Pajāpati.\\
Why is that?\\
(1) Because he has fully understood it, I say.\\
(2) Because he is free from lust through the destruction of lust.\\
(3) Because he is free from hate through the destruction of hate.\\
(4) Because he is free from delusion through the destruction of delusion.\\
}
\hspace{0pt} \\ 
\end{leftcolumn*}

\begin{rightcolumn}
\PaliColumn{"pajāpatiṁ pajāpatito abhijānāti;\\
pajāpatiṁ pajāpatito abhiññāya pajāpatiṁ na maññati,\\
pajāpatismiṁ na maññati,
pajāpatito na maññati,\\
pajāpatiṁ meti na maññati,
pajāpatiṁ nābhinandati.\\
taṁ kissa hetu?\\
(1) ‘pariññātaṁ tassā’ti vadāmi.\\
(2) khayā rāgassa, vītarāgattā.\\
(3) khayā dosassa, vītadosattā.\\
(4) khayā mohassa, vītamohattā.\\
}
\hspace{0pt} \\ 
\end{rightcolumn}
\end{samepage}

\begin{samepage}
\ensurevspace{4\baselineskip}
\begin{leftcolumn*}
\EnglishColumn{"He too directly knows Brahmā as Brahmā. \\
having directly known Brahmā as Brahmā,
he does not conceive [himself as] Brahmā,\\
he does not conceive [himself] in Brahmā,
he does not conceive [himself apart] from Brahmā,\\
he does not conceive Brahmā to be ‘mine,’
he does not delight in Brahmā.\\
Why is that?\\
(1) Because he has fully understood it, I say.\\
(2) Because he is free from lust through the destruction of lust.\\
(3) Because he is free from hate through the destruction of hate.\\
(4) Because he is free from delusion through the destruction of delusion.\\
}
\hspace{0pt} \\ 
\end{leftcolumn*}

\begin{rightcolumn}
\PaliColumn{"brahmaṁ brahmato abhijānāti;\\
brahmaṁ brahmato abhiññāya brahmaṁ na maññati,\\
brahmasmiṁ na maññati,
brahmato na maññati,\\
brahmaṁ meti na maññati,
brahmaṁ nābhinandati.\\
taṁ kissa hetu?\\
(1) ‘pariññātaṁ tassā’ti vadāmi.\\
(2) khayā rāgassa, vītarāgattā.\\
(3) khayā dosassa, vītadosattā.\\
(4) khayā mohassa, vītamohattā.\\
}
\hspace{0pt} \\ 
\end{rightcolumn}
\end{samepage}

\begin{samepage}
\ensurevspace{4\baselineskip}
\begin{leftcolumn*}
\EnglishColumn{(SR - Streaming Radiance)\\
"He too directly knows the Gods of SR as the Gods of SR. \\
having directly known the Gods of SR as the Gods of SR,
he does not conceive [himself as] the Gods of SR,\\
he does not conceive [himself] in the Gods of SR,
he does not conceive [himself apart] from the Gods of SR,\\
he does not conceive the Gods of SR to be ‘mine,’
he does not delight in the Gods of SR.\\
Why is that?\\
(1) Because he has fully understood it, I say.\\
(2) Because he is free from lust through the destruction of lust.\\
(3) Because he is free from hate through the destruction of hate.\\
(4) Because he is free from delusion through the destruction of delusion.\\
}
\hspace{0pt} \\ 
\end{leftcolumn*}

\begin{rightcolumn}
\PaliColumn{"ābhassare ābhassarato abhijānāti;\\
ābhassare ābhassarato abhiññāya ābhassare na maññati,\\
ābhassaresu na maññati,
ābhassarato na maññati,\\
ābhassare meti na maññati,
ābhassare nābhinandati.\\
taṁ kissa hetu?\\
(1) ‘pariññātaṁ tassā’ti vadāmi.\\
(2) khayā rāgassa, vītarāgattā.\\
(3) khayā dosassa, vītadosattā.\\
(4) khayā mohassa, vītamohattā.\\
}
\hspace{0pt} \\ 
\end{rightcolumn}
\end{samepage}

\begin{samepage}
\ensurevspace{4\baselineskip}
\begin{leftcolumn*}
\EnglishColumn{(RG - Refulgent Glory)\\
"He too directly knows the Gods of RG as the Gods of RG. \\
having directly known the Gods of RG as the Gods of RG,
he does not conceive [himself as] the Gods of RG,\\
he does not conceive [himself] in the Gods of RG,
he does not conceive [himself apart] from the Gods of RG,\\
he does not conceive the Gods of RG to be ‘mine,’
he does not delight in the Gods of RG.\\
Why is that?\\
(1) Because he has fully understood it, I say.\\
(2) Because he is free from lust through the destruction of lust.\\
(3) Because he is free from hate through the destruction of hate.\\
(4) Because he is free from delusion through the destruction of delusion.\\
}
\hspace{0pt} \\ 
\end{leftcolumn*}

\begin{rightcolumn}
\PaliColumn{"subhakiṇhe subhakiṇhato abhijānāti;\\
subhakiṇhe subhakiṇhato abhiññāya subhakiṇhe na maññati,\\
subhakiṇhesu na maññati,
subhakiṇhato na maññati,\\
subhakiṇhe meti na maññati,
subhakiṇhe nābhinandati.\\
taṁ kissa hetu?\\
(1) ‘pariññātaṁ tassā’ti vadāmi.\\
(2) khayā rāgassa, vītarāgattā.\\
(3) khayā dosassa, vītadosattā.\\
(4) khayā mohassa, vītamohattā.\\
}
\hspace{0pt} \\ 
\end{rightcolumn}
\end{samepage}

\begin{samepage}
\ensurevspace{4\baselineskip}
\begin{leftcolumn*}
\EnglishColumn{(GF - Great Fruit)\\
"He too directly knows the Gods of GF as the Gods of GF.\\
having directly known the Gods of GF as the Gods of GF,
he does not conceive [himself as] the Gods of GF,\\
he does not conceive [himself] in the Gods of GF,
he does not conceive [himself apart] from the Gods of GF,\\
he does not conceive the Gods of GF to be ‘mine,’
he does not delight in the Gods of GF.\\
Why is that?\\
(1) Because he has fully understood it, I say.\\
(2) Because he is free from lust through the destruction of lust.\\
(3) Because he is free from hate through the destruction of hate.\\
(4) Because he is free from delusion through the destruction of delusion.\\
}
\hspace{0pt} \\ 
\end{leftcolumn*}

\begin{rightcolumn}
\PaliColumn{"vehapphale vehapphalato abhijānāti;\\
vehapphale vehapphalato abhiññāya vehapphale na maññati,\\
vehapphalesu na maññati,
vehapphalato na maññati,\\
vehapphale meti na maññati,
vehapphale nābhinandati.\\
taṁ kissa hetu?\\
(1) ‘pariññātaṁ tassā’ti vadāmi.\\
(2) khayā rāgassa, vītarāgattā.\\
(3) khayā dosassa, vītadosattā.\\
(4) khayā mohassa, vītamohattā.\\
}
\hspace{0pt} \\ 
\end{rightcolumn}
\end{samepage}

\begin{samepage}
\ensurevspace{4\baselineskip}
\begin{leftcolumn*}
\EnglishColumn{"He too directly knows the Overlord as the Overlord. \\
having directly known the Overlord as the Overlord,
he does not conceive [himself as] the Overlord,\\
he does not conceive [himself] in the Overlord,
he does not conceive [himself apart] from the Overlord,\\
he does not conceive the Overlord to be ‘mine,’
he does not delight in the Overlord.\\
Why is that?\\
(1) Because he has fully understood it, I say.\\
(2) Because he is free from lust through the destruction of lust.\\
(3) Because he is free from hate through the destruction of hate.\\
(4) Because he is free from delusion through the destruction of delusion.\\
}
\hspace{0pt} \\ 
\end{leftcolumn*}

\begin{rightcolumn}
\PaliColumn{"abhibhuṁ abhibhūto abhijānāti;\\
abhibhuṁ abhibhūto abhiññāya abhibhuṁ na maññati,\\
abhibhusmiṁ na maññati,
abhibhūto na maññati,\\
abhibhuṁ meti na maññati,
abhibhuṁ nābhinandati.\\
taṁ kissa hetu?\\
(1) ‘pariññātaṁ tassā’ti vadāmi.\\
(2) khayā rāgassa, vītarāgattā.\\
(3) khayā dosassa, vītadosattā.\\
(4) khayā mohassa, vītamohattā.\\
}
\hspace{0pt} \\ 
\end{rightcolumn}
\end{samepage}

\begin{samepage}
\ensurevspace{4\baselineskip}
\begin{leftcolumn*}
\EnglishColumn{(US - unbound space)\\
"He too directly knows the base of US as the base of US. \\
having directly known the base of US as the base of US,
he does not conceive [himself as] the base of US,\\
he does not conceive [himself] in the base of US,
he does not conceive [himself apart] from the base of US,\\
he does not conceive the base of US to be ‘mine,’
he does not delight in the base of US.\\
Why is that?\\
(1) Because he has fully understood it, I say.\\
(2) Because he is free from lust through the destruction of lust.\\
(3) Because he is free from hate through the destruction of hate.\\
(4) Because he is free from delusion through the destruction of delusion.\\
}
\hspace{0pt} \\ 
\end{leftcolumn*}

\begin{rightcolumn}
\PaliColumn{"ākāsānañcāyatanaṁ ākāsānañcāyatanato abhijānāti;\\
ākāsānañcāyatanaṁ ākāsānañcāyatanato abhiññāya ākāsānañcāyatanaṁ na maññati,\\
ākāsānañcāyatanasmiṁ na maññati,
ākāsānañcāyatanato na maññati,\\
ākāsānañcāyatanaṁ meti na maññati,
ākāsānañcāyatanaṁ nābhinandati.\\
taṁ kissa hetu?\\
(1) ‘pariññātaṁ tassā’ti vadāmi.\\
(2) khayā rāgassa, vītarāgattā.\\
(3) khayā dosassa, vītadosattā.\\
(4) khayā mohassa, vītamohattā.\\
}
\hspace{0pt} \\ 
\end{rightcolumn}
\end{samepage}

\begin{samepage}
\ensurevspace{4\baselineskip}
\begin{leftcolumn*}
\EnglishColumn{(UC - unbound consciousness)\\
"He too directly knows the base of UC as the base of UC. \\
having directly known the base of UC as the base of UC,
he does not conceive [himself as] the base of UC,\\
he does not conceive [himself] in the base of UC,
he does not conceive [himself apart] from the base of UC,\\
he does not conceive the base of UC to be ‘mine,’
he does not delight in the base of UC.\\
Why is that?\\
(1) Because he has fully understood it, I say.\\
(2) Because he is free from lust through the destruction of lust.\\
(3) Because he is free from hate through the destruction of hate.\\
(4) Because he is free from delusion through the destruction of delusion.\\
}
\hspace{0pt} \\ 
\end{leftcolumn*}

\begin{rightcolumn}
\PaliColumn{"viññāṇañcāyatanaṁ viññāṇañcāyatanato abhijānāti;\\
viññāṇañcāyatanaṁ viññāṇañcāyatanato abhiññāya viññāṇañcāyatanaṁ na maññati,\\
viññāṇañcāyatanasmiṁ na maññati,
viññāṇañcāyatanato na maññati,\\
viññāṇañcāyatanaṁ meti na maññati,
viññāṇañcāyatanaṁ nābhinandati.\\
taṁ kissa hetu?\\
(1) ‘pariññātaṁ tassā’ti vadāmi.\\
(2) khayā rāgassa, vītarāgattā.\\
(3) khayā dosassa, vītadosattā.\\
(4) khayā mohassa, vītamohattā.\\
}
\hspace{0pt} \\ 
\end{rightcolumn}
\end{samepage}

\begin{samepage}
\ensurevspace{4\baselineskip}
\begin{leftcolumn*}
\EnglishColumn{(NT - no-thingness)\\
"He too directly knows the base of NT as the base of NT. \\
having directly known the base of NT as the base of NT,
he does not conceive [himself as] the base of NT,\\
he does not conceive [himself] in the base of NT,
he does not conceive [himself apart] from the base of NT,\\
he does not conceive the base of NT to be ‘mine,’
he does not delight in the base of NT.\\
Why is that?\\
(1) Because he has fully understood it, I say.\\
(2) Because he is free from lust through the destruction of lust.\\
(3) Because he is free from hate through the destruction of hate.\\
(4) Because he is free from delusion through the destruction of delusion.\\
}
\hspace{0pt} \\ 
\end{leftcolumn*}

\begin{rightcolumn}
\PaliColumn{"ākiñcaññāyatanaṁ ākiñcaññāyatanato abhijānāti;\\
ākiñcaññāyatanaṁ ākiñcaññāyatanato abhiññāya ākiñcaññāyatanaṁ na maññati,\\
ākiñcaññāyatanasmiṁ na maññati,
ākiñcaññāyatanato na maññati,\\
ākiñcaññāyatanaṁ meti na maññati,
ākiñcaññāyatanaṁ nābhinandati.\\
taṁ kissa hetu?\\
(1) ‘pariññātaṁ tassā’ti vadāmi.\\
(2) khayā rāgassa, vītarāgattā.\\
(3) khayā dosassa, vītadosattā.\\
(4) khayā mohassa, vītamohattā.\\
}
\hspace{0pt} \\ 
\end{rightcolumn}
\end{samepage}

\begin{samepage}
\ensurevspace{4\baselineskip}
\begin{leftcolumn*}
\EnglishColumn{(NPnNP - neither-perception-nor-non-perception)\\
"He too directly knows the base of NPnNP as the base of NPnNP. \\
having directly known the base of NPnNP as the base of NPnNP,
he does not conceive [himself as] the base of NPnNP,\\
he does not conceive [himself] in the base of NPnNP,
he does not conceive [himself apart] from the base of NPnNP,\\
he does not conceive the base of NPnNP to be ‘mine,’
he does not delight in the base of NPnNP.\\
Why is that?\\
(1) Because he has fully understood it, I say.\\
(2) Because he is free from lust through the destruction of lust.\\
(3) Because he is free from hate through the destruction of hate.\\
(4) Because he is free from delusion through the destruction of delusion.\\
}
\hspace{0pt} \\ 
\end{leftcolumn*}

\begin{rightcolumn}
\PaliColumn{"nevasaññānāsaññāyatanaṁ nevasaññānāsaññāyatanato abhijānāti;\\
nevasaññānāsaññāyatanaṁ nevasaññānāsaññāyatanato abhiññāya nevasaññānāsaññāyatanaṁ na maññati,\\
nevasaññānāsaññāyatanasmiṁ na maññati,
nevasaññānāsaññāyatanato na maññati,\\
nevasaññānāsaññāyatanaṁ meti na maññati,
nevasaññānāsaññāyatanaṁ nābhinandati.\\
taṁ kissa hetu?\\
(1) ‘pariññātaṁ tassā’ti vadāmi.\\
(2) khayā rāgassa, vītarāgattā.\\
(3) khayā dosassa, vītadosattā.\\
(4) khayā mohassa, vītamohattā.\\
}
\hspace{0pt} \\ 
\end{rightcolumn}
\end{samepage}

\begin{samepage}
\ensurevspace{4\baselineskip}
\begin{leftcolumn*}
\EnglishColumn{"He too directly knows the seen as the seen. \\
having directly known the seen as the seen,
he does not conceive [himself as] the seen,\\
he does not conceive [himself] in the seen,
he does not conceive [himself apart] from the seen,\\
he does not conceive the seen to be ‘mine,’
he does not delight in the seen.\\
Why is that?\\
(1) Because he has fully understood it, I say.\\
(2) Because he is free from lust through the destruction of lust.\\
(3) Because he is free from hate through the destruction of hate.\\
(4) Because he is free from delusion through the destruction of delusion.\\
}
\hspace{0pt} \\ 
\end{leftcolumn*}

\begin{rightcolumn}
\PaliColumn{"diṭṭhaṁ diṭṭhato abhijānāti;\\
diṭṭhaṁ diṭṭhato abhiññāya diṭṭhaṁ na maññati,\\
diṭṭhasmiṁ na maññati,
diṭṭhato na maññati,\\
diṭṭhaṁ meti na maññati,
diṭṭhaṁ nābhinandati.\\
taṁ kissa hetu?\\
(1) ‘pariññātaṁ tassā’ti vadāmi.\\
(2) khayā rāgassa, vītarāgattā.\\
(3) khayā dosassa, vītadosattā.\\
(4) khayā mohassa, vītamohattā.\\
}
\hspace{0pt} \\ 
\end{rightcolumn}
\end{samepage}

\begin{samepage}
\ensurevspace{4\baselineskip}
\begin{leftcolumn*}
\EnglishColumn{"He too directly knows the heard as the heard. \\
having directly known the heard as the heard,
he does not conceive [himself as] the heard,\\
he does not conceive [himself] in the heard,
he does not conceive [himself apart] from the heard,\\
he does not conceive the heard to be ‘mine,’
he does not delight in the heard.\\
Why is that?\\
(1) Because he has fully understood it, I say.\\
(2) Because he is free from lust through the destruction of lust.\\
(3) Because he is free from hate through the destruction of hate.\\
(4) Because he is free from delusion through the destruction of delusion.\\
}
\hspace{0pt} \\ 
\end{leftcolumn*}

\begin{rightcolumn}
\PaliColumn{"sutaṁ sutato abhijānāti;\\
sutaṁ sutato abhiññāya sutaṁ na maññati,\\
sutasmiṁ na maññati,
sutato na maññati,\\
sutaṁ meti na maññati,
sutaṁ nābhinandati.\\
taṁ kissa hetu?\\
(1) ‘pariññātaṁ tassā’ti vadāmi.\\
(2) khayā rāgassa, vītarāgattā.\\
(3) khayā dosassa, vītadosattā.\\
(4) khayā mohassa, vītamohattā.\\
}
\hspace{0pt} \\ 
\end{rightcolumn}
\end{samepage}

\begin{samepage}
\ensurevspace{4\baselineskip}
\begin{leftcolumn*}
\EnglishColumn{"He too directly knows the sensed as the sensed. \\
having directly known the sensed as the sensed,
he does not conceive [himself as] the sensed,\\
he does not conceive [himself] in the sensed,
he does not conceive [himself apart] from the sensed,\\
he does not conceive the sensed to be ‘mine,’
he does not delight in the sensed.\\
Why is that?\\
(1) Because he has fully understood it, I say.\\
(2) Because he is free from lust through the destruction of lust.\\
(3) Because he is free from hate through the destruction of hate.\\
(4) Because he is free from delusion through the destruction of delusion.\\
}
\hspace{0pt} \\ 
\end{leftcolumn*}

\begin{rightcolumn}
\PaliColumn{"mutaṁ mutato abhijānāti;\\
mutaṁ mutato abhiññāya mutaṁ na maññati,\\
mutasmiṁ na maññati,
mutato na maññati,\\
mutaṁ meti na maññati,
mutaṁ nābhinandati.\\
taṁ kissa hetu?\\
(1) ‘pariññātaṁ tassā’ti vadāmi.\\
(2) khayā rāgassa, vītarāgattā.\\
(3) khayā dosassa, vītadosattā.\\
(4) khayā mohassa, vītamohattā.\\
}
\hspace{0pt} \\ 
\end{rightcolumn}
\end{samepage}

\begin{samepage}
\ensurevspace{4\baselineskip}
\begin{leftcolumn*}
\EnglishColumn{"He too directly knows the cognized as the cognized. \\
having directly known the cognized as the cognized,
he does not conceive [himself as] the cognized,\\
he does not conceive [himself] in the cognized,
he does not conceive [himself apart] from the cognized,\\
he does not conceive the cognized to be ‘mine,’
he does not delight in the cognized.\\
Why is that?\\
(1) Because he has fully understood it, I say.\\
(2) Because he is free from lust through the destruction of lust.\\
(3) Because he is free from hate through the destruction of hate.\\
(4) Because he is free from delusion through the destruction of delusion.\\
}
\hspace{0pt} \\ 
\end{leftcolumn*}

\begin{rightcolumn}
\PaliColumn{"viññātaṁ viññātato abhijānāti;\\
viññātaṁ viññātato abhiññāya viññātaṁ na maññati,\\
viññātasmiṁ na maññati,
viññātato na maññati,\\
viññātaṁ meti na maññati,
viññātaṁ nābhinandati.\\
taṁ kissa hetu?\\
(1) ‘pariññātaṁ tassā’ti vadāmi.\\
(2) khayā rāgassa, vītarāgattā.\\
(3) khayā dosassa, vītadosattā.\\
(4) khayā mohassa, vītamohattā.\\
}
\hspace{0pt} \\ 
\end{rightcolumn}
\end{samepage}

\begin{samepage}
\ensurevspace{4\baselineskip}
\begin{leftcolumn*}
\EnglishColumn{"He too directly knows unity as unity. \\
having directly known unity as unity,
he does not conceive [himself as] unity,\\
he does not conceive [himself] in unity,
he does not conceive [himself apart] from unity,\\
he does not conceive unity to be ‘mine,’
he does not delight in unity.\\
Why is that?\\
(1) Because he has fully understood it, I say.\\
(2) Because he is free from lust through the destruction of lust.\\
(3) Because he is free from hate through the destruction of hate.\\
(4) Because he is free from delusion through the destruction of delusion.\\
}
\hspace{0pt} \\ 
\end{leftcolumn*}

\begin{rightcolumn}
\PaliColumn{"ekattaṁ ekattato abhijānāti;\\
ekattaṁ ekattato abhiññāya ekattaṁ na maññati,\\
ekattasmiṁ na maññati,
ekattato na maññati,\\
ekattaṁ meti na maññati,
ekattaṁ nābhinandati.\\
taṁ kissa hetu?\\
(1) ‘pariññātaṁ tassā’ti vadāmi.\\
(2) khayā rāgassa, vītarāgattā.\\
(3) khayā dosassa, vītadosattā.\\
(4) khayā mohassa, vītamohattā.\\
}
\hspace{0pt} \\ 
\end{rightcolumn}
\end{samepage}

\begin{samepage}
\ensurevspace{4\baselineskip}
\begin{leftcolumn*}
\EnglishColumn{"He too directly knows diversity as diversity. \\
having directly known diversity as diversity,
he does not conceive [himself as] diversity,\\
he does not conceive [himself] in diversity,
he does not conceive [himself apart] from diversity,\\
he does not conceive diversity to be ‘mine,’
he does not delight in diversity.\\
Why is that?\\
(1) Because he has fully understood it, I say.\\
(2) Because he is free from lust through the destruction of lust.\\
(3) Because he is free from hate through the destruction of hate.\\
(4) Because he is free from delusion through the destruction of delusion.\\
}
\hspace{0pt} \\ 
\end{leftcolumn*}

\begin{rightcolumn}
\PaliColumn{"nānattaṁ nānattato abhijānāti;\\
nānattaṁ nānattato abhiññāya nānattaṁ na maññati,\\
nānattasmiṁ na maññati,
nānattato na maññati,\\
nānattaṁ meti na maññati,
nānattaṁ nābhinandati.\\
taṁ kissa hetu?\\
(1) ‘pariññātaṁ tassā’ti vadāmi.\\
(2) khayā rāgassa, vītarāgattā.\\
(3) khayā dosassa, vītadosattā.\\
(4) khayā mohassa, vītamohattā.\\
}
\hspace{0pt} \\ 
\end{rightcolumn}
\end{samepage}

\begin{samepage}
\ensurevspace{4\baselineskip}
\begin{leftcolumn*}
\EnglishColumn{"He too directly knows all as all. \\
having directly known all as all,
he does not conceive [himself as] all,\\
he does not conceive [himself] in all,
he does not conceive [himself apart] from all,\\
he does not conceive all to be ‘mine,’
he does not delight in all.\\
Why is that?\\
(1) Because he has fully understood it, I say.\\
(2) Because he is free from lust through the destruction of lust.\\
(3) Because he is free from hate through the destruction of hate.\\
(4) Because he is free from delusion through the destruction of delusion.\\
}
\hspace{0pt} \\ 
\end{leftcolumn*}

\begin{rightcolumn}
\PaliColumn{"sabbaṁ sabbato abhijānāti;\\
sabbaṁ sabbato abhiññāya sabbaṁ na maññati,\\
sabbasmiṁ na maññati,
sabbato na maññati,\\
sabbaṁ meti na maññati,
sabbaṁ nābhinandati.\\
taṁ kissa hetu?\\
(1) ‘pariññātaṁ tassā’ti vadāmi.\\
(2) khayā rāgassa, vītarāgattā.\\
(3) khayā dosassa, vītadosattā.\\
(4) khayā mohassa, vītamohattā.\\
}
\hspace{0pt} \\ 
\end{rightcolumn}
\end{samepage}

\begin{samepage}
\ensurevspace{4\baselineskip}
\begin{leftcolumn*}
\EnglishColumn{"He too directly knows Nibbāna as Nibbāna. \\
having directly known Nibbāna as Nibbāna,
he does not conceive [himself as] Nibbāna,\\
he does not conceive [himself] in Nibbāna,
he does not conceive [himself apart] from Nibbāna,\\
he does not conceive Nibbāna to be ‘mine,’
he does not delight in Nibbāna.\\
Why is that?\\
(1) Because he has fully understood it, I say.\\
(2) Because he is free from lust through the destruction of lust.\\
(3) Because he is free from hate through the destruction of hate.\\
(4) Because he is free from delusion through the destruction of delusion.\\
}
\hspace{0pt} \\ 
\end{leftcolumn*}

\begin{rightcolumn}
\PaliColumn{"nibbānaṁ nibbānato abhijānāti\\
nibbānaṁ nibbānato abhiññāya nibbānaṁ na maññati,\\
nibbānasmiṁ na maññati,
nibbānato na maññati,\\
nibbānaṁ meti na maññati,
nibbānaṁ nābhinandati.\\
taṁ kissa hetu?\\
(1) ‘pariññātaṁ tassā’ti vadāmi.\\
(2) khayā rāgassa, vītarāgattā.\\
(3) khayā dosassa, vītadosattā.\\
(4) khayā mohassa, vītamohattā.\\
}
\hspace{0pt} \\ 
\end{rightcolumn}
\end{samepage}

\begin{samepage}
\ensurevspace{4\baselineskip}
\begin{leftcolumn*}
\EnglishColumn{(THE TATHĀGATA — 1 \& 2)
}
\hspace{0pt} \\ 
\end{leftcolumn*}

\begin{rightcolumn}
\PaliColumn{(TATHĀGATO- 1 \& 2)
}
\hspace{0pt} \\ 
\end{rightcolumn}
\end{samepage}

\begin{samepage}
\ensurevspace{4\baselineskip}
\begin{leftcolumn*}
\EnglishColumn{147. “Bhikkhus, the Tathāgata, too, accomplished and fully enlightened,
}
\hspace{0pt} \\ 
\end{leftcolumn*}

\begin{rightcolumn}
\PaliColumn{12. “tathāgatopi, bhikkhave, arahaṁ sammāsambuddho
}
\hspace{0pt} \\ 
\end{rightcolumn}
\end{samepage}

\begin{samepage}
\ensurevspace{4\baselineskip}
\begin{leftcolumn*}
\EnglishColumn{directly knows earth as earth.\\
Having directly known earth as earth,
he does not conceive [himself as] earth,\\
he does not conceive [himself] in earth,
he does not conceive [himself apart] from earth,\\
he does not conceive earth to be ‘mine,’
he does not delight in earth.\\
Why is that?
(1) Because the Tathāgata has fully understood it to the end, I say.\\
(2) Because he has understood that delight is the root of suffering,
and that with being [as condition] there is birth,
and that for whatever has come to be there is ageing and death.
Therefore, bhikkhus, through the complete destruction, fading away, cessation, giving up, and relinquishing of cravings,
the Tathāgata has awakened to supreme full enlightenment, I say.”\\
}
\hspace{0pt} \\ 
\end{leftcolumn*}

\begin{rightcolumn}
\PaliColumn{pathaviṁ pathavito abhijānāti;\\
pathaviṁ pathavito abhiññāya pathaviṁ na maññati,\\
pathaviyā na maññati,
pathavito na maññati,\\
pathaviṁ meti na maññati,
pathaviṁ nābhinandati.\\
taṁ kissa hetu?
(1) ‘pariññātantaṁ tathāgatassā’ti vadāmi.\\
(2) ‘nandī dukkhassa mūlan’ti — iti viditvā ‘bhavā jāti bhūtassa jarāmaraṇan’ti. tasmātiha, bhikkhave, ‘tathāgato sabbaso taṇhānaṁ khayā virāgā nirodhā cāgā paṭinissaggā anuttaraṁ sammāsambodhiṁ abhisambuddho’ti vadāmi.\\
}
\hspace{0pt} \\ 
\end{rightcolumn}
\end{samepage}

\begin{samepage}
\ensurevspace{4\baselineskip}
\begin{leftcolumn*}
\EnglishColumn{"He too directly knows water as water.\\
Having directly known water as water,
he does not conceive [himself as] water,\\
he does not conceive [himself] in water,
he does not conceive [himself apart] from water,\\
he does not conceive water to be ‘mine,’
he does not delight in water.\\
Why is that?
(1) Because the Tathāgata has fully understood it to the end, I say.\\
(2) Because he has understood that delight is the root of suffering,
and that with being [as condition] there is birth,
and that for whatever has come to be there is ageing and death.
Therefore, bhikkhus, through the complete destruction, fading away, cessation, giving up, and relinquishing of cravings,
the Tathāgata has awakened to supreme full enlightenment, I say.”\\
}
\hspace{0pt} \\ 
\end{leftcolumn*}

\begin{rightcolumn}
\PaliColumn{āpaṁ āpato abhijānāti;\\
āpaṁ āpato abhiññāya āpaṁ na maññati,\\
āpasmiṁ na maññati,
āpato na maññati,\\
āpaṁ meti na maññati,
āpaṁ nābhinandati.\\
taṁ kissa hetu?
(1) ‘pariññātantaṁ tathāgatassā’ti vadāmi.\\
(2) ‘nandī dukkhassa mūlan’ti — iti viditvā ‘bhavā jāti bhūtassa jarāmaraṇan’ti. tasmātiha, bhikkhave, ‘tathāgato sabbaso taṇhānaṁ khayā virāgā nirodhā cāgā paṭinissaggā anuttaraṁ sammāsambodhiṁ abhisambuddho’ti vadāmi.\\
}
\hspace{0pt} \\ 
\end{rightcolumn}
\end{samepage}

\begin{samepage}
\ensurevspace{4\baselineskip}
\begin{leftcolumn*}
\EnglishColumn{"He too directly knows fire as fire.\\
Having directly known fire as fire,
he does not conceive [himself as] fire,\\
he does not conceive [himself] in fire,
he does not conceive [himself apart] from fire,\\
he does not conceive fire to be ‘mine,’
he does not delight in fire.\\
Why is that?
(1) Because the Tathāgata has fully understood it to the end, I say.\\
(2) Because he has understood that delight is the root of suffering,
and that with being [as condition] there is birth,
and that for whatever has come to be there is ageing and death.
Therefore, bhikkhus, through the complete destruction, fading away, cessation, giving up, and relinquishing of cravings,
the Tathāgata has awakened to supreme full enlightenment, I say.”\\
}
\hspace{0pt} \\ 
\end{leftcolumn*}

\begin{rightcolumn}
\PaliColumn{tejaṁ tejato abhijānāti;\\
tejaṁ tejato abhiññāya tejaṁ na maññati,\\
tejasmiṁ na maññati,
tejato na maññati,\\
tejaṁ meti na maññati,
tejaṁ nābhinandati.\\
taṁ kissa hetu?
(1) ‘pariññātantaṁ tathāgatassā’ti vadāmi.\\
(2) ‘nandī dukkhassa mūlan’ti — iti viditvā ‘bhavā jāti bhūtassa jarāmaraṇan’ti. tasmātiha, bhikkhave, ‘tathāgato sabbaso taṇhānaṁ khayā virāgā nirodhā cāgā paṭinissaggā anuttaraṁ sammāsambodhiṁ abhisambuddho’ti vadāmi.\\
}
\hspace{0pt} \\ 
\end{rightcolumn}
\end{samepage}

\begin{samepage}
\ensurevspace{4\baselineskip}
\begin{leftcolumn*}
\EnglishColumn{"He too directly knows air as air.\\
Having directly known air as air,
he does not conceive [himself as] air,\\
he does not conceive [himself] in air,
he does not conceive [himself apart] from air,\\
he does not conceive air to be ‘mine,’
he does not delight in air.\\
Why is that?
(1) Because the Tathāgata has fully understood it to the end, I say.\\
(2) Because he has understood that delight is the root of suffering,
and that with being [as condition] there is birth,
and that for whatever has come to be there is ageing and death.
Therefore, bhikkhus, through the complete destruction, fading away, cessation, giving up, and relinquishing of cravings,
the Tathāgata has awakened to supreme full enlightenment, I say.”\\
}
\hspace{0pt} \\ 
\end{leftcolumn*}

\begin{rightcolumn}
\PaliColumn{vāyaṁ vāyato abhijānāti;\\
vāyaṁ vāyato abhiññāya vāyaṁ na maññati,\\
vāyasmiṁ na maññati,
vāyato na maññati,\\
vāyaṁ meti na maññati,
vāyaṁ nābhinandati.\\
taṁ kissa hetu?
(1) ‘pariññātantaṁ tathāgatassā’ti vadāmi.\\
(2) ‘nandī dukkhassa mūlan’ti — iti viditvā ‘bhavā jāti bhūtassa jarāmaraṇan’ti. tasmātiha, bhikkhave, ‘tathāgato sabbaso taṇhānaṁ khayā virāgā nirodhā cāgā paṭinissaggā anuttaraṁ sammāsambodhiṁ abhisambuddho’ti vadāmi.\\
}
\hspace{0pt} \\ 
\end{rightcolumn}
\end{samepage}

\begin{samepage}
\ensurevspace{4\baselineskip}
\begin{leftcolumn*}
\EnglishColumn{"He too directly knows beings as beings.\\
Having directly known beings as beings,
he does not conceive [himself as] beings,\\
he does not conceive [himself] in beings,
he does not conceive [himself apart] from beings,\\
he does not conceive beings to be ‘mine,’
he does not delight in beings.\\
Why is that?
(1) Because the Tathāgata has fully understood it to the end, I say.\\
(2) Because he has understood that delight is the root of suffering,
and that with being [as condition] there is birth,
and that for whatever has come to be there is ageing and death.
Therefore, bhikkhus, through the complete destruction, fading away, cessation, giving up, and relinquishing of cravings,
the Tathāgata has awakened to supreme full enlightenment, I say.”\\
}
\hspace{0pt} \\ 
\end{leftcolumn*}

\begin{rightcolumn}
\PaliColumn{bhūte bhūtato abhijānāti;\\
bhūte bhūtato abhiññāya bhūte na maññati,\\
bhūtesu na maññati,
bhūtato na maññati,\\
bhūte meti na maññati,
bhūte nābhinandati.\\
taṁ kissa hetu?
(1) ‘pariññātantaṁ tathāgatassā’ti vadāmi.\\
(2) ‘nandī dukkhassa mūlan’ti — iti viditvā ‘bhavā jāti bhūtassa jarāmaraṇan’ti. tasmātiha, bhikkhave, ‘tathāgato sabbaso taṇhānaṁ khayā virāgā nirodhā cāgā paṭinissaggā anuttaraṁ sammāsambodhiṁ abhisambuddho’ti vadāmi.\\
}
\hspace{0pt} \\ 
\end{rightcolumn}
\end{samepage}

\begin{samepage}
\ensurevspace{4\baselineskip}
\begin{leftcolumn*}
\EnglishColumn{"He too directly knows gods as gods.\\
Having directly known gods as gods,
he does not conceive [himself as] gods,\\
he does not conceive [himself] in gods,
he does not conceive [himself apart] from gods,\\
he does not conceive gods to be ‘mine,’
he does not delight in gods.\\
Why is that?
(1) Because the Tathāgata has fully understood it to the end, I say.\\
(2) Because he has understood that delight is the root of suffering,
and that with being [as condition] there is birth,
and that for whatever has come to be there is ageing and death.
Therefore, bhikkhus, through the complete destruction, fading away, cessation, giving up, and relinquishing of cravings,
the Tathāgata has awakened to supreme full enlightenment, I say.”\\
}
\hspace{0pt} \\ 
\end{leftcolumn*}

\begin{rightcolumn}
\PaliColumn{deve devato abhijānāti;\\
deve devato abhiññāya deve na maññati,\\
devesu na maññati,
devato na maññati,\\
deve meti na maññati,
deve nābhinandati.\\
taṁ kissa hetu?
(1) ‘pariññātantaṁ tathāgatassā’ti vadāmi.\\
(2) ‘nandī dukkhassa mūlan’ti — iti viditvā ‘bhavā jāti bhūtassa jarāmaraṇan’ti. tasmātiha, bhikkhave, ‘tathāgato sabbaso taṇhānaṁ khayā virāgā nirodhā cāgā paṭinissaggā anuttaraṁ sammāsambodhiṁ abhisambuddho’ti vadāmi.\\
}
\hspace{0pt} \\ 
\end{rightcolumn}
\end{samepage}

\begin{samepage}
\ensurevspace{4\baselineskip}
\begin{leftcolumn*}
\EnglishColumn{"He too directly knows Pajāpati as Pajāpati.\\
Having directly known Pajāpati as Pajāpati,
he does not conceive [himself as] Pajāpati,\\
he does not conceive [himself] in Pajāpati,
he does not conceive [himself apart] from Pajāpati,\\
he does not conceive Pajāpati to be ‘mine,’
he does not delight in Pajāpati.\\
Why is that?
(1) Because the Tathāgata has fully understood it to the end, I say.\\
(2) Because he has understood that delight is the root of suffering,
and that with being [as condition] there is birth,
and that for whatever has come to be there is ageing and death.
Therefore, bhikkhus, through the complete destruction, fading away, cessation, giving up, and relinquishing of cravings,
the Tathāgata has awakened to supreme full enlightenment, I say.”\\
}
\hspace{0pt} \\ 
\end{leftcolumn*}

\begin{rightcolumn}
\PaliColumn{pajāpatiṁ pajāpatito abhijānāti;\\
pajāpatiṁ pajāpatito abhiññāya pajāpatiṁ na maññati,\\
pajāpatismiṁ na maññati,
pajāpatito na maññati,\\
pajāpatiṁ meti na maññati,
pajāpatiṁ nābhinandati.\\
taṁ kissa hetu?
(1) ‘pariññātantaṁ tathāgatassā’ti vadāmi.\\
(2) ‘nandī dukkhassa mūlan’ti — iti viditvā ‘bhavā jāti bhūtassa jarāmaraṇan’ti. tasmātiha, bhikkhave, ‘tathāgato sabbaso taṇhānaṁ khayā virāgā nirodhā cāgā paṭinissaggā anuttaraṁ sammāsambodhiṁ abhisambuddho’ti vadāmi.\\
}
\hspace{0pt} \\ 
\end{rightcolumn}
\end{samepage}

\begin{samepage}
\ensurevspace{4\baselineskip}
\begin{leftcolumn*}
\EnglishColumn{"He too directly knows Brahmā as Brahmā.\\
Having directly known Brahmā as Brahmā,
he does not conceive [himself as] Brahmā,\\
he does not conceive [himself] in Brahmā,
he does not conceive [himself apart] from Brahmā,\\
he does not conceive Brahmā to be ‘mine,’
he does not delight in Brahmā.\\
Why is that?
(1) Because the Tathāgata has fully understood it to the end, I say.\\
(2) Because he has understood that delight is the root of suffering,
and that with being [as condition] there is birth,
and that for whatever has come to be there is ageing and death.
Therefore, bhikkhus, through the complete destruction, fading away, cessation, giving up, and relinquishing of cravings,
the Tathāgata has awakened to supreme full enlightenment, I say.”\\
}
\hspace{0pt} \\ 
\end{leftcolumn*}

\begin{rightcolumn}
\PaliColumn{brahmaṁ brahmato abhijānāti;\\
brahmaṁ brahmato abhiññāya brahmaṁ na maññati,\\
brahmasmiṁ na maññati,
brahmato na maññati,\\
brahmaṁ meti na maññati,
brahmaṁ nābhinandati.\\
taṁ kissa hetu?
(1) ‘pariññātantaṁ tathāgatassā’ti vadāmi.\\
(2) ‘nandī dukkhassa mūlan’ti — iti viditvā ‘bhavā jāti bhūtassa jarāmaraṇan’ti. tasmātiha, bhikkhave, ‘tathāgato sabbaso taṇhānaṁ khayā virāgā nirodhā cāgā paṭinissaggā anuttaraṁ sammāsambodhiṁ abhisambuddho’ti vadāmi.\\
}
\hspace{0pt} \\ 
\end{rightcolumn}
\end{samepage}

\begin{samepage}
\ensurevspace{4\baselineskip}
\begin{leftcolumn*}
\EnglishColumn{(SR - Streaming Radiance)\\
"He too directly knows the Gods of SR as the Gods of SR.\\
Having directly known the Gods of SR as the Gods of SR,
he does not conceive [himself as] the Gods of SR,\\
he does not conceive [himself] in the Gods of SR,
he does not conceive [himself apart] from the Gods of SR,\\
he does not conceive the Gods of SR to be ‘mine,’
he does not delight in the Gods of SR.\\
Why is that?
(1) Because the Tathāgata has fully understood it to the end, I say.\\
(2) Because he has understood that delight is the root of suffering,
and that with being [as condition] there is birth,
and that for whatever has come to be there is ageing and death.
Therefore, bhikkhus, through the complete destruction, fading away, cessation, giving up, and relinquishing of cravings,
the Tathāgata has awakened to supreme full enlightenment, I say.”\\
}
\hspace{0pt} \\ 
\end{leftcolumn*}

\begin{rightcolumn}
\PaliColumn{ābhassare ābhassarato abhijānāti;\\
ābhassare ābhassarato abhiññāya ābhassare na maññati,\\
ābhassaresu na maññati,
ābhassarato na maññati,\\
ābhassare meti na maññati,
ābhassare nābhinandati.\\
taṁ kissa hetu?
(1) ‘nandī dukkhassa mūlan’ti — iti viditvā ‘bhavā jāti bhūtassa jarāmaraṇan’ti. ‘pariññātantaṁ tathāgatassā’ti vadāmi.\\
(2) tasmātiha, bhikkhave, ‘tathāgato sabbaso taṇhānaṁ khayā virāgā nirodhā cāgā paṭinissaggā anuttaraṁ sammāsambodhiṁ abhisambuddho’ti vadāmi.\\
}
\hspace{0pt} \\ 
\end{rightcolumn}
\end{samepage}

\begin{samepage}
\ensurevspace{4\baselineskip}
\begin{leftcolumn*}
\EnglishColumn{(RG - Refulgent Glory)\\
"He too directly knows the Gods of RG as the Gods of RG.\\
Having directly known the Gods of RG as the Gods of RG,
he does not conceive [himself as] the Gods of RG,\\
he does not conceive [himself] in the Gods of RG,
he does not conceive [himself apart] from the Gods of RG,\\
he does not conceive the Gods of RG to be ‘mine,’
he does not delight in the Gods of RG.\\
Why is that?
(1) Because the Tathāgata has fully understood it to the end, I say.\\
(2) Because he has understood that delight is the root of suffering,
and that with being [as condition] there is birth,
and that for whatever has come to be there is ageing and death.
Therefore, bhikkhus, through the complete destruction, fading away, cessation, giving up, and relinquishing of cravings,
the Tathāgata has awakened to supreme full enlightenment, I say.”\\
}
\hspace{0pt} \\ 
\end{leftcolumn*}

\begin{rightcolumn}
\PaliColumn{subhakiṇhe subhakiṇhato abhijānāti;\\
subhakiṇhe subhakiṇhato abhiññāya subhakiṇhe na maññati,\\
subhakiṇhesu na maññati,
subhakiṇhato na maññati,\\
subhakiṇhe meti na maññati,
subhakiṇhe nābhinandati.\\
taṁ kissa hetu?
(1) ‘pariññātantaṁ tathāgatassā’ti vadāmi.\\
(2) ‘nandī dukkhassa mūlan’ti — iti viditvā ‘bhavā jāti bhūtassa jarāmaraṇan’ti. tasmātiha, bhikkhave, ‘tathāgato sabbaso taṇhānaṁ khayā virāgā nirodhā cāgā paṭinissaggā anuttaraṁ sammāsambodhiṁ abhisambuddho’ti vadāmi.\\
}
\hspace{0pt} \\ 
\end{rightcolumn}
\end{samepage}

\begin{samepage}
\ensurevspace{4\baselineskip}
\begin{leftcolumn*}
\EnglishColumn{(GF - Great Fruit)\\
"He too directly knows the Gods of GF as the Gods of GF.\\
Having directly known the Gods of GF as the Gods of GF,
he does not conceive [himself as] the Gods of GF,\\
he does not conceive [himself] in the Gods of GF,
he does not conceive [himself apart] from the Gods of GF,\\
he does not conceive the Gods of GF to be ‘mine,’
he does not delight in the Gods of GF.\\
Why is that?
(1) Because the Tathāgata has fully understood it to the end, I say.\\
(2) Because he has understood that delight is the root of suffering,
and that with being [as condition] there is birth,
and that for whatever has come to be there is ageing and death.
Therefore, bhikkhus, through the complete destruction, fading away, cessation, giving up, and relinquishing of cravings,
the Tathāgata has awakened to supreme full enlightenment, I say.”\\
}
\hspace{0pt} \\ 
\end{leftcolumn*}

\begin{rightcolumn}
\PaliColumn{vehapphale vehapphalato abhijānāti;\\
vehapphale vehapphalato abhiññāya vehapphale na maññati,\\
vehapphalesu na maññati,
vehapphalato na maññati,\\
vehapphale meti na maññati,
vehapphale nābhinandati.\\
taṁ kissa hetu?
(1) ‘pariññātantaṁ tathāgatassā’ti vadāmi.\\
(2) ‘nandī dukkhassa mūlan’ti — iti viditvā ‘bhavā jāti bhūtassa jarāmaraṇan’ti. tasmātiha, bhikkhave, ‘tathāgato sabbaso taṇhānaṁ khayā virāgā nirodhā cāgā paṭinissaggā anuttaraṁ sammāsambodhiṁ abhisambuddho’ti vadāmi.\\
}
\hspace{0pt} \\ 
\end{rightcolumn}
\end{samepage}

\begin{samepage}
\ensurevspace{4\baselineskip}
\begin{leftcolumn*}
\EnglishColumn{"He too directly knows the Overlord as the Overlord.\\
Having directly known the Overlord as the Overlord,
he does not conceive [himself as] the Overlord,\\
he does not conceive [himself] in the Overlord,
he does not conceive [himself apart] from the Overlord,\\
he does not conceive the Overlord to be ‘mine,’
he does not delight in the Overlord.\\
Why is that?
(1) Because the Tathāgata has fully understood it to the end, I say.\\
(2) Because he has understood that delight is the root of suffering,
and that with being [as condition] there is birth,
and that for whatever has come to be there is ageing and death.
Therefore, bhikkhus, through the complete destruction, fading away, cessation, giving up, and relinquishing of cravings,
the Tathāgata has awakened to supreme full enlightenment, I say.”\\
}
\hspace{0pt} \\ 
\end{leftcolumn*}

\begin{rightcolumn}
\PaliColumn{abhibhuṁ abhibhūto abhijānāti;\\
abhibhuṁ abhibhūto abhiññāya abhibhuṁ na maññati,\\
abhibhusmiṁ na maññati,
abhibhūto na maññati,\\
abhibhuṁ meti na maññati,
abhibhuṁ nābhinandati.\\
taṁ kissa hetu?
(1) ‘pariññātantaṁ tathāgatassā’ti vadāmi.\\
(2) ‘nandī dukkhassa mūlan’ti — iti viditvā ‘bhavā jāti bhūtassa jarāmaraṇan’ti. tasmātiha, bhikkhave, ‘tathāgato sabbaso taṇhānaṁ khayā virāgā nirodhā cāgā paṭinissaggā anuttaraṁ sammāsambodhiṁ abhisambuddho’ti vadāmi.\\
}
\hspace{0pt} \\ 
\end{rightcolumn}
\end{samepage}

\begin{samepage}
\ensurevspace{4\baselineskip}
\begin{leftcolumn*}
\EnglishColumn{(US - unbound space)\\
"He too directly knows the base of US as the base of US.\\
Having directly known the base of US as the base of US,
he does not conceive [himself as] the base of US,\\
he does not conceive [himself] in the base of US,
he does not conceive [himself apart] from the base of US,\\
he does not conceive the base of US to be ‘mine,’
he does not delight in the base of US.\\
Why is that?
(1) Because the Tathāgata has fully understood it to the end, I say.\\
(2) Because he has understood that delight is the root of suffering,
and that with being [as condition] there is birth,
and that for whatever has come to be there is ageing and death.
Therefore, bhikkhus, through the complete destruction, fading away, cessation, giving up, and relinquishing of cravings,
the Tathāgata has awakened to supreme full enlightenment, I say.”\\
}
\hspace{0pt} \\ 
\end{leftcolumn*}

\begin{rightcolumn}
\PaliColumn{ākāsānañcāyatanaṁ ākāsānañcāyatanato abhijānāti;\\
ākāsānañcāyatanaṁ ākāsānañcāyatanato abhiññāya ākāsānañcāyatanaṁ na maññati,\\
ākāsānañcāyatanasmiṁ na maññati,
ākāsānañcāyatanato na maññati,\\
ākāsānañcāyatanaṁ meti na maññati,
ākāsānañcāyatanaṁ nābhinandati.\\
taṁ kissa hetu?
(1) ‘pariññātantaṁ tathāgatassā’ti vadāmi.\\
(2) ‘nandī dukkhassa mūlan’ti — iti viditvā ‘bhavā jāti bhūtassa jarāmaraṇan’ti. tasmātiha, bhikkhave, ‘tathāgato sabbaso taṇhānaṁ khayā virāgā nirodhā cāgā paṭinissaggā anuttaraṁ sammāsambodhiṁ abhisambuddho’ti vadāmi.\\
}
\hspace{0pt} \\ 
\end{rightcolumn}
\end{samepage}

\begin{samepage}
\ensurevspace{4\baselineskip}
\begin{leftcolumn*}
\EnglishColumn{(UC - unbound consciousness)\\
"He too directly knows the base of UC as the base of UC.\\
Having directly known the base of UC as the base of UC,
he does not conceive [himself as] the base of UC,\\
he does not conceive [himself] in the base of UC,
he does not conceive [himself apart] from the base of UC,\\
he does not conceive the base of UC to be ‘mine,’
he does not delight in the base of UC.\\
Why is that?
(1) Because the Tathāgata has fully understood it to the end, I say.\\
(2) Because he has understood that delight is the root of suffering,
and that with being [as condition] there is birth,
and that for whatever has come to be there is ageing and death.
Therefore, bhikkhus, through the complete destruction, fading away, cessation, giving up, and relinquishing of cravings,
the Tathāgata has awakened to supreme full enlightenment, I say.”\\
}
\hspace{0pt} \\ 
\end{leftcolumn*}

\begin{rightcolumn}
\PaliColumn{viññāṇañcāyatanaṁ viññāṇañcāyatanato abhijānāti;\\
viññāṇañcāyatanaṁ viññāṇañcāyatanato abhiññāya viññāṇañcāyatanaṁ na maññati,\\
viññāṇañcāyatanasmiṁ na maññati,
viññāṇañcāyatanato na maññati,\\
viññāṇañcāyatanaṁ meti na maññati,
viññāṇañcāyatanaṁ nābhinandati.\\
taṁ kissa hetu?
(1) ‘pariññātantaṁ tathāgatassā’ti vadāmi.\\
(2) ‘nandī dukkhassa mūlan’ti — iti viditvā ‘bhavā jāti bhūtassa jarāmaraṇan’ti. tasmātiha, bhikkhave, ‘tathāgato sabbaso taṇhānaṁ khayā virāgā nirodhā cāgā paṭinissaggā anuttaraṁ sammāsambodhiṁ abhisambuddho’ti vadāmi.\\
}
\hspace{0pt} \\ 
\end{rightcolumn}
\end{samepage}

\begin{samepage}
\ensurevspace{4\baselineskip}
\begin{leftcolumn*}
\EnglishColumn{(NT - no-thingness)\\
"He too directly knows the base of NT as the base of NT.\\
Having directly known the base of NT as the base of NT,
he does not conceive [himself as] the base of NT,\\
he does not conceive [himself] in the base of NT,
he does not conceive [himself apart] from the base of NT,\\
he does not conceive the base of NT to be ‘mine,’
he does not delight in the base of NT.\\
Why is that?
(1) Because the Tathāgata has fully understood it to the end, I say.\\
(2) Because he has understood that delight is the root of suffering,
and that with being [as condition] there is birth,
and that for whatever has come to be there is ageing and death.
Therefore, bhikkhus, through the complete destruction, fading away, cessation, giving up, and relinquishing of cravings,
the Tathāgata has awakened to supreme full enlightenment, I say.”\\
}
\hspace{0pt} \\ 
\end{leftcolumn*}

\begin{rightcolumn}
\PaliColumn{ākiñcaññāyatanaṁ ākiñcaññāyatanato abhijānāti;\\
ākiñcaññāyatanaṁ ākiñcaññāyatanato abhiññāya ākiñcaññāyatanaṁ na maññati,\\
ākiñcaññāyatanasmiṁ na maññati,
ākiñcaññāyatanato na maññati,\\
ākiñcaññāyatanaṁ meti na maññati,
ākiñcaññāyatanaṁ nābhinandati.\\
taṁ kissa hetu?
(1) ‘pariññātantaṁ tathāgatassā’ti vadāmi.\\
(2) ‘nandī dukkhassa mūlan’ti — iti viditvā ‘bhavā jāti bhūtassa jarāmaraṇan’ti. tasmātiha, bhikkhave, ‘tathāgato sabbaso taṇhānaṁ khayā virāgā nirodhā cāgā paṭinissaggā anuttaraṁ sammāsambodhiṁ abhisambuddho’ti vadāmi.\\
}
\hspace{0pt} \\ 
\end{rightcolumn}
\end{samepage}

\begin{samepage}
\ensurevspace{4\baselineskip}
\begin{leftcolumn*}
\EnglishColumn{(NPnNP - neither-perception-nor-non-perception)\\
"He too directly knows the base of NPnNP as the base of NPnNP.\\
Having directly known the base of NPnNP as the base of NPnNP,
he does not conceive [himself as] the base of NPnNP,\\
he does not conceive [himself] in the base of NPnNP,
he does not conceive [himself apart] from the base of NPnNP,\\
he does not conceive the base of NPnNP to be ‘mine,’
he does not delight in the base of NPnNP.\\
Why is that?
(1) Because the Tathāgata has fully understood it to the end, I say.\\
(2) Because he has understood that delight is the root of suffering,
and that with being [as condition] there is birth,
and that for whatever has come to be there is ageing and death.
Therefore, bhikkhus, through the complete destruction, fading away, cessation, giving up, and relinquishing of cravings,
the Tathāgata has awakened to supreme full enlightenment, I say.”\\
}
\hspace{0pt} \\ 
\end{leftcolumn*}

\begin{rightcolumn}
\PaliColumn{nevasaññānāsaññāyatanaṁ nevasaññānāsaññāyatanato abhijānāti;\\
nevasaññānāsaññāyatanaṁ nevasaññānāsaññāyatanato abhiññāya nevasaññānāsaññāyatanaṁ na maññati,\\
nevasaññānāsaññāyatanasmiṁ na maññati,
nevasaññānāsaññāyatanato na maññati,\\
nevasaññānāsaññāyatanaṁ meti na maññati,
nevasaññānāsaññāyatanaṁ nābhinandati.\\
taṁ kissa hetu?
(1) ‘pariññātantaṁ tathāgatassā’ti vadāmi.\\
(2) ‘nandī dukkhassa mūlan’ti — iti viditvā ‘bhavā jāti bhūtassa jarāmaraṇan’ti. tasmātiha, bhikkhave, ‘tathāgato sabbaso taṇhānaṁ khayā virāgā nirodhā cāgā paṭinissaggā anuttaraṁ sammāsambodhiṁ abhisambuddho’ti vadāmi.\\
}
\hspace{0pt} \\ 
\end{rightcolumn}
\end{samepage}

\begin{samepage}
\ensurevspace{4\baselineskip}
\begin{leftcolumn*}
\EnglishColumn{"He too directly knows the seen as the seen.\\
Having directly known the seen as the seen,
he does not conceive [himself as] the seen,\\
he does not conceive [himself] in the seen,
he does not conceive [himself apart] from the seen,\\
he does not conceive the seen to be ‘mine,’
he does not delight in the seen.\\
Why is that?
(1) Because the Tathāgata has fully understood it to the end, I say.\\
(2) Because he has understood that delight is the root of suffering,
and that with being [as condition] there is birth,
and that for whatever has come to be there is ageing and death.
Therefore, bhikkhus, through the complete destruction, fading away, cessation, giving up, and relinquishing of cravings,
the Tathāgata has awakened to supreme full enlightenment, I say.”\\
}
\hspace{0pt} \\ 
\end{leftcolumn*}

\begin{rightcolumn}
\PaliColumn{diṭṭhaṁ diṭṭhato abhijānāti;\\
diṭṭhaṁ diṭṭhato abhiññāya diṭṭhaṁ na maññati,\\
diṭṭhasmiṁ na maññati,
diṭṭhato na maññati,\\
diṭṭhaṁ meti na maññati,
diṭṭhaṁ nābhinandati.\\
taṁ kissa hetu?
(1) ‘pariññātantaṁ tathāgatassā’ti vadāmi.\\
(2) ‘nandī dukkhassa mūlan’ti — iti viditvā ‘bhavā jāti bhūtassa jarāmaraṇan’ti. tasmātiha, bhikkhave, ‘tathāgato sabbaso taṇhānaṁ khayā virāgā nirodhā cāgā paṭinissaggā anuttaraṁ sammāsambodhiṁ abhisambuddho’ti vadāmi.\\
}
\hspace{0pt} \\ 
\end{rightcolumn}
\end{samepage}

\begin{samepage}
\ensurevspace{4\baselineskip}
\begin{leftcolumn*}
\EnglishColumn{"He too directly knows the heard as the heard.\\
Having directly known the heard as the heard,
he does not conceive [himself as] the heard,\\
he does not conceive [himself] in the heard,
he does not conceive [himself apart] from the heard,\\
he does not conceive the heard to be ‘mine,’
he does not delight in the heard.\\
Why is that?
(1) Because the Tathāgata has fully understood it to the end, I say.\\
(2) Because he has understood that delight is the root of suffering,
and that with being [as condition] there is birth,
and that for whatever has come to be there is ageing and death.
Therefore, bhikkhus, through the complete destruction, fading away, cessation, giving up, and relinquishing of cravings,
the Tathāgata has awakened to supreme full enlightenment, I say.”\\
}
\hspace{0pt} \\ 
\end{leftcolumn*}

\begin{rightcolumn}
\PaliColumn{sutaṁ sutato abhijānāti;\\
sutaṁ sutato abhiññāya sutaṁ na maññati,\\
sutasmiṁ na maññati,
sutato na maññati,\\
sutaṁ meti na maññati,
sutaṁ nābhinandati.\\
taṁ kissa hetu?
(1) ‘pariññātantaṁ tathāgatassā’ti vadāmi.\\
(2) ‘nandī dukkhassa mūlan’ti — iti viditvā ‘bhavā jāti bhūtassa jarāmaraṇan’ti. tasmātiha, bhikkhave, ‘tathāgato sabbaso taṇhānaṁ khayā virāgā nirodhā cāgā paṭinissaggā anuttaraṁ sammāsambodhiṁ abhisambuddho’ti vadāmi.\\
}
\hspace{0pt} \\ 
\end{rightcolumn}
\end{samepage}

\begin{samepage}
\ensurevspace{4\baselineskip}
\begin{leftcolumn*}
\EnglishColumn{"He too directly knows the sensed as the sensed.\\
Having directly known the sensed as the sensed,
he does not conceive [himself as] the sensed,\\
he does not conceive [himself] in the sensed,
he does not conceive [himself apart] from the sensed,\\
he does not conceive the sensed to be ‘mine,’
he does not delight in the sensed.\\
Why is that?
(1) Because the Tathāgata has fully understood it to the end, I say.\\
(2) Because he has understood that delight is the root of suffering,
and that with being [as condition] there is birth,
and that for whatever has come to be there is ageing and death.
Therefore, bhikkhus, through the complete destruction, fading away, cessation, giving up, and relinquishing of cravings,
the Tathāgata has awakened to supreme full enlightenment, I say.”\\
}
\hspace{0pt} \\ 
\end{leftcolumn*}

\begin{rightcolumn}
\PaliColumn{mutaṁ mutato abhijānāti;\\
mutaṁ mutato abhiññāya mutaṁ na maññati,\\
mutasmiṁ na maññati,
mutato na maññati,\\
mutaṁ meti na maññati,
mutaṁ nābhinandati.\\
taṁ kissa hetu?
(1) ‘pariññātantaṁ tathāgatassā’ti vadāmi.\\
(2) ‘nandī dukkhassa mūlan’ti — iti viditvā ‘bhavā jāti bhūtassa jarāmaraṇan’ti. tasmātiha, bhikkhave, ‘tathāgato sabbaso taṇhānaṁ khayā virāgā nirodhā cāgā paṭinissaggā anuttaraṁ sammāsambodhiṁ abhisambuddho’ti vadāmi.\\
}
\hspace{0pt} \\ 
\end{rightcolumn}
\end{samepage}

\begin{samepage}
\ensurevspace{4\baselineskip}
\begin{leftcolumn*}
\EnglishColumn{"He too directly knows the cognized as the cognized.\\
Having directly known the cognized as the cognized,
he does not conceive [himself as] the cognized,\\
he does not conceive [himself] in the cognized,
he does not conceive [himself apart] from the cognized,\\
he does not conceive the cognized to be ‘mine,’
he does not delight in the cognized.\\
Why is that?
(1) Because the Tathāgata has fully understood it to the end, I say.\\
(2) Because he has understood that delight is the root of suffering,
and that with being [as condition] there is birth,
and that for whatever has come to be there is ageing and death.
Therefore, bhikkhus, through the complete destruction, fading away, cessation, giving up, and relinquishing of cravings,
the Tathāgata has awakened to supreme full enlightenment, I say.”\\
}
\hspace{0pt} \\ 
\end{leftcolumn*}

\begin{rightcolumn}
\PaliColumn{viññātaṁ viññātato abhijānāti;\\
viññātaṁ viññātato abhiññāya viññātaṁ na maññati,\\
viññātasmiṁ na maññati,
viññātato na maññati,\\
viññātaṁ meti na maññati,
viññātaṁ nābhinandati.\\
taṁ kissa hetu?
(1) ‘pariññātantaṁ tathāgatassā’ti vadāmi.\\
(2) ‘nandī dukkhassa mūlan’ti — iti viditvā ‘bhavā jāti bhūtassa jarāmaraṇan’ti. tasmātiha, bhikkhave, ‘tathāgato sabbaso taṇhānaṁ khayā virāgā nirodhā cāgā paṭinissaggā anuttaraṁ sammāsambodhiṁ abhisambuddho’ti vadāmi.\\
}
\hspace{0pt} \\ 
\end{rightcolumn}
\end{samepage}

\begin{samepage}
\ensurevspace{4\baselineskip}
\begin{leftcolumn*}
\EnglishColumn{"He too directly knows unity as unity.\\
Having directly known unity as unity,
he does not conceive [himself as] unity,\\
he does not conceive [himself] in unity,
he does not conceive [himself apart] from unity,\\
he does not conceive unity to be ‘mine,’
he does not delight in unity.\\
Why is that?
(1) Because the Tathāgata has fully understood it to the end, I say.\\
(2) Because he has understood that delight is the root of suffering,
and that with being [as condition] there is birth,
and that for whatever has come to be there is ageing and death.
Therefore, bhikkhus, through the complete destruction, fading away, cessation, giving up, and relinquishing of cravings,
the Tathāgata has awakened to supreme full enlightenment, I say.”\\
}
\hspace{0pt} \\ 
\end{leftcolumn*}

\begin{rightcolumn}
\PaliColumn{ekattaṁ ekattato abhijānāti;\\
ekattaṁ ekattato abhiññāya ekattaṁ na maññati,\\
ekattasmiṁ na maññati,
ekattato na maññati,\\
ekattaṁ meti na maññati,
ekattaṁ nābhinandati.\\
taṁ kissa hetu?
(1) ‘pariññātantaṁ tathāgatassā’ti vadāmi.\\
(2) ‘nandī dukkhassa mūlan’ti — iti viditvā ‘bhavā jāti bhūtassa jarāmaraṇan’ti. tasmātiha, bhikkhave, ‘tathāgato sabbaso taṇhānaṁ khayā virāgā nirodhā cāgā paṭinissaggā anuttaraṁ sammāsambodhiṁ abhisambuddho’ti vadāmi.\\
}
\hspace{0pt} \\ 
\end{rightcolumn}
\end{samepage}

\begin{samepage}
\ensurevspace{4\baselineskip}
\begin{leftcolumn*}
\EnglishColumn{"He too directly knows diversity as diversity.\\
Having directly known diversity as diversity,
he does not conceive [himself as] diversity,\\
he does not conceive [himself] in diversity,
he does not conceive [himself apart] from diversity,\\
he does not conceive diversity to be ‘mine,’
he does not delight in diversity.\\
Why is that?
(1) Because the Tathāgata has fully understood it to the end, I say.\\
(2) Because he has understood that delight is the root of suffering,
and that with being [as condition] there is birth,
and that for whatever has come to be there is ageing and death.
Therefore, bhikkhus, through the complete destruction, fading away, cessation, giving up, and relinquishing of cravings,
the Tathāgata has awakened to supreme full enlightenment, I say.”\\
}
\hspace{0pt} \\ 
\end{leftcolumn*}

\begin{rightcolumn}
\PaliColumn{nānattaṁ nānattato abhijānāti;\\
nānattaṁ nānattato abhiññāya nānattaṁ na maññati,\\
nānattasmiṁ na maññati,
nānattato na maññati,\\
nānattaṁ meti na maññati,
vānattaṁ nābhinandati.\\
taṁ kissa hetu?
(1) ‘pariññātantaṁ tathāgatassā’ti vadāmi.\\
(2) ‘nandī dukkhassa mūlan’ti — iti viditvā ‘bhavā jāti bhūtassa jarāmaraṇan’ti. tasmātiha, bhikkhave, ‘tathāgato sabbaso taṇhānaṁ khayā virāgā nirodhā cāgā paṭinissaggā anuttaraṁ sammāsambodhiṁ abhisambuddho’ti vadāmi.\\
}
\hspace{0pt} \\ 
\end{rightcolumn}
\end{samepage}

\begin{samepage}
\ensurevspace{4\baselineskip}
\begin{leftcolumn*}
\EnglishColumn{"He too directly knows all as all.\\
Having directly known all as all,
he does not conceive [himself as] all,\\
he does not conceive [himself] in all,
he does not conceive [himself apart] from all,\\
he does not conceive all to be ‘mine,’
he does not delight in all.\\
Why is that?
(1) Because the Tathāgata has fully understood it to the end, I say.\\
(2) Because he has understood that delight is the root of suffering,
and that with being [as condition] there is birth,
and that for whatever has come to be there is ageing and death.
Therefore, bhikkhus, through the complete destruction, fading away, cessation, giving up, and relinquishing of cravings,
the Tathāgata has awakened to supreme full enlightenment, I say.”\\
}
\hspace{0pt} \\ 
\end{leftcolumn*}

\begin{rightcolumn}
\PaliColumn{sabbaṁ sabbato abhijānāti;\\
sabbaṁ sabbato abhiññāya sabbaṁ na maññati,\\
sabbasmiṁ na maññati,
sabbato na maññati,\\
sabbaṁ meti na maññati,
sabbaṁ nābhinandati.\\
taṁ kissa hetu?
(1) ‘pariññātantaṁ tathāgatassā’ti vadāmi.\\
(2) ‘nandī dukkhassa mūlan’ti — iti viditvā ‘bhavā jāti bhūtassa jarāmaraṇan’ti. tasmātiha, bhikkhave, ‘tathāgato sabbaso taṇhānaṁ khayā virāgā nirodhā cāgā paṭinissaggā anuttaraṁ sammāsambodhiṁ abhisambuddho’ti vadāmi.\\
}
\hspace{0pt} \\ 
\end{rightcolumn}
\end{samepage}

\begin{samepage}
\ensurevspace{4\baselineskip}
\begin{leftcolumn*}
\EnglishColumn{"He too directly knows Nibbāna as Nibbāna.\\
Having directly known Nibbāna as Nibbāna,
he does not conceive [himself as] Nibbāna,\\
he does not conceive [himself] in Nibbāna,
he does not conceive [himself apart] from Nibbāna,\\
he does not conceive Nibbāna to be ‘mine,’
he does not delight in Nibbāna.\\
Why is that?
(1) Because the Tathāgata has fully understood it to the end, I say.\\
(2) Because he has understood that delight is the root of suffering,
and that with being [as condition] there is birth,
and that for whatever has come to be there is ageing and death.
Therefore, bhikkhus, through the complete destruction, fading away, cessation, giving up, and relinquishing of cravings,
the Tathāgata has awakened to supreme full enlightenment, I say.”\\
}
\hspace{0pt} \\ 
\end{leftcolumn*}

\begin{rightcolumn}
\PaliColumn{nibbānaṁ nibbānato abhijānāti;\\
nibbānaṁ nibbānato abhiññāya nibbānaṁ na maññati,\\
nibbānasmiṁ na maññati,
nibbānato na maññati,\\
nibbānaṁ meti na maññati,
nibbānaṁ nābhinandati.\\
taṁ kissa hetu?
(1) ‘pariññātantaṁ tathāgatassā’ti vadāmi.\\
(2) ‘nandī dukkhassa mūlan’ti — iti viditvā ‘bhavā jāti bhūtassa jarāmaraṇan’ti. tasmātiha, bhikkhave, ‘tathāgato sabbaso taṇhānaṁ khayā virāgā nirodhā cāgā paṭinissaggā anuttaraṁ sammāsambodhiṁ abhisambuddho’ti vadāmi.\\
}
\hspace{0pt} \\ 
\end{rightcolumn}
\end{samepage}

\begin{samepage}
\ensurevspace{4\baselineskip}
\begin{leftcolumn*}
\EnglishColumn{That is what the Blessed One said. But those bhikkhus did not delight in the Blessed One’s words.\\
}
\hspace{0pt} \\ 
\end{leftcolumn*}

\begin{rightcolumn}
\PaliColumn{idamavoca bhagavā. na te bhikkhū bhagavato bhāsitaṁ abhinandunti.\\
}
\hspace{0pt} \\ 
\end{rightcolumn}
\end{samepage}

\begin{samepage}
\ensurevspace{4\baselineskip}
\begin{leftcolumn*}
\EnglishColumn{
}
\hspace{0pt} \\ 
\end{leftcolumn*}

\begin{rightcolumn}
\PaliColumn{mūlapariyāyasuttaṁ niṭṭhitaṁ paṭhamaṁ.
}
\hspace{0pt} \\ 
\end{rightcolumn}
\end{samepage}

