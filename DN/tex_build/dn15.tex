
\begin{absolutelynopagebreak}
\setstretch{.7}
{\PaliGlossA{dīgha nikāya 15}}\\
\begin{addmargin}[1em]{2em}
\setstretch{.5}
{\PaliGlossB{Long Discourses 15}}\\
\end{addmargin}
\end{absolutelynopagebreak}

\begin{absolutelynopagebreak}
\setstretch{.7}
{\PaliGlossA{mahānidānasutta}}\\
\begin{addmargin}[1em]{2em}
\setstretch{.5}
{\PaliGlossB{The Great Discourse on Causation}}\\
\end{addmargin}
\end{absolutelynopagebreak}

\begin{absolutelynopagebreak}
\setstretch{.7}
{\PaliGlossA{1. paṭiccasamuppāda}}\\
\begin{addmargin}[1em]{2em}
\setstretch{.5}
{\PaliGlossB{1. Dependent Origination}}\\
\end{addmargin}
\end{absolutelynopagebreak}

\begin{absolutelynopagebreak}
\setstretch{.7}
{\PaliGlossA{evaṃ me sutaṃ—}}\\
\begin{addmargin}[1em]{2em}
\setstretch{.5}
{\PaliGlossB{So I have heard.}}\\
\end{addmargin}
\end{absolutelynopagebreak}

\begin{absolutelynopagebreak}
\setstretch{.7}
{\PaliGlossA{ekaṃ samayaṃ bhagavā kurūsu viharati kammāsadhammaṃ nāma kurūnaṃ nigamo.}}\\
\begin{addmargin}[1em]{2em}
\setstretch{.5}
{\PaliGlossB{At one time the Buddha was staying in the land of the Kurus, near the Kuru town named Kammāsadamma.}}\\
\end{addmargin}
\end{absolutelynopagebreak}

\begin{absolutelynopagebreak}
\setstretch{.7}
{\PaliGlossA{atha kho āyasmā ānando yena bhagavā tenupasaṅkami, upasaṅkamitvā bhagavantaṃ abhivādetvā ekamantaṃ nisīdi. ekamantaṃ nisinno kho āyasmā ānando bhagavantaṃ etadavoca:}}\\
\begin{addmargin}[1em]{2em}
\setstretch{.5}
{\PaliGlossB{Then Venerable Ānanda went up to the Buddha, bowed, sat down to one side, and said to him,}}\\
\end{addmargin}
\end{absolutelynopagebreak}

\begin{absolutelynopagebreak}
\setstretch{.7}
{\PaliGlossA{“acchariyaṃ, bhante, abbhutaṃ, bhante.}}\\
\begin{addmargin}[1em]{2em}
\setstretch{.5}
{\PaliGlossB{“It’s incredible, sir, it’s amazing,}}\\
\end{addmargin}
\end{absolutelynopagebreak}

\begin{absolutelynopagebreak}
\setstretch{.7}
{\PaliGlossA{yāva gambhīro cāyaṃ, bhante, paṭiccasamuppādo gambhīrāvabhāso ca, atha ca pana me uttānakuttānako viya khāyatī”ti.}}\\
\begin{addmargin}[1em]{2em}
\setstretch{.5}
{\PaliGlossB{in that this dependent origination is deep and appears deep, yet to me it seems as plain as can be.”}}\\
\end{addmargin}
\end{absolutelynopagebreak}

\begin{absolutelynopagebreak}
\setstretch{.7}
{\PaliGlossA{“mā hevaṃ, ānanda, avaca, mā hevaṃ, ānanda, avaca.}}\\
\begin{addmargin}[1em]{2em}
\setstretch{.5}
{\PaliGlossB{“Don’t say that, Ānanda, don’t say that!}}\\
\end{addmargin}
\end{absolutelynopagebreak}

\begin{absolutelynopagebreak}
\setstretch{.7}
{\PaliGlossA{gambhīro cāyaṃ, ānanda, paṭiccasamuppādo gambhīrāvabhāso ca.}}\\
\begin{addmargin}[1em]{2em}
\setstretch{.5}
{\PaliGlossB{This dependent origination is deep and appears deep.}}\\
\end{addmargin}
\end{absolutelynopagebreak}

\begin{absolutelynopagebreak}
\setstretch{.7}
{\PaliGlossA{etassa, ānanda, dhammassa ananubodhā appaṭivedhā evamayaṃ pajā tantākulakajātā kulagaṇṭhikajātā muñjapabbajabhūtā apāyaṃ duggatiṃ vinipātaṃ saṃsāraṃ nātivattati.}}\\
\begin{addmargin}[1em]{2em}
\setstretch{.5}
{\PaliGlossB{It is because of not understanding and not penetrating this teaching that this population has become tangled like string, knotted like a ball of thread, and matted like rushes and reeds, and it doesn’t escape the places of loss, the bad places, the underworld, transmigration.}}\\
\end{addmargin}
\end{absolutelynopagebreak}

\begin{absolutelynopagebreak}
\setstretch{.7}
{\PaliGlossA{‘atthi idappaccayā jarāmaraṇan’ti iti puṭṭhena satā, ānanda, atthītissa vacanīyaṃ.}}\\
\begin{addmargin}[1em]{2em}
\setstretch{.5}
{\PaliGlossB{When asked, ‘Is there a specific condition for old age and death?’ you should answer, ‘There is.’}}\\
\end{addmargin}
\end{absolutelynopagebreak}

\begin{absolutelynopagebreak}
\setstretch{.7}
{\PaliGlossA{‘kiṃpaccayā jarāmaraṇan’ti iti ce vadeyya, ‘jātipaccayā jarāmaraṇan’ti iccassa vacanīyaṃ.}}\\
\begin{addmargin}[1em]{2em}
\setstretch{.5}
{\PaliGlossB{If they say, ‘What is a condition for old age and death?’ you should answer, ‘Rebirth is a condition for old age and death.’}}\\
\end{addmargin}
\end{absolutelynopagebreak}

\begin{absolutelynopagebreak}
\setstretch{.7}
{\PaliGlossA{‘atthi idappaccayā jātī’ti iti puṭṭhena satā, ānanda, atthītissa vacanīyaṃ.}}\\
\begin{addmargin}[1em]{2em}
\setstretch{.5}
{\PaliGlossB{When asked, ‘Is there a specific condition for rebirth?’ you should answer, ‘There is.’}}\\
\end{addmargin}
\end{absolutelynopagebreak}

\begin{absolutelynopagebreak}
\setstretch{.7}
{\PaliGlossA{‘kiṃpaccayā jātī’ti iti ce vadeyya, ‘bhavapaccayā jātī’ti iccassa vacanīyaṃ.}}\\
\begin{addmargin}[1em]{2em}
\setstretch{.5}
{\PaliGlossB{If they say, ‘What is a condition for rebirth?’ you should answer, ‘Continued existence is a condition for rebirth.’}}\\
\end{addmargin}
\end{absolutelynopagebreak}

\begin{absolutelynopagebreak}
\setstretch{.7}
{\PaliGlossA{‘atthi idappaccayā bhavo’ti iti puṭṭhena satā, ānanda, atthītissa vacanīyaṃ.}}\\
\begin{addmargin}[1em]{2em}
\setstretch{.5}
{\PaliGlossB{When asked, ‘Is there a specific condition for continued existence?’ you should answer, ‘There is.’}}\\
\end{addmargin}
\end{absolutelynopagebreak}

\begin{absolutelynopagebreak}
\setstretch{.7}
{\PaliGlossA{‘kiṃpaccayā bhavo’ti iti ce vadeyya, ‘upādānapaccayā bhavo’ti iccassa vacanīyaṃ.}}\\
\begin{addmargin}[1em]{2em}
\setstretch{.5}
{\PaliGlossB{If they say, ‘What is a condition for continued existence?’ you should answer, ‘Grasping is a condition for continued existence.’}}\\
\end{addmargin}
\end{absolutelynopagebreak}

\begin{absolutelynopagebreak}
\setstretch{.7}
{\PaliGlossA{‘atthi idappaccayā upādānan’ti iti puṭṭhena satā, ānanda, atthītissa vacanīyaṃ.}}\\
\begin{addmargin}[1em]{2em}
\setstretch{.5}
{\PaliGlossB{When asked, ‘Is there a specific condition for grasping?’ you should answer, ‘There is.’}}\\
\end{addmargin}
\end{absolutelynopagebreak}

\begin{absolutelynopagebreak}
\setstretch{.7}
{\PaliGlossA{‘kiṃpaccayā upādānan’ti iti ce vadeyya, ‘taṇhāpaccayā upādānan’ti iccassa vacanīyaṃ.}}\\
\begin{addmargin}[1em]{2em}
\setstretch{.5}
{\PaliGlossB{If they say, ‘What is a condition for grasping?’ you should answer, ‘Craving is a condition for grasping.’}}\\
\end{addmargin}
\end{absolutelynopagebreak}

\begin{absolutelynopagebreak}
\setstretch{.7}
{\PaliGlossA{‘atthi idappaccayā taṇhā’ti iti puṭṭhena satā, ānanda, atthītissa vacanīyaṃ.}}\\
\begin{addmargin}[1em]{2em}
\setstretch{.5}
{\PaliGlossB{When asked, ‘Is there a specific condition for craving?’ you should answer, ‘There is.’}}\\
\end{addmargin}
\end{absolutelynopagebreak}

\begin{absolutelynopagebreak}
\setstretch{.7}
{\PaliGlossA{‘kiṃpaccayā taṇhā’ti iti ce vadeyya, ‘vedanāpaccayā taṇhā’ti iccassa vacanīyaṃ.}}\\
\begin{addmargin}[1em]{2em}
\setstretch{.5}
{\PaliGlossB{If they say, ‘What is a condition for craving?’ you should answer, ‘Feeling is a condition for craving.’}}\\
\end{addmargin}
\end{absolutelynopagebreak}

\begin{absolutelynopagebreak}
\setstretch{.7}
{\PaliGlossA{‘atthi idappaccayā vedanā’ti iti puṭṭhena satā, ānanda, atthītissa vacanīyaṃ.}}\\
\begin{addmargin}[1em]{2em}
\setstretch{.5}
{\PaliGlossB{When asked, ‘Is there a specific condition for feeling?’ you should answer, ‘There is.’}}\\
\end{addmargin}
\end{absolutelynopagebreak}

\begin{absolutelynopagebreak}
\setstretch{.7}
{\PaliGlossA{‘kiṃpaccayā vedanā’ti iti ce vadeyya, ‘phassapaccayā vedanā’ti iccassa vacanīyaṃ.}}\\
\begin{addmargin}[1em]{2em}
\setstretch{.5}
{\PaliGlossB{If they say, ‘What is a condition for feeling?’ you should answer, ‘Contact is a condition for feeling.’}}\\
\end{addmargin}
\end{absolutelynopagebreak}

\begin{absolutelynopagebreak}
\setstretch{.7}
{\PaliGlossA{‘atthi idappaccayā phasso’ti iti puṭṭhena satā, ānanda, atthītissa vacanīyaṃ.}}\\
\begin{addmargin}[1em]{2em}
\setstretch{.5}
{\PaliGlossB{When asked, ‘Is there a specific condition for contact?’ you should answer, ‘There is.’}}\\
\end{addmargin}
\end{absolutelynopagebreak}

\begin{absolutelynopagebreak}
\setstretch{.7}
{\PaliGlossA{‘kiṃpaccayā phasso’ti iti ce vadeyya, ‘nāmarūpapaccayā phasso’ti iccassa vacanīyaṃ.}}\\
\begin{addmargin}[1em]{2em}
\setstretch{.5}
{\PaliGlossB{If they say, ‘What is a condition for contact?’ you should answer, ‘Name and form are conditions for contact.’}}\\
\end{addmargin}
\end{absolutelynopagebreak}

\begin{absolutelynopagebreak}
\setstretch{.7}
{\PaliGlossA{‘atthi idappaccayā nāmarūpan’ti iti puṭṭhena satā, ānanda, atthītissa vacanīyaṃ.}}\\
\begin{addmargin}[1em]{2em}
\setstretch{.5}
{\PaliGlossB{When asked, ‘Is there a specific condition for name and form?’ you should answer, ‘There is.’}}\\
\end{addmargin}
\end{absolutelynopagebreak}

\begin{absolutelynopagebreak}
\setstretch{.7}
{\PaliGlossA{‘kiṃpaccayā nāmarūpan’ti iti ce vadeyya, ‘viññāṇapaccayā nāmarūpan’ti iccassa vacanīyaṃ.}}\\
\begin{addmargin}[1em]{2em}
\setstretch{.5}
{\PaliGlossB{If they say, ‘What is a condition for name and form?’ you should answer, ‘Consciousness is a condition for name and form.’}}\\
\end{addmargin}
\end{absolutelynopagebreak}

\begin{absolutelynopagebreak}
\setstretch{.7}
{\PaliGlossA{‘atthi idappaccayā viññāṇan’ti iti puṭṭhena satā, ānanda, atthītissa vacanīyaṃ.}}\\
\begin{addmargin}[1em]{2em}
\setstretch{.5}
{\PaliGlossB{When asked, ‘Is there a specific condition for consciousness?’ you should answer, ‘There is.’}}\\
\end{addmargin}
\end{absolutelynopagebreak}

\begin{absolutelynopagebreak}
\setstretch{.7}
{\PaliGlossA{‘kiṃpaccayā viññāṇan’ti iti ce vadeyya, ‘nāmarūpapaccayā viññāṇan’ti iccassa vacanīyaṃ.}}\\
\begin{addmargin}[1em]{2em}
\setstretch{.5}
{\PaliGlossB{If they say, ‘What is a condition for consciousness?’ you should answer, ‘Name and form are conditions for consciousness.’}}\\
\end{addmargin}
\end{absolutelynopagebreak}

\begin{absolutelynopagebreak}
\setstretch{.7}
{\PaliGlossA{iti kho, ānanda, nāmarūpapaccayā viññāṇaṃ, viññāṇapaccayā nāmarūpaṃ, nāmarūpapaccayā phasso, phassapaccayā vedanā, vedanāpaccayā taṇhā, taṇhāpaccayā upādānaṃ, upādānapaccayā bhavo, bhavapaccayā jāti, jātipaccayā jarāmaraṇaṃ sokaparidevadukkhadomanassupāyāsā sambhavanti.}}\\
\begin{addmargin}[1em]{2em}
\setstretch{.5}
{\PaliGlossB{So: name and form are conditions for consciousness. Consciousness is a condition for name and form. Name and form are conditions for contact. Contact is a condition for feeling. Feeling is a condition for craving. Craving is a condition for grasping. Grasping is a condition for continued existence. Continued existence is a condition for rebirth. Rebirth is a condition for old age and death, sorrow, lamentation, pain, sadness, and distress to come to be.}}\\
\end{addmargin}
\end{absolutelynopagebreak}

\begin{absolutelynopagebreak}
\setstretch{.7}
{\PaliGlossA{evametassa kevalassa dukkhakkhandhassa samudayo hoti.}}\\
\begin{addmargin}[1em]{2em}
\setstretch{.5}
{\PaliGlossB{That is how this entire mass of suffering originates.}}\\
\end{addmargin}
\end{absolutelynopagebreak}

\begin{absolutelynopagebreak}
\setstretch{.7}
{\PaliGlossA{‘jātipaccayā jarāmaraṇan’ti iti kho panetaṃ vuttaṃ, tadānanda, imināpetaṃ pariyāyena veditabbaṃ, yathā jātipaccayā jarāmaraṇaṃ.}}\\
\begin{addmargin}[1em]{2em}
\setstretch{.5}
{\PaliGlossB{‘Rebirth is a condition for old age and death’—that’s what I said. And this is a way to understand how this is so.}}\\
\end{addmargin}
\end{absolutelynopagebreak}

\begin{absolutelynopagebreak}
\setstretch{.7}
{\PaliGlossA{jāti ca hi, ānanda, nābhavissa sabbena sabbaṃ sabbathā sabbaṃ kassaci kimhici, seyyathidaṃ—}}\\
\begin{addmargin}[1em]{2em}
\setstretch{.5}
{\PaliGlossB{Suppose there were totally and utterly no rebirth for anyone anywhere.}}\\
\end{addmargin}
\end{absolutelynopagebreak}

\begin{absolutelynopagebreak}
\setstretch{.7}
{\PaliGlossA{devānaṃ vā devattāya, gandhabbānaṃ vā gandhabbattāya, yakkhānaṃ vā yakkhattāya, bhūtānaṃ vā bhūtattāya, manussānaṃ vā manussattāya, catuppadānaṃ vā catuppadattāya, pakkhīnaṃ vā pakkhittāya, sarīsapānaṃ vā sarīsapattāya, tesaṃ tesañca hi, ānanda, sattānaṃ tadattāya jāti nābhavissa. sabbaso jātiyā asati jātinirodhā api nu kho jarāmaraṇaṃ paññāyethā”ti?}}\\
\begin{addmargin}[1em]{2em}
\setstretch{.5}
{\PaliGlossB{That is, there were no rebirth of sentient beings into their various realms—of gods, fairies, spirits, creatures, humans, quadrupeds, birds, or reptiles, each into their own realm. When there’s no rebirth at all, with the cessation of rebirth, would old age and death still be found?”}}\\
\end{addmargin}
\end{absolutelynopagebreak}

\begin{absolutelynopagebreak}
\setstretch{.7}
{\PaliGlossA{“no hetaṃ, bhante”.}}\\
\begin{addmargin}[1em]{2em}
\setstretch{.5}
{\PaliGlossB{“No, sir.”}}\\
\end{addmargin}
\end{absolutelynopagebreak}

\begin{absolutelynopagebreak}
\setstretch{.7}
{\PaliGlossA{“tasmātihānanda, eseva hetu etaṃ nidānaṃ esa samudayo esa paccayo jarāmaraṇassa, yadidaṃ jāti.}}\\
\begin{addmargin}[1em]{2em}
\setstretch{.5}
{\PaliGlossB{“That’s why this is the cause, source, origin, and condition of old age and death, namely rebirth.}}\\
\end{addmargin}
\end{absolutelynopagebreak}

\begin{absolutelynopagebreak}
\setstretch{.7}
{\PaliGlossA{‘bhavapaccayā jātī’ti iti kho panetaṃ vuttaṃ, tadānanda, imināpetaṃ pariyāyena veditabbaṃ, yathā bhavapaccayā jāti.}}\\
\begin{addmargin}[1em]{2em}
\setstretch{.5}
{\PaliGlossB{‘Continued existence is a condition for rebirth’—that’s what I said. And this is a way to understand how this is so.}}\\
\end{addmargin}
\end{absolutelynopagebreak}

\begin{absolutelynopagebreak}
\setstretch{.7}
{\PaliGlossA{bhavo ca hi, ānanda, nābhavissa sabbena sabbaṃ sabbathā sabbaṃ kassaci kimhici, seyyathidaṃ—}}\\
\begin{addmargin}[1em]{2em}
\setstretch{.5}
{\PaliGlossB{Suppose there were totally and utterly no continued existence for anyone anywhere.}}\\
\end{addmargin}
\end{absolutelynopagebreak}

\begin{absolutelynopagebreak}
\setstretch{.7}
{\PaliGlossA{kāmabhavo vā rūpabhavo vā arūpabhavo vā, sabbaso bhave asati bhavanirodhā api nu kho jāti paññāyethā”ti?}}\\
\begin{addmargin}[1em]{2em}
\setstretch{.5}
{\PaliGlossB{That is, continued existence in the sensual realm, the realm of luminous form, or the formless realm. When there’s no continued existence at all, with the cessation of continued existence, would rebirth still be found?”}}\\
\end{addmargin}
\end{absolutelynopagebreak}

\begin{absolutelynopagebreak}
\setstretch{.7}
{\PaliGlossA{“no hetaṃ, bhante”.}}\\
\begin{addmargin}[1em]{2em}
\setstretch{.5}
{\PaliGlossB{“No, sir.”}}\\
\end{addmargin}
\end{absolutelynopagebreak}

\begin{absolutelynopagebreak}
\setstretch{.7}
{\PaliGlossA{“tasmātihānanda, eseva hetu etaṃ nidānaṃ esa samudayo esa paccayo jātiyā, yadidaṃ bhavo.}}\\
\begin{addmargin}[1em]{2em}
\setstretch{.5}
{\PaliGlossB{“That’s why this is the cause, source, origin, and condition of rebirth, namely continued existence.}}\\
\end{addmargin}
\end{absolutelynopagebreak}

\begin{absolutelynopagebreak}
\setstretch{.7}
{\PaliGlossA{‘upādānapaccayā bhavo’ti iti kho panetaṃ vuttaṃ, tadānanda, imināpetaṃ pariyāyena veditabbaṃ, yathā upādānapaccayā bhavo.}}\\
\begin{addmargin}[1em]{2em}
\setstretch{.5}
{\PaliGlossB{‘Grasping is a condition for continued existence’—that’s what I said. And this is a way to understand how this is so.}}\\
\end{addmargin}
\end{absolutelynopagebreak}

\begin{absolutelynopagebreak}
\setstretch{.7}
{\PaliGlossA{upādānañca hi, ānanda, nābhavissa sabbena sabbaṃ sabbathā sabbaṃ kassaci kimhici, seyyathidaṃ—}}\\
\begin{addmargin}[1em]{2em}
\setstretch{.5}
{\PaliGlossB{Suppose there were totally and utterly no grasping for anyone anywhere.}}\\
\end{addmargin}
\end{absolutelynopagebreak}

\begin{absolutelynopagebreak}
\setstretch{.7}
{\PaliGlossA{kāmupādānaṃ vā diṭṭhupādānaṃ vā sīlabbatupādānaṃ vā attavādupādānaṃ vā, sabbaso upādāne asati upādānanirodhā api nu kho bhavo paññāyethā”ti?}}\\
\begin{addmargin}[1em]{2em}
\setstretch{.5}
{\PaliGlossB{That is, grasping at sensual pleasures, views, precepts and observances, and theories of a self. When there’s no grasping at all, with the cessation of grasping, would continued existence still be found?”}}\\
\end{addmargin}
\end{absolutelynopagebreak}

\begin{absolutelynopagebreak}
\setstretch{.7}
{\PaliGlossA{“no hetaṃ, bhante”.}}\\
\begin{addmargin}[1em]{2em}
\setstretch{.5}
{\PaliGlossB{“No, sir.”}}\\
\end{addmargin}
\end{absolutelynopagebreak}

\begin{absolutelynopagebreak}
\setstretch{.7}
{\PaliGlossA{“tasmātihānanda, eseva hetu etaṃ nidānaṃ esa samudayo esa paccayo bhavassa, yadidaṃ upādānaṃ.}}\\
\begin{addmargin}[1em]{2em}
\setstretch{.5}
{\PaliGlossB{“That’s why this is the cause, source, origin, and condition of continued existence, namely grasping.}}\\
\end{addmargin}
\end{absolutelynopagebreak}

\begin{absolutelynopagebreak}
\setstretch{.7}
{\PaliGlossA{‘taṇhāpaccayā upādānan’ti iti kho panetaṃ vuttaṃ tadānanda, imināpetaṃ pariyāyena veditabbaṃ, yathā taṇhāpaccayā upādānaṃ.}}\\
\begin{addmargin}[1em]{2em}
\setstretch{.5}
{\PaliGlossB{‘Craving is a condition for grasping’—that’s what I said. And this is a way to understand how this is so.}}\\
\end{addmargin}
\end{absolutelynopagebreak}

\begin{absolutelynopagebreak}
\setstretch{.7}
{\PaliGlossA{taṇhā ca hi, ānanda, nābhavissa sabbena sabbaṃ sabbathā sabbaṃ kassaci kimhici, seyyathidaṃ—}}\\
\begin{addmargin}[1em]{2em}
\setstretch{.5}
{\PaliGlossB{Suppose there were totally and utterly no craving for anyone anywhere.}}\\
\end{addmargin}
\end{absolutelynopagebreak}

\begin{absolutelynopagebreak}
\setstretch{.7}
{\PaliGlossA{rūpataṇhā saddataṇhā gandhataṇhā rasataṇhā phoṭṭhabbataṇhā dhammataṇhā, sabbaso taṇhāya asati taṇhānirodhā api nu kho upādānaṃ paññāyethā”ti?}}\\
\begin{addmargin}[1em]{2em}
\setstretch{.5}
{\PaliGlossB{That is, craving for sights, sounds, smells, tastes, touches, and thoughts. When there’s no craving at all, with the cessation of craving, would grasping still be found?”}}\\
\end{addmargin}
\end{absolutelynopagebreak}

\begin{absolutelynopagebreak}
\setstretch{.7}
{\PaliGlossA{“no hetaṃ, bhante”.}}\\
\begin{addmargin}[1em]{2em}
\setstretch{.5}
{\PaliGlossB{“No, sir.”}}\\
\end{addmargin}
\end{absolutelynopagebreak}

\begin{absolutelynopagebreak}
\setstretch{.7}
{\PaliGlossA{“tasmātihānanda, eseva hetu etaṃ nidānaṃ esa samudayo esa paccayo upādānassa, yadidaṃ taṇhā.}}\\
\begin{addmargin}[1em]{2em}
\setstretch{.5}
{\PaliGlossB{“That’s why this is the cause, source, origin, and condition of grasping, namely craving.}}\\
\end{addmargin}
\end{absolutelynopagebreak}

\begin{absolutelynopagebreak}
\setstretch{.7}
{\PaliGlossA{‘vedanāpaccayā taṇhā’ti iti kho panetaṃ vuttaṃ, tadānanda, imināpetaṃ pariyāyena veditabbaṃ, yathā vedanāpaccayā taṇhā.}}\\
\begin{addmargin}[1em]{2em}
\setstretch{.5}
{\PaliGlossB{‘Feeling is a condition for craving’—that’s what I said. And this is a way to understand how this is so.}}\\
\end{addmargin}
\end{absolutelynopagebreak}

\begin{absolutelynopagebreak}
\setstretch{.7}
{\PaliGlossA{vedanā ca hi, ānanda, nābhavissa sabbena sabbaṃ sabbathā sabbaṃ kassaci kimhici, seyyathidaṃ—}}\\
\begin{addmargin}[1em]{2em}
\setstretch{.5}
{\PaliGlossB{Suppose there were totally and utterly no feeling for anyone anywhere.}}\\
\end{addmargin}
\end{absolutelynopagebreak}

\begin{absolutelynopagebreak}
\setstretch{.7}
{\PaliGlossA{cakkhusamphassajā vedanā sotasamphassajā vedanā ghānasamphassajā vedanā jivhāsamphassajā vedanā kāyasamphassajā vedanā manosamphassajā vedanā, sabbaso vedanāya asati vedanānirodhā api nu kho taṇhā paññāyethā”ti?}}\\
\begin{addmargin}[1em]{2em}
\setstretch{.5}
{\PaliGlossB{That is, feeling born of contact through the eye, ear, nose, tongue, body, and mind. When there’s no feeling at all, with the cessation of feeling, would craving still be found?”}}\\
\end{addmargin}
\end{absolutelynopagebreak}

\begin{absolutelynopagebreak}
\setstretch{.7}
{\PaliGlossA{“no hetaṃ, bhante”.}}\\
\begin{addmargin}[1em]{2em}
\setstretch{.5}
{\PaliGlossB{“No, sir.”}}\\
\end{addmargin}
\end{absolutelynopagebreak}

\begin{absolutelynopagebreak}
\setstretch{.7}
{\PaliGlossA{“tasmātihānanda, eseva hetu etaṃ nidānaṃ esa samudayo esa paccayo taṇhāya, yadidaṃ vedanā.}}\\
\begin{addmargin}[1em]{2em}
\setstretch{.5}
{\PaliGlossB{“That’s why this is the cause, source, origin, and condition of craving, namely feeling.}}\\
\end{addmargin}
\end{absolutelynopagebreak}

\begin{absolutelynopagebreak}
\setstretch{.7}
{\PaliGlossA{iti kho panetaṃ, ānanda, vedanaṃ paṭicca taṇhā, taṇhaṃ paṭicca pariyesanā, pariyesanaṃ paṭicca lābho, lābhaṃ paṭicca vinicchayo, vinicchayaṃ paṭicca chandarāgo, chandarāgaṃ paṭicca ajjhosānaṃ, ajjhosānaṃ paṭicca pariggaho, pariggahaṃ paṭicca macchariyaṃ, macchariyaṃ paṭicca ārakkho.}}\\
\begin{addmargin}[1em]{2em}
\setstretch{.5}
{\PaliGlossB{So it is, Ānanda, that feeling is a cause of craving. Craving is a cause of seeking. Seeking is a cause of gaining material possessions. Gaining material possessions is a cause of assessing. Assessing is a cause of desire and lust. Desire and lust is a cause of attachment. Attachment is a cause of possessiveness. Possessiveness is a cause of stinginess. Stinginess is a cause of safeguarding.}}\\
\end{addmargin}
\end{absolutelynopagebreak}

\begin{absolutelynopagebreak}
\setstretch{.7}
{\PaliGlossA{ārakkhādhikaraṇaṃ daṇḍādānasatthādānakalahaviggahavivādatuvaṃtuvaṃpesuññamusāvādā aneke pāpakā akusalā dhammā sambhavanti.}}\\
\begin{addmargin}[1em]{2em}
\setstretch{.5}
{\PaliGlossB{Owing to safeguarding, many bad, unskillful things come to be: taking up the rod and the sword, quarrels, arguments, and fights, accusations, divisive speech, and lies.}}\\
\end{addmargin}
\end{absolutelynopagebreak}

\begin{absolutelynopagebreak}
\setstretch{.7}
{\PaliGlossA{‘ārakkhādhikaraṇaṃ daṇḍādānasatthādānakalahaviggahavivādatuvaṃtuvaṃpesuññamusāvādā aneke pāpakā akusalā dhammā sambhavantī’ti iti kho panetaṃ vuttaṃ, tadānanda, imināpetaṃ pariyāyena veditabbaṃ, yathā ārakkhādhikaraṇaṃ daṇḍādānasatthādānakalahaviggahavivādatuvaṃtuvaṃpesuññamusāvādā aneke pāpakā akusalā dhammā sambhavanti.}}\\
\begin{addmargin}[1em]{2em}
\setstretch{.5}
{\PaliGlossB{‘Owing to safeguarding, many bad, unskillful things come to be: taking up the rod and the sword, quarrels, arguments, and fights, accusations, divisive speech, and lies’—that’s what I said. And this is a way to understand how this is so.}}\\
\end{addmargin}
\end{absolutelynopagebreak}

\begin{absolutelynopagebreak}
\setstretch{.7}
{\PaliGlossA{ārakkho ca hi, ānanda, nābhavissa sabbena sabbaṃ sabbathā sabbaṃ kassaci kimhici, sabbaso ārakkhe asati ārakkhanirodhā api nu kho daṇḍādānasatthādānakalahaviggahavivādatuvaṃtuvaṃpesuññamusāvādā aneke pāpakā akusalā dhammā sambhaveyyun”ti?}}\\
\begin{addmargin}[1em]{2em}
\setstretch{.5}
{\PaliGlossB{Suppose there were totally and utterly no safeguarding for anyone anywhere. When there’s no safeguarding at all, with the cessation of safeguarding, would those many bad, unskillful things still come to be?”}}\\
\end{addmargin}
\end{absolutelynopagebreak}

\begin{absolutelynopagebreak}
\setstretch{.7}
{\PaliGlossA{“no hetaṃ, bhante”.}}\\
\begin{addmargin}[1em]{2em}
\setstretch{.5}
{\PaliGlossB{“No, sir.”}}\\
\end{addmargin}
\end{absolutelynopagebreak}

\begin{absolutelynopagebreak}
\setstretch{.7}
{\PaliGlossA{“tasmātihānanda, eseva hetu etaṃ nidānaṃ esa samudayo esa paccayo daṇḍādānasatthādānakalahaviggahavivādatuvaṃtuvaṃpesuññamusāvādānaṃ anekesaṃ pāpakānaṃ akusalānaṃ dhammānaṃ sambhavāya yadidaṃ ārakkho.}}\\
\begin{addmargin}[1em]{2em}
\setstretch{.5}
{\PaliGlossB{“That’s why this is the cause, source, origin, and condition for the origination of those many bad, unskillful things, namely safeguarding.}}\\
\end{addmargin}
\end{absolutelynopagebreak}

\begin{absolutelynopagebreak}
\setstretch{.7}
{\PaliGlossA{‘macchariyaṃ paṭicca ārakkho’ti iti kho panetaṃ vuttaṃ, tadānanda, imināpetaṃ pariyāyena veditabbaṃ, yathā macchariyaṃ paṭicca ārakkho.}}\\
\begin{addmargin}[1em]{2em}
\setstretch{.5}
{\PaliGlossB{‘Stinginess is a cause of safeguarding’—that’s what I said. And this is a way to understand how this is so.}}\\
\end{addmargin}
\end{absolutelynopagebreak}

\begin{absolutelynopagebreak}
\setstretch{.7}
{\PaliGlossA{macchariyañca hi, ānanda, nābhavissa sabbena sabbaṃ sabbathā sabbaṃ kassaci kimhici, sabbaso macchariye asati macchariyanirodhā api nu kho ārakkho paññāyethā”ti?}}\\
\begin{addmargin}[1em]{2em}
\setstretch{.5}
{\PaliGlossB{Suppose there were totally and utterly no stinginess for anyone anywhere. When there’s no stinginess at all, with the cessation of stinginess, would safeguarding still be found?”}}\\
\end{addmargin}
\end{absolutelynopagebreak}

\begin{absolutelynopagebreak}
\setstretch{.7}
{\PaliGlossA{“no hetaṃ, bhante”.}}\\
\begin{addmargin}[1em]{2em}
\setstretch{.5}
{\PaliGlossB{“No, sir.”}}\\
\end{addmargin}
\end{absolutelynopagebreak}

\begin{absolutelynopagebreak}
\setstretch{.7}
{\PaliGlossA{“tasmātihānanda, eseva hetu etaṃ nidānaṃ esa samudayo esa paccayo ārakkhassa, yadidaṃ macchariyaṃ.}}\\
\begin{addmargin}[1em]{2em}
\setstretch{.5}
{\PaliGlossB{“That’s why this is the cause, source, origin, and condition of safeguarding, namely stinginess.}}\\
\end{addmargin}
\end{absolutelynopagebreak}

\begin{absolutelynopagebreak}
\setstretch{.7}
{\PaliGlossA{‘pariggahaṃ paṭicca macchariyan’ti iti kho panetaṃ vuttaṃ, tadānanda, imināpetaṃ pariyāyena veditabbaṃ, yathā pariggahaṃ paṭicca macchariyaṃ.}}\\
\begin{addmargin}[1em]{2em}
\setstretch{.5}
{\PaliGlossB{‘Possessiveness is a cause of stinginess’—that’s what I said. And this is a way to understand how this is so.}}\\
\end{addmargin}
\end{absolutelynopagebreak}

\begin{absolutelynopagebreak}
\setstretch{.7}
{\PaliGlossA{pariggaho ca hi, ānanda, nābhavissa sabbena sabbaṃ sabbathā sabbaṃ kassaci kimhici, sabbaso pariggahe asati pariggahanirodhā api nu kho macchariyaṃ paññāyethā”ti?}}\\
\begin{addmargin}[1em]{2em}
\setstretch{.5}
{\PaliGlossB{Suppose there were totally and utterly no possessiveness for anyone anywhere. When there’s no possessiveness at all, with the cessation of possessiveness, would stinginess still be found?”}}\\
\end{addmargin}
\end{absolutelynopagebreak}

\begin{absolutelynopagebreak}
\setstretch{.7}
{\PaliGlossA{“no hetaṃ, bhante”.}}\\
\begin{addmargin}[1em]{2em}
\setstretch{.5}
{\PaliGlossB{“No, sir.”}}\\
\end{addmargin}
\end{absolutelynopagebreak}

\begin{absolutelynopagebreak}
\setstretch{.7}
{\PaliGlossA{“tasmātihānanda, eseva hetu etaṃ nidānaṃ esa samudayo esa paccayo macchariyassa, yadidaṃ pariggaho.}}\\
\begin{addmargin}[1em]{2em}
\setstretch{.5}
{\PaliGlossB{“That’s why this is the cause, source, origin, and condition of stinginess, namely possessiveness.}}\\
\end{addmargin}
\end{absolutelynopagebreak}

\begin{absolutelynopagebreak}
\setstretch{.7}
{\PaliGlossA{‘ajjhosānaṃ paṭicca pariggaho’ti iti kho panetaṃ vuttaṃ, tadānanda, imināpetaṃ pariyāyena veditabbaṃ, yathā ajjhosānaṃ paṭicca pariggaho.}}\\
\begin{addmargin}[1em]{2em}
\setstretch{.5}
{\PaliGlossB{‘Attachment is a cause of possessiveness’—that’s what I said. And this is a way to understand how this is so.}}\\
\end{addmargin}
\end{absolutelynopagebreak}

\begin{absolutelynopagebreak}
\setstretch{.7}
{\PaliGlossA{ajjhosānañca hi, ānanda, nābhavissa sabbena sabbaṃ sabbathā sabbaṃ kassaci kimhici, sabbaso ajjhosāne asati ajjhosānanirodhā api nu kho pariggaho paññāyethā”ti?}}\\
\begin{addmargin}[1em]{2em}
\setstretch{.5}
{\PaliGlossB{Suppose there were totally and utterly no attachment for anyone anywhere. When there’s no attachment at all, with the cessation of attachment, would possessiveness still be found?”}}\\
\end{addmargin}
\end{absolutelynopagebreak}

\begin{absolutelynopagebreak}
\setstretch{.7}
{\PaliGlossA{“no hetaṃ, bhante”.}}\\
\begin{addmargin}[1em]{2em}
\setstretch{.5}
{\PaliGlossB{“No, sir.”}}\\
\end{addmargin}
\end{absolutelynopagebreak}

\begin{absolutelynopagebreak}
\setstretch{.7}
{\PaliGlossA{“tasmātihānanda, eseva hetu etaṃ nidānaṃ esa samudayo esa paccayo pariggahassa—yadidaṃ ajjhosānaṃ.}}\\
\begin{addmargin}[1em]{2em}
\setstretch{.5}
{\PaliGlossB{“That’s why this is the cause, source, origin, and condition of possessiveness, namely attachment.}}\\
\end{addmargin}
\end{absolutelynopagebreak}

\begin{absolutelynopagebreak}
\setstretch{.7}
{\PaliGlossA{‘chandarāgaṃ paṭicca ajjhosānan’ti iti kho panetaṃ vuttaṃ, tadānanda, imināpetaṃ pariyāyena veditabbaṃ, yathā chandarāgaṃ paṭicca ajjhosānaṃ.}}\\
\begin{addmargin}[1em]{2em}
\setstretch{.5}
{\PaliGlossB{‘Desire and lust is a cause of attachment’—that’s what I said. And this is a way to understand how this is so.}}\\
\end{addmargin}
\end{absolutelynopagebreak}

\begin{absolutelynopagebreak}
\setstretch{.7}
{\PaliGlossA{chandarāgo ca hi, ānanda, nābhavissa sabbena sabbaṃ sabbathā sabbaṃ kassaci kimhici, sabbaso chandarāge asati chandarāganirodhā api nu kho ajjhosānaṃ paññāyethā”ti?}}\\
\begin{addmargin}[1em]{2em}
\setstretch{.5}
{\PaliGlossB{Suppose there were totally and utterly no desire and lust for anyone anywhere. When there’s no desire and lust at all, with the cessation of desire and lust, would attachment still be found?”}}\\
\end{addmargin}
\end{absolutelynopagebreak}

\begin{absolutelynopagebreak}
\setstretch{.7}
{\PaliGlossA{“no hetaṃ, bhante”.}}\\
\begin{addmargin}[1em]{2em}
\setstretch{.5}
{\PaliGlossB{“No, sir.”}}\\
\end{addmargin}
\end{absolutelynopagebreak}

\begin{absolutelynopagebreak}
\setstretch{.7}
{\PaliGlossA{“tasmātihānanda, eseva hetu etaṃ nidānaṃ esa samudayo esa paccayo ajjhosānassa, yadidaṃ chandarāgo.}}\\
\begin{addmargin}[1em]{2em}
\setstretch{.5}
{\PaliGlossB{“That’s why this is the cause, source, origin, and condition of attachment, namely desire and lust.}}\\
\end{addmargin}
\end{absolutelynopagebreak}

\begin{absolutelynopagebreak}
\setstretch{.7}
{\PaliGlossA{‘vinicchayaṃ paṭicca chandarāgo’ti iti kho panetaṃ vuttaṃ, tadānanda, imināpetaṃ pariyāyena veditabbaṃ, yathā vinicchayaṃ paṭicca chandarāgo.}}\\
\begin{addmargin}[1em]{2em}
\setstretch{.5}
{\PaliGlossB{‘Assessing is a cause of desire and lust’—that’s what I said. And this is a way to understand how this is so.}}\\
\end{addmargin}
\end{absolutelynopagebreak}

\begin{absolutelynopagebreak}
\setstretch{.7}
{\PaliGlossA{vinicchayo ca hi, ānanda, nābhavissa sabbena sabbaṃ sabbathā sabbaṃ kassaci kimhici, sabbaso vinicchaye asati vinicchayanirodhā api nu kho chandarāgo paññāyethā”ti?}}\\
\begin{addmargin}[1em]{2em}
\setstretch{.5}
{\PaliGlossB{Suppose there were totally and utterly no assessing for anyone anywhere. When there’s no assessing at all, with the cessation of assessing, would desire and lust still be found?”}}\\
\end{addmargin}
\end{absolutelynopagebreak}

\begin{absolutelynopagebreak}
\setstretch{.7}
{\PaliGlossA{“no hetaṃ, bhante”.}}\\
\begin{addmargin}[1em]{2em}
\setstretch{.5}
{\PaliGlossB{“No, sir.”}}\\
\end{addmargin}
\end{absolutelynopagebreak}

\begin{absolutelynopagebreak}
\setstretch{.7}
{\PaliGlossA{“tasmātihānanda, eseva hetu etaṃ nidānaṃ esa samudayo esa paccayo chandarāgassa, yadidaṃ vinicchayo.}}\\
\begin{addmargin}[1em]{2em}
\setstretch{.5}
{\PaliGlossB{“That’s why this is the cause, source, origin, and condition of desire and lust, namely assessing.}}\\
\end{addmargin}
\end{absolutelynopagebreak}

\begin{absolutelynopagebreak}
\setstretch{.7}
{\PaliGlossA{‘lābhaṃ paṭicca vinicchayo’ti iti kho panetaṃ vuttaṃ, tadānanda, imināpetaṃ pariyāyena veditabbaṃ, yathā lābhaṃ paṭicca vinicchayo.}}\\
\begin{addmargin}[1em]{2em}
\setstretch{.5}
{\PaliGlossB{‘Gaining material possessions is a cause of assessing’—that’s what I said. And this is a way to understand how this is so.}}\\
\end{addmargin}
\end{absolutelynopagebreak}

\begin{absolutelynopagebreak}
\setstretch{.7}
{\PaliGlossA{lābho ca hi, ānanda, nābhavissa sabbena sabbaṃ sabbathā sabbaṃ kassaci kimhici, sabbaso lābhe asati lābhanirodhā api nu kho vinicchayo paññāyethā”ti?}}\\
\begin{addmargin}[1em]{2em}
\setstretch{.5}
{\PaliGlossB{Suppose there were totally and utterly no gaining of material possessions for anyone anywhere. When there’s no gaining of material possessions at all, with the cessation of gaining material possessions, would assessing still be found?”}}\\
\end{addmargin}
\end{absolutelynopagebreak}

\begin{absolutelynopagebreak}
\setstretch{.7}
{\PaliGlossA{“no hetaṃ, bhante”.}}\\
\begin{addmargin}[1em]{2em}
\setstretch{.5}
{\PaliGlossB{“No, sir.”}}\\
\end{addmargin}
\end{absolutelynopagebreak}

\begin{absolutelynopagebreak}
\setstretch{.7}
{\PaliGlossA{“tasmātihānanda, eseva hetu etaṃ nidānaṃ esa samudayo esa paccayo vinicchayassa, yadidaṃ lābho.}}\\
\begin{addmargin}[1em]{2em}
\setstretch{.5}
{\PaliGlossB{“That’s why this is the cause, source, origin, and condition of assessing, namely the gaining of material possessions.}}\\
\end{addmargin}
\end{absolutelynopagebreak}

\begin{absolutelynopagebreak}
\setstretch{.7}
{\PaliGlossA{‘pariyesanaṃ paṭicca lābho’ti iti kho panetaṃ vuttaṃ, tadānanda, imināpetaṃ pariyāyena veditabbaṃ, yathā pariyesanaṃ paṭicca lābho.}}\\
\begin{addmargin}[1em]{2em}
\setstretch{.5}
{\PaliGlossB{‘Seeking is a cause of gaining material possessions’—that’s what I said. And this is a way to understand how this is so.}}\\
\end{addmargin}
\end{absolutelynopagebreak}

\begin{absolutelynopagebreak}
\setstretch{.7}
{\PaliGlossA{pariyesanā ca hi, ānanda, nābhavissa sabbena sabbaṃ sabbathā sabbaṃ kassaci kimhici, sabbaso pariyesanāya asati pariyesanānirodhā api nu kho lābho paññāyethā”ti?}}\\
\begin{addmargin}[1em]{2em}
\setstretch{.5}
{\PaliGlossB{Suppose there were totally and utterly no seeking for anyone anywhere. When there’s no seeking at all, with the cessation of seeking, would the gaining of material possessions still be found?”}}\\
\end{addmargin}
\end{absolutelynopagebreak}

\begin{absolutelynopagebreak}
\setstretch{.7}
{\PaliGlossA{“no hetaṃ, bhante”.}}\\
\begin{addmargin}[1em]{2em}
\setstretch{.5}
{\PaliGlossB{“No, sir.”}}\\
\end{addmargin}
\end{absolutelynopagebreak}

\begin{absolutelynopagebreak}
\setstretch{.7}
{\PaliGlossA{“tasmātihānanda, eseva hetu etaṃ nidānaṃ esa samudayo esa paccayo lābhassa, yadidaṃ pariyesanā.}}\\
\begin{addmargin}[1em]{2em}
\setstretch{.5}
{\PaliGlossB{“That’s why this is the cause, source, origin, and condition of gaining material possessions, namely seeking.}}\\
\end{addmargin}
\end{absolutelynopagebreak}

\begin{absolutelynopagebreak}
\setstretch{.7}
{\PaliGlossA{‘taṇhaṃ paṭicca pariyesanā’ti iti kho panetaṃ vuttaṃ, tadānanda, imināpetaṃ pariyāyena veditabbaṃ, yathā taṇhaṃ paṭicca pariyesanā.}}\\
\begin{addmargin}[1em]{2em}
\setstretch{.5}
{\PaliGlossB{‘Craving is a cause of seeking’—that’s what I said. And this is a way to understand how this is so.}}\\
\end{addmargin}
\end{absolutelynopagebreak}

\begin{absolutelynopagebreak}
\setstretch{.7}
{\PaliGlossA{taṇhā ca hi, ānanda, nābhavissa sabbena sabbaṃ sabbathā sabbaṃ kassaci kimhici, seyyathidaṃ—}}\\
\begin{addmargin}[1em]{2em}
\setstretch{.5}
{\PaliGlossB{Suppose there were totally and utterly no craving for anyone anywhere.}}\\
\end{addmargin}
\end{absolutelynopagebreak}

\begin{absolutelynopagebreak}
\setstretch{.7}
{\PaliGlossA{kāmataṇhā bhavataṇhā vibhavataṇhā, sabbaso taṇhāya asati taṇhānirodhā api nu kho pariyesanā paññāyethā”ti?}}\\
\begin{addmargin}[1em]{2em}
\setstretch{.5}
{\PaliGlossB{That is, craving for sensual pleasures, craving for continued existence, and craving to end existence. When there’s no craving at all, with the cessation of craving, would seeking still be found?”}}\\
\end{addmargin}
\end{absolutelynopagebreak}

\begin{absolutelynopagebreak}
\setstretch{.7}
{\PaliGlossA{“no hetaṃ, bhante”.}}\\
\begin{addmargin}[1em]{2em}
\setstretch{.5}
{\PaliGlossB{“No, sir.”}}\\
\end{addmargin}
\end{absolutelynopagebreak}

\begin{absolutelynopagebreak}
\setstretch{.7}
{\PaliGlossA{“tasmātihānanda, eseva hetu etaṃ nidānaṃ esa samudayo esa paccayo pariyesanāya, yadidaṃ taṇhā.}}\\
\begin{addmargin}[1em]{2em}
\setstretch{.5}
{\PaliGlossB{“That’s why this is the cause, source, origin, and condition of seeking, namely craving.}}\\
\end{addmargin}
\end{absolutelynopagebreak}

\begin{absolutelynopagebreak}
\setstretch{.7}
{\PaliGlossA{iti kho, ānanda, ime dve dhammā dvayena vedanāya ekasamosaraṇā bhavanti.}}\\
\begin{addmargin}[1em]{2em}
\setstretch{.5}
{\PaliGlossB{And so, Ānanda, these two things are united by the two aspects of feeling.}}\\
\end{addmargin}
\end{absolutelynopagebreak}

\begin{absolutelynopagebreak}
\setstretch{.7}
{\PaliGlossA{‘phassapaccayā vedanā’ti iti kho panetaṃ vuttaṃ, tadānanda, imināpetaṃ pariyāyena veditabbaṃ, yathā phassapaccayā vedanā.}}\\
\begin{addmargin}[1em]{2em}
\setstretch{.5}
{\PaliGlossB{‘Contact is a condition for feeling’—that’s what I said. And this is a way to understand how this is so.}}\\
\end{addmargin}
\end{absolutelynopagebreak}

\begin{absolutelynopagebreak}
\setstretch{.7}
{\PaliGlossA{phasso ca hi, ānanda, nābhavissa sabbena sabbaṃ sabbathā sabbaṃ kassaci kimhici, seyyathidaṃ—}}\\
\begin{addmargin}[1em]{2em}
\setstretch{.5}
{\PaliGlossB{Suppose there were totally and utterly no contact for anyone anywhere.}}\\
\end{addmargin}
\end{absolutelynopagebreak}

\begin{absolutelynopagebreak}
\setstretch{.7}
{\PaliGlossA{cakkhusamphasso sotasamphasso ghānasamphasso jivhāsamphasso kāyasamphasso manosamphasso, sabbaso phasse asati phassanirodhā api nu kho vedanā paññāyethā”ti?}}\\
\begin{addmargin}[1em]{2em}
\setstretch{.5}
{\PaliGlossB{That is, contact through the eye, ear, nose, tongue, body, and mind. When there’s no contact at all, with the cessation of contact, would feeling still be found?”}}\\
\end{addmargin}
\end{absolutelynopagebreak}

\begin{absolutelynopagebreak}
\setstretch{.7}
{\PaliGlossA{“no hetaṃ, bhante”.}}\\
\begin{addmargin}[1em]{2em}
\setstretch{.5}
{\PaliGlossB{“No, sir.”}}\\
\end{addmargin}
\end{absolutelynopagebreak}

\begin{absolutelynopagebreak}
\setstretch{.7}
{\PaliGlossA{“tasmātihānanda, eseva hetu etaṃ nidānaṃ esa samudayo esa paccayo vedanāya, yadidaṃ phasso.}}\\
\begin{addmargin}[1em]{2em}
\setstretch{.5}
{\PaliGlossB{“That’s why this is the cause, source, origin, and condition of feeling, namely contact.}}\\
\end{addmargin}
\end{absolutelynopagebreak}

\begin{absolutelynopagebreak}
\setstretch{.7}
{\PaliGlossA{‘nāmarūpapaccayā phasso’ti iti kho panetaṃ vuttaṃ, tadānanda, imināpetaṃ pariyāyena veditabbaṃ, yathā nāmarūpapaccayā phasso.}}\\
\begin{addmargin}[1em]{2em}
\setstretch{.5}
{\PaliGlossB{‘Name and form are conditions for contact’—that’s what I said. And this is a way to understand how this is so.}}\\
\end{addmargin}
\end{absolutelynopagebreak}

\begin{absolutelynopagebreak}
\setstretch{.7}
{\PaliGlossA{yehi, ānanda, ākārehi yehi liṅgehi yehi nimittehi yehi uddesehi nāmakāyassa paññatti hoti, tesu ākāresu tesu liṅgesu tesu nimittesu tesu uddesesu asati api nu kho rūpakāye adhivacanasamphasso paññāyethā”ti?}}\\
\begin{addmargin}[1em]{2em}
\setstretch{.5}
{\PaliGlossB{Suppose there were none of the features, attributes, signs, and details by which the category of mental phenomena is found. Would linguistic contact still be found in the category of physical phenomena?”}}\\
\end{addmargin}
\end{absolutelynopagebreak}

\begin{absolutelynopagebreak}
\setstretch{.7}
{\PaliGlossA{“no hetaṃ, bhante”.}}\\
\begin{addmargin}[1em]{2em}
\setstretch{.5}
{\PaliGlossB{“No, sir.”}}\\
\end{addmargin}
\end{absolutelynopagebreak}

\begin{absolutelynopagebreak}
\setstretch{.7}
{\PaliGlossA{“yehi, ānanda, ākārehi yehi liṅgehi yehi nimittehi yehi uddesehi rūpakāyassa paññatti hoti, tesu ākāresu … pe … tesu uddesesu asati api nu kho nāmakāye paṭighasamphasso paññāyethā”ti?}}\\
\begin{addmargin}[1em]{2em}
\setstretch{.5}
{\PaliGlossB{“Suppose there were none of the features, attributes, signs, and details by which the category of physical phenomena is found. Would impingement contact still be found in the category of mental phenomena?”}}\\
\end{addmargin}
\end{absolutelynopagebreak}

\begin{absolutelynopagebreak}
\setstretch{.7}
{\PaliGlossA{“no hetaṃ, bhante”.}}\\
\begin{addmargin}[1em]{2em}
\setstretch{.5}
{\PaliGlossB{“No, sir.”}}\\
\end{addmargin}
\end{absolutelynopagebreak}

\begin{absolutelynopagebreak}
\setstretch{.7}
{\PaliGlossA{“yehi, ānanda, ākārehi … pe … yehi uddesehi nāmakāyassa ca rūpakāyassa ca paññatti hoti, tesu ākāresu … pe … tesu uddesesu asati api nu kho adhivacanasamphasso vā paṭighasamphasso vā paññāyethā”ti?}}\\
\begin{addmargin}[1em]{2em}
\setstretch{.5}
{\PaliGlossB{“Suppose there were none of the features, attributes, signs, and details by which the categories of mental or physical phenomena are found. Would either linguistic contact or impingement contact still be found?”}}\\
\end{addmargin}
\end{absolutelynopagebreak}

\begin{absolutelynopagebreak}
\setstretch{.7}
{\PaliGlossA{“no hetaṃ, bhante”.}}\\
\begin{addmargin}[1em]{2em}
\setstretch{.5}
{\PaliGlossB{“No, sir.”}}\\
\end{addmargin}
\end{absolutelynopagebreak}

\begin{absolutelynopagebreak}
\setstretch{.7}
{\PaliGlossA{“yehi, ānanda, ākārehi … pe … yehi uddesehi nāmarūpassa paññatti hoti, tesu ākāresu … pe … tesu uddesesu asati api nu kho phasso paññāyethā”ti?}}\\
\begin{addmargin}[1em]{2em}
\setstretch{.5}
{\PaliGlossB{“Suppose there were none of the features, attributes, signs, and details by which name and form are found. Would contact still be found?”}}\\
\end{addmargin}
\end{absolutelynopagebreak}

\begin{absolutelynopagebreak}
\setstretch{.7}
{\PaliGlossA{“no hetaṃ, bhante”.}}\\
\begin{addmargin}[1em]{2em}
\setstretch{.5}
{\PaliGlossB{“No, sir.”}}\\
\end{addmargin}
\end{absolutelynopagebreak}

\begin{absolutelynopagebreak}
\setstretch{.7}
{\PaliGlossA{“tasmātihānanda, eseva hetu etaṃ nidānaṃ esa samudayo esa paccayo phassassa, yadidaṃ nāmarūpaṃ.}}\\
\begin{addmargin}[1em]{2em}
\setstretch{.5}
{\PaliGlossB{“That’s why this is the cause, source, origin, and condition of contact, namely name and form.}}\\
\end{addmargin}
\end{absolutelynopagebreak}

\begin{absolutelynopagebreak}
\setstretch{.7}
{\PaliGlossA{‘viññāṇapaccayā nāmarūpan’ti iti kho panetaṃ vuttaṃ, tadānanda, imināpetaṃ pariyāyena veditabbaṃ, yathā viññāṇapaccayā nāmarūpaṃ.}}\\
\begin{addmargin}[1em]{2em}
\setstretch{.5}
{\PaliGlossB{‘Consciousness is a condition for name and form’—that’s what I said. And this is a way to understand how this is so.}}\\
\end{addmargin}
\end{absolutelynopagebreak}

\begin{absolutelynopagebreak}
\setstretch{.7}
{\PaliGlossA{viññāṇañca hi, ānanda, mātukucchismiṃ na okkamissatha, api nu kho nāmarūpaṃ mātukucchismiṃ samuccissathā”ti?}}\\
\begin{addmargin}[1em]{2em}
\setstretch{.5}
{\PaliGlossB{If consciousness were not conceived in the mother’s womb, would name and form coagulate there?”}}\\
\end{addmargin}
\end{absolutelynopagebreak}

\begin{absolutelynopagebreak}
\setstretch{.7}
{\PaliGlossA{“no hetaṃ, bhante”.}}\\
\begin{addmargin}[1em]{2em}
\setstretch{.5}
{\PaliGlossB{“No, sir.”}}\\
\end{addmargin}
\end{absolutelynopagebreak}

\begin{absolutelynopagebreak}
\setstretch{.7}
{\PaliGlossA{“viññāṇañca hi, ānanda, mātukucchismiṃ okkamitvā vokkamissatha, api nu kho nāmarūpaṃ itthattāya abhinibbattissathā”ti?}}\\
\begin{addmargin}[1em]{2em}
\setstretch{.5}
{\PaliGlossB{“If consciousness, after being conceived in the mother’s womb, were to be miscarried, would name and form be born into this state of existence?”}}\\
\end{addmargin}
\end{absolutelynopagebreak}

\begin{absolutelynopagebreak}
\setstretch{.7}
{\PaliGlossA{“no hetaṃ, bhante”.}}\\
\begin{addmargin}[1em]{2em}
\setstretch{.5}
{\PaliGlossB{“No, sir.”}}\\
\end{addmargin}
\end{absolutelynopagebreak}

\begin{absolutelynopagebreak}
\setstretch{.7}
{\PaliGlossA{“viññāṇañca hi, ānanda, daharasseva sato vocchijjissatha kumārakassa vā kumārikāya vā, api nu kho nāmarūpaṃ vuddhiṃ virūḷhiṃ vepullaṃ āpajjissathā”ti?}}\\
\begin{addmargin}[1em]{2em}
\setstretch{.5}
{\PaliGlossB{“If the consciousness of a young boy or girl were to be cut off, would name and form achieve growth, increase, and maturity?”}}\\
\end{addmargin}
\end{absolutelynopagebreak}

\begin{absolutelynopagebreak}
\setstretch{.7}
{\PaliGlossA{“no hetaṃ, bhante”.}}\\
\begin{addmargin}[1em]{2em}
\setstretch{.5}
{\PaliGlossB{“No, sir.”}}\\
\end{addmargin}
\end{absolutelynopagebreak}

\begin{absolutelynopagebreak}
\setstretch{.7}
{\PaliGlossA{“tasmātihānanda, eseva hetu etaṃ nidānaṃ esa samudayo esa paccayo nāmarūpassa—yadidaṃ viññāṇaṃ.}}\\
\begin{addmargin}[1em]{2em}
\setstretch{.5}
{\PaliGlossB{“That’s why this is the cause, source, origin, and condition of name and form, namely consciousness.}}\\
\end{addmargin}
\end{absolutelynopagebreak}

\begin{absolutelynopagebreak}
\setstretch{.7}
{\PaliGlossA{‘nāmarūpapaccayā viññāṇan’ti iti kho panetaṃ vuttaṃ, tadānanda, imināpetaṃ pariyāyena veditabbaṃ, yathā nāmarūpapaccayā viññāṇaṃ.}}\\
\begin{addmargin}[1em]{2em}
\setstretch{.5}
{\PaliGlossB{‘Name and form are conditions for consciousness’—that’s what I said. And this is a way to understand how this is so.}}\\
\end{addmargin}
\end{absolutelynopagebreak}

\begin{absolutelynopagebreak}
\setstretch{.7}
{\PaliGlossA{viññāṇañca hi, ānanda, nāmarūpe patiṭṭhaṃ na labhissatha, api nu kho āyatiṃ jātijarāmaraṇaṃ dukkhasamudayasambhavo paññāyethā”ti?}}\\
\begin{addmargin}[1em]{2em}
\setstretch{.5}
{\PaliGlossB{If consciousness were not to become established in name and form, would the coming to be of the origin of suffering—of rebirth, old age, and death in the future—be found?”}}\\
\end{addmargin}
\end{absolutelynopagebreak}

\begin{absolutelynopagebreak}
\setstretch{.7}
{\PaliGlossA{“no hetaṃ, bhante”.}}\\
\begin{addmargin}[1em]{2em}
\setstretch{.5}
{\PaliGlossB{“No, sir.”}}\\
\end{addmargin}
\end{absolutelynopagebreak}

\begin{absolutelynopagebreak}
\setstretch{.7}
{\PaliGlossA{“tasmātihānanda, eseva hetu etaṃ nidānaṃ esa samudayo esa paccayo viññāṇassa yadidaṃ nāmarūpaṃ.}}\\
\begin{addmargin}[1em]{2em}
\setstretch{.5}
{\PaliGlossB{“That’s why this is the cause, source, origin, and condition of consciousness, namely name and form.}}\\
\end{addmargin}
\end{absolutelynopagebreak}

\begin{absolutelynopagebreak}
\setstretch{.7}
{\PaliGlossA{ettāvatā kho, ānanda, jāyetha vā jīyetha vā mīyetha vā cavetha vā upapajjetha vā.}}\\
\begin{addmargin}[1em]{2em}
\setstretch{.5}
{\PaliGlossB{This is the extent to which one may be reborn, grow old, die, pass away, or reappear.}}\\
\end{addmargin}
\end{absolutelynopagebreak}

\begin{absolutelynopagebreak}
\setstretch{.7}
{\PaliGlossA{ettāvatā adhivacanapatho, ettāvatā niruttipatho, ettāvatā paññattipatho, ettāvatā paññāvacaraṃ, ettāvatā vaṭṭaṃ vattati itthattaṃ paññāpanāya yadidaṃ nāmarūpaṃ saha viññāṇena aññamaññapaccayatā pavattati.}}\\
\begin{addmargin}[1em]{2em}
\setstretch{.5}
{\PaliGlossB{This is how far the scope of language, terminology, and description extends; how far the sphere of wisdom extends; how far the cycle of rebirths continues so that this state of existence is to be found; namely, name and form together with consciousness.}}\\
\end{addmargin}
\end{absolutelynopagebreak}

\begin{absolutelynopagebreak}
\setstretch{.7}
{\PaliGlossA{2. attapaññatti}}\\
\begin{addmargin}[1em]{2em}
\setstretch{.5}
{\PaliGlossB{2. Describing the Self}}\\
\end{addmargin}
\end{absolutelynopagebreak}

\begin{absolutelynopagebreak}
\setstretch{.7}
{\PaliGlossA{kittāvatā ca, ānanda, attānaṃ paññapento paññapeti?}}\\
\begin{addmargin}[1em]{2em}
\setstretch{.5}
{\PaliGlossB{How do those who describe the self describe it?}}\\
\end{addmargin}
\end{absolutelynopagebreak}

\begin{absolutelynopagebreak}
\setstretch{.7}
{\PaliGlossA{rūpiṃ vā hi, ānanda, parittaṃ attānaṃ paññapento paññapeti:}}\\
\begin{addmargin}[1em]{2em}
\setstretch{.5}
{\PaliGlossB{They describe it as physical and limited:}}\\
\end{addmargin}
\end{absolutelynopagebreak}

\begin{absolutelynopagebreak}
\setstretch{.7}
{\PaliGlossA{‘rūpī me paritto attā’ti.}}\\
\begin{addmargin}[1em]{2em}
\setstretch{.5}
{\PaliGlossB{‘My self is physical and limited.’}}\\
\end{addmargin}
\end{absolutelynopagebreak}

\begin{absolutelynopagebreak}
\setstretch{.7}
{\PaliGlossA{rūpiṃ vā hi, ānanda, anantaṃ attānaṃ paññapento paññapeti:}}\\
\begin{addmargin}[1em]{2em}
\setstretch{.5}
{\PaliGlossB{Or they describe it as physical and infinite:}}\\
\end{addmargin}
\end{absolutelynopagebreak}

\begin{absolutelynopagebreak}
\setstretch{.7}
{\PaliGlossA{‘rūpī me ananto attā’ti.}}\\
\begin{addmargin}[1em]{2em}
\setstretch{.5}
{\PaliGlossB{‘My self is physical and infinite.’}}\\
\end{addmargin}
\end{absolutelynopagebreak}

\begin{absolutelynopagebreak}
\setstretch{.7}
{\PaliGlossA{arūpiṃ vā hi, ānanda, parittaṃ attānaṃ paññapento paññapeti:}}\\
\begin{addmargin}[1em]{2em}
\setstretch{.5}
{\PaliGlossB{Or they describe it as formless and limited:}}\\
\end{addmargin}
\end{absolutelynopagebreak}

\begin{absolutelynopagebreak}
\setstretch{.7}
{\PaliGlossA{‘arūpī me paritto attā’ti.}}\\
\begin{addmargin}[1em]{2em}
\setstretch{.5}
{\PaliGlossB{‘My self is formless and limited.’}}\\
\end{addmargin}
\end{absolutelynopagebreak}

\begin{absolutelynopagebreak}
\setstretch{.7}
{\PaliGlossA{arūpiṃ vā hi, ānanda, anantaṃ attānaṃ paññapento paññapeti:}}\\
\begin{addmargin}[1em]{2em}
\setstretch{.5}
{\PaliGlossB{Or they describe it as formless and infinite:}}\\
\end{addmargin}
\end{absolutelynopagebreak}

\begin{absolutelynopagebreak}
\setstretch{.7}
{\PaliGlossA{‘arūpī me ananto attā’ti.}}\\
\begin{addmargin}[1em]{2em}
\setstretch{.5}
{\PaliGlossB{‘My self is formless and infinite.’}}\\
\end{addmargin}
\end{absolutelynopagebreak}

\begin{absolutelynopagebreak}
\setstretch{.7}
{\PaliGlossA{tatrānanda, yo so rūpiṃ parittaṃ attānaṃ paññapento paññapeti.}}\\
\begin{addmargin}[1em]{2em}
\setstretch{.5}
{\PaliGlossB{Now, take those who describe the self as physical and limited.}}\\
\end{addmargin}
\end{absolutelynopagebreak}

\begin{absolutelynopagebreak}
\setstretch{.7}
{\PaliGlossA{etarahi vā so rūpiṃ parittaṃ attānaṃ paññapento paññapeti, tattha bhāviṃ vā so rūpiṃ parittaṃ attānaṃ paññapento paññapeti, ‘atathaṃ vā pana santaṃ tathattāya upakappessāmī’ti iti vā panassa hoti.}}\\
\begin{addmargin}[1em]{2em}
\setstretch{.5}
{\PaliGlossB{They describe the self as physical and limited in the present; or in some future life; or else they think: ‘Though it is not like that, I will ensure it is provided with what it needs to become like that.’}}\\
\end{addmargin}
\end{absolutelynopagebreak}

\begin{absolutelynopagebreak}
\setstretch{.7}
{\PaliGlossA{evaṃ santaṃ kho, ānanda, rūpiṃ parittattānudiṭṭhi anusetīti iccālaṃ vacanāya.}}\\
\begin{addmargin}[1em]{2em}
\setstretch{.5}
{\PaliGlossB{This being so, it’s appropriate to say that a view of self as physical and limited underlies them.}}\\
\end{addmargin}
\end{absolutelynopagebreak}

\begin{absolutelynopagebreak}
\setstretch{.7}
{\PaliGlossA{tatrānanda, yo so rūpiṃ anantaṃ attānaṃ paññapento paññapeti.}}\\
\begin{addmargin}[1em]{2em}
\setstretch{.5}
{\PaliGlossB{Now, take those who describe the self as physical and infinite …}}\\
\end{addmargin}
\end{absolutelynopagebreak}

\begin{absolutelynopagebreak}
\setstretch{.7}
{\PaliGlossA{etarahi vā so rūpiṃ anantaṃ attānaṃ paññapento paññapeti, tattha bhāviṃ vā so rūpiṃ anantaṃ attānaṃ paññapento paññapeti, ‘atathaṃ vā pana santaṃ tathattāya upakappessāmī’ti iti vā panassa hoti.}}\\
\begin{addmargin}[1em]{2em}
\setstretch{.5}
{\PaliGlossB{    -}}\\
\end{addmargin}
\end{absolutelynopagebreak}

\begin{absolutelynopagebreak}
\setstretch{.7}
{\PaliGlossA{evaṃ santaṃ kho, ānanda, rūpiṃ anantattānudiṭṭhi anusetīti iccālaṃ vacanāya.}}\\
\begin{addmargin}[1em]{2em}
\setstretch{.5}
{\PaliGlossB{    -}}\\
\end{addmargin}
\end{absolutelynopagebreak}

\begin{absolutelynopagebreak}
\setstretch{.7}
{\PaliGlossA{tatrānanda, yo so arūpiṃ parittaṃ attānaṃ paññapento paññapeti.}}\\
\begin{addmargin}[1em]{2em}
\setstretch{.5}
{\PaliGlossB{formless and limited …}}\\
\end{addmargin}
\end{absolutelynopagebreak}

\begin{absolutelynopagebreak}
\setstretch{.7}
{\PaliGlossA{etarahi vā so arūpiṃ parittaṃ attānaṃ paññapento paññapeti, tattha bhāviṃ vā so arūpiṃ parittaṃ attānaṃ paññapento paññapeti, ‘atathaṃ vā pana santaṃ tathattāya upakappessāmī’ti iti vā panassa hoti.}}\\
\begin{addmargin}[1em]{2em}
\setstretch{.5}
{\PaliGlossB{    -}}\\
\end{addmargin}
\end{absolutelynopagebreak}

\begin{absolutelynopagebreak}
\setstretch{.7}
{\PaliGlossA{evaṃ santaṃ kho, ānanda, arūpiṃ parittattānudiṭṭhi anusetīti iccālaṃ vacanāya.}}\\
\begin{addmargin}[1em]{2em}
\setstretch{.5}
{\PaliGlossB{    -}}\\
\end{addmargin}
\end{absolutelynopagebreak}

\begin{absolutelynopagebreak}
\setstretch{.7}
{\PaliGlossA{tatrānanda, yo so arūpiṃ anantaṃ attānaṃ paññapento paññapeti.}}\\
\begin{addmargin}[1em]{2em}
\setstretch{.5}
{\PaliGlossB{formless and infinite.}}\\
\end{addmargin}
\end{absolutelynopagebreak}

\begin{absolutelynopagebreak}
\setstretch{.7}
{\PaliGlossA{etarahi vā so arūpiṃ anantaṃ attānaṃ paññapento paññapeti, tattha bhāviṃ vā so arūpiṃ anantaṃ attānaṃ paññapento paññapeti, ‘atathaṃ vā pana santaṃ tathattāya upakappessāmī’ti iti vā panassa hoti.}}\\
\begin{addmargin}[1em]{2em}
\setstretch{.5}
{\PaliGlossB{They describe the self as formless and infinite in the present; or in some future life; or else they think: ‘Though it is not like that, I will ensure it is provided with what it needs to become like that.’}}\\
\end{addmargin}
\end{absolutelynopagebreak}

\begin{absolutelynopagebreak}
\setstretch{.7}
{\PaliGlossA{evaṃ santaṃ kho, ānanda, arūpiṃ anantattānudiṭṭhi anusetīti iccālaṃ vacanāya.}}\\
\begin{addmargin}[1em]{2em}
\setstretch{.5}
{\PaliGlossB{This being so, it’s appropriate to say that a view of self as formless and infinite underlies them.}}\\
\end{addmargin}
\end{absolutelynopagebreak}

\begin{absolutelynopagebreak}
\setstretch{.7}
{\PaliGlossA{ettāvatā kho, ānanda, attānaṃ paññapento paññapeti.}}\\
\begin{addmargin}[1em]{2em}
\setstretch{.5}
{\PaliGlossB{That’s how those who describe the self describe it.}}\\
\end{addmargin}
\end{absolutelynopagebreak}

\begin{absolutelynopagebreak}
\setstretch{.7}
{\PaliGlossA{3. naattapaññatti}}\\
\begin{addmargin}[1em]{2em}
\setstretch{.5}
{\PaliGlossB{3. Not Describing the Self}}\\
\end{addmargin}
\end{absolutelynopagebreak}

\begin{absolutelynopagebreak}
\setstretch{.7}
{\PaliGlossA{kittāvatā ca, ānanda, attānaṃ na paññapento na paññapeti?}}\\
\begin{addmargin}[1em]{2em}
\setstretch{.5}
{\PaliGlossB{How do those who don’t describe the self not describe it?}}\\
\end{addmargin}
\end{absolutelynopagebreak}

\begin{absolutelynopagebreak}
\setstretch{.7}
{\PaliGlossA{rūpiṃ vā hi, ānanda, parittaṃ attānaṃ na paññapento na paññapeti:}}\\
\begin{addmargin}[1em]{2em}
\setstretch{.5}
{\PaliGlossB{They don’t describe it as physical and limited …}}\\
\end{addmargin}
\end{absolutelynopagebreak}

\begin{absolutelynopagebreak}
\setstretch{.7}
{\PaliGlossA{‘rūpī me paritto attā’ti.}}\\
\begin{addmargin}[1em]{2em}
\setstretch{.5}
{\PaliGlossB{    -}}\\
\end{addmargin}
\end{absolutelynopagebreak}

\begin{absolutelynopagebreak}
\setstretch{.7}
{\PaliGlossA{rūpiṃ vā hi, ānanda, anantaṃ attānaṃ na paññapento na paññapeti:}}\\
\begin{addmargin}[1em]{2em}
\setstretch{.5}
{\PaliGlossB{physical and infinite …}}\\
\end{addmargin}
\end{absolutelynopagebreak}

\begin{absolutelynopagebreak}
\setstretch{.7}
{\PaliGlossA{‘rūpī me ananto attā’ti.}}\\
\begin{addmargin}[1em]{2em}
\setstretch{.5}
{\PaliGlossB{    -}}\\
\end{addmargin}
\end{absolutelynopagebreak}

\begin{absolutelynopagebreak}
\setstretch{.7}
{\PaliGlossA{arūpiṃ vā hi, ānanda, parittaṃ attānaṃ na paññapento na paññapeti:}}\\
\begin{addmargin}[1em]{2em}
\setstretch{.5}
{\PaliGlossB{formless and limited …}}\\
\end{addmargin}
\end{absolutelynopagebreak}

\begin{absolutelynopagebreak}
\setstretch{.7}
{\PaliGlossA{‘arūpī me paritto attā’ti.}}\\
\begin{addmargin}[1em]{2em}
\setstretch{.5}
{\PaliGlossB{    -}}\\
\end{addmargin}
\end{absolutelynopagebreak}

\begin{absolutelynopagebreak}
\setstretch{.7}
{\PaliGlossA{arūpiṃ vā hi, ānanda, anantaṃ attānaṃ na paññapento na paññapeti:}}\\
\begin{addmargin}[1em]{2em}
\setstretch{.5}
{\PaliGlossB{formless and infinite:}}\\
\end{addmargin}
\end{absolutelynopagebreak}

\begin{absolutelynopagebreak}
\setstretch{.7}
{\PaliGlossA{‘arūpī me ananto attā’ti.}}\\
\begin{addmargin}[1em]{2em}
\setstretch{.5}
{\PaliGlossB{‘My self is formless and infinite.’}}\\
\end{addmargin}
\end{absolutelynopagebreak}

\begin{absolutelynopagebreak}
\setstretch{.7}
{\PaliGlossA{tatrānanda, yo so rūpiṃ parittaṃ attānaṃ na paññapento na paññapeti.}}\\
\begin{addmargin}[1em]{2em}
\setstretch{.5}
{\PaliGlossB{Now, take those who don’t describe the self as physical and limited …}}\\
\end{addmargin}
\end{absolutelynopagebreak}

\begin{absolutelynopagebreak}
\setstretch{.7}
{\PaliGlossA{etarahi vā so rūpiṃ parittaṃ attānaṃ na paññapento na paññapeti, tattha bhāviṃ vā so rūpiṃ parittaṃ attānaṃ na paññapento na paññapeti, ‘atathaṃ vā pana santaṃ tathattāya upakappessāmī’ti iti vā panassa na hoti.}}\\
\begin{addmargin}[1em]{2em}
\setstretch{.5}
{\PaliGlossB{    -}}\\
\end{addmargin}
\end{absolutelynopagebreak}

\begin{absolutelynopagebreak}
\setstretch{.7}
{\PaliGlossA{evaṃ santaṃ kho, ānanda, rūpiṃ parittattānudiṭṭhi nānusetīti iccālaṃ vacanāya.}}\\
\begin{addmargin}[1em]{2em}
\setstretch{.5}
{\PaliGlossB{    -}}\\
\end{addmargin}
\end{absolutelynopagebreak}

\begin{absolutelynopagebreak}
\setstretch{.7}
{\PaliGlossA{tatrānanda, yo so rūpiṃ anantaṃ attānaṃ na paññapento na paññapeti.}}\\
\begin{addmargin}[1em]{2em}
\setstretch{.5}
{\PaliGlossB{physical and infinite …}}\\
\end{addmargin}
\end{absolutelynopagebreak}

\begin{absolutelynopagebreak}
\setstretch{.7}
{\PaliGlossA{etarahi vā so rūpiṃ anantaṃ attānaṃ na paññapento na paññapeti, tattha bhāviṃ vā so rūpiṃ anantaṃ attānaṃ na paññapento na paññapeti, ‘atathaṃ vā pana santaṃ tathattāya upakappessāmī’ti iti vā panassa na hoti.}}\\
\begin{addmargin}[1em]{2em}
\setstretch{.5}
{\PaliGlossB{    -}}\\
\end{addmargin}
\end{absolutelynopagebreak}

\begin{absolutelynopagebreak}
\setstretch{.7}
{\PaliGlossA{evaṃ santaṃ kho, ānanda, rūpiṃ anantattānudiṭṭhi nānusetīti iccālaṃ vacanāya.}}\\
\begin{addmargin}[1em]{2em}
\setstretch{.5}
{\PaliGlossB{    -}}\\
\end{addmargin}
\end{absolutelynopagebreak}

\begin{absolutelynopagebreak}
\setstretch{.7}
{\PaliGlossA{tatrānanda, yo so arūpiṃ parittaṃ attānaṃ na paññapento na paññapeti.}}\\
\begin{addmargin}[1em]{2em}
\setstretch{.5}
{\PaliGlossB{formless and limited …}}\\
\end{addmargin}
\end{absolutelynopagebreak}

\begin{absolutelynopagebreak}
\setstretch{.7}
{\PaliGlossA{etarahi vā so arūpiṃ parittaṃ attānaṃ na paññapento na paññapeti, tattha bhāviṃ vā so arūpiṃ parittaṃ attānaṃ na paññapento na paññapeti, ‘atathaṃ vā pana santaṃ tathattāya upakappessāmī’ti iti vā panassa na hoti.}}\\
\begin{addmargin}[1em]{2em}
\setstretch{.5}
{\PaliGlossB{    -}}\\
\end{addmargin}
\end{absolutelynopagebreak}

\begin{absolutelynopagebreak}
\setstretch{.7}
{\PaliGlossA{evaṃ santaṃ kho, ānanda, arūpiṃ parittattānudiṭṭhi nānusetīti iccālaṃ vacanāya.}}\\
\begin{addmargin}[1em]{2em}
\setstretch{.5}
{\PaliGlossB{    -}}\\
\end{addmargin}
\end{absolutelynopagebreak}

\begin{absolutelynopagebreak}
\setstretch{.7}
{\PaliGlossA{tatrānanda, yo so arūpiṃ anantaṃ attānaṃ na paññapento na paññapeti.}}\\
\begin{addmargin}[1em]{2em}
\setstretch{.5}
{\PaliGlossB{formless and infinite.}}\\
\end{addmargin}
\end{absolutelynopagebreak}

\begin{absolutelynopagebreak}
\setstretch{.7}
{\PaliGlossA{etarahi vā so arūpiṃ anantaṃ attānaṃ na paññapento na paññapeti, tattha bhāviṃ vā so arūpiṃ anantaṃ attānaṃ na paññapento na paññapeti, ‘atathaṃ vā pana santaṃ tathattāya upakappessāmī’ti iti vā panassa na hoti.}}\\
\begin{addmargin}[1em]{2em}
\setstretch{.5}
{\PaliGlossB{They don’t describe the self as formless and infinite in the present; or in some future life; and they don’t think: ‘Though it is not like that, I will ensure it is provided with what it needs to become like that.’}}\\
\end{addmargin}
\end{absolutelynopagebreak}

\begin{absolutelynopagebreak}
\setstretch{.7}
{\PaliGlossA{evaṃ santaṃ kho, ānanda, arūpiṃ anantattānudiṭṭhi nānusetīti iccālaṃ vacanāya.}}\\
\begin{addmargin}[1em]{2em}
\setstretch{.5}
{\PaliGlossB{This being so, it’s appropriate to say that a view of self as formless and infinite doesn’t underlie them.}}\\
\end{addmargin}
\end{absolutelynopagebreak}

\begin{absolutelynopagebreak}
\setstretch{.7}
{\PaliGlossA{ettāvatā kho, ānanda, attānaṃ na paññapento na paññapeti.}}\\
\begin{addmargin}[1em]{2em}
\setstretch{.5}
{\PaliGlossB{That’s how those who don’t describe the self don’t describe it.}}\\
\end{addmargin}
\end{absolutelynopagebreak}

\begin{absolutelynopagebreak}
\setstretch{.7}
{\PaliGlossA{4. attasamanupassanā}}\\
\begin{addmargin}[1em]{2em}
\setstretch{.5}
{\PaliGlossB{4. Regarding a Self}}\\
\end{addmargin}
\end{absolutelynopagebreak}

\begin{absolutelynopagebreak}
\setstretch{.7}
{\PaliGlossA{kittāvatā ca, ānanda, attānaṃ samanupassamāno samanupassati?}}\\
\begin{addmargin}[1em]{2em}
\setstretch{.5}
{\PaliGlossB{How do those who regard the self regard it?}}\\
\end{addmargin}
\end{absolutelynopagebreak}

\begin{absolutelynopagebreak}
\setstretch{.7}
{\PaliGlossA{vedanaṃ vā hi, ānanda, attānaṃ samanupassamāno samanupassati:}}\\
\begin{addmargin}[1em]{2em}
\setstretch{.5}
{\PaliGlossB{They regard feeling as self:}}\\
\end{addmargin}
\end{absolutelynopagebreak}

\begin{absolutelynopagebreak}
\setstretch{.7}
{\PaliGlossA{‘vedanā me attā’ti.}}\\
\begin{addmargin}[1em]{2em}
\setstretch{.5}
{\PaliGlossB{‘Feeling is my self.’}}\\
\end{addmargin}
\end{absolutelynopagebreak}

\begin{absolutelynopagebreak}
\setstretch{.7}
{\PaliGlossA{‘na heva kho me vedanā attā, appaṭisaṃvedano me attā’ti iti vā hi, ānanda, attānaṃ samanupassamāno samanupassati.}}\\
\begin{addmargin}[1em]{2em}
\setstretch{.5}
{\PaliGlossB{Or they regard it like this: ‘Feeling is definitely not my self. My self does not experience feeling.’}}\\
\end{addmargin}
\end{absolutelynopagebreak}

\begin{absolutelynopagebreak}
\setstretch{.7}
{\PaliGlossA{‘na heva kho me vedanā attā, nopi appaṭisaṃvedano me attā, attā me vediyati, vedanādhammo hi me attā’ti iti vā hi, ānanda, attānaṃ samanupassamāno samanupassati.}}\\
\begin{addmargin}[1em]{2em}
\setstretch{.5}
{\PaliGlossB{Or they regard it like this: ‘Feeling is definitely not my self. But it’s not that my self does not experience feeling. My self feels, for my self is liable to feel.’}}\\
\end{addmargin}
\end{absolutelynopagebreak}

\begin{absolutelynopagebreak}
\setstretch{.7}
{\PaliGlossA{tatrānanda, yo so evamāha:}}\\
\begin{addmargin}[1em]{2em}
\setstretch{.5}
{\PaliGlossB{Now, as to those who say:}}\\
\end{addmargin}
\end{absolutelynopagebreak}

\begin{absolutelynopagebreak}
\setstretch{.7}
{\PaliGlossA{‘vedanā me attā’ti,}}\\
\begin{addmargin}[1em]{2em}
\setstretch{.5}
{\PaliGlossB{‘Feeling is my self.’}}\\
\end{addmargin}
\end{absolutelynopagebreak}

\begin{absolutelynopagebreak}
\setstretch{.7}
{\PaliGlossA{so evamassa vacanīyo:}}\\
\begin{addmargin}[1em]{2em}
\setstretch{.5}
{\PaliGlossB{You should say this to them:}}\\
\end{addmargin}
\end{absolutelynopagebreak}

\begin{absolutelynopagebreak}
\setstretch{.7}
{\PaliGlossA{‘tisso kho imā, āvuso, vedanā—}}\\
\begin{addmargin}[1em]{2em}
\setstretch{.5}
{\PaliGlossB{‘Reverend, there are three feelings:}}\\
\end{addmargin}
\end{absolutelynopagebreak}

\begin{absolutelynopagebreak}
\setstretch{.7}
{\PaliGlossA{sukhā vedanā dukkhā vedanā adukkhamasukhā vedanā.}}\\
\begin{addmargin}[1em]{2em}
\setstretch{.5}
{\PaliGlossB{pleasant, painful, and neutral.}}\\
\end{addmargin}
\end{absolutelynopagebreak}

\begin{absolutelynopagebreak}
\setstretch{.7}
{\PaliGlossA{imāsaṃ kho tvaṃ tissannaṃ vedanānaṃ katamaṃ attato samanupassasī’ti?}}\\
\begin{addmargin}[1em]{2em}
\setstretch{.5}
{\PaliGlossB{Which one of these do you regard as self?’}}\\
\end{addmargin}
\end{absolutelynopagebreak}

\begin{absolutelynopagebreak}
\setstretch{.7}
{\PaliGlossA{yasmiṃ, ānanda, samaye sukhaṃ vedanaṃ vedeti, neva tasmiṃ samaye dukkhaṃ vedanaṃ vedeti, na adukkhamasukhaṃ vedanaṃ vedeti;}}\\
\begin{addmargin}[1em]{2em}
\setstretch{.5}
{\PaliGlossB{Ānanda, at a time when you feel a pleasant feeling, you don’t feel a painful or neutral feeling;}}\\
\end{addmargin}
\end{absolutelynopagebreak}

\begin{absolutelynopagebreak}
\setstretch{.7}
{\PaliGlossA{sukhaṃyeva tasmiṃ samaye vedanaṃ vedeti.}}\\
\begin{addmargin}[1em]{2em}
\setstretch{.5}
{\PaliGlossB{you only feel a pleasant feeling.}}\\
\end{addmargin}
\end{absolutelynopagebreak}

\begin{absolutelynopagebreak}
\setstretch{.7}
{\PaliGlossA{yasmiṃ, ānanda, samaye dukkhaṃ vedanaṃ vedeti, neva tasmiṃ samaye sukhaṃ vedanaṃ vedeti, na adukkhamasukhaṃ vedanaṃ vedeti;}}\\
\begin{addmargin}[1em]{2em}
\setstretch{.5}
{\PaliGlossB{At a time when you feel a painful feeling, you don’t feel a pleasant or neutral feeling;}}\\
\end{addmargin}
\end{absolutelynopagebreak}

\begin{absolutelynopagebreak}
\setstretch{.7}
{\PaliGlossA{dukkhaṃyeva tasmiṃ samaye vedanaṃ vedeti.}}\\
\begin{addmargin}[1em]{2em}
\setstretch{.5}
{\PaliGlossB{you only feel a painful feeling.}}\\
\end{addmargin}
\end{absolutelynopagebreak}

\begin{absolutelynopagebreak}
\setstretch{.7}
{\PaliGlossA{yasmiṃ, ānanda, samaye adukkhamasukhaṃ vedanaṃ vedeti, neva tasmiṃ samaye sukhaṃ vedanaṃ vedeti, na dukkhaṃ vedanaṃ vedeti;}}\\
\begin{addmargin}[1em]{2em}
\setstretch{.5}
{\PaliGlossB{At a time when you feel a neutral feeling, you don’t feel a pleasant or painful feeling;}}\\
\end{addmargin}
\end{absolutelynopagebreak}

\begin{absolutelynopagebreak}
\setstretch{.7}
{\PaliGlossA{adukkhamasukhaṃyeva tasmiṃ samaye vedanaṃ vedeti.}}\\
\begin{addmargin}[1em]{2em}
\setstretch{.5}
{\PaliGlossB{you only feel a neutral feeling.}}\\
\end{addmargin}
\end{absolutelynopagebreak}

\begin{absolutelynopagebreak}
\setstretch{.7}
{\PaliGlossA{sukhāpi kho, ānanda, vedanā aniccā saṅkhatā paṭiccasamuppannā khayadhammā vayadhammā virāgadhammā nirodhadhammā.}}\\
\begin{addmargin}[1em]{2em}
\setstretch{.5}
{\PaliGlossB{Pleasant feelings, painful feelings, and neutral feelings are all impermanent, conditioned, dependently originated, liable to end, vanish, fade away, and cease.}}\\
\end{addmargin}
\end{absolutelynopagebreak}

\begin{absolutelynopagebreak}
\setstretch{.7}
{\PaliGlossA{dukkhāpi kho, ānanda, vedanā aniccā saṅkhatā paṭiccasamuppannā khayadhammā vayadhammā virāgadhammā nirodhadhammā.}}\\
\begin{addmargin}[1em]{2em}
\setstretch{.5}
{\PaliGlossB{    -}}\\
\end{addmargin}
\end{absolutelynopagebreak}

\begin{absolutelynopagebreak}
\setstretch{.7}
{\PaliGlossA{adukkhamasukhāpi kho, ānanda, vedanā aniccā saṅkhatā paṭiccasamuppannā khayadhammā vayadhammā virāgadhammā nirodhadhammā.}}\\
\begin{addmargin}[1em]{2em}
\setstretch{.5}
{\PaliGlossB{    -}}\\
\end{addmargin}
\end{absolutelynopagebreak}

\begin{absolutelynopagebreak}
\setstretch{.7}
{\PaliGlossA{tassa sukhaṃ vedanaṃ vediyamānassa ‘eso me attā’ti hoti.}}\\
\begin{addmargin}[1em]{2em}
\setstretch{.5}
{\PaliGlossB{When feeling a pleasant feeling they think: ‘This is my self.’}}\\
\end{addmargin}
\end{absolutelynopagebreak}

\begin{absolutelynopagebreak}
\setstretch{.7}
{\PaliGlossA{tassāyeva sukhāya vedanāya nirodhā ‘byagā me attā’ti hoti.}}\\
\begin{addmargin}[1em]{2em}
\setstretch{.5}
{\PaliGlossB{When their pleasant feeling ceases they think: ‘My self has disappeared.’}}\\
\end{addmargin}
\end{absolutelynopagebreak}

\begin{absolutelynopagebreak}
\setstretch{.7}
{\PaliGlossA{dukkhaṃ vedanaṃ vediyamānassa ‘eso me attā’ti hoti.}}\\
\begin{addmargin}[1em]{2em}
\setstretch{.5}
{\PaliGlossB{When feeling a painful feeling they think: ‘This is my self.’}}\\
\end{addmargin}
\end{absolutelynopagebreak}

\begin{absolutelynopagebreak}
\setstretch{.7}
{\PaliGlossA{tassāyeva dukkhāya vedanāya nirodhā ‘byagā me attā’ti hoti.}}\\
\begin{addmargin}[1em]{2em}
\setstretch{.5}
{\PaliGlossB{When their painful feeling ceases they think: ‘My self has disappeared.’}}\\
\end{addmargin}
\end{absolutelynopagebreak}

\begin{absolutelynopagebreak}
\setstretch{.7}
{\PaliGlossA{adukkhamasukhaṃ vedanaṃ vediyamānassa ‘eso me attā’ti hoti.}}\\
\begin{addmargin}[1em]{2em}
\setstretch{.5}
{\PaliGlossB{When feeling a neutral feeling they think: ‘This is my self.’}}\\
\end{addmargin}
\end{absolutelynopagebreak}

\begin{absolutelynopagebreak}
\setstretch{.7}
{\PaliGlossA{tassāyeva adukkhamasukhāya vedanāya nirodhā ‘byagā me attā’ti hoti.}}\\
\begin{addmargin}[1em]{2em}
\setstretch{.5}
{\PaliGlossB{When their neutral feeling ceases they think: ‘My self has disappeared.’}}\\
\end{addmargin}
\end{absolutelynopagebreak}

\begin{absolutelynopagebreak}
\setstretch{.7}
{\PaliGlossA{iti so diṭṭheva dhamme aniccasukhadukkhavokiṇṇaṃ uppādavayadhammaṃ attānaṃ samanupassamāno samanupassati, yo so evamāha: ‘vedanā me attā’ti.}}\\
\begin{addmargin}[1em]{2em}
\setstretch{.5}
{\PaliGlossB{So those who say ‘feeling is my self’ regard as self that which is evidently impermanent, a mixture of pleasure and pain, and liable to rise and fall.}}\\
\end{addmargin}
\end{absolutelynopagebreak}

\begin{absolutelynopagebreak}
\setstretch{.7}
{\PaliGlossA{tasmātihānanda, etena petaṃ nakkhamati: ‘vedanā me attā’ti samanupassituṃ.}}\\
\begin{addmargin}[1em]{2em}
\setstretch{.5}
{\PaliGlossB{That’s why it’s not acceptable to regard feeling as self.}}\\
\end{addmargin}
\end{absolutelynopagebreak}

\begin{absolutelynopagebreak}
\setstretch{.7}
{\PaliGlossA{tatrānanda, yo so evamāha:}}\\
\begin{addmargin}[1em]{2em}
\setstretch{.5}
{\PaliGlossB{Now, as to those who say:}}\\
\end{addmargin}
\end{absolutelynopagebreak}

\begin{absolutelynopagebreak}
\setstretch{.7}
{\PaliGlossA{‘na heva kho me vedanā attā, appaṭisaṃvedano me attā’ti, so evamassa vacanīyo:}}\\
\begin{addmargin}[1em]{2em}
\setstretch{.5}
{\PaliGlossB{‘Feeling is definitely not my self. My self does not experience feeling.’ You should say this to them,}}\\
\end{addmargin}
\end{absolutelynopagebreak}

\begin{absolutelynopagebreak}
\setstretch{.7}
{\PaliGlossA{‘yattha panāvuso, sabbaso vedayitaṃ natthi api nu kho, tattha “ayamahamasmī”ti siyā’”ti?}}\\
\begin{addmargin}[1em]{2em}
\setstretch{.5}
{\PaliGlossB{‘But reverend, where there is nothing felt at all, would the thought “I am” occur there?’”}}\\
\end{addmargin}
\end{absolutelynopagebreak}

\begin{absolutelynopagebreak}
\setstretch{.7}
{\PaliGlossA{“no hetaṃ, bhante”.}}\\
\begin{addmargin}[1em]{2em}
\setstretch{.5}
{\PaliGlossB{“No, sir.”}}\\
\end{addmargin}
\end{absolutelynopagebreak}

\begin{absolutelynopagebreak}
\setstretch{.7}
{\PaliGlossA{“tasmātihānanda, etena petaṃ nakkhamati: ‘na heva kho me vedanā attā, appaṭisaṃvedano me attā’ti samanupassituṃ.}}\\
\begin{addmargin}[1em]{2em}
\setstretch{.5}
{\PaliGlossB{“That’s why it’s not acceptable to regard self as that which does not experience feeling.}}\\
\end{addmargin}
\end{absolutelynopagebreak}

\begin{absolutelynopagebreak}
\setstretch{.7}
{\PaliGlossA{tatrānanda, yo so evamāha:}}\\
\begin{addmargin}[1em]{2em}
\setstretch{.5}
{\PaliGlossB{Now, as to those who say:}}\\
\end{addmargin}
\end{absolutelynopagebreak}

\begin{absolutelynopagebreak}
\setstretch{.7}
{\PaliGlossA{‘na heva kho me vedanā attā, nopi appaṭisaṃvedano me attā, attā me vediyati, vedanādhammo hi me attā’ti.}}\\
\begin{addmargin}[1em]{2em}
\setstretch{.5}
{\PaliGlossB{‘Feeling is definitely not my self. But it’s not that my self does not experience feeling. My self feels, for my self is liable to feel.’}}\\
\end{addmargin}
\end{absolutelynopagebreak}

\begin{absolutelynopagebreak}
\setstretch{.7}
{\PaliGlossA{so evamassa vacanīyo—}}\\
\begin{addmargin}[1em]{2em}
\setstretch{.5}
{\PaliGlossB{You should say this to them,}}\\
\end{addmargin}
\end{absolutelynopagebreak}

\begin{absolutelynopagebreak}
\setstretch{.7}
{\PaliGlossA{vedanā ca hi, āvuso, sabbena sabbaṃ sabbathā sabbaṃ aparisesā nirujjheyyuṃ.}}\\
\begin{addmargin}[1em]{2em}
\setstretch{.5}
{\PaliGlossB{‘Suppose feelings were to totally and utterly cease without anything left over.}}\\
\end{addmargin}
\end{absolutelynopagebreak}

\begin{absolutelynopagebreak}
\setstretch{.7}
{\PaliGlossA{sabbaso vedanāya asati vedanānirodhā api nu kho tattha ‘ayamahamasmī’ti siyā”ti?}}\\
\begin{addmargin}[1em]{2em}
\setstretch{.5}
{\PaliGlossB{When there’s no feeling at all, with the cessation of feeling, would the thought “I am this” occur there?’”}}\\
\end{addmargin}
\end{absolutelynopagebreak}

\begin{absolutelynopagebreak}
\setstretch{.7}
{\PaliGlossA{“no hetaṃ, bhante”.}}\\
\begin{addmargin}[1em]{2em}
\setstretch{.5}
{\PaliGlossB{“No, sir.”}}\\
\end{addmargin}
\end{absolutelynopagebreak}

\begin{absolutelynopagebreak}
\setstretch{.7}
{\PaliGlossA{“tasmātihānanda, etena petaṃ nakkhamati: ‘na heva kho me vedanā attā, nopi appaṭisaṃvedano me attā, attā me vediyati, vedanādhammo hi me attā’ti samanupassituṃ.}}\\
\begin{addmargin}[1em]{2em}
\setstretch{.5}
{\PaliGlossB{“That’s why it’s not acceptable to regard self as that which is liable to feel.}}\\
\end{addmargin}
\end{absolutelynopagebreak}

\begin{absolutelynopagebreak}
\setstretch{.7}
{\PaliGlossA{yato kho, ānanda, bhikkhu neva vedanaṃ attānaṃ samanupassati, nopi appaṭisaṃvedanaṃ attānaṃ samanupassati, nopi ‘attā me vediyati, vedanādhammo hi me attā’ti samanupassati.}}\\
\begin{addmargin}[1em]{2em}
\setstretch{.5}
{\PaliGlossB{    -}}\\
\end{addmargin}
\end{absolutelynopagebreak}

\begin{absolutelynopagebreak}
\setstretch{.7}
{\PaliGlossA{so evaṃ na samanupassanto na ca kiñci loke upādiyati,}}\\
\begin{addmargin}[1em]{2em}
\setstretch{.5}
{\PaliGlossB{Not regarding anything in this way, they don’t grasp at anything in the world.}}\\
\end{addmargin}
\end{absolutelynopagebreak}

\begin{absolutelynopagebreak}
\setstretch{.7}
{\PaliGlossA{anupādiyaṃ na paritassati, aparitassaṃ paccattaññeva parinibbāyati,}}\\
\begin{addmargin}[1em]{2em}
\setstretch{.5}
{\PaliGlossB{Not grasping, they’re not anxious. Not being anxious, they personally become extinguished.}}\\
\end{addmargin}
\end{absolutelynopagebreak}

\begin{absolutelynopagebreak}
\setstretch{.7}
{\PaliGlossA{‘khīṇā jāti, vusitaṃ brahmacariyaṃ, kataṃ karaṇīyaṃ, nāparaṃ itthattāyā’ti pajānāti.}}\\
\begin{addmargin}[1em]{2em}
\setstretch{.5}
{\PaliGlossB{They understand: ‘Rebirth is ended, the spiritual journey has been completed, what had to be done has been done, there is no return to any state of existence.’}}\\
\end{addmargin}
\end{absolutelynopagebreak}

\begin{absolutelynopagebreak}
\setstretch{.7}
{\PaliGlossA{evaṃ vimuttacittaṃ kho, ānanda, bhikkhuṃ yo evaṃ vadeyya:}}\\
\begin{addmargin}[1em]{2em}
\setstretch{.5}
{\PaliGlossB{It wouldn’t be appropriate to say that a mendicant whose mind is freed like this holds the following views:}}\\
\end{addmargin}
\end{absolutelynopagebreak}

\begin{absolutelynopagebreak}
\setstretch{.7}
{\PaliGlossA{‘hoti tathāgato paraṃ maraṇā itissa diṭṭhī’ti, tadakallaṃ.}}\\
\begin{addmargin}[1em]{2em}
\setstretch{.5}
{\PaliGlossB{‘A Realized One exists after death’;}}\\
\end{addmargin}
\end{absolutelynopagebreak}

\begin{absolutelynopagebreak}
\setstretch{.7}
{\PaliGlossA{‘na hoti tathāgato paraṃ maraṇā itissa diṭṭhī’ti, tadakallaṃ.}}\\
\begin{addmargin}[1em]{2em}
\setstretch{.5}
{\PaliGlossB{‘A Realized One doesn’t exist after death’;}}\\
\end{addmargin}
\end{absolutelynopagebreak}

\begin{absolutelynopagebreak}
\setstretch{.7}
{\PaliGlossA{‘hoti ca na ca hoti tathāgato paraṃ maraṇā itissa diṭṭhī’ti, tadakallaṃ.}}\\
\begin{addmargin}[1em]{2em}
\setstretch{.5}
{\PaliGlossB{‘A Realized One both exists and doesn’t exist after death’;}}\\
\end{addmargin}
\end{absolutelynopagebreak}

\begin{absolutelynopagebreak}
\setstretch{.7}
{\PaliGlossA{‘neva hoti na na hoti tathāgato paraṃ maraṇā itissa diṭṭhī’ti, tadakallaṃ.}}\\
\begin{addmargin}[1em]{2em}
\setstretch{.5}
{\PaliGlossB{‘A Realized One neither exists nor doesn’t exist after death’.}}\\
\end{addmargin}
\end{absolutelynopagebreak}

\begin{absolutelynopagebreak}
\setstretch{.7}
{\PaliGlossA{taṃ kissa hetu?}}\\
\begin{addmargin}[1em]{2em}
\setstretch{.5}
{\PaliGlossB{Why is that?}}\\
\end{addmargin}
\end{absolutelynopagebreak}

\begin{absolutelynopagebreak}
\setstretch{.7}
{\PaliGlossA{yāvatā, ānanda, adhivacanaṃ yāvatā adhivacanapatho, yāvatā nirutti yāvatā niruttipatho, yāvatā paññatti yāvatā paññattipatho, yāvatā paññā yāvatā paññāvacaraṃ, yāvatā vaṭṭaṃ, yāvatā vaṭṭati, tadabhiññāvimutto bhikkhu, tadabhiññāvimuttaṃ bhikkhuṃ ‘na jānāti na passati itissa diṭṭhī’ti, tadakallaṃ.}}\\
\begin{addmargin}[1em]{2em}
\setstretch{.5}
{\PaliGlossB{A mendicant is freed by directly knowing this: how far language and the scope of language extend; how far terminology and the scope of terminology extend; how far description and the scope of description extend; how far wisdom and the sphere of wisdom extend; how far the cycle of rebirths and its continuation extend. It wouldn’t be appropriate to say that a mendicant freed by directly knowing this holds the view: ‘There is no such thing as knowing and seeing.’}}\\
\end{addmargin}
\end{absolutelynopagebreak}

\begin{absolutelynopagebreak}
\setstretch{.7}
{\PaliGlossA{5. sattaviññāṇaṭṭhiti}}\\
\begin{addmargin}[1em]{2em}
\setstretch{.5}
{\PaliGlossB{5. Planes of Consciousness}}\\
\end{addmargin}
\end{absolutelynopagebreak}

\begin{absolutelynopagebreak}
\setstretch{.7}
{\PaliGlossA{satta kho, ānanda, viññāṇaṭṭhitiyo, dve āyatanāni.}}\\
\begin{addmargin}[1em]{2em}
\setstretch{.5}
{\PaliGlossB{Ānanda, there are seven planes of consciousness and two dimensions.}}\\
\end{addmargin}
\end{absolutelynopagebreak}

\begin{absolutelynopagebreak}
\setstretch{.7}
{\PaliGlossA{katamā satta?}}\\
\begin{addmargin}[1em]{2em}
\setstretch{.5}
{\PaliGlossB{What seven?}}\\
\end{addmargin}
\end{absolutelynopagebreak}

\begin{absolutelynopagebreak}
\setstretch{.7}
{\PaliGlossA{santānanda, sattā nānattakāyā nānattasaññino, seyyathāpi manussā, ekacce ca devā, ekacce ca vinipātikā.}}\\
\begin{addmargin}[1em]{2em}
\setstretch{.5}
{\PaliGlossB{There are sentient beings that are diverse in body and diverse in perception, such as human beings, some gods, and some beings in the underworld.}}\\
\end{addmargin}
\end{absolutelynopagebreak}

\begin{absolutelynopagebreak}
\setstretch{.7}
{\PaliGlossA{ayaṃ paṭhamā viññāṇaṭṭhiti.}}\\
\begin{addmargin}[1em]{2em}
\setstretch{.5}
{\PaliGlossB{This is the first plane of consciousness.}}\\
\end{addmargin}
\end{absolutelynopagebreak}

\begin{absolutelynopagebreak}
\setstretch{.7}
{\PaliGlossA{santānanda, sattā nānattakāyā ekattasaññino, seyyathāpi devā brahmakāyikā paṭhamābhinibbattā.}}\\
\begin{addmargin}[1em]{2em}
\setstretch{.5}
{\PaliGlossB{There are sentient beings that are diverse in body and unified in perception, such as the gods reborn in Brahmā’s Host through the first absorption.}}\\
\end{addmargin}
\end{absolutelynopagebreak}

\begin{absolutelynopagebreak}
\setstretch{.7}
{\PaliGlossA{ayaṃ dutiyā viññāṇaṭṭhiti.}}\\
\begin{addmargin}[1em]{2em}
\setstretch{.5}
{\PaliGlossB{This is the second plane of consciousness.}}\\
\end{addmargin}
\end{absolutelynopagebreak}

\begin{absolutelynopagebreak}
\setstretch{.7}
{\PaliGlossA{santānanda, sattā ekattakāyā nānattasaññino, seyyathāpi devā ābhassarā.}}\\
\begin{addmargin}[1em]{2em}
\setstretch{.5}
{\PaliGlossB{There are sentient beings that are unified in body and diverse in perception, such as the gods of streaming radiance.}}\\
\end{addmargin}
\end{absolutelynopagebreak}

\begin{absolutelynopagebreak}
\setstretch{.7}
{\PaliGlossA{ayaṃ tatiyā viññāṇaṭṭhiti.}}\\
\begin{addmargin}[1em]{2em}
\setstretch{.5}
{\PaliGlossB{This is the third plane of consciousness.}}\\
\end{addmargin}
\end{absolutelynopagebreak}

\begin{absolutelynopagebreak}
\setstretch{.7}
{\PaliGlossA{santānanda, sattā ekattakāyā ekattasaññino, seyyathāpi devā subhakiṇhā.}}\\
\begin{addmargin}[1em]{2em}
\setstretch{.5}
{\PaliGlossB{There are sentient beings that are unified in body and unified in perception, such as the gods replete with glory.}}\\
\end{addmargin}
\end{absolutelynopagebreak}

\begin{absolutelynopagebreak}
\setstretch{.7}
{\PaliGlossA{ayaṃ catutthī viññāṇaṭṭhiti.}}\\
\begin{addmargin}[1em]{2em}
\setstretch{.5}
{\PaliGlossB{This is the fourth plane of consciousness.}}\\
\end{addmargin}
\end{absolutelynopagebreak}

\begin{absolutelynopagebreak}
\setstretch{.7}
{\PaliGlossA{santānanda, sattā sabbaso rūpasaññānaṃ samatikkamā paṭighasaññānaṃ atthaṅgamā nānattasaññānaṃ amanasikārā ‘ananto ākāso’ti ākāsānañcāyatanūpagā.}}\\
\begin{addmargin}[1em]{2em}
\setstretch{.5}
{\PaliGlossB{There are sentient beings that have gone totally beyond perceptions of form. With the ending of perceptions of impingement, not focusing on perceptions of diversity, aware that ‘space is infinite’, they have been reborn in the dimension of infinite space.}}\\
\end{addmargin}
\end{absolutelynopagebreak}

\begin{absolutelynopagebreak}
\setstretch{.7}
{\PaliGlossA{ayaṃ pañcamī viññāṇaṭṭhiti.}}\\
\begin{addmargin}[1em]{2em}
\setstretch{.5}
{\PaliGlossB{This is the fifth plane of consciousness.}}\\
\end{addmargin}
\end{absolutelynopagebreak}

\begin{absolutelynopagebreak}
\setstretch{.7}
{\PaliGlossA{santānanda, sattā sabbaso ākāsānañcāyatanaṃ samatikkamma ‘anantaṃ viññāṇan’ti viññāṇañcāyatanūpagā.}}\\
\begin{addmargin}[1em]{2em}
\setstretch{.5}
{\PaliGlossB{There are sentient beings that have gone totally beyond the dimension of infinite space. Aware that ‘consciousness is infinite’, they have been reborn in the dimension of infinite consciousness.}}\\
\end{addmargin}
\end{absolutelynopagebreak}

\begin{absolutelynopagebreak}
\setstretch{.7}
{\PaliGlossA{ayaṃ chaṭṭhī viññāṇaṭṭhiti.}}\\
\begin{addmargin}[1em]{2em}
\setstretch{.5}
{\PaliGlossB{This is the sixth plane of consciousness.}}\\
\end{addmargin}
\end{absolutelynopagebreak}

\begin{absolutelynopagebreak}
\setstretch{.7}
{\PaliGlossA{santānanda, sattā sabbaso viññāṇañcāyatanaṃ samatikkamma ‘natthi kiñcī’ti ākiñcaññāyatanūpagā.}}\\
\begin{addmargin}[1em]{2em}
\setstretch{.5}
{\PaliGlossB{There are sentient beings that have gone totally beyond the dimension of infinite consciousness. Aware that ‘there is nothing at all’, they have been reborn in the dimension of nothingness.}}\\
\end{addmargin}
\end{absolutelynopagebreak}

\begin{absolutelynopagebreak}
\setstretch{.7}
{\PaliGlossA{ayaṃ sattamī viññāṇaṭṭhiti.}}\\
\begin{addmargin}[1em]{2em}
\setstretch{.5}
{\PaliGlossB{This is the seventh plane of consciousness.}}\\
\end{addmargin}
\end{absolutelynopagebreak}

\begin{absolutelynopagebreak}
\setstretch{.7}
{\PaliGlossA{asaññasattāyatanaṃ nevasaññānāsaññāyatanameva dutiyaṃ.}}\\
\begin{addmargin}[1em]{2em}
\setstretch{.5}
{\PaliGlossB{Then there’s the dimension of non-percipient beings, and secondly, the dimension of neither perception nor non-perception.}}\\
\end{addmargin}
\end{absolutelynopagebreak}

\begin{absolutelynopagebreak}
\setstretch{.7}
{\PaliGlossA{tatrānanda, yāyaṃ paṭhamā viññāṇaṭṭhiti nānattakāyā nānattasaññino, seyyathāpi manussā, ekacce ca devā, ekacce ca vinipātikā.}}\\
\begin{addmargin}[1em]{2em}
\setstretch{.5}
{\PaliGlossB{Now, regarding these seven planes of consciousness and two dimensions,}}\\
\end{addmargin}
\end{absolutelynopagebreak}

\begin{absolutelynopagebreak}
\setstretch{.7}
{\PaliGlossA{yo nu kho, ānanda, tañca pajānāti, tassā ca samudayaṃ pajānāti, tassā ca atthaṅgamaṃ pajānāti, tassā ca assādaṃ pajānāti, tassā ca ādīnavaṃ pajānāti, tassā ca nissaraṇaṃ pajānāti, kallaṃ nu tena tadabhinanditun”ti?}}\\
\begin{addmargin}[1em]{2em}
\setstretch{.5}
{\PaliGlossB{is it appropriate for someone who understands them—and their origin, ending, gratification, drawback, and escape—to take pleasure in them?”}}\\
\end{addmargin}
\end{absolutelynopagebreak}

\begin{absolutelynopagebreak}
\setstretch{.7}
{\PaliGlossA{“no hetaṃ, bhante” … pe …}}\\
\begin{addmargin}[1em]{2em}
\setstretch{.5}
{\PaliGlossB{“No, sir.”}}\\
\end{addmargin}
\end{absolutelynopagebreak}

\begin{absolutelynopagebreak}
\setstretch{.7}
{\PaliGlossA{“tatrānanda, yamidaṃ asaññasattāyatanaṃ.}}\\
\begin{addmargin}[1em]{2em}
\setstretch{.5}
{\PaliGlossB{    -}}\\
\end{addmargin}
\end{absolutelynopagebreak}

\begin{absolutelynopagebreak}
\setstretch{.7}
{\PaliGlossA{yo nu kho, ānanda, tañca pajānāti, tassa ca samudayaṃ pajānāti, tassa ca atthaṅgamaṃ pajānāti, tassa ca assādaṃ pajānāti, tassa ca ādīnavaṃ pajānāti, tassa ca nissaraṇaṃ pajānāti, kallaṃ nu tena tadabhinanditun”ti?}}\\
\begin{addmargin}[1em]{2em}
\setstretch{.5}
{\PaliGlossB{    -}}\\
\end{addmargin}
\end{absolutelynopagebreak}

\begin{absolutelynopagebreak}
\setstretch{.7}
{\PaliGlossA{“no hetaṃ, bhante”.}}\\
\begin{addmargin}[1em]{2em}
\setstretch{.5}
{\PaliGlossB{    -}}\\
\end{addmargin}
\end{absolutelynopagebreak}

\begin{absolutelynopagebreak}
\setstretch{.7}
{\PaliGlossA{“tatrānanda, yamidaṃ nevasaññānāsaññāyatanaṃ.}}\\
\begin{addmargin}[1em]{2em}
\setstretch{.5}
{\PaliGlossB{    -}}\\
\end{addmargin}
\end{absolutelynopagebreak}

\begin{absolutelynopagebreak}
\setstretch{.7}
{\PaliGlossA{yo nu kho, ānanda, tañca pajānāti, tassa ca samudayaṃ pajānāti, tassa ca atthaṅgamaṃ pajānāti, tassa ca assādaṃ pajānāti, tassa ca ādīnavaṃ pajānāti, tassa ca nissaraṇaṃ pajānāti, kallaṃ nu tena tadabhinanditun”ti?}}\\
\begin{addmargin}[1em]{2em}
\setstretch{.5}
{\PaliGlossB{    -}}\\
\end{addmargin}
\end{absolutelynopagebreak}

\begin{absolutelynopagebreak}
\setstretch{.7}
{\PaliGlossA{“no hetaṃ, bhante”.}}\\
\begin{addmargin}[1em]{2em}
\setstretch{.5}
{\PaliGlossB{    -}}\\
\end{addmargin}
\end{absolutelynopagebreak}

\begin{absolutelynopagebreak}
\setstretch{.7}
{\PaliGlossA{“yato kho, ānanda, bhikkhu imāsañca sattannaṃ viññāṇaṭṭhitīnaṃ imesañca dvinnaṃ āyatanānaṃ samudayañca atthaṅgamañca assādañca ādīnavañca nissaraṇañca yathābhūtaṃ viditvā anupādā vimutto hoti, ayaṃ vuccatānanda, bhikkhu paññāvimutto.}}\\
\begin{addmargin}[1em]{2em}
\setstretch{.5}
{\PaliGlossB{“When a mendicant, having truly understood the origin, ending, gratification, drawback, and escape regarding these seven planes of consciousness and these two dimensions, is freed by not grasping, they’re called a mendicant who is freed by wisdom.}}\\
\end{addmargin}
\end{absolutelynopagebreak}

\begin{absolutelynopagebreak}
\setstretch{.7}
{\PaliGlossA{6. aṭṭhavimokkha}}\\
\begin{addmargin}[1em]{2em}
\setstretch{.5}
{\PaliGlossB{6. The Eight Liberations}}\\
\end{addmargin}
\end{absolutelynopagebreak}

\begin{absolutelynopagebreak}
\setstretch{.7}
{\PaliGlossA{aṭṭha kho ime, ānanda, vimokkhā.}}\\
\begin{addmargin}[1em]{2em}
\setstretch{.5}
{\PaliGlossB{Ānanda, there are these eight liberations.}}\\
\end{addmargin}
\end{absolutelynopagebreak}

\begin{absolutelynopagebreak}
\setstretch{.7}
{\PaliGlossA{katame aṭṭha?}}\\
\begin{addmargin}[1em]{2em}
\setstretch{.5}
{\PaliGlossB{What eight?}}\\
\end{addmargin}
\end{absolutelynopagebreak}

\begin{absolutelynopagebreak}
\setstretch{.7}
{\PaliGlossA{rūpī rūpāni passati}}\\
\begin{addmargin}[1em]{2em}
\setstretch{.5}
{\PaliGlossB{Having physical form, they see visions.}}\\
\end{addmargin}
\end{absolutelynopagebreak}

\begin{absolutelynopagebreak}
\setstretch{.7}
{\PaliGlossA{ayaṃ paṭhamo vimokkho.}}\\
\begin{addmargin}[1em]{2em}
\setstretch{.5}
{\PaliGlossB{This is the first liberation.}}\\
\end{addmargin}
\end{absolutelynopagebreak}

\begin{absolutelynopagebreak}
\setstretch{.7}
{\PaliGlossA{ajjhattaṃ arūpasaññī bahiddhā rūpāni passati,}}\\
\begin{addmargin}[1em]{2em}
\setstretch{.5}
{\PaliGlossB{Not perceiving form internally, they see visions externally.}}\\
\end{addmargin}
\end{absolutelynopagebreak}

\begin{absolutelynopagebreak}
\setstretch{.7}
{\PaliGlossA{ayaṃ dutiyo vimokkho.}}\\
\begin{addmargin}[1em]{2em}
\setstretch{.5}
{\PaliGlossB{This is the second liberation.}}\\
\end{addmargin}
\end{absolutelynopagebreak}

\begin{absolutelynopagebreak}
\setstretch{.7}
{\PaliGlossA{subhanteva adhimutto hoti,}}\\
\begin{addmargin}[1em]{2em}
\setstretch{.5}
{\PaliGlossB{They’re focused only on beauty.}}\\
\end{addmargin}
\end{absolutelynopagebreak}

\begin{absolutelynopagebreak}
\setstretch{.7}
{\PaliGlossA{ayaṃ tatiyo vimokkho.}}\\
\begin{addmargin}[1em]{2em}
\setstretch{.5}
{\PaliGlossB{This is the third liberation.}}\\
\end{addmargin}
\end{absolutelynopagebreak}

\begin{absolutelynopagebreak}
\setstretch{.7}
{\PaliGlossA{sabbaso rūpasaññānaṃ samatikkamā paṭighasaññānaṃ atthaṅgamā nānattasaññānaṃ amanasikārā ‘ananto ākāso’ti ākāsānañcāyatanaṃ upasampajja viharati,}}\\
\begin{addmargin}[1em]{2em}
\setstretch{.5}
{\PaliGlossB{Going totally beyond perceptions of form, with the ending of perceptions of impingement, not focusing on perceptions of diversity, aware that ‘space is infinite’, they enter and remain in the dimension of infinite space.}}\\
\end{addmargin}
\end{absolutelynopagebreak}

\begin{absolutelynopagebreak}
\setstretch{.7}
{\PaliGlossA{ayaṃ catuttho vimokkho.}}\\
\begin{addmargin}[1em]{2em}
\setstretch{.5}
{\PaliGlossB{This is the fourth liberation.}}\\
\end{addmargin}
\end{absolutelynopagebreak}

\begin{absolutelynopagebreak}
\setstretch{.7}
{\PaliGlossA{sabbaso ākāsānañcāyatanaṃ samatikkamma ‘anantaṃ viññāṇan’ti viññāṇañcāyatanaṃ upasampajja viharati,}}\\
\begin{addmargin}[1em]{2em}
\setstretch{.5}
{\PaliGlossB{Going totally beyond the dimension of infinite space, aware that ‘consciousness is infinite’, they enter and remain in the dimension of infinite consciousness.}}\\
\end{addmargin}
\end{absolutelynopagebreak}

\begin{absolutelynopagebreak}
\setstretch{.7}
{\PaliGlossA{ayaṃ pañcamo vimokkho.}}\\
\begin{addmargin}[1em]{2em}
\setstretch{.5}
{\PaliGlossB{This is the fifth liberation.}}\\
\end{addmargin}
\end{absolutelynopagebreak}

\begin{absolutelynopagebreak}
\setstretch{.7}
{\PaliGlossA{sabbaso viññāṇañcāyatanaṃ samatikkamma ‘natthi kiñcī’ti ākiñcaññāyatanaṃ upasampajja viharati,}}\\
\begin{addmargin}[1em]{2em}
\setstretch{.5}
{\PaliGlossB{Going totally beyond the dimension of infinite consciousness, aware that ‘there is nothing at all’, they enter and remain in the dimension of nothingness.}}\\
\end{addmargin}
\end{absolutelynopagebreak}

\begin{absolutelynopagebreak}
\setstretch{.7}
{\PaliGlossA{ayaṃ chaṭṭho vimokkho.}}\\
\begin{addmargin}[1em]{2em}
\setstretch{.5}
{\PaliGlossB{This is the sixth liberation.}}\\
\end{addmargin}
\end{absolutelynopagebreak}

\begin{absolutelynopagebreak}
\setstretch{.7}
{\PaliGlossA{sabbaso ākiñcaññāyatanaṃ samatikkamma nevasaññānāsaññāyatanaṃ upasampajja viharati,}}\\
\begin{addmargin}[1em]{2em}
\setstretch{.5}
{\PaliGlossB{Going totally beyond the dimension of nothingness, they enter and remain in the dimension of neither perception nor non-perception.}}\\
\end{addmargin}
\end{absolutelynopagebreak}

\begin{absolutelynopagebreak}
\setstretch{.7}
{\PaliGlossA{ayaṃ sattamo vimokkho.}}\\
\begin{addmargin}[1em]{2em}
\setstretch{.5}
{\PaliGlossB{This is the seventh liberation.}}\\
\end{addmargin}
\end{absolutelynopagebreak}

\begin{absolutelynopagebreak}
\setstretch{.7}
{\PaliGlossA{sabbaso nevasaññānāsaññāyatanaṃ samatikkamma saññāvedayitanirodhaṃ upasampajja viharati,}}\\
\begin{addmargin}[1em]{2em}
\setstretch{.5}
{\PaliGlossB{Going totally beyond the dimension of neither perception nor non-perception, they enter and remain in the cessation of perception and feeling.}}\\
\end{addmargin}
\end{absolutelynopagebreak}

\begin{absolutelynopagebreak}
\setstretch{.7}
{\PaliGlossA{ayaṃ aṭṭhamo vimokkho.}}\\
\begin{addmargin}[1em]{2em}
\setstretch{.5}
{\PaliGlossB{This is the eighth liberation.}}\\
\end{addmargin}
\end{absolutelynopagebreak}

\begin{absolutelynopagebreak}
\setstretch{.7}
{\PaliGlossA{ime kho, ānanda, aṭṭha vimokkhā.}}\\
\begin{addmargin}[1em]{2em}
\setstretch{.5}
{\PaliGlossB{These are the eight liberations.}}\\
\end{addmargin}
\end{absolutelynopagebreak}

\begin{absolutelynopagebreak}
\setstretch{.7}
{\PaliGlossA{yato kho, ānanda, bhikkhu ime aṭṭha vimokkhe anulomampi samāpajjati, paṭilomampi samāpajjati, anulomapaṭilomampi samāpajjati, yatthicchakaṃ yadicchakaṃ yāvaticchakaṃ samāpajjatipi vuṭṭhātipi.}}\\
\begin{addmargin}[1em]{2em}
\setstretch{.5}
{\PaliGlossB{When a mendicant enters into and withdraws from these eight liberations—in forward order, in reverse order, and in forward and reverse order—wherever they wish, whenever they wish, and for as long as they wish;}}\\
\end{addmargin}
\end{absolutelynopagebreak}

\begin{absolutelynopagebreak}
\setstretch{.7}
{\PaliGlossA{āsavānañca khayā anāsavaṃ cetovimuttiṃ paññāvimuttiṃ diṭṭheva dhamme sayaṃ abhiññā sacchikatvā upasampajja viharati, ayaṃ vuccatānanda, bhikkhu ubhatobhāgavimutto.}}\\
\begin{addmargin}[1em]{2em}
\setstretch{.5}
{\PaliGlossB{and when they realize the undefiled freedom of heart and freedom by wisdom in this very life, and live having realized it with their own insight due to the ending of defilements, they’re called a mendicant who is freed both ways.}}\\
\end{addmargin}
\end{absolutelynopagebreak}

\begin{absolutelynopagebreak}
\setstretch{.7}
{\PaliGlossA{imāya ca, ānanda, ubhatobhāgavimuttiyā aññā ubhatobhāgavimutti uttaritarā vā paṇītatarā vā natthī”ti.}}\\
\begin{addmargin}[1em]{2em}
\setstretch{.5}
{\PaliGlossB{And, Ānanda, there is no other freedom both ways that is better or finer than this.”}}\\
\end{addmargin}
\end{absolutelynopagebreak}

\begin{absolutelynopagebreak}
\setstretch{.7}
{\PaliGlossA{idamavoca bhagavā.}}\\
\begin{addmargin}[1em]{2em}
\setstretch{.5}
{\PaliGlossB{That is what the Buddha said.}}\\
\end{addmargin}
\end{absolutelynopagebreak}

\begin{absolutelynopagebreak}
\setstretch{.7}
{\PaliGlossA{attamano āyasmā ānando bhagavato bhāsitaṃ abhinandīti.}}\\
\begin{addmargin}[1em]{2em}
\setstretch{.5}
{\PaliGlossB{Satisfied, Venerable Ānanda was happy with what the Buddha said.}}\\
\end{addmargin}
\end{absolutelynopagebreak}

\begin{absolutelynopagebreak}
\setstretch{.7}
{\PaliGlossA{mahānidānasuttaṃ niṭṭhitaṃ dutiyaṃ.}}\\
\begin{addmargin}[1em]{2em}
\setstretch{.5}
{\PaliGlossB{    -}}\\
\end{addmargin}
\end{absolutelynopagebreak}
