
\begin{absolutelynopagebreak}
\setstretch{.7}
{\PaliGlossA{dīgha nikāya 10}}\\
\begin{addmargin}[1em]{2em}
\setstretch{.5}
{\PaliGlossB{Long Discourses 10}}\\
\end{addmargin}
\end{absolutelynopagebreak}

\begin{absolutelynopagebreak}
\setstretch{.7}
{\PaliGlossA{subhasutta}}\\
\begin{addmargin}[1em]{2em}
\setstretch{.5}
{\PaliGlossB{With Subha}}\\
\end{addmargin}
\end{absolutelynopagebreak}

\begin{absolutelynopagebreak}
\setstretch{.7}
{\PaliGlossA{evaṃ me sutaṃ—}}\\
\begin{addmargin}[1em]{2em}
\setstretch{.5}
{\PaliGlossB{So I have heard.}}\\
\end{addmargin}
\end{absolutelynopagebreak}

\begin{absolutelynopagebreak}
\setstretch{.7}
{\PaliGlossA{ekaṃ samayaṃ āyasmā ānando sāvatthiyaṃ viharati jetavane anāthapiṇḍikassa ārāme aciraparinibbute bhagavati.}}\\
\begin{addmargin}[1em]{2em}
\setstretch{.5}
{\PaliGlossB{At one time Venerable Ānanda was staying near Sāvatthī in Jeta’s Grove, Anāthapiṇḍika’s monastery. It was not long after the Buddha had become fully extinguished.}}\\
\end{addmargin}
\end{absolutelynopagebreak}

\begin{absolutelynopagebreak}
\setstretch{.7}
{\PaliGlossA{tena kho pana samayena subho māṇavo todeyyaputto sāvatthiyaṃ paṭivasati kenacideva karaṇīyena.}}\\
\begin{addmargin}[1em]{2em}
\setstretch{.5}
{\PaliGlossB{Now at that time the brahmin student Subha, Todeyya’s son, was residing in Sāvatthī on some business.}}\\
\end{addmargin}
\end{absolutelynopagebreak}

\begin{absolutelynopagebreak}
\setstretch{.7}
{\PaliGlossA{atha kho subho māṇavo todeyyaputto aññataraṃ māṇavakaṃ āmantesi:}}\\
\begin{addmargin}[1em]{2em}
\setstretch{.5}
{\PaliGlossB{Then he addressed a certain student,}}\\
\end{addmargin}
\end{absolutelynopagebreak}

\begin{absolutelynopagebreak}
\setstretch{.7}
{\PaliGlossA{“ehi tvaṃ, māṇavaka, yena samaṇo ānando tenupasaṅkama; upasaṅkamitvā mama vacanena samaṇaṃ ānandaṃ appābādhaṃ appātaṅkaṃ lahuṭṭhānaṃ balaṃ phāsuvihāraṃ puccha:}}\\
\begin{addmargin}[1em]{2em}
\setstretch{.5}
{\PaliGlossB{“Here, student, go to the ascetic Ānanda and in my name bow with your head to his feet. Ask him if he is healthy and well, nimble, strong, and living comfortably.}}\\
\end{addmargin}
\end{absolutelynopagebreak}

\begin{absolutelynopagebreak}
\setstretch{.7}
{\PaliGlossA{‘subho māṇavo todeyyaputto bhavantaṃ ānandaṃ appābādhaṃ appātaṅkaṃ lahuṭṭhānaṃ balaṃ phāsuvihāraṃ pucchatī’ti.}}\\
\begin{addmargin}[1em]{2em}
\setstretch{.5}
{\PaliGlossB{    -}}\\
\end{addmargin}
\end{absolutelynopagebreak}

\begin{absolutelynopagebreak}
\setstretch{.7}
{\PaliGlossA{evañca vadehi:}}\\
\begin{addmargin}[1em]{2em}
\setstretch{.5}
{\PaliGlossB{And then say:}}\\
\end{addmargin}
\end{absolutelynopagebreak}

\begin{absolutelynopagebreak}
\setstretch{.7}
{\PaliGlossA{‘sādhu kira bhavaṃ ānando yena subhassa māṇavassa todeyyaputtassa nivesanaṃ tenupasaṅkamatu anukampaṃ upādāyā’”ti.}}\\
\begin{addmargin}[1em]{2em}
\setstretch{.5}
{\PaliGlossB{‘Sir, please visit the student Subha, Todeyya’s son, at his home out of compassion.’”}}\\
\end{addmargin}
\end{absolutelynopagebreak}

\begin{absolutelynopagebreak}
\setstretch{.7}
{\PaliGlossA{“evaṃ, bho”ti kho so māṇavako subhassa māṇavassa todeyyaputtassa paṭissutvā yenāyasmā ānando tenupasaṅkami; upasaṅkamitvā āyasmatā ānandena saddhiṃ sammodi.}}\\
\begin{addmargin}[1em]{2em}
\setstretch{.5}
{\PaliGlossB{“Yes, sir,” replied the student, and did as he was asked.}}\\
\end{addmargin}
\end{absolutelynopagebreak}

\begin{absolutelynopagebreak}
\setstretch{.7}
{\PaliGlossA{sammodanīyaṃ kathaṃ sāraṇīyaṃ vītisāretvā ekamantaṃ nisīdi. ekamantaṃ nisinno kho so māṇavako āyasmantaṃ ānandaṃ etadavoca:}}\\
\begin{addmargin}[1em]{2em}
\setstretch{.5}
{\PaliGlossB{    -}}\\
\end{addmargin}
\end{absolutelynopagebreak}

\begin{absolutelynopagebreak}
\setstretch{.7}
{\PaliGlossA{“subho māṇavo todeyyaputto bhavantaṃ ānandaṃ appābādhaṃ appātaṅkaṃ lahuṭṭhānaṃ balaṃ phāsuvihāraṃ pucchati;}}\\
\begin{addmargin}[1em]{2em}
\setstretch{.5}
{\PaliGlossB{    -}}\\
\end{addmargin}
\end{absolutelynopagebreak}

\begin{absolutelynopagebreak}
\setstretch{.7}
{\PaliGlossA{evañca vadeti:}}\\
\begin{addmargin}[1em]{2em}
\setstretch{.5}
{\PaliGlossB{    -}}\\
\end{addmargin}
\end{absolutelynopagebreak}

\begin{absolutelynopagebreak}
\setstretch{.7}
{\PaliGlossA{‘sādhu kira bhavaṃ ānando yena subhassa māṇavassa todeyyaputtassa nivesanaṃ tenupasaṅkamatu anukampaṃ upādāyā’”ti.}}\\
\begin{addmargin}[1em]{2em}
\setstretch{.5}
{\PaliGlossB{    -}}\\
\end{addmargin}
\end{absolutelynopagebreak}

\begin{absolutelynopagebreak}
\setstretch{.7}
{\PaliGlossA{evaṃ vutte, āyasmā ānando taṃ māṇavakaṃ etadavoca:}}\\
\begin{addmargin}[1em]{2em}
\setstretch{.5}
{\PaliGlossB{When he had spoken, Venerable Ānanda said to him,}}\\
\end{addmargin}
\end{absolutelynopagebreak}

\begin{absolutelynopagebreak}
\setstretch{.7}
{\PaliGlossA{“akālo kho, māṇavaka.}}\\
\begin{addmargin}[1em]{2em}
\setstretch{.5}
{\PaliGlossB{“It’s not the right time, student.}}\\
\end{addmargin}
\end{absolutelynopagebreak}

\begin{absolutelynopagebreak}
\setstretch{.7}
{\PaliGlossA{atthi me ajja bhesajjamattā pītā.}}\\
\begin{addmargin}[1em]{2em}
\setstretch{.5}
{\PaliGlossB{I’ve drunk sufficient refreshments for today.}}\\
\end{addmargin}
\end{absolutelynopagebreak}

\begin{absolutelynopagebreak}
\setstretch{.7}
{\PaliGlossA{appevanāma svepi upasaṅkameyyāma kālañca samayañca upādāyā”ti.}}\\
\begin{addmargin}[1em]{2em}
\setstretch{.5}
{\PaliGlossB{But hopefully tomorrow I’ll get a chance to visit him.”}}\\
\end{addmargin}
\end{absolutelynopagebreak}

\begin{absolutelynopagebreak}
\setstretch{.7}
{\PaliGlossA{“evaṃ, bho”ti kho so māṇavako āyasmato ānandassa paṭissutvā uṭṭhāyāsanā yena subho māṇavo todeyyaputto tenupasaṅkami; upasaṅkamitvā subhaṃ māṇavaṃ todeyyaputtaṃ etadavoca:}}\\
\begin{addmargin}[1em]{2em}
\setstretch{.5}
{\PaliGlossB{“Yes, sir,” replied the student. He went back to Subha, and told him what had happened, adding,}}\\
\end{addmargin}
\end{absolutelynopagebreak}

\begin{absolutelynopagebreak}
\setstretch{.7}
{\PaliGlossA{“avocumhā kho mayaṃ bhoto vacanena taṃ bhavantaṃ ānandaṃ:}}\\
\begin{addmargin}[1em]{2em}
\setstretch{.5}
{\PaliGlossB{    -}}\\
\end{addmargin}
\end{absolutelynopagebreak}

\begin{absolutelynopagebreak}
\setstretch{.7}
{\PaliGlossA{‘subho māṇavo todeyyaputto bhavantaṃ ānandaṃ appābādhaṃ appātaṅkaṃ lahuṭṭhānaṃ balaṃ phāsuvihāraṃ pucchati, evañca vadeti:}}\\
\begin{addmargin}[1em]{2em}
\setstretch{.5}
{\PaliGlossB{    -}}\\
\end{addmargin}
\end{absolutelynopagebreak}

\begin{absolutelynopagebreak}
\setstretch{.7}
{\PaliGlossA{“sādhu kira bhavaṃ ānando yena subhassa māṇavassa todeyyaputtassa nivesanaṃ tenupasaṅkamatu anukampaṃ upādāyā”’ti.}}\\
\begin{addmargin}[1em]{2em}
\setstretch{.5}
{\PaliGlossB{    -}}\\
\end{addmargin}
\end{absolutelynopagebreak}

\begin{absolutelynopagebreak}
\setstretch{.7}
{\PaliGlossA{evaṃ vutte, bho, samaṇo ānando maṃ etadavoca:}}\\
\begin{addmargin}[1em]{2em}
\setstretch{.5}
{\PaliGlossB{    -}}\\
\end{addmargin}
\end{absolutelynopagebreak}

\begin{absolutelynopagebreak}
\setstretch{.7}
{\PaliGlossA{‘akālo kho, māṇavaka.}}\\
\begin{addmargin}[1em]{2em}
\setstretch{.5}
{\PaliGlossB{    -}}\\
\end{addmargin}
\end{absolutelynopagebreak}

\begin{absolutelynopagebreak}
\setstretch{.7}
{\PaliGlossA{atthi me ajja bhesajjamattā pītā.}}\\
\begin{addmargin}[1em]{2em}
\setstretch{.5}
{\PaliGlossB{    -}}\\
\end{addmargin}
\end{absolutelynopagebreak}

\begin{absolutelynopagebreak}
\setstretch{.7}
{\PaliGlossA{appevanāma svepi upasaṅkameyyāma kālañca samayañca upādāyā’ti.}}\\
\begin{addmargin}[1em]{2em}
\setstretch{.5}
{\PaliGlossB{    -}}\\
\end{addmargin}
\end{absolutelynopagebreak}

\begin{absolutelynopagebreak}
\setstretch{.7}
{\PaliGlossA{ettāvatāpi kho, bho, katameva etaṃ, yato kho so bhavaṃ ānando okāsamakāsi svātanāyapi upasaṅkamanāyā”ti.}}\\
\begin{addmargin}[1em]{2em}
\setstretch{.5}
{\PaliGlossB{“This much, sir, I managed to do. At least Master Ānanda will take the opportunity to visit tomorrow.”}}\\
\end{addmargin}
\end{absolutelynopagebreak}

\begin{absolutelynopagebreak}
\setstretch{.7}
{\PaliGlossA{atha kho āyasmā ānando tassā rattiyā accayena pubbaṇhasamayaṃ nivāsetvā pattacīvaramādāya cetakena bhikkhunā pacchāsamaṇena yena subhassa māṇavassa todeyyaputtassa nivesanaṃ tenupasaṅkami; upasaṅkamitvā paññatte āsane nisīdi. atha kho subho māṇavo todeyyaputto yenāyasmā ānando tenupasaṅkami; upasaṅkamitvā āyasmatā ānandena saddhiṃ sammodi.}}\\
\begin{addmargin}[1em]{2em}
\setstretch{.5}
{\PaliGlossB{Then when the night had passed, Ānanda robed up in the morning and, taking his bowl and robe, went with Venerable Cetaka as his second monk to Subha’s home, where he sat on the seat spread out. Then Subha went up to Ānanda, and exchanged greetings with him.}}\\
\end{addmargin}
\end{absolutelynopagebreak}

\begin{absolutelynopagebreak}
\setstretch{.7}
{\PaliGlossA{sammodanīyaṃ kathaṃ sāraṇīyaṃ vītisāretvā ekamantaṃ nisīdi. ekamantaṃ nisinno kho subho māṇavo todeyyaputto āyasmantaṃ ānandaṃ etadavoca:}}\\
\begin{addmargin}[1em]{2em}
\setstretch{.5}
{\PaliGlossB{When the greetings and polite conversation were over, he sat down to one side and said to Ānanda:}}\\
\end{addmargin}
\end{absolutelynopagebreak}

\begin{absolutelynopagebreak}
\setstretch{.7}
{\PaliGlossA{“bhavañhi ānando tassa bhoto gotamassa dīgharattaṃ upaṭṭhāko santikāvacaro samīpacārī.}}\\
\begin{addmargin}[1em]{2em}
\setstretch{.5}
{\PaliGlossB{“Master Ānanda, you were Master Gotama’s attendant. You were close to him, living in his presence.}}\\
\end{addmargin}
\end{absolutelynopagebreak}

\begin{absolutelynopagebreak}
\setstretch{.7}
{\PaliGlossA{bhavametaṃ ānando jāneyya, yesaṃ so bhavaṃ gotamo dhammānaṃ vaṇṇavādī ahosi, yattha ca imaṃ janataṃ samādapesi nivesesi patiṭṭhāpesi.}}\\
\begin{addmargin}[1em]{2em}
\setstretch{.5}
{\PaliGlossB{You ought to know what things Master Gotama praised, and in which he encouraged, settled, and grounded all these people.}}\\
\end{addmargin}
\end{absolutelynopagebreak}

\begin{absolutelynopagebreak}
\setstretch{.7}
{\PaliGlossA{katamesānaṃ kho, bho ānanda, dhammānaṃ so bhavaṃ gotamo vaṇṇavādī ahosi; kattha ca imaṃ janataṃ samādapesi nivesesi patiṭṭhāpesī”ti?}}\\
\begin{addmargin}[1em]{2em}
\setstretch{.5}
{\PaliGlossB{What were those things?”}}\\
\end{addmargin}
\end{absolutelynopagebreak}

\begin{absolutelynopagebreak}
\setstretch{.7}
{\PaliGlossA{“tiṇṇaṃ kho, māṇava, khandhānaṃ so bhagavā vaṇṇavādī ahosi; ettha ca imaṃ janataṃ samādapesi nivesesi patiṭṭhāpesi.}}\\
\begin{addmargin}[1em]{2em}
\setstretch{.5}
{\PaliGlossB{“Student, the Buddha praised three sets of things, and that’s what he encouraged, settled, and grounded all these people in.}}\\
\end{addmargin}
\end{absolutelynopagebreak}

\begin{absolutelynopagebreak}
\setstretch{.7}
{\PaliGlossA{katamesaṃ tiṇṇaṃ?}}\\
\begin{addmargin}[1em]{2em}
\setstretch{.5}
{\PaliGlossB{What three?}}\\
\end{addmargin}
\end{absolutelynopagebreak}

\begin{absolutelynopagebreak}
\setstretch{.7}
{\PaliGlossA{ariyassa sīlakkhandhassa, ariyassa samādhikkhandhassa, ariyassa paññākkhandhassa.}}\\
\begin{addmargin}[1em]{2em}
\setstretch{.5}
{\PaliGlossB{The entire spectrum of noble ethics, immersion, and wisdom.}}\\
\end{addmargin}
\end{absolutelynopagebreak}

\begin{absolutelynopagebreak}
\setstretch{.7}
{\PaliGlossA{imesaṃ kho, māṇava, tiṇṇaṃ khandhānaṃ so bhagavā vaṇṇavādī ahosi; ettha ca imaṃ janataṃ samādapesi nivesesi patiṭṭhāpesī”ti.}}\\
\begin{addmargin}[1em]{2em}
\setstretch{.5}
{\PaliGlossB{These are the three sets of things that the Buddha praised.”}}\\
\end{addmargin}
\end{absolutelynopagebreak}

\begin{absolutelynopagebreak}
\setstretch{.7}
{\PaliGlossA{1. sīlakkhandha}}\\
\begin{addmargin}[1em]{2em}
\setstretch{.5}
{\PaliGlossB{1. The Entire Spectrum of Ethics}}\\
\end{addmargin}
\end{absolutelynopagebreak}

\begin{absolutelynopagebreak}
\setstretch{.7}
{\PaliGlossA{“katamo pana so, bho ānanda, ariyo sīlakkhandho, yassa so bhavaṃ gotamo vaṇṇavādī ahosi, yattha ca imaṃ janataṃ samādapesi nivesesi patiṭṭhāpesī”ti?}}\\
\begin{addmargin}[1em]{2em}
\setstretch{.5}
{\PaliGlossB{“But what was that noble spectrum of ethics that the Buddha praised?”}}\\
\end{addmargin}
\end{absolutelynopagebreak}

\begin{absolutelynopagebreak}
\setstretch{.7}
{\PaliGlossA{“idha, māṇava, tathāgato loke uppajjati arahaṃ sammāsambuddho vijjācaraṇasampanno sugato lokavidū anuttaro purisadammasārathi satthā devamanussānaṃ buddho bhagavā.}}\\
\begin{addmargin}[1em]{2em}
\setstretch{.5}
{\PaliGlossB{“Student, it’s when a Realized One arises in the world, perfected, a fully awakened Buddha, accomplished in knowledge and conduct, holy, knower of the world, supreme guide for those who wish to train, teacher of gods and humans, awakened, blessed.}}\\
\end{addmargin}
\end{absolutelynopagebreak}

\begin{absolutelynopagebreak}
\setstretch{.7}
{\PaliGlossA{so imaṃ lokaṃ sadevakaṃ samārakaṃ sabrahmakaṃ sassamaṇabrāhmaṇiṃ pajaṃ sadevamanussaṃ sayaṃ abhiññā sacchikatvā pavedeti.}}\\
\begin{addmargin}[1em]{2em}
\setstretch{.5}
{\PaliGlossB{He has realized with his own insight this world—with its gods, Māras and Brahmās, this population with its ascetics and brahmins, gods and humans—and he makes it known to others.}}\\
\end{addmargin}
\end{absolutelynopagebreak}

\begin{absolutelynopagebreak}
\setstretch{.7}
{\PaliGlossA{so dhammaṃ deseti ādikalyāṇaṃ majjhekalyāṇaṃ pariyosānakalyāṇaṃ sātthaṃ sabyañjanaṃ kevalaparipuṇṇaṃ parisuddhaṃ brahmacariyaṃ pakāseti.}}\\
\begin{addmargin}[1em]{2em}
\setstretch{.5}
{\PaliGlossB{He teaches Dhamma that’s good in the beginning, good in the middle, and good in the end, meaningful and well-phrased. And he reveals a spiritual practice that’s entirely full and pure.}}\\
\end{addmargin}
\end{absolutelynopagebreak}

\begin{absolutelynopagebreak}
\setstretch{.7}
{\PaliGlossA{taṃ dhammaṃ suṇāti gahapati vā gahapatiputto vā aññatarasmiṃ vā kule paccājāto.}}\\
\begin{addmargin}[1em]{2em}
\setstretch{.5}
{\PaliGlossB{A householder hears that teaching, or a householder’s child, or someone reborn in some clan.}}\\
\end{addmargin}
\end{absolutelynopagebreak}

\begin{absolutelynopagebreak}
\setstretch{.7}
{\PaliGlossA{so taṃ dhammaṃ sutvā tathāgate saddhaṃ paṭilabhati.}}\\
\begin{addmargin}[1em]{2em}
\setstretch{.5}
{\PaliGlossB{They gain faith in the Realized One,}}\\
\end{addmargin}
\end{absolutelynopagebreak}

\begin{absolutelynopagebreak}
\setstretch{.7}
{\PaliGlossA{so tena saddhāpaṭilābhena samannāgato iti paṭisañcikkhati:}}\\
\begin{addmargin}[1em]{2em}
\setstretch{.5}
{\PaliGlossB{and reflect:}}\\
\end{addmargin}
\end{absolutelynopagebreak}

\begin{absolutelynopagebreak}
\setstretch{.7}
{\PaliGlossA{‘sambādho gharāvāso rajopatho, abbhokāso pabbajjā,}}\\
\begin{addmargin}[1em]{2em}
\setstretch{.5}
{\PaliGlossB{‘Living in a house is cramped and dirty, but the life of one gone forth is wide open.}}\\
\end{addmargin}
\end{absolutelynopagebreak}

\begin{absolutelynopagebreak}
\setstretch{.7}
{\PaliGlossA{nayidaṃ sukaraṃ agāraṃ ajjhāvasatā ekantaparipuṇṇaṃ ekantaparisuddhaṃ saṅkhalikhitaṃ brahmacariyaṃ carituṃ.}}\\
\begin{addmargin}[1em]{2em}
\setstretch{.5}
{\PaliGlossB{It’s not easy for someone living at home to lead the spiritual life utterly full and pure, like a polished shell.}}\\
\end{addmargin}
\end{absolutelynopagebreak}

\begin{absolutelynopagebreak}
\setstretch{.7}
{\PaliGlossA{yannūnāhaṃ kesamassuṃ ohāretvā kāsāyāni vatthāni acchādetvā agārasmā anagāriyaṃ pabbajeyyan’ti.}}\\
\begin{addmargin}[1em]{2em}
\setstretch{.5}
{\PaliGlossB{Why don’t I shave off my hair and beard, dress in ocher robes, and go forth from the lay life to homelessness?’}}\\
\end{addmargin}
\end{absolutelynopagebreak}

\begin{absolutelynopagebreak}
\setstretch{.7}
{\PaliGlossA{so aparena samayena appaṃ vā bhogakkhandhaṃ pahāya mahantaṃ vā bhogakkhandhaṃ pahāya appaṃ vā ñātiparivaṭṭaṃ pahāya mahantaṃ vā ñātiparivaṭṭaṃ pahāya kesamassuṃ ohāretvā kāsāyāni vatthāni acchādetvā agārasmā anagāriyaṃ pabbajati.}}\\
\begin{addmargin}[1em]{2em}
\setstretch{.5}
{\PaliGlossB{After some time they give up a large or small fortune, and a large or small family circle. They shave off hair and beard, dress in ocher robes, and go forth from the lay life to homelessness.}}\\
\end{addmargin}
\end{absolutelynopagebreak}

\begin{absolutelynopagebreak}
\setstretch{.7}
{\PaliGlossA{so evaṃ pabbajito samāno pātimokkhasaṃvarasaṃvuto viharati, ācāragocarasampanno, anumattesu vajjesu bhayadassāvī, samādāya sikkhati sikkhāpadesu, kāyakammavacīkammena samannāgato kusalena, parisuddhājīvo, sīlasampanno, indriyesu guttadvāro, satisampajaññena samannāgato, santuṭṭho.}}\\
\begin{addmargin}[1em]{2em}
\setstretch{.5}
{\PaliGlossB{Once they’ve gone forth, they live restrained in the monastic code, conducting themselves well and seeking alms in suitable places. Seeing danger in the slightest fault, they keep the rules they’ve undertaken. They act skillfully by body and speech. They’re purified in livelihood and accomplished in ethical conduct. They guard the sense doors, have mindfulness and situational awareness, and are content.}}\\
\end{addmargin}
\end{absolutelynopagebreak}

\begin{absolutelynopagebreak}
\setstretch{.7}
{\PaliGlossA{kathañca, māṇava, bhikkhu sīlasampanno hoti?}}\\
\begin{addmargin}[1em]{2em}
\setstretch{.5}
{\PaliGlossB{And how is a mendicant accomplished in ethics?}}\\
\end{addmargin}
\end{absolutelynopagebreak}

\begin{absolutelynopagebreak}
\setstretch{.7}
{\PaliGlossA{idha, māṇava, bhikkhu pāṇātipātaṃ pahāya pāṇātipātā paṭivirato hoti, nihitadaṇḍo nihitasattho lajjī dayāpanno, sabbapāṇabhūtahitānukampī viharati.}}\\
\begin{addmargin}[1em]{2em}
\setstretch{.5}
{\PaliGlossB{It’s when a mendicant gives up killing living creatures. They renounce the rod and the sword. They’re scrupulous and kind, living full of compassion for all living beings. …}}\\
\end{addmargin}
\end{absolutelynopagebreak}

\begin{absolutelynopagebreak}
\setstretch{.7}
{\PaliGlossA{yampi, māṇava, bhikkhu pāṇātipātaṃ pahāya pāṇātipātā paṭivirato hoti, nihitadaṇḍo nihitasattho lajjī dayāpanno, sabbapāṇabhūtahitānukampī viharati;}}\\
\begin{addmargin}[1em]{2em}
\setstretch{.5}
{\PaliGlossB{    -}}\\
\end{addmargin}
\end{absolutelynopagebreak}

\begin{absolutelynopagebreak}
\setstretch{.7}
{\PaliGlossA{idampissa hoti sīlasmiṃ.}}\\
\begin{addmargin}[1em]{2em}
\setstretch{.5}
{\PaliGlossB{This pertains to their ethics.}}\\
\end{addmargin}
\end{absolutelynopagebreak}

\begin{absolutelynopagebreak}
\setstretch{.7}
{\PaliGlossA{yathā vā paneke bhonto samaṇabrāhmaṇā saddhādeyyāni bhojanāni bhuñjitvā te evarūpāya tiracchānavijjāya micchājīvena jīvitaṃ kappenti,}}\\
\begin{addmargin}[1em]{2em}
\setstretch{.5}
{\PaliGlossB{There are some ascetics and brahmins who, while enjoying food given in faith, still earn a living by unworthy branches of knowledge, by wrong livelihood.}}\\
\end{addmargin}
\end{absolutelynopagebreak}

\begin{absolutelynopagebreak}
\setstretch{.7}
{\PaliGlossA{seyyathidaṃ—santikammaṃ paṇidhikammaṃ bhūtakammaṃ bhūrikammaṃ vassakammaṃ vossakammaṃ vatthukammaṃ vatthuparikammaṃ ācamanaṃ nhāpanaṃ juhanaṃ vamanaṃ virecanaṃ uddhaṃvirecanaṃ adhovirecanaṃ sīsavirecanaṃ kaṇṇatelaṃ nettatappanaṃ natthukammaṃ añjanaṃ paccañjanaṃ sālākiyaṃ sallakattiyaṃ dārakatikicchā mūlabhesajjānaṃ anuppadānaṃ osadhīnaṃ paṭimokkho}}\\
\begin{addmargin}[1em]{2em}
\setstretch{.5}
{\PaliGlossB{This includes rites for propitiation, for granting wishes, for ghosts, for the earth, for rain, for property settlement, and for preparing and consecrating house sites, and rites involving rinsing and bathing, and oblations. It also includes administering emetics, purgatives, expectorants, and phlegmagogues; administering ear-oils, eye restoratives, nasal medicine, ointments, and counter-ointments; surgery with needle and scalpel, treating children, prescribing root medicines, and binding on herbs.}}\\
\end{addmargin}
\end{absolutelynopagebreak}

\begin{absolutelynopagebreak}
\setstretch{.7}
{\PaliGlossA{iti vā iti evarūpāya tiracchānavijjāya micchājīvā paṭivirato hoti.}}\\
\begin{addmargin}[1em]{2em}
\setstretch{.5}
{\PaliGlossB{They refrain from such unworthy branches of knowledge, such wrong livelihood. …}}\\
\end{addmargin}
\end{absolutelynopagebreak}

\begin{absolutelynopagebreak}
\setstretch{.7}
{\PaliGlossA{yampi, māṇava, bhikkhu yathā vā paneke bhonto samaṇabrāhmaṇā saddhādeyyāni bhojanāni bhuñjitvā te evarūpāya tiracchānavijjāya micchājīvena jīvitaṃ kappenti,}}\\
\begin{addmargin}[1em]{2em}
\setstretch{.5}
{\PaliGlossB{    -}}\\
\end{addmargin}
\end{absolutelynopagebreak}

\begin{absolutelynopagebreak}
\setstretch{.7}
{\PaliGlossA{seyyathidaṃ—santikammaṃ paṇidhikammaṃ … pe … osadhīnaṃ paṭimokkho}}\\
\begin{addmargin}[1em]{2em}
\setstretch{.5}
{\PaliGlossB{    -}}\\
\end{addmargin}
\end{absolutelynopagebreak}

\begin{absolutelynopagebreak}
\setstretch{.7}
{\PaliGlossA{iti vā iti evarūpāya tiracchānavijjāya micchājīvā paṭivirato hoti.}}\\
\begin{addmargin}[1em]{2em}
\setstretch{.5}
{\PaliGlossB{    -}}\\
\end{addmargin}
\end{absolutelynopagebreak}

\begin{absolutelynopagebreak}
\setstretch{.7}
{\PaliGlossA{idampissa hoti sīlasmiṃ.}}\\
\begin{addmargin}[1em]{2em}
\setstretch{.5}
{\PaliGlossB{This pertains to their ethics.}}\\
\end{addmargin}
\end{absolutelynopagebreak}

\begin{absolutelynopagebreak}
\setstretch{.7}
{\PaliGlossA{sa kho so, māṇava, bhikkhu evaṃ sīlasampanno na kutoci bhayaṃ samanupassati, yadidaṃ sīlasaṃvarato.}}\\
\begin{addmargin}[1em]{2em}
\setstretch{.5}
{\PaliGlossB{A mendicant thus accomplished in ethics sees no danger in any quarter in regards to their ethical restraint.}}\\
\end{addmargin}
\end{absolutelynopagebreak}

\begin{absolutelynopagebreak}
\setstretch{.7}
{\PaliGlossA{seyyathāpi, māṇava, rājā khattiyo muddhāvasitto nihatapaccāmitto na kutoci bhayaṃ samanupassati, yadidaṃ paccatthikato;}}\\
\begin{addmargin}[1em]{2em}
\setstretch{.5}
{\PaliGlossB{It’s like a king who has defeated his enemies. He sees no danger from his foes in any quarter.}}\\
\end{addmargin}
\end{absolutelynopagebreak}

\begin{absolutelynopagebreak}
\setstretch{.7}
{\PaliGlossA{evameva kho, māṇava, bhikkhu evaṃ sīlasampanno na kutoci bhayaṃ samanupassati, yadidaṃ sīlasaṃvarato.}}\\
\begin{addmargin}[1em]{2em}
\setstretch{.5}
{\PaliGlossB{A mendicant thus accomplished in ethics sees no danger in any quarter in regards to their ethical restraint.}}\\
\end{addmargin}
\end{absolutelynopagebreak}

\begin{absolutelynopagebreak}
\setstretch{.7}
{\PaliGlossA{so iminā ariyena sīlakkhandhena samannāgato ajjhattaṃ anavajjasukhaṃ paṭisaṃvedeti.}}\\
\begin{addmargin}[1em]{2em}
\setstretch{.5}
{\PaliGlossB{When they have this entire spectrum of noble ethics, they experience a blameless happiness inside themselves.}}\\
\end{addmargin}
\end{absolutelynopagebreak}

\begin{absolutelynopagebreak}
\setstretch{.7}
{\PaliGlossA{evaṃ kho, māṇava, bhikkhu sīlasampanno hoti.}}\\
\begin{addmargin}[1em]{2em}
\setstretch{.5}
{\PaliGlossB{That’s how a mendicant is accomplished in ethics.}}\\
\end{addmargin}
\end{absolutelynopagebreak}

\begin{absolutelynopagebreak}
\setstretch{.7}
{\PaliGlossA{ayaṃ kho so, māṇava, ariyo sīlakkhandho yassa so bhagavā vaṇṇavādī ahosi, yattha ca imaṃ janataṃ samādapesi nivesesi patiṭṭhāpesi.}}\\
\begin{addmargin}[1em]{2em}
\setstretch{.5}
{\PaliGlossB{This is that noble spectrum of ethics that the Buddha praised.}}\\
\end{addmargin}
\end{absolutelynopagebreak}

\begin{absolutelynopagebreak}
\setstretch{.7}
{\PaliGlossA{atthi cevettha uttarikaraṇīyan”ti.}}\\
\begin{addmargin}[1em]{2em}
\setstretch{.5}
{\PaliGlossB{But there is still more to be done.”}}\\
\end{addmargin}
\end{absolutelynopagebreak}

\begin{absolutelynopagebreak}
\setstretch{.7}
{\PaliGlossA{“acchariyaṃ, bho ānanda, abbhutaṃ, bho ānanda.}}\\
\begin{addmargin}[1em]{2em}
\setstretch{.5}
{\PaliGlossB{“It’s incredible, Master Ānanda, it’s amazing,}}\\
\end{addmargin}
\end{absolutelynopagebreak}

\begin{absolutelynopagebreak}
\setstretch{.7}
{\PaliGlossA{so cāyaṃ, bho ānanda, ariyo sīlakkhandho paripuṇṇo, no aparipuṇṇo.}}\\
\begin{addmargin}[1em]{2em}
\setstretch{.5}
{\PaliGlossB{This noble spectrum of ethics is complete, not lacking anything!}}\\
\end{addmargin}
\end{absolutelynopagebreak}

\begin{absolutelynopagebreak}
\setstretch{.7}
{\PaliGlossA{evaṃ paripuṇṇañcāhaṃ, bho ānanda, ariyaṃ sīlakkhandhaṃ ito bahiddhā aññesu samaṇabrāhmaṇesu na samanupassāmi.}}\\
\begin{addmargin}[1em]{2em}
\setstretch{.5}
{\PaliGlossB{Such a complete spectrum of ethics cannot be seen among the other ascetics and brahmins.}}\\
\end{addmargin}
\end{absolutelynopagebreak}

\begin{absolutelynopagebreak}
\setstretch{.7}
{\PaliGlossA{evaṃ paripuṇṇañca, bho ānanda, ariyaṃ sīlakkhandhaṃ ito bahiddhā aññe samaṇabrāhmaṇā attani samanupasseyyuṃ, te tāvatakeneva attamanā assu:}}\\
\begin{addmargin}[1em]{2em}
\setstretch{.5}
{\PaliGlossB{Were other ascetics and brahmins to see such a complete spectrum of noble ethics in themselves, they’d be delighted with just that much:}}\\
\end{addmargin}
\end{absolutelynopagebreak}

\begin{absolutelynopagebreak}
\setstretch{.7}
{\PaliGlossA{‘alamettāvatā, katamettāvatā, anuppatto no sāmaññattho, natthi no kiñci uttarikaraṇīyan’ti.}}\\
\begin{addmargin}[1em]{2em}
\setstretch{.5}
{\PaliGlossB{‘At this point it’s enough; at this point our work is done. We’ve reached the goal of our ascetic life. There is nothing more to be done.’}}\\
\end{addmargin}
\end{absolutelynopagebreak}

\begin{absolutelynopagebreak}
\setstretch{.7}
{\PaliGlossA{atha ca pana bhavaṃ ānando evamāha:}}\\
\begin{addmargin}[1em]{2em}
\setstretch{.5}
{\PaliGlossB{And yet you say:}}\\
\end{addmargin}
\end{absolutelynopagebreak}

\begin{absolutelynopagebreak}
\setstretch{.7}
{\PaliGlossA{‘atthi cevettha uttarikaraṇīyan’”ti.}}\\
\begin{addmargin}[1em]{2em}
\setstretch{.5}
{\PaliGlossB{‘But there is still more to be done.’}}\\
\end{addmargin}
\end{absolutelynopagebreak}

\begin{absolutelynopagebreak}
\setstretch{.7}
{\PaliGlossA{2. samādhikkhandha}}\\
\begin{addmargin}[1em]{2em}
\setstretch{.5}
{\PaliGlossB{2. The Spectrum of Immersion}}\\
\end{addmargin}
\end{absolutelynopagebreak}

\begin{absolutelynopagebreak}
\setstretch{.7}
{\PaliGlossA{“katamo pana so, bho ānanda, ariyo samādhikkhandho, yassa so bhavaṃ gotamo vaṇṇavādī ahosi, yattha ca imaṃ janataṃ samādapesi nivesesi patiṭṭhāpesī”ti?}}\\
\begin{addmargin}[1em]{2em}
\setstretch{.5}
{\PaliGlossB{But what, Master Ānanda, was that noble spectrum of immersion that the Buddha praised?”}}\\
\end{addmargin}
\end{absolutelynopagebreak}

\begin{absolutelynopagebreak}
\setstretch{.7}
{\PaliGlossA{“kathañca, māṇava, bhikkhu indriyesu guttadvāro hoti?}}\\
\begin{addmargin}[1em]{2em}
\setstretch{.5}
{\PaliGlossB{“And how, student, does a mendicant guard the sense doors?}}\\
\end{addmargin}
\end{absolutelynopagebreak}

\begin{absolutelynopagebreak}
\setstretch{.7}
{\PaliGlossA{idha, māṇava, bhikkhu cakkhunā rūpaṃ disvā na nimittaggāhī hoti nānubyañjanaggāhī;}}\\
\begin{addmargin}[1em]{2em}
\setstretch{.5}
{\PaliGlossB{When a mendicant sees a sight with their eyes, they don’t get caught up in the features and details.}}\\
\end{addmargin}
\end{absolutelynopagebreak}

\begin{absolutelynopagebreak}
\setstretch{.7}
{\PaliGlossA{yatvādhikaraṇamenaṃ cakkhundriyaṃ asaṃvutaṃ viharantaṃ abhijjhādomanassā pāpakā akusalā dhammā anvāssaveyyuṃ; tassa saṃvarāya paṭipajjati, rakkhati cakkhundriyaṃ, cakkhundriye saṃvaraṃ āpajjati.}}\\
\begin{addmargin}[1em]{2em}
\setstretch{.5}
{\PaliGlossB{If the faculty of sight were left unrestrained, bad unskillful qualities of desire and aversion would become overwhelming. For this reason, they practice restraint, protecting the faculty of sight, and achieving its restraint.}}\\
\end{addmargin}
\end{absolutelynopagebreak}

\begin{absolutelynopagebreak}
\setstretch{.7}
{\PaliGlossA{sotena saddaṃ sutvā … pe …}}\\
\begin{addmargin}[1em]{2em}
\setstretch{.5}
{\PaliGlossB{When they hear a sound with their ears …}}\\
\end{addmargin}
\end{absolutelynopagebreak}

\begin{absolutelynopagebreak}
\setstretch{.7}
{\PaliGlossA{ghānena gandhaṃ ghāyitvā …}}\\
\begin{addmargin}[1em]{2em}
\setstretch{.5}
{\PaliGlossB{When they smell an odor with their nose …}}\\
\end{addmargin}
\end{absolutelynopagebreak}

\begin{absolutelynopagebreak}
\setstretch{.7}
{\PaliGlossA{jivhāya rasaṃ sāyitvā …}}\\
\begin{addmargin}[1em]{2em}
\setstretch{.5}
{\PaliGlossB{When they taste a flavor with their tongue …}}\\
\end{addmargin}
\end{absolutelynopagebreak}

\begin{absolutelynopagebreak}
\setstretch{.7}
{\PaliGlossA{kāyena phoṭṭhabbaṃ phusitvā …}}\\
\begin{addmargin}[1em]{2em}
\setstretch{.5}
{\PaliGlossB{When they feel a touch with their body …}}\\
\end{addmargin}
\end{absolutelynopagebreak}

\begin{absolutelynopagebreak}
\setstretch{.7}
{\PaliGlossA{manasā dhammaṃ viññāya na nimittaggāhī hoti nānubyañjanaggāhī;}}\\
\begin{addmargin}[1em]{2em}
\setstretch{.5}
{\PaliGlossB{When they know a thought with their mind, they don’t get caught up in the features and details.}}\\
\end{addmargin}
\end{absolutelynopagebreak}

\begin{absolutelynopagebreak}
\setstretch{.7}
{\PaliGlossA{yatvādhikaraṇamenaṃ manindriyaṃ asaṃvutaṃ viharantaṃ abhijjhādomanassā pāpakā akusalā dhammā anvāssaveyyuṃ; tassa saṃvarāya paṭipajjati, rakkhati manindriyaṃ, manindriye saṃvaraṃ āpajjati.}}\\
\begin{addmargin}[1em]{2em}
\setstretch{.5}
{\PaliGlossB{If the faculty of mind were left unrestrained, bad unskillful qualities of desire and aversion would become overwhelming. For this reason, they practice restraint, protecting the faculty of mind, and achieving its restraint.}}\\
\end{addmargin}
\end{absolutelynopagebreak}

\begin{absolutelynopagebreak}
\setstretch{.7}
{\PaliGlossA{so iminā ariyena indriyasaṃvarena samannāgato ajjhattaṃ abyāsekasukhaṃ paṭisaṃvedeti.}}\\
\begin{addmargin}[1em]{2em}
\setstretch{.5}
{\PaliGlossB{When they have this noble sense restraint, they experience an unsullied bliss inside themselves.}}\\
\end{addmargin}
\end{absolutelynopagebreak}

\begin{absolutelynopagebreak}
\setstretch{.7}
{\PaliGlossA{evaṃ kho, māṇava, bhikkhu indriyesu guttadvāro hoti.}}\\
\begin{addmargin}[1em]{2em}
\setstretch{.5}
{\PaliGlossB{That’s how a mendicant guards the sense doors.}}\\
\end{addmargin}
\end{absolutelynopagebreak}

\begin{absolutelynopagebreak}
\setstretch{.7}
{\PaliGlossA{kathañca, māṇava, bhikkhu satisampajaññena samannāgato hoti?}}\\
\begin{addmargin}[1em]{2em}
\setstretch{.5}
{\PaliGlossB{And how does a mendicant have mindfulness and situational awareness?}}\\
\end{addmargin}
\end{absolutelynopagebreak}

\begin{absolutelynopagebreak}
\setstretch{.7}
{\PaliGlossA{idha, māṇava, bhikkhu abhikkante paṭikkante sampajānakārī hoti, ālokite vilokite sampajānakārī hoti, samiñjite pasārite sampajānakārī hoti, saṅghāṭipattacīvaradhāraṇe sampajānakārī hoti, asite pīte khāyite sāyite sampajānakārī hoti, uccārapassāvakamme sampajānakārī hoti, gate ṭhite nisinne sutte jāgarite bhāsite tuṇhībhāve sampajānakārī hoti.}}\\
\begin{addmargin}[1em]{2em}
\setstretch{.5}
{\PaliGlossB{It’s when a mendicant acts with situational awareness when going out and coming back; when looking ahead and aside; when bending and extending the limbs; when bearing the outer robe, bowl and robes; when eating, drinking, chewing, and tasting; when urinating and defecating; when walking, standing, sitting, sleeping, waking, speaking, and keeping silent.}}\\
\end{addmargin}
\end{absolutelynopagebreak}

\begin{absolutelynopagebreak}
\setstretch{.7}
{\PaliGlossA{evaṃ kho, māṇava, bhikkhu satisampajaññena samannāgato hoti.}}\\
\begin{addmargin}[1em]{2em}
\setstretch{.5}
{\PaliGlossB{That’s how a mendicant has mindfulness and situational awareness.}}\\
\end{addmargin}
\end{absolutelynopagebreak}

\begin{absolutelynopagebreak}
\setstretch{.7}
{\PaliGlossA{kathañca, māṇava, bhikkhu santuṭṭho hoti?}}\\
\begin{addmargin}[1em]{2em}
\setstretch{.5}
{\PaliGlossB{And how is a mendicant content?}}\\
\end{addmargin}
\end{absolutelynopagebreak}

\begin{absolutelynopagebreak}
\setstretch{.7}
{\PaliGlossA{idha, māṇava, bhikkhu santuṭṭho hoti kāyaparihārikena cīvarena kucchiparihārikena piṇḍapātena. so yena yeneva pakkamati, samādāyeva pakkamati.}}\\
\begin{addmargin}[1em]{2em}
\setstretch{.5}
{\PaliGlossB{It’s when a mendicant is content with robes to look after the body and alms-food to look after the belly. Wherever they go, they set out taking only these things.}}\\
\end{addmargin}
\end{absolutelynopagebreak}

\begin{absolutelynopagebreak}
\setstretch{.7}
{\PaliGlossA{seyyathāpi, māṇava, pakkhī sakuṇo yena yeneva ḍeti, sapattabhārova ḍeti;}}\\
\begin{addmargin}[1em]{2em}
\setstretch{.5}
{\PaliGlossB{They’re like a bird: wherever it flies, wings are its only burden.}}\\
\end{addmargin}
\end{absolutelynopagebreak}

\begin{absolutelynopagebreak}
\setstretch{.7}
{\PaliGlossA{evameva kho, māṇava, bhikkhu santuṭṭho hoti kāyaparihārikena cīvarena kucchiparihārikena piṇḍapātena. so yena yeneva pakkamati, samādāyeva pakkamati.}}\\
\begin{addmargin}[1em]{2em}
\setstretch{.5}
{\PaliGlossB{In the same way, a mendicant is content with robes to look after the body and alms-food to look after the belly. Wherever they go, they set out taking only these things.}}\\
\end{addmargin}
\end{absolutelynopagebreak}

\begin{absolutelynopagebreak}
\setstretch{.7}
{\PaliGlossA{evaṃ kho, māṇava, bhikkhu santuṭṭho hoti.}}\\
\begin{addmargin}[1em]{2em}
\setstretch{.5}
{\PaliGlossB{That’s how a mendicant is content.}}\\
\end{addmargin}
\end{absolutelynopagebreak}

\begin{absolutelynopagebreak}
\setstretch{.7}
{\PaliGlossA{so iminā ca ariyena sīlakkhandhena samannāgato, iminā ca ariyena indriyasaṃvarena samannāgato, iminā ca ariyena satisampajaññena samannāgato, imāya ca ariyāya santuṭṭhiyā samannāgato}}\\
\begin{addmargin}[1em]{2em}
\setstretch{.5}
{\PaliGlossB{When they have this noble spectrum of ethics, this noble sense restraint, this noble mindfulness and situational awareness, and this noble contentment,}}\\
\end{addmargin}
\end{absolutelynopagebreak}

\begin{absolutelynopagebreak}
\setstretch{.7}
{\PaliGlossA{vivittaṃ senāsanaṃ bhajati araññaṃ rukkhamūlaṃ pabbataṃ kandaraṃ giriguhaṃ susānaṃ vanapatthaṃ abbhokāsaṃ palālapuñjaṃ.}}\\
\begin{addmargin}[1em]{2em}
\setstretch{.5}
{\PaliGlossB{they frequent a secluded lodging—a wilderness, the root of a tree, a hill, a ravine, a mountain cave, a charnel ground, a forest, the open air, a heap of straw.}}\\
\end{addmargin}
\end{absolutelynopagebreak}

\begin{absolutelynopagebreak}
\setstretch{.7}
{\PaliGlossA{so pacchābhattaṃ piṇḍapātapaṭikkanto nisīdati pallaṅkaṃ ābhujitvā, ujuṃ kāyaṃ paṇidhāya, parimukhaṃ satiṃ upaṭṭhapetvā.}}\\
\begin{addmargin}[1em]{2em}
\setstretch{.5}
{\PaliGlossB{After the meal, they return from alms-round, sit down cross-legged with their body straight, and establish mindfulness right there.}}\\
\end{addmargin}
\end{absolutelynopagebreak}

\begin{absolutelynopagebreak}
\setstretch{.7}
{\PaliGlossA{so abhijjhaṃ loke pahāya vigatābhijjhena cetasā viharati abhijjhāya cittaṃ parisodheti.}}\\
\begin{addmargin}[1em]{2em}
\setstretch{.5}
{\PaliGlossB{Giving up desire for the world, they meditate with a heart rid of desire, cleansing the mind of desire.}}\\
\end{addmargin}
\end{absolutelynopagebreak}

\begin{absolutelynopagebreak}
\setstretch{.7}
{\PaliGlossA{byāpādapadosaṃ pahāya abyāpannacitto viharati sabbapāṇabhūtahitānukampī byāpādapadosā cittaṃ parisodheti.}}\\
\begin{addmargin}[1em]{2em}
\setstretch{.5}
{\PaliGlossB{Giving up ill will and malevolence, they meditate with a mind rid of ill will, full of compassion for all living beings, cleansing the mind of ill will.}}\\
\end{addmargin}
\end{absolutelynopagebreak}

\begin{absolutelynopagebreak}
\setstretch{.7}
{\PaliGlossA{thinamiddhaṃ pahāya vigatathinamiddho viharati ālokasaññī sato sampajāno, thinamiddhā cittaṃ parisodheti.}}\\
\begin{addmargin}[1em]{2em}
\setstretch{.5}
{\PaliGlossB{Giving up dullness and drowsiness, they meditate with a mind rid of dullness and drowsiness, perceiving light, mindful and aware, cleansing the mind of dullness and drowsiness.}}\\
\end{addmargin}
\end{absolutelynopagebreak}

\begin{absolutelynopagebreak}
\setstretch{.7}
{\PaliGlossA{uddhaccakukkuccaṃ pahāya anuddhato viharati ajjhattaṃ vūpasantacitto uddhaccakukkuccā cittaṃ parisodheti.}}\\
\begin{addmargin}[1em]{2em}
\setstretch{.5}
{\PaliGlossB{Giving up restlessness and remorse, they meditate without restlessness, their mind peaceful inside, cleansing the mind of restlessness and remorse.}}\\
\end{addmargin}
\end{absolutelynopagebreak}

\begin{absolutelynopagebreak}
\setstretch{.7}
{\PaliGlossA{vicikicchaṃ pahāya tiṇṇavicikiccho viharati akathaṅkathī kusalesu dhammesu, vicikicchāya cittaṃ parisodheti.}}\\
\begin{addmargin}[1em]{2em}
\setstretch{.5}
{\PaliGlossB{Giving up doubt, they meditate having gone beyond doubt, not undecided about skillful qualities, cleansing the mind of doubt.}}\\
\end{addmargin}
\end{absolutelynopagebreak}

\begin{absolutelynopagebreak}
\setstretch{.7}
{\PaliGlossA{seyyathāpi, māṇava, puriso iṇaṃ ādāya kammante payojeyya.}}\\
\begin{addmargin}[1em]{2em}
\setstretch{.5}
{\PaliGlossB{Suppose a man who has gotten into debt were to apply himself to work,}}\\
\end{addmargin}
\end{absolutelynopagebreak}

\begin{absolutelynopagebreak}
\setstretch{.7}
{\PaliGlossA{tassa te kammantā samijjheyyuṃ.}}\\
\begin{addmargin}[1em]{2em}
\setstretch{.5}
{\PaliGlossB{and his efforts proved successful.}}\\
\end{addmargin}
\end{absolutelynopagebreak}

\begin{absolutelynopagebreak}
\setstretch{.7}
{\PaliGlossA{so yāni ca porāṇāni iṇamūlāni tāni ca byantiṃ kareyya, siyā cassa uttariṃ avasiṭṭhaṃ dārabharaṇāya.}}\\
\begin{addmargin}[1em]{2em}
\setstretch{.5}
{\PaliGlossB{He would pay off the original loan and have enough left over to support his partner.}}\\
\end{addmargin}
\end{absolutelynopagebreak}

\begin{absolutelynopagebreak}
\setstretch{.7}
{\PaliGlossA{tassa evamassa:}}\\
\begin{addmargin}[1em]{2em}
\setstretch{.5}
{\PaliGlossB{Thinking about this,}}\\
\end{addmargin}
\end{absolutelynopagebreak}

\begin{absolutelynopagebreak}
\setstretch{.7}
{\PaliGlossA{‘ahaṃ kho pubbe iṇaṃ ādāya kammante payojesiṃ.}}\\
\begin{addmargin}[1em]{2em}
\setstretch{.5}
{\PaliGlossB{    -}}\\
\end{addmargin}
\end{absolutelynopagebreak}

\begin{absolutelynopagebreak}
\setstretch{.7}
{\PaliGlossA{tassa me te kammantā samijjhiṃsu.}}\\
\begin{addmargin}[1em]{2em}
\setstretch{.5}
{\PaliGlossB{    -}}\\
\end{addmargin}
\end{absolutelynopagebreak}

\begin{absolutelynopagebreak}
\setstretch{.7}
{\PaliGlossA{sohaṃ yāni ca porāṇāni iṇamūlāni tāni ca byantiṃ akāsiṃ, atthi ca me uttariṃ avasiṭṭhaṃ dārabharaṇāyā’ti.}}\\
\begin{addmargin}[1em]{2em}
\setstretch{.5}
{\PaliGlossB{    -}}\\
\end{addmargin}
\end{absolutelynopagebreak}

\begin{absolutelynopagebreak}
\setstretch{.7}
{\PaliGlossA{so tatonidānaṃ labhetha pāmojjaṃ, adhigaccheyya somanassaṃ.}}\\
\begin{addmargin}[1em]{2em}
\setstretch{.5}
{\PaliGlossB{he’d be filled with joy and happiness.}}\\
\end{addmargin}
\end{absolutelynopagebreak}

\begin{absolutelynopagebreak}
\setstretch{.7}
{\PaliGlossA{seyyathāpi, māṇava, puriso ābādhiko assa dukkhito bāḷhagilāno; bhattañcassa nacchādeyya, na cassa kāye balamattā.}}\\
\begin{addmargin}[1em]{2em}
\setstretch{.5}
{\PaliGlossB{Suppose there was a person who was sick, suffering, gravely ill. They’d lose their appetite and get physically weak.}}\\
\end{addmargin}
\end{absolutelynopagebreak}

\begin{absolutelynopagebreak}
\setstretch{.7}
{\PaliGlossA{so aparena samayena tamhā ābādhā mucceyya, bhattañcassa chādeyya, siyā cassa kāye balamattā.}}\\
\begin{addmargin}[1em]{2em}
\setstretch{.5}
{\PaliGlossB{But after some time they’d recover from that illness, and regain their appetite and their strength.}}\\
\end{addmargin}
\end{absolutelynopagebreak}

\begin{absolutelynopagebreak}
\setstretch{.7}
{\PaliGlossA{tassa evamassa:}}\\
\begin{addmargin}[1em]{2em}
\setstretch{.5}
{\PaliGlossB{Thinking about this,}}\\
\end{addmargin}
\end{absolutelynopagebreak}

\begin{absolutelynopagebreak}
\setstretch{.7}
{\PaliGlossA{‘ahaṃ kho pubbe ābādhiko ahosiṃ dukkhito bāḷhagilāno, bhattañca me nacchādesi, na ca me āsi kāye balamattā.}}\\
\begin{addmargin}[1em]{2em}
\setstretch{.5}
{\PaliGlossB{    -}}\\
\end{addmargin}
\end{absolutelynopagebreak}

\begin{absolutelynopagebreak}
\setstretch{.7}
{\PaliGlossA{somhi etarahi tamhā ābādhā mutto bhattañca me chādeti, atthi ca me kāye balamattā’ti.}}\\
\begin{addmargin}[1em]{2em}
\setstretch{.5}
{\PaliGlossB{    -}}\\
\end{addmargin}
\end{absolutelynopagebreak}

\begin{absolutelynopagebreak}
\setstretch{.7}
{\PaliGlossA{so tatonidānaṃ labhetha pāmojjaṃ, adhigaccheyya somanassaṃ.}}\\
\begin{addmargin}[1em]{2em}
\setstretch{.5}
{\PaliGlossB{they’d be filled with joy and happiness.}}\\
\end{addmargin}
\end{absolutelynopagebreak}

\begin{absolutelynopagebreak}
\setstretch{.7}
{\PaliGlossA{seyyathāpi, māṇava, puriso bandhanāgāre baddho assa.}}\\
\begin{addmargin}[1em]{2em}
\setstretch{.5}
{\PaliGlossB{Suppose a person was imprisoned in a jail.}}\\
\end{addmargin}
\end{absolutelynopagebreak}

\begin{absolutelynopagebreak}
\setstretch{.7}
{\PaliGlossA{so aparena samayena tamhā bandhanāgārā mucceyya sotthinā abbhayena, na cassa kiñci bhogānaṃ vayo.}}\\
\begin{addmargin}[1em]{2em}
\setstretch{.5}
{\PaliGlossB{But after some time they were released from jail, safe and sound, with no loss of wealth.}}\\
\end{addmargin}
\end{absolutelynopagebreak}

\begin{absolutelynopagebreak}
\setstretch{.7}
{\PaliGlossA{tassa evamassa:}}\\
\begin{addmargin}[1em]{2em}
\setstretch{.5}
{\PaliGlossB{Thinking about this,}}\\
\end{addmargin}
\end{absolutelynopagebreak}

\begin{absolutelynopagebreak}
\setstretch{.7}
{\PaliGlossA{‘ahaṃ kho pubbe bandhanāgāre baddho ahosiṃ.}}\\
\begin{addmargin}[1em]{2em}
\setstretch{.5}
{\PaliGlossB{    -}}\\
\end{addmargin}
\end{absolutelynopagebreak}

\begin{absolutelynopagebreak}
\setstretch{.7}
{\PaliGlossA{somhi etarahi tamhā bandhanāgārā mutto sotthinā abbhayena, natthi ca me kiñci bhogānaṃ vayo’ti.}}\\
\begin{addmargin}[1em]{2em}
\setstretch{.5}
{\PaliGlossB{    -}}\\
\end{addmargin}
\end{absolutelynopagebreak}

\begin{absolutelynopagebreak}
\setstretch{.7}
{\PaliGlossA{so tatonidānaṃ labhetha pāmojjaṃ, adhigaccheyya somanassaṃ.}}\\
\begin{addmargin}[1em]{2em}
\setstretch{.5}
{\PaliGlossB{they’d be filled with joy and happiness.}}\\
\end{addmargin}
\end{absolutelynopagebreak}

\begin{absolutelynopagebreak}
\setstretch{.7}
{\PaliGlossA{seyyathāpi, māṇava, puriso dāso assa anattādhīno parādhīno na yenakāmaṅgamo.}}\\
\begin{addmargin}[1em]{2em}
\setstretch{.5}
{\PaliGlossB{Suppose a person was a bondservant. They belonged to someone else and were unable to go where they wish.}}\\
\end{addmargin}
\end{absolutelynopagebreak}

\begin{absolutelynopagebreak}
\setstretch{.7}
{\PaliGlossA{so aparena samayena tamhā dāsabyā mucceyya, attādhīno aparādhīno bhujisso yenakāmaṅgamo.}}\\
\begin{addmargin}[1em]{2em}
\setstretch{.5}
{\PaliGlossB{But after some time they’d be freed from servitude and become their own master, an emancipated individual able to go where they wish.}}\\
\end{addmargin}
\end{absolutelynopagebreak}

\begin{absolutelynopagebreak}
\setstretch{.7}
{\PaliGlossA{tassa evamassa:}}\\
\begin{addmargin}[1em]{2em}
\setstretch{.5}
{\PaliGlossB{Thinking about this,}}\\
\end{addmargin}
\end{absolutelynopagebreak}

\begin{absolutelynopagebreak}
\setstretch{.7}
{\PaliGlossA{‘ahaṃ kho pubbe dāso ahosiṃ anattādhīno parādhīno na yenakāmaṅgamo.}}\\
\begin{addmargin}[1em]{2em}
\setstretch{.5}
{\PaliGlossB{    -}}\\
\end{addmargin}
\end{absolutelynopagebreak}

\begin{absolutelynopagebreak}
\setstretch{.7}
{\PaliGlossA{somhi etarahi tamhā dāsabyā mutto attādhīno aparādhīno bhujisso yenakāmaṅgamo’ti.}}\\
\begin{addmargin}[1em]{2em}
\setstretch{.5}
{\PaliGlossB{    -}}\\
\end{addmargin}
\end{absolutelynopagebreak}

\begin{absolutelynopagebreak}
\setstretch{.7}
{\PaliGlossA{so tatonidānaṃ labhetha pāmojjaṃ, adhigaccheyya somanassaṃ.}}\\
\begin{addmargin}[1em]{2em}
\setstretch{.5}
{\PaliGlossB{they’d be filled with joy and happiness.}}\\
\end{addmargin}
\end{absolutelynopagebreak}

\begin{absolutelynopagebreak}
\setstretch{.7}
{\PaliGlossA{seyyathāpi, māṇava, puriso sadhano sabhogo kantāraddhānamaggaṃ paṭipajjeyya dubbhikkhaṃ sappaṭibhayaṃ.}}\\
\begin{addmargin}[1em]{2em}
\setstretch{.5}
{\PaliGlossB{Suppose there was a person with wealth and property who was traveling along a desert road, which was perilous, with nothing to eat.}}\\
\end{addmargin}
\end{absolutelynopagebreak}

\begin{absolutelynopagebreak}
\setstretch{.7}
{\PaliGlossA{so aparena samayena taṃ kantāraṃ nitthareyya, sotthinā gāmantaṃ anupāpuṇeyya khemaṃ appaṭibhayaṃ.}}\\
\begin{addmargin}[1em]{2em}
\setstretch{.5}
{\PaliGlossB{But after some time they crossed over the desert safely, reaching the neighborhood of a village, a sanctuary free of peril.}}\\
\end{addmargin}
\end{absolutelynopagebreak}

\begin{absolutelynopagebreak}
\setstretch{.7}
{\PaliGlossA{tassa evamassa:}}\\
\begin{addmargin}[1em]{2em}
\setstretch{.5}
{\PaliGlossB{Thinking about this,}}\\
\end{addmargin}
\end{absolutelynopagebreak}

\begin{absolutelynopagebreak}
\setstretch{.7}
{\PaliGlossA{‘ahaṃ kho pubbe sadhano sabhogo kantāraddhānamaggaṃ paṭipajjiṃ dubbhikkhaṃ sappaṭibhayaṃ.}}\\
\begin{addmargin}[1em]{2em}
\setstretch{.5}
{\PaliGlossB{    -}}\\
\end{addmargin}
\end{absolutelynopagebreak}

\begin{absolutelynopagebreak}
\setstretch{.7}
{\PaliGlossA{somhi etarahi taṃ kantāraṃ nitthiṇṇo, sotthinā gāmantaṃ anuppatto khemaṃ appaṭibhayan’ti.}}\\
\begin{addmargin}[1em]{2em}
\setstretch{.5}
{\PaliGlossB{    -}}\\
\end{addmargin}
\end{absolutelynopagebreak}

\begin{absolutelynopagebreak}
\setstretch{.7}
{\PaliGlossA{so tatonidānaṃ labhetha pāmojjaṃ, adhigaccheyya somanassaṃ.}}\\
\begin{addmargin}[1em]{2em}
\setstretch{.5}
{\PaliGlossB{they’d be filled with joy and happiness.}}\\
\end{addmargin}
\end{absolutelynopagebreak}

\begin{absolutelynopagebreak}
\setstretch{.7}
{\PaliGlossA{evameva kho, māṇava, bhikkhu yathā iṇaṃ yathā rogaṃ yathā bandhanāgāraṃ yathā dāsabyaṃ yathā kantāraddhānamaggaṃ, evaṃ ime pañca nīvaraṇe appahīne attani samanupassati.}}\\
\begin{addmargin}[1em]{2em}
\setstretch{.5}
{\PaliGlossB{In the same way, as long as these five hindrances are not given up inside themselves, a mendicant regards them as a debt, a disease, a prison, slavery, and a desert crossing.}}\\
\end{addmargin}
\end{absolutelynopagebreak}

\begin{absolutelynopagebreak}
\setstretch{.7}
{\PaliGlossA{seyyathāpi, māṇava, yathā āṇaṇyaṃ yathā ārogyaṃ yathā bandhanāmokkhaṃ yathā bhujissaṃ yathā khemantabhūmiṃ.}}\\
\begin{addmargin}[1em]{2em}
\setstretch{.5}
{\PaliGlossB{But when these five hindrances are given up inside themselves, a mendicant regards this as freedom from debt, good health, release from prison, emancipation, and sanctuary.}}\\
\end{addmargin}
\end{absolutelynopagebreak}

\begin{absolutelynopagebreak}
\setstretch{.7}
{\PaliGlossA{evameva kho bhikkhu ime pañca nīvaraṇe pahīne attani samanupassati.}}\\
\begin{addmargin}[1em]{2em}
\setstretch{.5}
{\PaliGlossB{    -}}\\
\end{addmargin}
\end{absolutelynopagebreak}

\begin{absolutelynopagebreak}
\setstretch{.7}
{\PaliGlossA{tassime pañca nīvaraṇe pahīne attani samanupassato pāmojjaṃ jāyati, pamuditassa pīti jāyati, pītimanassa kāyo passambhati, passaddhakāyo sukhaṃ vedeti, sukhino cittaṃ samādhiyati.}}\\
\begin{addmargin}[1em]{2em}
\setstretch{.5}
{\PaliGlossB{Seeing that the hindrances have been given up in them, joy springs up. Being joyful, rapture springs up. When the mind is full of rapture, the body becomes tranquil. When the body is tranquil, they feel bliss. And when blissful, the mind becomes immersed.}}\\
\end{addmargin}
\end{absolutelynopagebreak}

\begin{absolutelynopagebreak}
\setstretch{.7}
{\PaliGlossA{so vivicceva kāmehi vivicca akusalehi dhammehi savitakkaṃ savicāraṃ vivekajaṃ pītisukhaṃ paṭhamaṃ jhānaṃ upasampajja viharati.}}\\
\begin{addmargin}[1em]{2em}
\setstretch{.5}
{\PaliGlossB{Quite secluded from sensual pleasures, secluded from unskillful qualities, they enter and remain in the first absorption, which has the rapture and bliss born of seclusion, while placing the mind and keeping it connected.}}\\
\end{addmargin}
\end{absolutelynopagebreak}

\begin{absolutelynopagebreak}
\setstretch{.7}
{\PaliGlossA{so imameva kāyaṃ vivekajena pītisukhena abhisandeti parisandeti paripūreti parippharati, nāssa kiñci sabbāvato kāyassa vivekajena pītisukhena apphuṭaṃ hoti.}}\\
\begin{addmargin}[1em]{2em}
\setstretch{.5}
{\PaliGlossB{They drench, steep, fill, and spread their body with rapture and bliss born of seclusion. There’s no part of the body that’s not spread with rapture and bliss born of seclusion.}}\\
\end{addmargin}
\end{absolutelynopagebreak}

\begin{absolutelynopagebreak}
\setstretch{.7}
{\PaliGlossA{seyyathāpi, māṇava, dakkho nhāpako vā nhāpakantevāsī vā kaṃsathāle nhānīyacuṇṇāni ākiritvā udakena paripphosakaṃ paripphosakaṃ sanneyya. sāyaṃ nhānīyapiṇḍi snehānugatā snehaparetā santarabāhirā phuṭā snehena, na ca paggharaṇī.}}\\
\begin{addmargin}[1em]{2em}
\setstretch{.5}
{\PaliGlossB{It’s like when a deft bathroom attendant or their apprentice pours bath powder into a bronze dish, sprinkling it little by little with water. They knead it until the ball of bath powder is soaked and saturated with moisture, spread through inside and out; yet no moisture oozes out.}}\\
\end{addmargin}
\end{absolutelynopagebreak}

\begin{absolutelynopagebreak}
\setstretch{.7}
{\PaliGlossA{evameva kho, māṇava, bhikkhu imameva kāyaṃ vivekajena pītisukhena abhisandeti parisandeti paripūreti parippharati, nāssa kiñci sabbāvato kāyassa vivekajena pītisukhena apphuṭaṃ hoti.}}\\
\begin{addmargin}[1em]{2em}
\setstretch{.5}
{\PaliGlossB{In the same way, a mendicant drenches, steeps, fills, and spreads their body with rapture and bliss born of seclusion. There’s no part of the body that’s not spread with rapture and bliss born of seclusion.}}\\
\end{addmargin}
\end{absolutelynopagebreak}

\begin{absolutelynopagebreak}
\setstretch{.7}
{\PaliGlossA{yampi, māṇava, bhikkhu vivicceva kāmehi vivicca akusalehi dhammehi savitakkaṃ savicāraṃ vivekajaṃ pītisukhaṃ paṭhamaṃ jhānaṃ upasampajja viharati.}}\\
\begin{addmargin}[1em]{2em}
\setstretch{.5}
{\PaliGlossB{    -}}\\
\end{addmargin}
\end{absolutelynopagebreak}

\begin{absolutelynopagebreak}
\setstretch{.7}
{\PaliGlossA{so imameva kāyaṃ vivekajena pītisukhena abhisandeti parisandeti paripūreti parippharati, nāssa kiñci sabbāvato kāyassa vivekajena pītisukhena apphuṭaṃ hoti.}}\\
\begin{addmargin}[1em]{2em}
\setstretch{.5}
{\PaliGlossB{    -}}\\
\end{addmargin}
\end{absolutelynopagebreak}

\begin{absolutelynopagebreak}
\setstretch{.7}
{\PaliGlossA{idampissa hoti samādhismiṃ.}}\\
\begin{addmargin}[1em]{2em}
\setstretch{.5}
{\PaliGlossB{This pertains to their immersion.}}\\
\end{addmargin}
\end{absolutelynopagebreak}

\begin{absolutelynopagebreak}
\setstretch{.7}
{\PaliGlossA{puna caparaṃ, māṇava, bhikkhu vitakkavicārānaṃ vūpasamā ajjhattaṃ sampasādanaṃ cetaso ekodibhāvaṃ avitakkaṃ avicāraṃ samādhijaṃ pītisukhaṃ dutiyaṃ jhānaṃ upasampajja viharati.}}\\
\begin{addmargin}[1em]{2em}
\setstretch{.5}
{\PaliGlossB{Furthermore, as the placing of the mind and keeping it connected are stilled, a mendicant enters and remains in the second absorption, which has the rapture and bliss born of immersion, with internal clarity and confidence, and unified mind, without placing the mind and keeping it connected.}}\\
\end{addmargin}
\end{absolutelynopagebreak}

\begin{absolutelynopagebreak}
\setstretch{.7}
{\PaliGlossA{so imameva kāyaṃ samādhijena pītisukhena abhisandeti parisandeti paripūreti parippharati, nāssa kiñci sabbāvato kāyassa samādhijena pītisukhena apphuṭaṃ hoti.}}\\
\begin{addmargin}[1em]{2em}
\setstretch{.5}
{\PaliGlossB{They drench, steep, fill, and spread their body with rapture and bliss born of immersion. There’s no part of the body that’s not spread with rapture and bliss born of immersion.}}\\
\end{addmargin}
\end{absolutelynopagebreak}

\begin{absolutelynopagebreak}
\setstretch{.7}
{\PaliGlossA{seyyathāpi, māṇava, udakarahado gambhīro ubbhidodako. tassa nevassa puratthimāya disāya udakassa āyamukhaṃ, na dakkhiṇāya disāya udakassa āyamukhaṃ, na pacchimāya disāya udakassa āyamukhaṃ, na uttarāya disāya udakassa āyamukhaṃ, devo ca na kālena kālaṃ sammā dhāraṃ anupaveccheyya. atha kho tamhāva udakarahadā sītā vāridhārā ubbhijjitvā tameva udakarahadaṃ sītena vārinā abhisandeyya parisandeyya paripūreyya paripphareyya, nāssa kiñci sabbāvato udakarahadassa sītena vārinā apphuṭaṃ assa.}}\\
\begin{addmargin}[1em]{2em}
\setstretch{.5}
{\PaliGlossB{It’s like a deep lake fed by spring water. There’s no inlet to the east, west, north, or south, and no rainfall to replenish it from time to time. But the stream of cool water welling up in the lake drenches, steeps, fills, and spreads throughout the lake. There’s no part of the lake that’s not spread through with cool water.}}\\
\end{addmargin}
\end{absolutelynopagebreak}

\begin{absolutelynopagebreak}
\setstretch{.7}
{\PaliGlossA{evameva kho, māṇava, bhikkhu … pe …}}\\
\begin{addmargin}[1em]{2em}
\setstretch{.5}
{\PaliGlossB{In the same way, a mendicant drenches, steeps, fills, and spreads their body with rapture and bliss born of immersion. There’s no part of the body that’s not spread with rapture and bliss born of immersion.}}\\
\end{addmargin}
\end{absolutelynopagebreak}

\begin{absolutelynopagebreak}
\setstretch{.7}
{\PaliGlossA{yampi, māṇava, bhikkhu vitakkavicārānaṃ vūpasamā … pe … dutiyaṃ jhānaṃ upasampajja viharati.}}\\
\begin{addmargin}[1em]{2em}
\setstretch{.5}
{\PaliGlossB{    -}}\\
\end{addmargin}
\end{absolutelynopagebreak}

\begin{absolutelynopagebreak}
\setstretch{.7}
{\PaliGlossA{so imameva kāyaṃ samādhijena pītisukhena abhisandeti parisandeti paripūreti parippharati, nāssa kiñci sabbāvato kāyassa samādhijena pītisukhena apphuṭaṃ hoti.}}\\
\begin{addmargin}[1em]{2em}
\setstretch{.5}
{\PaliGlossB{    -}}\\
\end{addmargin}
\end{absolutelynopagebreak}

\begin{absolutelynopagebreak}
\setstretch{.7}
{\PaliGlossA{idampissa hoti samādhismiṃ.}}\\
\begin{addmargin}[1em]{2em}
\setstretch{.5}
{\PaliGlossB{This pertains to their immersion.}}\\
\end{addmargin}
\end{absolutelynopagebreak}

\begin{absolutelynopagebreak}
\setstretch{.7}
{\PaliGlossA{puna caparaṃ, māṇava, bhikkhu pītiyā ca virāgā upekkhako ca viharati sato sampajāno, sukhañca kāyena paṭisaṃvedeti, yaṃ taṃ ariyā ācikkhanti: ‘upekkhako satimā sukhavihārī’ti, tatiyaṃ jhānaṃ upasampajja viharati.}}\\
\begin{addmargin}[1em]{2em}
\setstretch{.5}
{\PaliGlossB{Furthermore, with the fading away of rapture, a mendicant enters and remains in the third absorption, where they meditate with equanimity, mindful and aware, personally experiencing the bliss of which the noble ones declare, ‘Equanimous and mindful, one meditates in bliss.’}}\\
\end{addmargin}
\end{absolutelynopagebreak}

\begin{absolutelynopagebreak}
\setstretch{.7}
{\PaliGlossA{so imameva kāyaṃ nippītikena sukhena abhisandeti parisandeti paripūreti parippharati, nāssa kiñci sabbāvato kāyassa nippītikena sukhena apphuṭaṃ hoti.}}\\
\begin{addmargin}[1em]{2em}
\setstretch{.5}
{\PaliGlossB{They drench, steep, fill, and spread their body with bliss free of rapture. There’s no part of the body that’s not spread with bliss free of rapture.}}\\
\end{addmargin}
\end{absolutelynopagebreak}

\begin{absolutelynopagebreak}
\setstretch{.7}
{\PaliGlossA{seyyathāpi, māṇava, uppaliniyaṃ vā paduminiyaṃ vā puṇḍarīkiniyaṃ vā appekaccāni uppalāni vā padumāni vā puṇḍarīkāni vā udake jātāni udake saṃvaḍḍhāni udakānuggatāni antonimuggaposīni, tāni yāva caggā yāva ca mūlā sītena vārinā abhisannāni parisannāni paripūrāni paripphuṭāni, nāssa kiñci sabbāvataṃ uppalānaṃ vā padumānaṃ vā puṇḍarīkānaṃ vā sītena vārinā apphuṭaṃ assa.}}\\
\begin{addmargin}[1em]{2em}
\setstretch{.5}
{\PaliGlossB{It’s like a pool with blue water lilies, or pink or white lotuses. Some of them sprout and grow in the water without rising above it, thriving underwater. From the tip to the root they’re drenched, steeped, filled, and soaked with cool water. There’s no part of them that’s not soaked with cool water.}}\\
\end{addmargin}
\end{absolutelynopagebreak}

\begin{absolutelynopagebreak}
\setstretch{.7}
{\PaliGlossA{evameva kho, māṇava, bhikkhu … pe …}}\\
\begin{addmargin}[1em]{2em}
\setstretch{.5}
{\PaliGlossB{In the same way, a mendicant drenches, steeps, fills, and spreads their body with bliss free of rapture. There’s no part of the body that’s not spread with bliss free of rapture.}}\\
\end{addmargin}
\end{absolutelynopagebreak}

\begin{absolutelynopagebreak}
\setstretch{.7}
{\PaliGlossA{yampi, māṇava, bhikkhu pītiyā ca virāgā … pe … tatiyaṃ jhānaṃ upasampajja viharati.}}\\
\begin{addmargin}[1em]{2em}
\setstretch{.5}
{\PaliGlossB{    -}}\\
\end{addmargin}
\end{absolutelynopagebreak}

\begin{absolutelynopagebreak}
\setstretch{.7}
{\PaliGlossA{so imameva kāyaṃ nippītikena sukhena abhisandeti parisandeti paripūreti parippharati, nāssa kiñci sabbāvato kāyassa nippītikena sukhena apphuṭaṃ hoti.}}\\
\begin{addmargin}[1em]{2em}
\setstretch{.5}
{\PaliGlossB{    -}}\\
\end{addmargin}
\end{absolutelynopagebreak}

\begin{absolutelynopagebreak}
\setstretch{.7}
{\PaliGlossA{idampissa hoti samādhismiṃ.}}\\
\begin{addmargin}[1em]{2em}
\setstretch{.5}
{\PaliGlossB{This pertains to their immersion.}}\\
\end{addmargin}
\end{absolutelynopagebreak}

\begin{absolutelynopagebreak}
\setstretch{.7}
{\PaliGlossA{puna caparaṃ, māṇava, bhikkhu sukhassa ca pahānā dukkhassa ca pahānā pubbeva somanassadomanassānaṃ atthaṅgamā adukkhamasukhaṃ upekkhāsatipārisuddhiṃ catutthaṃ jhānaṃ upasampajja viharati.}}\\
\begin{addmargin}[1em]{2em}
\setstretch{.5}
{\PaliGlossB{Furthermore, giving up pleasure and pain, and ending former happiness and sadness, a mendicant enters and remains in the fourth absorption, without pleasure or pain, with pure equanimity and mindfulness.}}\\
\end{addmargin}
\end{absolutelynopagebreak}

\begin{absolutelynopagebreak}
\setstretch{.7}
{\PaliGlossA{so imameva kāyaṃ parisuddhena cetasā pariyodātena pharitvā nisinno hoti;}}\\
\begin{addmargin}[1em]{2em}
\setstretch{.5}
{\PaliGlossB{They sit spreading their body through with pure bright mind.}}\\
\end{addmargin}
\end{absolutelynopagebreak}

\begin{absolutelynopagebreak}
\setstretch{.7}
{\PaliGlossA{nāssa kiñci sabbāvato kāyassa parisuddhena cetasā pariyodātena apphuṭaṃ hoti.}}\\
\begin{addmargin}[1em]{2em}
\setstretch{.5}
{\PaliGlossB{There’s no part of the body that’s not spread with pure bright mind.}}\\
\end{addmargin}
\end{absolutelynopagebreak}

\begin{absolutelynopagebreak}
\setstretch{.7}
{\PaliGlossA{seyyathāpi, māṇava, puriso odātena vatthena sasīsaṃ pārupitvā nisinno assa, nāssa kiñci sabbāvato kāyassa odātena vatthena apphuṭaṃ assa.}}\\
\begin{addmargin}[1em]{2em}
\setstretch{.5}
{\PaliGlossB{It’s like someone sitting wrapped from head to foot with white cloth. There’s no part of the body that’s not spread over with white cloth.}}\\
\end{addmargin}
\end{absolutelynopagebreak}

\begin{absolutelynopagebreak}
\setstretch{.7}
{\PaliGlossA{evameva kho, māṇava, bhikkhu … pe …}}\\
\begin{addmargin}[1em]{2em}
\setstretch{.5}
{\PaliGlossB{In the same way, a mendicant sits spreading their body through with pure bright mind. There's no part of their body that's not spread with pure bright mind.}}\\
\end{addmargin}
\end{absolutelynopagebreak}

\begin{absolutelynopagebreak}
\setstretch{.7}
{\PaliGlossA{yampi, māṇava, bhikkhu sukhassa ca pahānā dukkhassa ca pahānā pubbeva somanassadomanassānaṃ atthaṅgamā adukkhamasukhaṃ upekkhāsatipārisuddhiṃ catutthaṃ jhānaṃ upasampajja viharati.}}\\
\begin{addmargin}[1em]{2em}
\setstretch{.5}
{\PaliGlossB{    -}}\\
\end{addmargin}
\end{absolutelynopagebreak}

\begin{absolutelynopagebreak}
\setstretch{.7}
{\PaliGlossA{so imameva kāyaṃ parisuddhena cetasā pariyodātena pharitvā nisinno hoti; nāssa kiñci sabbāvato kāyassa parisuddhena cetasā pariyodātena apphuṭaṃ hoti.}}\\
\begin{addmargin}[1em]{2em}
\setstretch{.5}
{\PaliGlossB{    -}}\\
\end{addmargin}
\end{absolutelynopagebreak}

\begin{absolutelynopagebreak}
\setstretch{.7}
{\PaliGlossA{idampissa hoti samādhismiṃ.}}\\
\begin{addmargin}[1em]{2em}
\setstretch{.5}
{\PaliGlossB{This pertains to their immersion.}}\\
\end{addmargin}
\end{absolutelynopagebreak}

\begin{absolutelynopagebreak}
\setstretch{.7}
{\PaliGlossA{ayaṃ kho so, māṇava, ariyo samādhikkhandho yassa so bhagavā vaṇṇavādī ahosi, yattha ca imaṃ janataṃ samādapesi nivesesi patiṭṭhāpesi.}}\\
\begin{addmargin}[1em]{2em}
\setstretch{.5}
{\PaliGlossB{This is that noble spectrum of immersion that the Buddha praised.}}\\
\end{addmargin}
\end{absolutelynopagebreak}

\begin{absolutelynopagebreak}
\setstretch{.7}
{\PaliGlossA{atthi cevettha uttarikaraṇīyan”ti.}}\\
\begin{addmargin}[1em]{2em}
\setstretch{.5}
{\PaliGlossB{But there is still more to be done.”}}\\
\end{addmargin}
\end{absolutelynopagebreak}

\begin{absolutelynopagebreak}
\setstretch{.7}
{\PaliGlossA{“acchariyaṃ, bho ānanda, abbhutaṃ, bho ānanda.}}\\
\begin{addmargin}[1em]{2em}
\setstretch{.5}
{\PaliGlossB{“It’s incredible, Master Ānanda, it’s amazing!}}\\
\end{addmargin}
\end{absolutelynopagebreak}

\begin{absolutelynopagebreak}
\setstretch{.7}
{\PaliGlossA{so cāyaṃ, bho ānanda, ariyo samādhikkhandho paripuṇṇo, no aparipuṇṇo.}}\\
\begin{addmargin}[1em]{2em}
\setstretch{.5}
{\PaliGlossB{This noble spectrum of immersion is complete, not lacking anything!}}\\
\end{addmargin}
\end{absolutelynopagebreak}

\begin{absolutelynopagebreak}
\setstretch{.7}
{\PaliGlossA{evaṃ paripuṇṇañcāhaṃ, bho ānanda, ariyaṃ samādhikkhandhaṃ ito bahiddhā aññesu samaṇabrāhmaṇesu na samanupassāmi.}}\\
\begin{addmargin}[1em]{2em}
\setstretch{.5}
{\PaliGlossB{Such a complete spectrum of immersion cannot be seen among the other ascetics and brahmins.}}\\
\end{addmargin}
\end{absolutelynopagebreak}

\begin{absolutelynopagebreak}
\setstretch{.7}
{\PaliGlossA{evaṃ paripuṇṇañca, bho ānanda, ariyaṃ samādhikkhandhaṃ ito bahiddhā aññe samaṇabrāhmaṇā attani samanupasseyyuṃ, te tāvatakeneva attamanā assu:}}\\
\begin{addmargin}[1em]{2em}
\setstretch{.5}
{\PaliGlossB{Were other ascetics and brahmins to see such a complete spectrum of noble immersion in themselves, they’d be delighted with just that much:}}\\
\end{addmargin}
\end{absolutelynopagebreak}

\begin{absolutelynopagebreak}
\setstretch{.7}
{\PaliGlossA{‘alamettāvatā, katamettāvatā, anuppatto no sāmaññattho, natthi no kiñci uttarikaraṇīyan’ti.}}\\
\begin{addmargin}[1em]{2em}
\setstretch{.5}
{\PaliGlossB{‘At this point it’s enough; at this point our work is done. We’ve reached the goal of our ascetic life. There is nothing more to be done.’}}\\
\end{addmargin}
\end{absolutelynopagebreak}

\begin{absolutelynopagebreak}
\setstretch{.7}
{\PaliGlossA{atha ca pana bhavaṃ ānando evamāha:}}\\
\begin{addmargin}[1em]{2em}
\setstretch{.5}
{\PaliGlossB{And yet you say:}}\\
\end{addmargin}
\end{absolutelynopagebreak}

\begin{absolutelynopagebreak}
\setstretch{.7}
{\PaliGlossA{‘atthi cevettha uttarikaraṇīyan’ti.}}\\
\begin{addmargin}[1em]{2em}
\setstretch{.5}
{\PaliGlossB{‘But there is still more to be done.’}}\\
\end{addmargin}
\end{absolutelynopagebreak}

\begin{absolutelynopagebreak}
\setstretch{.7}
{\PaliGlossA{3. paññākkhandha}}\\
\begin{addmargin}[1em]{2em}
\setstretch{.5}
{\PaliGlossB{3. The Spectrum of Wisdom}}\\
\end{addmargin}
\end{absolutelynopagebreak}

\begin{absolutelynopagebreak}
\setstretch{.7}
{\PaliGlossA{katamo pana so, bho ānanda, ariyo paññākkhandho, yassa bho bhavaṃ gotamo vaṇṇavādī ahosi, yattha ca imaṃ janataṃ samādapesi nivesesi patiṭṭhāpesī”ti?}}\\
\begin{addmargin}[1em]{2em}
\setstretch{.5}
{\PaliGlossB{But what, Master Ānanda, was that noble spectrum of wisdom that the Buddha praised?”}}\\
\end{addmargin}
\end{absolutelynopagebreak}

\begin{absolutelynopagebreak}
\setstretch{.7}
{\PaliGlossA{“so evaṃ samāhite citte parisuddhe pariyodāte anaṅgaṇe vigatūpakkilese mudubhūte kammaniye ṭhite āneñjappatte ñāṇadassanāya cittaṃ abhinīharati abhininnāmeti.}}\\
\begin{addmargin}[1em]{2em}
\setstretch{.5}
{\PaliGlossB{“When their mind has become immersed in samādhi like this—purified, bright, flawless, rid of corruptions, pliable, workable, steady, and imperturbable—they extend it and project it toward knowledge and vision.}}\\
\end{addmargin}
\end{absolutelynopagebreak}

\begin{absolutelynopagebreak}
\setstretch{.7}
{\PaliGlossA{so evaṃ pajānāti:}}\\
\begin{addmargin}[1em]{2em}
\setstretch{.5}
{\PaliGlossB{They understand:}}\\
\end{addmargin}
\end{absolutelynopagebreak}

\begin{absolutelynopagebreak}
\setstretch{.7}
{\PaliGlossA{‘ayaṃ kho me kāyo rūpī cātumahābhūtiko mātāpettikasambhavo odanakummāsūpacayo aniccucchādanaparimaddanabhedanaviddhaṃsanadhammo; idañca pana me viññāṇaṃ ettha sitaṃ ettha paṭibaddhan’ti.}}\\
\begin{addmargin}[1em]{2em}
\setstretch{.5}
{\PaliGlossB{‘This body of mine is physical. It’s made up of the four primary elements, produced by mother and father, built up from rice and porridge, liable to impermanence, to wearing away and erosion, to breaking up and destruction. And this consciousness of mine is attached to it, tied to it.’}}\\
\end{addmargin}
\end{absolutelynopagebreak}

\begin{absolutelynopagebreak}
\setstretch{.7}
{\PaliGlossA{seyyathāpi, māṇava, maṇi veḷuriyo subho jātimā aṭṭhaṃso suparikammakato accho vippasanno anāvilo sabbākārasampanno. tatrāssa suttaṃ āvutaṃ nīlaṃ vā pītaṃ vā lohitaṃ vā odātaṃ vā paṇḍusuttaṃ vā.}}\\
\begin{addmargin}[1em]{2em}
\setstretch{.5}
{\PaliGlossB{Suppose there was a beryl gem that was naturally beautiful, eight-faceted, well-worked, transparent, clear, and unclouded, endowed with all good qualities. And it was strung with a thread of blue, yellow, red, white, or golden brown.}}\\
\end{addmargin}
\end{absolutelynopagebreak}

\begin{absolutelynopagebreak}
\setstretch{.7}
{\PaliGlossA{tamenaṃ cakkhumā puriso hatthe karitvā paccavekkheyya: ‘ayaṃ kho maṇi veḷuriyo subho jātimā aṭṭhaṃso suparikammakato accho vippasanno anāvilo sabbākārasampanno. tatridaṃ suttaṃ āvutaṃ nīlaṃ vā pītaṃ vā lohitaṃ vā odātaṃ vā paṇḍusuttaṃ vā’ti.}}\\
\begin{addmargin}[1em]{2em}
\setstretch{.5}
{\PaliGlossB{And someone with good eyesight were to take it in their hand and examine it: ‘This beryl gem is naturally beautiful, eight-faceted, well-worked, transparent, clear, and unclouded, endowed with all good qualities. And it’s strung with a thread of blue, yellow, red, white, or golden brown.’}}\\
\end{addmargin}
\end{absolutelynopagebreak}

\begin{absolutelynopagebreak}
\setstretch{.7}
{\PaliGlossA{evameva kho, māṇava, bhikkhu evaṃ samāhite citte parisuddhe pariyodāte anaṅgaṇe vigatūpakkilese mudubhūte kammaniye ṭhite āneñjappatte ñāṇadassanāya cittaṃ abhinīharati abhininnāmeti.}}\\
\begin{addmargin}[1em]{2em}
\setstretch{.5}
{\PaliGlossB{In the same way, when their mind has become immersed in samādhi like this—purified, bright, flawless, rid of corruptions, pliable, workable, steady, and imperturbable—they extend it and project it toward knowledge and vision.}}\\
\end{addmargin}
\end{absolutelynopagebreak}

\begin{absolutelynopagebreak}
\setstretch{.7}
{\PaliGlossA{so evaṃ pajānāti:}}\\
\begin{addmargin}[1em]{2em}
\setstretch{.5}
{\PaliGlossB{    -}}\\
\end{addmargin}
\end{absolutelynopagebreak}

\begin{absolutelynopagebreak}
\setstretch{.7}
{\PaliGlossA{‘ayaṃ kho me kāyo rūpī cātumahābhūtiko mātāpettikasambhavo odanakummāsūpacayo aniccucchādanaparimaddanabhedanaviddhaṃsanadhammo. idañca pana me viññāṇaṃ ettha sitaṃ ettha paṭibaddhan’ti.}}\\
\begin{addmargin}[1em]{2em}
\setstretch{.5}
{\PaliGlossB{    -}}\\
\end{addmargin}
\end{absolutelynopagebreak}

\begin{absolutelynopagebreak}
\setstretch{.7}
{\PaliGlossA{yampi, māṇava, bhikkhu evaṃ samāhite citte … pe …}}\\
\begin{addmargin}[1em]{2em}
\setstretch{.5}
{\PaliGlossB{    -}}\\
\end{addmargin}
\end{absolutelynopagebreak}

\begin{absolutelynopagebreak}
\setstretch{.7}
{\PaliGlossA{āneñjappatte ñāṇadassanāya cittaṃ abhinīharati abhininnāmeti.}}\\
\begin{addmargin}[1em]{2em}
\setstretch{.5}
{\PaliGlossB{    -}}\\
\end{addmargin}
\end{absolutelynopagebreak}

\begin{absolutelynopagebreak}
\setstretch{.7}
{\PaliGlossA{so evaṃ pajānāti … pe …}}\\
\begin{addmargin}[1em]{2em}
\setstretch{.5}
{\PaliGlossB{    -}}\\
\end{addmargin}
\end{absolutelynopagebreak}

\begin{absolutelynopagebreak}
\setstretch{.7}
{\PaliGlossA{ettha paṭibaddhanti.}}\\
\begin{addmargin}[1em]{2em}
\setstretch{.5}
{\PaliGlossB{    -}}\\
\end{addmargin}
\end{absolutelynopagebreak}

\begin{absolutelynopagebreak}
\setstretch{.7}
{\PaliGlossA{idampissa hoti paññāya.}}\\
\begin{addmargin}[1em]{2em}
\setstretch{.5}
{\PaliGlossB{This pertains to their wisdom.}}\\
\end{addmargin}
\end{absolutelynopagebreak}

\begin{absolutelynopagebreak}
\setstretch{.7}
{\PaliGlossA{so evaṃ samāhite citte parisuddhe pariyodāte anaṅgaṇe vigatūpakkilese mudubhūte kammaniye ṭhite āneñjappatte manomayaṃ kāyaṃ abhinimmānāya cittaṃ abhinīharati abhininnāmeti.}}\\
\begin{addmargin}[1em]{2em}
\setstretch{.5}
{\PaliGlossB{When their mind has become immersed in samādhi like this—purified, bright, flawless, rid of corruptions, pliable, workable, steady, and imperturbable—they extend it and project it toward the creation of a mind-made body.}}\\
\end{addmargin}
\end{absolutelynopagebreak}

\begin{absolutelynopagebreak}
\setstretch{.7}
{\PaliGlossA{so imamhā kāyā aññaṃ kāyaṃ abhinimmināti rūpiṃ manomayaṃ sabbaṅgapaccaṅgiṃ ahīnindriyaṃ.}}\\
\begin{addmargin}[1em]{2em}
\setstretch{.5}
{\PaliGlossB{From this body they create another body, physical, mind-made, complete in all its various parts, not deficient in any faculty.}}\\
\end{addmargin}
\end{absolutelynopagebreak}

\begin{absolutelynopagebreak}
\setstretch{.7}
{\PaliGlossA{seyyathāpi, māṇava, puriso muñjamhā īsikaṃ pavāheyya.}}\\
\begin{addmargin}[1em]{2em}
\setstretch{.5}
{\PaliGlossB{Suppose a person was to draw a reed out from its sheath.}}\\
\end{addmargin}
\end{absolutelynopagebreak}

\begin{absolutelynopagebreak}
\setstretch{.7}
{\PaliGlossA{tassa evamassa:}}\\
\begin{addmargin}[1em]{2em}
\setstretch{.5}
{\PaliGlossB{They’d think:}}\\
\end{addmargin}
\end{absolutelynopagebreak}

\begin{absolutelynopagebreak}
\setstretch{.7}
{\PaliGlossA{‘ayaṃ muñjo ayaṃ īsikā; añño muñjo aññā īsikā; muñjamhā tveva īsikā pavāḷhā’ti.}}\\
\begin{addmargin}[1em]{2em}
\setstretch{.5}
{\PaliGlossB{‘This is the reed, this is the sheath. The reed and the sheath are different things. The reed has been drawn out from the sheath.’}}\\
\end{addmargin}
\end{absolutelynopagebreak}

\begin{absolutelynopagebreak}
\setstretch{.7}
{\PaliGlossA{seyyathā vā pana, māṇava, puriso asiṃ kosiyā pavāheyya.}}\\
\begin{addmargin}[1em]{2em}
\setstretch{.5}
{\PaliGlossB{Or suppose a person was to draw a sword out from its scabbard.}}\\
\end{addmargin}
\end{absolutelynopagebreak}

\begin{absolutelynopagebreak}
\setstretch{.7}
{\PaliGlossA{tassa evamassa:}}\\
\begin{addmargin}[1em]{2em}
\setstretch{.5}
{\PaliGlossB{They’d think:}}\\
\end{addmargin}
\end{absolutelynopagebreak}

\begin{absolutelynopagebreak}
\setstretch{.7}
{\PaliGlossA{‘ayaṃ asi, ayaṃ kosi; añño asi, aññā kosi; kosiyā tveva asi pavāḷho’ti.}}\\
\begin{addmargin}[1em]{2em}
\setstretch{.5}
{\PaliGlossB{‘This is the sword, this is the scabbard. The sword and the scabbard are different things. The sword has been drawn out from the scabbard.’}}\\
\end{addmargin}
\end{absolutelynopagebreak}

\begin{absolutelynopagebreak}
\setstretch{.7}
{\PaliGlossA{seyyathā vā pana, māṇava, puriso ahiṃ karaṇḍā uddhareyya.}}\\
\begin{addmargin}[1em]{2em}
\setstretch{.5}
{\PaliGlossB{Or suppose a person was to draw a snake out from its slough.}}\\
\end{addmargin}
\end{absolutelynopagebreak}

\begin{absolutelynopagebreak}
\setstretch{.7}
{\PaliGlossA{tassa evamassa:}}\\
\begin{addmargin}[1em]{2em}
\setstretch{.5}
{\PaliGlossB{They’d think:}}\\
\end{addmargin}
\end{absolutelynopagebreak}

\begin{absolutelynopagebreak}
\setstretch{.7}
{\PaliGlossA{‘ayaṃ ahi, ayaṃ karaṇḍo; añño ahi, añño karaṇḍo; karaṇḍā tveva ahi ubbhato’ti.}}\\
\begin{addmargin}[1em]{2em}
\setstretch{.5}
{\PaliGlossB{‘This is the snake, this is the slough. The snake and the slough are different things. The snake has been drawn out from the slough.’}}\\
\end{addmargin}
\end{absolutelynopagebreak}

\begin{absolutelynopagebreak}
\setstretch{.7}
{\PaliGlossA{evameva kho, māṇava, bhikkhu … pe …}}\\
\begin{addmargin}[1em]{2em}
\setstretch{.5}
{\PaliGlossB{In the same way, when their mind has become immersed in samādhi like this—purified, bright, flawless, rid of corruptions, pliable, workable, steady, and imperturbable—they extend it and project it toward the creation of a mind-made body.}}\\
\end{addmargin}
\end{absolutelynopagebreak}

\begin{absolutelynopagebreak}
\setstretch{.7}
{\PaliGlossA{yampi, māṇava, bhikkhu evaṃ samāhite citte parisuddhe pariyodāte anaṅgaṇe vigatūpakkilese mudubhūte kammaniye ṭhite āneñjappatte manomayaṃ kāyaṃ abhinimmānāya cittaṃ abhinīharati abhininnāmeti … pe ….}}\\
\begin{addmargin}[1em]{2em}
\setstretch{.5}
{\PaliGlossB{    -}}\\
\end{addmargin}
\end{absolutelynopagebreak}

\begin{absolutelynopagebreak}
\setstretch{.7}
{\PaliGlossA{idampissa hoti paññāya.}}\\
\begin{addmargin}[1em]{2em}
\setstretch{.5}
{\PaliGlossB{This pertains to their wisdom.}}\\
\end{addmargin}
\end{absolutelynopagebreak}

\begin{absolutelynopagebreak}
\setstretch{.7}
{\PaliGlossA{so evaṃ samāhite citte parisuddhe pariyodāte anaṅgaṇe vigatūpakkilese mudubhūte kammaniye ṭhite āneñjappatte iddhividhāya cittaṃ abhinīharati abhininnāmeti.}}\\
\begin{addmargin}[1em]{2em}
\setstretch{.5}
{\PaliGlossB{When their mind has become immersed in samādhi like this—purified, bright, flawless, rid of corruptions, pliable, workable, steady, and imperturbable—they extend it and project it toward psychic power.}}\\
\end{addmargin}
\end{absolutelynopagebreak}

\begin{absolutelynopagebreak}
\setstretch{.7}
{\PaliGlossA{so anekavihitaṃ iddhividhaṃ paccanubhoti—ekopi hutvā bahudhā hoti, bahudhāpi hutvā eko hoti; āvibhāvaṃ tirobhāvaṃ tirokuṭṭaṃ tiropākāraṃ tiropabbataṃ asajjamāno gacchati seyyathāpi ākāse; pathaviyāpi ummujjanimujjaṃ karoti seyyathāpi udake; udakepi abhijjamāne gacchati seyyathāpi pathaviyaṃ; ākāsepi pallaṅkena kamati seyyathāpi pakkhī sakuṇo; imepi candimasūriye evaṃ mahiddhike evaṃ mahānubhāve pāṇinā parāmasati parimajjati; yāva brahmalokāpi kāyena vasaṃ vatteti.}}\\
\begin{addmargin}[1em]{2em}
\setstretch{.5}
{\PaliGlossB{They wield the many kinds of psychic power: multiplying themselves and becoming one again; going unimpeded through a wall, a rampart, or a mountain as if through space; diving in and out of the earth as if it were water; walking on water as if it were earth; flying cross-legged through the sky like a bird; touching and stroking with the hand the sun and moon, so mighty and powerful; controlling the body as far as the Brahmā realm.}}\\
\end{addmargin}
\end{absolutelynopagebreak}

\begin{absolutelynopagebreak}
\setstretch{.7}
{\PaliGlossA{seyyathāpi, māṇava, dakkho kumbhakāro vā kumbhakārantevāsī vā suparikammakatāya mattikāya yaññadeva bhājanavikatiṃ ākaṅkheyya, taṃ tadeva kareyya abhinipphādeyya.}}\\
\begin{addmargin}[1em]{2em}
\setstretch{.5}
{\PaliGlossB{Suppose an expert potter or their apprentice had some well-prepared clay. They could produce any kind of pot that they like.}}\\
\end{addmargin}
\end{absolutelynopagebreak}

\begin{absolutelynopagebreak}
\setstretch{.7}
{\PaliGlossA{seyyathā vā pana, māṇava, dakkho dantakāro vā dantakārantevāsī vā suparikammakatasmiṃ dantasmiṃ yaññadeva dantavikatiṃ ākaṅkheyya, taṃ tadeva kareyya abhinipphādeyya.}}\\
\begin{addmargin}[1em]{2em}
\setstretch{.5}
{\PaliGlossB{Or suppose an expert ivory-carver or their apprentice had some well-prepared ivory. They could produce any kind of ivory item that they like.}}\\
\end{addmargin}
\end{absolutelynopagebreak}

\begin{absolutelynopagebreak}
\setstretch{.7}
{\PaliGlossA{seyyathā vā pana, māṇava, dakkho suvaṇṇakāro vā suvaṇṇakārantevāsī vā suparikammakatasmiṃ suvaṇṇasmiṃ yaññadeva suvaṇṇavikatiṃ ākaṅkheyya, taṃ tadeva kareyya abhinipphādeyya.}}\\
\begin{addmargin}[1em]{2em}
\setstretch{.5}
{\PaliGlossB{Or suppose an expert goldsmith or their apprentice had some well-prepared gold. They could produce any kind of gold item that they like.}}\\
\end{addmargin}
\end{absolutelynopagebreak}

\begin{absolutelynopagebreak}
\setstretch{.7}
{\PaliGlossA{evameva kho, māṇava, bhikkhu … pe … yampi māṇava bhikkhu evaṃ samāhite citte parisuddhe pariyodāte anaṅgaṇe vigatūpakkilese mudubhūte kammaniye ṭhite āneñjappatte iddhividhāya cittaṃ abhinīharati abhininnāmeti.}}\\
\begin{addmargin}[1em]{2em}
\setstretch{.5}
{\PaliGlossB{In the same way, when their mind has become immersed in samādhi like this—purified, bright, flawless, rid of corruptions, pliable, workable, steady, and imperturbable—they extend it and project it toward psychic power.}}\\
\end{addmargin}
\end{absolutelynopagebreak}

\begin{absolutelynopagebreak}
\setstretch{.7}
{\PaliGlossA{so anekavihitaṃ iddhividhaṃ paccanubhoti—ekopi hutvā bahudhā hoti … pe … yāva brahmalokāpi kāyena vasaṃ vatteti.}}\\
\begin{addmargin}[1em]{2em}
\setstretch{.5}
{\PaliGlossB{    -}}\\
\end{addmargin}
\end{absolutelynopagebreak}

\begin{absolutelynopagebreak}
\setstretch{.7}
{\PaliGlossA{idampissa hoti paññāya.}}\\
\begin{addmargin}[1em]{2em}
\setstretch{.5}
{\PaliGlossB{This pertains to their wisdom.}}\\
\end{addmargin}
\end{absolutelynopagebreak}

\begin{absolutelynopagebreak}
\setstretch{.7}
{\PaliGlossA{so evaṃ samāhite citte … pe … āneñjappatte dibbāya sotadhātuyā cittaṃ abhinīharati abhininnāmeti}}\\
\begin{addmargin}[1em]{2em}
\setstretch{.5}
{\PaliGlossB{When their mind has become immersed in samādhi like this—purified, bright, flawless, rid of corruptions, pliable, workable, steady, and imperturbable—they extend it and project it toward clairaudience.}}\\
\end{addmargin}
\end{absolutelynopagebreak}

\begin{absolutelynopagebreak}
\setstretch{.7}
{\PaliGlossA{so dibbāya sotadhātuyā visuddhāya atikkantamānusikāya ubho sadde suṇāti dibbe ca mānuse ca ye dūre santike ca.}}\\
\begin{addmargin}[1em]{2em}
\setstretch{.5}
{\PaliGlossB{With clairaudience that is purified and superhuman, they hear both kinds of sounds, human and divine, whether near or far.}}\\
\end{addmargin}
\end{absolutelynopagebreak}

\begin{absolutelynopagebreak}
\setstretch{.7}
{\PaliGlossA{seyyathāpi, māṇava, puriso addhānamaggappaṭipanno. so suṇeyya bherisaddampi mudiṅgasaddampi saṅkhapaṇavadindimasaddampi. tassa evamassa—bherisaddo itipi mudiṅgasaddo itipi saṅkhapaṇavadindimasaddo itipi.}}\\
\begin{addmargin}[1em]{2em}
\setstretch{.5}
{\PaliGlossB{Suppose there was a person traveling along the road. They’d hear the sound of drums, clay drums, horns, kettledrums, and tom-toms. They’d think: ‘That’s the sound of drums,’ and ‘that’s the sound of clay-drums,’ and ‘that’s the sound of horns, kettledrums, and tom-toms.’}}\\
\end{addmargin}
\end{absolutelynopagebreak}

\begin{absolutelynopagebreak}
\setstretch{.7}
{\PaliGlossA{evameva kho, māṇava, bhikkhu … pe …}}\\
\begin{addmargin}[1em]{2em}
\setstretch{.5}
{\PaliGlossB{In the same way, when their mind has become immersed in samādhi like this—purified, bright, flawless, rid of corruptions, pliable, workable, steady, and imperturbable—they extend it and project it toward clairaudience.}}\\
\end{addmargin}
\end{absolutelynopagebreak}

\begin{absolutelynopagebreak}
\setstretch{.7}
{\PaliGlossA{yampi, māṇava, bhikkhu evaṃ samāhite citte … pe … āneñjappatte dibbāya sotadhātuyā cittaṃ abhinīharati abhininnāmeti.}}\\
\begin{addmargin}[1em]{2em}
\setstretch{.5}
{\PaliGlossB{    -}}\\
\end{addmargin}
\end{absolutelynopagebreak}

\begin{absolutelynopagebreak}
\setstretch{.7}
{\PaliGlossA{so dibbāya sotadhātuyā visuddhāya atikkantamānusikāya ubho sadde suṇāti dibbe ca mānuse ca ye dūre santike ca.}}\\
\begin{addmargin}[1em]{2em}
\setstretch{.5}
{\PaliGlossB{    -}}\\
\end{addmargin}
\end{absolutelynopagebreak}

\begin{absolutelynopagebreak}
\setstretch{.7}
{\PaliGlossA{idampissa hoti paññāya.}}\\
\begin{addmargin}[1em]{2em}
\setstretch{.5}
{\PaliGlossB{This pertains to their wisdom.}}\\
\end{addmargin}
\end{absolutelynopagebreak}

\begin{absolutelynopagebreak}
\setstretch{.7}
{\PaliGlossA{so evaṃ samāhite citte parisuddhe pariyodāte anaṅgaṇe vigatūpakkilese mudubhūte kammaniye ṭhite āneñjappatte cetopariyañāṇāya cittaṃ abhinīharati abhininnāmeti.}}\\
\begin{addmargin}[1em]{2em}
\setstretch{.5}
{\PaliGlossB{When their mind has become immersed in samādhi like this—purified, bright, flawless, rid of corruptions, pliable, workable, steady, and imperturbable—they extend it and project it toward comprehending the minds of others.}}\\
\end{addmargin}
\end{absolutelynopagebreak}

\begin{absolutelynopagebreak}
\setstretch{.7}
{\PaliGlossA{so parasattānaṃ parapuggalānaṃ cetasā ceto paricca pajānāti, sarāgaṃ vā cittaṃ ‘sarāgaṃ cittan’ti pajānāti, vītarāgaṃ vā cittaṃ ‘vītarāgaṃ cittan’ti pajānāti, sadosaṃ vā cittaṃ ‘sadosaṃ cittan’ti pajānāti, vītadosaṃ vā cittaṃ ‘vītadosaṃ cittan’ti pajānāti, samohaṃ vā cittaṃ ‘samohaṃ cittan’ti pajānāti, vītamohaṃ vā cittaṃ ‘vītamohaṃ cittan’ti pajānāti, saṅkhittaṃ vā cittaṃ ‘saṅkhittaṃ cittan’ti pajānāti, vikkhittaṃ vā cittaṃ ‘vikkhittaṃ cittan’ti pajānāti, mahaggataṃ vā cittaṃ ‘mahaggataṃ cittan’ti pajānāti, amahaggataṃ vā cittaṃ ‘amahaggataṃ cittan’ti pajānāti, sauttaraṃ vā cittaṃ ‘sauttaraṃ cittan’ti pajānāti, anuttaraṃ vā cittaṃ ‘anuttaraṃ cittan’ti pajānāti, samāhitaṃ vā cittaṃ ‘samāhitaṃ cittan’ti pajānāti, asamāhitaṃ vā cittaṃ ‘asamāhitaṃ cittan’ti pajānāti, vimuttaṃ vā cittaṃ ‘vimuttaṃ cittan’ti pajānāti, avimuttaṃ vā cittaṃ ‘avimuttaṃ cittan’ti pajānāti.}}\\
\begin{addmargin}[1em]{2em}
\setstretch{.5}
{\PaliGlossB{They understand mind with greed as ‘mind with greed’, and mind without greed as ‘mind without greed’. They understand mind with hate … mind without hate … mind with delusion … mind without delusion … constricted mind … scattered mind … expansive mind … unexpansive mind … mind that is not supreme … mind that is supreme … immersed mind … unimmersed mind … freed mind … They understand unfreed mind as ‘unfreed mind’.}}\\
\end{addmargin}
\end{absolutelynopagebreak}

\begin{absolutelynopagebreak}
\setstretch{.7}
{\PaliGlossA{seyyathāpi, māṇava, itthī vā puriso vā daharo yuvā maṇḍanajātiko ādāse vā parisuddhe pariyodāte acche vā udakapatte sakaṃ mukhanimittaṃ paccavekkhamāno sakaṇikaṃ vā sakaṇikanti jāneyya, akaṇikaṃ vā akaṇikanti jāneyya.}}\\
\begin{addmargin}[1em]{2em}
\setstretch{.5}
{\PaliGlossB{Suppose there was a woman or man who was young, youthful, and fond of adornments, and they check their own reflection in a clean bright mirror or a clear bowl of water. If they had a spot they’d know ‘I have a spot,’ and if they had no spots they’d know ‘I have no spots.’}}\\
\end{addmargin}
\end{absolutelynopagebreak}

\begin{absolutelynopagebreak}
\setstretch{.7}
{\PaliGlossA{evameva kho, māṇava, bhikkhu … pe …}}\\
\begin{addmargin}[1em]{2em}
\setstretch{.5}
{\PaliGlossB{In the same way, when their mind has become immersed in samādhi like this—purified, bright, flawless, rid of corruptions, pliable, workable, steady, and imperturbable—they extend it and project it toward comprehending the minds of others.}}\\
\end{addmargin}
\end{absolutelynopagebreak}

\begin{absolutelynopagebreak}
\setstretch{.7}
{\PaliGlossA{yampi, māṇava, bhikkhu evaṃ samāhite … pe … āneñjappatte cetopariyañāṇāya cittaṃ abhinīharati abhininnāmeti.}}\\
\begin{addmargin}[1em]{2em}
\setstretch{.5}
{\PaliGlossB{    -}}\\
\end{addmargin}
\end{absolutelynopagebreak}

\begin{absolutelynopagebreak}
\setstretch{.7}
{\PaliGlossA{so parasattānaṃ parapuggalānaṃ cetasā ceto paricca pajānāti, sarāgaṃ vā cittaṃ sarāgaṃ cittanti pajānāti … pe … avimuttaṃ vā cittaṃ avimuttaṃ cittanti pajānāti.}}\\
\begin{addmargin}[1em]{2em}
\setstretch{.5}
{\PaliGlossB{    -}}\\
\end{addmargin}
\end{absolutelynopagebreak}

\begin{absolutelynopagebreak}
\setstretch{.7}
{\PaliGlossA{idampissa hoti paññāya.}}\\
\begin{addmargin}[1em]{2em}
\setstretch{.5}
{\PaliGlossB{This pertains to their wisdom.}}\\
\end{addmargin}
\end{absolutelynopagebreak}

\begin{absolutelynopagebreak}
\setstretch{.7}
{\PaliGlossA{so evaṃ samāhite citte … pe … āneñjappatte pubbenivāsānussatiñāṇāya cittaṃ abhinīharati abhininnāmeti}}\\
\begin{addmargin}[1em]{2em}
\setstretch{.5}
{\PaliGlossB{When their mind has become immersed in samādhi like this—purified, bright, flawless, rid of corruptions, pliable, workable, steady, and imperturbable—they extend it and project it toward recollection of past lives.}}\\
\end{addmargin}
\end{absolutelynopagebreak}

\begin{absolutelynopagebreak}
\setstretch{.7}
{\PaliGlossA{so anekavihitaṃ pubbenivāsaṃ anussarati. seyyathidaṃ—ekampi jātiṃ dvepi jātiyo tissopi jātiyo catassopi jātiyo pañcapi jātiyo dasapi jātiyo vīsampi jātiyo tiṃsampi jātiyo cattālīsampi jātiyo paññāsampi jātiyo jātisatampi jātisahassampi jātisatasahassampi anekepi saṃvaṭṭakappe anekepi vivaṭṭakappe anekepi saṃvaṭṭavivaṭṭakappe: ‘amutrāsiṃ evaṃnāmo evaṅgotto evaṃvaṇṇo evamāhāro evaṃsukhadukkhappaṭisaṃvedī evamāyupariyanto. so tato cuto amutra udapādiṃ; tatrāpāsiṃ evaṃnāmo evaṅgotto evaṃvaṇṇo evamāhāro evaṃsukhadukkhappaṭisaṃvedī evamāyupariyanto; so tato cuto idhūpapanno’ti. iti sākāraṃ sauddesaṃ anekavihitaṃ pubbenivāsaṃ anussarati.}}\\
\begin{addmargin}[1em]{2em}
\setstretch{.5}
{\PaliGlossB{They recollect many kinds of past lives, that is, one, two, three, four, five, ten, twenty, thirty, forty, fifty, a hundred, a thousand, a hundred thousand rebirths; many eons of the world contracting, many eons of the world expanding, many eons of the world contracting and expanding. They remember: ‘There, I was named this, my clan was that, I looked like this, and that was my food. This was how I felt pleasure and pain, and that was how my life ended. When I passed away from that place I was reborn somewhere else. Passing away from there, I was reborn elsewhere, and there I had such a name, such a family, such appearance, such food, such experience of happiness and suffering, and such a life-span. Passing away from there, I was reborn here.’ And so they recollect their many kinds of past lives, with features and details.}}\\
\end{addmargin}
\end{absolutelynopagebreak}

\begin{absolutelynopagebreak}
\setstretch{.7}
{\PaliGlossA{seyyathāpi, māṇava, puriso sakamhā gāmā aññaṃ gāmaṃ gaccheyya; tamhāpi gāmā aññaṃ gāmaṃ gaccheyya; so tamhā gāmā sakaṃyeva gāmaṃ paccāgaccheyya. tassa evamassa: ‘ahaṃ kho sakamhā gāmā amuṃ gāmaṃ agacchiṃ, tatra evaṃ aṭṭhāsiṃ evaṃ nisīdiṃ evaṃ abhāsiṃ evaṃ tuṇhī ahosiṃ. so tamhāpi gāmā amuṃ gāmaṃ gacchiṃ, tatrāpi evaṃ aṭṭhāsiṃ evaṃ nisīdiṃ evaṃ abhāsiṃ evaṃ tuṇhī ahosiṃ. somhi tamhā gāmā sakaṃyeva gāmaṃ paccāgato’ti.}}\\
\begin{addmargin}[1em]{2em}
\setstretch{.5}
{\PaliGlossB{Suppose a person was to leave their home village and go to another village. From that village they’d go to yet another village. And from that village they’d return to their home village. They’d think: ‘I went from my home village to another village. There I stood like this, sat like that, spoke like this, or kept silent like that. From that village I went to yet another village. There too I stood like this, sat like that, spoke like this, or kept silent like that. And from that village I returned to my home village.’}}\\
\end{addmargin}
\end{absolutelynopagebreak}

\begin{absolutelynopagebreak}
\setstretch{.7}
{\PaliGlossA{evameva kho, māṇava, bhikkhu … pe … yampi, māṇava, bhikkhu evaṃ samāhite citte … pe … āneñjappatte pubbenivāsānussatiñāṇāya cittaṃ abhinīharati abhininnāmeti.}}\\
\begin{addmargin}[1em]{2em}
\setstretch{.5}
{\PaliGlossB{In the same way, when their mind has become immersed in samādhi like this—purified, bright, flawless, rid of corruptions, pliable, workable, steady, and imperturbable—they extend it and project it toward recollection of past lives.}}\\
\end{addmargin}
\end{absolutelynopagebreak}

\begin{absolutelynopagebreak}
\setstretch{.7}
{\PaliGlossA{so anekavihitaṃ pubbenivāsaṃ anussarati. seyyathidaṃ—ekampi jātiṃ … pe … iti sākāraṃ sauddesaṃ anekavihitaṃ pubbenivāsaṃ anussarati.}}\\
\begin{addmargin}[1em]{2em}
\setstretch{.5}
{\PaliGlossB{    -}}\\
\end{addmargin}
\end{absolutelynopagebreak}

\begin{absolutelynopagebreak}
\setstretch{.7}
{\PaliGlossA{idampissa hoti paññāya.}}\\
\begin{addmargin}[1em]{2em}
\setstretch{.5}
{\PaliGlossB{This pertains to their wisdom.}}\\
\end{addmargin}
\end{absolutelynopagebreak}

\begin{absolutelynopagebreak}
\setstretch{.7}
{\PaliGlossA{so evaṃ samāhite citte … pe … āneñjappatte sattānaṃ cutūpapātañāṇāya cittaṃ abhinīharati abhininnāmeti.}}\\
\begin{addmargin}[1em]{2em}
\setstretch{.5}
{\PaliGlossB{When their mind has become immersed in samādhi like this—purified, bright, flawless, rid of corruptions, pliable, workable, steady, and imperturbable—they extend it and project it toward knowledge of the death and rebirth of sentient beings.}}\\
\end{addmargin}
\end{absolutelynopagebreak}

\begin{absolutelynopagebreak}
\setstretch{.7}
{\PaliGlossA{so dibbena cakkhunā visuddhena atikkantamānusakena satte passati cavamāne upapajjamāne hīne paṇīte suvaṇṇe dubbaṇṇe sugate duggate, yathākammūpage satte pajānāti: ‘ime vata bhonto sattā kāyaduccaritena samannāgatā vacīduccaritena samannāgatā manoduccaritena samannāgatā ariyānaṃ upavādakā micchādiṭṭhikā micchādiṭṭhikammasamādānā. te kāyassa bhedā paraṃ maraṇā apāyaṃ duggatiṃ vinipātaṃ nirayaṃ upapannā. ime vā pana bhonto sattā kāyasucaritena samannāgatā vacīsucaritena samannāgatā manosucaritena samannāgatā ariyānaṃ anupavādakā sammādiṭṭhikā sammādiṭṭhikammasamādānā. te kāyassa bhedā paraṃ maraṇā sugatiṃ saggaṃ lokaṃ upapannā’ti. iti dibbena cakkhunā visuddhena atikkantamānusakena satte passati cavamāne upapajjamāne hīne paṇīte suvaṇṇe dubbaṇṇe sugate duggate, yathākammūpage satte pajānāti.}}\\
\begin{addmargin}[1em]{2em}
\setstretch{.5}
{\PaliGlossB{With clairvoyance that is purified and superhuman, they see sentient beings passing away and being reborn—inferior and superior, beautiful and ugly, in a good place or a bad place. They understand how sentient beings are reborn according to their deeds. ‘These dear beings did bad things by way of body, speech, and mind. They spoke ill of the noble ones; they had wrong view; and they chose to act out of that wrong view. When their body breaks up, after death, they’re reborn in a place of loss, a bad place, the underworld, hell. These dear beings, however, did good things by way of body, speech, and mind. They never spoke ill of the noble ones; they had right view; and they chose to act out of that right view. When their body breaks up, after death, they’re reborn in a good place, a heavenly realm.’ And so, with clairvoyance that is purified and superhuman, they see sentient beings passing away and being reborn—inferior and superior, beautiful and ugly, in a good place or a bad place. They understand how sentient beings are reborn according to their deeds.}}\\
\end{addmargin}
\end{absolutelynopagebreak}

\begin{absolutelynopagebreak}
\setstretch{.7}
{\PaliGlossA{seyyathāpi, māṇava, majjhesiṅghāṭake pāsādo, tattha cakkhumā puriso ṭhito passeyya manusse gehaṃ pavisantepi nikkhamantepi rathikāyapi vīthiṃ sañcarante majjhesiṅghāṭake nisinnepi. tassa evamassa: ‘ete manussā gehaṃ pavisanti, ete nikkhamanti, ete rathikāya vīthiṃ sañcaranti, ete majjhesiṅghāṭake nisinnā’ti.}}\\
\begin{addmargin}[1em]{2em}
\setstretch{.5}
{\PaliGlossB{Suppose there was a stilt longhouse at the central square. A person with good eyesight standing there might see people entering and leaving a house, walking along the streets and paths, and sitting at the central square. They’d think: ‘These are people entering and leaving a house, walking along the streets and paths, and sitting at the central square.’}}\\
\end{addmargin}
\end{absolutelynopagebreak}

\begin{absolutelynopagebreak}
\setstretch{.7}
{\PaliGlossA{evameva kho, māṇava, bhikkhu … pe …}}\\
\begin{addmargin}[1em]{2em}
\setstretch{.5}
{\PaliGlossB{In the same way, when their mind has become immersed in samādhi like this—purified, bright, flawless, rid of corruptions, pliable, workable, steady, and imperturbable—they extend and project it toward knowledge of the death and rebirth of sentient beings.}}\\
\end{addmargin}
\end{absolutelynopagebreak}

\begin{absolutelynopagebreak}
\setstretch{.7}
{\PaliGlossA{yampi, māṇava, bhikkhu evaṃ samāhite citte … pe … āneñjappatte sattānaṃ cutūpapātañāṇāya cittaṃ abhinīharati abhininnāmeti.}}\\
\begin{addmargin}[1em]{2em}
\setstretch{.5}
{\PaliGlossB{    -}}\\
\end{addmargin}
\end{absolutelynopagebreak}

\begin{absolutelynopagebreak}
\setstretch{.7}
{\PaliGlossA{so dibbena cakkhunā visuddhena atikkantamānusakena satte passati cavamāne upapajjamāne hīne paṇīte suvaṇṇe dubbaṇṇe sugate duggate, yathākammūpage satte pajānāti.}}\\
\begin{addmargin}[1em]{2em}
\setstretch{.5}
{\PaliGlossB{    -}}\\
\end{addmargin}
\end{absolutelynopagebreak}

\begin{absolutelynopagebreak}
\setstretch{.7}
{\PaliGlossA{idampissa hoti paññāya.}}\\
\begin{addmargin}[1em]{2em}
\setstretch{.5}
{\PaliGlossB{This pertains to their wisdom.}}\\
\end{addmargin}
\end{absolutelynopagebreak}

\begin{absolutelynopagebreak}
\setstretch{.7}
{\PaliGlossA{so evaṃ samāhite citte parisuddhe pariyodāte anaṅgaṇe vigatūpakkilese mudubhūte kammaniye ṭhite āneñjappatte āsavānaṃ khayañāṇāya cittaṃ abhinīharati abhininnāmeti.}}\\
\begin{addmargin}[1em]{2em}
\setstretch{.5}
{\PaliGlossB{When their mind has become immersed in samādhi like this—purified, bright, flawless, rid of corruptions, pliable, workable, steady, and imperturbable—they extend it and project it toward knowledge of the ending of defilements.}}\\
\end{addmargin}
\end{absolutelynopagebreak}

\begin{absolutelynopagebreak}
\setstretch{.7}
{\PaliGlossA{so idaṃ dukkhanti yathābhūtaṃ pajānāti, ayaṃ dukkhasamudayoti yathābhūtaṃ pajānāti, ayaṃ dukkhanirodhoti yathābhūtaṃ pajānāti, ayaṃ dukkhanirodhagāminī paṭipadāti yathābhūtaṃ pajānāti;}}\\
\begin{addmargin}[1em]{2em}
\setstretch{.5}
{\PaliGlossB{They truly understand: ‘This is suffering’ … ‘This is the origin of suffering’ … ‘This is the cessation of suffering’ … ‘This is the practice that leads to the cessation of suffering’.}}\\
\end{addmargin}
\end{absolutelynopagebreak}

\begin{absolutelynopagebreak}
\setstretch{.7}
{\PaliGlossA{ime āsavāti yathābhūtaṃ pajānāti, ayaṃ āsavasamudayoti yathābhūtaṃ pajānāti, ayaṃ āsavanirodhoti yathābhūtaṃ pajānāti, ayaṃ āsavanirodhagāminī paṭipadāti yathābhūtaṃ pajānāti.}}\\
\begin{addmargin}[1em]{2em}
\setstretch{.5}
{\PaliGlossB{They truly understand: ‘These are defilements’ … ‘This is the origin of defilements’ … ‘This is the cessation of defilements’ … ‘This is the practice that leads to the cessation of defilements’.}}\\
\end{addmargin}
\end{absolutelynopagebreak}

\begin{absolutelynopagebreak}
\setstretch{.7}
{\PaliGlossA{tassa evaṃ jānato evaṃ passato kāmāsavāpi cittaṃ vimuccati, bhavāsavāpi cittaṃ vimuccati, avijjāsavāpi cittaṃ vimuccati,}}\\
\begin{addmargin}[1em]{2em}
\setstretch{.5}
{\PaliGlossB{Knowing and seeing like this, their mind is freed from the defilements of sensuality, desire to be reborn, and ignorance.}}\\
\end{addmargin}
\end{absolutelynopagebreak}

\begin{absolutelynopagebreak}
\setstretch{.7}
{\PaliGlossA{vimuttasmiṃ vimuttamiti ñāṇaṃ hoti.}}\\
\begin{addmargin}[1em]{2em}
\setstretch{.5}
{\PaliGlossB{When they’re freed, they know they’re freed.}}\\
\end{addmargin}
\end{absolutelynopagebreak}

\begin{absolutelynopagebreak}
\setstretch{.7}
{\PaliGlossA{‘khīṇā jāti, vusitaṃ brahmacariyaṃ, kataṃ karaṇīyaṃ, nāparaṃ itthattāyā’ti pajānāti.}}\\
\begin{addmargin}[1em]{2em}
\setstretch{.5}
{\PaliGlossB{They understand: ‘Rebirth is ended, the spiritual journey has been completed, what had to be done has been done, there is no return to any state of existence.’}}\\
\end{addmargin}
\end{absolutelynopagebreak}

\begin{absolutelynopagebreak}
\setstretch{.7}
{\PaliGlossA{seyyathāpi, māṇava, pabbatasaṅkhepe udakarahado accho vippasanno anāvilo. tattha cakkhumā puriso tīre ṭhito passeyya sippisambukampi sakkharakathalampi macchagumbampi carantampi tiṭṭhantampi. tassa evamassa: ‘ayaṃ kho udakarahado accho vippasanno anāvilo. tatrime sippisambukāpi sakkharakathalāpi macchagumbāpi carantipi tiṭṭhantipī’ti.}}\\
\begin{addmargin}[1em]{2em}
\setstretch{.5}
{\PaliGlossB{Suppose that in a mountain glen there was a lake that was transparent, clear, and unclouded. A person with good eyesight standing on the bank would see the mussel shells, gravel and pebbles, and schools of fish swimming about or staying still. They’d think: ‘This lake is transparent, clear, and unclouded. And here are the mussel shells, gravel and pebbles, and schools of fish swimming about or staying still.’}}\\
\end{addmargin}
\end{absolutelynopagebreak}

\begin{absolutelynopagebreak}
\setstretch{.7}
{\PaliGlossA{evameva kho, māṇava, bhikkhu … pe …}}\\
\begin{addmargin}[1em]{2em}
\setstretch{.5}
{\PaliGlossB{In the same way, when their mind has become immersed in samādhi like this—purified, bright, flawless, rid of corruptions, pliable, workable, steady, and imperturbable—they extend it and project it toward knowledge of the ending of defilements.}}\\
\end{addmargin}
\end{absolutelynopagebreak}

\begin{absolutelynopagebreak}
\setstretch{.7}
{\PaliGlossA{yampi, māṇava, bhikkhu evaṃ samāhite citte … pe … āneñjappatte āsavānaṃ khayañāṇāya cittaṃ abhinīharati abhininnāmeti.}}\\
\begin{addmargin}[1em]{2em}
\setstretch{.5}
{\PaliGlossB{    -}}\\
\end{addmargin}
\end{absolutelynopagebreak}

\begin{absolutelynopagebreak}
\setstretch{.7}
{\PaliGlossA{so idaṃ dukkhanti yathābhūtaṃ pajānāti … pe …}}\\
\begin{addmargin}[1em]{2em}
\setstretch{.5}
{\PaliGlossB{    -}}\\
\end{addmargin}
\end{absolutelynopagebreak}

\begin{absolutelynopagebreak}
\setstretch{.7}
{\PaliGlossA{āsavanirodhagāminī paṭipadāti yathābhūtaṃ pajānāti.}}\\
\begin{addmargin}[1em]{2em}
\setstretch{.5}
{\PaliGlossB{    -}}\\
\end{addmargin}
\end{absolutelynopagebreak}

\begin{absolutelynopagebreak}
\setstretch{.7}
{\PaliGlossA{tassa evaṃ jānato evaṃ passato kāmāsavāpi cittaṃ vimuccati, bhavāsavāpi cittaṃ vimuccati, avijjāsavāpi cittaṃ vimuccati, vimuttasmiṃ vimuttamiti ñāṇaṃ hoti, ‘khīṇā jāti, vusitaṃ brahmacariyaṃ, kataṃ karaṇīyaṃ, nāparaṃ itthattāyā’ti pajānāti.}}\\
\begin{addmargin}[1em]{2em}
\setstretch{.5}
{\PaliGlossB{    -}}\\
\end{addmargin}
\end{absolutelynopagebreak}

\begin{absolutelynopagebreak}
\setstretch{.7}
{\PaliGlossA{idampissa hoti paññāya.}}\\
\begin{addmargin}[1em]{2em}
\setstretch{.5}
{\PaliGlossB{This pertains to their wisdom.}}\\
\end{addmargin}
\end{absolutelynopagebreak}

\begin{absolutelynopagebreak}
\setstretch{.7}
{\PaliGlossA{ayaṃ kho so, māṇava, ariyo paññākkhandho yassa so bhagavā vaṇṇavādī ahosi, yattha ca imaṃ janataṃ samādapesi nivesesi patiṭṭhāpesi.}}\\
\begin{addmargin}[1em]{2em}
\setstretch{.5}
{\PaliGlossB{This is that noble spectrum of wisdom that the Buddha praised.}}\\
\end{addmargin}
\end{absolutelynopagebreak}

\begin{absolutelynopagebreak}
\setstretch{.7}
{\PaliGlossA{natthi cevettha uttarikaraṇīyan”ti.}}\\
\begin{addmargin}[1em]{2em}
\setstretch{.5}
{\PaliGlossB{And there is nothing more to be done.”}}\\
\end{addmargin}
\end{absolutelynopagebreak}

\begin{absolutelynopagebreak}
\setstretch{.7}
{\PaliGlossA{“acchariyaṃ, bho ānanda, abbhutaṃ, bho ānanda.}}\\
\begin{addmargin}[1em]{2em}
\setstretch{.5}
{\PaliGlossB{“It’s incredible, Master Ānanda, it’s amazing!}}\\
\end{addmargin}
\end{absolutelynopagebreak}

\begin{absolutelynopagebreak}
\setstretch{.7}
{\PaliGlossA{so cāyaṃ, bho ānanda, ariyo paññākkhandho paripuṇṇo, no aparipuṇṇo.}}\\
\begin{addmargin}[1em]{2em}
\setstretch{.5}
{\PaliGlossB{This noble spectrum of wisdom is complete, not lacking anything!}}\\
\end{addmargin}
\end{absolutelynopagebreak}

\begin{absolutelynopagebreak}
\setstretch{.7}
{\PaliGlossA{evaṃ paripuṇṇañcāhaṃ, bho ānanda, ariyaṃ paññākkhandhaṃ ito bahiddhā aññesu samaṇabrāhmaṇesu na samanupassāmi.}}\\
\begin{addmargin}[1em]{2em}
\setstretch{.5}
{\PaliGlossB{Such a complete spectrum of wisdom cannot be seen among the other ascetics and brahmins.}}\\
\end{addmargin}
\end{absolutelynopagebreak}

\begin{absolutelynopagebreak}
\setstretch{.7}
{\PaliGlossA{natthi cevettha uttarikaraṇīyaṃ.}}\\
\begin{addmargin}[1em]{2em}
\setstretch{.5}
{\PaliGlossB{And there is nothing more to be done.}}\\
\end{addmargin}
\end{absolutelynopagebreak}

\begin{absolutelynopagebreak}
\setstretch{.7}
{\PaliGlossA{abhikkantaṃ, bho ānanda, abhikkantaṃ, bho ānanda.}}\\
\begin{addmargin}[1em]{2em}
\setstretch{.5}
{\PaliGlossB{Excellent, Master Ānanda! Excellent!}}\\
\end{addmargin}
\end{absolutelynopagebreak}

\begin{absolutelynopagebreak}
\setstretch{.7}
{\PaliGlossA{seyyathāpi, bho ānanda, nikkujjitaṃ vā ukkujjeyya, paṭicchannaṃ vā vivareyya, mūḷhassa vā maggaṃ ācikkheyya, andhakāre vā telapajjotaṃ dhāreyya: ‘cakkhumanto rūpāni dakkhantī’ti; evamevaṃ bhotā ānandena anekapariyāyena dhammo pakāsito.}}\\
\begin{addmargin}[1em]{2em}
\setstretch{.5}
{\PaliGlossB{As if he were righting the overturned, or revealing the hidden, or pointing out the path to the lost, or lighting a lamp in the dark so people with good eyes can see what’s there, Master Ānanda has made the teaching clear in many ways.}}\\
\end{addmargin}
\end{absolutelynopagebreak}

\begin{absolutelynopagebreak}
\setstretch{.7}
{\PaliGlossA{esāhaṃ, bho ānanda, taṃ bhavantaṃ gotamaṃ saraṇaṃ gacchāmi dhammañca bhikkhusaṃghañca.}}\\
\begin{addmargin}[1em]{2em}
\setstretch{.5}
{\PaliGlossB{I go for refuge to Master Gotama, to the teaching, and to the mendicant Saṅgha.}}\\
\end{addmargin}
\end{absolutelynopagebreak}

\begin{absolutelynopagebreak}
\setstretch{.7}
{\PaliGlossA{upāsakaṃ maṃ bhavaṃ ānando dhāretu ajjatagge pāṇupetaṃ saraṇaṃ gatan”ti.}}\\
\begin{addmargin}[1em]{2em}
\setstretch{.5}
{\PaliGlossB{From this day forth, may Master Ānanda remember me as a lay follower who has gone for refuge for life.”}}\\
\end{addmargin}
\end{absolutelynopagebreak}

\begin{absolutelynopagebreak}
\setstretch{.7}
{\PaliGlossA{subhasuttaṃ niṭṭhitaṃ dasamaṃ.}}\\
\begin{addmargin}[1em]{2em}
\setstretch{.5}
{\PaliGlossB{    -}}\\
\end{addmargin}
\end{absolutelynopagebreak}
