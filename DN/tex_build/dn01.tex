
\begin{absolutelynopagebreak}
\setstretch{.7}
{\PaliGlossA{dīgha nikāya 1}}\\
\begin{addmargin}[1em]{2em}
\setstretch{.5}
{\PaliGlossB{Long Discourses 1}}\\
\end{addmargin}
\end{absolutelynopagebreak}

\begin{absolutelynopagebreak}
\setstretch{.7}
{\PaliGlossA{brahmajālasutta}}\\
\begin{addmargin}[1em]{2em}
\setstretch{.5}
{\PaliGlossB{The Prime Net}}\\
\end{addmargin}
\end{absolutelynopagebreak}

\begin{absolutelynopagebreak}
\setstretch{.7}
{\PaliGlossA{1. paribbājakakathā}}\\
\begin{addmargin}[1em]{2em}
\setstretch{.5}
{\PaliGlossB{1. Talk on Wanderers}}\\
\end{addmargin}
\end{absolutelynopagebreak}

\begin{absolutelynopagebreak}
\setstretch{.7}
{\PaliGlossA{evaṃ me sutaṃ—}}\\
\begin{addmargin}[1em]{2em}
\setstretch{.5}
{\PaliGlossB{So I have heard.}}\\
\end{addmargin}
\end{absolutelynopagebreak}

\begin{absolutelynopagebreak}
\setstretch{.7}
{\PaliGlossA{ekaṃ samayaṃ bhagavā antarā ca rājagahaṃ antarā ca nāḷandaṃ addhānamaggappaṭipanno hoti mahatā bhikkhusaṃghena saddhiṃ pañcamattehi bhikkhusatehi.}}\\
\begin{addmargin}[1em]{2em}
\setstretch{.5}
{\PaliGlossB{At one time the Buddha was traveling along the road between Rājagaha and Nālanda together with a large Saṅgha of around five hundred mendicants.}}\\
\end{addmargin}
\end{absolutelynopagebreak}

\begin{absolutelynopagebreak}
\setstretch{.7}
{\PaliGlossA{suppiyopi kho paribbājako antarā ca rājagahaṃ antarā ca nāḷandaṃ addhānamaggappaṭipanno hoti saddhiṃ antevāsinā brahmadattena māṇavena.}}\\
\begin{addmargin}[1em]{2em}
\setstretch{.5}
{\PaliGlossB{The wanderer Suppiya was also traveling along the same road, together with his pupil, the brahmin student Brahmadatta.}}\\
\end{addmargin}
\end{absolutelynopagebreak}

\begin{absolutelynopagebreak}
\setstretch{.7}
{\PaliGlossA{tatra sudaṃ suppiyo paribbājako anekapariyāyena buddhassa avaṇṇaṃ bhāsati, dhammassa avaṇṇaṃ bhāsati, saṃghassa avaṇṇaṃ bhāsati;}}\\
\begin{addmargin}[1em]{2em}
\setstretch{.5}
{\PaliGlossB{Meanwhile, Suppiya criticized the Buddha, the teaching, and the Saṅgha in many ways,}}\\
\end{addmargin}
\end{absolutelynopagebreak}

\begin{absolutelynopagebreak}
\setstretch{.7}
{\PaliGlossA{suppiyassa pana paribbājakassa antevāsī brahmadatto māṇavo anekapariyāyena buddhassa vaṇṇaṃ bhāsati, dhammassa vaṇṇaṃ bhāsati, saṃghassa vaṇṇaṃ bhāsati.}}\\
\begin{addmargin}[1em]{2em}
\setstretch{.5}
{\PaliGlossB{but his pupil Brahmadatta praised them in many ways.}}\\
\end{addmargin}
\end{absolutelynopagebreak}

\begin{absolutelynopagebreak}
\setstretch{.7}
{\PaliGlossA{itiha te ubho ācariyantevāsī aññamaññassa ujuvipaccanīkavādā bhagavantaṃ piṭṭhito piṭṭhito anubandhā honti bhikkhusaṃghañca.}}\\
\begin{addmargin}[1em]{2em}
\setstretch{.5}
{\PaliGlossB{And so both teacher and pupil followed behind the Buddha and the Saṅgha of mendicants directly contradicting each other.}}\\
\end{addmargin}
\end{absolutelynopagebreak}

\begin{absolutelynopagebreak}
\setstretch{.7}
{\PaliGlossA{atha kho bhagavā ambalaṭṭhikāyaṃ rājāgārake ekarattivāsaṃ upagacchi saddhiṃ bhikkhusaṃghena.}}\\
\begin{addmargin}[1em]{2em}
\setstretch{.5}
{\PaliGlossB{Then the Buddha took up residence for the night in the royal rest-house in Ambalaṭṭhikā together with the Saṅgha of mendicants.}}\\
\end{addmargin}
\end{absolutelynopagebreak}

\begin{absolutelynopagebreak}
\setstretch{.7}
{\PaliGlossA{suppiyopi kho paribbājako ambalaṭṭhikāyaṃ rājāgārake ekarattivāsaṃ upagacchi antevāsinā brahmadattena māṇavena.}}\\
\begin{addmargin}[1em]{2em}
\setstretch{.5}
{\PaliGlossB{And Suppiya and Brahmadatta did likewise.}}\\
\end{addmargin}
\end{absolutelynopagebreak}

\begin{absolutelynopagebreak}
\setstretch{.7}
{\PaliGlossA{tatrapi sudaṃ suppiyo paribbājako anekapariyāyena buddhassa avaṇṇaṃ bhāsati, dhammassa avaṇṇaṃ bhāsati, saṃghassa avaṇṇaṃ bhāsati;}}\\
\begin{addmargin}[1em]{2em}
\setstretch{.5}
{\PaliGlossB{There too, Suppiya criticized the Buddha, the teaching, and the Saṅgha in many ways,}}\\
\end{addmargin}
\end{absolutelynopagebreak}

\begin{absolutelynopagebreak}
\setstretch{.7}
{\PaliGlossA{suppiyassa pana paribbājakassa antevāsī brahmadatto māṇavo anekapariyāyena buddhassa vaṇṇaṃ bhāsati, dhammassa vaṇṇaṃ bhāsati, saṃghassa vaṇṇaṃ bhāsati.}}\\
\begin{addmargin}[1em]{2em}
\setstretch{.5}
{\PaliGlossB{but his pupil Brahmadatta praised them in many ways.}}\\
\end{addmargin}
\end{absolutelynopagebreak}

\begin{absolutelynopagebreak}
\setstretch{.7}
{\PaliGlossA{itiha te ubho ācariyantevāsī aññamaññassa ujuvipaccanīkavādā viharanti.}}\\
\begin{addmargin}[1em]{2em}
\setstretch{.5}
{\PaliGlossB{And so both teacher and pupil kept on directly contradicting each other.}}\\
\end{addmargin}
\end{absolutelynopagebreak}

\begin{absolutelynopagebreak}
\setstretch{.7}
{\PaliGlossA{atha kho sambahulānaṃ bhikkhūnaṃ rattiyā paccūsasamayaṃ paccuṭṭhitānaṃ maṇḍalamāḷe sannisinnānaṃ sannipatitānaṃ ayaṃ saṅkhiyadhammo udapādi:}}\\
\begin{addmargin}[1em]{2em}
\setstretch{.5}
{\PaliGlossB{Then several mendicants rose at the crack of dawn and sat together in the pavilion, where the topic of evaluation came up:}}\\
\end{addmargin}
\end{absolutelynopagebreak}

\begin{absolutelynopagebreak}
\setstretch{.7}
{\PaliGlossA{“acchariyaṃ, āvuso, abbhutaṃ, āvuso, yāvañcidaṃ tena bhagavatā jānatā passatā arahatā sammāsambuddhena sattānaṃ nānādhimuttikatā suppaṭividitā.}}\\
\begin{addmargin}[1em]{2em}
\setstretch{.5}
{\PaliGlossB{“It’s incredible, reverends, it’s amazing how the diverse attitudes of sentient beings have been clearly comprehended by the Blessed One, who knows and sees, the perfected one, the fully awakened Buddha.}}\\
\end{addmargin}
\end{absolutelynopagebreak}

\begin{absolutelynopagebreak}
\setstretch{.7}
{\PaliGlossA{ayañhi suppiyo paribbājako anekapariyāyena buddhassa avaṇṇaṃ bhāsati, dhammassa avaṇṇaṃ bhāsati, saṅghassa avaṇṇaṃ bhāsati;}}\\
\begin{addmargin}[1em]{2em}
\setstretch{.5}
{\PaliGlossB{For this Suppiya criticizes the Buddha, the teaching, and the Saṅgha in many ways,}}\\
\end{addmargin}
\end{absolutelynopagebreak}

\begin{absolutelynopagebreak}
\setstretch{.7}
{\PaliGlossA{suppiyassa pana paribbājakassa antevāsī brahmadatto māṇavo anekapariyāyena buddhassa vaṇṇaṃ bhāsati, dhammassa vaṇṇaṃ bhāsati, saṅghassa vaṇṇaṃ bhāsati.}}\\
\begin{addmargin}[1em]{2em}
\setstretch{.5}
{\PaliGlossB{while his pupil Brahmadatta praises them in many ways.}}\\
\end{addmargin}
\end{absolutelynopagebreak}

\begin{absolutelynopagebreak}
\setstretch{.7}
{\PaliGlossA{itihame ubho ācariyantevāsī aññamaññassa ujuvipaccanīkavādā bhagavantaṃ piṭṭhito piṭṭhito anubandhā honti bhikkhusaṅghañcā”ti.}}\\
\begin{addmargin}[1em]{2em}
\setstretch{.5}
{\PaliGlossB{And so both teacher and pupil followed behind the Buddha and the Saṅgha of mendicants directly contradicting each other.”}}\\
\end{addmargin}
\end{absolutelynopagebreak}

\begin{absolutelynopagebreak}
\setstretch{.7}
{\PaliGlossA{atha kho bhagavā tesaṃ bhikkhūnaṃ imaṃ saṅkhiyadhammaṃ viditvā yena maṇḍalamāḷo tenupasaṅkami; upasaṅkamitvā paññatte āsane nisīdi. nisajja kho bhagavā bhikkhū āmantesi:}}\\
\begin{addmargin}[1em]{2em}
\setstretch{.5}
{\PaliGlossB{When the Buddha found out about this discussion on evaluation among the mendicants, he went to the pavilion, where he sat on the seat spread out and addressed the mendicants,}}\\
\end{addmargin}
\end{absolutelynopagebreak}

\begin{absolutelynopagebreak}
\setstretch{.7}
{\PaliGlossA{“kāya nuttha, bhikkhave, etarahi kathāya sannisinnā sannipatitā, kā ca pana vo antarākathā vippakatā”ti?}}\\
\begin{addmargin}[1em]{2em}
\setstretch{.5}
{\PaliGlossB{“Mendicants, what were you sitting talking about just now? What conversation was left unfinished?”}}\\
\end{addmargin}
\end{absolutelynopagebreak}

\begin{absolutelynopagebreak}
\setstretch{.7}
{\PaliGlossA{evaṃ vutte, te bhikkhū bhagavantaṃ etadavocuṃ:}}\\
\begin{addmargin}[1em]{2em}
\setstretch{.5}
{\PaliGlossB{The mendicants told him what had happened, adding,}}\\
\end{addmargin}
\end{absolutelynopagebreak}

\begin{absolutelynopagebreak}
\setstretch{.7}
{\PaliGlossA{“idha, bhante, amhākaṃ rattiyā paccūsasamayaṃ paccuṭṭhitānaṃ maṇḍalamāḷe sannisinnānaṃ sannipatitānaṃ ayaṃ saṅkhiyadhammo udapādi:}}\\
\begin{addmargin}[1em]{2em}
\setstretch{.5}
{\PaliGlossB{    -}}\\
\end{addmargin}
\end{absolutelynopagebreak}

\begin{absolutelynopagebreak}
\setstretch{.7}
{\PaliGlossA{‘acchariyaṃ, āvuso, abbhutaṃ, āvuso, yāvañcidaṃ tena bhagavatā jānatā passatā arahatā sammāsambuddhena sattānaṃ nānādhimuttikatā suppaṭividitā.}}\\
\begin{addmargin}[1em]{2em}
\setstretch{.5}
{\PaliGlossB{    -}}\\
\end{addmargin}
\end{absolutelynopagebreak}

\begin{absolutelynopagebreak}
\setstretch{.7}
{\PaliGlossA{ayañhi suppiyo paribbājako anekapariyāyena buddhassa avaṇṇaṃ bhāsati, dhammassa avaṇṇaṃ bhāsati, saṅghassa avaṇṇaṃ bhāsati;}}\\
\begin{addmargin}[1em]{2em}
\setstretch{.5}
{\PaliGlossB{    -}}\\
\end{addmargin}
\end{absolutelynopagebreak}

\begin{absolutelynopagebreak}
\setstretch{.7}
{\PaliGlossA{suppiyassa pana paribbājakassa antevāsī brahmadatto māṇavo anekapariyāyena buddhassa vaṇṇaṃ bhāsati, dhammassa vaṇṇaṃ bhāsati, saṅghassa vaṇṇaṃ bhāsati.}}\\
\begin{addmargin}[1em]{2em}
\setstretch{.5}
{\PaliGlossB{    -}}\\
\end{addmargin}
\end{absolutelynopagebreak}

\begin{absolutelynopagebreak}
\setstretch{.7}
{\PaliGlossA{itihame ubho ācariyantevāsī aññamaññassa ujuvipaccanīkavādā bhagavantaṃ piṭṭhito piṭṭhito anubandhā honti bhikkhusaṅghañcā’ti.}}\\
\begin{addmargin}[1em]{2em}
\setstretch{.5}
{\PaliGlossB{    -}}\\
\end{addmargin}
\end{absolutelynopagebreak}

\begin{absolutelynopagebreak}
\setstretch{.7}
{\PaliGlossA{ayaṃ kho no, bhante, antarākathā vippakatā, atha bhagavā anuppatto”ti.}}\\
\begin{addmargin}[1em]{2em}
\setstretch{.5}
{\PaliGlossB{“This was our conversation that was unfinished when the Buddha arrived.”}}\\
\end{addmargin}
\end{absolutelynopagebreak}

\begin{absolutelynopagebreak}
\setstretch{.7}
{\PaliGlossA{“mamaṃ vā, bhikkhave, pare avaṇṇaṃ bhāseyyuṃ, dhammassa vā avaṇṇaṃ bhāseyyuṃ, saṅghassa vā avaṇṇaṃ bhāseyyuṃ, tatra tumhehi na āghāto na appaccayo na cetaso anabhiraddhi karaṇīyā.}}\\
\begin{addmargin}[1em]{2em}
\setstretch{.5}
{\PaliGlossB{“Mendicants, if others criticize me, the teaching, or the Saṅgha, don’t make yourselves resentful, bitter, and exasperated.}}\\
\end{addmargin}
\end{absolutelynopagebreak}

\begin{absolutelynopagebreak}
\setstretch{.7}
{\PaliGlossA{mamaṃ vā, bhikkhave, pare avaṇṇaṃ bhāseyyuṃ, dhammassa vā avaṇṇaṃ bhāseyyuṃ, saṅghassa vā avaṇṇaṃ bhāseyyuṃ, tatra ce tumhe assatha kupitā vā anattamanā vā, tumhaṃ yevassa tena antarāyo.}}\\
\begin{addmargin}[1em]{2em}
\setstretch{.5}
{\PaliGlossB{You’ll get angry and upset, which would be an obstacle for you alone.}}\\
\end{addmargin}
\end{absolutelynopagebreak}

\begin{absolutelynopagebreak}
\setstretch{.7}
{\PaliGlossA{mamaṃ vā, bhikkhave, pare avaṇṇaṃ bhāseyyuṃ, dhammassa vā avaṇṇaṃ bhāseyyuṃ, saṅghassa vā avaṇṇaṃ bhāseyyuṃ, tatra ce tumhe assatha kupitā vā anattamanā vā, api nu tumhe paresaṃ subhāsitaṃ dubbhāsitaṃ ājāneyyāthā”ti?}}\\
\begin{addmargin}[1em]{2em}
\setstretch{.5}
{\PaliGlossB{If others were to criticize me, the teaching, or the Saṅgha, and you got angry and upset, would you be able to understand whether they spoke well or poorly?”}}\\
\end{addmargin}
\end{absolutelynopagebreak}

\begin{absolutelynopagebreak}
\setstretch{.7}
{\PaliGlossA{“no hetaṃ, bhante”.}}\\
\begin{addmargin}[1em]{2em}
\setstretch{.5}
{\PaliGlossB{“No, sir.”}}\\
\end{addmargin}
\end{absolutelynopagebreak}

\begin{absolutelynopagebreak}
\setstretch{.7}
{\PaliGlossA{“mamaṃ vā, bhikkhave, pare avaṇṇaṃ bhāseyyuṃ, dhammassa vā avaṇṇaṃ bhāseyyuṃ, saṅghassa vā avaṇṇaṃ bhāseyyuṃ, tatra tumhehi abhūtaṃ abhūtato nibbeṭhetabbaṃ:}}\\
\begin{addmargin}[1em]{2em}
\setstretch{.5}
{\PaliGlossB{“If others criticize me, the teaching, or the Saṅgha, you should explain that what is untrue is in fact untrue:}}\\
\end{addmargin}
\end{absolutelynopagebreak}

\begin{absolutelynopagebreak}
\setstretch{.7}
{\PaliGlossA{‘itipetaṃ abhūtaṃ, itipetaṃ atacchaṃ, natthi cetaṃ amhesu, na ca panetaṃ amhesu saṃvijjatī’ti.}}\\
\begin{addmargin}[1em]{2em}
\setstretch{.5}
{\PaliGlossB{‘This is why that’s untrue, this is why that’s false. There’s no such thing in us, it’s not found among us.’}}\\
\end{addmargin}
\end{absolutelynopagebreak}

\begin{absolutelynopagebreak}
\setstretch{.7}
{\PaliGlossA{mamaṃ vā, bhikkhave, pare vaṇṇaṃ bhāseyyuṃ, dhammassa vā vaṇṇaṃ bhāseyyuṃ, saṃghassa vā vaṇṇaṃ bhāseyyuṃ, tatra tumhehi na ānando na somanassaṃ na cetaso uppilāvitattaṃ karaṇīyaṃ.}}\\
\begin{addmargin}[1em]{2em}
\setstretch{.5}
{\PaliGlossB{If others praise me, the teaching, or the Saṅgha, don’t make yourselves thrilled, elated, and excited.}}\\
\end{addmargin}
\end{absolutelynopagebreak}

\begin{absolutelynopagebreak}
\setstretch{.7}
{\PaliGlossA{mamaṃ vā, bhikkhave, pare vaṇṇaṃ bhāseyyuṃ, dhammassa vā vaṇṇaṃ bhāseyyuṃ, saṃghassa vā vaṇṇaṃ bhāseyyuṃ, tatra ce tumhe assatha ānandino sumanā uppilāvitā tumhaṃ yevassa tena antarāyo.}}\\
\begin{addmargin}[1em]{2em}
\setstretch{.5}
{\PaliGlossB{You’ll get thrilled, elated, and excited, which would be an obstacle for you alone.}}\\
\end{addmargin}
\end{absolutelynopagebreak}

\begin{absolutelynopagebreak}
\setstretch{.7}
{\PaliGlossA{mamaṃ vā, bhikkhave, pare vaṇṇaṃ bhāseyyuṃ, dhammassa vā vaṇṇaṃ bhāseyyuṃ, saṃghassa vā vaṇṇaṃ bhāseyyuṃ, tatra tumhehi bhūtaṃ bhūtato paṭijānitabbaṃ:}}\\
\begin{addmargin}[1em]{2em}
\setstretch{.5}
{\PaliGlossB{If others praise me, the teaching, or the Saṅgha, you should acknowledge that what is true is in fact true:}}\\
\end{addmargin}
\end{absolutelynopagebreak}

\begin{absolutelynopagebreak}
\setstretch{.7}
{\PaliGlossA{‘itipetaṃ bhūtaṃ, itipetaṃ tacchaṃ, atthi cetaṃ amhesu, saṃvijjati ca panetaṃ amhesū’ti.}}\\
\begin{addmargin}[1em]{2em}
\setstretch{.5}
{\PaliGlossB{‘This is why that’s true, this is why that’s correct. There is such a thing in us, it is found among us.’}}\\
\end{addmargin}
\end{absolutelynopagebreak}

\begin{absolutelynopagebreak}
\setstretch{.7}
{\PaliGlossA{2. sīla}}\\
\begin{addmargin}[1em]{2em}
\setstretch{.5}
{\PaliGlossB{2. Ethics}}\\
\end{addmargin}
\end{absolutelynopagebreak}

\begin{absolutelynopagebreak}
\setstretch{.7}
{\PaliGlossA{2.1. cūḷasīla}}\\
\begin{addmargin}[1em]{2em}
\setstretch{.5}
{\PaliGlossB{2.1. The Shorter Section on Ethics}}\\
\end{addmargin}
\end{absolutelynopagebreak}

\begin{absolutelynopagebreak}
\setstretch{.7}
{\PaliGlossA{appamattakaṃ kho panetaṃ, bhikkhave, oramattakaṃ sīlamattakaṃ, yena puthujjano tathāgatassa vaṇṇaṃ vadamāno vadeyya.}}\\
\begin{addmargin}[1em]{2em}
\setstretch{.5}
{\PaliGlossB{When an ordinary person speaks praise of the Realized One, they speak only of trivial, insignificant details of mere ethics.}}\\
\end{addmargin}
\end{absolutelynopagebreak}

\begin{absolutelynopagebreak}
\setstretch{.7}
{\PaliGlossA{katamañca taṃ, bhikkhave, appamattakaṃ oramattakaṃ sīlamattakaṃ, yena puthujjano tathāgatassa vaṇṇaṃ vadamāno vadeyya?}}\\
\begin{addmargin}[1em]{2em}
\setstretch{.5}
{\PaliGlossB{And what are the trivial, insignificant details of mere ethics that an ordinary person speaks of?}}\\
\end{addmargin}
\end{absolutelynopagebreak}

\begin{absolutelynopagebreak}
\setstretch{.7}
{\PaliGlossA{‘pāṇātipātaṃ pahāya pāṇātipātā paṭivirato samaṇo gotamo nihitadaṇḍo, nihitasattho, lajjī, dayāpanno, sabbapāṇabhūtahitānukampī viharatī’ti—}}\\
\begin{addmargin}[1em]{2em}
\setstretch{.5}
{\PaliGlossB{‘The ascetic Gotama has given up killing living creatures. He has renounced the rod and the sword. He’s scrupulous and kind, living full of compassion for all living beings.’}}\\
\end{addmargin}
\end{absolutelynopagebreak}

\begin{absolutelynopagebreak}
\setstretch{.7}
{\PaliGlossA{iti vā hi, bhikkhave, puthujjano tathāgatassa vaṇṇaṃ vadamāno vadeyya.}}\\
\begin{addmargin}[1em]{2em}
\setstretch{.5}
{\PaliGlossB{Such is an ordinary person’s praise of the Realized One.}}\\
\end{addmargin}
\end{absolutelynopagebreak}

\begin{absolutelynopagebreak}
\setstretch{.7}
{\PaliGlossA{‘adinnādānaṃ pahāya adinnādānā paṭivirato samaṇo gotamo dinnādāyī dinnapāṭikaṅkhī, athenena sucibhūtena attanā viharatī’ti—}}\\
\begin{addmargin}[1em]{2em}
\setstretch{.5}
{\PaliGlossB{‘The ascetic Gotama has given up stealing. He takes only what’s given, and expects only what’s given. He keeps himself clean by not thieving.’}}\\
\end{addmargin}
\end{absolutelynopagebreak}

\begin{absolutelynopagebreak}
\setstretch{.7}
{\PaliGlossA{iti vā hi, bhikkhave, puthujjano tathāgatassa vaṇṇaṃ vadamāno vadeyya.}}\\
\begin{addmargin}[1em]{2em}
\setstretch{.5}
{\PaliGlossB{Such is an ordinary person’s praise of the Realized One.}}\\
\end{addmargin}
\end{absolutelynopagebreak}

\begin{absolutelynopagebreak}
\setstretch{.7}
{\PaliGlossA{‘abrahmacariyaṃ pahāya brahmacārī samaṇo gotamo ārācārī virato methunā gāmadhammā’ti—}}\\
\begin{addmargin}[1em]{2em}
\setstretch{.5}
{\PaliGlossB{‘The ascetic Gotama has given up unchastity. He is celibate, set apart, avoiding the common practice of sex.’}}\\
\end{addmargin}
\end{absolutelynopagebreak}

\begin{absolutelynopagebreak}
\setstretch{.7}
{\PaliGlossA{iti vā hi, bhikkhave, puthujjano tathāgatassa vaṇṇaṃ vadamāno vadeyya.}}\\
\begin{addmargin}[1em]{2em}
\setstretch{.5}
{\PaliGlossB{Such is an ordinary person’s praise of the Realized One.}}\\
\end{addmargin}
\end{absolutelynopagebreak}

\begin{absolutelynopagebreak}
\setstretch{.7}
{\PaliGlossA{‘musāvādaṃ pahāya musāvādā paṭivirato samaṇo gotamo saccavādī saccasandho theto paccayiko avisaṃvādako lokassā’ti—}}\\
\begin{addmargin}[1em]{2em}
\setstretch{.5}
{\PaliGlossB{‘The ascetic Gotama has given up lying. He speaks the truth and sticks to the truth. He’s honest and trustworthy, and doesn’t trick the world with his words.’}}\\
\end{addmargin}
\end{absolutelynopagebreak}

\begin{absolutelynopagebreak}
\setstretch{.7}
{\PaliGlossA{iti vā hi, bhikkhave, puthujjano tathāgatassa vaṇṇaṃ vadamāno vadeyya.}}\\
\begin{addmargin}[1em]{2em}
\setstretch{.5}
{\PaliGlossB{Such is an ordinary person’s praise of the Realized One.}}\\
\end{addmargin}
\end{absolutelynopagebreak}

\begin{absolutelynopagebreak}
\setstretch{.7}
{\PaliGlossA{‘pisuṇaṃ vācaṃ pahāya pisuṇāya vācāya paṭivirato samaṇo gotamo, ito sutvā na amutra akkhātā imesaṃ bhedāya, amutra vā sutvā na imesaṃ akkhātā amūsaṃ bhedāya. iti bhinnānaṃ vā sandhātā, sahitānaṃ vā anuppadātā samaggārāmo samaggarato samagganandī samaggakaraṇiṃ vācaṃ bhāsitā’ti—}}\\
\begin{addmargin}[1em]{2em}
\setstretch{.5}
{\PaliGlossB{‘The ascetic Gotama has given up divisive speech. He doesn’t repeat in one place what he heard in another so as to divide people against each other. Instead, he reconciles those who are divided, supporting unity, delighting in harmony, loving harmony, speaking words that promote harmony.’}}\\
\end{addmargin}
\end{absolutelynopagebreak}

\begin{absolutelynopagebreak}
\setstretch{.7}
{\PaliGlossA{iti vā hi, bhikkhave, puthujjano tathāgatassa vaṇṇaṃ vadamāno vadeyya.}}\\
\begin{addmargin}[1em]{2em}
\setstretch{.5}
{\PaliGlossB{Such is an ordinary person’s praise of the Realized One.}}\\
\end{addmargin}
\end{absolutelynopagebreak}

\begin{absolutelynopagebreak}
\setstretch{.7}
{\PaliGlossA{‘pharusaṃ vācaṃ pahāya pharusāya vācāya paṭivirato samaṇo gotamo, yā sā vācā nelā kaṇṇasukhā pemanīyā hadayaṅgamā porī bahujanakantā bahujanamanāpā tathārūpiṃ vācaṃ bhāsitā’ti—}}\\
\begin{addmargin}[1em]{2em}
\setstretch{.5}
{\PaliGlossB{‘The ascetic Gotama has given up harsh speech. He speaks in a way that’s mellow, pleasing to the ear, lovely, going to the heart, polite, likable and agreeable to the people.’}}\\
\end{addmargin}
\end{absolutelynopagebreak}

\begin{absolutelynopagebreak}
\setstretch{.7}
{\PaliGlossA{iti vā hi, bhikkhave, puthujjano tathāgatassa vaṇṇaṃ vadamāno vadeyya.}}\\
\begin{addmargin}[1em]{2em}
\setstretch{.5}
{\PaliGlossB{Such is an ordinary person’s praise of the Realized One.}}\\
\end{addmargin}
\end{absolutelynopagebreak}

\begin{absolutelynopagebreak}
\setstretch{.7}
{\PaliGlossA{‘samphappalāpaṃ pahāya samphappalāpā paṭivirato samaṇo gotamo kālavādī bhūtavādī atthavādī dhammavādī vinayavādī, nidhānavatiṃ vācaṃ bhāsitā kālena sāpadesaṃ pariyantavatiṃ atthasaṃhitan’ti—}}\\
\begin{addmargin}[1em]{2em}
\setstretch{.5}
{\PaliGlossB{‘The ascetic Gotama has given up talking nonsense. His words are timely, true, and meaningful, in line with the teaching and training. He says things at the right time which are valuable, reasonable, succinct, and beneficial.’}}\\
\end{addmargin}
\end{absolutelynopagebreak}

\begin{absolutelynopagebreak}
\setstretch{.7}
{\PaliGlossA{iti vā hi, bhikkhave, puthujjano tathāgatassa vaṇṇaṃ vadamāno vadeyya.}}\\
\begin{addmargin}[1em]{2em}
\setstretch{.5}
{\PaliGlossB{Such is an ordinary person’s praise of the Realized One.}}\\
\end{addmargin}
\end{absolutelynopagebreak}

\begin{absolutelynopagebreak}
\setstretch{.7}
{\PaliGlossA{‘bījagāmabhūtagāmasamārambhā paṭivirato samaṇo gotamo’ti—}}\\
\begin{addmargin}[1em]{2em}
\setstretch{.5}
{\PaliGlossB{‘The ascetic Gotama refrains from injuring plants and seeds.’}}\\
\end{addmargin}
\end{absolutelynopagebreak}

\begin{absolutelynopagebreak}
\setstretch{.7}
{\PaliGlossA{iti vā hi, bhikkhave … pe ….}}\\
\begin{addmargin}[1em]{2em}
\setstretch{.5}
{\PaliGlossB{    -}}\\
\end{addmargin}
\end{absolutelynopagebreak}

\begin{absolutelynopagebreak}
\setstretch{.7}
{\PaliGlossA{‘ekabhattiko samaṇo gotamo rattūparato virato vikālabhojanā ….}}\\
\begin{addmargin}[1em]{2em}
\setstretch{.5}
{\PaliGlossB{‘He eats in one part of the day, abstaining from eating at night and food at the wrong time.’}}\\
\end{addmargin}
\end{absolutelynopagebreak}

\begin{absolutelynopagebreak}
\setstretch{.7}
{\PaliGlossA{naccagītavāditavisūkadassanā paṭivirato samaṇo gotamo ….}}\\
\begin{addmargin}[1em]{2em}
\setstretch{.5}
{\PaliGlossB{‘He refrains from dancing, singing, music, and seeing shows.’}}\\
\end{addmargin}
\end{absolutelynopagebreak}

\begin{absolutelynopagebreak}
\setstretch{.7}
{\PaliGlossA{mālāgandhavilepanadhāraṇamaṇḍanavibhūsanaṭṭhānā paṭivirato samaṇo gotamo ….}}\\
\begin{addmargin}[1em]{2em}
\setstretch{.5}
{\PaliGlossB{‘He refrains from beautifying and adorning himself with garlands, perfumes, and makeup.’}}\\
\end{addmargin}
\end{absolutelynopagebreak}

\begin{absolutelynopagebreak}
\setstretch{.7}
{\PaliGlossA{uccāsayanamahāsayanā paṭivirato samaṇo gotamo ….}}\\
\begin{addmargin}[1em]{2em}
\setstretch{.5}
{\PaliGlossB{‘He refrains from high and luxurious beds.’}}\\
\end{addmargin}
\end{absolutelynopagebreak}

\begin{absolutelynopagebreak}
\setstretch{.7}
{\PaliGlossA{jātarūparajatapaṭiggahaṇā paṭivirato samaṇo gotamo ….}}\\
\begin{addmargin}[1em]{2em}
\setstretch{.5}
{\PaliGlossB{‘He refrains from receiving gold and money,}}\\
\end{addmargin}
\end{absolutelynopagebreak}

\begin{absolutelynopagebreak}
\setstretch{.7}
{\PaliGlossA{āmakadhaññapaṭiggahaṇā paṭivirato samaṇo gotamo ….}}\\
\begin{addmargin}[1em]{2em}
\setstretch{.5}
{\PaliGlossB{raw grains,}}\\
\end{addmargin}
\end{absolutelynopagebreak}

\begin{absolutelynopagebreak}
\setstretch{.7}
{\PaliGlossA{āmakamaṃsapaṭiggahaṇā paṭivirato samaṇo gotamo ….}}\\
\begin{addmargin}[1em]{2em}
\setstretch{.5}
{\PaliGlossB{raw meat,}}\\
\end{addmargin}
\end{absolutelynopagebreak}

\begin{absolutelynopagebreak}
\setstretch{.7}
{\PaliGlossA{itthikumārikapaṭiggahaṇā paṭivirato samaṇo gotamo ….}}\\
\begin{addmargin}[1em]{2em}
\setstretch{.5}
{\PaliGlossB{women and girls,}}\\
\end{addmargin}
\end{absolutelynopagebreak}

\begin{absolutelynopagebreak}
\setstretch{.7}
{\PaliGlossA{dāsidāsapaṭiggahaṇā paṭivirato samaṇo gotamo ….}}\\
\begin{addmargin}[1em]{2em}
\setstretch{.5}
{\PaliGlossB{male and female bondservants,}}\\
\end{addmargin}
\end{absolutelynopagebreak}

\begin{absolutelynopagebreak}
\setstretch{.7}
{\PaliGlossA{ajeḷakapaṭiggahaṇā paṭivirato samaṇo gotamo ….}}\\
\begin{addmargin}[1em]{2em}
\setstretch{.5}
{\PaliGlossB{goats and sheep,}}\\
\end{addmargin}
\end{absolutelynopagebreak}

\begin{absolutelynopagebreak}
\setstretch{.7}
{\PaliGlossA{kukkuṭasūkarapaṭiggahaṇā paṭivirato samaṇo gotamo ….}}\\
\begin{addmargin}[1em]{2em}
\setstretch{.5}
{\PaliGlossB{chickens and pigs,}}\\
\end{addmargin}
\end{absolutelynopagebreak}

\begin{absolutelynopagebreak}
\setstretch{.7}
{\PaliGlossA{hatthigavassavaḷavapaṭiggahaṇā paṭivirato samaṇo gotamo ….}}\\
\begin{addmargin}[1em]{2em}
\setstretch{.5}
{\PaliGlossB{elephants, cows, horses, and mares,}}\\
\end{addmargin}
\end{absolutelynopagebreak}

\begin{absolutelynopagebreak}
\setstretch{.7}
{\PaliGlossA{khettavatthupaṭiggahaṇā paṭivirato samaṇo gotamo ….}}\\
\begin{addmargin}[1em]{2em}
\setstretch{.5}
{\PaliGlossB{and fields and land.’}}\\
\end{addmargin}
\end{absolutelynopagebreak}

\begin{absolutelynopagebreak}
\setstretch{.7}
{\PaliGlossA{dūteyyapahiṇagamanānuyogā paṭivirato samaṇo gotamo ….}}\\
\begin{addmargin}[1em]{2em}
\setstretch{.5}
{\PaliGlossB{‘He refrains from running errands and messages;}}\\
\end{addmargin}
\end{absolutelynopagebreak}

\begin{absolutelynopagebreak}
\setstretch{.7}
{\PaliGlossA{kayavikkayā paṭivirato samaṇo gotamo ….}}\\
\begin{addmargin}[1em]{2em}
\setstretch{.5}
{\PaliGlossB{buying and selling;}}\\
\end{addmargin}
\end{absolutelynopagebreak}

\begin{absolutelynopagebreak}
\setstretch{.7}
{\PaliGlossA{tulākūṭakaṃsakūṭamānakūṭā paṭivirato samaṇo gotamo ….}}\\
\begin{addmargin}[1em]{2em}
\setstretch{.5}
{\PaliGlossB{falsifying weights, metals, or measures;}}\\
\end{addmargin}
\end{absolutelynopagebreak}

\begin{absolutelynopagebreak}
\setstretch{.7}
{\PaliGlossA{ukkoṭanavañcananikatisāciyogā paṭivirato samaṇo gotamo ….}}\\
\begin{addmargin}[1em]{2em}
\setstretch{.5}
{\PaliGlossB{bribery, fraud, cheating, and duplicity;}}\\
\end{addmargin}
\end{absolutelynopagebreak}

\begin{absolutelynopagebreak}
\setstretch{.7}
{\PaliGlossA{chedanavadhabandhanaviparāmosaālopasahasākārā paṭivirato samaṇo gotamo’ti—}}\\
\begin{addmargin}[1em]{2em}
\setstretch{.5}
{\PaliGlossB{mutilation, murder, abduction, banditry, plunder, and violence.’}}\\
\end{addmargin}
\end{absolutelynopagebreak}

\begin{absolutelynopagebreak}
\setstretch{.7}
{\PaliGlossA{iti vā hi, bhikkhave, puthujjano tathāgatassa vaṇṇaṃ vadamāno vadeyya.}}\\
\begin{addmargin}[1em]{2em}
\setstretch{.5}
{\PaliGlossB{Such is an ordinary person’s praise of the Realized One.}}\\
\end{addmargin}
\end{absolutelynopagebreak}

\begin{absolutelynopagebreak}
\setstretch{.7}
{\PaliGlossA{cūḷasīlaṃ niṭṭhitaṃ.}}\\
\begin{addmargin}[1em]{2em}
\setstretch{.5}
{\PaliGlossB{The shorter section on ethics is finished.}}\\
\end{addmargin}
\end{absolutelynopagebreak}

\begin{absolutelynopagebreak}
\setstretch{.7}
{\PaliGlossA{2.2. majjhimasīla}}\\
\begin{addmargin}[1em]{2em}
\setstretch{.5}
{\PaliGlossB{2.2. The Middle Section on Ethics}}\\
\end{addmargin}
\end{absolutelynopagebreak}

\begin{absolutelynopagebreak}
\setstretch{.7}
{\PaliGlossA{‘yathā vā paneke bhonto samaṇabrāhmaṇā saddhādeyyāni bhojanāni bhuñjitvā te evarūpaṃ bījagāmabhūtagāmasamārambhaṃ anuyuttā viharanti,}}\\
\begin{addmargin}[1em]{2em}
\setstretch{.5}
{\PaliGlossB{‘There are some ascetics and brahmins who, while enjoying food given in faith, still engage in injuring plants and seeds.}}\\
\end{addmargin}
\end{absolutelynopagebreak}

\begin{absolutelynopagebreak}
\setstretch{.7}
{\PaliGlossA{seyyathidaṃ—mūlabījaṃ khandhabījaṃ phaḷubījaṃ aggabījaṃ bījabījameva pañcamaṃ;}}\\
\begin{addmargin}[1em]{2em}
\setstretch{.5}
{\PaliGlossB{These include plants propagated from roots, stems, cuttings, or joints; and those from regular seeds as the fifth.}}\\
\end{addmargin}
\end{absolutelynopagebreak}

\begin{absolutelynopagebreak}
\setstretch{.7}
{\PaliGlossA{iti evarūpā bījagāmabhūtagāmasamārambhā paṭivirato samaṇo gotamo’ti—}}\\
\begin{addmargin}[1em]{2em}
\setstretch{.5}
{\PaliGlossB{The ascetic Gotama refrains from such injury to plants and seeds.’}}\\
\end{addmargin}
\end{absolutelynopagebreak}

\begin{absolutelynopagebreak}
\setstretch{.7}
{\PaliGlossA{iti vā hi, bhikkhave, puthujjano tathāgatassa vaṇṇaṃ vadamāno vadeyya.}}\\
\begin{addmargin}[1em]{2em}
\setstretch{.5}
{\PaliGlossB{Such is an ordinary person’s praise of the Realized One.}}\\
\end{addmargin}
\end{absolutelynopagebreak}

\begin{absolutelynopagebreak}
\setstretch{.7}
{\PaliGlossA{‘yathā vā paneke bhonto samaṇabrāhmaṇā saddhādeyyāni bhojanāni bhuñjitvā te evarūpaṃ sannidhikāraparibhogaṃ anuyuttā viharanti,}}\\
\begin{addmargin}[1em]{2em}
\setstretch{.5}
{\PaliGlossB{‘There are some ascetics and brahmins who, while enjoying food given in faith, still engage in storing up goods for their own use.}}\\
\end{addmargin}
\end{absolutelynopagebreak}

\begin{absolutelynopagebreak}
\setstretch{.7}
{\PaliGlossA{seyyathidaṃ—annasannidhiṃ pānasannidhiṃ vatthasannidhiṃ yānasannidhiṃ sayanasannidhiṃ gandhasannidhiṃ āmisasannidhiṃ}}\\
\begin{addmargin}[1em]{2em}
\setstretch{.5}
{\PaliGlossB{This includes such things as food, drink, clothes, vehicles, bedding, fragrance, and material possessions.}}\\
\end{addmargin}
\end{absolutelynopagebreak}

\begin{absolutelynopagebreak}
\setstretch{.7}
{\PaliGlossA{iti vā iti evarūpā sannidhikāraparibhogā paṭivirato samaṇo gotamo’ti—}}\\
\begin{addmargin}[1em]{2em}
\setstretch{.5}
{\PaliGlossB{The ascetic Gotama refrains from storing up such goods.’}}\\
\end{addmargin}
\end{absolutelynopagebreak}

\begin{absolutelynopagebreak}
\setstretch{.7}
{\PaliGlossA{iti vā hi, bhikkhave, puthujjano tathāgatassa vaṇṇaṃ vadamāno vadeyya.}}\\
\begin{addmargin}[1em]{2em}
\setstretch{.5}
{\PaliGlossB{Such is an ordinary person’s praise of the Realized One.}}\\
\end{addmargin}
\end{absolutelynopagebreak}

\begin{absolutelynopagebreak}
\setstretch{.7}
{\PaliGlossA{‘yathā vā paneke bhonto samaṇabrāhmaṇā saddhādeyyāni bhojanāni bhuñjitvā te evarūpaṃ visūkadassanaṃ anuyuttā viharanti,}}\\
\begin{addmargin}[1em]{2em}
\setstretch{.5}
{\PaliGlossB{‘There are some ascetics and brahmins who, while enjoying food given in faith, still engage in seeing shows.}}\\
\end{addmargin}
\end{absolutelynopagebreak}

\begin{absolutelynopagebreak}
\setstretch{.7}
{\PaliGlossA{seyyathidaṃ—naccaṃ gītaṃ vāditaṃ pekkhaṃ akkhānaṃ pāṇissaraṃ vetāḷaṃ kumbhathūṇaṃ sobhanakaṃ caṇḍālaṃ vaṃsaṃ dhovanaṃ hatthiyuddhaṃ assayuddhaṃ mahiṃsayuddhaṃ usabhayuddhaṃ ajayuddhaṃ meṇḍayuddhaṃ kukkuṭayuddhaṃ vaṭṭakayuddhaṃ daṇḍayuddhaṃ muṭṭhiyuddhaṃ nibbuddhaṃ uyyodhikaṃ balaggaṃ senābyūhaṃ anīkadassanaṃ}}\\
\begin{addmargin}[1em]{2em}
\setstretch{.5}
{\PaliGlossB{This includes such things as dancing, singing, music, performances, and storytelling; clapping, gongs, and kettle-drums; art exhibitions and acrobatic displays; battles of elephants, horses, buffaloes, bulls, goats, rams, chickens, and quails; staff-fights, boxing, and wrestling; combat, roll calls of the armed forces, battle-formations, and regimental reviews.}}\\
\end{addmargin}
\end{absolutelynopagebreak}

\begin{absolutelynopagebreak}
\setstretch{.7}
{\PaliGlossA{iti vā iti evarūpā visūkadassanā paṭivirato samaṇo gotamo’ti—}}\\
\begin{addmargin}[1em]{2em}
\setstretch{.5}
{\PaliGlossB{The ascetic Gotama refrains from such shows.’}}\\
\end{addmargin}
\end{absolutelynopagebreak}

\begin{absolutelynopagebreak}
\setstretch{.7}
{\PaliGlossA{iti vā hi, bhikkhave, puthujjano tathāgatassa vaṇṇaṃ vadamāno vadeyya.}}\\
\begin{addmargin}[1em]{2em}
\setstretch{.5}
{\PaliGlossB{Such is an ordinary person’s praise of the Realized One.}}\\
\end{addmargin}
\end{absolutelynopagebreak}

\begin{absolutelynopagebreak}
\setstretch{.7}
{\PaliGlossA{‘yathā vā paneke bhonto samaṇabrāhmaṇā saddhādeyyāni bhojanāni bhuñjitvā te evarūpaṃ jūtappamādaṭṭhānānuyogaṃ anuyuttā viharanti,}}\\
\begin{addmargin}[1em]{2em}
\setstretch{.5}
{\PaliGlossB{‘There are some ascetics and brahmins who, while enjoying food given in faith, still engage in gambling that causes negligence.}}\\
\end{addmargin}
\end{absolutelynopagebreak}

\begin{absolutelynopagebreak}
\setstretch{.7}
{\PaliGlossA{seyyathidaṃ—aṭṭhapadaṃ dasapadaṃ ākāsaṃ parihārapathaṃ santikaṃ khalikaṃ ghaṭikaṃ salākahatthaṃ akkhaṃ paṅgacīraṃ vaṅkakaṃ mokkhacikaṃ ciṅgulikaṃ pattāḷhakaṃ rathakaṃ dhanukaṃ akkharikaṃ manesikaṃ yathāvajjaṃ}}\\
\begin{addmargin}[1em]{2em}
\setstretch{.5}
{\PaliGlossB{This includes such things as checkers, draughts, checkers in the air, hopscotch, spillikins, board-games, tip-cat, drawing straws, dice, leaf-flutes, toy plows, somersaults, pinwheels, toy measures, toy carts, toy bows, guessing words from syllables, and guessing another’s thoughts.}}\\
\end{addmargin}
\end{absolutelynopagebreak}

\begin{absolutelynopagebreak}
\setstretch{.7}
{\PaliGlossA{iti vā iti evarūpā jūtappamādaṭṭhānānuyogā paṭivirato samaṇo gotamo’ti—}}\\
\begin{addmargin}[1em]{2em}
\setstretch{.5}
{\PaliGlossB{The ascetic Gotama refrains from such gambling.’}}\\
\end{addmargin}
\end{absolutelynopagebreak}

\begin{absolutelynopagebreak}
\setstretch{.7}
{\PaliGlossA{iti vā hi, bhikkhave, puthujjano tathāgatassa vaṇṇaṃ vadamāno vadeyya.}}\\
\begin{addmargin}[1em]{2em}
\setstretch{.5}
{\PaliGlossB{Such is an ordinary person’s praise of the Realized One.}}\\
\end{addmargin}
\end{absolutelynopagebreak}

\begin{absolutelynopagebreak}
\setstretch{.7}
{\PaliGlossA{‘yathā vā paneke bhonto samaṇabrāhmaṇā saddhādeyyāni bhojanāni bhuñjitvā te evarūpaṃ uccāsayanamahāsayanaṃ anuyuttā viharanti,}}\\
\begin{addmargin}[1em]{2em}
\setstretch{.5}
{\PaliGlossB{‘There are some ascetics and brahmins who, while enjoying food given in faith, still make use of high and luxurious bedding.}}\\
\end{addmargin}
\end{absolutelynopagebreak}

\begin{absolutelynopagebreak}
\setstretch{.7}
{\PaliGlossA{seyyathidaṃ—āsandiṃ pallaṅkaṃ gonakaṃ cittakaṃ paṭikaṃ paṭalikaṃ tūlikaṃ vikatikaṃ uddalomiṃ ekantalomiṃ kaṭṭissaṃ koseyyaṃ kuttakaṃ hatthattharaṃ assattharaṃ rathattharaṃ ajinappaveṇiṃ kadalimigapavarapaccattharaṇaṃ sauttaracchadaṃ ubhatolohitakūpadhānaṃ}}\\
\begin{addmargin}[1em]{2em}
\setstretch{.5}
{\PaliGlossB{This includes such things as sofas, couches, woolen covers—shag-piled, colorful, white, embroidered with flowers, quilted, embroidered with animals, double- or single-fringed—and silk covers studded with gems, as well as silken sheets, woven carpets, rugs for elephants, horses, or chariots, antelope hide rugs, and spreads of fine deer hide, with a canopy above and red cushions at both ends.}}\\
\end{addmargin}
\end{absolutelynopagebreak}

\begin{absolutelynopagebreak}
\setstretch{.7}
{\PaliGlossA{iti vā iti evarūpā uccāsayanamahāsayanā paṭivirato samaṇo gotamo’ti—}}\\
\begin{addmargin}[1em]{2em}
\setstretch{.5}
{\PaliGlossB{The ascetic Gotama refrains from such bedding.’}}\\
\end{addmargin}
\end{absolutelynopagebreak}

\begin{absolutelynopagebreak}
\setstretch{.7}
{\PaliGlossA{iti vā hi, bhikkhave, puthujjano tathāgatassa vaṇṇaṃ vadamāno vadeyya.}}\\
\begin{addmargin}[1em]{2em}
\setstretch{.5}
{\PaliGlossB{Such is an ordinary person’s praise of the Realized One.}}\\
\end{addmargin}
\end{absolutelynopagebreak}

\begin{absolutelynopagebreak}
\setstretch{.7}
{\PaliGlossA{‘yathā vā paneke bhonto samaṇabrāhmaṇā saddhādeyyāni bhojanāni bhuñjitvā te evarūpaṃ maṇḍanavibhūsanaṭṭhānānuyogaṃ anuyuttā viharanti, seyyathidaṃ—}}\\
\begin{addmargin}[1em]{2em}
\setstretch{.5}
{\PaliGlossB{‘There are some ascetics and brahmins who, while enjoying food given in faith, still engage in beautifying and adorning themselves with garlands, fragrance, and makeup.}}\\
\end{addmargin}
\end{absolutelynopagebreak}

\begin{absolutelynopagebreak}
\setstretch{.7}
{\PaliGlossA{ucchādanaṃ parimaddanaṃ nhāpanaṃ sambāhanaṃ ādāsaṃ añjanaṃ mālāgandhavilepanaṃ mukhacuṇṇaṃ mukhalepanaṃ hatthabandhaṃ sikhābandhaṃ daṇḍaṃ nāḷikaṃ asiṃ chattaṃ citrupāhanaṃ uṇhīsaṃ maṇiṃ vālabījaniṃ odātāni vatthāni dīghadasāni}}\\
\begin{addmargin}[1em]{2em}
\setstretch{.5}
{\PaliGlossB{This includes such things as applying beauty products by anointing, massaging, bathing, and rubbing; mirrors, ointments, garlands, fragrances, and makeup; face-powder, foundation, bracelets, headbands, fancy walking-sticks or containers, rapiers, parasols, fancy sandals, turbans, jewelry, chowries, and long-fringed white robes.}}\\
\end{addmargin}
\end{absolutelynopagebreak}

\begin{absolutelynopagebreak}
\setstretch{.7}
{\PaliGlossA{iti vā iti evarūpā maṇḍanavibhūsanaṭṭhānānuyogā paṭivirato samaṇo gotamo’ti—}}\\
\begin{addmargin}[1em]{2em}
\setstretch{.5}
{\PaliGlossB{The ascetic Gotama refrains from such beautification and adornment.’}}\\
\end{addmargin}
\end{absolutelynopagebreak}

\begin{absolutelynopagebreak}
\setstretch{.7}
{\PaliGlossA{iti vā hi, bhikkhave, puthujjano tathāgatassa vaṇṇaṃ vadamāno vadeyya.}}\\
\begin{addmargin}[1em]{2em}
\setstretch{.5}
{\PaliGlossB{Such is an ordinary person’s praise of the Realized One.}}\\
\end{addmargin}
\end{absolutelynopagebreak}

\begin{absolutelynopagebreak}
\setstretch{.7}
{\PaliGlossA{‘yathā vā paneke bhonto samaṇabrāhmaṇā saddhādeyyāni bhojanāni bhuñjitvā te evarūpaṃ tiracchānakathaṃ anuyuttā viharanti,}}\\
\begin{addmargin}[1em]{2em}
\setstretch{.5}
{\PaliGlossB{‘There are some ascetics and brahmins who, while enjoying food given in faith, still engage in unworthy talk. This includes such topics as}}\\
\end{addmargin}
\end{absolutelynopagebreak}

\begin{absolutelynopagebreak}
\setstretch{.7}
{\PaliGlossA{seyyathidaṃ—rājakathaṃ corakathaṃ mahāmattakathaṃ senākathaṃ bhayakathaṃ yuddhakathaṃ annakathaṃ pānakathaṃ vatthakathaṃ sayanakathaṃ mālākathaṃ gandhakathaṃ ñātikathaṃ yānakathaṃ gāmakathaṃ nigamakathaṃ nagarakathaṃ janapadakathaṃ itthikathaṃ sūrakathaṃ visikhākathaṃ kumbhaṭṭhānakathaṃ pubbapetakathaṃ nānattakathaṃ lokakkhāyikaṃ samuddakkhāyikaṃ itibhavābhavakathaṃ}}\\
\begin{addmargin}[1em]{2em}
\setstretch{.5}
{\PaliGlossB{talk about kings, bandits, and ministers; talk about armies, threats, and wars; talk about food, drink, clothes, and beds; talk about garlands and fragrances; talk about family, vehicles, villages, towns, cities, and countries; talk about women and heroes; street talk and well talk; talk about the departed; motley talk; tales of land and sea; and talk about being reborn in this or that state of existence.}}\\
\end{addmargin}
\end{absolutelynopagebreak}

\begin{absolutelynopagebreak}
\setstretch{.7}
{\PaliGlossA{iti vā iti evarūpāya tiracchānakathāya paṭivirato samaṇo gotamo’ti—}}\\
\begin{addmargin}[1em]{2em}
\setstretch{.5}
{\PaliGlossB{The ascetic Gotama refrains from such unworthy talk.’}}\\
\end{addmargin}
\end{absolutelynopagebreak}

\begin{absolutelynopagebreak}
\setstretch{.7}
{\PaliGlossA{iti vā hi, bhikkhave, puthujjano tathāgatassa vaṇṇaṃ vadamāno vadeyya.}}\\
\begin{addmargin}[1em]{2em}
\setstretch{.5}
{\PaliGlossB{Such is an ordinary person’s praise of the Realized One.}}\\
\end{addmargin}
\end{absolutelynopagebreak}

\begin{absolutelynopagebreak}
\setstretch{.7}
{\PaliGlossA{‘yathā vā paneke bhonto samaṇabrāhmaṇā saddhādeyyāni bhojanāni bhuñjitvā te evarūpaṃ viggāhikakathaṃ anuyuttā viharanti,}}\\
\begin{addmargin}[1em]{2em}
\setstretch{.5}
{\PaliGlossB{‘There are some ascetics and brahmins who, while enjoying food given in faith, still engage in arguments.}}\\
\end{addmargin}
\end{absolutelynopagebreak}

\begin{absolutelynopagebreak}
\setstretch{.7}
{\PaliGlossA{seyyathidaṃ—na tvaṃ imaṃ dhammavinayaṃ ājānāsi, ahaṃ imaṃ dhammavinayaṃ ājānāmi, kiṃ tvaṃ imaṃ dhammavinayaṃ ājānissasi, micchā paṭipanno tvamasi, ahamasmi sammā paṭipanno, sahitaṃ me, asahitaṃ te, purevacanīyaṃ pacchā avaca, pacchāvacanīyaṃ pure avaca, adhiciṇṇaṃ te viparāvattaṃ, āropito te vādo, niggahito tvamasi, cara vādappamokkhāya, nibbeṭhehi vā sace pahosīti}}\\
\begin{addmargin}[1em]{2em}
\setstretch{.5}
{\PaliGlossB{They say such things as: “You don’t understand this teaching and training. I understand this teaching and training. What, you understand this teaching and training? You’re practicing wrong. I’m practicing right. I stay on topic, you don’t. You said last what you should have said first. You said first what you should have said last. What you’ve thought so much about has been disproved. Your doctrine is refuted. Go on, save your doctrine! You’re trapped; get yourself out of this—if you can!”}}\\
\end{addmargin}
\end{absolutelynopagebreak}

\begin{absolutelynopagebreak}
\setstretch{.7}
{\PaliGlossA{iti vā iti evarūpāya viggāhikakathāya paṭivirato samaṇo gotamo’ti—}}\\
\begin{addmargin}[1em]{2em}
\setstretch{.5}
{\PaliGlossB{The ascetic Gotama refrains from such argumentative talk.’}}\\
\end{addmargin}
\end{absolutelynopagebreak}

\begin{absolutelynopagebreak}
\setstretch{.7}
{\PaliGlossA{iti vā hi, bhikkhave, puthujjano tathāgatassa vaṇṇaṃ vadamāno vadeyya.}}\\
\begin{addmargin}[1em]{2em}
\setstretch{.5}
{\PaliGlossB{Such is an ordinary person’s praise of the Realized One.}}\\
\end{addmargin}
\end{absolutelynopagebreak}

\begin{absolutelynopagebreak}
\setstretch{.7}
{\PaliGlossA{‘yathā vā paneke bhonto samaṇabrāhmaṇā saddhādeyyāni bhojanāni bhuñjitvā te evarūpaṃ dūteyyapahiṇagamanānuyogaṃ anuyuttā viharanti,}}\\
\begin{addmargin}[1em]{2em}
\setstretch{.5}
{\PaliGlossB{‘There are some ascetics and brahmins who, while enjoying food given in faith, still engage in running errands and messages.}}\\
\end{addmargin}
\end{absolutelynopagebreak}

\begin{absolutelynopagebreak}
\setstretch{.7}
{\PaliGlossA{seyyathidaṃ—raññaṃ, rājamahāmattānaṃ, khattiyānaṃ, brāhmaṇānaṃ, gahapatikānaṃ, kumārānaṃ “idha gaccha, amutrāgaccha, idaṃ hara, amutra idaṃ āharā”ti}}\\
\begin{addmargin}[1em]{2em}
\setstretch{.5}
{\PaliGlossB{This includes running errands for rulers, ministers, aristocrats, brahmins, householders, or princes who say: “Go here, go there. Take this, bring that from there.”}}\\
\end{addmargin}
\end{absolutelynopagebreak}

\begin{absolutelynopagebreak}
\setstretch{.7}
{\PaliGlossA{iti vā iti evarūpā dūteyyapahiṇagamanānuyogā paṭivirato samaṇo gotamo’ti—}}\\
\begin{addmargin}[1em]{2em}
\setstretch{.5}
{\PaliGlossB{The ascetic Gotama refrains from such errands.’}}\\
\end{addmargin}
\end{absolutelynopagebreak}

\begin{absolutelynopagebreak}
\setstretch{.7}
{\PaliGlossA{iti vā hi, bhikkhave, puthujjano tathāgatassa vaṇṇaṃ vadamāno vadeyya.}}\\
\begin{addmargin}[1em]{2em}
\setstretch{.5}
{\PaliGlossB{Such is an ordinary person’s praise of the Realized One.}}\\
\end{addmargin}
\end{absolutelynopagebreak}

\begin{absolutelynopagebreak}
\setstretch{.7}
{\PaliGlossA{‘yathā vā paneke bhonto samaṇabrāhmaṇā saddhādeyyāni bhojanāni bhuñjitvā te kuhakā ca honti, lapakā ca nemittikā ca nippesikā ca, lābhena lābhaṃ nijigīsitāro ca}}\\
\begin{addmargin}[1em]{2em}
\setstretch{.5}
{\PaliGlossB{‘There are some ascetics and brahmins who, while enjoying food given in faith, still engage in deceit, flattery, hinting, and belittling, and using material possessions to pursue other material possessions.}}\\
\end{addmargin}
\end{absolutelynopagebreak}

\begin{absolutelynopagebreak}
\setstretch{.7}
{\PaliGlossA{iti evarūpā kuhanalapanā paṭivirato samaṇo gotamo’ti—}}\\
\begin{addmargin}[1em]{2em}
\setstretch{.5}
{\PaliGlossB{The ascetic Gotama refrains from such deceit and flattery.’}}\\
\end{addmargin}
\end{absolutelynopagebreak}

\begin{absolutelynopagebreak}
\setstretch{.7}
{\PaliGlossA{iti vā hi, bhikkhave, puthujjano tathāgatassa vaṇṇaṃ vadamāno vadeyya.}}\\
\begin{addmargin}[1em]{2em}
\setstretch{.5}
{\PaliGlossB{Such is an ordinary person’s praise of the Realized One.}}\\
\end{addmargin}
\end{absolutelynopagebreak}

\begin{absolutelynopagebreak}
\setstretch{.7}
{\PaliGlossA{majjhimasīlaṃ niṭṭhitaṃ.}}\\
\begin{addmargin}[1em]{2em}
\setstretch{.5}
{\PaliGlossB{The middle section on ethics is finished.}}\\
\end{addmargin}
\end{absolutelynopagebreak}

\begin{absolutelynopagebreak}
\setstretch{.7}
{\PaliGlossA{2.3. mahāsīla}}\\
\begin{addmargin}[1em]{2em}
\setstretch{.5}
{\PaliGlossB{2.3. The Large Section on Ethics}}\\
\end{addmargin}
\end{absolutelynopagebreak}

\begin{absolutelynopagebreak}
\setstretch{.7}
{\PaliGlossA{‘yathā vā paneke bhonto samaṇabrāhmaṇā saddhādeyyāni bhojanāni bhuñjitvā te evarūpāya tiracchānavijjāya micchājīvena jīvitaṃ kappenti,}}\\
\begin{addmargin}[1em]{2em}
\setstretch{.5}
{\PaliGlossB{‘There are some ascetics and brahmins who, while enjoying food given in faith, still earn a living by unworthy branches of knowledge, by wrong livelihood.}}\\
\end{addmargin}
\end{absolutelynopagebreak}

\begin{absolutelynopagebreak}
\setstretch{.7}
{\PaliGlossA{seyyathidaṃ—aṅgaṃ nimittaṃ uppātaṃ supinaṃ lakkhaṇaṃ mūsikacchinnaṃ aggihomaṃ dabbihomaṃ thusahomaṃ kaṇahomaṃ taṇḍulahomaṃ sappihomaṃ telahomaṃ mukhahomaṃ lohitahomaṃ aṅgavijjā vatthuvijjā khattavijjā sivavijjā bhūtavijjā bhūrivijjā ahivijjā visavijjā vicchikavijjā mūsikavijjā sakuṇavijjā vāyasavijjā pakkajjhānaṃ saraparittāṇaṃ migacakkaṃ}}\\
\begin{addmargin}[1em]{2em}
\setstretch{.5}
{\PaliGlossB{This includes such fields as limb-reading, omenology, divining celestial portents, interpreting dreams, divining bodily marks, divining holes in cloth gnawed by mice, fire offerings, ladle offerings, offerings of husks, rice powder, rice, ghee, or oil; offerings from the mouth, blood sacrifices, palmistry; geomancy for building sites, fields, and cemeteries; exorcisms, earth magic, snake charming, poisons; the crafts of the scorpion, the rat, the bird, and the crow; prophesying life span, chanting for protection, and animal cries.}}\\
\end{addmargin}
\end{absolutelynopagebreak}

\begin{absolutelynopagebreak}
\setstretch{.7}
{\PaliGlossA{iti vā iti evarūpāya tiracchānavijjāya micchājīvā paṭivirato samaṇo gotamo’ti—}}\\
\begin{addmargin}[1em]{2em}
\setstretch{.5}
{\PaliGlossB{The ascetic Gotama refrains from such unworthy branches of knowledge, such wrong livelihood.’}}\\
\end{addmargin}
\end{absolutelynopagebreak}

\begin{absolutelynopagebreak}
\setstretch{.7}
{\PaliGlossA{iti vā hi, bhikkhave, puthujjano tathāgatassa vaṇṇaṃ vadamāno vadeyya.}}\\
\begin{addmargin}[1em]{2em}
\setstretch{.5}
{\PaliGlossB{Such is an ordinary person’s praise of the Realized One.}}\\
\end{addmargin}
\end{absolutelynopagebreak}

\begin{absolutelynopagebreak}
\setstretch{.7}
{\PaliGlossA{‘yathā vā paneke bhonto samaṇabrāhmaṇā saddhādeyyāni bhojanāni bhuñjitvā te evarūpāya tiracchānavijjāya micchājīvena jīvitaṃ kappenti,}}\\
\begin{addmargin}[1em]{2em}
\setstretch{.5}
{\PaliGlossB{‘There are some ascetics and brahmins who, while enjoying food given in faith, still earn a living by unworthy branches of knowledge, by wrong livelihood.}}\\
\end{addmargin}
\end{absolutelynopagebreak}

\begin{absolutelynopagebreak}
\setstretch{.7}
{\PaliGlossA{seyyathidaṃ—maṇilakkhaṇaṃ vatthalakkhaṇaṃ daṇḍalakkhaṇaṃ satthalakkhaṇaṃ asilakkhaṇaṃ usulakkhaṇaṃ dhanulakkhaṇaṃ āvudhalakkhaṇaṃ itthilakkhaṇaṃ purisalakkhaṇaṃ kumāralakkhaṇaṃ kumārilakkhaṇaṃ dāsalakkhaṇaṃ dāsilakkhaṇaṃ hatthilakkhaṇaṃ assalakkhaṇaṃ mahiṃsalakkhaṇaṃ usabhalakkhaṇaṃ golakkhaṇaṃ ajalakkhaṇaṃ meṇḍalakkhaṇaṃ kukkuṭalakkhaṇaṃ vaṭṭakalakkhaṇaṃ godhālakkhaṇaṃ kaṇṇikālakkhaṇaṃ kacchapalakkhaṇaṃ migalakkhaṇaṃ}}\\
\begin{addmargin}[1em]{2em}
\setstretch{.5}
{\PaliGlossB{This includes reading the marks of gems, cloth, clubs, swords, spears, arrows, weapons, women, men, boys, girls, male and female bondservants, elephants, horses, buffaloes, bulls, cows, goats, rams, chickens, quails, monitor lizards, rabbits, tortoises, or deer.}}\\
\end{addmargin}
\end{absolutelynopagebreak}

\begin{absolutelynopagebreak}
\setstretch{.7}
{\PaliGlossA{iti vā iti evarūpāya tiracchānavijjāya micchājīvā paṭivirato samaṇo gotamo’ti—}}\\
\begin{addmargin}[1em]{2em}
\setstretch{.5}
{\PaliGlossB{The ascetic Gotama refrains from such unworthy branches of knowledge, such wrong livelihood.’}}\\
\end{addmargin}
\end{absolutelynopagebreak}

\begin{absolutelynopagebreak}
\setstretch{.7}
{\PaliGlossA{iti vā hi, bhikkhave, puthujjano tathāgatassa vaṇṇaṃ vadamāno vadeyya.}}\\
\begin{addmargin}[1em]{2em}
\setstretch{.5}
{\PaliGlossB{Such is an ordinary person’s praise of the Realized One.}}\\
\end{addmargin}
\end{absolutelynopagebreak}

\begin{absolutelynopagebreak}
\setstretch{.7}
{\PaliGlossA{‘yathā vā paneke bhonto samaṇabrāhmaṇā saddhādeyyāni bhojanāni bhuñjitvā te evarūpāya tiracchānavijjāya micchājīvena jīvitaṃ kappenti,}}\\
\begin{addmargin}[1em]{2em}
\setstretch{.5}
{\PaliGlossB{‘There are some ascetics and brahmins who, while enjoying food given in faith, still earn a living by unworthy branches of knowledge, by wrong livelihood.}}\\
\end{addmargin}
\end{absolutelynopagebreak}

\begin{absolutelynopagebreak}
\setstretch{.7}
{\PaliGlossA{seyyathidaṃ—raññaṃ niyyānaṃ bhavissati, raññaṃ aniyyānaṃ bhavissati, abbhantarānaṃ raññaṃ upayānaṃ bhavissati, bāhirānaṃ raññaṃ apayānaṃ bhavissati, bāhirānaṃ raññaṃ upayānaṃ bhavissati, abbhantarānaṃ raññaṃ apayānaṃ bhavissati, abbhantarānaṃ raññaṃ jayo bhavissati, bāhirānaṃ raññaṃ parājayo bhavissati, bāhirānaṃ raññaṃ jayo bhavissati, abbhantarānaṃ raññaṃ parājayo bhavissati, iti imassa jayo bhavissati, imassa parājayo bhavissati}}\\
\begin{addmargin}[1em]{2em}
\setstretch{.5}
{\PaliGlossB{This includes making predictions that the king will march forth or march back; or that our king will attack and the enemy king will retreat, or vice versa; or that our king will triumph and the enemy king will be defeated, or vice versa; and so there will be victory for one and defeat for the other.}}\\
\end{addmargin}
\end{absolutelynopagebreak}

\begin{absolutelynopagebreak}
\setstretch{.7}
{\PaliGlossA{iti vā iti evarūpāya tiracchānavijjāya micchājīvā paṭivirato samaṇo gotamo’ti—}}\\
\begin{addmargin}[1em]{2em}
\setstretch{.5}
{\PaliGlossB{The ascetic Gotama refrains from such unworthy branches of knowledge, such wrong livelihood.’}}\\
\end{addmargin}
\end{absolutelynopagebreak}

\begin{absolutelynopagebreak}
\setstretch{.7}
{\PaliGlossA{iti vā hi, bhikkhave, puthujjano tathāgatassa vaṇṇaṃ vadamāno vadeyya.}}\\
\begin{addmargin}[1em]{2em}
\setstretch{.5}
{\PaliGlossB{Such is an ordinary person’s praise of the Realized One.}}\\
\end{addmargin}
\end{absolutelynopagebreak}

\begin{absolutelynopagebreak}
\setstretch{.7}
{\PaliGlossA{‘yathā vā paneke bhonto samaṇabrāhmaṇā saddhādeyyāni bhojanāni bhuñjitvā te evarūpāya tiracchānavijjāya micchājīvena jīvitaṃ kappenti,}}\\
\begin{addmargin}[1em]{2em}
\setstretch{.5}
{\PaliGlossB{‘There are some ascetics and brahmins who, while enjoying food given in faith, still earn a living by unworthy branches of knowledge, by wrong livelihood.}}\\
\end{addmargin}
\end{absolutelynopagebreak}

\begin{absolutelynopagebreak}
\setstretch{.7}
{\PaliGlossA{seyyathidaṃ—candaggāho bhavissati, sūriyaggāho bhavissati, nakkhattaggāho bhavissati, candimasūriyānaṃ pathagamanaṃ bhavissati, candimasūriyānaṃ uppathagamanaṃ bhavissati, nakkhattānaṃ pathagamanaṃ bhavissati, nakkhattānaṃ uppathagamanaṃ bhavissati, ukkāpāto bhavissati, disāḍāho bhavissati, bhūmicālo bhavissati, devadudrabhi bhavissati, candimasūriyanakkhattānaṃ uggamanaṃ ogamanaṃ saṃkilesaṃ vodānaṃ bhavissati, evaṃvipāko candaggāho bhavissati, evaṃvipāko sūriyaggāho bhavissati, evaṃvipāko nakkhattaggāho bhavissati, evaṃvipākaṃ candimasūriyānaṃ pathagamanaṃ bhavissati, evaṃvipākaṃ candimasūriyānaṃ uppathagamanaṃ bhavissati, evaṃvipākaṃ nakkhattānaṃ pathagamanaṃ bhavissati, evaṃvipākaṃ nakkhattānaṃ uppathagamanaṃ bhavissati, evaṃvipāko ukkāpāto bhavissati, evaṃvipāko disāḍāho bhavissati, evaṃvipāko bhūmicālo bhavissati, evaṃvipāko devadudrabhi bhavissati, evaṃvipākaṃ candimasūriyanakkhattānaṃ uggamanaṃ ogamanaṃ saṃkilesaṃ vodānaṃ bhavissati}}\\
\begin{addmargin}[1em]{2em}
\setstretch{.5}
{\PaliGlossB{This includes making predictions that there will be an eclipse of the moon, or sun, or stars; that the sun, moon, and stars will be in conjunction or in opposition; that there will be a meteor shower, a fiery sky, an earthquake, thunder; that there will be a rising, a setting, a darkening, a brightening of the moon, sun, and stars. And it also includes making predictions about the results of all such phenomena.}}\\
\end{addmargin}
\end{absolutelynopagebreak}

\begin{absolutelynopagebreak}
\setstretch{.7}
{\PaliGlossA{iti vā iti evarūpāya tiracchānavijjāya micchājīvā paṭivirato samaṇo gotamo’ti—}}\\
\begin{addmargin}[1em]{2em}
\setstretch{.5}
{\PaliGlossB{The ascetic Gotama refrains from such unworthy branches of knowledge, such wrong livelihood.’}}\\
\end{addmargin}
\end{absolutelynopagebreak}

\begin{absolutelynopagebreak}
\setstretch{.7}
{\PaliGlossA{iti vā hi, bhikkhave, puthujjano tathāgatassa vaṇṇaṃ vadamāno vadeyya.}}\\
\begin{addmargin}[1em]{2em}
\setstretch{.5}
{\PaliGlossB{Such is an ordinary person’s praise of the Realized One.}}\\
\end{addmargin}
\end{absolutelynopagebreak}

\begin{absolutelynopagebreak}
\setstretch{.7}
{\PaliGlossA{‘yathā vā paneke bhonto samaṇabrāhmaṇā saddhādeyyāni bhojanāni bhuñjitvā te evarūpāya tiracchānavijjāya micchājīvena jīvitaṃ kappenti,}}\\
\begin{addmargin}[1em]{2em}
\setstretch{.5}
{\PaliGlossB{‘There are some ascetics and brahmins who, while enjoying food given in faith, still earn a living by unworthy branches of knowledge, by wrong livelihood.}}\\
\end{addmargin}
\end{absolutelynopagebreak}

\begin{absolutelynopagebreak}
\setstretch{.7}
{\PaliGlossA{seyyathidaṃ—suvuṭṭhikā bhavissati, dubbuṭṭhikā bhavissati, subhikkhaṃ bhavissati, dubbhikkhaṃ bhavissati, khemaṃ bhavissati, bhayaṃ bhavissati, rogo bhavissati, ārogyaṃ bhavissati, muddā, gaṇanā, saṅkhānaṃ, kāveyyaṃ, lokāyataṃ}}\\
\begin{addmargin}[1em]{2em}
\setstretch{.5}
{\PaliGlossB{This includes predicting whether there will be plenty of rain or drought; plenty to eat or famine; an abundant harvest or a bad harvest; security or peril; sickness or health. It also includes such occupations as computing, accounting, calculating, poetry, and cosmology.}}\\
\end{addmargin}
\end{absolutelynopagebreak}

\begin{absolutelynopagebreak}
\setstretch{.7}
{\PaliGlossA{iti vā iti evarūpāya tiracchānavijjāya micchājīvā paṭivirato samaṇo gotamo’ti—}}\\
\begin{addmargin}[1em]{2em}
\setstretch{.5}
{\PaliGlossB{The ascetic Gotama refrains from such unworthy branches of knowledge, such wrong livelihood.’}}\\
\end{addmargin}
\end{absolutelynopagebreak}

\begin{absolutelynopagebreak}
\setstretch{.7}
{\PaliGlossA{iti vā hi, bhikkhave, puthujjano tathāgatassa vaṇṇaṃ vadamāno vadeyya.}}\\
\begin{addmargin}[1em]{2em}
\setstretch{.5}
{\PaliGlossB{Such is an ordinary person’s praise of the Realized One.}}\\
\end{addmargin}
\end{absolutelynopagebreak}

\begin{absolutelynopagebreak}
\setstretch{.7}
{\PaliGlossA{‘yathā vā paneke bhonto samaṇabrāhmaṇā saddhādeyyāni bhojanāni bhuñjitvā te evarūpāya tiracchānavijjāya micchājīvena jīvitaṃ kappenti,}}\\
\begin{addmargin}[1em]{2em}
\setstretch{.5}
{\PaliGlossB{‘There are some ascetics and brahmins who, while enjoying food given in faith, still earn a living by unworthy branches of knowledge, by wrong livelihood.}}\\
\end{addmargin}
\end{absolutelynopagebreak}

\begin{absolutelynopagebreak}
\setstretch{.7}
{\PaliGlossA{seyyathidaṃ—āvāhanaṃ vivāhanaṃ saṃvaraṇaṃ vivaraṇaṃ saṃkiraṇaṃ vikiraṇaṃ subhagakaraṇaṃ dubbhagakaraṇaṃ viruddhagabbhakaraṇaṃ jivhānibandhanaṃ hanusaṃhananaṃ hatthābhijappanaṃ hanujappanaṃ kaṇṇajappanaṃ ādāsapañhaṃ kumārikapañhaṃ devapañhaṃ ādiccupaṭṭhānaṃ mahatupaṭṭhānaṃ abbhujjalanaṃ sirivhāyanaṃ}}\\
\begin{addmargin}[1em]{2em}
\setstretch{.5}
{\PaliGlossB{This includes making arrangements for giving and taking in marriage; for engagement and divorce; and for scattering rice inwards or outwards at the wedding ceremony. It also includes casting spells for good or bad luck, curses to prevent conception, bind the tongue, or lock the jaws; charms for the hands and ears; questioning a mirror, a girl, or a god as an oracle; worshiping the sun, worshiping the Great One, breathing fire, and invoking Siri, the goddess of luck.}}\\
\end{addmargin}
\end{absolutelynopagebreak}

\begin{absolutelynopagebreak}
\setstretch{.7}
{\PaliGlossA{iti vā iti evarūpāya tiracchānavijjāya micchājīvā paṭivirato samaṇo gotamo’ti—}}\\
\begin{addmargin}[1em]{2em}
\setstretch{.5}
{\PaliGlossB{The ascetic Gotama refrains from such unworthy branches of knowledge, such wrong livelihood.’}}\\
\end{addmargin}
\end{absolutelynopagebreak}

\begin{absolutelynopagebreak}
\setstretch{.7}
{\PaliGlossA{iti vā hi, bhikkhave, puthujjano tathāgatassa vaṇṇaṃ vadamāno vadeyya.}}\\
\begin{addmargin}[1em]{2em}
\setstretch{.5}
{\PaliGlossB{Such is an ordinary person’s praise of the Realized One.}}\\
\end{addmargin}
\end{absolutelynopagebreak}

\begin{absolutelynopagebreak}
\setstretch{.7}
{\PaliGlossA{‘yathā vā paneke bhonto samaṇabrāhmaṇā saddhādeyyāni bhojanāni bhuñjitvā te evarūpāya tiracchānavijjāya micchājīvena jīvitaṃ kappenti,}}\\
\begin{addmargin}[1em]{2em}
\setstretch{.5}
{\PaliGlossB{‘There are some ascetics and brahmins who, while enjoying food given in faith, still earn a living by unworthy branches of knowledge, by wrong livelihood.}}\\
\end{addmargin}
\end{absolutelynopagebreak}

\begin{absolutelynopagebreak}
\setstretch{.7}
{\PaliGlossA{seyyathidaṃ—santikammaṃ paṇidhikammaṃ bhūtakammaṃ bhūrikammaṃ vassakammaṃ vossakammaṃ vatthukammaṃ vatthuparikammaṃ ācamanaṃ nhāpanaṃ juhanaṃ vamanaṃ virecanaṃ uddhaṃvirecanaṃ adhovirecanaṃ sīsavirecanaṃ kaṇṇatelaṃ nettatappanaṃ natthukammaṃ añjanaṃ paccañjanaṃ sālākiyaṃ sallakattiyaṃ dārakatikicchā mūlabhesajjānaṃ anuppadānaṃ osadhīnaṃ paṭimokkho}}\\
\begin{addmargin}[1em]{2em}
\setstretch{.5}
{\PaliGlossB{This includes rites for propitiation, for granting wishes, for ghosts, for the earth, for rain, for property settlement, and for preparing and consecrating house sites, and rites involving rinsing and bathing, and oblations. It also includes administering emetics, purgatives, expectorants, and phlegmagogues; administering ear-oils, eye restoratives, nasal medicine, ointments, and counter-ointments; surgery with needle and scalpel, treating children, prescribing root medicines, and binding on herbs.}}\\
\end{addmargin}
\end{absolutelynopagebreak}

\begin{absolutelynopagebreak}
\setstretch{.7}
{\PaliGlossA{iti vā iti evarūpāya tiracchānavijjāya micchājīvā paṭivirato samaṇo gotamo’ti—}}\\
\begin{addmargin}[1em]{2em}
\setstretch{.5}
{\PaliGlossB{The ascetic Gotama refrains from such unworthy branches of knowledge, such wrong livelihood.’}}\\
\end{addmargin}
\end{absolutelynopagebreak}

\begin{absolutelynopagebreak}
\setstretch{.7}
{\PaliGlossA{iti vā hi, bhikkhave, puthujjano tathāgatassa vaṇṇaṃ vadamāno vadeyya.}}\\
\begin{addmargin}[1em]{2em}
\setstretch{.5}
{\PaliGlossB{Such is an ordinary person’s praise of the Realized One.}}\\
\end{addmargin}
\end{absolutelynopagebreak}

\begin{absolutelynopagebreak}
\setstretch{.7}
{\PaliGlossA{idaṃ kho, bhikkhave, appamattakaṃ oramattakaṃ sīlamattakaṃ, yena puthujjano tathāgatassa vaṇṇaṃ vadamāno vadeyya.}}\\
\begin{addmargin}[1em]{2em}
\setstretch{.5}
{\PaliGlossB{These are the trivial, insignificant details of mere ethics that an ordinary person speaks of when they speak praise of the Realized One.}}\\
\end{addmargin}
\end{absolutelynopagebreak}

\begin{absolutelynopagebreak}
\setstretch{.7}
{\PaliGlossA{mahāsīlaṃ niṭṭhitaṃ.}}\\
\begin{addmargin}[1em]{2em}
\setstretch{.5}
{\PaliGlossB{The longer section on ethics is finished.}}\\
\end{addmargin}
\end{absolutelynopagebreak}

\begin{absolutelynopagebreak}
\setstretch{.7}
{\PaliGlossA{3. diṭṭhi}}\\
\begin{addmargin}[1em]{2em}
\setstretch{.5}
{\PaliGlossB{3. Views}}\\
\end{addmargin}
\end{absolutelynopagebreak}

\begin{absolutelynopagebreak}
\setstretch{.7}
{\PaliGlossA{3.1. pubbantakappika}}\\
\begin{addmargin}[1em]{2em}
\setstretch{.5}
{\PaliGlossB{3.1. Theories About the Past}}\\
\end{addmargin}
\end{absolutelynopagebreak}

\begin{absolutelynopagebreak}
\setstretch{.7}
{\PaliGlossA{atthi, bhikkhave, aññeva dhammā gambhīrā duddasā duranubodhā santā paṇītā atakkāvacarā nipuṇā paṇḍitavedanīyā, ye tathāgato sayaṃ abhiññā sacchikatvā pavedeti, yehi tathāgatassa yathābhuccaṃ vaṇṇaṃ sammā vadamānā vadeyyuṃ.}}\\
\begin{addmargin}[1em]{2em}
\setstretch{.5}
{\PaliGlossB{There are other principles—deep, hard to see, hard to understand, peaceful, sublime, beyond the scope of reason, subtle, comprehensible to the astute—which the Realized One makes known after realizing them with his own insight. Those who genuinely praise the Realized One would rightly speak of these things.}}\\
\end{addmargin}
\end{absolutelynopagebreak}

\begin{absolutelynopagebreak}
\setstretch{.7}
{\PaliGlossA{katame ca te, bhikkhave, dhammā gambhīrā duddasā duranubodhā santā paṇītā atakkāvacarā nipuṇā paṇḍitavedanīyā, ye tathāgato sayaṃ abhiññā sacchikatvā pavedeti, yehi tathāgatassa yathābhuccaṃ vaṇṇaṃ sammā vadamānā vadeyyuṃ?}}\\
\begin{addmargin}[1em]{2em}
\setstretch{.5}
{\PaliGlossB{And what are these principles?}}\\
\end{addmargin}
\end{absolutelynopagebreak}

\begin{absolutelynopagebreak}
\setstretch{.7}
{\PaliGlossA{santi, bhikkhave, eke samaṇabrāhmaṇā pubbantakappikā pubbantānudiṭṭhino, pubbantaṃ ārabbha anekavihitāni adhimuttipadāni abhivadanti aṭṭhārasahi vatthūhi.}}\\
\begin{addmargin}[1em]{2em}
\setstretch{.5}
{\PaliGlossB{There are some ascetics and brahmins who theorize about the past, and assert various hypotheses concerning the past on eighteen grounds.}}\\
\end{addmargin}
\end{absolutelynopagebreak}

\begin{absolutelynopagebreak}
\setstretch{.7}
{\PaliGlossA{te ca bhonto samaṇabrāhmaṇā kimāgamma kimārabbha pubbantakappikā pubbantānudiṭṭhino pubbantaṃ ārabbha anekavihitāni adhimuttipadāni abhivadanti aṭṭhārasahi vatthūhi?}}\\
\begin{addmargin}[1em]{2em}
\setstretch{.5}
{\PaliGlossB{And what are the eighteen grounds on which they rely?}}\\
\end{addmargin}
\end{absolutelynopagebreak}

\begin{absolutelynopagebreak}
\setstretch{.7}
{\PaliGlossA{3.1.1. sassatavāda}}\\
\begin{addmargin}[1em]{2em}
\setstretch{.5}
{\PaliGlossB{3.1.1. Eternalism}}\\
\end{addmargin}
\end{absolutelynopagebreak}

\begin{absolutelynopagebreak}
\setstretch{.7}
{\PaliGlossA{santi, bhikkhave, eke samaṇabrāhmaṇā sassatavādā, sassataṃ attānañca lokañca paññapenti catūhi vatthūhi.}}\\
\begin{addmargin}[1em]{2em}
\setstretch{.5}
{\PaliGlossB{There are some ascetics and brahmins who are eternalists, who assert that the self and the cosmos are eternal on four grounds.}}\\
\end{addmargin}
\end{absolutelynopagebreak}

\begin{absolutelynopagebreak}
\setstretch{.7}
{\PaliGlossA{te ca bhonto samaṇabrāhmaṇā kimāgamma kimārabbha sassatavādā sassataṃ attānañca lokañca paññapenti catūhi vatthūhi?}}\\
\begin{addmargin}[1em]{2em}
\setstretch{.5}
{\PaliGlossB{And what are the four grounds on which they rely?}}\\
\end{addmargin}
\end{absolutelynopagebreak}

\begin{absolutelynopagebreak}
\setstretch{.7}
{\PaliGlossA{idha, bhikkhave, ekacco samaṇo vā brāhmaṇo vā ātappamanvāya padhānamanvāya anuyogamanvāya appamādamanvāya sammāmanasikāramanvāya tathārūpaṃ cetosamādhiṃ phusati, yathāsamāhite citte () anekavihitaṃ pubbenivāsaṃ anussarati.}}\\
\begin{addmargin}[1em]{2em}
\setstretch{.5}
{\PaliGlossB{It’s when some ascetic or brahmin—by dint of keen, resolute, committed, and diligent effort, and right focus—experiences an immersion of the heart of such a kind that they recollect their many kinds of past lives.}}\\
\end{addmargin}
\end{absolutelynopagebreak}

\begin{absolutelynopagebreak}
\setstretch{.7}
{\PaliGlossA{seyyathidaṃ—ekampi jātiṃ dvepi jātiyo tissopi jātiyo catassopi jātiyo pañcapi jātiyo dasapi jātiyo vīsampi jātiyo tiṃsampi jātiyo cattālīsampi jātiyo paññāsampi jātiyo jātisatampi jātisahassampi jātisatasahassampi anekānipi jātisatāni anekānipi jātisahassāni anekānipi jātisatasahassāni: ‘amutrāsiṃ evaṃnāmo evaṃgotto evaṃvaṇṇo evamāhāro evaṃsukhadukkhappaṭisaṃvedī evamāyupariyanto, so tato cuto amutra udapādiṃ; tatrāpāsiṃ evaṃnāmo evaṃgotto evaṃvaṇṇo evamāhāro evaṃsukhadukkhappaṭisaṃvedī evamāyupariyanto, so tato cuto idhūpapanno’ti. iti sākāraṃ sauddesaṃ anekavihitaṃ pubbenivāsaṃ anussarati.}}\\
\begin{addmargin}[1em]{2em}
\setstretch{.5}
{\PaliGlossB{That is: one, two, three, four, five, ten, twenty, thirty, forty, fifty, a hundred, a thousand, a hundred thousand rebirths; many eons of the cosmos contracting, many eons of the cosmos expanding, many eons of the cosmos contracting and expanding. They remember: ‘There, I was named this, my clan was that, I looked like this, and that was my food. This was how I felt pleasure and pain, and that was how my life ended. When I passed away from that place I was reborn somewhere else. There, too, I was named this, my clan was that, I looked like this, and that was my food. This was how I felt pleasure and pain, and that was how my life ended. When I passed away from that place I was reborn here.’ And so they recollect their many kinds of past lives, with features and details.}}\\
\end{addmargin}
\end{absolutelynopagebreak}

\begin{absolutelynopagebreak}
\setstretch{.7}
{\PaliGlossA{so evamāha:}}\\
\begin{addmargin}[1em]{2em}
\setstretch{.5}
{\PaliGlossB{They say:}}\\
\end{addmargin}
\end{absolutelynopagebreak}

\begin{absolutelynopagebreak}
\setstretch{.7}
{\PaliGlossA{‘sassato attā ca loko ca vañjho kūṭaṭṭho esikaṭṭhāyiṭṭhito;}}\\
\begin{addmargin}[1em]{2em}
\setstretch{.5}
{\PaliGlossB{‘The self and the cosmos are eternal, barren, steady as a mountain peak, standing firm like a pillar.}}\\
\end{addmargin}
\end{absolutelynopagebreak}

\begin{absolutelynopagebreak}
\setstretch{.7}
{\PaliGlossA{te ca sattā sandhāvanti saṃsaranti cavanti upapajjanti, atthi tveva sassatisamaṃ.}}\\
\begin{addmargin}[1em]{2em}
\setstretch{.5}
{\PaliGlossB{They remain the same for all eternity, while these sentient beings wander and transmigrate and pass away and rearise.}}\\
\end{addmargin}
\end{absolutelynopagebreak}

\begin{absolutelynopagebreak}
\setstretch{.7}
{\PaliGlossA{taṃ kissa hetu?}}\\
\begin{addmargin}[1em]{2em}
\setstretch{.5}
{\PaliGlossB{Why is that?}}\\
\end{addmargin}
\end{absolutelynopagebreak}

\begin{absolutelynopagebreak}
\setstretch{.7}
{\PaliGlossA{ahañhi ātappamanvāya padhānamanvāya anuyogamanvāya appamādamanvāya sammāmanasikāramanvāya tathārūpaṃ cetosamādhiṃ phusāmi, yathāsamāhite citte anekavihitaṃ pubbenivāsaṃ anussarāmi.}}\\
\begin{addmargin}[1em]{2em}
\setstretch{.5}
{\PaliGlossB{Because by dint of keen, resolute, committed, and diligent effort, and right focus I experience an immersion of the heart of such a kind that I recollect my many kinds of past lives,}}\\
\end{addmargin}
\end{absolutelynopagebreak}

\begin{absolutelynopagebreak}
\setstretch{.7}
{\PaliGlossA{seyyathidaṃ—ekampi jātiṃ dvepi jātiyo tissopi jātiyo catassopi jātiyo pañcapi jātiyo dasapi jātiyo vīsampi jātiyo tiṃsampi jātiyo cattālīsampi jātiyo paññāsampi jātiyo jātisatampi jātisahassampi jātisatasahassampi anekānipi jātisatāni anekānipi jātisahassāni anekānipi jātisatasahassāni: “amutrāsiṃ evaṃnāmo evaṅgotto evaṃvaṇṇo evamāhāro evaṃsukhadukkhappaṭisaṃvedī evamāyupariyanto, so tato cuto amutra udapādiṃ; tatrāpāsiṃ evaṃnāmo evaṅgotto evaṃvaṇṇo evamāhāro evaṃsukhadukkhappaṭisaṃvedī evamāyupariyanto, so tato cuto idhūpapanno”ti. iti sākāraṃ sauddesaṃ anekavihitaṃ pubbenivāsaṃ anussarāmi.}}\\
\begin{addmargin}[1em]{2em}
\setstretch{.5}
{\PaliGlossB{with features and details.}}\\
\end{addmargin}
\end{absolutelynopagebreak}

\begin{absolutelynopagebreak}
\setstretch{.7}
{\PaliGlossA{imināmahaṃ etaṃ jānāmi:}}\\
\begin{addmargin}[1em]{2em}
\setstretch{.5}
{\PaliGlossB{Because of this I know:}}\\
\end{addmargin}
\end{absolutelynopagebreak}

\begin{absolutelynopagebreak}
\setstretch{.7}
{\PaliGlossA{“yathā sassato attā ca loko ca vañjho kūṭaṭṭho esikaṭṭhāyiṭṭhito;}}\\
\begin{addmargin}[1em]{2em}
\setstretch{.5}
{\PaliGlossB{“The self and the cosmos are eternal, barren, steady as a mountain peak, standing firm like a pillar.}}\\
\end{addmargin}
\end{absolutelynopagebreak}

\begin{absolutelynopagebreak}
\setstretch{.7}
{\PaliGlossA{te ca sattā sandhāvanti saṃsaranti cavanti upapajjanti, atthi tveva sassatisaman”’ti.}}\\
\begin{addmargin}[1em]{2em}
\setstretch{.5}
{\PaliGlossB{They remain the same for all eternity, while these sentient beings wander and transmigrate and pass away and rearise.’}}\\
\end{addmargin}
\end{absolutelynopagebreak}

\begin{absolutelynopagebreak}
\setstretch{.7}
{\PaliGlossA{idaṃ, bhikkhave, paṭhamaṃ ṭhānaṃ, yaṃ āgamma yaṃ ārabbha eke samaṇabrāhmaṇā sassatavādā sassataṃ attānañca lokañca paññapenti. (1: 1)}}\\
\begin{addmargin}[1em]{2em}
\setstretch{.5}
{\PaliGlossB{This is the first ground on which some ascetics and brahmins rely to assert that the self and the cosmos are eternal.}}\\
\end{addmargin}
\end{absolutelynopagebreak}

\begin{absolutelynopagebreak}
\setstretch{.7}
{\PaliGlossA{dutiye ca bhonto samaṇabrāhmaṇā kimāgamma kimārabbha sassatavādā sassataṃ attānañca lokañca paññapenti?}}\\
\begin{addmargin}[1em]{2em}
\setstretch{.5}
{\PaliGlossB{And what is the second ground on which they rely?}}\\
\end{addmargin}
\end{absolutelynopagebreak}

\begin{absolutelynopagebreak}
\setstretch{.7}
{\PaliGlossA{idha, bhikkhave, ekacco samaṇo vā brāhmaṇo vā ātappamanvāya padhānamanvāya anuyogamanvāya appamādamanvāya sammāmanasikāramanvāya tathārūpaṃ cetosamādhiṃ phusati, yathāsamāhite citte anekavihitaṃ pubbenivāsaṃ anussarati.}}\\
\begin{addmargin}[1em]{2em}
\setstretch{.5}
{\PaliGlossB{It’s when some ascetic or brahmin—by dint of keen, resolute, committed, and diligent effort, and right focus—experiences an immersion of the heart of such a kind that they recollect their many kinds of past lives.}}\\
\end{addmargin}
\end{absolutelynopagebreak}

\begin{absolutelynopagebreak}
\setstretch{.7}
{\PaliGlossA{seyyathidaṃ—ekampi saṃvaṭṭavivaṭṭaṃ dvepi saṃvaṭṭavivaṭṭāni tīṇipi saṃvaṭṭavivaṭṭāni cattāripi saṃvaṭṭavivaṭṭāni pañcapi saṃvaṭṭavivaṭṭāni dasapi saṃvaṭṭavivaṭṭāni: ‘amutrāsiṃ evaṃnāmo evaṅgotto evaṃvaṇṇo evamāhāro evaṃsukhadukkhappaṭisaṃvedī evamāyupariyanto, so tato cuto amutra udapādiṃ; tatrāpāsiṃ evaṃnāmo evaṅgotto evaṃvaṇṇo evamāhāro evaṃsukhadukkhappaṭisaṃvedī evamāyupariyanto, so tato cuto idhūpapanno’ti. iti sākāraṃ sauddesaṃ anekavihitaṃ pubbenivāsaṃ anussarati.}}\\
\begin{addmargin}[1em]{2em}
\setstretch{.5}
{\PaliGlossB{That is: one eon of the cosmos contracting and expanding; two, three, four, five, or ten eons of the cosmos contracting and expanding. They remember: ‘There, I was named this, my clan was that, I looked like this, and that was my food. This was how I felt pleasure and pain, and that was how my life ended. When I passed away from that place I was reborn somewhere else. There, too, I was named this, my clan was that, I looked like this, and that was my food. This was how I felt pleasure and pain, and that was how my life ended. When I passed away from that place I was reborn here.’ And so they recollect their many kinds of past lives, with features and details.}}\\
\end{addmargin}
\end{absolutelynopagebreak}

\begin{absolutelynopagebreak}
\setstretch{.7}
{\PaliGlossA{so evamāha:}}\\
\begin{addmargin}[1em]{2em}
\setstretch{.5}
{\PaliGlossB{They say:}}\\
\end{addmargin}
\end{absolutelynopagebreak}

\begin{absolutelynopagebreak}
\setstretch{.7}
{\PaliGlossA{‘sassato attā ca loko ca vañjho kūṭaṭṭho esikaṭṭhāyiṭṭhito;}}\\
\begin{addmargin}[1em]{2em}
\setstretch{.5}
{\PaliGlossB{‘The self and the cosmos are eternal, barren, steady as a mountain peak, standing firm like a pillar.}}\\
\end{addmargin}
\end{absolutelynopagebreak}

\begin{absolutelynopagebreak}
\setstretch{.7}
{\PaliGlossA{te ca sattā sandhāvanti saṃsaranti cavanti upapajjanti, atthi tveva sassatisamaṃ.}}\\
\begin{addmargin}[1em]{2em}
\setstretch{.5}
{\PaliGlossB{They remain the same for all eternity, while these sentient beings wander and transmigrate and pass away and rearise.}}\\
\end{addmargin}
\end{absolutelynopagebreak}

\begin{absolutelynopagebreak}
\setstretch{.7}
{\PaliGlossA{taṃ kissa hetu?}}\\
\begin{addmargin}[1em]{2em}
\setstretch{.5}
{\PaliGlossB{Why is that?}}\\
\end{addmargin}
\end{absolutelynopagebreak}

\begin{absolutelynopagebreak}
\setstretch{.7}
{\PaliGlossA{ahañhi ātappamanvāya padhānamanvāya anuyogamanvāya appamādamanvāya sammāmanasikāramanvāya tathārūpaṃ cetosamādhiṃ phusāmi yathāsamāhite citte anekavihitaṃ pubbenivāsaṃ anussarāmi.}}\\
\begin{addmargin}[1em]{2em}
\setstretch{.5}
{\PaliGlossB{Because by dint of keen, resolute, committed, and diligent effort, and right focus I experience an immersion of the heart of such a kind that I recollect my many kinds of past lives,}}\\
\end{addmargin}
\end{absolutelynopagebreak}

\begin{absolutelynopagebreak}
\setstretch{.7}
{\PaliGlossA{seyyathidaṃ—ekampi saṃvaṭṭavivaṭṭaṃ dvepi saṃvaṭṭavivaṭṭāni tīṇipi saṃvaṭṭavivaṭṭāni cattāripi saṃvaṭṭavivaṭṭāni pañcapi saṃvaṭṭavivaṭṭāni dasapi saṃvaṭṭavivaṭṭāni: “amutrāsiṃ evaṃnāmo evaṅgotto evaṃvaṇṇo evamāhāro evaṃsukhadukkhappaṭisaṃvedī evamāyupariyanto, so tato cuto amutra udapādiṃ; tatrāpāsiṃ evaṃnāmo evaṅgotto evaṃvaṇṇo evamāhāro evaṃsukhadukkhappaṭisaṃvedī evamāyupariyanto, so tato cuto idhūpapanno”ti. iti sākāraṃ sauddesaṃ anekavihitaṃ pubbenivāsaṃ anussarāmi.}}\\
\begin{addmargin}[1em]{2em}
\setstretch{.5}
{\PaliGlossB{with features and details.}}\\
\end{addmargin}
\end{absolutelynopagebreak}

\begin{absolutelynopagebreak}
\setstretch{.7}
{\PaliGlossA{imināmahaṃ etaṃ jānāmi:}}\\
\begin{addmargin}[1em]{2em}
\setstretch{.5}
{\PaliGlossB{Because of this I know:}}\\
\end{addmargin}
\end{absolutelynopagebreak}

\begin{absolutelynopagebreak}
\setstretch{.7}
{\PaliGlossA{“yathā sassato attā ca loko ca vañjho kūṭaṭṭho esikaṭṭhāyiṭṭhito, te ca sattā sandhāvanti saṃsaranti cavanti upapajjanti, atthi tveva sassatisaman”’ti.}}\\
\begin{addmargin}[1em]{2em}
\setstretch{.5}
{\PaliGlossB{“The self and the cosmos are eternal, barren, steady as a mountain peak, standing firm like a pillar. They remain the same for all eternity, while these sentient beings wander and transmigrate and pass away and rearise.”’}}\\
\end{addmargin}
\end{absolutelynopagebreak}

\begin{absolutelynopagebreak}
\setstretch{.7}
{\PaliGlossA{idaṃ, bhikkhave, dutiyaṃ ṭhānaṃ, yaṃ āgamma yaṃ ārabbha eke samaṇabrāhmaṇā sassatavādā sassataṃ attānañca lokañca paññapenti. (2: 2)}}\\
\begin{addmargin}[1em]{2em}
\setstretch{.5}
{\PaliGlossB{This is the second ground on which some ascetics and brahmins rely to assert that the self and the cosmos are eternal.}}\\
\end{addmargin}
\end{absolutelynopagebreak}

\begin{absolutelynopagebreak}
\setstretch{.7}
{\PaliGlossA{tatiye ca bhonto samaṇabrāhmaṇā kimāgamma kimārabbha sassatavādā sassataṃ attānañca lokañca paññapenti?}}\\
\begin{addmargin}[1em]{2em}
\setstretch{.5}
{\PaliGlossB{And what is the third ground on which they rely?}}\\
\end{addmargin}
\end{absolutelynopagebreak}

\begin{absolutelynopagebreak}
\setstretch{.7}
{\PaliGlossA{idha, bhikkhave, ekacco samaṇo vā brāhmaṇo vā ātappamanvāya padhānamanvāya anuyogamanvāya appamādamanvāya sammāmanasikāramanvāya tathārūpaṃ cetosamādhiṃ phusati, yathāsamāhite citte anekavihitaṃ pubbenivāsaṃ anussarati.}}\\
\begin{addmargin}[1em]{2em}
\setstretch{.5}
{\PaliGlossB{It’s when some ascetic or brahmin—by dint of keen, resolute, committed, and diligent effort, and right focus—experiences an immersion of the heart of such a kind that they recollect their many kinds of past lives.}}\\
\end{addmargin}
\end{absolutelynopagebreak}

\begin{absolutelynopagebreak}
\setstretch{.7}
{\PaliGlossA{seyyathidaṃ—dasapi saṃvaṭṭavivaṭṭāni vīsampi saṃvaṭṭavivaṭṭāni tiṃsampi saṃvaṭṭavivaṭṭāni cattālīsampi saṃvaṭṭavivaṭṭāni: ‘amutrāsiṃ evaṃnāmo evaṅgotto evaṃvaṇṇo evamāhāro evaṃsukhadukkhappaṭisaṃvedī evamāyupariyanto, so tato cuto amutra udapādiṃ; tatrāpāsiṃ evaṃnāmo evaṅgotto evaṃvaṇṇo evamāhāro evaṃsukhadukkhappaṭisaṃvedī evamāyupariyanto, so tato cuto idhūpapanno’ti. iti sākāraṃ sauddesaṃ anekavihitaṃ pubbenivāsaṃ anussarati.}}\\
\begin{addmargin}[1em]{2em}
\setstretch{.5}
{\PaliGlossB{That is: ten eons of the cosmos contracting and expanding; twenty, thirty, or forty eons of the cosmos contracting and expanding. They remember: ‘There, I was named this, my clan was that, I looked like this, and that was my food. This was how I felt pleasure and pain, and that was how my life ended. When I passed away from that place I was reborn somewhere else. There, too, I was named this, my clan was that, I looked like this, and that was my food. This was how I felt pleasure and pain, and that was how my life ended. When I passed away from that place I was reborn here.’ And so they recollect their many kinds of past lives, with features and details.}}\\
\end{addmargin}
\end{absolutelynopagebreak}

\begin{absolutelynopagebreak}
\setstretch{.7}
{\PaliGlossA{so evamāha:}}\\
\begin{addmargin}[1em]{2em}
\setstretch{.5}
{\PaliGlossB{They say:}}\\
\end{addmargin}
\end{absolutelynopagebreak}

\begin{absolutelynopagebreak}
\setstretch{.7}
{\PaliGlossA{‘sassato attā ca loko ca vañjho kūṭaṭṭho esikaṭṭhāyiṭṭhito;}}\\
\begin{addmargin}[1em]{2em}
\setstretch{.5}
{\PaliGlossB{‘The self and the cosmos are eternal, barren, steady as a mountain peak, standing firm like a pillar.}}\\
\end{addmargin}
\end{absolutelynopagebreak}

\begin{absolutelynopagebreak}
\setstretch{.7}
{\PaliGlossA{te ca sattā sandhāvanti saṃsaranti cavanti upapajjanti, atthi tveva sassatisamaṃ.}}\\
\begin{addmargin}[1em]{2em}
\setstretch{.5}
{\PaliGlossB{They remain the same for all eternity, while these sentient beings wander and transmigrate and pass away and rearise.}}\\
\end{addmargin}
\end{absolutelynopagebreak}

\begin{absolutelynopagebreak}
\setstretch{.7}
{\PaliGlossA{taṃ kissa hetu?}}\\
\begin{addmargin}[1em]{2em}
\setstretch{.5}
{\PaliGlossB{Why is that?}}\\
\end{addmargin}
\end{absolutelynopagebreak}

\begin{absolutelynopagebreak}
\setstretch{.7}
{\PaliGlossA{ahañhi ātappamanvāya padhānamanvāya anuyogamanvāya appamādamanvāya sammāmanasikāramanvāya tathārūpaṃ cetosamādhiṃ phusāmi, yathāsamāhite citte anekavihitaṃ pubbenivāsaṃ anussarāmi.}}\\
\begin{addmargin}[1em]{2em}
\setstretch{.5}
{\PaliGlossB{Because by dint of keen, resolute, committed, and diligent effort, and right focus I experience an immersion of the heart of such a kind that I recollect my many kinds of past lives,}}\\
\end{addmargin}
\end{absolutelynopagebreak}

\begin{absolutelynopagebreak}
\setstretch{.7}
{\PaliGlossA{seyyathidaṃ—dasapi saṃvaṭṭavivaṭṭāni vīsampi saṃvaṭṭavivaṭṭāni tiṃsampi saṃvaṭṭavivaṭṭāni cattālīsampi saṃvaṭṭavivaṭṭāni: “amutrāsiṃ evaṃnāmo evaṅgotto evaṃvaṇṇo evamāhāro evaṃsukhadukkhappaṭisaṃvedī evamāyupariyanto, so tato cuto amutra udapādiṃ; tatrāpāsiṃ evaṃnāmo evaṅgotto evaṃvaṇṇo evamāhāro evaṃsukhadukkhappaṭisaṃvedī evamāyupariyanto, so tato cuto idhūpapanno”ti. iti sākāraṃ sauddesaṃ anekavihitaṃ pubbenivāsaṃ anussarāmi.}}\\
\begin{addmargin}[1em]{2em}
\setstretch{.5}
{\PaliGlossB{with features and details.}}\\
\end{addmargin}
\end{absolutelynopagebreak}

\begin{absolutelynopagebreak}
\setstretch{.7}
{\PaliGlossA{imināmahaṃ etaṃ jānāmi:}}\\
\begin{addmargin}[1em]{2em}
\setstretch{.5}
{\PaliGlossB{Because of this I know:}}\\
\end{addmargin}
\end{absolutelynopagebreak}

\begin{absolutelynopagebreak}
\setstretch{.7}
{\PaliGlossA{“yathā sassato attā ca loko ca vañjho kūṭaṭṭho esikaṭṭhāyiṭṭhito, te ca sattā sandhāvanti saṃsaranti cavanti upapajjanti, atthi tveva sassatisaman”’ti.}}\\
\begin{addmargin}[1em]{2em}
\setstretch{.5}
{\PaliGlossB{“The self and the cosmos are eternal, barren, steady as a mountain peak, standing firm like a pillar. They remain the same for all eternity, while these sentient beings wander and transmigrate and pass away and rearise.”’}}\\
\end{addmargin}
\end{absolutelynopagebreak}

\begin{absolutelynopagebreak}
\setstretch{.7}
{\PaliGlossA{idaṃ, bhikkhave, tatiyaṃ ṭhānaṃ, yaṃ āgamma yaṃ ārabbha eke samaṇabrāhmaṇā sassatavādā sassataṃ attānañca lokañca paññapenti. (3: 3)}}\\
\begin{addmargin}[1em]{2em}
\setstretch{.5}
{\PaliGlossB{This is the third ground on which some ascetics and brahmins rely to assert that the self and the cosmos are eternal.}}\\
\end{addmargin}
\end{absolutelynopagebreak}

\begin{absolutelynopagebreak}
\setstretch{.7}
{\PaliGlossA{catutthe ca bhonto samaṇabrāhmaṇā kimāgamma kimārabbha sassatavādā sassataṃ attānañca lokañca paññapenti?}}\\
\begin{addmargin}[1em]{2em}
\setstretch{.5}
{\PaliGlossB{And what is the fourth ground on which they rely?}}\\
\end{addmargin}
\end{absolutelynopagebreak}

\begin{absolutelynopagebreak}
\setstretch{.7}
{\PaliGlossA{idha, bhikkhave, ekacco samaṇo vā brāhmaṇo vā takkī hoti vīmaṃsī, so takkapariyāhataṃ vīmaṃsānucaritaṃ sayaṃ paṭibhānaṃ evamāha:}}\\
\begin{addmargin}[1em]{2em}
\setstretch{.5}
{\PaliGlossB{It’s when some ascetic or brahmin relies on logic and inquiry. They speak of what they have worked out by logic, following a line of inquiry, expressing their own perspective:}}\\
\end{addmargin}
\end{absolutelynopagebreak}

\begin{absolutelynopagebreak}
\setstretch{.7}
{\PaliGlossA{‘sassato attā ca loko ca vañjho kūṭaṭṭho esikaṭṭhāyiṭṭhito;}}\\
\begin{addmargin}[1em]{2em}
\setstretch{.5}
{\PaliGlossB{‘The self and the cosmos are eternal, barren, steady as a mountain peak, standing firm like a pillar.}}\\
\end{addmargin}
\end{absolutelynopagebreak}

\begin{absolutelynopagebreak}
\setstretch{.7}
{\PaliGlossA{te ca sattā sandhāvanti saṃsaranti cavanti upapajjanti, atthi tveva sassatisaman’ti.}}\\
\begin{addmargin}[1em]{2em}
\setstretch{.5}
{\PaliGlossB{They remain the same for all eternity, while these sentient beings wander and transmigrate and pass away and rearise.’}}\\
\end{addmargin}
\end{absolutelynopagebreak}

\begin{absolutelynopagebreak}
\setstretch{.7}
{\PaliGlossA{idaṃ, bhikkhave, catutthaṃ ṭhānaṃ, yaṃ āgamma yaṃ ārabbha eke samaṇabrāhmaṇā sassatavādā sassataṃ attānañca lokañca paññapenti. (4: 4)}}\\
\begin{addmargin}[1em]{2em}
\setstretch{.5}
{\PaliGlossB{This is the fourth ground on which some ascetics and brahmins rely to assert that the self and the cosmos are eternal.}}\\
\end{addmargin}
\end{absolutelynopagebreak}

\begin{absolutelynopagebreak}
\setstretch{.7}
{\PaliGlossA{imehi kho te, bhikkhave, samaṇabrāhmaṇā sassatavādā sassataṃ attānañca lokañca paññapenti catūhi vatthūhi.}}\\
\begin{addmargin}[1em]{2em}
\setstretch{.5}
{\PaliGlossB{These are the four grounds on which those ascetics and brahmins assert that the self and the cosmos are eternal.}}\\
\end{addmargin}
\end{absolutelynopagebreak}

\begin{absolutelynopagebreak}
\setstretch{.7}
{\PaliGlossA{ye hi keci, bhikkhave, samaṇā vā brāhmaṇā vā sassatavādā sassataṃ attānañca lokañca paññapenti, sabbe te imeheva catūhi vatthūhi, etesaṃ vā aññatarena; natthi ito bahiddhā.}}\\
\begin{addmargin}[1em]{2em}
\setstretch{.5}
{\PaliGlossB{Any ascetics and brahmins who assert that the self and the cosmos are eternal do so on one or other of these four grounds. Outside of this there is none.}}\\
\end{addmargin}
\end{absolutelynopagebreak}

\begin{absolutelynopagebreak}
\setstretch{.7}
{\PaliGlossA{tayidaṃ, bhikkhave, tathāgato pajānāti:}}\\
\begin{addmargin}[1em]{2em}
\setstretch{.5}
{\PaliGlossB{The Realized One understands this:}}\\
\end{addmargin}
\end{absolutelynopagebreak}

\begin{absolutelynopagebreak}
\setstretch{.7}
{\PaliGlossA{‘ime diṭṭhiṭṭhānā evaṃgahitā evaṃparāmaṭṭhā evaṃgatikā bhavanti evaṃabhisamparāyā’ti,}}\\
\begin{addmargin}[1em]{2em}
\setstretch{.5}
{\PaliGlossB{‘If you hold on to and attach to these grounds for views it leads to such and such a destiny in the next life.’}}\\
\end{addmargin}
\end{absolutelynopagebreak}

\begin{absolutelynopagebreak}
\setstretch{.7}
{\PaliGlossA{tañca tathāgato pajānāti, tato ca uttaritaraṃ pajānāti; tañca pajānanaṃ na parāmasati, aparāmasato cassa paccattaññeva nibbuti viditā.}}\\
\begin{addmargin}[1em]{2em}
\setstretch{.5}
{\PaliGlossB{He understands this, and what goes beyond this. Yet since he does not misapprehend that understanding, he has realized extinguishment within himself.}}\\
\end{addmargin}
\end{absolutelynopagebreak}

\begin{absolutelynopagebreak}
\setstretch{.7}
{\PaliGlossA{vedanānaṃ samudayañca atthaṅgamañca assādañca ādīnavañca nissaraṇañca yathābhūtaṃ viditvā anupādāvimutto, bhikkhave, tathāgato.}}\\
\begin{addmargin}[1em]{2em}
\setstretch{.5}
{\PaliGlossB{Having truly understood the origin, ending, gratification, drawback, and escape from feelings, the Realized One is freed through not grasping.}}\\
\end{addmargin}
\end{absolutelynopagebreak}

\begin{absolutelynopagebreak}
\setstretch{.7}
{\PaliGlossA{ime kho te, bhikkhave, dhammā gambhīrā duddasā duranubodhā santā paṇītā atakkāvacarā nipuṇā paṇḍitavedanīyā, ye tathāgato sayaṃ abhiññā sacchikatvā pavedeti, yehi tathāgatassa yathābhuccaṃ vaṇṇaṃ sammā vadamānā vadeyyuṃ.}}\\
\begin{addmargin}[1em]{2em}
\setstretch{.5}
{\PaliGlossB{These are the principles—deep, hard to see, hard to understand, peaceful, sublime, beyond the scope of reason, subtle, comprehensible to the astute—which the Realized One makes known after realizing them with his own insight. And those who genuinely praise the Realized One would rightly speak of these things.}}\\
\end{addmargin}
\end{absolutelynopagebreak}

\begin{absolutelynopagebreak}
\setstretch{.7}
{\PaliGlossA{paṭhamabhāṇavāro.}}\\
\begin{addmargin}[1em]{2em}
\setstretch{.5}
{\PaliGlossB{    -}}\\
\end{addmargin}
\end{absolutelynopagebreak}

\begin{absolutelynopagebreak}
\setstretch{.7}
{\PaliGlossA{3.1.2. ekaccasassatavāda}}\\
\begin{addmargin}[1em]{2em}
\setstretch{.5}
{\PaliGlossB{3.1.2. Partial Eternalism}}\\
\end{addmargin}
\end{absolutelynopagebreak}

\begin{absolutelynopagebreak}
\setstretch{.7}
{\PaliGlossA{santi, bhikkhave, eke samaṇabrāhmaṇā ekaccasassatikā ekaccaasassatikā ekaccaṃ sassataṃ ekaccaṃ asassataṃ attānañca lokañca paññapenti catūhi vatthūhi.}}\\
\begin{addmargin}[1em]{2em}
\setstretch{.5}
{\PaliGlossB{There are some ascetics and brahmins who are partial eternalists, who assert that the self and the cosmos are partially eternal and partially not eternal on four grounds.}}\\
\end{addmargin}
\end{absolutelynopagebreak}

\begin{absolutelynopagebreak}
\setstretch{.7}
{\PaliGlossA{te ca bhonto samaṇabrāhmaṇā kimāgamma kimārabbha ekaccasassatikā ekaccaasassatikā ekaccaṃ sassataṃ ekaccaṃ asassataṃ attānañca lokañca paññapenti catūhi vatthūhi?}}\\
\begin{addmargin}[1em]{2em}
\setstretch{.5}
{\PaliGlossB{And what are the four grounds on which they rely?}}\\
\end{addmargin}
\end{absolutelynopagebreak}

\begin{absolutelynopagebreak}
\setstretch{.7}
{\PaliGlossA{hoti kho so, bhikkhave, samayo, yaṃ kadāci karahaci dīghassa addhuno accayena ayaṃ loko saṃvaṭṭati.}}\\
\begin{addmargin}[1em]{2em}
\setstretch{.5}
{\PaliGlossB{There comes a time when, after a very long period has passed, this cosmos contracts.}}\\
\end{addmargin}
\end{absolutelynopagebreak}

\begin{absolutelynopagebreak}
\setstretch{.7}
{\PaliGlossA{saṃvaṭṭamāne loke yebhuyyena sattā ābhassarasaṃvattanikā honti.}}\\
\begin{addmargin}[1em]{2em}
\setstretch{.5}
{\PaliGlossB{As the cosmos contracts, sentient beings are mostly headed for the realm of streaming radiance.}}\\
\end{addmargin}
\end{absolutelynopagebreak}

\begin{absolutelynopagebreak}
\setstretch{.7}
{\PaliGlossA{te tattha honti manomayā pītibhakkhā sayaṃpabhā antalikkhacarā subhaṭṭhāyino, ciraṃ dīghamaddhānaṃ tiṭṭhanti.}}\\
\begin{addmargin}[1em]{2em}
\setstretch{.5}
{\PaliGlossB{There they are mind-made, feeding on rapture, self-luminous, moving through the sky, steadily glorious, and they remain like that for a very long time.}}\\
\end{addmargin}
\end{absolutelynopagebreak}

\begin{absolutelynopagebreak}
\setstretch{.7}
{\PaliGlossA{hoti kho so, bhikkhave, samayo, yaṃ kadāci karahaci dīghassa addhuno accayena ayaṃ loko vivaṭṭati.}}\\
\begin{addmargin}[1em]{2em}
\setstretch{.5}
{\PaliGlossB{There comes a time when, after a very long period has passed, this cosmos expands.}}\\
\end{addmargin}
\end{absolutelynopagebreak}

\begin{absolutelynopagebreak}
\setstretch{.7}
{\PaliGlossA{vivaṭṭamāne loke suññaṃ brahmavimānaṃ pātubhavati.}}\\
\begin{addmargin}[1em]{2em}
\setstretch{.5}
{\PaliGlossB{As it expands an empty mansion of Brahmā appears.}}\\
\end{addmargin}
\end{absolutelynopagebreak}

\begin{absolutelynopagebreak}
\setstretch{.7}
{\PaliGlossA{atha kho aññataro satto āyukkhayā vā puññakkhayā vā ābhassarakāyā cavitvā suññaṃ brahmavimānaṃ upapajjati.}}\\
\begin{addmargin}[1em]{2em}
\setstretch{.5}
{\PaliGlossB{Then a certain sentient being—due to the running out of their life-span or merit—passes away from that host of radiant deities and is reborn in that empty mansion of Brahmā.}}\\
\end{addmargin}
\end{absolutelynopagebreak}

\begin{absolutelynopagebreak}
\setstretch{.7}
{\PaliGlossA{so tattha hoti manomayo pītibhakkho sayampabho antalikkhacaro subhaṭṭhāyī, ciraṃ dīghamaddhānaṃ tiṭṭhati.}}\\
\begin{addmargin}[1em]{2em}
\setstretch{.5}
{\PaliGlossB{There they are mind-made, feeding on rapture, self-luminous, moving through the sky, steadily glorious, and they remain like that for a very long time.}}\\
\end{addmargin}
\end{absolutelynopagebreak}

\begin{absolutelynopagebreak}
\setstretch{.7}
{\PaliGlossA{tassa tattha ekakassa dīgharattaṃ nivusitattā anabhirati paritassanā uppajjati:}}\\
\begin{addmargin}[1em]{2em}
\setstretch{.5}
{\PaliGlossB{But after staying there all alone for a long time, they become dissatisfied and anxious:}}\\
\end{addmargin}
\end{absolutelynopagebreak}

\begin{absolutelynopagebreak}
\setstretch{.7}
{\PaliGlossA{‘aho vata aññepi sattā itthattaṃ āgaccheyyun’ti.}}\\
\begin{addmargin}[1em]{2em}
\setstretch{.5}
{\PaliGlossB{‘Oh, if only another being would come to this state of existence.’}}\\
\end{addmargin}
\end{absolutelynopagebreak}

\begin{absolutelynopagebreak}
\setstretch{.7}
{\PaliGlossA{atha aññepi sattā āyukkhayā vā puññakkhayā vā ābhassarakāyā cavitvā brahmavimānaṃ upapajjanti tassa sattassa sahabyataṃ.}}\\
\begin{addmargin}[1em]{2em}
\setstretch{.5}
{\PaliGlossB{Then other sentient beings—due to the running out of their life-span or merit—pass away from that host of radiant deities and are reborn in that empty mansion of Brahmā in company with that being.}}\\
\end{addmargin}
\end{absolutelynopagebreak}

\begin{absolutelynopagebreak}
\setstretch{.7}
{\PaliGlossA{tepi tattha honti manomayā pītibhakkhā sayaṃpabhā antalikkhacarā subhaṭṭhāyino, ciraṃ dīghamaddhānaṃ tiṭṭhanti.}}\\
\begin{addmargin}[1em]{2em}
\setstretch{.5}
{\PaliGlossB{There they too are mind-made, feeding on rapture, self-luminous, moving through the sky, steadily glorious, and they remain like that for a very long time.}}\\
\end{addmargin}
\end{absolutelynopagebreak}

\begin{absolutelynopagebreak}
\setstretch{.7}
{\PaliGlossA{tatra, bhikkhave, yo so satto paṭhamaṃ upapanno tassa evaṃ hoti:}}\\
\begin{addmargin}[1em]{2em}
\setstretch{.5}
{\PaliGlossB{Now, the being who was reborn there first thinks:}}\\
\end{addmargin}
\end{absolutelynopagebreak}

\begin{absolutelynopagebreak}
\setstretch{.7}
{\PaliGlossA{‘ahamasmi brahmā mahābrahmā abhibhū anabhibhūto aññadatthudaso vasavattī issaro kattā nimmātā seṭṭho sajitā vasī pitā bhūtabhabyānaṃ.}}\\
\begin{addmargin}[1em]{2em}
\setstretch{.5}
{\PaliGlossB{‘I am Brahmā, the Great Brahmā, the Undefeated, the Champion, the Universal Seer, the Wielder of Power, the Lord God, the Maker, the Author, the Best, the Begetter, the Controller, the Father of those who have been born and those yet to be born.}}\\
\end{addmargin}
\end{absolutelynopagebreak}

\begin{absolutelynopagebreak}
\setstretch{.7}
{\PaliGlossA{mayā ime sattā nimmitā.}}\\
\begin{addmargin}[1em]{2em}
\setstretch{.5}
{\PaliGlossB{These beings were created by me!}}\\
\end{addmargin}
\end{absolutelynopagebreak}

\begin{absolutelynopagebreak}
\setstretch{.7}
{\PaliGlossA{taṃ kissa hetu?}}\\
\begin{addmargin}[1em]{2em}
\setstretch{.5}
{\PaliGlossB{Why is that?}}\\
\end{addmargin}
\end{absolutelynopagebreak}

\begin{absolutelynopagebreak}
\setstretch{.7}
{\PaliGlossA{mamañhi pubbe etadahosi:}}\\
\begin{addmargin}[1em]{2em}
\setstretch{.5}
{\PaliGlossB{Because first I thought:}}\\
\end{addmargin}
\end{absolutelynopagebreak}

\begin{absolutelynopagebreak}
\setstretch{.7}
{\PaliGlossA{“aho vata aññepi sattā itthattaṃ āgaccheyyun”ti.}}\\
\begin{addmargin}[1em]{2em}
\setstretch{.5}
{\PaliGlossB{“Oh, if only another being would come to this state of existence.”}}\\
\end{addmargin}
\end{absolutelynopagebreak}

\begin{absolutelynopagebreak}
\setstretch{.7}
{\PaliGlossA{iti mama ca manopaṇidhi, ime ca sattā itthattaṃ āgatā’ti.}}\\
\begin{addmargin}[1em]{2em}
\setstretch{.5}
{\PaliGlossB{Such was my heart’s wish, and then these creatures came to this state of existence.’}}\\
\end{addmargin}
\end{absolutelynopagebreak}

\begin{absolutelynopagebreak}
\setstretch{.7}
{\PaliGlossA{yepi te sattā pacchā upapannā, tesampi evaṃ hoti:}}\\
\begin{addmargin}[1em]{2em}
\setstretch{.5}
{\PaliGlossB{And the beings who were reborn there later also think:}}\\
\end{addmargin}
\end{absolutelynopagebreak}

\begin{absolutelynopagebreak}
\setstretch{.7}
{\PaliGlossA{‘ayaṃ kho bhavaṃ brahmā mahābrahmā abhibhū anabhibhūto aññadatthudaso vasavattī issaro kattā nimmātā seṭṭho sajitā vasī pitā bhūtabhabyānaṃ.}}\\
\begin{addmargin}[1em]{2em}
\setstretch{.5}
{\PaliGlossB{‘This must be Brahmā, the Great Brahmā, the Undefeated, the Champion, the Universal Seer, the Wielder of Power, the Lord God, the Maker, the Author, the Best, the Begetter, the Controller, the Father of those who have been born and those yet to be born.}}\\
\end{addmargin}
\end{absolutelynopagebreak}

\begin{absolutelynopagebreak}
\setstretch{.7}
{\PaliGlossA{iminā mayaṃ bhotā brahmunā nimmitā.}}\\
\begin{addmargin}[1em]{2em}
\setstretch{.5}
{\PaliGlossB{And we have been created by him.}}\\
\end{addmargin}
\end{absolutelynopagebreak}

\begin{absolutelynopagebreak}
\setstretch{.7}
{\PaliGlossA{taṃ kissa hetu?}}\\
\begin{addmargin}[1em]{2em}
\setstretch{.5}
{\PaliGlossB{Why is that?}}\\
\end{addmargin}
\end{absolutelynopagebreak}

\begin{absolutelynopagebreak}
\setstretch{.7}
{\PaliGlossA{imañhi mayaṃ addasāma idha paṭhamaṃ upapannaṃ, mayaṃ panamha pacchā upapannā’ti.}}\\
\begin{addmargin}[1em]{2em}
\setstretch{.5}
{\PaliGlossB{Because we see that he was reborn here first, and we arrived later.’}}\\
\end{addmargin}
\end{absolutelynopagebreak}

\begin{absolutelynopagebreak}
\setstretch{.7}
{\PaliGlossA{tatra, bhikkhave, yo so satto paṭhamaṃ upapanno, so dīghāyukataro ca hoti vaṇṇavantataro ca mahesakkhataro ca.}}\\
\begin{addmargin}[1em]{2em}
\setstretch{.5}
{\PaliGlossB{And the being who was reborn first is more long-lived, beautiful, and illustrious than those who arrived later.}}\\
\end{addmargin}
\end{absolutelynopagebreak}

\begin{absolutelynopagebreak}
\setstretch{.7}
{\PaliGlossA{ye pana te sattā pacchā upapannā, te appāyukatarā ca honti dubbaṇṇatarā ca appesakkhatarā ca.}}\\
\begin{addmargin}[1em]{2em}
\setstretch{.5}
{\PaliGlossB{    -}}\\
\end{addmargin}
\end{absolutelynopagebreak}

\begin{absolutelynopagebreak}
\setstretch{.7}
{\PaliGlossA{ṭhānaṃ kho panetaṃ, bhikkhave, vijjati, yaṃ aññataro satto tamhā kāyā cavitvā itthattaṃ āgacchati.}}\\
\begin{addmargin}[1em]{2em}
\setstretch{.5}
{\PaliGlossB{It’s possible that one of those beings passes away from that host and is reborn in this state of existence.}}\\
\end{addmargin}
\end{absolutelynopagebreak}

\begin{absolutelynopagebreak}
\setstretch{.7}
{\PaliGlossA{itthattaṃ āgato samāno agārasmā anagāriyaṃ pabbajati.}}\\
\begin{addmargin}[1em]{2em}
\setstretch{.5}
{\PaliGlossB{Having done so, they go forth from the lay life to homelessness.}}\\
\end{addmargin}
\end{absolutelynopagebreak}

\begin{absolutelynopagebreak}
\setstretch{.7}
{\PaliGlossA{agārasmā anagāriyaṃ pabbajito samāno ātappamanvāya padhānamanvāya anuyogamanvāya appamādamanvāya sammāmanasikāramanvāya tathārūpaṃ cetosamādhiṃ phusati, yathāsamāhite citte taṃ pubbenivāsaṃ anussarati, tato paraṃ nānussarati.}}\\
\begin{addmargin}[1em]{2em}
\setstretch{.5}
{\PaliGlossB{By dint of keen, resolute, committed, and diligent effort, and right focus, they experience an immersion of the heart of such a kind that they recollect that past life, but no further.}}\\
\end{addmargin}
\end{absolutelynopagebreak}

\begin{absolutelynopagebreak}
\setstretch{.7}
{\PaliGlossA{so evamāha:}}\\
\begin{addmargin}[1em]{2em}
\setstretch{.5}
{\PaliGlossB{They say:}}\\
\end{addmargin}
\end{absolutelynopagebreak}

\begin{absolutelynopagebreak}
\setstretch{.7}
{\PaliGlossA{‘yo kho so bhavaṃ brahmā mahābrahmā abhibhū anabhibhūto aññadatthudaso vasavattī issaro kattā nimmātā seṭṭho sajitā vasī pitā bhūtabhabyānaṃ, yena mayaṃ bhotā brahmunā nimmitā, so nicco dhuvo sassato avipariṇāmadhammo sassatisamaṃ tatheva ṭhassati.}}\\
\begin{addmargin}[1em]{2em}
\setstretch{.5}
{\PaliGlossB{‘He who is Brahmā—the Great Brahmā, the Undefeated, the Champion, the Universal Seer, the Wielder of Power, the Lord God, the Maker, the Author, the Best, the Begetter, the Controller, the Father of those who have been born and those yet to be born—is permanent, everlasting, eternal, imperishable, remaining the same for all eternity.}}\\
\end{addmargin}
\end{absolutelynopagebreak}

\begin{absolutelynopagebreak}
\setstretch{.7}
{\PaliGlossA{ye pana mayaṃ ahumhā tena bhotā brahmunā nimmitā, te mayaṃ aniccā addhuvā appāyukā cavanadhammā itthattaṃ āgatā’ti.}}\\
\begin{addmargin}[1em]{2em}
\setstretch{.5}
{\PaliGlossB{We who were created by that Brahmā are impermanent, not lasting, short-lived, perishable, and have come to this state of existence.}}\\
\end{addmargin}
\end{absolutelynopagebreak}

\begin{absolutelynopagebreak}
\setstretch{.7}
{\PaliGlossA{idaṃ, bhikkhave, paṭhamaṃ ṭhānaṃ, yaṃ āgamma yaṃ ārabbha eke samaṇabrāhmaṇā ekaccasassatikā ekaccaasassatikā ekaccaṃ sassataṃ ekaccaṃ asassataṃ attānañca lokañca paññapenti. (1: 5)}}\\
\begin{addmargin}[1em]{2em}
\setstretch{.5}
{\PaliGlossB{This is the first ground on which some ascetics and brahmins rely to assert that the self and the cosmos are partially eternal.}}\\
\end{addmargin}
\end{absolutelynopagebreak}

\begin{absolutelynopagebreak}
\setstretch{.7}
{\PaliGlossA{dutiye ca bhonto samaṇabrāhmaṇā kimāgamma kimārabbha ekaccasassatikā ekaccaasassatikā ekaccaṃ sassataṃ ekaccaṃ asassataṃ attānañca lokañca paññapenti?}}\\
\begin{addmargin}[1em]{2em}
\setstretch{.5}
{\PaliGlossB{And what is the second ground on which they rely?}}\\
\end{addmargin}
\end{absolutelynopagebreak}

\begin{absolutelynopagebreak}
\setstretch{.7}
{\PaliGlossA{santi, bhikkhave, khiḍḍāpadosikā nāma devā, te ativelaṃ hassakhiḍḍāratidhammasamāpannā viharanti. tesaṃ ativelaṃ hassakhiḍḍāratidhammasamāpannānaṃ viharataṃ sati sammussati. satiyā sammosā te devā tamhā kāyā cavanti.}}\\
\begin{addmargin}[1em]{2em}
\setstretch{.5}
{\PaliGlossB{There are gods named ‘depraved by play.’ They spend too much time laughing, playing, and making merry. And in doing so, they lose their mindfulness, and they pass away from that host of gods.}}\\
\end{addmargin}
\end{absolutelynopagebreak}

\begin{absolutelynopagebreak}
\setstretch{.7}
{\PaliGlossA{ṭhānaṃ kho panetaṃ, bhikkhave, vijjati yaṃ aññataro satto tamhā kāyā cavitvā itthattaṃ āgacchati.}}\\
\begin{addmargin}[1em]{2em}
\setstretch{.5}
{\PaliGlossB{It’s possible that one of those beings passes away from that host and is reborn in this state of existence.}}\\
\end{addmargin}
\end{absolutelynopagebreak}

\begin{absolutelynopagebreak}
\setstretch{.7}
{\PaliGlossA{itthattaṃ āgato samāno agārasmā anagāriyaṃ pabbajati.}}\\
\begin{addmargin}[1em]{2em}
\setstretch{.5}
{\PaliGlossB{Having done so, they go forth from the lay life to homelessness.}}\\
\end{addmargin}
\end{absolutelynopagebreak}

\begin{absolutelynopagebreak}
\setstretch{.7}
{\PaliGlossA{agārasmā anagāriyaṃ pabbajito samāno ātappamanvāya padhānamanvāya anuyogamanvāya appamādamanvāya sammāmanasikāramanvāya tathārūpaṃ cetosamādhiṃ phusati, yathāsamāhite citte taṃ pubbenivāsaṃ anussarati, tato paraṃ nānussarati.}}\\
\begin{addmargin}[1em]{2em}
\setstretch{.5}
{\PaliGlossB{By dint of keen, resolute, committed, and diligent effort, and right focus, they experience an immersion of the heart of such a kind that they recollect that past life, but no further.}}\\
\end{addmargin}
\end{absolutelynopagebreak}

\begin{absolutelynopagebreak}
\setstretch{.7}
{\PaliGlossA{so evamāha:}}\\
\begin{addmargin}[1em]{2em}
\setstretch{.5}
{\PaliGlossB{They say:}}\\
\end{addmargin}
\end{absolutelynopagebreak}

\begin{absolutelynopagebreak}
\setstretch{.7}
{\PaliGlossA{‘ye kho te bhonto devā na khiḍḍāpadosikā, te na ativelaṃ hassakhiḍḍāratidhammasamāpannā viharanti. tesaṃ na ativelaṃ hassakhiḍḍāratidhammasamāpannānaṃ viharataṃ sati na sammussati. satiyā asammosā te devā tamhā kāyā na cavanti;}}\\
\begin{addmargin}[1em]{2em}
\setstretch{.5}
{\PaliGlossB{‘The gods not depraved by play don’t spend too much time laughing, playing, and making merry. So they don’t lose their mindfulness, and don’t pass away from that host of gods.}}\\
\end{addmargin}
\end{absolutelynopagebreak}

\begin{absolutelynopagebreak}
\setstretch{.7}
{\PaliGlossA{niccā dhuvā sassatā avipariṇāmadhammā sassatisamaṃ tatheva ṭhassanti.}}\\
\begin{addmargin}[1em]{2em}
\setstretch{.5}
{\PaliGlossB{They are permanent, everlasting, eternal, imperishable, remaining the same for all eternity.}}\\
\end{addmargin}
\end{absolutelynopagebreak}

\begin{absolutelynopagebreak}
\setstretch{.7}
{\PaliGlossA{ye pana mayaṃ ahumhā khiḍḍāpadosikā, te mayaṃ ativelaṃ hassakhiḍḍāratidhammasamāpannā viharimhā. tesaṃ no ativelaṃ hassakhiḍḍāratidhammasamāpannānaṃ viharataṃ sati sammussati. satiyā sammosā evaṃ mayaṃ tamhā kāyā cutā}}\\
\begin{addmargin}[1em]{2em}
\setstretch{.5}
{\PaliGlossB{But we who were depraved by play spent too much time laughing, playing, and making merry. In doing so, we lost our mindfulness, and passed away from that host of gods.}}\\
\end{addmargin}
\end{absolutelynopagebreak}

\begin{absolutelynopagebreak}
\setstretch{.7}
{\PaliGlossA{aniccā addhuvā appāyukā cavanadhammā itthattaṃ āgatā’ti.}}\\
\begin{addmargin}[1em]{2em}
\setstretch{.5}
{\PaliGlossB{We are impermanent, not lasting, short-lived, perishable, and have come to this state of existence.’}}\\
\end{addmargin}
\end{absolutelynopagebreak}

\begin{absolutelynopagebreak}
\setstretch{.7}
{\PaliGlossA{idaṃ, bhikkhave, dutiyaṃ ṭhānaṃ, yaṃ āgamma yaṃ ārabbha eke samaṇabrāhmaṇā ekaccasassatikā ekaccaasassatikā ekaccaṃ sassataṃ ekaccaṃ asassataṃ attānañca lokañca paññapenti. (2: 6)}}\\
\begin{addmargin}[1em]{2em}
\setstretch{.5}
{\PaliGlossB{This is the second ground on which some ascetics and brahmins rely to assert that the self and the cosmos are partially eternal.}}\\
\end{addmargin}
\end{absolutelynopagebreak}

\begin{absolutelynopagebreak}
\setstretch{.7}
{\PaliGlossA{tatiye ca bhonto samaṇabrāhmaṇā kimāgamma kimārabbha ekaccasassatikā ekaccaasassatikā ekaccaṃ sassataṃ ekaccaṃ asassataṃ attānañca lokañca paññapenti?}}\\
\begin{addmargin}[1em]{2em}
\setstretch{.5}
{\PaliGlossB{And what is the third ground on which they rely?}}\\
\end{addmargin}
\end{absolutelynopagebreak}

\begin{absolutelynopagebreak}
\setstretch{.7}
{\PaliGlossA{santi, bhikkhave, manopadosikā nāma devā, te ativelaṃ aññamaññaṃ upanijjhāyanti. te ativelaṃ aññamaññaṃ upanijjhāyantā aññamaññamhi cittāni padūsenti. te aññamaññaṃ paduṭṭhacittā kilantakāyā kilantacittā. te devā tamhā kāyā cavanti.}}\\
\begin{addmargin}[1em]{2em}
\setstretch{.5}
{\PaliGlossB{There are gods named ‘malevolent’. They spend too much time gazing at each other, so they grow angry with each other, and their bodies and minds get tired. They pass away from that host of gods.}}\\
\end{addmargin}
\end{absolutelynopagebreak}

\begin{absolutelynopagebreak}
\setstretch{.7}
{\PaliGlossA{ṭhānaṃ kho panetaṃ, bhikkhave, vijjati yaṃ aññataro satto tamhā kāyā cavitvā itthattaṃ āgacchati.}}\\
\begin{addmargin}[1em]{2em}
\setstretch{.5}
{\PaliGlossB{It’s possible that one of those beings passes away from that host and is reborn in this state of existence.}}\\
\end{addmargin}
\end{absolutelynopagebreak}

\begin{absolutelynopagebreak}
\setstretch{.7}
{\PaliGlossA{itthattaṃ āgato samāno agārasmā anagāriyaṃ pabbajati.}}\\
\begin{addmargin}[1em]{2em}
\setstretch{.5}
{\PaliGlossB{Having done so, they go forth from the lay life to homelessness.}}\\
\end{addmargin}
\end{absolutelynopagebreak}

\begin{absolutelynopagebreak}
\setstretch{.7}
{\PaliGlossA{agārasmā anagāriyaṃ pabbajito samāno ātappamanvāya padhānamanvāya anuyogamanvāya appamādamanvāya sammāmanasikāramanvāya tathārūpaṃ cetosamādhiṃ phusati, yathāsamāhite citte taṃ pubbenivāsaṃ anussarati, tato paraṃ nānussarati.}}\\
\begin{addmargin}[1em]{2em}
\setstretch{.5}
{\PaliGlossB{By dint of keen, resolute, committed, and diligent effort, and right focus, they experience an immersion of the heart of such a kind that they recollect that past life, but no further.}}\\
\end{addmargin}
\end{absolutelynopagebreak}

\begin{absolutelynopagebreak}
\setstretch{.7}
{\PaliGlossA{so evamāha:}}\\
\begin{addmargin}[1em]{2em}
\setstretch{.5}
{\PaliGlossB{They say:}}\\
\end{addmargin}
\end{absolutelynopagebreak}

\begin{absolutelynopagebreak}
\setstretch{.7}
{\PaliGlossA{‘ye kho te bhonto devā na manopadosikā, te nātivelaṃ aññamaññaṃ upanijjhāyanti. te nātivelaṃ aññamaññaṃ upanijjhāyantā aññamaññamhi cittāni nappadūsenti. te aññamaññaṃ appaduṭṭhacittā akilantakāyā akilantacittā. te devā tamhā kāyā na cavanti,}}\\
\begin{addmargin}[1em]{2em}
\setstretch{.5}
{\PaliGlossB{‘The gods who are not malevolent don’t spend too much time gazing at each other, so they don’t grow angry with each other, their bodies and minds don’t get tired, and they don’t pass away from that host of gods.}}\\
\end{addmargin}
\end{absolutelynopagebreak}

\begin{absolutelynopagebreak}
\setstretch{.7}
{\PaliGlossA{niccā dhuvā sassatā avipariṇāmadhammā sassatisamaṃ tatheva ṭhassanti.}}\\
\begin{addmargin}[1em]{2em}
\setstretch{.5}
{\PaliGlossB{They are permanent, everlasting, eternal, imperishable, remaining the same for all eternity.}}\\
\end{addmargin}
\end{absolutelynopagebreak}

\begin{absolutelynopagebreak}
\setstretch{.7}
{\PaliGlossA{ye pana mayaṃ ahumhā manopadosikā, te mayaṃ ativelaṃ aññamaññaṃ upanijjhāyimhā. te mayaṃ ativelaṃ aññamaññaṃ upanijjhāyantā aññamaññamhi cittāni padūsimhā, te mayaṃ aññamaññaṃ paduṭṭhacittā kilantakāyā kilantacittā. evaṃ mayaṃ tamhā kāyā cutā}}\\
\begin{addmargin}[1em]{2em}
\setstretch{.5}
{\PaliGlossB{But we who were malevolent spent too much time gazing at each other, we grew angry with each other, our bodies and minds got tired, and we passed away from that host of gods.}}\\
\end{addmargin}
\end{absolutelynopagebreak}

\begin{absolutelynopagebreak}
\setstretch{.7}
{\PaliGlossA{aniccā addhuvā appāyukā cavanadhammā itthattaṃ āgatā’ti.}}\\
\begin{addmargin}[1em]{2em}
\setstretch{.5}
{\PaliGlossB{We are impermanent, not lasting, short-lived, perishable, and have come to this state of existence.’}}\\
\end{addmargin}
\end{absolutelynopagebreak}

\begin{absolutelynopagebreak}
\setstretch{.7}
{\PaliGlossA{idaṃ, bhikkhave, tatiyaṃ ṭhānaṃ, yaṃ āgamma yaṃ ārabbha eke samaṇabrāhmaṇā ekaccasassatikā ekaccaasassatikā ekaccaṃ sassataṃ ekaccaṃ asassataṃ attānañca lokañca paññapenti. (3: 7)}}\\
\begin{addmargin}[1em]{2em}
\setstretch{.5}
{\PaliGlossB{This is the third ground on which some ascetics and brahmins rely to assert that the self and the cosmos are partially eternal.}}\\
\end{addmargin}
\end{absolutelynopagebreak}

\begin{absolutelynopagebreak}
\setstretch{.7}
{\PaliGlossA{catutthe ca bhonto samaṇabrāhmaṇā kimāgamma kimārabbha ekaccasassatikā ekaccaasassatikā ekaccaṃ sassataṃ ekaccaṃ asassataṃ attānañca lokañca paññapenti?}}\\
\begin{addmargin}[1em]{2em}
\setstretch{.5}
{\PaliGlossB{And what is the fourth ground on which they rely?}}\\
\end{addmargin}
\end{absolutelynopagebreak}

\begin{absolutelynopagebreak}
\setstretch{.7}
{\PaliGlossA{idha, bhikkhave, ekacco samaṇo vā brāhmaṇo vā takkī hoti vīmaṃsī. so takkapariyāhataṃ vīmaṃsānucaritaṃ sayampaṭibhānaṃ evamāha:}}\\
\begin{addmargin}[1em]{2em}
\setstretch{.5}
{\PaliGlossB{It’s when some ascetic or brahmin relies on logic and inquiry. They speak of what they have worked out by logic, following a line of inquiry, expressing their own perspective:}}\\
\end{addmargin}
\end{absolutelynopagebreak}

\begin{absolutelynopagebreak}
\setstretch{.7}
{\PaliGlossA{‘yaṃ kho idaṃ vuccati cakkhuṃ itipi sotaṃ itipi ghānaṃ itipi jivhā itipi kāyo itipi, ayaṃ attā anicco addhuvo asassato vipariṇāmadhammo.}}\\
\begin{addmargin}[1em]{2em}
\setstretch{.5}
{\PaliGlossB{‘That which is called “the eye” or “the ear” or “the nose” or “the tongue” or “the body”: that self is impermanent, not lasting, transient, perishable.}}\\
\end{addmargin}
\end{absolutelynopagebreak}

\begin{absolutelynopagebreak}
\setstretch{.7}
{\PaliGlossA{yañca kho idaṃ vuccati cittanti vā manoti vā viññāṇanti vā ayaṃ attā nicco dhuvo sassato avipariṇāmadhammo sassatisamaṃ tatheva ṭhassatī’ti.}}\\
\begin{addmargin}[1em]{2em}
\setstretch{.5}
{\PaliGlossB{That which is called “mind” or “sentience” or “consciousness”: that self is permanent, everlasting, eternal, imperishable, remaining the same for all eternity.’}}\\
\end{addmargin}
\end{absolutelynopagebreak}

\begin{absolutelynopagebreak}
\setstretch{.7}
{\PaliGlossA{idaṃ, bhikkhave, catutthaṃ ṭhānaṃ, yaṃ āgamma yaṃ ārabbha eke samaṇabrāhmaṇā ekaccasassatikā ekaccaasassatikā ekaccaṃ sassataṃ ekaccaṃ asassataṃ attānañca lokañca paññapenti. (4: 8)}}\\
\begin{addmargin}[1em]{2em}
\setstretch{.5}
{\PaliGlossB{This is the fourth ground on which some ascetics and brahmins rely to assert that the self and the cosmos are partially eternal.}}\\
\end{addmargin}
\end{absolutelynopagebreak}

\begin{absolutelynopagebreak}
\setstretch{.7}
{\PaliGlossA{imehi kho te, bhikkhave, samaṇabrāhmaṇā ekaccasassatikā ekaccaasassatikā ekaccaṃ sassataṃ ekaccaṃ asassataṃ attānañca lokañca paññapenti catūhi vatthūhi.}}\\
\begin{addmargin}[1em]{2em}
\setstretch{.5}
{\PaliGlossB{These are the four grounds on which those ascetics and brahmins assert that the self and the cosmos are partially eternal and partially not eternal.}}\\
\end{addmargin}
\end{absolutelynopagebreak}

\begin{absolutelynopagebreak}
\setstretch{.7}
{\PaliGlossA{ye hi keci, bhikkhave, samaṇā vā brāhmaṇā vā ekaccasassatikā ekaccaasassatikā ekaccaṃ sassataṃ ekaccaṃ asassataṃ attānañca lokañca paññapenti, sabbe te imeheva catūhi vatthūhi, etesaṃ vā aññatarena; natthi ito bahiddhā.}}\\
\begin{addmargin}[1em]{2em}
\setstretch{.5}
{\PaliGlossB{Any ascetics and brahmins who assert that the self and the cosmos are partially eternal and partially not eternal do so on one or other of these four grounds. Outside of this there is none.}}\\
\end{addmargin}
\end{absolutelynopagebreak}

\begin{absolutelynopagebreak}
\setstretch{.7}
{\PaliGlossA{tayidaṃ, bhikkhave, tathāgato pajānāti:}}\\
\begin{addmargin}[1em]{2em}
\setstretch{.5}
{\PaliGlossB{The Realized One understands this:}}\\
\end{addmargin}
\end{absolutelynopagebreak}

\begin{absolutelynopagebreak}
\setstretch{.7}
{\PaliGlossA{‘ime diṭṭhiṭṭhānā evaṅgahitā evaṃparāmaṭṭhā evaṅgatikā bhavanti evaṃabhisamparāyā’ti.}}\\
\begin{addmargin}[1em]{2em}
\setstretch{.5}
{\PaliGlossB{‘If you hold on to and attach to these grounds for views it leads to such and such a destiny in the next life.’}}\\
\end{addmargin}
\end{absolutelynopagebreak}

\begin{absolutelynopagebreak}
\setstretch{.7}
{\PaliGlossA{tañca tathāgato pajānāti, tato ca uttaritaraṃ pajānāti, tañca pajānanaṃ na parāmasati, aparāmasato cassa paccattaññeva nibbuti viditā.}}\\
\begin{addmargin}[1em]{2em}
\setstretch{.5}
{\PaliGlossB{He understands this, and what goes beyond this. Yet since he does not misapprehend that understanding, he has realized extinguishment within himself.}}\\
\end{addmargin}
\end{absolutelynopagebreak}

\begin{absolutelynopagebreak}
\setstretch{.7}
{\PaliGlossA{vedanānaṃ samudayañca atthaṅgamañca assādañca ādīnavañca nissaraṇañca yathābhūtaṃ viditvā anupādāvimutto, bhikkhave, tathāgato.}}\\
\begin{addmargin}[1em]{2em}
\setstretch{.5}
{\PaliGlossB{Having truly understood the origin, ending, gratification, drawback, and escape from feelings, the Realized One is freed through not grasping.}}\\
\end{addmargin}
\end{absolutelynopagebreak}

\begin{absolutelynopagebreak}
\setstretch{.7}
{\PaliGlossA{ime kho te, bhikkhave, dhammā gambhīrā duddasā duranubodhā santā paṇītā atakkāvacarā nipuṇā paṇḍitavedanīyā, ye tathāgato sayaṃ abhiññā sacchikatvā pavedeti, yehi tathāgatassa yathābhuccaṃ vaṇṇaṃ sammā vadamānā vadeyyuṃ.}}\\
\begin{addmargin}[1em]{2em}
\setstretch{.5}
{\PaliGlossB{These are the principles—deep, hard to see, hard to understand, peaceful, sublime, beyond the scope of reason, subtle, comprehensible to the astute—which the Realized One makes known after realizing them with his own insight. And those who genuinely praise the Realized One would rightly speak of these things.}}\\
\end{addmargin}
\end{absolutelynopagebreak}

\begin{absolutelynopagebreak}
\setstretch{.7}
{\PaliGlossA{3.1.3. antānantavāda}}\\
\begin{addmargin}[1em]{2em}
\setstretch{.5}
{\PaliGlossB{3.1.3. The Cosmos is Finite or Infinite}}\\
\end{addmargin}
\end{absolutelynopagebreak}

\begin{absolutelynopagebreak}
\setstretch{.7}
{\PaliGlossA{santi, bhikkhave, eke samaṇabrāhmaṇā antānantikā antānantaṃ lokassa paññapenti catūhi vatthūhi.}}\\
\begin{addmargin}[1em]{2em}
\setstretch{.5}
{\PaliGlossB{There are some ascetics and brahmins who theorize about size, and assert that the cosmos is finite or infinite on four grounds.}}\\
\end{addmargin}
\end{absolutelynopagebreak}

\begin{absolutelynopagebreak}
\setstretch{.7}
{\PaliGlossA{te ca bhonto samaṇabrāhmaṇā kimāgamma kimārabbha antānantikā antānantaṃ lokassa paññapenti catūhi vatthūhi?}}\\
\begin{addmargin}[1em]{2em}
\setstretch{.5}
{\PaliGlossB{And what are the four grounds on which they rely?}}\\
\end{addmargin}
\end{absolutelynopagebreak}

\begin{absolutelynopagebreak}
\setstretch{.7}
{\PaliGlossA{idha, bhikkhave, ekacco samaṇo vā brāhmaṇo vā ātappamanvāya padhānamanvāya anuyogamanvāya appamādamanvāya sammāmanasikāramanvāya tathārūpaṃ cetosamādhiṃ phusati, yathāsamāhite citte antasaññī lokasmiṃ viharati.}}\\
\begin{addmargin}[1em]{2em}
\setstretch{.5}
{\PaliGlossB{It’s when some ascetic or brahmin—by dint of keen, resolute, committed, and diligent effort, and right focus—experiences an immersion of the heart of such a kind that they meditate perceiving the cosmos as finite.}}\\
\end{addmargin}
\end{absolutelynopagebreak}

\begin{absolutelynopagebreak}
\setstretch{.7}
{\PaliGlossA{so evamāha:}}\\
\begin{addmargin}[1em]{2em}
\setstretch{.5}
{\PaliGlossB{They say:}}\\
\end{addmargin}
\end{absolutelynopagebreak}

\begin{absolutelynopagebreak}
\setstretch{.7}
{\PaliGlossA{‘antavā ayaṃ loko parivaṭumo.}}\\
\begin{addmargin}[1em]{2em}
\setstretch{.5}
{\PaliGlossB{‘The cosmos is finite and bounded.}}\\
\end{addmargin}
\end{absolutelynopagebreak}

\begin{absolutelynopagebreak}
\setstretch{.7}
{\PaliGlossA{taṃ kissa hetu?}}\\
\begin{addmargin}[1em]{2em}
\setstretch{.5}
{\PaliGlossB{Why is that?}}\\
\end{addmargin}
\end{absolutelynopagebreak}

\begin{absolutelynopagebreak}
\setstretch{.7}
{\PaliGlossA{ahañhi ātappamanvāya padhānamanvāya anuyogamanvāya appamādamanvāya sammāmanasikāramanvāya tathārūpaṃ cetosamādhiṃ phusāmi, yathāsamāhite citte antasaññī lokasmiṃ viharāmi.}}\\
\begin{addmargin}[1em]{2em}
\setstretch{.5}
{\PaliGlossB{Because by dint of keen, resolute, committed, and diligent effort, and right focus I experience an immersion of the heart of such a kind that I meditate perceiving the cosmos as finite.}}\\
\end{addmargin}
\end{absolutelynopagebreak}

\begin{absolutelynopagebreak}
\setstretch{.7}
{\PaliGlossA{imināmahaṃ etaṃ jānāmi:}}\\
\begin{addmargin}[1em]{2em}
\setstretch{.5}
{\PaliGlossB{Because of this I know:}}\\
\end{addmargin}
\end{absolutelynopagebreak}

\begin{absolutelynopagebreak}
\setstretch{.7}
{\PaliGlossA{“yathā antavā ayaṃ loko parivaṭumo”’ti.}}\\
\begin{addmargin}[1em]{2em}
\setstretch{.5}
{\PaliGlossB{“The cosmos is finite and bounded.”’}}\\
\end{addmargin}
\end{absolutelynopagebreak}

\begin{absolutelynopagebreak}
\setstretch{.7}
{\PaliGlossA{idaṃ, bhikkhave, paṭhamaṃ ṭhānaṃ, yaṃ āgamma yaṃ ārabbha eke samaṇabrāhmaṇā antānantikā antānantaṃ lokassa paññapenti. (1: 9)}}\\
\begin{addmargin}[1em]{2em}
\setstretch{.5}
{\PaliGlossB{This is the first ground on which some ascetics and brahmins rely to assert that the cosmos is finite or infinite.}}\\
\end{addmargin}
\end{absolutelynopagebreak}

\begin{absolutelynopagebreak}
\setstretch{.7}
{\PaliGlossA{dutiye ca bhonto samaṇabrāhmaṇā kimāgamma kimārabbha antānantikā antānantaṃ lokassa paññapenti?}}\\
\begin{addmargin}[1em]{2em}
\setstretch{.5}
{\PaliGlossB{And what is the second ground on which they rely?}}\\
\end{addmargin}
\end{absolutelynopagebreak}

\begin{absolutelynopagebreak}
\setstretch{.7}
{\PaliGlossA{idha, bhikkhave, ekacco samaṇo vā brāhmaṇo vā ātappamanvāya padhānamanvāya anuyogamanvāya appamādamanvāya sammāmanasikāramanvāya tathārūpaṃ cetosamādhiṃ phusati, yathāsamāhite citte anantasaññī lokasmiṃ viharati.}}\\
\begin{addmargin}[1em]{2em}
\setstretch{.5}
{\PaliGlossB{It’s when some ascetic or brahmin—by dint of keen, resolute, committed, and diligent effort, and right focus—experiences an immersion of the heart of such a kind that they meditate perceiving the cosmos as infinite.}}\\
\end{addmargin}
\end{absolutelynopagebreak}

\begin{absolutelynopagebreak}
\setstretch{.7}
{\PaliGlossA{so evamāha:}}\\
\begin{addmargin}[1em]{2em}
\setstretch{.5}
{\PaliGlossB{They say:}}\\
\end{addmargin}
\end{absolutelynopagebreak}

\begin{absolutelynopagebreak}
\setstretch{.7}
{\PaliGlossA{‘ananto ayaṃ loko apariyanto.}}\\
\begin{addmargin}[1em]{2em}
\setstretch{.5}
{\PaliGlossB{‘The cosmos is infinite and unbounded.}}\\
\end{addmargin}
\end{absolutelynopagebreak}

\begin{absolutelynopagebreak}
\setstretch{.7}
{\PaliGlossA{ye te samaṇabrāhmaṇā evamāhaṃsu:}}\\
\begin{addmargin}[1em]{2em}
\setstretch{.5}
{\PaliGlossB{The ascetics and brahmins who say that}}\\
\end{addmargin}
\end{absolutelynopagebreak}

\begin{absolutelynopagebreak}
\setstretch{.7}
{\PaliGlossA{“antavā ayaṃ loko parivaṭumo”ti, tesaṃ musā.}}\\
\begin{addmargin}[1em]{2em}
\setstretch{.5}
{\PaliGlossB{the cosmos is finite are wrong.}}\\
\end{addmargin}
\end{absolutelynopagebreak}

\begin{absolutelynopagebreak}
\setstretch{.7}
{\PaliGlossA{ananto ayaṃ loko apariyanto.}}\\
\begin{addmargin}[1em]{2em}
\setstretch{.5}
{\PaliGlossB{The cosmos is infinite and unbounded.}}\\
\end{addmargin}
\end{absolutelynopagebreak}

\begin{absolutelynopagebreak}
\setstretch{.7}
{\PaliGlossA{taṃ kissa hetu?}}\\
\begin{addmargin}[1em]{2em}
\setstretch{.5}
{\PaliGlossB{Why is that?}}\\
\end{addmargin}
\end{absolutelynopagebreak}

\begin{absolutelynopagebreak}
\setstretch{.7}
{\PaliGlossA{ahañhi ātappamanvāya padhānamanvāya anuyogamanvāya appamādamanvāya sammāmanasikāramanvāya tathārūpaṃ cetosamādhiṃ phusāmi, yathāsamāhite citte anantasaññī lokasmiṃ viharāmi.}}\\
\begin{addmargin}[1em]{2em}
\setstretch{.5}
{\PaliGlossB{Because by dint of keen, resolute, committed, and diligent effort, and right focus I experience an immersion of the heart of such a kind that I meditate perceiving the cosmos as infinite.}}\\
\end{addmargin}
\end{absolutelynopagebreak}

\begin{absolutelynopagebreak}
\setstretch{.7}
{\PaliGlossA{imināmahaṃ etaṃ jānāmi:}}\\
\begin{addmargin}[1em]{2em}
\setstretch{.5}
{\PaliGlossB{Because of this I know:}}\\
\end{addmargin}
\end{absolutelynopagebreak}

\begin{absolutelynopagebreak}
\setstretch{.7}
{\PaliGlossA{“yathā ananto ayaṃ loko apariyanto”’ti.}}\\
\begin{addmargin}[1em]{2em}
\setstretch{.5}
{\PaliGlossB{“The cosmos is infinite and unbounded.”’}}\\
\end{addmargin}
\end{absolutelynopagebreak}

\begin{absolutelynopagebreak}
\setstretch{.7}
{\PaliGlossA{idaṃ, bhikkhave, dutiyaṃ ṭhānaṃ, yaṃ āgamma yaṃ ārabbha eke samaṇabrāhmaṇā antānantikā antānantaṃ lokassa paññapenti. (2: 10)}}\\
\begin{addmargin}[1em]{2em}
\setstretch{.5}
{\PaliGlossB{This is the second ground on which some ascetics and brahmins rely to assert that the cosmos is finite or infinite.}}\\
\end{addmargin}
\end{absolutelynopagebreak}

\begin{absolutelynopagebreak}
\setstretch{.7}
{\PaliGlossA{tatiye ca bhonto samaṇabrāhmaṇā kimāgamma kimārabbha antānantikā antānantaṃ lokassa paññapenti?}}\\
\begin{addmargin}[1em]{2em}
\setstretch{.5}
{\PaliGlossB{And what is the third ground on which they rely?}}\\
\end{addmargin}
\end{absolutelynopagebreak}

\begin{absolutelynopagebreak}
\setstretch{.7}
{\PaliGlossA{idha, bhikkhave, ekacco samaṇo vā brāhmaṇo vā ātappamanvāya padhānamanvāya anuyogamanvāya appamādamanvāya sammāmanasikāramanvāya tathārūpaṃ cetosamādhiṃ phusati, yathāsamāhite citte uddhamadho antasaññī lokasmiṃ viharati, tiriyaṃ anantasaññī.}}\\
\begin{addmargin}[1em]{2em}
\setstretch{.5}
{\PaliGlossB{It’s when some ascetic or brahmin—by dint of keen, resolute, committed, and diligent effort, and right focus—experiences an immersion of the heart of such a kind that they meditate perceiving the cosmos as finite vertically but infinite horizontally.}}\\
\end{addmargin}
\end{absolutelynopagebreak}

\begin{absolutelynopagebreak}
\setstretch{.7}
{\PaliGlossA{so evamāha:}}\\
\begin{addmargin}[1em]{2em}
\setstretch{.5}
{\PaliGlossB{They say:}}\\
\end{addmargin}
\end{absolutelynopagebreak}

\begin{absolutelynopagebreak}
\setstretch{.7}
{\PaliGlossA{‘antavā ca ayaṃ loko ananto ca.}}\\
\begin{addmargin}[1em]{2em}
\setstretch{.5}
{\PaliGlossB{‘The cosmos is both finite and infinite.}}\\
\end{addmargin}
\end{absolutelynopagebreak}

\begin{absolutelynopagebreak}
\setstretch{.7}
{\PaliGlossA{ye te samaṇabrāhmaṇā evamāhaṃsu:}}\\
\begin{addmargin}[1em]{2em}
\setstretch{.5}
{\PaliGlossB{The ascetics and brahmins who say that}}\\
\end{addmargin}
\end{absolutelynopagebreak}

\begin{absolutelynopagebreak}
\setstretch{.7}
{\PaliGlossA{“antavā ayaṃ loko parivaṭumo”ti, tesaṃ musā.}}\\
\begin{addmargin}[1em]{2em}
\setstretch{.5}
{\PaliGlossB{the cosmos is finite are wrong,}}\\
\end{addmargin}
\end{absolutelynopagebreak}

\begin{absolutelynopagebreak}
\setstretch{.7}
{\PaliGlossA{yepi te samaṇabrāhmaṇā evamāhaṃsu:}}\\
\begin{addmargin}[1em]{2em}
\setstretch{.5}
{\PaliGlossB{and so are those who say that}}\\
\end{addmargin}
\end{absolutelynopagebreak}

\begin{absolutelynopagebreak}
\setstretch{.7}
{\PaliGlossA{“ananto ayaṃ loko apariyanto”ti, tesampi musā.}}\\
\begin{addmargin}[1em]{2em}
\setstretch{.5}
{\PaliGlossB{the cosmos is infinite.}}\\
\end{addmargin}
\end{absolutelynopagebreak}

\begin{absolutelynopagebreak}
\setstretch{.7}
{\PaliGlossA{antavā ca ayaṃ loko ananto ca.}}\\
\begin{addmargin}[1em]{2em}
\setstretch{.5}
{\PaliGlossB{The cosmos is both finite and infinite.}}\\
\end{addmargin}
\end{absolutelynopagebreak}

\begin{absolutelynopagebreak}
\setstretch{.7}
{\PaliGlossA{taṃ kissa hetu?}}\\
\begin{addmargin}[1em]{2em}
\setstretch{.5}
{\PaliGlossB{Why is that?}}\\
\end{addmargin}
\end{absolutelynopagebreak}

\begin{absolutelynopagebreak}
\setstretch{.7}
{\PaliGlossA{ahañhi ātappamanvāya padhānamanvāya anuyogamanvāya appamādamanvāya sammāmanasikāramanvāya tathārūpaṃ cetosamādhiṃ phusāmi, yathāsamāhite citte uddhamadho antasaññī lokasmiṃ viharāmi, tiriyaṃ anantasaññī.}}\\
\begin{addmargin}[1em]{2em}
\setstretch{.5}
{\PaliGlossB{Because by dint of keen, resolute, committed, and diligent effort, and right focus I experience an immersion of the heart of such a kind that I meditate perceiving the cosmos as finite vertically but infinite horizontally.}}\\
\end{addmargin}
\end{absolutelynopagebreak}

\begin{absolutelynopagebreak}
\setstretch{.7}
{\PaliGlossA{imināmahaṃ etaṃ jānāmi:}}\\
\begin{addmargin}[1em]{2em}
\setstretch{.5}
{\PaliGlossB{Because of this I know:}}\\
\end{addmargin}
\end{absolutelynopagebreak}

\begin{absolutelynopagebreak}
\setstretch{.7}
{\PaliGlossA{“yathā antavā ca ayaṃ loko ananto cā”’ti.}}\\
\begin{addmargin}[1em]{2em}
\setstretch{.5}
{\PaliGlossB{“The cosmos is both finite and infinite.”’}}\\
\end{addmargin}
\end{absolutelynopagebreak}

\begin{absolutelynopagebreak}
\setstretch{.7}
{\PaliGlossA{idaṃ, bhikkhave, tatiyaṃ ṭhānaṃ, yaṃ āgamma yaṃ ārabbha eke samaṇabrāhmaṇā antānantikā antānantaṃ lokassa paññapenti. (3: 11)}}\\
\begin{addmargin}[1em]{2em}
\setstretch{.5}
{\PaliGlossB{This is the third ground on which some ascetics and brahmins rely to assert that the cosmos is finite or infinite.}}\\
\end{addmargin}
\end{absolutelynopagebreak}

\begin{absolutelynopagebreak}
\setstretch{.7}
{\PaliGlossA{catutthe ca bhonto samaṇabrāhmaṇā kimāgamma kimārabbha antānantikā antānantaṃ lokassa paññapenti?}}\\
\begin{addmargin}[1em]{2em}
\setstretch{.5}
{\PaliGlossB{And what is the fourth ground on which they rely?}}\\
\end{addmargin}
\end{absolutelynopagebreak}

\begin{absolutelynopagebreak}
\setstretch{.7}
{\PaliGlossA{idha, bhikkhave, ekacco samaṇo vā brāhmaṇo vā takkī hoti vīmaṃsī. so takkapariyāhataṃ vīmaṃsānucaritaṃ sayampaṭibhānaṃ evamāha:}}\\
\begin{addmargin}[1em]{2em}
\setstretch{.5}
{\PaliGlossB{It’s when some ascetic or brahmin relies on logic and inquiry. They speak of what they have worked out by logic, following a line of inquiry, expressing their own perspective:}}\\
\end{addmargin}
\end{absolutelynopagebreak}

\begin{absolutelynopagebreak}
\setstretch{.7}
{\PaliGlossA{‘nevāyaṃ loko antavā, na panānanto.}}\\
\begin{addmargin}[1em]{2em}
\setstretch{.5}
{\PaliGlossB{‘The cosmos is neither finite nor infinite.}}\\
\end{addmargin}
\end{absolutelynopagebreak}

\begin{absolutelynopagebreak}
\setstretch{.7}
{\PaliGlossA{ye te samaṇabrāhmaṇā evamāhaṃsu:}}\\
\begin{addmargin}[1em]{2em}
\setstretch{.5}
{\PaliGlossB{The ascetics and brahmins who say that}}\\
\end{addmargin}
\end{absolutelynopagebreak}

\begin{absolutelynopagebreak}
\setstretch{.7}
{\PaliGlossA{“antavā ayaṃ loko parivaṭumo”ti, tesaṃ musā.}}\\
\begin{addmargin}[1em]{2em}
\setstretch{.5}
{\PaliGlossB{the cosmos is finite are wrong,}}\\
\end{addmargin}
\end{absolutelynopagebreak}

\begin{absolutelynopagebreak}
\setstretch{.7}
{\PaliGlossA{yepi te samaṇabrāhmaṇā evamāhaṃsu:}}\\
\begin{addmargin}[1em]{2em}
\setstretch{.5}
{\PaliGlossB{as are those who say that}}\\
\end{addmargin}
\end{absolutelynopagebreak}

\begin{absolutelynopagebreak}
\setstretch{.7}
{\PaliGlossA{“ananto ayaṃ loko apariyanto”ti, tesampi musā.}}\\
\begin{addmargin}[1em]{2em}
\setstretch{.5}
{\PaliGlossB{the cosmos is infinite,}}\\
\end{addmargin}
\end{absolutelynopagebreak}

\begin{absolutelynopagebreak}
\setstretch{.7}
{\PaliGlossA{yepi te samaṇabrāhmaṇā evamāhaṃsu:}}\\
\begin{addmargin}[1em]{2em}
\setstretch{.5}
{\PaliGlossB{and also those who say that}}\\
\end{addmargin}
\end{absolutelynopagebreak}

\begin{absolutelynopagebreak}
\setstretch{.7}
{\PaliGlossA{“antavā ca ayaṃ loko ananto cā”ti, tesampi musā.}}\\
\begin{addmargin}[1em]{2em}
\setstretch{.5}
{\PaliGlossB{the cosmos is both finite and infinite.}}\\
\end{addmargin}
\end{absolutelynopagebreak}

\begin{absolutelynopagebreak}
\setstretch{.7}
{\PaliGlossA{nevāyaṃ loko antavā, na panānanto’ti.}}\\
\begin{addmargin}[1em]{2em}
\setstretch{.5}
{\PaliGlossB{The cosmos is neither finite nor infinite.’}}\\
\end{addmargin}
\end{absolutelynopagebreak}

\begin{absolutelynopagebreak}
\setstretch{.7}
{\PaliGlossA{idaṃ, bhikkhave, catutthaṃ ṭhānaṃ, yaṃ āgamma yaṃ ārabbha eke samaṇabrāhmaṇā antānantikā antānantaṃ lokassa paññapenti. (4: 12)}}\\
\begin{addmargin}[1em]{2em}
\setstretch{.5}
{\PaliGlossB{This is the fourth ground on which some ascetics and brahmins rely to assert that the cosmos is finite or infinite.}}\\
\end{addmargin}
\end{absolutelynopagebreak}

\begin{absolutelynopagebreak}
\setstretch{.7}
{\PaliGlossA{imehi kho te, bhikkhave, samaṇabrāhmaṇā antānantikā antānantaṃ lokassa paññapenti catūhi vatthūhi.}}\\
\begin{addmargin}[1em]{2em}
\setstretch{.5}
{\PaliGlossB{These are the four grounds on which those ascetics and brahmins assert that the cosmos is finite or infinite.}}\\
\end{addmargin}
\end{absolutelynopagebreak}

\begin{absolutelynopagebreak}
\setstretch{.7}
{\PaliGlossA{ye hi keci, bhikkhave, samaṇā vā brāhmaṇā vā antānantikā antānantaṃ lokassa paññapenti, sabbe te imeheva catūhi vatthūhi, etesaṃ vā aññatarena; natthi ito bahiddhā.}}\\
\begin{addmargin}[1em]{2em}
\setstretch{.5}
{\PaliGlossB{Any ascetics and brahmins who assert that the cosmos is finite or infinite do so on one or other of these four grounds. Outside of this there is none.}}\\
\end{addmargin}
\end{absolutelynopagebreak}

\begin{absolutelynopagebreak}
\setstretch{.7}
{\PaliGlossA{tayidaṃ, bhikkhave, tathāgato pajānāti:}}\\
\begin{addmargin}[1em]{2em}
\setstretch{.5}
{\PaliGlossB{The Realized One understands this:}}\\
\end{addmargin}
\end{absolutelynopagebreak}

\begin{absolutelynopagebreak}
\setstretch{.7}
{\PaliGlossA{‘ime diṭṭhiṭṭhānā evaṅgahitā evaṃparāmaṭṭhā evaṅgatikā bhavanti evaṃabhisamparāyā’ti.}}\\
\begin{addmargin}[1em]{2em}
\setstretch{.5}
{\PaliGlossB{‘If you hold on to and attach to these grounds for views it leads to such and such a destiny in the next life.’}}\\
\end{addmargin}
\end{absolutelynopagebreak}

\begin{absolutelynopagebreak}
\setstretch{.7}
{\PaliGlossA{tañca tathāgato pajānāti, tato ca uttaritaraṃ pajānāti, tañca pajānanaṃ na parāmasati, aparāmasato cassa paccattaññeva nibbuti viditā.}}\\
\begin{addmargin}[1em]{2em}
\setstretch{.5}
{\PaliGlossB{He understands this, and what goes beyond this. Yet since he does not misapprehend that understanding, he has realized extinguishment within himself.}}\\
\end{addmargin}
\end{absolutelynopagebreak}

\begin{absolutelynopagebreak}
\setstretch{.7}
{\PaliGlossA{vedanānaṃ samudayañca atthaṅgamañca assādañca ādīnavañca nissaraṇañca yathābhūtaṃ viditvā anupādāvimutto, bhikkhave, tathāgato.}}\\
\begin{addmargin}[1em]{2em}
\setstretch{.5}
{\PaliGlossB{Having truly understood the origin, ending, gratification, drawback, and escape from feelings, the Realized One is freed through not grasping.}}\\
\end{addmargin}
\end{absolutelynopagebreak}

\begin{absolutelynopagebreak}
\setstretch{.7}
{\PaliGlossA{ime kho te, bhikkhave, dhammā gambhīrā duddasā duranubodhā santā paṇītā atakkāvacarā nipuṇā paṇḍitavedanīyā, ye tathāgato sayaṃ abhiññā sacchikatvā pavedeti, yehi tathāgatassa yathābhuccaṃ vaṇṇaṃ sammā vadamānā vadeyyuṃ.}}\\
\begin{addmargin}[1em]{2em}
\setstretch{.5}
{\PaliGlossB{These are the principles—deep, hard to see, hard to understand, peaceful, sublime, beyond the scope of reason, subtle, comprehensible to the astute—which the Realized One makes known after realizing them with his own insight. And those who genuinely praise the Realized One would rightly speak of these things.}}\\
\end{addmargin}
\end{absolutelynopagebreak}

\begin{absolutelynopagebreak}
\setstretch{.7}
{\PaliGlossA{3.1.4. amarāvikkhepavāda}}\\
\begin{addmargin}[1em]{2em}
\setstretch{.5}
{\PaliGlossB{3.1.4. Equivocators}}\\
\end{addmargin}
\end{absolutelynopagebreak}

\begin{absolutelynopagebreak}
\setstretch{.7}
{\PaliGlossA{santi, bhikkhave, eke samaṇabrāhmaṇā amarāvikkhepikā, tattha tattha pañhaṃ puṭṭhā samānā vācāvikkhepaṃ āpajjanti amarāvikkhepaṃ catūhi vatthūhi.}}\\
\begin{addmargin}[1em]{2em}
\setstretch{.5}
{\PaliGlossB{There are some ascetics and brahmins who are equivocators. Whenever they’re asked a question, they resort to evasiveness and equivocation on four grounds.}}\\
\end{addmargin}
\end{absolutelynopagebreak}

\begin{absolutelynopagebreak}
\setstretch{.7}
{\PaliGlossA{te ca bhonto samaṇabrāhmaṇā kimāgamma kimārabbha amarāvikkhepikā tattha tattha pañhaṃ puṭṭhā samānā vācāvikkhepaṃ āpajjanti amarāvikkhepaṃ catūhi vatthūhi?}}\\
\begin{addmargin}[1em]{2em}
\setstretch{.5}
{\PaliGlossB{And what are the four grounds on which they rely?}}\\
\end{addmargin}
\end{absolutelynopagebreak}

\begin{absolutelynopagebreak}
\setstretch{.7}
{\PaliGlossA{idha, bhikkhave, ekacco samaṇo vā brāhmaṇo vā ‘idaṃ kusalan’ti yathābhūtaṃ nappajānāti, ‘idaṃ akusalan’ti yathābhūtaṃ nappajānāti.}}\\
\begin{addmargin}[1em]{2em}
\setstretch{.5}
{\PaliGlossB{It’s when some ascetic or brahmin doesn’t truly understand what is skillful and what is unskillful.}}\\
\end{addmargin}
\end{absolutelynopagebreak}

\begin{absolutelynopagebreak}
\setstretch{.7}
{\PaliGlossA{tassa evaṃ hoti:}}\\
\begin{addmargin}[1em]{2em}
\setstretch{.5}
{\PaliGlossB{They think:}}\\
\end{addmargin}
\end{absolutelynopagebreak}

\begin{absolutelynopagebreak}
\setstretch{.7}
{\PaliGlossA{‘ahaṃ kho “idaṃ kusalan”ti yathābhūtaṃ nappajānāmi, “idaṃ akusalan”ti yathābhūtaṃ nappajānāmi.}}\\
\begin{addmargin}[1em]{2em}
\setstretch{.5}
{\PaliGlossB{‘I don’t truly understand what is skillful and what is unskillful.}}\\
\end{addmargin}
\end{absolutelynopagebreak}

\begin{absolutelynopagebreak}
\setstretch{.7}
{\PaliGlossA{ahañce kho pana “idaṃ kusalan”ti yathābhūtaṃ appajānanto, “idaṃ akusalan”ti yathābhūtaṃ appajānanto, “idaṃ kusalan”ti vā byākareyyaṃ, “idaṃ akusalan”ti vā byākareyyaṃ, taṃ mamassa musā.}}\\
\begin{addmargin}[1em]{2em}
\setstretch{.5}
{\PaliGlossB{If I were to declare that something was skillful or unskillful I might be wrong.}}\\
\end{addmargin}
\end{absolutelynopagebreak}

\begin{absolutelynopagebreak}
\setstretch{.7}
{\PaliGlossA{yaṃ mamassa musā, so mamassa vighāto.}}\\
\begin{addmargin}[1em]{2em}
\setstretch{.5}
{\PaliGlossB{That would be stressful for me,}}\\
\end{addmargin}
\end{absolutelynopagebreak}

\begin{absolutelynopagebreak}
\setstretch{.7}
{\PaliGlossA{yo mamassa vighāto so mamassa antarāyo’ti.}}\\
\begin{addmargin}[1em]{2em}
\setstretch{.5}
{\PaliGlossB{and that stress would be an obstacle.’}}\\
\end{addmargin}
\end{absolutelynopagebreak}

\begin{absolutelynopagebreak}
\setstretch{.7}
{\PaliGlossA{iti so musāvādabhayā musāvādaparijegucchā nevidaṃ kusalanti byākaroti, na panidaṃ akusalanti byākaroti, tattha tattha pañhaṃ puṭṭho samāno vācāvikkhepaṃ āpajjati amarāvikkhepaṃ:}}\\
\begin{addmargin}[1em]{2em}
\setstretch{.5}
{\PaliGlossB{So from fear and disgust with false speech they avoid stating whether something is skillful or unskillful. Whenever they’re asked a question, they resort to evasiveness and equivocation:}}\\
\end{addmargin}
\end{absolutelynopagebreak}

\begin{absolutelynopagebreak}
\setstretch{.7}
{\PaliGlossA{‘evantipi me no; tathātipi me no; aññathātipi me no; notipi me no; no notipi me no’ti.}}\\
\begin{addmargin}[1em]{2em}
\setstretch{.5}
{\PaliGlossB{‘I don’t say it’s like this. I don’t say it’s like that. I don’t say it’s otherwise. I don’t say it’s not so. And I don’t deny it’s not so.’}}\\
\end{addmargin}
\end{absolutelynopagebreak}

\begin{absolutelynopagebreak}
\setstretch{.7}
{\PaliGlossA{idaṃ, bhikkhave, paṭhamaṃ ṭhānaṃ, yaṃ āgamma yaṃ ārabbha eke samaṇabrāhmaṇā amarāvikkhepikā tattha tattha pañhaṃ puṭṭhā samānā vācāvikkhepaṃ āpajjanti amarāvikkhepaṃ. (1: 13)}}\\
\begin{addmargin}[1em]{2em}
\setstretch{.5}
{\PaliGlossB{This is the first ground on which some ascetics and brahmins rely when resorting to evasiveness and equivocation.}}\\
\end{addmargin}
\end{absolutelynopagebreak}

\begin{absolutelynopagebreak}
\setstretch{.7}
{\PaliGlossA{dutiye ca bhonto samaṇabrāhmaṇā kimāgamma kimārabbha amarāvikkhepikā tattha tattha pañhaṃ puṭṭhā samānā vācāvikkhepaṃ āpajjanti amarāvikkhepaṃ?}}\\
\begin{addmargin}[1em]{2em}
\setstretch{.5}
{\PaliGlossB{And what is the second ground on which they rely?}}\\
\end{addmargin}
\end{absolutelynopagebreak}

\begin{absolutelynopagebreak}
\setstretch{.7}
{\PaliGlossA{idha, bhikkhave, ekacco samaṇo vā brāhmaṇo vā ‘idaṃ kusalan’ti yathābhūtaṃ nappajānāti, ‘idaṃ akusalan’ti yathābhūtaṃ nappajānāti.}}\\
\begin{addmargin}[1em]{2em}
\setstretch{.5}
{\PaliGlossB{It’s when some ascetic or brahmin doesn’t truly understand what is skillful and what is unskillful.}}\\
\end{addmargin}
\end{absolutelynopagebreak}

\begin{absolutelynopagebreak}
\setstretch{.7}
{\PaliGlossA{tassa evaṃ hoti:}}\\
\begin{addmargin}[1em]{2em}
\setstretch{.5}
{\PaliGlossB{They think:}}\\
\end{addmargin}
\end{absolutelynopagebreak}

\begin{absolutelynopagebreak}
\setstretch{.7}
{\PaliGlossA{‘ahaṃ kho “idaṃ kusalan”ti yathābhūtaṃ nappajānāmi, “idaṃ akusalan”ti yathābhūtaṃ nappajānāmi.}}\\
\begin{addmargin}[1em]{2em}
\setstretch{.5}
{\PaliGlossB{‘I don’t truly understand what is skillful and what is unskillful.}}\\
\end{addmargin}
\end{absolutelynopagebreak}

\begin{absolutelynopagebreak}
\setstretch{.7}
{\PaliGlossA{ahañce kho pana “idaṃ kusalan”ti yathābhūtaṃ appajānanto, “idaṃ akusalan”ti yathābhūtaṃ appajānanto, “idaṃ kusalan”ti vā byākareyyaṃ, “idaṃ akusalan”ti vā byākareyyaṃ, tattha me assa chando vā rāgo vā doso vā paṭigho vā.}}\\
\begin{addmargin}[1em]{2em}
\setstretch{.5}
{\PaliGlossB{If I were to declare that something was skillful or unskillful I might feel desire or greed or hate or repulsion.}}\\
\end{addmargin}
\end{absolutelynopagebreak}

\begin{absolutelynopagebreak}
\setstretch{.7}
{\PaliGlossA{yattha me assa chando vā rāgo vā doso vā paṭigho vā, taṃ mamassa upādānaṃ.}}\\
\begin{addmargin}[1em]{2em}
\setstretch{.5}
{\PaliGlossB{That would be grasping on my part.}}\\
\end{addmargin}
\end{absolutelynopagebreak}

\begin{absolutelynopagebreak}
\setstretch{.7}
{\PaliGlossA{yaṃ mamassa upādānaṃ, so mamassa vighāto.}}\\
\begin{addmargin}[1em]{2em}
\setstretch{.5}
{\PaliGlossB{That would be stressful for me,}}\\
\end{addmargin}
\end{absolutelynopagebreak}

\begin{absolutelynopagebreak}
\setstretch{.7}
{\PaliGlossA{yo mamassa vighāto, so mamassa antarāyo’ti.}}\\
\begin{addmargin}[1em]{2em}
\setstretch{.5}
{\PaliGlossB{and that stress would be an obstacle.’}}\\
\end{addmargin}
\end{absolutelynopagebreak}

\begin{absolutelynopagebreak}
\setstretch{.7}
{\PaliGlossA{iti so upādānabhayā upādānaparijegucchā nevidaṃ kusalanti byākaroti, na panidaṃ akusalanti byākaroti, tattha tattha pañhaṃ puṭṭho samāno vācāvikkhepaṃ āpajjati amarāvikkhepaṃ:}}\\
\begin{addmargin}[1em]{2em}
\setstretch{.5}
{\PaliGlossB{So from fear and disgust with grasping they avoid stating whether something is skillful or unskillful. Whenever they’re asked a question, they resort to evasiveness and equivocation:}}\\
\end{addmargin}
\end{absolutelynopagebreak}

\begin{absolutelynopagebreak}
\setstretch{.7}
{\PaliGlossA{‘evantipi me no; tathātipi me no; aññathātipi me no; notipi me no; no notipi me no’ti.}}\\
\begin{addmargin}[1em]{2em}
\setstretch{.5}
{\PaliGlossB{‘I don’t say it’s like this. I don’t say it’s like that. I don’t say it’s otherwise. I don’t say it’s not so. And I don’t deny it’s not so.’}}\\
\end{addmargin}
\end{absolutelynopagebreak}

\begin{absolutelynopagebreak}
\setstretch{.7}
{\PaliGlossA{idaṃ, bhikkhave, dutiyaṃ ṭhānaṃ, yaṃ āgamma yaṃ ārabbha eke samaṇabrāhmaṇā amarāvikkhepikā tattha tattha pañhaṃ puṭṭhā samānā vācāvikkhepaṃ āpajjanti amarāvikkhepaṃ. (2: 14)}}\\
\begin{addmargin}[1em]{2em}
\setstretch{.5}
{\PaliGlossB{This is the second ground on which some ascetics and brahmins rely when resorting to evasiveness and equivocation.}}\\
\end{addmargin}
\end{absolutelynopagebreak}

\begin{absolutelynopagebreak}
\setstretch{.7}
{\PaliGlossA{tatiye ca bhonto samaṇabrāhmaṇā kimāgamma kimārabbha amarāvikkhepikā tattha tattha pañhaṃ puṭṭhā samānā vācāvikkhepaṃ āpajjanti amarāvikkhepaṃ?}}\\
\begin{addmargin}[1em]{2em}
\setstretch{.5}
{\PaliGlossB{And what is the third ground on which they rely?}}\\
\end{addmargin}
\end{absolutelynopagebreak}

\begin{absolutelynopagebreak}
\setstretch{.7}
{\PaliGlossA{idha, bhikkhave, ekacco samaṇo vā brāhmaṇo vā ‘idaṃ kusalan’ti yathābhūtaṃ nappajānāti, ‘idaṃ akusalan’ti yathābhūtaṃ nappajānāti.}}\\
\begin{addmargin}[1em]{2em}
\setstretch{.5}
{\PaliGlossB{It’s when some ascetic or brahmin doesn’t truly understand what is skillful and what is unskillful.}}\\
\end{addmargin}
\end{absolutelynopagebreak}

\begin{absolutelynopagebreak}
\setstretch{.7}
{\PaliGlossA{tassa evaṃ hoti:}}\\
\begin{addmargin}[1em]{2em}
\setstretch{.5}
{\PaliGlossB{They think:}}\\
\end{addmargin}
\end{absolutelynopagebreak}

\begin{absolutelynopagebreak}
\setstretch{.7}
{\PaliGlossA{‘ahaṃ kho “idaṃ kusalan”ti yathābhūtaṃ nappajānāmi, “idaṃ akusalan”ti yathābhūtaṃ nappajānāmi.}}\\
\begin{addmargin}[1em]{2em}
\setstretch{.5}
{\PaliGlossB{‘I don’t truly understand what is skillful and what is unskillful.}}\\
\end{addmargin}
\end{absolutelynopagebreak}

\begin{absolutelynopagebreak}
\setstretch{.7}
{\PaliGlossA{ahañce kho pana “idaṃ kusalan”ti yathābhūtaṃ appajānanto “idaṃ akusalan”ti yathābhūtaṃ appajānanto “idaṃ kusalan”ti vā byākareyyaṃ, “idaṃ akusalan”ti vā byākareyyaṃ;}}\\
\begin{addmargin}[1em]{2em}
\setstretch{.5}
{\PaliGlossB{Suppose I were to declare that something was skillful or unskillful.}}\\
\end{addmargin}
\end{absolutelynopagebreak}

\begin{absolutelynopagebreak}
\setstretch{.7}
{\PaliGlossA{santi hi kho samaṇabrāhmaṇā paṇḍitā nipuṇā kataparappavādā vālavedhirūpā, te bhindantā maññe caranti paññāgatena diṭṭhigatāni,}}\\
\begin{addmargin}[1em]{2em}
\setstretch{.5}
{\PaliGlossB{There are clever ascetics and brahmins who are subtle, accomplished in the doctrines of others, hair-splitters. You’d think they live to demolish convictions with their intellect.}}\\
\end{addmargin}
\end{absolutelynopagebreak}

\begin{absolutelynopagebreak}
\setstretch{.7}
{\PaliGlossA{te maṃ tattha samanuyuñjeyyuṃ samanugāheyyuṃ samanubhāseyyuṃ.}}\\
\begin{addmargin}[1em]{2em}
\setstretch{.5}
{\PaliGlossB{They might pursue, press, and grill me about that.}}\\
\end{addmargin}
\end{absolutelynopagebreak}

\begin{absolutelynopagebreak}
\setstretch{.7}
{\PaliGlossA{ye maṃ tattha samanuyuñjeyyuṃ samanugāheyyuṃ samanubhāseyyuṃ, tesāhaṃ na sampāyeyyaṃ.}}\\
\begin{addmargin}[1em]{2em}
\setstretch{.5}
{\PaliGlossB{I’d be stumped by such a grilling.}}\\
\end{addmargin}
\end{absolutelynopagebreak}

\begin{absolutelynopagebreak}
\setstretch{.7}
{\PaliGlossA{yesāhaṃ na sampāyeyyaṃ, so mamassa vighāto.}}\\
\begin{addmargin}[1em]{2em}
\setstretch{.5}
{\PaliGlossB{That would be stressful for me,}}\\
\end{addmargin}
\end{absolutelynopagebreak}

\begin{absolutelynopagebreak}
\setstretch{.7}
{\PaliGlossA{yo mamassa vighāto, so mamassa antarāyo’ti.}}\\
\begin{addmargin}[1em]{2em}
\setstretch{.5}
{\PaliGlossB{and that stress would be an obstacle.’}}\\
\end{addmargin}
\end{absolutelynopagebreak}

\begin{absolutelynopagebreak}
\setstretch{.7}
{\PaliGlossA{iti so anuyogabhayā anuyogaparijegucchā nevidaṃ kusalanti byākaroti, na panidaṃ akusalanti byākaroti, tattha tattha pañhaṃ puṭṭho samāno vācāvikkhepaṃ āpajjati amarāvikkhepaṃ:}}\\
\begin{addmargin}[1em]{2em}
\setstretch{.5}
{\PaliGlossB{So from fear and disgust with examination they avoid stating whether something is skillful or unskillful. Whenever they’re asked a question, they resort to evasiveness and equivocation:}}\\
\end{addmargin}
\end{absolutelynopagebreak}

\begin{absolutelynopagebreak}
\setstretch{.7}
{\PaliGlossA{‘evantipi me no; tathātipi me no; aññathātipi me no; notipi me no; no notipi me no’ti.}}\\
\begin{addmargin}[1em]{2em}
\setstretch{.5}
{\PaliGlossB{‘I don’t say it’s like this. I don’t say it’s like that. I don’t say it’s otherwise. I don’t say it’s not so. And I don’t deny it’s not so.’}}\\
\end{addmargin}
\end{absolutelynopagebreak}

\begin{absolutelynopagebreak}
\setstretch{.7}
{\PaliGlossA{idaṃ, bhikkhave, tatiyaṃ ṭhānaṃ, yaṃ āgamma yaṃ ārabbha eke samaṇabrāhmaṇā amarāvikkhepikā tattha tattha pañhaṃ puṭṭhā samānā vācāvikkhepaṃ āpajjanti amarāvikkhepaṃ. (3: 15)}}\\
\begin{addmargin}[1em]{2em}
\setstretch{.5}
{\PaliGlossB{This is the third ground on which some ascetics and brahmins rely when resorting to evasiveness and equivocation.}}\\
\end{addmargin}
\end{absolutelynopagebreak}

\begin{absolutelynopagebreak}
\setstretch{.7}
{\PaliGlossA{catutthe ca bhonto samaṇabrāhmaṇā kimāgamma kimārabbha amarāvikkhepikā tattha tattha pañhaṃ puṭṭhā samānā vācāvikkhepaṃ āpajjanti amarāvikkhepaṃ?}}\\
\begin{addmargin}[1em]{2em}
\setstretch{.5}
{\PaliGlossB{And what is the fourth ground on which they rely?}}\\
\end{addmargin}
\end{absolutelynopagebreak}

\begin{absolutelynopagebreak}
\setstretch{.7}
{\PaliGlossA{idha, bhikkhave, ekacco samaṇo vā brāhmaṇo vā mando hoti momūho.}}\\
\begin{addmargin}[1em]{2em}
\setstretch{.5}
{\PaliGlossB{It’s when some ascetic or brahmin is dull and stupid.}}\\
\end{addmargin}
\end{absolutelynopagebreak}

\begin{absolutelynopagebreak}
\setstretch{.7}
{\PaliGlossA{so mandattā momūhattā tattha tattha pañhaṃ puṭṭho samāno vācāvikkhepaṃ āpajjati amarāvikkhepaṃ:}}\\
\begin{addmargin}[1em]{2em}
\setstretch{.5}
{\PaliGlossB{Because of that, whenever they’re asked a question, they resort to evasiveness and equivocation:}}\\
\end{addmargin}
\end{absolutelynopagebreak}

\begin{absolutelynopagebreak}
\setstretch{.7}
{\PaliGlossA{‘atthi paro loko’ti iti ce maṃ pucchasi, ‘atthi paro loko’ti iti ce me assa, ‘atthi paro loko’ti iti te naṃ byākareyyaṃ,}}\\
\begin{addmargin}[1em]{2em}
\setstretch{.5}
{\PaliGlossB{‘Suppose you were to ask me whether there is another world. If I believed there was, I would say so.}}\\
\end{addmargin}
\end{absolutelynopagebreak}

\begin{absolutelynopagebreak}
\setstretch{.7}
{\PaliGlossA{‘evantipi me no, tathātipi me no, aññathātipi me no, notipi me no, no notipi me no’ti.}}\\
\begin{addmargin}[1em]{2em}
\setstretch{.5}
{\PaliGlossB{But I don’t say it’s like this. I don’t say it’s like that. I don’t say it’s otherwise. I don’t say it’s not so. And I don’t deny it’s not so.}}\\
\end{addmargin}
\end{absolutelynopagebreak}

\begin{absolutelynopagebreak}
\setstretch{.7}
{\PaliGlossA{‘natthi paro loko … pe …}}\\
\begin{addmargin}[1em]{2em}
\setstretch{.5}
{\PaliGlossB{Suppose you were to ask me whether there is no other world …}}\\
\end{addmargin}
\end{absolutelynopagebreak}

\begin{absolutelynopagebreak}
\setstretch{.7}
{\PaliGlossA{‘atthi ca natthi ca paro loko … pe …}}\\
\begin{addmargin}[1em]{2em}
\setstretch{.5}
{\PaliGlossB{whether there both is and is not another world …}}\\
\end{addmargin}
\end{absolutelynopagebreak}

\begin{absolutelynopagebreak}
\setstretch{.7}
{\PaliGlossA{‘nevatthi na natthi paro loko … pe …}}\\
\begin{addmargin}[1em]{2em}
\setstretch{.5}
{\PaliGlossB{whether there neither is nor is not another world …}}\\
\end{addmargin}
\end{absolutelynopagebreak}

\begin{absolutelynopagebreak}
\setstretch{.7}
{\PaliGlossA{‘atthi sattā opapātikā … pe …}}\\
\begin{addmargin}[1em]{2em}
\setstretch{.5}
{\PaliGlossB{whether there are beings who are reborn spontaneously …}}\\
\end{addmargin}
\end{absolutelynopagebreak}

\begin{absolutelynopagebreak}
\setstretch{.7}
{\PaliGlossA{‘natthi sattā opapātikā … pe …}}\\
\begin{addmargin}[1em]{2em}
\setstretch{.5}
{\PaliGlossB{whether there are not beings who are reborn spontaneously …}}\\
\end{addmargin}
\end{absolutelynopagebreak}

\begin{absolutelynopagebreak}
\setstretch{.7}
{\PaliGlossA{‘atthi ca natthi ca sattā opapātikā … pe …}}\\
\begin{addmargin}[1em]{2em}
\setstretch{.5}
{\PaliGlossB{whether there both are and are not beings who are reborn spontaneously …}}\\
\end{addmargin}
\end{absolutelynopagebreak}

\begin{absolutelynopagebreak}
\setstretch{.7}
{\PaliGlossA{‘nevatthi na natthi sattā opapātikā … pe …}}\\
\begin{addmargin}[1em]{2em}
\setstretch{.5}
{\PaliGlossB{whether there neither are nor are not beings who are reborn spontaneously …}}\\
\end{addmargin}
\end{absolutelynopagebreak}

\begin{absolutelynopagebreak}
\setstretch{.7}
{\PaliGlossA{‘atthi sukatadukkaṭānaṃ kammānaṃ phalaṃ vipāko … pe …}}\\
\begin{addmargin}[1em]{2em}
\setstretch{.5}
{\PaliGlossB{whether there is fruit and result of good and bad deeds …}}\\
\end{addmargin}
\end{absolutelynopagebreak}

\begin{absolutelynopagebreak}
\setstretch{.7}
{\PaliGlossA{‘natthi sukatadukkaṭānaṃ kammānaṃ phalaṃ vipāko … pe …}}\\
\begin{addmargin}[1em]{2em}
\setstretch{.5}
{\PaliGlossB{whether there is not fruit and result of good and bad deeds …}}\\
\end{addmargin}
\end{absolutelynopagebreak}

\begin{absolutelynopagebreak}
\setstretch{.7}
{\PaliGlossA{‘atthi ca natthi ca sukatadukkaṭānaṃ kammānaṃ phalaṃ vipāko … pe …}}\\
\begin{addmargin}[1em]{2em}
\setstretch{.5}
{\PaliGlossB{whether there both is and is not fruit and result of good and bad deeds …}}\\
\end{addmargin}
\end{absolutelynopagebreak}

\begin{absolutelynopagebreak}
\setstretch{.7}
{\PaliGlossA{‘nevatthi na natthi sukatadukkaṭānaṃ kammānaṃ phalaṃ vipāko … pe …}}\\
\begin{addmargin}[1em]{2em}
\setstretch{.5}
{\PaliGlossB{whether there neither is nor is not fruit and result of good and bad deeds …}}\\
\end{addmargin}
\end{absolutelynopagebreak}

\begin{absolutelynopagebreak}
\setstretch{.7}
{\PaliGlossA{‘hoti tathāgato paraṃ maraṇā … pe …}}\\
\begin{addmargin}[1em]{2em}
\setstretch{.5}
{\PaliGlossB{whether a Realized One exists after death …}}\\
\end{addmargin}
\end{absolutelynopagebreak}

\begin{absolutelynopagebreak}
\setstretch{.7}
{\PaliGlossA{‘na hoti tathāgato paraṃ maraṇā … pe …}}\\
\begin{addmargin}[1em]{2em}
\setstretch{.5}
{\PaliGlossB{whether a Realized One doesn’t exist after death …}}\\
\end{addmargin}
\end{absolutelynopagebreak}

\begin{absolutelynopagebreak}
\setstretch{.7}
{\PaliGlossA{‘hoti ca na ca hoti tathāgato paraṃ maraṇā … pe …}}\\
\begin{addmargin}[1em]{2em}
\setstretch{.5}
{\PaliGlossB{whether a Realized One both exists and doesn’t exist after death …}}\\
\end{addmargin}
\end{absolutelynopagebreak}

\begin{absolutelynopagebreak}
\setstretch{.7}
{\PaliGlossA{‘neva hoti na na hoti tathāgato paraṃ maraṇā’ti iti ce maṃ pucchasi, ‘neva hoti na na hoti tathāgato paraṃ maraṇā’ti iti ce me assa, ‘neva hoti na na hoti tathāgato paraṃ maraṇā’ti iti te naṃ byākareyyaṃ,}}\\
\begin{addmargin}[1em]{2em}
\setstretch{.5}
{\PaliGlossB{whether a Realized One neither exists nor doesn’t exist after death. If I believed there was, I would say so.}}\\
\end{addmargin}
\end{absolutelynopagebreak}

\begin{absolutelynopagebreak}
\setstretch{.7}
{\PaliGlossA{‘evantipi me no, tathātipi me no, aññathātipi me no, notipi me no, no notipi me no’ti.}}\\
\begin{addmargin}[1em]{2em}
\setstretch{.5}
{\PaliGlossB{But I don’t say it’s like this. I don’t say it’s like that. I don’t say it’s otherwise. I don’t say it’s not so. And I don’t deny it’s not so.’}}\\
\end{addmargin}
\end{absolutelynopagebreak}

\begin{absolutelynopagebreak}
\setstretch{.7}
{\PaliGlossA{idaṃ, bhikkhave, catutthaṃ ṭhānaṃ, yaṃ āgamma yaṃ ārabbha eke samaṇabrāhmaṇā amarāvikkhepikā tattha tattha pañhaṃ puṭṭhā samānā vācāvikkhepaṃ āpajjanti amarāvikkhepaṃ. (4: 16)}}\\
\begin{addmargin}[1em]{2em}
\setstretch{.5}
{\PaliGlossB{This is the fourth ground on which some ascetics and brahmins rely when resorting to evasiveness and equivocation.}}\\
\end{addmargin}
\end{absolutelynopagebreak}

\begin{absolutelynopagebreak}
\setstretch{.7}
{\PaliGlossA{imehi kho te, bhikkhave, samaṇabrāhmaṇā amarāvikkhepikā tattha tattha pañhaṃ puṭṭhā samānā vācāvikkhepaṃ āpajjanti amarāvikkhepaṃ catūhi vatthūhi.}}\\
\begin{addmargin}[1em]{2em}
\setstretch{.5}
{\PaliGlossB{These are the four grounds on which those ascetics and brahmins who are equivocators resort to evasiveness and equivocation whenever they’re asked a question.}}\\
\end{addmargin}
\end{absolutelynopagebreak}

\begin{absolutelynopagebreak}
\setstretch{.7}
{\PaliGlossA{ye hi keci, bhikkhave, samaṇā vā brāhmaṇā vā amarāvikkhepikā tattha tattha pañhaṃ puṭṭhā samānā vācāvikkhepaṃ āpajjanti amarāvikkhepaṃ, sabbe te imeheva catūhi vatthūhi, etesaṃ vā aññatarena, natthi ito bahiddhā …}}\\
\begin{addmargin}[1em]{2em}
\setstretch{.5}
{\PaliGlossB{Any ascetics and brahmins who resort to equivocation do so on one or other of these four grounds. Outside of this there is none.}}\\
\end{addmargin}
\end{absolutelynopagebreak}

\begin{absolutelynopagebreak}
\setstretch{.7}
{\PaliGlossA{pe …}}\\
\begin{addmargin}[1em]{2em}
\setstretch{.5}
{\PaliGlossB{The Realized One understands this …}}\\
\end{addmargin}
\end{absolutelynopagebreak}

\begin{absolutelynopagebreak}
\setstretch{.7}
{\PaliGlossA{yehi tathāgatassa yathābhuccaṃ vaṇṇaṃ sammā vadamānā vadeyyuṃ.}}\\
\begin{addmargin}[1em]{2em}
\setstretch{.5}
{\PaliGlossB{And those who genuinely praise the Realized One would rightly speak of these things.}}\\
\end{addmargin}
\end{absolutelynopagebreak}

\begin{absolutelynopagebreak}
\setstretch{.7}
{\PaliGlossA{3.1.5. adhiccasamuppannavāda}}\\
\begin{addmargin}[1em]{2em}
\setstretch{.5}
{\PaliGlossB{3.1.5. Doctrines of Origination by Chance}}\\
\end{addmargin}
\end{absolutelynopagebreak}

\begin{absolutelynopagebreak}
\setstretch{.7}
{\PaliGlossA{santi, bhikkhave, eke samaṇabrāhmaṇā adhiccasamuppannikā adhiccasamuppannaṃ attānañca lokañca paññapenti dvīhi vatthūhi.}}\\
\begin{addmargin}[1em]{2em}
\setstretch{.5}
{\PaliGlossB{There are some ascetics and brahmins who theorize about chance. They assert that the self and the cosmos arose by chance on two grounds.}}\\
\end{addmargin}
\end{absolutelynopagebreak}

\begin{absolutelynopagebreak}
\setstretch{.7}
{\PaliGlossA{te ca bhonto samaṇabrāhmaṇā kimāgamma kimārabbha adhiccasamuppannikā adhiccasamuppannaṃ attānañca lokañca paññapenti dvīhi vatthūhi?}}\\
\begin{addmargin}[1em]{2em}
\setstretch{.5}
{\PaliGlossB{And what are the two grounds on which they rely?}}\\
\end{addmargin}
\end{absolutelynopagebreak}

\begin{absolutelynopagebreak}
\setstretch{.7}
{\PaliGlossA{santi, bhikkhave, asaññasattā nāma devā.}}\\
\begin{addmargin}[1em]{2em}
\setstretch{.5}
{\PaliGlossB{There are gods named ‘non-percipient beings’.}}\\
\end{addmargin}
\end{absolutelynopagebreak}

\begin{absolutelynopagebreak}
\setstretch{.7}
{\PaliGlossA{saññuppādā ca pana te devā tamhā kāyā cavanti.}}\\
\begin{addmargin}[1em]{2em}
\setstretch{.5}
{\PaliGlossB{When perception arises they pass away from that host of gods.}}\\
\end{addmargin}
\end{absolutelynopagebreak}

\begin{absolutelynopagebreak}
\setstretch{.7}
{\PaliGlossA{ṭhānaṃ kho panetaṃ, bhikkhave, vijjati, yaṃ aññataro satto tamhā kāyā cavitvā itthattaṃ āgacchati.}}\\
\begin{addmargin}[1em]{2em}
\setstretch{.5}
{\PaliGlossB{It’s possible that one of those beings passes away from that host and is reborn in this state of existence.}}\\
\end{addmargin}
\end{absolutelynopagebreak}

\begin{absolutelynopagebreak}
\setstretch{.7}
{\PaliGlossA{itthattaṃ āgato samāno agārasmā anagāriyaṃ pabbajati.}}\\
\begin{addmargin}[1em]{2em}
\setstretch{.5}
{\PaliGlossB{Having done so, they go forth from the lay life to homelessness.}}\\
\end{addmargin}
\end{absolutelynopagebreak}

\begin{absolutelynopagebreak}
\setstretch{.7}
{\PaliGlossA{agārasmā anagāriyaṃ pabbajito samāno ātappamanvāya padhānamanvāya anuyogamanvāya appamādamanvāya sammāmanasikāramanvāya tathārūpaṃ cetosamādhiṃ phusati, yathāsamāhite citte saññuppādaṃ anussarati, tato paraṃ nānussarati.}}\\
\begin{addmargin}[1em]{2em}
\setstretch{.5}
{\PaliGlossB{By dint of keen, resolute, committed, and diligent effort, and right focus, they experience an immersion of the heart of such a kind that they recollect the arising of perception, but no further.}}\\
\end{addmargin}
\end{absolutelynopagebreak}

\begin{absolutelynopagebreak}
\setstretch{.7}
{\PaliGlossA{so evamāha:}}\\
\begin{addmargin}[1em]{2em}
\setstretch{.5}
{\PaliGlossB{They say:}}\\
\end{addmargin}
\end{absolutelynopagebreak}

\begin{absolutelynopagebreak}
\setstretch{.7}
{\PaliGlossA{‘adhiccasamuppanno attā ca loko ca.}}\\
\begin{addmargin}[1em]{2em}
\setstretch{.5}
{\PaliGlossB{‘The self and the cosmos arose by chance.}}\\
\end{addmargin}
\end{absolutelynopagebreak}

\begin{absolutelynopagebreak}
\setstretch{.7}
{\PaliGlossA{taṃ kissa hetu?}}\\
\begin{addmargin}[1em]{2em}
\setstretch{.5}
{\PaliGlossB{Why is that?}}\\
\end{addmargin}
\end{absolutelynopagebreak}

\begin{absolutelynopagebreak}
\setstretch{.7}
{\PaliGlossA{ahañhi pubbe nāhosiṃ, somhi etarahi ahutvā santatāya pariṇato’ti.}}\\
\begin{addmargin}[1em]{2em}
\setstretch{.5}
{\PaliGlossB{Because formerly I didn’t exist. Now, having not been, I’ve sprung into existence.’}}\\
\end{addmargin}
\end{absolutelynopagebreak}

\begin{absolutelynopagebreak}
\setstretch{.7}
{\PaliGlossA{idaṃ, bhikkhave, paṭhamaṃ ṭhānaṃ, yaṃ āgamma yaṃ ārabbha eke samaṇabrāhmaṇā adhiccasamuppannikā adhiccasamuppannaṃ attānañca lokañca paññapenti. (1: 17)}}\\
\begin{addmargin}[1em]{2em}
\setstretch{.5}
{\PaliGlossB{This is the first ground on which some ascetics and brahmins rely to assert that the self and the cosmos arose by chance.}}\\
\end{addmargin}
\end{absolutelynopagebreak}

\begin{absolutelynopagebreak}
\setstretch{.7}
{\PaliGlossA{dutiye ca bhonto samaṇabrāhmaṇā kimāgamma kimārabbha adhiccasamuppannikā adhiccasamuppannaṃ attānañca lokañca paññapenti?}}\\
\begin{addmargin}[1em]{2em}
\setstretch{.5}
{\PaliGlossB{And what is the second ground on which they rely?}}\\
\end{addmargin}
\end{absolutelynopagebreak}

\begin{absolutelynopagebreak}
\setstretch{.7}
{\PaliGlossA{idha, bhikkhave, ekacco samaṇo vā brāhmaṇo vā takkī hoti vīmaṃsī.}}\\
\begin{addmargin}[1em]{2em}
\setstretch{.5}
{\PaliGlossB{It’s when some ascetic or brahmin relies on logic and inquiry.}}\\
\end{addmargin}
\end{absolutelynopagebreak}

\begin{absolutelynopagebreak}
\setstretch{.7}
{\PaliGlossA{so takkapariyāhataṃ vīmaṃsānucaritaṃ sayampaṭibhānaṃ evamāha:}}\\
\begin{addmargin}[1em]{2em}
\setstretch{.5}
{\PaliGlossB{They speak of what they have worked out by logic, following a line of inquiry, expressing their own perspective:}}\\
\end{addmargin}
\end{absolutelynopagebreak}

\begin{absolutelynopagebreak}
\setstretch{.7}
{\PaliGlossA{‘adhiccasamuppanno attā ca loko cā’ti.}}\\
\begin{addmargin}[1em]{2em}
\setstretch{.5}
{\PaliGlossB{‘The self and the cosmos arose by chance.’}}\\
\end{addmargin}
\end{absolutelynopagebreak}

\begin{absolutelynopagebreak}
\setstretch{.7}
{\PaliGlossA{idaṃ, bhikkhave, dutiyaṃ ṭhānaṃ, yaṃ āgamma yaṃ ārabbha eke samaṇabrāhmaṇā adhiccasamuppannikā adhiccasamuppannaṃ attānañca lokañca paññapenti. (2: 18)}}\\
\begin{addmargin}[1em]{2em}
\setstretch{.5}
{\PaliGlossB{This is the second ground on which some ascetics and brahmins rely to assert that the self and the cosmos arose by chance.}}\\
\end{addmargin}
\end{absolutelynopagebreak}

\begin{absolutelynopagebreak}
\setstretch{.7}
{\PaliGlossA{imehi kho te, bhikkhave, samaṇabrāhmaṇā adhiccasamuppannikā adhiccasamuppannaṃ attānañca lokañca paññapenti dvīhi vatthūhi.}}\\
\begin{addmargin}[1em]{2em}
\setstretch{.5}
{\PaliGlossB{These are the two grounds on which those ascetics and brahmins who theorize about chance assert that the self and the cosmos arose by chance.}}\\
\end{addmargin}
\end{absolutelynopagebreak}

\begin{absolutelynopagebreak}
\setstretch{.7}
{\PaliGlossA{ye hi keci, bhikkhave, samaṇā vā brāhmaṇā vā adhiccasamuppannikā adhiccasamuppannaṃ attānañca lokañca paññapenti, sabbe te imeheva dvīhi vatthūhi, etesaṃ vā aññatarena, natthi ito bahiddhā …}}\\
\begin{addmargin}[1em]{2em}
\setstretch{.5}
{\PaliGlossB{Any ascetics and brahmins who theorize about chance do so on one or other of these two grounds. Outside of this there is none.}}\\
\end{addmargin}
\end{absolutelynopagebreak}

\begin{absolutelynopagebreak}
\setstretch{.7}
{\PaliGlossA{pe …}}\\
\begin{addmargin}[1em]{2em}
\setstretch{.5}
{\PaliGlossB{The Realized One understands this …}}\\
\end{addmargin}
\end{absolutelynopagebreak}

\begin{absolutelynopagebreak}
\setstretch{.7}
{\PaliGlossA{yehi tathāgatassa yathābhuccaṃ vaṇṇaṃ sammā vadamānā vadeyyuṃ.}}\\
\begin{addmargin}[1em]{2em}
\setstretch{.5}
{\PaliGlossB{And those who genuinely praise the Realized One would rightly speak of these things.}}\\
\end{addmargin}
\end{absolutelynopagebreak}

\begin{absolutelynopagebreak}
\setstretch{.7}
{\PaliGlossA{imehi kho te, bhikkhave, samaṇabrāhmaṇā pubbantakappikā pubbantānudiṭṭhino pubbantaṃ ārabbha anekavihitāni adhimuttipadāni abhivadanti aṭṭhārasahi vatthūhi.}}\\
\begin{addmargin}[1em]{2em}
\setstretch{.5}
{\PaliGlossB{These are the eighteen grounds on which those ascetics and brahmins who theorize about the past assert various hypotheses concerning the past.}}\\
\end{addmargin}
\end{absolutelynopagebreak}

\begin{absolutelynopagebreak}
\setstretch{.7}
{\PaliGlossA{ye hi keci, bhikkhave, samaṇā vā brāhmaṇā vā pubbantakappikā pubbantānudiṭṭhino pubbantamārabbha anekavihitāni adhimuttipadāni abhivadanti, sabbe te imeheva aṭṭhārasahi vatthūhi, etesaṃ vā aññatarena, natthi ito bahiddhā.}}\\
\begin{addmargin}[1em]{2em}
\setstretch{.5}
{\PaliGlossB{Any ascetics and brahmins who theorize about the past do so on one or other of these eighteen grounds. Outside of this there is none.}}\\
\end{addmargin}
\end{absolutelynopagebreak}

\begin{absolutelynopagebreak}
\setstretch{.7}
{\PaliGlossA{tayidaṃ, bhikkhave, tathāgato pajānāti:}}\\
\begin{addmargin}[1em]{2em}
\setstretch{.5}
{\PaliGlossB{The Realized One understands this:}}\\
\end{addmargin}
\end{absolutelynopagebreak}

\begin{absolutelynopagebreak}
\setstretch{.7}
{\PaliGlossA{‘ime diṭṭhiṭṭhānā evaṅgahitā evaṃparāmaṭṭhā evaṅgatikā bhavanti evaṃabhisamparāyā’ti.}}\\
\begin{addmargin}[1em]{2em}
\setstretch{.5}
{\PaliGlossB{‘If you hold on to and attach to these grounds for views it leads to such and such a destiny in the next life.’}}\\
\end{addmargin}
\end{absolutelynopagebreak}

\begin{absolutelynopagebreak}
\setstretch{.7}
{\PaliGlossA{tañca tathāgato pajānāti, tato ca uttaritaraṃ pajānāti, tañca pajānanaṃ na parāmasati, aparāmasato cassa paccattaññeva nibbuti viditā.}}\\
\begin{addmargin}[1em]{2em}
\setstretch{.5}
{\PaliGlossB{He understands this, and what goes beyond this. Yet since he does not misapprehend that understanding, he has realized extinguishment within himself.}}\\
\end{addmargin}
\end{absolutelynopagebreak}

\begin{absolutelynopagebreak}
\setstretch{.7}
{\PaliGlossA{vedanānaṃ samudayañca atthaṅgamañca assādañca ādīnavañca nissaraṇañca yathābhūtaṃ viditvā anupādāvimutto, bhikkhave, tathāgato.}}\\
\begin{addmargin}[1em]{2em}
\setstretch{.5}
{\PaliGlossB{Having truly understood the origin, ending, gratification, drawback, and escape from feelings, the Realized One is freed through not grasping.}}\\
\end{addmargin}
\end{absolutelynopagebreak}

\begin{absolutelynopagebreak}
\setstretch{.7}
{\PaliGlossA{ime kho te, bhikkhave, dhammā gambhīrā duddasā duranubodhā santā paṇītā atakkāvacarā nipuṇā paṇḍitavedanīyā, ye tathāgato sayaṃ abhiññā sacchikatvā pavedeti, yehi tathāgatassa yathābhuccaṃ vaṇṇaṃ sammā vadamānā vadeyyuṃ.}}\\
\begin{addmargin}[1em]{2em}
\setstretch{.5}
{\PaliGlossB{These are the principles—deep, hard to see, hard to understand, peaceful, sublime, beyond the scope of reason, subtle, comprehensible to the astute—which the Realized One makes known after realizing them with his own insight. And those who genuinely praise the Realized One would rightly speak of these things.}}\\
\end{addmargin}
\end{absolutelynopagebreak}

\begin{absolutelynopagebreak}
\setstretch{.7}
{\PaliGlossA{dutiyabhāṇavāro.}}\\
\begin{addmargin}[1em]{2em}
\setstretch{.5}
{\PaliGlossB{    -}}\\
\end{addmargin}
\end{absolutelynopagebreak}

\begin{absolutelynopagebreak}
\setstretch{.7}
{\PaliGlossA{3.2. aparantakappika}}\\
\begin{addmargin}[1em]{2em}
\setstretch{.5}
{\PaliGlossB{3.2. Theories About the Future}}\\
\end{addmargin}
\end{absolutelynopagebreak}

\begin{absolutelynopagebreak}
\setstretch{.7}
{\PaliGlossA{santi, bhikkhave, eke samaṇabrāhmaṇā aparantakappikā aparantānudiṭṭhino, aparantaṃ ārabbha anekavihitāni adhimuttipadāni abhivadanti catucattārīsāya vatthūhi.}}\\
\begin{addmargin}[1em]{2em}
\setstretch{.5}
{\PaliGlossB{There are some ascetics and brahmins who theorize about the future, and assert various hypotheses concerning the future on forty-four grounds.}}\\
\end{addmargin}
\end{absolutelynopagebreak}

\begin{absolutelynopagebreak}
\setstretch{.7}
{\PaliGlossA{te ca bhonto samaṇabrāhmaṇā kimāgamma kimārabbha aparantakappikā aparantānudiṭṭhino aparantaṃ ārabbha anekavihitāni adhimuttipadāni abhivadanti catucattārīsāya vatthūhi?}}\\
\begin{addmargin}[1em]{2em}
\setstretch{.5}
{\PaliGlossB{And what are the forty-four grounds on which they rely?}}\\
\end{addmargin}
\end{absolutelynopagebreak}

\begin{absolutelynopagebreak}
\setstretch{.7}
{\PaliGlossA{3.2.1. saññīvāda}}\\
\begin{addmargin}[1em]{2em}
\setstretch{.5}
{\PaliGlossB{3.2.1. Percipient Life After Death}}\\
\end{addmargin}
\end{absolutelynopagebreak}

\begin{absolutelynopagebreak}
\setstretch{.7}
{\PaliGlossA{santi, bhikkhave, eke samaṇabrāhmaṇā uddhamāghātanikā saññīvādā uddhamāghātanaṃ saññiṃ attānaṃ paññapenti soḷasahi vatthūhi.}}\\
\begin{addmargin}[1em]{2em}
\setstretch{.5}
{\PaliGlossB{There are some ascetics and brahmins who say there is life after death, and assert that the self lives on after death in a percipient form on sixteen grounds.}}\\
\end{addmargin}
\end{absolutelynopagebreak}

\begin{absolutelynopagebreak}
\setstretch{.7}
{\PaliGlossA{te ca bhonto samaṇabrāhmaṇā kimāgamma kimārabbha uddhamāghātanikā saññīvādā uddhamāghātanaṃ saññiṃ attānaṃ paññapenti soḷasahi vatthūhi?}}\\
\begin{addmargin}[1em]{2em}
\setstretch{.5}
{\PaliGlossB{And what are the sixteen grounds on which they rely?}}\\
\end{addmargin}
\end{absolutelynopagebreak}

\begin{absolutelynopagebreak}
\setstretch{.7}
{\PaliGlossA{‘rūpī attā hoti arogo paraṃ maraṇā saññī’ti naṃ paññapenti. (1: 19)}}\\
\begin{addmargin}[1em]{2em}
\setstretch{.5}
{\PaliGlossB{They assert: ‘The self is sound and percipient after death, and it is physical …}}\\
\end{addmargin}
\end{absolutelynopagebreak}

\begin{absolutelynopagebreak}
\setstretch{.7}
{\PaliGlossA{‘arūpī attā hoti arogo paraṃ maraṇā saññī’ti naṃ paññapenti. (2: 20)}}\\
\begin{addmargin}[1em]{2em}
\setstretch{.5}
{\PaliGlossB{non-physical …}}\\
\end{addmargin}
\end{absolutelynopagebreak}

\begin{absolutelynopagebreak}
\setstretch{.7}
{\PaliGlossA{‘rūpī ca arūpī ca attā hoti … pe …. (3: 21)}}\\
\begin{addmargin}[1em]{2em}
\setstretch{.5}
{\PaliGlossB{both physical and non-physical …}}\\
\end{addmargin}
\end{absolutelynopagebreak}

\begin{absolutelynopagebreak}
\setstretch{.7}
{\PaliGlossA{‘nevarūpī nārūpī attā hoti …. (4: 22)}}\\
\begin{addmargin}[1em]{2em}
\setstretch{.5}
{\PaliGlossB{neither physical nor non-physical …}}\\
\end{addmargin}
\end{absolutelynopagebreak}

\begin{absolutelynopagebreak}
\setstretch{.7}
{\PaliGlossA{‘antavā attā hoti …. (5: 23)}}\\
\begin{addmargin}[1em]{2em}
\setstretch{.5}
{\PaliGlossB{finite …}}\\
\end{addmargin}
\end{absolutelynopagebreak}

\begin{absolutelynopagebreak}
\setstretch{.7}
{\PaliGlossA{‘anantavā attā hoti …. (6: 24)}}\\
\begin{addmargin}[1em]{2em}
\setstretch{.5}
{\PaliGlossB{infinite …}}\\
\end{addmargin}
\end{absolutelynopagebreak}

\begin{absolutelynopagebreak}
\setstretch{.7}
{\PaliGlossA{‘antavā ca anantavā ca attā hoti …. (7: 25)}}\\
\begin{addmargin}[1em]{2em}
\setstretch{.5}
{\PaliGlossB{both finite and infinite …}}\\
\end{addmargin}
\end{absolutelynopagebreak}

\begin{absolutelynopagebreak}
\setstretch{.7}
{\PaliGlossA{‘nevantavā nānantavā attā hoti …. (8: 26)}}\\
\begin{addmargin}[1em]{2em}
\setstretch{.5}
{\PaliGlossB{neither finite nor infinite …}}\\
\end{addmargin}
\end{absolutelynopagebreak}

\begin{absolutelynopagebreak}
\setstretch{.7}
{\PaliGlossA{‘ekattasaññī attā hoti …. (9: 27)}}\\
\begin{addmargin}[1em]{2em}
\setstretch{.5}
{\PaliGlossB{of unified perception …}}\\
\end{addmargin}
\end{absolutelynopagebreak}

\begin{absolutelynopagebreak}
\setstretch{.7}
{\PaliGlossA{‘nānattasaññī attā hoti …. (10: 28)}}\\
\begin{addmargin}[1em]{2em}
\setstretch{.5}
{\PaliGlossB{of diverse perception …}}\\
\end{addmargin}
\end{absolutelynopagebreak}

\begin{absolutelynopagebreak}
\setstretch{.7}
{\PaliGlossA{‘parittasaññī attā hoti …. (11: 29)}}\\
\begin{addmargin}[1em]{2em}
\setstretch{.5}
{\PaliGlossB{of limited perception …}}\\
\end{addmargin}
\end{absolutelynopagebreak}

\begin{absolutelynopagebreak}
\setstretch{.7}
{\PaliGlossA{‘appamāṇasaññī attā hoti …. (12: 30)}}\\
\begin{addmargin}[1em]{2em}
\setstretch{.5}
{\PaliGlossB{of limitless perception …}}\\
\end{addmargin}
\end{absolutelynopagebreak}

\begin{absolutelynopagebreak}
\setstretch{.7}
{\PaliGlossA{‘ekantasukhī attā hoti …. (13: 31)}}\\
\begin{addmargin}[1em]{2em}
\setstretch{.5}
{\PaliGlossB{experiences nothing but happiness …}}\\
\end{addmargin}
\end{absolutelynopagebreak}

\begin{absolutelynopagebreak}
\setstretch{.7}
{\PaliGlossA{‘ekantadukkhī attā hoti …. (14: 32)}}\\
\begin{addmargin}[1em]{2em}
\setstretch{.5}
{\PaliGlossB{experiences nothing but suffering …}}\\
\end{addmargin}
\end{absolutelynopagebreak}

\begin{absolutelynopagebreak}
\setstretch{.7}
{\PaliGlossA{‘sukhadukkhī attā hoti …. (15: 33)}}\\
\begin{addmargin}[1em]{2em}
\setstretch{.5}
{\PaliGlossB{experiences both happiness and suffering …}}\\
\end{addmargin}
\end{absolutelynopagebreak}

\begin{absolutelynopagebreak}
\setstretch{.7}
{\PaliGlossA{‘adukkhamasukhī attā hoti arogo paraṃ maraṇā saññī’ti naṃ paññapenti. (16: 34)}}\\
\begin{addmargin}[1em]{2em}
\setstretch{.5}
{\PaliGlossB{experiences neither happiness nor suffering.’}}\\
\end{addmargin}
\end{absolutelynopagebreak}

\begin{absolutelynopagebreak}
\setstretch{.7}
{\PaliGlossA{imehi kho te, bhikkhave, samaṇabrāhmaṇā uddhamāghātanikā saññīvādā uddhamāghātanaṃ saññiṃ attānaṃ paññapenti soḷasahi vatthūhi.}}\\
\begin{addmargin}[1em]{2em}
\setstretch{.5}
{\PaliGlossB{These are the sixteen grounds on which those ascetics and brahmins assert that the self lives on after death in a percipient form.}}\\
\end{addmargin}
\end{absolutelynopagebreak}

\begin{absolutelynopagebreak}
\setstretch{.7}
{\PaliGlossA{ye hi keci, bhikkhave, samaṇā vā brāhmaṇā vā uddhamāghātanikā saññīvādā uddhamāghātanaṃ saññiṃ attānaṃ paññapenti, sabbe te imeheva soḷasahi vatthūhi, etesaṃ vā aññatarena, natthi ito bahiddhā …}}\\
\begin{addmargin}[1em]{2em}
\setstretch{.5}
{\PaliGlossB{Any ascetics and brahmins who assert that the self lives on after death in a percipient form do so on one or other of these sixteen grounds. Outside of this there is none.}}\\
\end{addmargin}
\end{absolutelynopagebreak}

\begin{absolutelynopagebreak}
\setstretch{.7}
{\PaliGlossA{pe …}}\\
\begin{addmargin}[1em]{2em}
\setstretch{.5}
{\PaliGlossB{The Realized One understands this …}}\\
\end{addmargin}
\end{absolutelynopagebreak}

\begin{absolutelynopagebreak}
\setstretch{.7}
{\PaliGlossA{yehi tathāgatassa yathābhuccaṃ vaṇṇaṃ sammā vadamānā vadeyyuṃ.}}\\
\begin{addmargin}[1em]{2em}
\setstretch{.5}
{\PaliGlossB{And those who genuinely praise the Realized One would rightly speak of these things.}}\\
\end{addmargin}
\end{absolutelynopagebreak}

\begin{absolutelynopagebreak}
\setstretch{.7}
{\PaliGlossA{3.2.2. asaññīvāda}}\\
\begin{addmargin}[1em]{2em}
\setstretch{.5}
{\PaliGlossB{3.2.2. Non-Percipient Life After Death}}\\
\end{addmargin}
\end{absolutelynopagebreak}

\begin{absolutelynopagebreak}
\setstretch{.7}
{\PaliGlossA{santi, bhikkhave, eke samaṇabrāhmaṇā uddhamāghātanikā asaññīvādā uddhamāghātanaṃ asaññiṃ attānaṃ paññapenti aṭṭhahi vatthūhi.}}\\
\begin{addmargin}[1em]{2em}
\setstretch{.5}
{\PaliGlossB{There are some ascetics and brahmins who say there is life after death, and assert that the self lives on after death in a non-percipient form on eight grounds.}}\\
\end{addmargin}
\end{absolutelynopagebreak}

\begin{absolutelynopagebreak}
\setstretch{.7}
{\PaliGlossA{te ca bhonto samaṇabrāhmaṇā kimāgamma kimārabbha uddhamāghātanikā asaññīvādā uddhamāghātanaṃ asaññiṃ attānaṃ paññapenti aṭṭhahi vatthūhi?}}\\
\begin{addmargin}[1em]{2em}
\setstretch{.5}
{\PaliGlossB{And what are the eight grounds on which they rely?}}\\
\end{addmargin}
\end{absolutelynopagebreak}

\begin{absolutelynopagebreak}
\setstretch{.7}
{\PaliGlossA{‘rūpī attā hoti arogo paraṃ maraṇā asaññī’ti naṃ paññapenti. (1: 35)}}\\
\begin{addmargin}[1em]{2em}
\setstretch{.5}
{\PaliGlossB{They assert: ‘The self is sound and non-percipient after death, and it is physical …}}\\
\end{addmargin}
\end{absolutelynopagebreak}

\begin{absolutelynopagebreak}
\setstretch{.7}
{\PaliGlossA{‘arūpī attā hoti arogo paraṃ maraṇā asaññī’ti naṃ paññapenti. (2: 36)}}\\
\begin{addmargin}[1em]{2em}
\setstretch{.5}
{\PaliGlossB{non-physical …}}\\
\end{addmargin}
\end{absolutelynopagebreak}

\begin{absolutelynopagebreak}
\setstretch{.7}
{\PaliGlossA{‘rūpī ca arūpī ca attā hoti … pe …. (3: 37)}}\\
\begin{addmargin}[1em]{2em}
\setstretch{.5}
{\PaliGlossB{both physical and non-physical …}}\\
\end{addmargin}
\end{absolutelynopagebreak}

\begin{absolutelynopagebreak}
\setstretch{.7}
{\PaliGlossA{‘nevarūpī nārūpī attā hoti …. (4: 38)}}\\
\begin{addmargin}[1em]{2em}
\setstretch{.5}
{\PaliGlossB{neither physical nor non-physical …}}\\
\end{addmargin}
\end{absolutelynopagebreak}

\begin{absolutelynopagebreak}
\setstretch{.7}
{\PaliGlossA{‘antavā attā hoti …. (5: 39)}}\\
\begin{addmargin}[1em]{2em}
\setstretch{.5}
{\PaliGlossB{finite …}}\\
\end{addmargin}
\end{absolutelynopagebreak}

\begin{absolutelynopagebreak}
\setstretch{.7}
{\PaliGlossA{‘anantavā attā hoti …. (6: 40)}}\\
\begin{addmargin}[1em]{2em}
\setstretch{.5}
{\PaliGlossB{infinite …}}\\
\end{addmargin}
\end{absolutelynopagebreak}

\begin{absolutelynopagebreak}
\setstretch{.7}
{\PaliGlossA{‘antavā ca anantavā ca attā hoti …. (7: 41)}}\\
\begin{addmargin}[1em]{2em}
\setstretch{.5}
{\PaliGlossB{both finite and infinite …}}\\
\end{addmargin}
\end{absolutelynopagebreak}

\begin{absolutelynopagebreak}
\setstretch{.7}
{\PaliGlossA{‘nevantavā nānantavā attā hoti arogo paraṃ maraṇā asaññī’ti naṃ paññapenti. (8: 42)}}\\
\begin{addmargin}[1em]{2em}
\setstretch{.5}
{\PaliGlossB{neither finite nor infinite.’}}\\
\end{addmargin}
\end{absolutelynopagebreak}

\begin{absolutelynopagebreak}
\setstretch{.7}
{\PaliGlossA{imehi kho te, bhikkhave, samaṇabrāhmaṇā uddhamāghātanikā asaññīvādā uddhamāghātanaṃ asaññiṃ attānaṃ paññapenti aṭṭhahi vatthūhi.}}\\
\begin{addmargin}[1em]{2em}
\setstretch{.5}
{\PaliGlossB{These are the eight grounds on which those ascetics and brahmins assert that the self lives on after death in a non-percipient form.}}\\
\end{addmargin}
\end{absolutelynopagebreak}

\begin{absolutelynopagebreak}
\setstretch{.7}
{\PaliGlossA{ye hi keci, bhikkhave, samaṇā vā brāhmaṇā vā uddhamāghātanikā asaññīvādā uddhamāghātanaṃ asaññiṃ attānaṃ paññapenti, sabbe te imeheva aṭṭhahi vatthūhi, etesaṃ vā aññatarena, natthi ito bahiddhā …}}\\
\begin{addmargin}[1em]{2em}
\setstretch{.5}
{\PaliGlossB{Any ascetics and brahmins who assert that the self lives on after death in a non-percipient form do so on one or other of these eight grounds. Outside of this there is none.}}\\
\end{addmargin}
\end{absolutelynopagebreak}

\begin{absolutelynopagebreak}
\setstretch{.7}
{\PaliGlossA{pe …}}\\
\begin{addmargin}[1em]{2em}
\setstretch{.5}
{\PaliGlossB{The Realized One understands this …}}\\
\end{addmargin}
\end{absolutelynopagebreak}

\begin{absolutelynopagebreak}
\setstretch{.7}
{\PaliGlossA{yehi tathāgatassa yathābhuccaṃ vaṇṇaṃ sammā vadamānā vadeyyuṃ.}}\\
\begin{addmargin}[1em]{2em}
\setstretch{.5}
{\PaliGlossB{And those who genuinely praise the Realized One would rightly speak of these things.}}\\
\end{addmargin}
\end{absolutelynopagebreak}

\begin{absolutelynopagebreak}
\setstretch{.7}
{\PaliGlossA{3.2.3. nevasaññīnāsaññīvāda}}\\
\begin{addmargin}[1em]{2em}
\setstretch{.5}
{\PaliGlossB{3.2.3. Neither Percipient Nor Non-Percipient Life After Death}}\\
\end{addmargin}
\end{absolutelynopagebreak}

\begin{absolutelynopagebreak}
\setstretch{.7}
{\PaliGlossA{santi, bhikkhave, eke samaṇabrāhmaṇā uddhamāghātanikā nevasaññīnāsaññīvādā, uddhamāghātanaṃ nevasaññīnāsaññiṃ attānaṃ paññapenti aṭṭhahi vatthūhi.}}\\
\begin{addmargin}[1em]{2em}
\setstretch{.5}
{\PaliGlossB{There are some ascetics and brahmins who say there is life after death, and assert that the self lives on after death in a neither percipient nor non-percipient form on eight grounds.}}\\
\end{addmargin}
\end{absolutelynopagebreak}

\begin{absolutelynopagebreak}
\setstretch{.7}
{\PaliGlossA{te ca bhonto samaṇabrāhmaṇā kimāgamma kimārabbha uddhamāghātanikā nevasaññīnāsaññīvādā uddhamāghātanaṃ nevasaññīnāsaññiṃ attānaṃ paññapenti aṭṭhahi vatthūhi?}}\\
\begin{addmargin}[1em]{2em}
\setstretch{.5}
{\PaliGlossB{And what are the eight grounds on which they rely?}}\\
\end{addmargin}
\end{absolutelynopagebreak}

\begin{absolutelynopagebreak}
\setstretch{.7}
{\PaliGlossA{‘rūpī attā hoti arogo paraṃ maraṇā nevasaññīnāsaññī’ti naṃ paññapenti. (1: 43)}}\\
\begin{addmargin}[1em]{2em}
\setstretch{.5}
{\PaliGlossB{They assert: ‘The self is sound and neither percipient nor non-percipient after death, and it is physical …}}\\
\end{addmargin}
\end{absolutelynopagebreak}

\begin{absolutelynopagebreak}
\setstretch{.7}
{\PaliGlossA{‘arūpī attā hoti … pe …. (2: 44)}}\\
\begin{addmargin}[1em]{2em}
\setstretch{.5}
{\PaliGlossB{non-physical …}}\\
\end{addmargin}
\end{absolutelynopagebreak}

\begin{absolutelynopagebreak}
\setstretch{.7}
{\PaliGlossA{‘rūpī ca arūpī ca attā hoti …. (3: 45)}}\\
\begin{addmargin}[1em]{2em}
\setstretch{.5}
{\PaliGlossB{both physical and non-physical …}}\\
\end{addmargin}
\end{absolutelynopagebreak}

\begin{absolutelynopagebreak}
\setstretch{.7}
{\PaliGlossA{‘nevarūpī nārūpī attā hoti …. (4: 46)}}\\
\begin{addmargin}[1em]{2em}
\setstretch{.5}
{\PaliGlossB{neither physical nor non-physical …}}\\
\end{addmargin}
\end{absolutelynopagebreak}

\begin{absolutelynopagebreak}
\setstretch{.7}
{\PaliGlossA{‘antavā attā hoti …. (5: 47)}}\\
\begin{addmargin}[1em]{2em}
\setstretch{.5}
{\PaliGlossB{finite …}}\\
\end{addmargin}
\end{absolutelynopagebreak}

\begin{absolutelynopagebreak}
\setstretch{.7}
{\PaliGlossA{‘anantavā attā hoti …. (6: 48)}}\\
\begin{addmargin}[1em]{2em}
\setstretch{.5}
{\PaliGlossB{infinite …}}\\
\end{addmargin}
\end{absolutelynopagebreak}

\begin{absolutelynopagebreak}
\setstretch{.7}
{\PaliGlossA{‘antavā ca anantavā ca attā hoti …. (7: 49)}}\\
\begin{addmargin}[1em]{2em}
\setstretch{.5}
{\PaliGlossB{both finite and infinite …}}\\
\end{addmargin}
\end{absolutelynopagebreak}

\begin{absolutelynopagebreak}
\setstretch{.7}
{\PaliGlossA{‘nevantavā nānantavā attā hoti arogo paraṃ maraṇā nevasaññīnāsaññī’ti naṃ paññapenti. (8: 50)}}\\
\begin{addmargin}[1em]{2em}
\setstretch{.5}
{\PaliGlossB{neither finite nor infinite.’}}\\
\end{addmargin}
\end{absolutelynopagebreak}

\begin{absolutelynopagebreak}
\setstretch{.7}
{\PaliGlossA{imehi kho te, bhikkhave, samaṇabrāhmaṇā uddhamāghātanikā nevasaññīnāsaññīvādā uddhamāghātanaṃ nevasaññīnāsaññiṃ attānaṃ paññapenti aṭṭhahi vatthūhi.}}\\
\begin{addmargin}[1em]{2em}
\setstretch{.5}
{\PaliGlossB{These are the eight grounds on which those ascetics and brahmins assert that the self lives on after death in a neither percipient nor non-percipient form.}}\\
\end{addmargin}
\end{absolutelynopagebreak}

\begin{absolutelynopagebreak}
\setstretch{.7}
{\PaliGlossA{ye hi keci, bhikkhave, samaṇā vā brāhmaṇā vā uddhamāghātanikā nevasaññīnāsaññīvādā uddhamāghātanaṃ nevasaññīnāsaññiṃ attānaṃ paññapenti, sabbe te imeheva aṭṭhahi vatthūhi …}}\\
\begin{addmargin}[1em]{2em}
\setstretch{.5}
{\PaliGlossB{Any ascetics and brahmins who assert that the self lives on after death in a neither percipient nor non-percipient form do so on one or other of these eight grounds. Outside of this there is none.}}\\
\end{addmargin}
\end{absolutelynopagebreak}

\begin{absolutelynopagebreak}
\setstretch{.7}
{\PaliGlossA{pe …}}\\
\begin{addmargin}[1em]{2em}
\setstretch{.5}
{\PaliGlossB{The Realized One understands this …}}\\
\end{addmargin}
\end{absolutelynopagebreak}

\begin{absolutelynopagebreak}
\setstretch{.7}
{\PaliGlossA{yehi tathāgatassa yathābhuccaṃ vaṇṇaṃ sammā vadamānā vadeyyuṃ.}}\\
\begin{addmargin}[1em]{2em}
\setstretch{.5}
{\PaliGlossB{And those who genuinely praise the Realized One would rightly speak of these things.}}\\
\end{addmargin}
\end{absolutelynopagebreak}

\begin{absolutelynopagebreak}
\setstretch{.7}
{\PaliGlossA{3.2.4. ucchedavāda}}\\
\begin{addmargin}[1em]{2em}
\setstretch{.5}
{\PaliGlossB{3.2.4. Annihilationism}}\\
\end{addmargin}
\end{absolutelynopagebreak}

\begin{absolutelynopagebreak}
\setstretch{.7}
{\PaliGlossA{santi, bhikkhave, eke samaṇabrāhmaṇā ucchedavādā sato sattassa ucchedaṃ vināsaṃ vibhavaṃ paññapenti sattahi vatthūhi.}}\\
\begin{addmargin}[1em]{2em}
\setstretch{.5}
{\PaliGlossB{There are some ascetics and brahmins who are annihilationists. They assert the annihilation, eradication, and obliteration of an existing being on seven grounds.}}\\
\end{addmargin}
\end{absolutelynopagebreak}

\begin{absolutelynopagebreak}
\setstretch{.7}
{\PaliGlossA{te ca bhonto samaṇabrāhmaṇā kimāgamma kimārabbha ucchedavādā sato sattassa ucchedaṃ vināsaṃ vibhavaṃ paññapenti sattahi vatthūhi?}}\\
\begin{addmargin}[1em]{2em}
\setstretch{.5}
{\PaliGlossB{And what are the seven grounds on which they rely?}}\\
\end{addmargin}
\end{absolutelynopagebreak}

\begin{absolutelynopagebreak}
\setstretch{.7}
{\PaliGlossA{idha, bhikkhave, ekacco samaṇo vā brāhmaṇo vā evaṃvādī hoti evaṃdiṭṭhi:}}\\
\begin{addmargin}[1em]{2em}
\setstretch{.5}
{\PaliGlossB{There are some ascetics and brahmins who have this doctrine and view:}}\\
\end{addmargin}
\end{absolutelynopagebreak}

\begin{absolutelynopagebreak}
\setstretch{.7}
{\PaliGlossA{‘yato kho, bho, ayaṃ attā rūpī cātumahābhūtiko mātāpettikasambhavo kāyassa bhedā ucchijjati vinassati, na hoti paraṃ maraṇā, ettāvatā kho, bho, ayaṃ attā sammā samucchinno hotī’ti.}}\\
\begin{addmargin}[1em]{2em}
\setstretch{.5}
{\PaliGlossB{‘This self is physical, made up of the four primary elements, and produced by mother and father. Since it’s annihilated and destroyed when the body breaks up, and doesn’t exist after death, that’s how this self becomes rightly annihilated.’}}\\
\end{addmargin}
\end{absolutelynopagebreak}

\begin{absolutelynopagebreak}
\setstretch{.7}
{\PaliGlossA{ittheke sato sattassa ucchedaṃ vināsaṃ vibhavaṃ paññapenti. (1: 51)}}\\
\begin{addmargin}[1em]{2em}
\setstretch{.5}
{\PaliGlossB{That is how some assert the annihilation of an existing being.}}\\
\end{addmargin}
\end{absolutelynopagebreak}

\begin{absolutelynopagebreak}
\setstretch{.7}
{\PaliGlossA{tamañño evamāha:}}\\
\begin{addmargin}[1em]{2em}
\setstretch{.5}
{\PaliGlossB{But someone else says to them:}}\\
\end{addmargin}
\end{absolutelynopagebreak}

\begin{absolutelynopagebreak}
\setstretch{.7}
{\PaliGlossA{‘atthi kho, bho, eso attā, yaṃ tvaṃ vadesi, neso natthīti vadāmi;}}\\
\begin{addmargin}[1em]{2em}
\setstretch{.5}
{\PaliGlossB{‘*That* self of which you speak does exist, I don’t deny it.}}\\
\end{addmargin}
\end{absolutelynopagebreak}

\begin{absolutelynopagebreak}
\setstretch{.7}
{\PaliGlossA{no ca kho, bho, ayaṃ attā ettāvatā sammā samucchinno hoti.}}\\
\begin{addmargin}[1em]{2em}
\setstretch{.5}
{\PaliGlossB{But that’s not how *this* self becomes rightly annihilated.}}\\
\end{addmargin}
\end{absolutelynopagebreak}

\begin{absolutelynopagebreak}
\setstretch{.7}
{\PaliGlossA{atthi kho, bho, añño attā dibbo rūpī kāmāvacaro kabaḷīkārāhārabhakkho.}}\\
\begin{addmargin}[1em]{2em}
\setstretch{.5}
{\PaliGlossB{There is another self that is divine, physical, sensual, consuming solid food.}}\\
\end{addmargin}
\end{absolutelynopagebreak}

\begin{absolutelynopagebreak}
\setstretch{.7}
{\PaliGlossA{taṃ tvaṃ na jānāsi na passasi.}}\\
\begin{addmargin}[1em]{2em}
\setstretch{.5}
{\PaliGlossB{You don’t know or see that.}}\\
\end{addmargin}
\end{absolutelynopagebreak}

\begin{absolutelynopagebreak}
\setstretch{.7}
{\PaliGlossA{tamahaṃ jānāmi passāmi.}}\\
\begin{addmargin}[1em]{2em}
\setstretch{.5}
{\PaliGlossB{But I know it and see it.}}\\
\end{addmargin}
\end{absolutelynopagebreak}

\begin{absolutelynopagebreak}
\setstretch{.7}
{\PaliGlossA{so kho, bho, attā yato kāyassa bhedā ucchijjati vinassati, na hoti paraṃ maraṇā, ettāvatā kho, bho, ayaṃ attā sammā samucchinno hotī’ti.}}\\
\begin{addmargin}[1em]{2em}
\setstretch{.5}
{\PaliGlossB{Since this self is annihilated and destroyed when the body breaks up, and doesn’t exist after death, that’s how this self becomes rightly annihilated.’}}\\
\end{addmargin}
\end{absolutelynopagebreak}

\begin{absolutelynopagebreak}
\setstretch{.7}
{\PaliGlossA{ittheke sato sattassa ucchedaṃ vināsaṃ vibhavaṃ paññapenti. (2: 52)}}\\
\begin{addmargin}[1em]{2em}
\setstretch{.5}
{\PaliGlossB{That is how some assert the annihilation of an existing being.}}\\
\end{addmargin}
\end{absolutelynopagebreak}

\begin{absolutelynopagebreak}
\setstretch{.7}
{\PaliGlossA{tamañño evamāha:}}\\
\begin{addmargin}[1em]{2em}
\setstretch{.5}
{\PaliGlossB{But someone else says to them:}}\\
\end{addmargin}
\end{absolutelynopagebreak}

\begin{absolutelynopagebreak}
\setstretch{.7}
{\PaliGlossA{‘atthi kho, bho, eso attā, yaṃ tvaṃ vadesi, neso natthīti vadāmi;}}\\
\begin{addmargin}[1em]{2em}
\setstretch{.5}
{\PaliGlossB{‘*That* self of which you speak does exist, I don’t deny it.}}\\
\end{addmargin}
\end{absolutelynopagebreak}

\begin{absolutelynopagebreak}
\setstretch{.7}
{\PaliGlossA{no ca kho, bho, ayaṃ attā ettāvatā sammā samucchinno hoti.}}\\
\begin{addmargin}[1em]{2em}
\setstretch{.5}
{\PaliGlossB{But that’s not how *this* self becomes rightly annihilated.}}\\
\end{addmargin}
\end{absolutelynopagebreak}

\begin{absolutelynopagebreak}
\setstretch{.7}
{\PaliGlossA{atthi kho, bho, añño attā dibbo rūpī manomayo sabbaṅgapaccaṅgī ahīnindriyo.}}\\
\begin{addmargin}[1em]{2em}
\setstretch{.5}
{\PaliGlossB{There is another self that is divine, physical, mind-made, complete in all its various parts, not deficient in any faculty.}}\\
\end{addmargin}
\end{absolutelynopagebreak}

\begin{absolutelynopagebreak}
\setstretch{.7}
{\PaliGlossA{taṃ tvaṃ na jānāsi na passasi.}}\\
\begin{addmargin}[1em]{2em}
\setstretch{.5}
{\PaliGlossB{You don’t know or see that.}}\\
\end{addmargin}
\end{absolutelynopagebreak}

\begin{absolutelynopagebreak}
\setstretch{.7}
{\PaliGlossA{tamahaṃ jānāmi passāmi.}}\\
\begin{addmargin}[1em]{2em}
\setstretch{.5}
{\PaliGlossB{But I know it and see it.}}\\
\end{addmargin}
\end{absolutelynopagebreak}

\begin{absolutelynopagebreak}
\setstretch{.7}
{\PaliGlossA{so kho, bho, attā yato kāyassa bhedā ucchijjati vinassati, na hoti paraṃ maraṇā, ettāvatā kho, bho, ayaṃ attā sammā samucchinno hotī’ti.}}\\
\begin{addmargin}[1em]{2em}
\setstretch{.5}
{\PaliGlossB{Since this self is annihilated and destroyed when the body breaks up, and doesn’t exist after death, that’s how this self becomes rightly annihilated.’}}\\
\end{addmargin}
\end{absolutelynopagebreak}

\begin{absolutelynopagebreak}
\setstretch{.7}
{\PaliGlossA{ittheke sato sattassa ucchedaṃ vināsaṃ vibhavaṃ paññapenti. (3: 53)}}\\
\begin{addmargin}[1em]{2em}
\setstretch{.5}
{\PaliGlossB{That is how some assert the annihilation of an existing being.}}\\
\end{addmargin}
\end{absolutelynopagebreak}

\begin{absolutelynopagebreak}
\setstretch{.7}
{\PaliGlossA{tamañño evamāha:}}\\
\begin{addmargin}[1em]{2em}
\setstretch{.5}
{\PaliGlossB{But someone else says to them:}}\\
\end{addmargin}
\end{absolutelynopagebreak}

\begin{absolutelynopagebreak}
\setstretch{.7}
{\PaliGlossA{‘atthi kho, bho, eso attā, yaṃ tvaṃ vadesi, neso natthīti vadāmi;}}\\
\begin{addmargin}[1em]{2em}
\setstretch{.5}
{\PaliGlossB{‘*That* self of which you speak does exist, I don’t deny it.}}\\
\end{addmargin}
\end{absolutelynopagebreak}

\begin{absolutelynopagebreak}
\setstretch{.7}
{\PaliGlossA{no ca kho, bho, ayaṃ attā ettāvatā sammā samucchinno hoti.}}\\
\begin{addmargin}[1em]{2em}
\setstretch{.5}
{\PaliGlossB{But that’s not how *this* self becomes rightly annihilated.}}\\
\end{addmargin}
\end{absolutelynopagebreak}

\begin{absolutelynopagebreak}
\setstretch{.7}
{\PaliGlossA{atthi kho, bho, añño attā sabbaso rūpasaññānaṃ samatikkamā paṭighasaññānaṃ atthaṅgamā nānattasaññānaṃ amanasikārā “ananto ākāso”ti ākāsānañcāyatanūpago.}}\\
\begin{addmargin}[1em]{2em}
\setstretch{.5}
{\PaliGlossB{There is another self which has gone totally beyond perceptions of form. With the ending of perceptions of impingement, not focusing on perceptions of diversity, aware that “space is infinite”, it’s reborn in the dimension of infinite space.}}\\
\end{addmargin}
\end{absolutelynopagebreak}

\begin{absolutelynopagebreak}
\setstretch{.7}
{\PaliGlossA{taṃ tvaṃ na jānāsi na passasi.}}\\
\begin{addmargin}[1em]{2em}
\setstretch{.5}
{\PaliGlossB{You don’t know or see that.}}\\
\end{addmargin}
\end{absolutelynopagebreak}

\begin{absolutelynopagebreak}
\setstretch{.7}
{\PaliGlossA{tamahaṃ jānāmi passāmi.}}\\
\begin{addmargin}[1em]{2em}
\setstretch{.5}
{\PaliGlossB{But I know it and see it.}}\\
\end{addmargin}
\end{absolutelynopagebreak}

\begin{absolutelynopagebreak}
\setstretch{.7}
{\PaliGlossA{so kho, bho, attā yato kāyassa bhedā ucchijjati vinassati, na hoti paraṃ maraṇā, ettāvatā kho, bho, ayaṃ attā sammā samucchinno hotī’ti.}}\\
\begin{addmargin}[1em]{2em}
\setstretch{.5}
{\PaliGlossB{Since this self is annihilated and destroyed when the body breaks up, and doesn’t exist after death, that’s how this self becomes rightly annihilated.’}}\\
\end{addmargin}
\end{absolutelynopagebreak}

\begin{absolutelynopagebreak}
\setstretch{.7}
{\PaliGlossA{ittheke sato sattassa ucchedaṃ vināsaṃ vibhavaṃ paññapenti. (4: 54)}}\\
\begin{addmargin}[1em]{2em}
\setstretch{.5}
{\PaliGlossB{That is how some assert the annihilation of an existing being.}}\\
\end{addmargin}
\end{absolutelynopagebreak}

\begin{absolutelynopagebreak}
\setstretch{.7}
{\PaliGlossA{tamañño evamāha:}}\\
\begin{addmargin}[1em]{2em}
\setstretch{.5}
{\PaliGlossB{But someone else says to them:}}\\
\end{addmargin}
\end{absolutelynopagebreak}

\begin{absolutelynopagebreak}
\setstretch{.7}
{\PaliGlossA{‘atthi kho, bho, eso attā yaṃ tvaṃ vadesi, neso natthīti vadāmi;}}\\
\begin{addmargin}[1em]{2em}
\setstretch{.5}
{\PaliGlossB{‘*That* self of which you speak does exist, I don’t deny it.}}\\
\end{addmargin}
\end{absolutelynopagebreak}

\begin{absolutelynopagebreak}
\setstretch{.7}
{\PaliGlossA{no ca kho, bho, ayaṃ attā ettāvatā sammā samucchinno hoti.}}\\
\begin{addmargin}[1em]{2em}
\setstretch{.5}
{\PaliGlossB{But that’s not how *this* self becomes rightly annihilated.}}\\
\end{addmargin}
\end{absolutelynopagebreak}

\begin{absolutelynopagebreak}
\setstretch{.7}
{\PaliGlossA{atthi kho, bho, añño attā sabbaso ākāsānañcāyatanaṃ samatikkamma “anantaṃ viññāṇan”ti viññāṇañcāyatanūpago.}}\\
\begin{addmargin}[1em]{2em}
\setstretch{.5}
{\PaliGlossB{There is another self which has gone totally beyond the dimension of infinite space. Aware that “consciousness is infinite”, it’s reborn in the dimension of infinite consciousness.}}\\
\end{addmargin}
\end{absolutelynopagebreak}

\begin{absolutelynopagebreak}
\setstretch{.7}
{\PaliGlossA{taṃ tvaṃ na jānāsi na passasi.}}\\
\begin{addmargin}[1em]{2em}
\setstretch{.5}
{\PaliGlossB{You don’t know or see that.}}\\
\end{addmargin}
\end{absolutelynopagebreak}

\begin{absolutelynopagebreak}
\setstretch{.7}
{\PaliGlossA{tamahaṃ jānāmi passāmi.}}\\
\begin{addmargin}[1em]{2em}
\setstretch{.5}
{\PaliGlossB{But I know it and see it.}}\\
\end{addmargin}
\end{absolutelynopagebreak}

\begin{absolutelynopagebreak}
\setstretch{.7}
{\PaliGlossA{so kho, bho, attā yato kāyassa bhedā ucchijjati vinassati, na hoti paraṃ maraṇā, ettāvatā kho, bho, ayaṃ attā sammā samucchinno hotī’ti.}}\\
\begin{addmargin}[1em]{2em}
\setstretch{.5}
{\PaliGlossB{Since this self is annihilated and destroyed when the body breaks up, and doesn’t exist after death, that’s how this self becomes rightly annihilated.’}}\\
\end{addmargin}
\end{absolutelynopagebreak}

\begin{absolutelynopagebreak}
\setstretch{.7}
{\PaliGlossA{ittheke sato sattassa ucchedaṃ vināsaṃ vibhavaṃ paññapenti. (5: 55)}}\\
\begin{addmargin}[1em]{2em}
\setstretch{.5}
{\PaliGlossB{That is how some assert the annihilation of an existing being.}}\\
\end{addmargin}
\end{absolutelynopagebreak}

\begin{absolutelynopagebreak}
\setstretch{.7}
{\PaliGlossA{tamañño evamāha:}}\\
\begin{addmargin}[1em]{2em}
\setstretch{.5}
{\PaliGlossB{But someone else says to them:}}\\
\end{addmargin}
\end{absolutelynopagebreak}

\begin{absolutelynopagebreak}
\setstretch{.7}
{\PaliGlossA{‘atthi kho, bho, so attā, yaṃ tvaṃ vadesi, neso natthīti vadāmi;}}\\
\begin{addmargin}[1em]{2em}
\setstretch{.5}
{\PaliGlossB{‘*That* self of which you speak does exist, I don’t deny it.}}\\
\end{addmargin}
\end{absolutelynopagebreak}

\begin{absolutelynopagebreak}
\setstretch{.7}
{\PaliGlossA{no ca kho, bho, ayaṃ attā ettāvatā sammā samucchinno hoti.}}\\
\begin{addmargin}[1em]{2em}
\setstretch{.5}
{\PaliGlossB{But that’s not how *this* self becomes rightly annihilated.}}\\
\end{addmargin}
\end{absolutelynopagebreak}

\begin{absolutelynopagebreak}
\setstretch{.7}
{\PaliGlossA{atthi kho, bho, añño attā sabbaso viññāṇañcāyatanaṃ samatikkamma “natthi kiñcī”ti ākiñcaññāyatanūpago.}}\\
\begin{addmargin}[1em]{2em}
\setstretch{.5}
{\PaliGlossB{There is another self that has gone totally beyond the dimension of infinite consciousness. Aware that “there is nothing at all”, it’s been reborn in the dimension of nothingness.}}\\
\end{addmargin}
\end{absolutelynopagebreak}

\begin{absolutelynopagebreak}
\setstretch{.7}
{\PaliGlossA{taṃ tvaṃ na jānāsi na passasi.}}\\
\begin{addmargin}[1em]{2em}
\setstretch{.5}
{\PaliGlossB{You don’t know or see that.}}\\
\end{addmargin}
\end{absolutelynopagebreak}

\begin{absolutelynopagebreak}
\setstretch{.7}
{\PaliGlossA{tamahaṃ jānāmi passāmi.}}\\
\begin{addmargin}[1em]{2em}
\setstretch{.5}
{\PaliGlossB{But I know it and see it.}}\\
\end{addmargin}
\end{absolutelynopagebreak}

\begin{absolutelynopagebreak}
\setstretch{.7}
{\PaliGlossA{so kho, bho, attā yato kāyassa bhedā ucchijjati vinassati, na hoti paraṃ maraṇā, ettāvatā kho, bho, ayaṃ attā sammā samucchinno hotī’ti.}}\\
\begin{addmargin}[1em]{2em}
\setstretch{.5}
{\PaliGlossB{Since this self is annihilated and destroyed when the body breaks up, and doesn’t exist after death, that’s how this self becomes rightly annihilated.’}}\\
\end{addmargin}
\end{absolutelynopagebreak}

\begin{absolutelynopagebreak}
\setstretch{.7}
{\PaliGlossA{ittheke sato sattassa ucchedaṃ vināsaṃ vibhavaṃ paññapenti. (6: 56)}}\\
\begin{addmargin}[1em]{2em}
\setstretch{.5}
{\PaliGlossB{That is how some assert the annihilation of an existing being.}}\\
\end{addmargin}
\end{absolutelynopagebreak}

\begin{absolutelynopagebreak}
\setstretch{.7}
{\PaliGlossA{tamañño evamāha:}}\\
\begin{addmargin}[1em]{2em}
\setstretch{.5}
{\PaliGlossB{But someone else says to them:}}\\
\end{addmargin}
\end{absolutelynopagebreak}

\begin{absolutelynopagebreak}
\setstretch{.7}
{\PaliGlossA{‘atthi kho, bho, eso attā, yaṃ tvaṃ vadesi, neso natthīti vadāmi;}}\\
\begin{addmargin}[1em]{2em}
\setstretch{.5}
{\PaliGlossB{‘*That* self of which you speak does exist, I don’t deny it.}}\\
\end{addmargin}
\end{absolutelynopagebreak}

\begin{absolutelynopagebreak}
\setstretch{.7}
{\PaliGlossA{no ca kho, bho, ayaṃ attā ettāvatā sammā samucchinno hoti.}}\\
\begin{addmargin}[1em]{2em}
\setstretch{.5}
{\PaliGlossB{But that’s not how *this* self becomes rightly annihilated.}}\\
\end{addmargin}
\end{absolutelynopagebreak}

\begin{absolutelynopagebreak}
\setstretch{.7}
{\PaliGlossA{atthi kho, bho, añño attā sabbaso ākiñcaññāyatanaṃ samatikkamma “santametaṃ paṇītametan”ti nevasaññānāsaññāyatanūpago.}}\\
\begin{addmargin}[1em]{2em}
\setstretch{.5}
{\PaliGlossB{There is another self that has gone totally beyond the dimension of nothingness. Aware that “this is peaceful, this is sublime”, it’s been reborn in the dimension of neither perception nor non-perception.}}\\
\end{addmargin}
\end{absolutelynopagebreak}

\begin{absolutelynopagebreak}
\setstretch{.7}
{\PaliGlossA{taṃ tvaṃ na jānāsi na passasi.}}\\
\begin{addmargin}[1em]{2em}
\setstretch{.5}
{\PaliGlossB{You don’t know or see that.}}\\
\end{addmargin}
\end{absolutelynopagebreak}

\begin{absolutelynopagebreak}
\setstretch{.7}
{\PaliGlossA{tamahaṃ jānāmi passāmi.}}\\
\begin{addmargin}[1em]{2em}
\setstretch{.5}
{\PaliGlossB{But I know it and see it.}}\\
\end{addmargin}
\end{absolutelynopagebreak}

\begin{absolutelynopagebreak}
\setstretch{.7}
{\PaliGlossA{so kho, bho, attā yato kāyassa bhedā ucchijjati vinassati, na hoti paraṃ maraṇā, ettāvatā kho, bho, ayaṃ attā sammā samucchinno hotī’ti.}}\\
\begin{addmargin}[1em]{2em}
\setstretch{.5}
{\PaliGlossB{Since this self is annihilated and destroyed when the body breaks up, and doesn’t exist after death, that’s how this self becomes rightly annihilated.’}}\\
\end{addmargin}
\end{absolutelynopagebreak}

\begin{absolutelynopagebreak}
\setstretch{.7}
{\PaliGlossA{ittheke sato sattassa ucchedaṃ vināsaṃ vibhavaṃ paññapenti. (7: 57)}}\\
\begin{addmargin}[1em]{2em}
\setstretch{.5}
{\PaliGlossB{That is how some assert the annihilation of an existing being.}}\\
\end{addmargin}
\end{absolutelynopagebreak}

\begin{absolutelynopagebreak}
\setstretch{.7}
{\PaliGlossA{imehi kho te, bhikkhave, samaṇabrāhmaṇā ucchedavādā sato sattassa ucchedaṃ vināsaṃ vibhavaṃ paññapenti sattahi vatthūhi.}}\\
\begin{addmargin}[1em]{2em}
\setstretch{.5}
{\PaliGlossB{These are the seven grounds on which those ascetics and brahmins assert the annihilation, eradication, and obliteration of an existing being.}}\\
\end{addmargin}
\end{absolutelynopagebreak}

\begin{absolutelynopagebreak}
\setstretch{.7}
{\PaliGlossA{ye hi keci, bhikkhave, samaṇā vā brāhmaṇā vā ucchedavādā sato sattassa ucchedaṃ vināsaṃ vibhavaṃ paññapenti, sabbe te imeheva sattahi vatthūhi …}}\\
\begin{addmargin}[1em]{2em}
\setstretch{.5}
{\PaliGlossB{Any ascetics and brahmins who assert the annihilation, eradication, and obliteration of an existing being do so on one or other of these seven grounds. Outside of this there is none.}}\\
\end{addmargin}
\end{absolutelynopagebreak}

\begin{absolutelynopagebreak}
\setstretch{.7}
{\PaliGlossA{pe …}}\\
\begin{addmargin}[1em]{2em}
\setstretch{.5}
{\PaliGlossB{The Realized One understands this …}}\\
\end{addmargin}
\end{absolutelynopagebreak}

\begin{absolutelynopagebreak}
\setstretch{.7}
{\PaliGlossA{yehi tathāgatassa yathābhuccaṃ vaṇṇaṃ sammā vadamānā vadeyyuṃ.}}\\
\begin{addmargin}[1em]{2em}
\setstretch{.5}
{\PaliGlossB{And those who genuinely praise the Realized One would rightly speak of these things.}}\\
\end{addmargin}
\end{absolutelynopagebreak}

\begin{absolutelynopagebreak}
\setstretch{.7}
{\PaliGlossA{3.2.5. diṭṭhadhammanibbānavāda}}\\
\begin{addmargin}[1em]{2em}
\setstretch{.5}
{\PaliGlossB{3.2.5. Extinguishment in the Present Life}}\\
\end{addmargin}
\end{absolutelynopagebreak}

\begin{absolutelynopagebreak}
\setstretch{.7}
{\PaliGlossA{santi, bhikkhave, eke samaṇabrāhmaṇā diṭṭhadhammanibbānavādā sato sattassa paramadiṭṭhadhammanibbānaṃ paññapenti pañcahi vatthūhi.}}\\
\begin{addmargin}[1em]{2em}
\setstretch{.5}
{\PaliGlossB{There are some ascetics and brahmins who speak of extinguishment in the present life. They assert the ultimate extinguishment of an existing being in the present life on five grounds.}}\\
\end{addmargin}
\end{absolutelynopagebreak}

\begin{absolutelynopagebreak}
\setstretch{.7}
{\PaliGlossA{te ca bhonto samaṇabrāhmaṇā kimāgamma kimārabbha diṭṭhadhammanibbānavādā sato sattassa paramadiṭṭhadhammanibbānaṃ paññapenti pañcahi vatthūhi?}}\\
\begin{addmargin}[1em]{2em}
\setstretch{.5}
{\PaliGlossB{And what are the five grounds on which they rely?}}\\
\end{addmargin}
\end{absolutelynopagebreak}

\begin{absolutelynopagebreak}
\setstretch{.7}
{\PaliGlossA{idha, bhikkhave, ekacco samaṇo vā brāhmaṇo vā evaṃvādī hoti evaṃdiṭṭhi:}}\\
\begin{addmargin}[1em]{2em}
\setstretch{.5}
{\PaliGlossB{There are some ascetics and brahmins who have this doctrine and view:}}\\
\end{addmargin}
\end{absolutelynopagebreak}

\begin{absolutelynopagebreak}
\setstretch{.7}
{\PaliGlossA{‘yato kho, bho, ayaṃ attā pañcahi kāmaguṇehi samappito samaṅgībhūto paricāreti, ettāvatā kho, bho, ayaṃ attā paramadiṭṭhadhammanibbānaṃ patto hotī’ti.}}\\
\begin{addmargin}[1em]{2em}
\setstretch{.5}
{\PaliGlossB{‘When this self amuses itself, supplied and provided with the five kinds of sensual stimulation, that’s how this self attains ultimate extinguishment in the present life.’}}\\
\end{addmargin}
\end{absolutelynopagebreak}

\begin{absolutelynopagebreak}
\setstretch{.7}
{\PaliGlossA{ittheke sato sattassa paramadiṭṭhadhammanibbānaṃ paññapenti. (1: 58)}}\\
\begin{addmargin}[1em]{2em}
\setstretch{.5}
{\PaliGlossB{That is how some assert the extinguishment of an existing being in the present life.}}\\
\end{addmargin}
\end{absolutelynopagebreak}

\begin{absolutelynopagebreak}
\setstretch{.7}
{\PaliGlossA{tamañño evamāha:}}\\
\begin{addmargin}[1em]{2em}
\setstretch{.5}
{\PaliGlossB{But someone else says to them:}}\\
\end{addmargin}
\end{absolutelynopagebreak}

\begin{absolutelynopagebreak}
\setstretch{.7}
{\PaliGlossA{‘atthi kho, bho, eso attā, yaṃ tvaṃ vadesi, neso natthīti vadāmi;}}\\
\begin{addmargin}[1em]{2em}
\setstretch{.5}
{\PaliGlossB{‘*That* self of which you speak does exist, I don’t deny it.}}\\
\end{addmargin}
\end{absolutelynopagebreak}

\begin{absolutelynopagebreak}
\setstretch{.7}
{\PaliGlossA{no ca kho, bho, ayaṃ attā ettāvatā paramadiṭṭhadhammanibbānaṃ patto hoti.}}\\
\begin{addmargin}[1em]{2em}
\setstretch{.5}
{\PaliGlossB{But that’s not how *this* self attains ultimate extinguishment in the present life.}}\\
\end{addmargin}
\end{absolutelynopagebreak}

\begin{absolutelynopagebreak}
\setstretch{.7}
{\PaliGlossA{taṃ kissa hetu?}}\\
\begin{addmargin}[1em]{2em}
\setstretch{.5}
{\PaliGlossB{Why is that?}}\\
\end{addmargin}
\end{absolutelynopagebreak}

\begin{absolutelynopagebreak}
\setstretch{.7}
{\PaliGlossA{kāmā hi, bho, aniccā dukkhā vipariṇāmadhammā, tesaṃ vipariṇāmaññathābhāvā uppajjanti sokaparidevadukkhadomanassupāyāsā.}}\\
\begin{addmargin}[1em]{2em}
\setstretch{.5}
{\PaliGlossB{Because sensual pleasures are impermanent, suffering, and perishable. Their decay and perishing give rise to sorrow, lamentation, pain, sadness, and distress.}}\\
\end{addmargin}
\end{absolutelynopagebreak}

\begin{absolutelynopagebreak}
\setstretch{.7}
{\PaliGlossA{yato kho, bho, ayaṃ attā vivicceva kāmehi vivicca akusalehi dhammehi savitakkaṃ savicāraṃ vivekajaṃ pītisukhaṃ paṭhamaṃ jhānaṃ upasampajja viharati, ettāvatā kho, bho, ayaṃ attā paramadiṭṭhadhammanibbānaṃ patto hotī’ti.}}\\
\begin{addmargin}[1em]{2em}
\setstretch{.5}
{\PaliGlossB{Quite secluded from sensual pleasures, secluded from unskillful qualities, this self enters and remains in the first absorption, which has the rapture and bliss born of seclusion, while placing the mind and keeping it connected. That’s how this self attains ultimate extinguishment in the present life.’}}\\
\end{addmargin}
\end{absolutelynopagebreak}

\begin{absolutelynopagebreak}
\setstretch{.7}
{\PaliGlossA{ittheke sato sattassa paramadiṭṭhadhammanibbānaṃ paññapenti. (2: 59)}}\\
\begin{addmargin}[1em]{2em}
\setstretch{.5}
{\PaliGlossB{That is how some assert the extinguishment of an existing being in the present life.}}\\
\end{addmargin}
\end{absolutelynopagebreak}

\begin{absolutelynopagebreak}
\setstretch{.7}
{\PaliGlossA{tamañño evamāha:}}\\
\begin{addmargin}[1em]{2em}
\setstretch{.5}
{\PaliGlossB{But someone else says to them:}}\\
\end{addmargin}
\end{absolutelynopagebreak}

\begin{absolutelynopagebreak}
\setstretch{.7}
{\PaliGlossA{‘atthi kho, bho, eso attā, yaṃ tvaṃ vadesi, neso natthīti vadāmi;}}\\
\begin{addmargin}[1em]{2em}
\setstretch{.5}
{\PaliGlossB{‘*That* self of which you speak does exist, I don’t deny it.}}\\
\end{addmargin}
\end{absolutelynopagebreak}

\begin{absolutelynopagebreak}
\setstretch{.7}
{\PaliGlossA{no ca kho, bho, ayaṃ attā ettāvatā paramadiṭṭhadhammanibbānaṃ patto hoti.}}\\
\begin{addmargin}[1em]{2em}
\setstretch{.5}
{\PaliGlossB{But that’s not how *this* self attains ultimate extinguishment in the present life.}}\\
\end{addmargin}
\end{absolutelynopagebreak}

\begin{absolutelynopagebreak}
\setstretch{.7}
{\PaliGlossA{taṃ kissa hetu?}}\\
\begin{addmargin}[1em]{2em}
\setstretch{.5}
{\PaliGlossB{Why is that?}}\\
\end{addmargin}
\end{absolutelynopagebreak}

\begin{absolutelynopagebreak}
\setstretch{.7}
{\PaliGlossA{yadeva tattha vitakkitaṃ vicāritaṃ, etenetaṃ oḷārikaṃ akkhāyati.}}\\
\begin{addmargin}[1em]{2em}
\setstretch{.5}
{\PaliGlossB{Because the placing of the mind and the keeping it connected there are coarse.}}\\
\end{addmargin}
\end{absolutelynopagebreak}

\begin{absolutelynopagebreak}
\setstretch{.7}
{\PaliGlossA{yato kho, bho, ayaṃ attā vitakkavicārānaṃ vūpasamā ajjhattaṃ sampasādanaṃ cetaso ekodibhāvaṃ avitakkaṃ avicāraṃ samādhijaṃ pītisukhaṃ dutiyaṃ jhānaṃ upasampajja viharati, ettāvatā kho, bho, ayaṃ attā paramadiṭṭhadhammanibbānaṃ patto hotī’ti.}}\\
\begin{addmargin}[1em]{2em}
\setstretch{.5}
{\PaliGlossB{But when the placing of the mind and keeping it connected are stilled, this self enters and remains in the second absorption, which has the rapture and bliss born of immersion, with internal clarity and confidence, and unified mind, without placing the mind and keeping it connected. That’s how this self attains ultimate extinguishment in the present life.’}}\\
\end{addmargin}
\end{absolutelynopagebreak}

\begin{absolutelynopagebreak}
\setstretch{.7}
{\PaliGlossA{ittheke sato sattassa paramadiṭṭhadhammanibbānaṃ paññapenti. (3: 60)}}\\
\begin{addmargin}[1em]{2em}
\setstretch{.5}
{\PaliGlossB{That is how some assert the extinguishment of an existing being in the present life.}}\\
\end{addmargin}
\end{absolutelynopagebreak}

\begin{absolutelynopagebreak}
\setstretch{.7}
{\PaliGlossA{tamañño evamāha:}}\\
\begin{addmargin}[1em]{2em}
\setstretch{.5}
{\PaliGlossB{But someone else says to them:}}\\
\end{addmargin}
\end{absolutelynopagebreak}

\begin{absolutelynopagebreak}
\setstretch{.7}
{\PaliGlossA{‘atthi kho, bho, eso attā, yaṃ tvaṃ vadesi, neso natthīti vadāmi;}}\\
\begin{addmargin}[1em]{2em}
\setstretch{.5}
{\PaliGlossB{‘*That* self of which you speak does exist, I don’t deny it.}}\\
\end{addmargin}
\end{absolutelynopagebreak}

\begin{absolutelynopagebreak}
\setstretch{.7}
{\PaliGlossA{no ca kho, bho, ayaṃ attā ettāvatā paramadiṭṭhadhammanibbānaṃ patto hoti.}}\\
\begin{addmargin}[1em]{2em}
\setstretch{.5}
{\PaliGlossB{But that’s not how *this* self attains ultimate extinguishment in the present life.}}\\
\end{addmargin}
\end{absolutelynopagebreak}

\begin{absolutelynopagebreak}
\setstretch{.7}
{\PaliGlossA{taṃ kissa hetu?}}\\
\begin{addmargin}[1em]{2em}
\setstretch{.5}
{\PaliGlossB{Why is that?}}\\
\end{addmargin}
\end{absolutelynopagebreak}

\begin{absolutelynopagebreak}
\setstretch{.7}
{\PaliGlossA{yadeva tattha pītigataṃ cetaso uppilāvitattaṃ, etenetaṃ oḷārikaṃ akkhāyati.}}\\
\begin{addmargin}[1em]{2em}
\setstretch{.5}
{\PaliGlossB{Because the rapture and emotional excitement there are coarse.}}\\
\end{addmargin}
\end{absolutelynopagebreak}

\begin{absolutelynopagebreak}
\setstretch{.7}
{\PaliGlossA{yato kho, bho, ayaṃ attā pītiyā ca virāgā upekkhako ca viharati, sato ca sampajāno, sukhañca kāyena paṭisaṃvedeti, yaṃ taṃ ariyā ācikkhanti “upekkhako satimā sukhavihārī”ti, tatiyaṃ jhānaṃ upasampajja viharati, ettāvatā kho, bho, ayaṃ attā paramadiṭṭhadhammanibbānaṃ patto hotī’ti.}}\\
\begin{addmargin}[1em]{2em}
\setstretch{.5}
{\PaliGlossB{But with the fading away of rapture, this self enters and remains in the third absorption, where it meditates with equanimity, mindful and aware, personally experiencing the bliss of which the noble ones declare, “Equanimous and mindful, one meditates in bliss”. That’s how this self attains ultimate extinguishment in the present life.’}}\\
\end{addmargin}
\end{absolutelynopagebreak}

\begin{absolutelynopagebreak}
\setstretch{.7}
{\PaliGlossA{ittheke sato sattassa paramadiṭṭhadhammanibbānaṃ paññapenti. (4: 61)}}\\
\begin{addmargin}[1em]{2em}
\setstretch{.5}
{\PaliGlossB{That is how some assert the extinguishment of an existing being in the present life.}}\\
\end{addmargin}
\end{absolutelynopagebreak}

\begin{absolutelynopagebreak}
\setstretch{.7}
{\PaliGlossA{tamañño evamāha:}}\\
\begin{addmargin}[1em]{2em}
\setstretch{.5}
{\PaliGlossB{But someone else says to them:}}\\
\end{addmargin}
\end{absolutelynopagebreak}

\begin{absolutelynopagebreak}
\setstretch{.7}
{\PaliGlossA{‘atthi kho, bho, eso attā, yaṃ tvaṃ vadesi, neso natthīti vadāmi;}}\\
\begin{addmargin}[1em]{2em}
\setstretch{.5}
{\PaliGlossB{‘*That* self of which you speak does exist, I don’t deny it.}}\\
\end{addmargin}
\end{absolutelynopagebreak}

\begin{absolutelynopagebreak}
\setstretch{.7}
{\PaliGlossA{no ca kho, bho, ayaṃ attā ettāvatā paramadiṭṭhadhammanibbānaṃ patto hoti.}}\\
\begin{addmargin}[1em]{2em}
\setstretch{.5}
{\PaliGlossB{But that’s not how *this* self attains ultimate extinguishment in the present life.}}\\
\end{addmargin}
\end{absolutelynopagebreak}

\begin{absolutelynopagebreak}
\setstretch{.7}
{\PaliGlossA{taṃ kissa hetu?}}\\
\begin{addmargin}[1em]{2em}
\setstretch{.5}
{\PaliGlossB{Why is that?}}\\
\end{addmargin}
\end{absolutelynopagebreak}

\begin{absolutelynopagebreak}
\setstretch{.7}
{\PaliGlossA{yadeva tattha sukhamiti cetaso ābhogo, etenetaṃ oḷārikaṃ akkhāyati.}}\\
\begin{addmargin}[1em]{2em}
\setstretch{.5}
{\PaliGlossB{Because the bliss and enjoyment there are coarse.}}\\
\end{addmargin}
\end{absolutelynopagebreak}

\begin{absolutelynopagebreak}
\setstretch{.7}
{\PaliGlossA{yato kho, bho, ayaṃ attā sukhassa ca pahānā dukkhassa ca pahānā pubbeva somanassadomanassānaṃ atthaṅgamā adukkhamasukhaṃ upekkhāsatipārisuddhiṃ catutthaṃ jhānaṃ upasampajja viharati, ettāvatā kho, bho, ayaṃ attā paramadiṭṭhadhammanibbānaṃ patto hotī’ti.}}\\
\begin{addmargin}[1em]{2em}
\setstretch{.5}
{\PaliGlossB{But giving up pleasure and pain, and ending former happiness and sadness, this self enters and remains in the fourth absorption, without pleasure or pain, with pure equanimity and mindfulness. That’s how this self attains ultimate extinguishment in the present life.’}}\\
\end{addmargin}
\end{absolutelynopagebreak}

\begin{absolutelynopagebreak}
\setstretch{.7}
{\PaliGlossA{ittheke sato sattassa paramadiṭṭhadhammanibbānaṃ paññapenti. (5: 62)}}\\
\begin{addmargin}[1em]{2em}
\setstretch{.5}
{\PaliGlossB{That is how some assert the extinguishment of an existing being in the present life.}}\\
\end{addmargin}
\end{absolutelynopagebreak}

\begin{absolutelynopagebreak}
\setstretch{.7}
{\PaliGlossA{imehi kho te, bhikkhave, samaṇabrāhmaṇā diṭṭhadhammanibbānavādā sato sattassa paramadiṭṭhadhammanibbānaṃ paññapenti pañcahi vatthūhi.}}\\
\begin{addmargin}[1em]{2em}
\setstretch{.5}
{\PaliGlossB{These are the five grounds on which those ascetics and brahmins assert the ultimate extinguishment of an existing being in the present life.}}\\
\end{addmargin}
\end{absolutelynopagebreak}

\begin{absolutelynopagebreak}
\setstretch{.7}
{\PaliGlossA{ye hi keci, bhikkhave, samaṇā vā brāhmaṇā vā diṭṭhadhammanibbānavādā sato sattassa paramadiṭṭhadhammanibbānaṃ paññapenti, sabbe te imeheva pañcahi vatthūhi …}}\\
\begin{addmargin}[1em]{2em}
\setstretch{.5}
{\PaliGlossB{Any ascetics and brahmins who assert the ultimate extinguishment of an existing being in the present life do so on one or other of these five grounds. Outside of this there is none.}}\\
\end{addmargin}
\end{absolutelynopagebreak}

\begin{absolutelynopagebreak}
\setstretch{.7}
{\PaliGlossA{pe …}}\\
\begin{addmargin}[1em]{2em}
\setstretch{.5}
{\PaliGlossB{The Realized One understands this …}}\\
\end{addmargin}
\end{absolutelynopagebreak}

\begin{absolutelynopagebreak}
\setstretch{.7}
{\PaliGlossA{yehi tathāgatassa yathābhuccaṃ vaṇṇaṃ sammā vadamānā vadeyyuṃ.}}\\
\begin{addmargin}[1em]{2em}
\setstretch{.5}
{\PaliGlossB{And those who genuinely praise the Realized One would rightly speak of these things.}}\\
\end{addmargin}
\end{absolutelynopagebreak}

\begin{absolutelynopagebreak}
\setstretch{.7}
{\PaliGlossA{imehi kho te, bhikkhave, samaṇabrāhmaṇā aparantakappikā aparantānudiṭṭhino aparantaṃ ārabbha anekavihitāni adhimuttipadāni abhivadanti catucattārīsāya vatthūhi.}}\\
\begin{addmargin}[1em]{2em}
\setstretch{.5}
{\PaliGlossB{These are the forty-four grounds on which those ascetics and brahmins who theorize about the future assert various hypotheses concerning the future.}}\\
\end{addmargin}
\end{absolutelynopagebreak}

\begin{absolutelynopagebreak}
\setstretch{.7}
{\PaliGlossA{ye hi keci, bhikkhave, samaṇā vā brāhmaṇā vā aparantakappikā aparantānudiṭṭhino aparantaṃ ārabbha anekavihitāni adhimuttipadāni abhivadanti, sabbe te imeheva catucattārīsāya vatthūhi …}}\\
\begin{addmargin}[1em]{2em}
\setstretch{.5}
{\PaliGlossB{Any ascetics and brahmins who theorize about the future do so on one or other of these forty-four grounds. Outside of this there is none.}}\\
\end{addmargin}
\end{absolutelynopagebreak}

\begin{absolutelynopagebreak}
\setstretch{.7}
{\PaliGlossA{pe …}}\\
\begin{addmargin}[1em]{2em}
\setstretch{.5}
{\PaliGlossB{The Realized One understands this …}}\\
\end{addmargin}
\end{absolutelynopagebreak}

\begin{absolutelynopagebreak}
\setstretch{.7}
{\PaliGlossA{yehi tathāgatassa yathābhuccaṃ vaṇṇaṃ sammā vadamānā vadeyyuṃ.}}\\
\begin{addmargin}[1em]{2em}
\setstretch{.5}
{\PaliGlossB{And those who genuinely praise the Realized One would rightly speak of these things.}}\\
\end{addmargin}
\end{absolutelynopagebreak}

\begin{absolutelynopagebreak}
\setstretch{.7}
{\PaliGlossA{imehi kho te, bhikkhave, samaṇabrāhmaṇā pubbantakappikā ca aparantakappikā ca pubbantāparantakappikā ca pubbantāparantānudiṭṭhino pubbantāparantaṃ ārabbha anekavihitāni adhimuttipadāni abhivadanti dvāsaṭṭhiyā vatthūhi.}}\\
\begin{addmargin}[1em]{2em}
\setstretch{.5}
{\PaliGlossB{These are the sixty-two grounds on which those ascetics and brahmins who theorize about the past and the future assert various hypotheses concerning the past and the future.}}\\
\end{addmargin}
\end{absolutelynopagebreak}

\begin{absolutelynopagebreak}
\setstretch{.7}
{\PaliGlossA{ye hi keci, bhikkhave, samaṇā vā brāhmaṇā vā pubbantakappikā vā aparantakappikā vā pubbantāparantakappikā vā pubbantāparantānudiṭṭhino pubbantāparantaṃ ārabbha anekavihitāni adhimuttipadāni abhivadanti, sabbe te imeheva dvāsaṭṭhiyā vatthūhi, etesaṃ vā aññatarena; natthi ito bahiddhā.}}\\
\begin{addmargin}[1em]{2em}
\setstretch{.5}
{\PaliGlossB{Any ascetics and brahmins who theorize about the past or the future do so on one or other of these sixty-two grounds. Outside of this there is none.}}\\
\end{addmargin}
\end{absolutelynopagebreak}

\begin{absolutelynopagebreak}
\setstretch{.7}
{\PaliGlossA{tayidaṃ, bhikkhave, tathāgato pajānāti:}}\\
\begin{addmargin}[1em]{2em}
\setstretch{.5}
{\PaliGlossB{The Realized One understands this:}}\\
\end{addmargin}
\end{absolutelynopagebreak}

\begin{absolutelynopagebreak}
\setstretch{.7}
{\PaliGlossA{‘ime diṭṭhiṭṭhānā evaṅgahitā evaṃparāmaṭṭhā evaṅgatikā bhavanti evaṃabhisamparāyā’ti.}}\\
\begin{addmargin}[1em]{2em}
\setstretch{.5}
{\PaliGlossB{‘If you hold on to and attach to these grounds for views it leads to such and such a destiny in the next life.’}}\\
\end{addmargin}
\end{absolutelynopagebreak}

\begin{absolutelynopagebreak}
\setstretch{.7}
{\PaliGlossA{tañca tathāgato pajānāti, tato ca uttaritaraṃ pajānāti, tañca pajānanaṃ na parāmasati, aparāmasato cassa paccattaññeva nibbuti viditā.}}\\
\begin{addmargin}[1em]{2em}
\setstretch{.5}
{\PaliGlossB{He understands this, and what goes beyond this. Yet since he does not misapprehend that understanding, he has realized extinguishment within himself.}}\\
\end{addmargin}
\end{absolutelynopagebreak}

\begin{absolutelynopagebreak}
\setstretch{.7}
{\PaliGlossA{vedanānaṃ samudayañca atthaṅgamañca assādañca ādīnavañca nissaraṇañca yathābhūtaṃ viditvā anupādāvimutto, bhikkhave, tathāgato.}}\\
\begin{addmargin}[1em]{2em}
\setstretch{.5}
{\PaliGlossB{Having truly understood the origin, ending, gratification, drawback, and escape from feelings, the Realized One is freed through not grasping.}}\\
\end{addmargin}
\end{absolutelynopagebreak}

\begin{absolutelynopagebreak}
\setstretch{.7}
{\PaliGlossA{ime kho te, bhikkhave, dhammā gambhīrā duddasā duranubodhā santā paṇītā atakkāvacarā nipuṇā paṇḍitavedanīyā, ye tathāgato sayaṃ abhiññā sacchikatvā pavedeti, yehi tathāgatassa yathābhuccaṃ vaṇṇaṃ sammā vadamānā vadeyyuṃ.}}\\
\begin{addmargin}[1em]{2em}
\setstretch{.5}
{\PaliGlossB{These are the principles—deep, hard to see, hard to understand, peaceful, sublime, beyond the scope of reason, subtle, comprehensible to the astute—which the Realized One makes known after realizing them with his own insight. And those who genuinely praise the Realized One would rightly speak of these things.}}\\
\end{addmargin}
\end{absolutelynopagebreak}

\begin{absolutelynopagebreak}
\setstretch{.7}
{\PaliGlossA{4. attālokapaññattivatthu}}\\
\begin{addmargin}[1em]{2em}
\setstretch{.5}
{\PaliGlossB{4. The Grounds For Assertions About the Self and the Cosmos}}\\
\end{addmargin}
\end{absolutelynopagebreak}

\begin{absolutelynopagebreak}
\setstretch{.7}
{\PaliGlossA{4.1. paritassitavipphanditavāra}}\\
\begin{addmargin}[1em]{2em}
\setstretch{.5}
{\PaliGlossB{4.1. Anxiety and Evasiveness}}\\
\end{addmargin}
\end{absolutelynopagebreak}

\begin{absolutelynopagebreak}
\setstretch{.7}
{\PaliGlossA{tatra, bhikkhave, ye te samaṇabrāhmaṇā sassatavādā sassataṃ attānañca lokañca paññapenti catūhi vatthūhi, tadapi tesaṃ bhavataṃ samaṇabrāhmaṇānaṃ ajānataṃ apassataṃ vedayitaṃ taṇhāgatānaṃ paritassitavipphanditameva.}}\\
\begin{addmargin}[1em]{2em}
\setstretch{.5}
{\PaliGlossB{Now, these things are only the feeling of those who do not know or see, the agitation and evasiveness of those under the sway of craving. Namely, when those ascetics and brahmins assert that the self and the cosmos are eternal on four grounds …}}\\
\end{addmargin}
\end{absolutelynopagebreak}

\begin{absolutelynopagebreak}
\setstretch{.7}
{\PaliGlossA{tatra, bhikkhave, ye te samaṇabrāhmaṇā ekaccasassatikā ekaccaasassatikā ekaccaṃ sassataṃ ekaccaṃ asassataṃ attānañca lokañca paññapenti catūhi vatthūhi, tadapi tesaṃ bhavataṃ samaṇabrāhmaṇānaṃ ajānataṃ apassataṃ vedayitaṃ taṇhāgatānaṃ paritassitavipphanditameva.}}\\
\begin{addmargin}[1em]{2em}
\setstretch{.5}
{\PaliGlossB{partially eternal on four grounds …}}\\
\end{addmargin}
\end{absolutelynopagebreak}

\begin{absolutelynopagebreak}
\setstretch{.7}
{\PaliGlossA{tatra, bhikkhave, ye te samaṇabrāhmaṇā antānantikā antānantaṃ lokassa paññapenti catūhi vatthūhi, tadapi tesaṃ bhavataṃ samaṇabrāhmaṇānaṃ ajānataṃ apassataṃ vedayitaṃ taṇhāgatānaṃ paritassitavipphanditameva.}}\\
\begin{addmargin}[1em]{2em}
\setstretch{.5}
{\PaliGlossB{finite or infinite on four grounds …}}\\
\end{addmargin}
\end{absolutelynopagebreak}

\begin{absolutelynopagebreak}
\setstretch{.7}
{\PaliGlossA{tatra, bhikkhave, ye te samaṇabrāhmaṇā amarāvikkhepikā tattha tattha pañhaṃ puṭṭhā samānā vācāvikkhepaṃ āpajjanti amarāvikkhepaṃ catūhi vatthūhi, tadapi tesaṃ bhavataṃ samaṇabrāhmaṇānaṃ ajānataṃ apassataṃ vedayitaṃ taṇhāgatānaṃ paritassitavipphanditameva.}}\\
\begin{addmargin}[1em]{2em}
\setstretch{.5}
{\PaliGlossB{or they resort to equivocation on four grounds …}}\\
\end{addmargin}
\end{absolutelynopagebreak}

\begin{absolutelynopagebreak}
\setstretch{.7}
{\PaliGlossA{tatra, bhikkhave, ye te samaṇabrāhmaṇā adhiccasamuppannikā adhiccasamuppannaṃ attānañca lokañca paññapenti dvīhi vatthūhi, tadapi tesaṃ bhavataṃ samaṇabrāhmaṇānaṃ ajānataṃ apassataṃ vedayitaṃ taṇhāgatānaṃ paritassitavipphanditameva.}}\\
\begin{addmargin}[1em]{2em}
\setstretch{.5}
{\PaliGlossB{or they assert that the self and the cosmos arose by chance on two grounds …}}\\
\end{addmargin}
\end{absolutelynopagebreak}

\begin{absolutelynopagebreak}
\setstretch{.7}
{\PaliGlossA{tatra, bhikkhave, ye te samaṇabrāhmaṇā pubbantakappikā pubbantānudiṭṭhino pubbantaṃ ārabbha anekavihitāni adhimuttipadāni abhivadanti aṭṭhārasahi vatthūhi, tadapi tesaṃ bhavataṃ samaṇabrāhmaṇānaṃ ajānataṃ apassataṃ vedayitaṃ taṇhāgatānaṃ paritassitavipphanditameva.}}\\
\begin{addmargin}[1em]{2em}
\setstretch{.5}
{\PaliGlossB{they theorize about the past on these eighteen grounds …}}\\
\end{addmargin}
\end{absolutelynopagebreak}

\begin{absolutelynopagebreak}
\setstretch{.7}
{\PaliGlossA{tatra, bhikkhave, ye te samaṇabrāhmaṇā uddhamāghātanikā saññīvādā uddhamāghātanaṃ saññiṃ attānaṃ paññapenti soḷasahi vatthūhi, tadapi tesaṃ bhavataṃ samaṇabrāhmaṇānaṃ ajānataṃ apassataṃ vedayitaṃ taṇhāgatānaṃ paritassitavipphanditameva.}}\\
\begin{addmargin}[1em]{2em}
\setstretch{.5}
{\PaliGlossB{or they assert that the self lives on after death in a percipient form on sixteen grounds …}}\\
\end{addmargin}
\end{absolutelynopagebreak}

\begin{absolutelynopagebreak}
\setstretch{.7}
{\PaliGlossA{tatra, bhikkhave, ye te samaṇabrāhmaṇā uddhamāghātanikā asaññīvādā uddhamāghātanaṃ asaññiṃ attānaṃ paññapenti aṭṭhahi vatthūhi, tadapi tesaṃ bhavataṃ samaṇabrāhmaṇānaṃ ajānataṃ apassataṃ vedayitaṃ taṇhāgatānaṃ paritassitavipphanditameva.}}\\
\begin{addmargin}[1em]{2em}
\setstretch{.5}
{\PaliGlossB{or that the self lives on after death in a non-percipient form on eight grounds …}}\\
\end{addmargin}
\end{absolutelynopagebreak}

\begin{absolutelynopagebreak}
\setstretch{.7}
{\PaliGlossA{tatra, bhikkhave, ye te samaṇabrāhmaṇā uddhamāghātanikā nevasaññīnāsaññīvādā uddhamāghātanaṃ nevasaññīnāsaññiṃ attānaṃ paññapenti aṭṭhahi vatthūhi, tadapi tesaṃ bhavataṃ samaṇabrāhmaṇānaṃ ajānataṃ apassataṃ vedayitaṃ taṇhāgatānaṃ paritassitavipphanditameva.}}\\
\begin{addmargin}[1em]{2em}
\setstretch{.5}
{\PaliGlossB{or that the self lives on after death in a neither percipient nor non-percipient form on eight grounds …}}\\
\end{addmargin}
\end{absolutelynopagebreak}

\begin{absolutelynopagebreak}
\setstretch{.7}
{\PaliGlossA{tatra, bhikkhave, ye te samaṇabrāhmaṇā ucchedavādā sato sattassa ucchedaṃ vināsaṃ vibhavaṃ paññapenti sattahi vatthūhi, tadapi tesaṃ bhavataṃ samaṇabrāhmaṇānaṃ ajānataṃ apassataṃ vedayitaṃ taṇhāgatānaṃ paritassitavipphanditameva.}}\\
\begin{addmargin}[1em]{2em}
\setstretch{.5}
{\PaliGlossB{or they assert the annihilation of an existing being on seven grounds …}}\\
\end{addmargin}
\end{absolutelynopagebreak}

\begin{absolutelynopagebreak}
\setstretch{.7}
{\PaliGlossA{tatra, bhikkhave, ye te samaṇabrāhmaṇā diṭṭhadhammanibbānavādā sato sattassa paramadiṭṭhadhammanibbānaṃ paññapenti pañcahi vatthūhi, tadapi tesaṃ bhavataṃ samaṇabrāhmaṇānaṃ ajānataṃ apassataṃ vedayitaṃ taṇhāgatānaṃ paritassitavipphanditameva.}}\\
\begin{addmargin}[1em]{2em}
\setstretch{.5}
{\PaliGlossB{or they assert the ultimate extinguishment of an existing being in the present life on five grounds …}}\\
\end{addmargin}
\end{absolutelynopagebreak}

\begin{absolutelynopagebreak}
\setstretch{.7}
{\PaliGlossA{tatra, bhikkhave, ye te samaṇabrāhmaṇā aparantakappikā aparantānudiṭṭhino aparantaṃ ārabbha anekavihitāni adhimuttipadāni abhivadanti catucattārīsāya vatthūhi, tadapi tesaṃ bhavataṃ samaṇabrāhmaṇānaṃ ajānataṃ apassataṃ vedayitaṃ taṇhāgatānaṃ paritassitavipphanditameva.}}\\
\begin{addmargin}[1em]{2em}
\setstretch{.5}
{\PaliGlossB{they theorize about the future on these forty-four grounds …}}\\
\end{addmargin}
\end{absolutelynopagebreak}

\begin{absolutelynopagebreak}
\setstretch{.7}
{\PaliGlossA{tatra, bhikkhave, ye te samaṇabrāhmaṇā pubbantakappikā ca aparantakappikā ca pubbantāparantakappikā ca pubbantāparantānudiṭṭhino pubbantāparantaṃ ārabbha anekavihitāni adhimuttipadāni abhivadanti dvāsaṭṭhiyā vatthūhi, tadapi tesaṃ bhavataṃ samaṇabrāhmaṇānaṃ ajānataṃ apassataṃ vedayitaṃ taṇhāgatānaṃ paritassitavipphanditameva.}}\\
\begin{addmargin}[1em]{2em}
\setstretch{.5}
{\PaliGlossB{When those ascetics and brahmins theorize about the past and the future on these sixty-two grounds, these things are only the feeling of those who do not know or see, the agitation and evasiveness of those under the sway of craving.}}\\
\end{addmargin}
\end{absolutelynopagebreak}

\begin{absolutelynopagebreak}
\setstretch{.7}
{\PaliGlossA{4.2. phassapaccayāvāra}}\\
\begin{addmargin}[1em]{2em}
\setstretch{.5}
{\PaliGlossB{4.2. Conditioned by Contact}}\\
\end{addmargin}
\end{absolutelynopagebreak}

\begin{absolutelynopagebreak}
\setstretch{.7}
{\PaliGlossA{tatra, bhikkhave, ye te samaṇabrāhmaṇā sassatavādā sassataṃ attānañca lokañca paññapenti catūhi vatthūhi, tadapi phassapaccayā.}}\\
\begin{addmargin}[1em]{2em}
\setstretch{.5}
{\PaliGlossB{Now, these things are conditioned by contact. Namely, when those ascetics and brahmins assert that the self and the cosmos are eternal on four grounds …}}\\
\end{addmargin}
\end{absolutelynopagebreak}

\begin{absolutelynopagebreak}
\setstretch{.7}
{\PaliGlossA{tatra, bhikkhave, ye te samaṇabrāhmaṇā ekaccasassatikā ekaccaasassatikā ekaccaṃ sassataṃ ekaccaṃ asassataṃ attānañca lokañca paññapenti catūhi vatthūhi, tadapi phassapaccayā.}}\\
\begin{addmargin}[1em]{2em}
\setstretch{.5}
{\PaliGlossB{partially eternal on four grounds …}}\\
\end{addmargin}
\end{absolutelynopagebreak}

\begin{absolutelynopagebreak}
\setstretch{.7}
{\PaliGlossA{tatra, bhikkhave, ye te samaṇabrāhmaṇā antānantikā antānantaṃ lokassa paññapenti catūhi vatthūhi, tadapi phassapaccayā.}}\\
\begin{addmargin}[1em]{2em}
\setstretch{.5}
{\PaliGlossB{finite or infinite on four grounds …}}\\
\end{addmargin}
\end{absolutelynopagebreak}

\begin{absolutelynopagebreak}
\setstretch{.7}
{\PaliGlossA{tatra, bhikkhave, ye te samaṇabrāhmaṇā amarāvikkhepikā tattha tattha pañhaṃ puṭṭhā samānā vācāvikkhepaṃ āpajjanti amarāvikkhepaṃ catūhi vatthūhi, tadapi phassapaccayā.}}\\
\begin{addmargin}[1em]{2em}
\setstretch{.5}
{\PaliGlossB{or they resort to equivocation on four grounds …}}\\
\end{addmargin}
\end{absolutelynopagebreak}

\begin{absolutelynopagebreak}
\setstretch{.7}
{\PaliGlossA{tatra, bhikkhave, ye te samaṇabrāhmaṇā adhiccasamuppannikā adhiccasamuppannaṃ attānañca lokañca paññapenti dvīhi vatthūhi, tadapi phassapaccayā.}}\\
\begin{addmargin}[1em]{2em}
\setstretch{.5}
{\PaliGlossB{or they assert that the self and the cosmos arose by chance on two grounds …}}\\
\end{addmargin}
\end{absolutelynopagebreak}

\begin{absolutelynopagebreak}
\setstretch{.7}
{\PaliGlossA{tatra, bhikkhave, ye te samaṇabrāhmaṇā pubbantakappikā pubbantānudiṭṭhino pubbantaṃ ārabbha anekavihitāni adhimuttipadāni abhivadanti aṭṭhārasahi vatthūhi, tadapi phassapaccayā.}}\\
\begin{addmargin}[1em]{2em}
\setstretch{.5}
{\PaliGlossB{they theorize about the past on these eighteen grounds …}}\\
\end{addmargin}
\end{absolutelynopagebreak}

\begin{absolutelynopagebreak}
\setstretch{.7}
{\PaliGlossA{tatra, bhikkhave, ye te samaṇabrāhmaṇā uddhamāghātanikā saññīvādā uddhamāghātanaṃ saññiṃ attānaṃ paññapenti soḷasahi vatthūhi, tadapi phassapaccayā.}}\\
\begin{addmargin}[1em]{2em}
\setstretch{.5}
{\PaliGlossB{or they assert that the self lives on after death in a percipient form on sixteen grounds …}}\\
\end{addmargin}
\end{absolutelynopagebreak}

\begin{absolutelynopagebreak}
\setstretch{.7}
{\PaliGlossA{tatra, bhikkhave, ye te samaṇabrāhmaṇā uddhamāghātanikā asaññīvādā uddhamāghātanaṃ asaññiṃ attānaṃ paññapenti aṭṭhahi vatthūhi, tadapi phassapaccayā.}}\\
\begin{addmargin}[1em]{2em}
\setstretch{.5}
{\PaliGlossB{or that the self lives on after death in a non-percipient form on eight grounds …}}\\
\end{addmargin}
\end{absolutelynopagebreak}

\begin{absolutelynopagebreak}
\setstretch{.7}
{\PaliGlossA{tatra, bhikkhave, ye te samaṇabrāhmaṇā uddhamāghātanikā nevasaññīnāsaññīvādā uddhamāghātanaṃ nevasaññīnāsaññiṃ attānaṃ paññapenti aṭṭhahi vatthūhi, tadapi phassapaccayā.}}\\
\begin{addmargin}[1em]{2em}
\setstretch{.5}
{\PaliGlossB{or that the self lives on after death in a neither percipient nor non-percipient form on eight grounds …}}\\
\end{addmargin}
\end{absolutelynopagebreak}

\begin{absolutelynopagebreak}
\setstretch{.7}
{\PaliGlossA{tatra, bhikkhave, ye te samaṇabrāhmaṇā ucchedavādā sato sattassa ucchedaṃ vināsaṃ vibhavaṃ paññapenti sattahi vatthūhi, tadapi phassapaccayā.}}\\
\begin{addmargin}[1em]{2em}
\setstretch{.5}
{\PaliGlossB{or they assert the annihilation of an existing being on seven grounds …}}\\
\end{addmargin}
\end{absolutelynopagebreak}

\begin{absolutelynopagebreak}
\setstretch{.7}
{\PaliGlossA{tatra, bhikkhave, ye te samaṇabrāhmaṇā diṭṭhadhammanibbānavādā sato sattassa paramadiṭṭhadhammanibbānaṃ paññapenti pañcahi vatthūhi, tadapi phassapaccayā.}}\\
\begin{addmargin}[1em]{2em}
\setstretch{.5}
{\PaliGlossB{or they assert the ultimate extinguishment of an existing being in the present life on five grounds …}}\\
\end{addmargin}
\end{absolutelynopagebreak}

\begin{absolutelynopagebreak}
\setstretch{.7}
{\PaliGlossA{tatra, bhikkhave, ye te samaṇabrāhmaṇā aparantakappikā aparantānudiṭṭhino aparantaṃ ārabbha anekavihitāni adhimuttipadāni abhivadanti catucattārīsāya vatthūhi, tadapi phassapaccayā.}}\\
\begin{addmargin}[1em]{2em}
\setstretch{.5}
{\PaliGlossB{they theorize about the future on these forty-four grounds …}}\\
\end{addmargin}
\end{absolutelynopagebreak}

\begin{absolutelynopagebreak}
\setstretch{.7}
{\PaliGlossA{tatra, bhikkhave, ye te samaṇabrāhmaṇā pubbantakappikā ca aparantakappikā ca pubbantāparantakappikā ca pubbantāparantānudiṭṭhino pubbantāparantaṃ ārabbha anekavihitāni adhimuttipadāni abhivadanti dvāsaṭṭhiyā vatthūhi, tadapi phassapaccayā.}}\\
\begin{addmargin}[1em]{2em}
\setstretch{.5}
{\PaliGlossB{When those ascetics and brahmins theorize about the past and the future on these sixty-two grounds, that too is conditioned by contact.}}\\
\end{addmargin}
\end{absolutelynopagebreak}

\begin{absolutelynopagebreak}
\setstretch{.7}
{\PaliGlossA{4.3. netaṃṭhānaṃvijjativāra}}\\
\begin{addmargin}[1em]{2em}
\setstretch{.5}
{\PaliGlossB{4.3. Not Possible}}\\
\end{addmargin}
\end{absolutelynopagebreak}

\begin{absolutelynopagebreak}
\setstretch{.7}
{\PaliGlossA{tatra, bhikkhave, ye te samaṇabrāhmaṇā sassatavādā sassataṃ attānañca lokañca paññapenti catūhi vatthūhi, te vata aññatra phassā paṭisaṃvedissantīti netaṃ ṭhānaṃ vijjati.}}\\
\begin{addmargin}[1em]{2em}
\setstretch{.5}
{\PaliGlossB{    -}}\\
\end{addmargin}
\end{absolutelynopagebreak}

\begin{absolutelynopagebreak}
\setstretch{.7}
{\PaliGlossA{tatra, bhikkhave, ye te samaṇabrāhmaṇā ekaccasassatikā ekaccaasassatikā ekaccaṃ sassataṃ ekaccaṃ asassataṃ attānañca lokañca paññapenti catūhi vatthūhi, te vata aññatra phassā paṭisaṃvedissantīti netaṃ ṭhānaṃ vijjati.}}\\
\begin{addmargin}[1em]{2em}
\setstretch{.5}
{\PaliGlossB{    -}}\\
\end{addmargin}
\end{absolutelynopagebreak}

\begin{absolutelynopagebreak}
\setstretch{.7}
{\PaliGlossA{tatra, bhikkhave, ye te samaṇabrāhmaṇā antānantikā antānantaṃ lokassa paññapenti catūhi vatthūhi, te vata aññatra phassā paṭisaṃvedissantīti netaṃ ṭhānaṃ vijjati.}}\\
\begin{addmargin}[1em]{2em}
\setstretch{.5}
{\PaliGlossB{    -}}\\
\end{addmargin}
\end{absolutelynopagebreak}

\begin{absolutelynopagebreak}
\setstretch{.7}
{\PaliGlossA{tatra, bhikkhave, ye te samaṇabrāhmaṇā amarāvikkhepikā tattha tattha pañhaṃ puṭṭhā samānā vācāvikkhepaṃ āpajjanti amarāvikkhepaṃ catūhi vatthūhi, te vata aññatra phassā paṭisaṃvedissantīti netaṃ ṭhānaṃ vijjati.}}\\
\begin{addmargin}[1em]{2em}
\setstretch{.5}
{\PaliGlossB{    -}}\\
\end{addmargin}
\end{absolutelynopagebreak}

\begin{absolutelynopagebreak}
\setstretch{.7}
{\PaliGlossA{tatra, bhikkhave, ye te samaṇabrāhmaṇā adhiccasamuppannikā adhiccasamuppannaṃ attānañca lokañca paññapenti dvīhi vatthūhi, te vata aññatra phassā paṭisaṃvedissantīti netaṃ ṭhānaṃ vijjati.}}\\
\begin{addmargin}[1em]{2em}
\setstretch{.5}
{\PaliGlossB{    -}}\\
\end{addmargin}
\end{absolutelynopagebreak}

\begin{absolutelynopagebreak}
\setstretch{.7}
{\PaliGlossA{tatra, bhikkhave, ye te samaṇabrāhmaṇā pubbantakappikā pubbantānudiṭṭhino pubbantaṃ ārabbha anekavihitāni adhimuttipadāni abhivadanti aṭṭhārasahi vatthūhi, te vata aññatra phassā paṭisaṃvedissantīti netaṃ ṭhānaṃ vijjati.}}\\
\begin{addmargin}[1em]{2em}
\setstretch{.5}
{\PaliGlossB{    -}}\\
\end{addmargin}
\end{absolutelynopagebreak}

\begin{absolutelynopagebreak}
\setstretch{.7}
{\PaliGlossA{tatra, bhikkhave, ye te samaṇabrāhmaṇā uddhamāghātanikā saññīvādā uddhamāghātanaṃ saññiṃ attānaṃ paññapenti soḷasahi vatthūhi, te vata aññatra phassā paṭisaṃvedissantīti netaṃ ṭhānaṃ vijjati.}}\\
\begin{addmargin}[1em]{2em}
\setstretch{.5}
{\PaliGlossB{    -}}\\
\end{addmargin}
\end{absolutelynopagebreak}

\begin{absolutelynopagebreak}
\setstretch{.7}
{\PaliGlossA{tatra, bhikkhave, ye te samaṇabrāhmaṇā uddhamāghātanikā asaññīvādā, uddhamāghātanaṃ asaññiṃ attānaṃ paññapenti aṭṭhahi vatthūhi, te vata aññatra phassā paṭisaṃvedissantīti netaṃ ṭhānaṃ vijjati.}}\\
\begin{addmargin}[1em]{2em}
\setstretch{.5}
{\PaliGlossB{    -}}\\
\end{addmargin}
\end{absolutelynopagebreak}

\begin{absolutelynopagebreak}
\setstretch{.7}
{\PaliGlossA{tatra, bhikkhave, ye te samaṇabrāhmaṇā uddhamāghātanikā nevasaññīnāsaññīvādā uddhamāghātanaṃ nevasaññīnāsaññiṃ attānaṃ paññapenti aṭṭhahi vatthūhi, te vata aññatra phassā paṭisaṃvedissantīti netaṃ ṭhānaṃ vijjati.}}\\
\begin{addmargin}[1em]{2em}
\setstretch{.5}
{\PaliGlossB{    -}}\\
\end{addmargin}
\end{absolutelynopagebreak}

\begin{absolutelynopagebreak}
\setstretch{.7}
{\PaliGlossA{tatra, bhikkhave, ye te samaṇabrāhmaṇā ucchedavādā sato sattassa ucchedaṃ vināsaṃ vibhavaṃ paññapenti sattahi vatthūhi, te vata aññatra phassā paṭisaṃvedissantīti netaṃ ṭhānaṃ vijjati.}}\\
\begin{addmargin}[1em]{2em}
\setstretch{.5}
{\PaliGlossB{    -}}\\
\end{addmargin}
\end{absolutelynopagebreak}

\begin{absolutelynopagebreak}
\setstretch{.7}
{\PaliGlossA{tatra, bhikkhave, ye te samaṇabrāhmaṇā diṭṭhadhammanibbānavādā sato sattassa paramadiṭṭhadhammanibbānaṃ paññapenti pañcahi vatthūhi, te vata aññatra phassā paṭisaṃvedissantīti netaṃ ṭhānaṃ vijjati.}}\\
\begin{addmargin}[1em]{2em}
\setstretch{.5}
{\PaliGlossB{    -}}\\
\end{addmargin}
\end{absolutelynopagebreak}

\begin{absolutelynopagebreak}
\setstretch{.7}
{\PaliGlossA{tatra, bhikkhave, ye te samaṇabrāhmaṇā aparantakappikā aparantānudiṭṭhino aparantaṃ ārabbha anekavihitāni adhimuttipadāni abhivadanti catucattārīsāya vatthūhi, te vata aññatra phassā paṭisaṃvedissantīti netaṃ ṭhānaṃ vijjati.}}\\
\begin{addmargin}[1em]{2em}
\setstretch{.5}
{\PaliGlossB{    -}}\\
\end{addmargin}
\end{absolutelynopagebreak}

\begin{absolutelynopagebreak}
\setstretch{.7}
{\PaliGlossA{tatra, bhikkhave, ye te samaṇabrāhmaṇā pubbantakappikā ca aparantakappikā ca pubbantāparantakappikā ca pubbantāparantānudiṭṭhino pubbantāparantaṃ ārabbha anekavihitāni adhimuttipadāni abhivadanti dvāsaṭṭhiyā vatthūhi, te vata aññatra phassā paṭisaṃvedissantīti netaṃ ṭhānaṃ vijjati.}}\\
\begin{addmargin}[1em]{2em}
\setstretch{.5}
{\PaliGlossB{Now, when those ascetics and brahmins theorize about the past and the future on these sixty-two grounds, it is not possible that they should experience these things without contact.}}\\
\end{addmargin}
\end{absolutelynopagebreak}

\begin{absolutelynopagebreak}
\setstretch{.7}
{\PaliGlossA{4.4. diṭṭhigatikādhiṭṭhānavaṭṭakathā}}\\
\begin{addmargin}[1em]{2em}
\setstretch{.5}
{\PaliGlossB{4.4. Dependent Origination}}\\
\end{addmargin}
\end{absolutelynopagebreak}

\begin{absolutelynopagebreak}
\setstretch{.7}
{\PaliGlossA{tatra, bhikkhave, ye te samaṇabrāhmaṇā sassatavādā sassataṃ attānañca lokañca paññapenti catūhi vatthūhi, yepi te samaṇabrāhmaṇā ekaccasassatikā ekaccaasassatikā … pe …}}\\
\begin{addmargin}[1em]{2em}
\setstretch{.5}
{\PaliGlossB{    -}}\\
\end{addmargin}
\end{absolutelynopagebreak}

\begin{absolutelynopagebreak}
\setstretch{.7}
{\PaliGlossA{yepi te samaṇabrāhmaṇā antānantikā …}}\\
\begin{addmargin}[1em]{2em}
\setstretch{.5}
{\PaliGlossB{    -}}\\
\end{addmargin}
\end{absolutelynopagebreak}

\begin{absolutelynopagebreak}
\setstretch{.7}
{\PaliGlossA{yepi te samaṇabrāhmaṇā amarāvikkhepikā …}}\\
\begin{addmargin}[1em]{2em}
\setstretch{.5}
{\PaliGlossB{    -}}\\
\end{addmargin}
\end{absolutelynopagebreak}

\begin{absolutelynopagebreak}
\setstretch{.7}
{\PaliGlossA{yepi te samaṇabrāhmaṇā adhiccasamuppannikā …}}\\
\begin{addmargin}[1em]{2em}
\setstretch{.5}
{\PaliGlossB{    -}}\\
\end{addmargin}
\end{absolutelynopagebreak}

\begin{absolutelynopagebreak}
\setstretch{.7}
{\PaliGlossA{yepi te samaṇabrāhmaṇā pubbantakappikā …}}\\
\begin{addmargin}[1em]{2em}
\setstretch{.5}
{\PaliGlossB{    -}}\\
\end{addmargin}
\end{absolutelynopagebreak}

\begin{absolutelynopagebreak}
\setstretch{.7}
{\PaliGlossA{yepi te samaṇabrāhmaṇā uddhamāghātanikā saññīvādā …}}\\
\begin{addmargin}[1em]{2em}
\setstretch{.5}
{\PaliGlossB{    -}}\\
\end{addmargin}
\end{absolutelynopagebreak}

\begin{absolutelynopagebreak}
\setstretch{.7}
{\PaliGlossA{yepi te samaṇabrāhmaṇā uddhamāghātanikā asaññīvādā …}}\\
\begin{addmargin}[1em]{2em}
\setstretch{.5}
{\PaliGlossB{    -}}\\
\end{addmargin}
\end{absolutelynopagebreak}

\begin{absolutelynopagebreak}
\setstretch{.7}
{\PaliGlossA{yepi te samaṇabrāhmaṇā uddhamāghātanikā nevasaññīnāsaññīvādā …}}\\
\begin{addmargin}[1em]{2em}
\setstretch{.5}
{\PaliGlossB{    -}}\\
\end{addmargin}
\end{absolutelynopagebreak}

\begin{absolutelynopagebreak}
\setstretch{.7}
{\PaliGlossA{yepi te samaṇabrāhmaṇā ucchedavādā …}}\\
\begin{addmargin}[1em]{2em}
\setstretch{.5}
{\PaliGlossB{    -}}\\
\end{addmargin}
\end{absolutelynopagebreak}

\begin{absolutelynopagebreak}
\setstretch{.7}
{\PaliGlossA{yepi te samaṇabrāhmaṇā diṭṭhadhammanibbānavādā …}}\\
\begin{addmargin}[1em]{2em}
\setstretch{.5}
{\PaliGlossB{    -}}\\
\end{addmargin}
\end{absolutelynopagebreak}

\begin{absolutelynopagebreak}
\setstretch{.7}
{\PaliGlossA{yepi te samaṇabrāhmaṇā aparantakappikā …}}\\
\begin{addmargin}[1em]{2em}
\setstretch{.5}
{\PaliGlossB{    -}}\\
\end{addmargin}
\end{absolutelynopagebreak}

\begin{absolutelynopagebreak}
\setstretch{.7}
{\PaliGlossA{yepi te samaṇabrāhmaṇā pubbantakappikā ca aparantakappikā ca pubbantāparantakappikā ca pubbantāparantānudiṭṭhino pubbantāparantaṃ ārabbha anekavihitāni adhimuttipadāni abhivadanti dvāsaṭṭhiyā vatthūhi, sabbe te chahi phassāyatanehi phussa phussa paṭisaṃvedenti tesaṃ vedanāpaccayā taṇhā, taṇhāpaccayā upādānaṃ, upādānapaccayā bhavo, bhavapaccayā jāti, jātipaccayā jarāmaraṇaṃ sokaparidevadukkhadomanassupāyāsā sambhavanti.}}\\
\begin{addmargin}[1em]{2em}
\setstretch{.5}
{\PaliGlossB{Now, when those ascetics and brahmins theorize about the past and the future on these sixty-two grounds, all of them experience this by repeated contact through the six fields of contact. Their feeling is a condition for craving. Craving is a condition for grasping. Grasping is a condition for continued existence. Continued existence is a condition for rebirth. Rebirth is a condition for old age and death, sorrow, lamentation, pain, sadness, and distress to come to be.}}\\
\end{addmargin}
\end{absolutelynopagebreak}

\begin{absolutelynopagebreak}
\setstretch{.7}
{\PaliGlossA{5. vivaṭṭakathādi}}\\
\begin{addmargin}[1em]{2em}
\setstretch{.5}
{\PaliGlossB{5. The End of the Round}}\\
\end{addmargin}
\end{absolutelynopagebreak}

\begin{absolutelynopagebreak}
\setstretch{.7}
{\PaliGlossA{yato kho, bhikkhave, bhikkhu channaṃ phassāyatanānaṃ samudayañca atthaṅgamañca assādañca ādīnavañca nissaraṇañca yathābhūtaṃ pajānāti, ayaṃ imehi sabbeheva uttaritaraṃ pajānāti.}}\\
\begin{addmargin}[1em]{2em}
\setstretch{.5}
{\PaliGlossB{When a mendicant truly understands the six fields of contacts’ origin, ending, gratification, drawback, and escape, they understand what lies beyond all these things.}}\\
\end{addmargin}
\end{absolutelynopagebreak}

\begin{absolutelynopagebreak}
\setstretch{.7}
{\PaliGlossA{ye hi keci, bhikkhave, samaṇā vā brāhmaṇā vā pubbantakappikā vā aparantakappikā vā pubbantāparantakappikā vā pubbantāparantānudiṭṭhino pubbantāparantaṃ ārabbha anekavihitāni adhimuttipadāni abhivadanti, sabbe te imeheva dvāsaṭṭhiyā vatthūhi antojālīkatā, ettha sitāva ummujjamānā ummujjanti, ettha pariyāpannā antojālīkatāva ummujjamānā ummujjanti.}}\\
\begin{addmargin}[1em]{2em}
\setstretch{.5}
{\PaliGlossB{All of these ascetics and brahmins who theorize about the past or the future are trapped in the net of these sixty-two grounds, so that wherever they emerge they are caught and trapped in this very net.}}\\
\end{addmargin}
\end{absolutelynopagebreak}

\begin{absolutelynopagebreak}
\setstretch{.7}
{\PaliGlossA{seyyathāpi, bhikkhave, dakkho kevaṭṭo vā kevaṭṭantevāsī vā sukhumacchikena jālena parittaṃ udakadahaṃ otthareyya. tassa evamassa: ‘ye kho keci imasmiṃ udakadahe oḷārikā pāṇā, sabbe te antojālīkatā. ettha sitāva ummujjamānā ummujjanti; ettha pariyāpannā antojālīkatāva ummujjamānā ummujjantī’ti;}}\\
\begin{addmargin}[1em]{2em}
\setstretch{.5}
{\PaliGlossB{Suppose a deft fisherman or his apprentice were to cast a fine-meshed net over a small pond. They’d think: ‘Any sizable creatures in this pond will be trapped in the net. Wherever they emerge they are caught and trapped in this very net.’}}\\
\end{addmargin}
\end{absolutelynopagebreak}

\begin{absolutelynopagebreak}
\setstretch{.7}
{\PaliGlossA{evameva kho, bhikkhave, ye hi keci samaṇā vā brāhmaṇā vā pubbantakappikā vā aparantakappikā vā pubbantāparantakappikā vā pubbantāparantānudiṭṭhino pubbantāparantaṃ ārabbha anekavihitāni adhimuttipadāni abhivadanti, sabbe te imeheva dvāsaṭṭhiyā vatthūhi antojālīkatā ettha sitāva ummujjamānā ummujjanti, ettha pariyāpannā antojālīkatāva ummujjamānā ummujjanti.}}\\
\begin{addmargin}[1em]{2em}
\setstretch{.5}
{\PaliGlossB{In the same way, all of these ascetics and brahmins who theorize about the past or the future are trapped in the net of these sixty-two grounds, so that wherever they emerge they are caught and trapped in this very net.}}\\
\end{addmargin}
\end{absolutelynopagebreak}

\begin{absolutelynopagebreak}
\setstretch{.7}
{\PaliGlossA{ucchinnabhavanettiko, bhikkhave, tathāgatassa kāyo tiṭṭhati.}}\\
\begin{addmargin}[1em]{2em}
\setstretch{.5}
{\PaliGlossB{The Realized One’s body remains, but his attachment to rebirth has been cut off.}}\\
\end{addmargin}
\end{absolutelynopagebreak}

\begin{absolutelynopagebreak}
\setstretch{.7}
{\PaliGlossA{yāvassa kāyo ṭhassati, tāva naṃ dakkhanti devamanussā.}}\\
\begin{addmargin}[1em]{2em}
\setstretch{.5}
{\PaliGlossB{As long as his body remains he will be seen by gods and humans.}}\\
\end{addmargin}
\end{absolutelynopagebreak}

\begin{absolutelynopagebreak}
\setstretch{.7}
{\PaliGlossA{kāyassa bhedā uddhaṃ jīvitapariyādānā na naṃ dakkhanti devamanussā.}}\\
\begin{addmargin}[1em]{2em}
\setstretch{.5}
{\PaliGlossB{But when his body breaks up, after life has ended, gods and humans will see him no more.}}\\
\end{addmargin}
\end{absolutelynopagebreak}

\begin{absolutelynopagebreak}
\setstretch{.7}
{\PaliGlossA{seyyathāpi, bhikkhave, ambapiṇḍiyā vaṇṭacchinnāya yāni kānici ambāni vaṇṭapaṭibandhāni, sabbāni tāni tadanvayāni bhavanti;}}\\
\begin{addmargin}[1em]{2em}
\setstretch{.5}
{\PaliGlossB{When the stalk of a bunch of mangoes is cut, all the mangoes attached to the stalk will follow along.}}\\
\end{addmargin}
\end{absolutelynopagebreak}

\begin{absolutelynopagebreak}
\setstretch{.7}
{\PaliGlossA{evameva kho, bhikkhave, ucchinnabhavanettiko tathāgatassa kāyo tiṭṭhati,}}\\
\begin{addmargin}[1em]{2em}
\setstretch{.5}
{\PaliGlossB{In the same way, the Realized One’s body remains, but his attachment to rebirth has been cut off.}}\\
\end{addmargin}
\end{absolutelynopagebreak}

\begin{absolutelynopagebreak}
\setstretch{.7}
{\PaliGlossA{yāvassa kāyo ṭhassati, tāva naṃ dakkhanti devamanussā,}}\\
\begin{addmargin}[1em]{2em}
\setstretch{.5}
{\PaliGlossB{As long as his body remains he will be seen by gods and humans.}}\\
\end{addmargin}
\end{absolutelynopagebreak}

\begin{absolutelynopagebreak}
\setstretch{.7}
{\PaliGlossA{kāyassa bhedā uddhaṃ jīvitapariyādānā na naṃ dakkhanti devamanussā”ti.}}\\
\begin{addmargin}[1em]{2em}
\setstretch{.5}
{\PaliGlossB{But when his body breaks up, after life has ended, gods and humans will see him no more.”}}\\
\end{addmargin}
\end{absolutelynopagebreak}

\begin{absolutelynopagebreak}
\setstretch{.7}
{\PaliGlossA{evaṃ vutte, āyasmā ānando bhagavantaṃ etadavoca:}}\\
\begin{addmargin}[1em]{2em}
\setstretch{.5}
{\PaliGlossB{When he had spoken, Venerable Ānanda said to the Buddha,}}\\
\end{addmargin}
\end{absolutelynopagebreak}

\begin{absolutelynopagebreak}
\setstretch{.7}
{\PaliGlossA{“acchariyaṃ, bhante, abbhutaṃ, bhante, ko nāmo ayaṃ, bhante, dhammapariyāyo”ti?}}\\
\begin{addmargin}[1em]{2em}
\setstretch{.5}
{\PaliGlossB{“It’s incredible, sir, it’s amazing! What is the name of this exposition of the teaching?”}}\\
\end{addmargin}
\end{absolutelynopagebreak}

\begin{absolutelynopagebreak}
\setstretch{.7}
{\PaliGlossA{“tasmātiha tvaṃ, ānanda, imaṃ dhammapariyāyaṃ atthajālantipi naṃ dhārehi, dhammajālantipi naṃ dhārehi, brahmajālantipi naṃ dhārehi, diṭṭhijālantipi naṃ dhārehi, anuttaro saṅgāmavijayotipi naṃ dhārehī”ti.}}\\
\begin{addmargin}[1em]{2em}
\setstretch{.5}
{\PaliGlossB{“Well, then, Ānanda, you may remember this exposition of the teaching as ‘The Net of Meaning’, or else ‘The Net of the Teaching’, or else ‘The Prime Net’, or else ‘The Net of Views’, or else ‘The Supreme Victory in Battle’.”}}\\
\end{addmargin}
\end{absolutelynopagebreak}

\begin{absolutelynopagebreak}
\setstretch{.7}
{\PaliGlossA{idamavoca bhagavā.}}\\
\begin{addmargin}[1em]{2em}
\setstretch{.5}
{\PaliGlossB{That is what the Buddha said.}}\\
\end{addmargin}
\end{absolutelynopagebreak}

\begin{absolutelynopagebreak}
\setstretch{.7}
{\PaliGlossA{attamanā te bhikkhū bhagavato bhāsitaṃ abhinandunti.}}\\
\begin{addmargin}[1em]{2em}
\setstretch{.5}
{\PaliGlossB{Satisfied, the mendicants were happy with what the Buddha said.}}\\
\end{addmargin}
\end{absolutelynopagebreak}

\begin{absolutelynopagebreak}
\setstretch{.7}
{\PaliGlossA{imasmiñca pana veyyākaraṇasmiṃ bhaññamāne dasasahassī lokadhātu akampitthāti.}}\\
\begin{addmargin}[1em]{2em}
\setstretch{.5}
{\PaliGlossB{And while this discourse was being spoken, the galaxy shook.}}\\
\end{addmargin}
\end{absolutelynopagebreak}

\begin{absolutelynopagebreak}
\setstretch{.7}
{\PaliGlossA{brahmajālasuttaṃ niṭṭhitaṃ paṭhamaṃ.}}\\
\begin{addmargin}[1em]{2em}
\setstretch{.5}
{\PaliGlossB{    -}}\\
\end{addmargin}
\end{absolutelynopagebreak}
